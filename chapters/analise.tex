\chapter{Análise dos dados}


Como não foi encontrado no estado da prática, uma aplicação que contenha as mesmas características do trabalho proposto, neste capítulo tem-se a análise de uma pesquisa acerca da motivação das pessoas perante a proposta deste trabalho. 
Assim, a seção \ref{objetivo_da_pesquisa} traz o objetivo da pesquisa realizada. 
A seção \ref{publico_alvo} busca definir o nicho correto de pessoas a quem a pesquisa deve ser direcionada. 
A seção \ref{pesquisa} contém uma apresentação da pesquisa aplicada 
e a seção \ref{resultados}, os resultados provenientes da pesquisa.


\section{Objetivo da pesquisa}\label{objetivo_da_pesquisa}


Como não foi encontrado um software que provesse as mesmas funcionalidades do trabalho proposto, então tem-se o mesmo como um novo produto no mercado. Entretanto, antes de expôr um produto, é recomendado que realize uma análise de mercado para saber a viabilidade da ideia. Baseado nisso, o objetivo da pesquisa deste trabalho é buscar saber se a ideia tem uma boa aceitação no mercado.


\section{Público alvo}\label{publico_alvo}


Diante do fato de o trabalho proposto estar ligado à área de desenvolvimento de software, independentemente da tecnologia usada para trabalhar, será adotado como público alvo da pesquisa as pessoas ligadas ao desenvolvimento e manutenção de software, podendo ser programador, analista, engenheiro de software, estudantes, pesquisadores ou áreas afins.


\section{Método de pesquisa}\label{pesquisa}


A fim de validar a proposta deste trabalho de conclusão de curso, trabalhará-se com dados primários, ou seja, dados coletados e analisados através de um formulário de pesquisa. Tal formulário é composto por dez perguntas, cada uma contendo quatro opções de resposta.


\section{Resultados}\label{resultados}


