\chapter{Referencial Teórico}


% Ser direto no começo, focando no que realmente será discutido A seção \ref{sec:apps_mobile} 
%A modularização de um software traz diversos benefícios para o desenvolvimento e manutenção do mesmo. 
 \section{Software como serviço}\label{sec:saas}


Um trabalho recente realizado por Lechessa et al. \cite{LechesaSS11} apresenta uma pesquisa qualitativa sobre os fatores determinantes para adoção ou não de um SaaS voltado para ERP na África do Sul. Esses autores indicam que os principais fatores determinantes para adoção desse mecanismo de software são sua fluidez quanto à rede e a segurança. Esses fatores estão presentes na aplicação desenvolvidas neste trabalho de conclusão de curso.
 

O artigo de \citep{journals/corr/RaiSM13} orienta como o avanço da computação em nuvem não é apenas um problema do ponto de vista da segurança, pois muitos trabalhos na literatura apenas a área apenas como mais um ponto de vulnerabilidade. Os autores de \citep{journals/corr/RaiSM13} realizaram estudos exploratórios junto a empresas usuárias de serviços em computação em nuvem e consideram que a perspectiva de SaaS também pode fortalecer a segurança nas aplicações de cloud computing, pois o software de autenticação compartilhado por várias aplicações em nuvem, oferece uma melhor padronização e consequente facilidade de prevenção a erros de vulnerabilidade específicas de cada módulo da pesquisa. Esse ponto de vista é muito importante para qualquer trabalho de ponta na área de SaaS.


A definição de SaaS encontra-se muito bem elaborada em um dos trabalhos listados na literatura. Segundo La e Chun \citep{La2009Systematic}, o princípio da definição de Software como um Serviço (Sofware as a Service - SaaS) é um serviço complementar para aplicações da computação em nuvem (cloud computing). As duas áreas estão interligadas, no entanto, não se confundem, pois o SaaS deve ser entendido como um mecanismo de suporte às soluções existentes na cloud. Os SaaS existem justamente para maximizar o reuso de serviços repetidos e não centrais em uma aplicação remota.


\section{Aplicações web}\label{sec:apps_web}


Para atender à fomentada demanda de aplicativos web, é necessário adotar métodos de desenvolvimentos que sejam ágeis, eficientes e de fácil manutenção. \cite{1372143} Propõe o uso do modelo MVC (Model, View e Controller) no atual desenvolvimento para softwares web. O modelo apresentado tornou-se um padrão popular e divide o software em camadas com propósito definido, tornando-o de mais fácil manutenção.]


\section{}\label{sec:}


