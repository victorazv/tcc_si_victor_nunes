%% RiSE Latex Template - version 0.5
%%
%% RiSE's latex template for thesis and dissertations
%% http://risetemplate.sourceforge.net
%%
%% (c) 2012 Yguaratã Cerqueira Cavalcanti (yguarata@gmail.com)
%%          Vinicius Cardoso Garcia (vinicius.garcia@gmail.com)
%%
%% This document was initially based on UFPEThesis template, from Paulo Gustavo
%% S. Fonseca.
%%
%% ACKNOWLEDGEMENTS
%%
%% We would like to thanks the RiSE's researchers community, the 
%% students from Federal University of Pernambuco, and other users that have
%% been contributing to this projects with comments and patches.
%%
%% GENERAL INSTRUCTIONS
%%
%% We strongly recommend you to compile your documents using pdflatex command.
%% It is also recommend use the texlipse plugin for Eclipse to edit your documents.
%%
%% Options for \documentclass command:
%%         * Idiom
%%           pt   - Portguese (default)
%%           en   - English
%%
%%         * Text type
%%           bsc  - B.Sc. Thesis
%%           msc  - M.Sc. Thesis (default)
%%           qual - PHD qualification (not tested yet)
%%           prop - PHD proposal (not tested yet)
%%           phd  - PHD thesis
%%
%%         * Media
%%           scr  - to eletronic version (PDF) / see the users guide
%%
%%         * Pagination
%%           oneside - unique face press
%%           twoside - two faces press
%%
%%		   * Line spacing
%%           singlespacing  - the same as using \linespread{1}
%%           onehalfspacing - the same as using \linespread{1.3}
%%           doublespacing  - the same as using \linespread{1.6}
%%
%% Reference commands. Use the following commands to make references in your
%% text:
%%          \figref  -- for Figure reference
%%          \tabref  -- for Table reference
%%          \eqnref  -- for equation reference
%%          \chapref -- for chapter reference
%%          \secref  -- for section reference
%%          \appref  -- for appendix reference
%%          \axiref  -- for axiom reference
%%          \conjref -- for conjecture reference
%%          \defref  -- for definition reference
%%          \lemref  -- for lemma reference
%%          \theoref -- for theorem reference
%%          \corref  -- for corollary reference
%%          \propref -- for proprosition reference
%%          \pgref   -- for page reference
%%
%%          Example: See \chapref{chap:introduction}. It will produce 
%%                   'See Chapter 1', in case of English language.

\documentclass[pt,twoside,onehalfspacing,bsc]{risethesis}

\usepackage[sort,square]{natbib} %package for bibliography
\bibliographystyle{ieeetr} %authordate1

\usepackage[utf8]{inputenc}
\usepackage[brazilian]{babel}
\usepackage[T1]{fontenc}

\usepackage{graphicx}

%% Change the following pdf author attribute name to your name.
\usepackage[linkcolor=blue,citecolor=blue,urlcolor=blue,colorlinks,pdfpagelabels,pdftitle={Bruno Cabral's Bachelor Thesis},pdfauthor={Bruno Cabral}]{hyperref}

\address{SALVADOR}

\universitypt{Universidade Federal da Bahia}
\universityen{Federal University of Bahia}

\departmentpt{Depertamento de Ciência da Computação}
\departmenten{Computer Science Department}

\programpt{Programa Multiinstitucional de Pós-graduação em Ciência da Computação}
\programen{Graduate in Computer Science}

\majorfieldpt{Ciência da Computação}
\majorfielden{Computer Science}

\title{Sistema de apoio à Pós graduação - UFBA}
\date{Outubro/2016}

\author{Victor de Azevedo Nunes}
\adviser{Ivan do Carmo Machado}

\begin{document}

\frontmatter
\frontpage
\presentationpage

\begin{dedicatory}
Eu dedico esta dissertação...
%I dedicate this dissertation to my family, girlfriend, friends and
%professors who gave me all necessary support to get here.
\end{dedicatory}

\acknowledgements
Meus agradecimentos...

\begin{epigraph}[]{Edward V Berard}
Walking on water and developing software from a specification are easy if both are frozen
\end{epigraph}

\resumo
% Escreva seu resumo no arquivo resumo.tex
Este trabalho de conclusão de curso utiliza elementos da engenharia de software para propor um software como serviço a fim de otimizar o processo de construção e manutenção dos softwares. Assim, o objetivo deste SaaS é gerenciar as permissões de acesso de sistemas cliente, provendo o reuso de software e padronizando as soluções. O sistema proposto neste trabalho encontra-se implementado e disponível no Github, e traz fundamentos sobre a arquitetura e tecnologias utilizadas, bem como avaliações sobre possibilidades de utilização do mesmo em diversos ambientes, como Web e mobile.

\begin{keywords}
Software; Reuso; SaaS; Web; PHP
\end{keywords}

\abstract
% Write your abstract in a file called abstract.tex
My abstract...

\begin{keywords}
key words...
\end{keywords}

% Summary (tables of contents)
\tableofcontents

% List of figures
\listoffigures

% List of tables
\listoftables

% List of acronyms
% Acronyms manual: http://linorg.usp.br/CTAN/macros/latex/contrib/acronym/acronym.pdf
\listofacronyms
\begin{acronym}[ACRONYM] 
% Change the word ACRONYM above to change the acronym column width.
% The column width is equals to the width of the word that you put.
% Read the manual about acronym package for more examples:
%   http://linorg.usp.br/CTAN/macros/latex/contrib/acronym/acronym.pdf

\acro{SPA}{Single Page Application}
\acro{JSON}{Javascript Object Notation}
\acro{PHP}{PHP: Hypertext Preprocessor}
\acro{SaaS}{Software as a Service}
\acro{ERP}{Enterprise Resource Planning}
\acro{QoS}{Quality of Service}
\acro{UML}{Unified Modeling Language}
\acro{MVC}{Model-View-Controller}
\acro{Ajax}{Asynchronous Javascript and XML}
\acro{HTML}{HyperText Markup Language}
\acro{CSS}{Cascading Style Sheets}
\acro{API}{Application Programming Interface}
\acro{DOM}{Document Object Model}
\acro{BPMN}{Business Process Model and Notation}
\acro{REST}{Representational State Transfer}

\end{acronym}

% List of listings
%\lstlistoflistings

\mainmatter

\chapter{Introdução}

\section{Motivação}

Organizar os procedimentos de um processo sempre nos traz vantagens. Apesar de no processo de implantação de um sistema, o mesmo burocratizar o processo, com o tempo temos o retorno da dedicação para a inserção dos dados. Com um certo volume de dados, é possível estruturar informações que num processo manual são difíceis de serem enxergadas. Assim, é possível depender menos das pessoas que organizam o processo, pois o legado de informações não estará mais somente na mente de alguns, mas sim documentado nos dados do sistema.

Além de colaborar na organização, também haverá uma grande colaboração no tempo gasto na gestão. Lidar com muitos papéis e confiar na mente humana para guardar informações, não é uma alternativa muito segura devido ao fato que as pessoas sempre estão sujeitas a sair do processo e levar contigo a experiência obtida. Experiência essa que faz com que os procedimentos sejam executados de forma mais eficiente. Entretanto, com um sistema inteligente, é possível auxiliar e tornar mais ágil a execução das tarefas.


\section{Problema}


De acordo com funcionários ligados ao o setor de pós graduação da UFBA, entrevistados a fim de um maior entendimento do cenário, apesar das semelhanças estruturais, a pós graduação gerida de forma diferencia da graduação. FULANO afirma que devido ao fato de não ter a mesma visibilidade, não tem acesso aos mesmos recursos de gestão acadêmica da graduação. O professores não executam somente atividades dentro da sala de aula, também tem diversas outras ocupações no setor. E muitos procedimentos realizados extra classe ainda se encontram sendo realizados de forma manual, estando mais vulnerável ao erro ou até mesmo à violação do processo. Também ocorre um grande desperdício de tempo pelos professores e gestores da área, devido ao diversos processos ainda realizados de forma manual, sem a devida documentação. Segundo FULANO, também entrevistado, esse tempo perdido implica numa redução da eficiência na sala de aula, pois o professor acaba por ter menos tempo disponível para o planejamento das atividades, o que gera impactos negativos aos alunos.


\section{Objetivos} %<o que deve ser feito/entregue>


Devido aos muitos processos sendo resolvidos de forma manual, propõe-se com solução um sistema moderno, arquitetado para ter funcionamento na web e com um módulo mobile, a fim de fornecer informações de forma rápida e eficiente para os professores através de notificações, já que o acesso à internet móvel é comum entre os possíveis usuários do sistema em questão.
O principal requisito para o sistema seria dispor recursos para reduzir o tempo desperdiçado pelos professores durante as atividades extra classe.


\section{Metodologia} %<como será feito | como resolver o problema apontado inicialmente>


%<analise de literatura | design | implementação | validação>
Baseando-se nas tecnologias gratuitas em alta no cenário atual do desenvolvimento web, dispomos de algumas opções eficientes para a implementação da solução. Dentre as possibilidades, considerando a facilidade para futura manutenção e continuidade do projeto, tende-se a optar por uma tecnologia popular. Como linguagem de programação, adota-se o PHP. A escolha é fundamentada de acordo com a pesquisa da RedMonk de 2015, que evidencia o uso das linguagens de programação de acordo com as discussões no StackOverflow e repositórios no GitHub. É possível constatar a popularidade do PHP no cenário atual com o gráfico da pesquisa citada, na qual o PHP é apresentado na terceira colocação, apenas atrás do lider JavaScript e do segundo colocado, o Java.

\begin{figure}
	\label{fig:graficoRedmonk}
	\includegraphics[width=1\textwidth]{img/grafico_redmonk}
	\caption{Ranking das liguagens de programação no Stack Overflow e Github}
\end{figure}


Ainda assim, para compor a interface do dado projeto, também ocorrerá o uso do líder JavaScript de forma intensa, provendo o elo com o as informações gerenciadas pelo PHP.


Entretanto, não seria inteligente desenvolver um sistema completo sem o auxílio de um framework. Dentre os frameworks disponíveis para PHP, hoje o destaque está com o Laravel, que se encontra no topo dentre os mais utilizados no momento. 


A WebHostFace, uma empresa de hospedagem, compilou várias estatísticas para criar um infográfico mostrando os frameworks PHP mais populares de 2015. Utilizando informações sobre os próprios clientes, o Google Trends, estatísticas de repositórios do GitHub e a pesquisa do SitePoint “Best PHP Frameworks 2015”, a WebHostFace elaborou o seguinte infográfico: 

\begin{figure}
	\label{fig:graficoWebhostface}
	\includegraphics[width=1\textwidth]{img/infografico_webhostface}
	\caption{Infográfico da WebhostFace, exibindo a popularidade dos Frameworks PHP em 2015}
\end{figure}

Assim, tem-se a evidência que o Laravel em 2015 teve a maior popularidade em projetos pessoais e tem a maior comunidade entre os concorrentes, o que o torna uma boa escolha para a escrita de um software que será continuado por terceiros.


Para elaborar os recursos de interface e integrar ao back-end PHP do sistema, será adotado o já conhecido AngularJS, ferramenta sólida e conhecida no aspecto em questão. 


Dados coletados via Google Trends, que propõe comparações entre termos pesquisados, revela a popularidade do AngularJs diante de alguns dos principais concorrentes. O gráfico abaixo evidencia o cenário.


%Como mostra a Figura \ref{fig:graficoGoogleTrendsFerramentasFront}. 
\begin{figure}
	\label{fig:graficoGoogleTrendsFerramentasFront}
	\includegraphics[width=1\textwidth]{img/grafico_ferramentas_front}
	\caption{Gráfico do Google Trends exibindo as pesquisas por ferramentas front-end}
\end{figure}


Junto ao Angular JS, será utilizada a agradável tendência de interface do Material Design da Google, que propõe layouts limpos e otimizados já conhecidos pelos usuários de smartphones Android. 


Para a elaboração da plataforma mobile do projeto, será utilizado o Ionic Framework, muito difundido e bastante pesquisado na área, o que fica evidenciado com o gráfico de pesquisbaixo, coletado via Google Trends buscando por frameworks de desenvolvimento híbrido mobile.


\begin{figure}
	\label{fig:graficoGoogleTrendsFerramentasHibridasMobile}
	\includegraphics[width=1\textwidth]{img/grafico_ferramentas_hibridas_mobile}
	\caption{Gráfico do Google Trends exibindo as pesquisas por Frameworks híbridos mobile}
\end{figure}	

Para layout da interface mobile, também será aplicado a tendência do Material Design, a fim de propor uma harmonia entre o módulo web e mobile para os usuários


\section{Resultados Esperados}


Como fruto de um sistema para pós-graduação da UFBA, espera-se que os professores tenham mais recursos para integrar as atividades e também prover melhores condições para acompanhamento da vida acadêmica dos alunos em questão. Também, que os novos colaboradores que entrarem no processo tenham facilidade de compreender o fluxo do setor ao navegar pelo sistema proposto.


\section{Fora de Escopo}


Interação com os alunos devido às complicações para realizar a integração com o sistema empregado na UFBA, gerenciado pela XXXXXX, o que causaria uma inviabilidade no projeto devido à necessidade de entrega do produto ser mais forte que o tempo necessário para executar o processo de obtenção de acesso ao sistema legado para realizar a integração.


\section{Estrutura do Trabalho}


<breve resumo sobre os capítulos do TCC>
\chapter{Referencial Teórico}


Projetar o desenvolvimento de um software requer muito planejamento, pois as falhas iniciais podem custar bastante caro ou até mesmo inviabilizar a continuação de um projeto. Assim, a escolha da arquitetura ideal para a aplicabilidade é essencial na concepção de um produto de software. 
De todo o modo, sempre busca-se fazer mais com menos. Diante de tal filosofia, temos neste capítulo, uma breve discussão sobre alguns elementos de projeto e arquitetura de software, a fim de contextualizar este trabalho de conclusão de curso.
O capítulo corrente é composto por quatro seções. A \ref{sec:saas} trata de Software como serviço, discutindo alguns elementos do contexto que são relevantes para o trabalho proposto. A \ref{sec:reuso}  discute sobre a empregabilidade do reuso de software. A \ref{sec:modularizacao} seção, trata sobre aspectos envolvidos na modularização dos softwares. Por fim, a \ref{sec:apps_web} aborda as aplicações web, discutindo sobre aspectos relevantes sobre a aplicação web que compõe este trabalho.


 \section{Software como serviço}\label{sec:saas}


Segundo La e Chun \citep{La2009Systematic}, o princípio da definição de Software como um Serviço (Sofware as a Service - SaaS) é um serviço complementar para aplicações da computação em nuvem (cloud computing). No entanto, as áreas não se confundem. SaaS deve ser entendido como um mecanismo de suporte às soluções existentes na cloud. Os SaaS existem justamente para maximizar o reuso de serviços repetidos e não centrais em uma aplicação remota.


Como vantagens, diversos fatores podem ser favoráveis para a adoção de um SaaS, como custo e manutenção dentre outros fatores aplicáveis a soluções específicas. Lechesaet al. \citep{LechesaSS11}, quantificam índices sobre os fatores determinantes para adoção ou não de um SaaS voltado para ERP na África do Sul. Os principais fatores determinantes para adoção desse mecanismo de software são sua fluidez quanto à rede e a segurança. Esses fatores estão presentes na aplicação desenvolvida neste trabalho de conclusão de curso.


Devido ao fato de ter um serviço constantemente na nuvem, fica o questionamento sobre a segurança da informação manipulada. Sabe-se que a vulnerabilidade na Web não é restrita ao SaaS, atingindo diversos âmbitos. Rai et al. \cite{journals/corr/RaiSM13} consideram que o avanço da computação em nuvem não é um problema apenas para os serviços Web do ponto de vista da segurança, pois muitos trabalhos na literatura mostram a área como mais um ponto de vulnerabilidade para diversos setores, a exemplo de infraestrutura. No mesmo artigo mencionado de Rai et al. \cite{journals/corr/RaiSM13}, também realizaram-se estudos exploratórios junto a empresas usuárias de serviços em computação em nuvem e consideram que a perspectiva de SaaS também pode fortalecer a segurança nas aplicações de cloud computing, pois o software de autenticação compartilhado por várias aplicações em nuvem, oferece uma melhor padronização e consequente facilidade de prevenção a erros de vulnerabilidade específicas de cada módulo da pesquisa. Esse ponto de vista é fundamenta para a compreensão de SaaS.


A arquitetura de armazenamento de dados de um Saas pode variar de acordo com a necessidade do contexto. Huixin \cite{7586486} descreve possíveis modelagens nesse sentido. Essa abordagem pode ser com um banco de dados único, fazendo com que diferentes clientes compartilhem o mesmo banco, diferindo os dados através de controle de usuário, ou isolando os diferentes clientes através de bancos de dados exclusivos para cada um. Esse fator também pode ser combinado com a arquitetura da aplicação, caso ofereça aplicação única para todos os clientes ou aplicação compartilhada. Diante das possíveis abordagens, a modelagem de dados do software pode ser decidida pela regra de negócio. Este trabalho optou por aplicação única e banco de dados compartilhado.

Devido ao conceito particular de obtenção de software, tanto pela visão do cliente como pela visão do vendedor, é necessário compreender os diversos aspectos que podem ser relevantes ao orçar um Saas. O recente trabalho de T. Kaur et al. \citep{6949281} orienta um modelo para compor o custo de um Saas. O custo total seria composto pelos fatores que dão suporte ao funcionamento do software. Tais fatores incluem infra-estrutura, configurabilidade, customização, parâmetros de QoS (Quality of service) como escalabilidade, disponibilidade, usabilidade, pontualidade e desempenho da resposta, portabilidade, custo total de propriedade e retorno do investimento. Esses fatores caracterizam o custo de forma eficaz, possibilitando ao fornecedor, prover um Serviço de acordo com a exigência do consumidor em vários pacotes de serviços.


O conceito de software como serviço se aplica neste trabalho de conclusão de curso, pois o mesmo estará disponível na web com alta disponibilidade adotando as características apresentadas para qualquer pessoa que desejar usá-lo. Assim, basta se cadastrar, configurá-lo e usar como um serviço, sem preocupação com a manutenção do mesmo.


\section{Reuso de software}\label{sec:reuso} %CRUISE BOOK CAPITULO 2


De acordo com o livro Software pratical reuse \citep{Ezran2002}, o reuso de software é a utilização de qualquer informação que um desenvolvedor pode necessitar no processo de criação de software. O livro de Basili e Rombach \citep{Basili1991} define reutilização de software como o uso de tudo o que está associado a um projeto de conhecimento.
Assim, o objetivo da reutilização de software é reciclar o design, código e outros componentes de um produto de software e assim reduzir o custo, o tempo e melhorar a qualidade do produto.
Segundo Keswani et al. \cite{6783445}, o componente reutilizável de software pode ser qualquer parte de seu desenvolvimento, como um fragmento de código, design, casos de teste, ou até mesmo a especificação de requisitos de uma funcionalidade do software. 

O reuso de software pode ter impacto positivo em diversos aspectos do software, vejamos alguns, conforme apresentados no C.R.U.I.S.E Book \citep{cruise2007} :

\begin{itemize}

\item Qualidade: As correções de erro tornam-se úteis em todos os locais em que ocorreu, padronizando e facilitando a manutenção.

\item Produtividade: O ganho de produtividade é alcançado devido ao menor número de artefatos desenvolvido. Isso resulta em menor esforço de teste e também análise e design, reduzindo custos.

\item Confiabilidade: A utilização de componentes bem testados aumenta a
confiança no software. Além disso, a utilização de um mesmo componente em vários sistemas, aumenta a possibilidade de detecção de erros e reforça a confiança no componente.

\item Redução do Esforço: A reutilização de software proporciona uma redução do tempo de desenvolvimento, o que reduz o tempo necessário para o produto ser disponibilizado no mercado para trazer rentabilidade.

\item Trabalho redundante e tempo de desenvolvimento: Desenvolver um sistema do
zero significa desenvolvimento redundante de muitos componentes, como requisitos, especificações, casos de uso, arquitetura, etc. Isso pode ser evitado quando estes estão disponíveis como componentes reutilizáveis e podem ser compartilhados, resultando em um processo de desenvolvimento otimizado.

\item Documentação: Embora a documentação seja muito importante para a
manutenção de um sistema, muitas vezes é negligenciada. A reutilização de componentes de software reduz a quantidade de documentação a ser escrita, entretanto depende da qualidade do que está escrito. Assim, apenas a estrutura do sistema e os novos artefatos desenvolvidos necessitam ser documentados.

\item Custo de manutenção: Menos defeitos e manutenções são esperados quando tem-se comprovada a qualidade dos componentes utilizados.

\item Tamanho da equipe: É comum haver casos em que a equipe de desenvolvimento sofre sobrecarga. Entretanto, dobrar o tamanho da equipe de desenvolvimento não necessariamente duplica produtividade. Se muitos componentes podem ser reutilizados, é possível desenvolver com equipes menores, levando a melhor comunicação e aumento da produtividade.

\end{itemize}

Apesar dos benefícios da reutilização de software, ela não é suficientemente aproveitada. Existem fatores que influenciam direta ou indiretamente na sua adoção. Esses fatores podem ser de aspecto gerencial, organizacional, econômico, conceitual ou técnico. Veremos a seguir alguns aspectos que podem gerar conflito com a cultura de reuso de software, segundo o C.R.U.I.S.E Book \citep{cruise2007} :
%(Sametinger, 1997). REVER

\begin{itemize}
	
\item Falta de apoio da gestão: Como a reutilização de software gera custos iniciais,
a medida pode não ser amplamente alcançada em uma organização sem o apoio de alto nível de gestão. Os gestores têm de ser informados sobre os custos iniciais e serem convencidos sobre economias futuras.

\item Gerenciamento do Projeto: Gerenciar projetos tradicionais é uma tarefa árdua, principalmente, os que praticam a reutilização de software. Utilizando a técnica em larga escala, tem-se impacto sobre todo o ciclo de vida do software.

\item Estruturas organizacionais inadequadas: As estruturas organizacionais devem
considerar diferentes necessidades que surgem quando a reutilização em larga escala está sendo adotada. Por exemplo, uma equipe particionada pode ser alocada somente para desenvolver, manter e certificar componentes reutilizáveis de software.

\item Incentivos de gestão: É comum a falta de incentivo para deixar os desenvolvedores gastarem tempo elaborando componentes do sistemas. A produtividade é muitas vezes medida apenas no tempo necessário para concluir um projeto. Assim, fazer qualquer trabalho além disso, embora benéfico para a empresa como um todo, diminui o seu sucesso. Mesmo quando os componentes reutilizáveis são utilizados, os benefícios obtidos são uma pequena fração do que poderia ser alcançado caso houvesse reutilização explícita, planejada e organizada.

\item Dificuldade de encontrar software reutilizável: Para reutilizar os componentes, devem existir formas eficientes de busca. Além disso, é importante ter um repositório bem organizado contendo componentes com um eficiente meio de acesso.

\item Não reutilização do software encontrado. O acesso fácil ao software existente
não necessariamente aumentar a reutilização. Os componentes reutilizáveis devem ser cuidadosamente especificados, projetados, implementados e documentados, pois em alguns casos, modificar e adaptar o código  pode ser mais custoso que a programação da funcionalidade necessária a partir do zero.

\item Modificação: É muito difícil encontrar um componente que funcione
exatamente da mesma maneira que queremos. Desta forma, são necessárias modificações e devem existir formas de determinar os seus efeitos sobre o componente.


\end{itemize}


A definição do reuso de software pode ser associada a este trabalho, pois ainda que como um serviço, estaremos adotando uma aplicação única para servir aos mais diversos softwares de uma empresa, por exemplo. Logo, ao adotar o projeto, pode ser interessante aplicar algumas métricas para tomar conhecimento de possíveis vantanjens como as citadas nessa seção, a exemplo de economia de tempo de desenvolvimento, e custo do projeto.


%Outra diretriz importante para a reutilização de software é reduzir o risco na criação de novos softwares. O risco tende a ser bastante reduzido se os componentes que estão sendo reutilizados têm as documentação, interfaces necessárias e devidamente testadas, fatores que contibruem para uma fácil integração.
%De acordo com Keswani et al. \citep{6783445}, para o reuso de software dar retornos apropriados, o processo deve ser sistemático e planejado. Qualquer organização que implemente a reutilização de software deve identificar os melhores métodos e estratégias de reutilização para obter a máxima produtividade. A reutilização de software ajuda a evitar software de engenharia a partir do zero, pois usa módulos de software existentes. A reutilização de software, embora seja uma tarefa difícil, especialmente para softwares antigos sem padrões de projeto, pode melhorar significativamente a produtividade e a qualidade de um produto de software. Embora a reutilização de software não seja um novo campo, ela pode dar grandes retornos em curto período de tempo.


\section{Modularização}\label{sec:modularizacao} %artigo de claudio pagina 222 introdução


%A modularidade vem desempenhando um papel predominante estágios emergentes das disciplinas de arquitetura de software [13]. Engenheiros de software consideram modularidade como princípio base na comparação entre arquiteturas alternativas  e arquitetura degeneração [9]. De fato, os engenheiros de software são incentivados a arquitecturas, baseando-se numa multiplicidade de mecanismos de modularidade disponíveis em: 
%(i) Linguagens de descrição de arquitetura (ADLs), como ACME [8], 
%(ii) catálogos de arquitetônicos [2, 13], e 
%(iii) conhecem bem princípios de alto nível, como interfaces de componentes estreitos, acoplamento arquitectónico reduzido e semelhantes.


Conforme é frisado por Wickramaarachchi e Lai \citep{7062705}, o conceito de modularização na indústria de software tem uma longa história e tem sido utilizado para melhorar o processo de desenvolvimento de software em diferentes estágios. Os principais conceitos por trás da modularização do software foram introduzidos por pesquisadores pioneiros há quarenta anos, com uma notável contribuição feita por Melvin Conway e David Parnas, que tem representação notável na engenharia de software.


Modularizar um software é um bom padrão a ser adotado. Segundo Wickramaarachchi e Lai \citep{7062705}, a modularização é importante na identificação de dependências e reduz as dificuldades diante de uma possível necessidade de grandes alterações. De uma perspectiva da engenharia de software, uma modularização geralmente tem várias vantagens, tais como: tornar a complexidade do software mais gerenciável, facilitar o trabalho paralelo e tornar o software mais maleável para acomodar o futuro incerto que um software pode ter. O objetivo final da modularização do software é aumentar a produtividade ea qualidade do software. Tal conceito encontra-se bastante difundido e estái incorporado em linguagens de programação e ferramentas de software. O presente trabalho favorece ao uso da modularização de um software e até mesmo pode ser considerado um módulo a ser acoplado a qualquer software, mediante a compatibilidade.


Ao realizar a adoção do trabalho proposto, fica bastante evidente a "responsabilidade" do mesmo no escopo do projeto. Assim, é possível tratá-lo como um módulo do projeto que o usa, mesmo que seja consumido como um serviço.


\section{Aplicações web}\label{sec:apps_web}


A popularidade das soluções Web aumentou exponencialmente na última década e todos os dias cresce o número de pessoas usuárias desse tipo de software. E seguindo um padrão próprio, Kumar et al. \citep{7813710} sugerem que para o desenvolvimento web, deve-se manter a prática eficaz de produzir diagramas UML. A abordagem baseada na web oferece uma maneira fácil e eficaz para gerenciar e controlar o processo de desenvolvimento por meio de artefatos de modelagem. Tal abordagem pode ser usada quando há uma exigência de lidar com mudanças muito rápidas e grandes em requisitos de forma muito eficaz em muito menos tempo, gerando assim um menor impacto. 


Para atender à fomentada demanda de aplicativos web, é necessário adotar métodos de desenvolvimentos que sejam ágeis, eficientes e de fácil manutenção. Yu Ping et al. \cite{1372143} propõem o uso do modelo MVC (Model, View e Controller) no desenvolvimento para softwares web. O modelo apresentado tornou-se um padrão popular e divide o software em camadas com propósito definido, tornando-o de mais fácil manutenção.


O Ajax (Asynchronous Javascript and XML) revolucionou a web. Conforme demonstrado por Yuping \citep{6845605}, ao usar a tecnologia Ajax, podemos enriquecer a experiência do usuário em aplicações baseadas em navegador de internet, e fornecer uma variedade de aplicações interativas para atender às necessidade de humanização das aplicações.
Os aplicativos Ajax em execução no navegador se comunicam com um servidor Web de forma assíncrona e atualizam apenas uma parte da página.


No artigo de Tesarik et al. \citep{4664349}, temos que o design de software SPA (Single page application) é uma maneira interessante de criar um software Web numa uma única página. Essa solução de página única sem navegação funciona apenas com base em técnicas dinâmicas e assíncronas, como o AJAX (citado no parágrafo anterior). No entanto, esta abordagem coloca o desenvolvedor antes de alguns desafios substanciais.
Para projetar a interface do usuário que mostra as informações, é importante projetar corretamente a tela para manipular os dados do aplicativo numa única página. O design da página deve ser elaborado para maximizar a decomposição da página em componentes distintos que encapsulam os principais casos de uso.
Também recomenda-se que para elaborar os artefatos visuais, explore os mais ricos recursos na implementação, como HTML5, JavaScript, Ajax, CSS3, e outras tecnologias que se apliquem. Entretanto, desenvolver uma rica interface com o uso de diversas tecnologias/frameworks pode ocasionar um esforço maior para explorar as possibilidades.
Tais componentes da interface de uma aplicação SPA normalmente são alimentados mediante o consumo de uma API Rest via requisições AJAX. Assim, é possível particionar as responsabilidades de processamento do software entre cliente e servidor.


Como é possível denotar no trabalho de Salvadori e Siqueire \citep{7195633}, REST é uma arquitetura muito popular para integração de aplicativos web, que permite compartilhar e reutilizar informações através de sistemas. Aplicações de grande escala baseadas em no REST deve ser implementadas utilizando estratégias e mecanismos para produzir sistemas que sejam fáceis de desenvolver, reutilizar e manter. As interfaces de integração fornecidas por esses sistemas, chamadas de Web API, têm uma influência importante nas características da implementação resultante, pois a sua forma de resposta deve ser casada com a interface que o consome.
 
 
Os elementos comentados nessa seção encontram-se presentes neste trabalho de conclusão de curso. Juntos, montam a estrutura tecnológica necessária aliada à arquitetura adotada. Esses elementos seguem tendências atuais dos softwares web, fazendo com que esse trabalho esteja composto por tecnologias modernas que estão com boa aceitação no mercado.

% %% RiSE Latex Template - version 0.5
%%
%% RiSE's latex template for thesis and dissertations
%% http://risetemplate.sourceforge.net
%%
%% (c) 2012 Yguaratã Cerqueira Cavalcanti (yguarata@gmail.com)
%%          Vinicius Cardoso Garcia (vinicius.garcia@gmail.com)
%%
%% This document was initially based on UFPEThesis template, from Paulo Gustavo
%% S. Fonseca.
%%
%% ACKNOWLEDGEMENTS
%%
%% We would like to thanks the RiSE's researchers community, the 
%% students from Federal University of Pernambuco, and other users that have
%% been contributing to this projects with comments and patches.
%%
%% GENERAL INSTRUCTIONS
%%
%% We strongly recommend you to compile your documents using pdflatex command.
%% It is also recommend use the texlipse plugin for Eclipse to edit your documents.
%%
%% Options for \documentclass command:
%%         * Idiom
%%           pt   - Portguese (default)
%%           en   - English
%%
%%         * Text type
%%           bsc  - B.Sc. Thesis
%%           msc  - M.Sc. Thesis (default)
%%           qual - PHD qualification (not tested yet)
%%           prop - PHD proposal (not tested yet)
%%           phd  - PHD thesis
%%
%%         * Media
%%           scr  - to eletronic version (PDF) / see the users guide
%%
%%         * Pagination
%%           oneside - unique face press
%%           twoside - two faces press
%%
%%		   * Line spacing
%%           singlespacing  - the same as using \linespread{1}
%%           onehalfspacing - the same as using \linespread{1.3}
%%           doublespacing  - the same as using \linespread{1.6}
%%
%% Reference commands. Use the following commands to make references in your
%% text:
%%          \figref  -- for Figure reference
%%          \tabref  -- for Table reference
%%          \eqnref  -- for equation reference
%%          \chapref -- for chapter reference
%%          \secref  -- for section reference
%%          \appref  -- for appendix reference
%%          \axiref  -- for axiom reference
%%          \conjref -- for conjecture reference
%%          \defref  -- for definition reference
%%          \lemref  -- for lemma reference
%%          \theoref -- for theorem reference
%%          \corref  -- for corollary reference
%%          \propref -- for proprosition reference
%%          \pgref   -- for page reference
%%
%%          Example: See \chapref{chap:introduction}. It will produce 
%%                   'See Chapter 1', in case of English language.

\documentclass[pt,twoside,onehalfspacing,bsc]{risethesis}

\usepackage[sort,square]{natbib} %package for bibliography
\bibliographystyle{ieeetr} %authordate1

\usepackage[utf8]{inputenc}
\usepackage[brazilian]{babel}
\usepackage[T1]{fontenc}

\usepackage{graphicx}

%% Change the following pdf author attribute name to your name.
\usepackage[linkcolor=blue,citecolor=blue,urlcolor=blue,colorlinks,pdfpagelabels,pdftitle={Bruno Cabral's Bachelor Thesis},pdfauthor={Bruno Cabral}]{hyperref}

\address{SALVADOR}

\universitypt{Universidade Federal da Bahia}
\universityen{Federal University of Bahia}

\departmentpt{Depertamento de Ciência da Computação}
\departmenten{Computer Science Department}

\programpt{Programa Multiinstitucional de Pós-graduação em Ciência da Computação}
\programen{Graduate in Computer Science}

\majorfieldpt{Ciência da Computação}
\majorfielden{Computer Science}

\title{Sistema de apoio à Pós graduação - UFBA}
\date{Outubro/2016}

\author{Victor de Azevedo Nunes}
\adviser{Ivan do Carmo Machado}

\begin{document}

\frontmatter
\frontpage
\presentationpage

\begin{dedicatory}
Eu dedico esta dissertação...
%I dedicate this dissertation to my family, girlfriend, friends and
%professors who gave me all necessary support to get here.
\end{dedicatory}

\acknowledgements
Meus agradecimentos...

\begin{epigraph}[]{Edward V Berard}
Walking on water and developing software from a specification are easy if both are frozen
\end{epigraph}

\resumo
% Escreva seu resumo no arquivo resumo.tex
Este trabalho de conclusão de curso utiliza elementos da engenharia de software para propor um software como serviço a fim de otimizar o processo de construção e manutenção dos softwares. Assim, o objetivo deste SaaS é gerenciar as permissões de acesso de sistemas cliente, provendo o reuso de software e padronizando as soluções. O sistema proposto neste trabalho encontra-se implementado e disponível no Github, e traz fundamentos sobre a arquitetura e tecnologias utilizadas, bem como avaliações sobre possibilidades de utilização do mesmo em diversos ambientes, como Web e mobile.

\begin{keywords}
Software; Reuso; SaaS; Web; PHP
\end{keywords}

\abstract
% Write your abstract in a file called abstract.tex
My abstract...

\begin{keywords}
key words...
\end{keywords}

% Summary (tables of contents)
\tableofcontents

% List of figures
\listoffigures

% List of tables
\listoftables

% List of acronyms
% Acronyms manual: http://linorg.usp.br/CTAN/macros/latex/contrib/acronym/acronym.pdf
\listofacronyms
\begin{acronym}[ACRONYM] 
% Change the word ACRONYM above to change the acronym column width.
% The column width is equals to the width of the word that you put.
% Read the manual about acronym package for more examples:
%   http://linorg.usp.br/CTAN/macros/latex/contrib/acronym/acronym.pdf

\acro{SPA}{Single Page Application}
\acro{JSON}{Javascript Object Notation}
\acro{PHP}{PHP: Hypertext Preprocessor}
\acro{SaaS}{Software as a Service}
\acro{ERP}{Enterprise Resource Planning}
\acro{QoS}{Quality of Service}
\acro{UML}{Unified Modeling Language}
\acro{MVC}{Model-View-Controller}
\acro{Ajax}{Asynchronous Javascript and XML}
\acro{HTML}{HyperText Markup Language}
\acro{CSS}{Cascading Style Sheets}
\acro{API}{Application Programming Interface}
\acro{DOM}{Document Object Model}
\acro{BPMN}{Business Process Model and Notation}
\acro{REST}{Representational State Transfer}

\end{acronym}

% List of listings
%\lstlistoflistings

\mainmatter

\chapter{Introdução}

\section{Motivação}

Organizar os procedimentos de um processo sempre nos traz vantagens. Apesar de no processo de implantação de um sistema, o mesmo burocratizar o processo, com o tempo temos o retorno da dedicação para a inserção dos dados. Com um certo volume de dados, é possível estruturar informações que num processo manual são difíceis de serem enxergadas. Assim, é possível depender menos das pessoas que organizam o processo, pois o legado de informações não estará mais somente na mente de alguns, mas sim documentado nos dados do sistema.

Além de colaborar na organização, também haverá uma grande colaboração no tempo gasto na gestão. Lidar com muitos papéis e confiar na mente humana para guardar informações, não é uma alternativa muito segura devido ao fato que as pessoas sempre estão sujeitas a sair do processo e levar contigo a experiência obtida. Experiência essa que faz com que os procedimentos sejam executados de forma mais eficiente. Entretanto, com um sistema inteligente, é possível auxiliar e tornar mais ágil a execução das tarefas.


\section{Problema}


De acordo com funcionários ligados ao o setor de pós graduação da UFBA, entrevistados a fim de um maior entendimento do cenário, apesar das semelhanças estruturais, a pós graduação gerida de forma diferencia da graduação. FULANO afirma que devido ao fato de não ter a mesma visibilidade, não tem acesso aos mesmos recursos de gestão acadêmica da graduação. O professores não executam somente atividades dentro da sala de aula, também tem diversas outras ocupações no setor. E muitos procedimentos realizados extra classe ainda se encontram sendo realizados de forma manual, estando mais vulnerável ao erro ou até mesmo à violação do processo. Também ocorre um grande desperdício de tempo pelos professores e gestores da área, devido ao diversos processos ainda realizados de forma manual, sem a devida documentação. Segundo FULANO, também entrevistado, esse tempo perdido implica numa redução da eficiência na sala de aula, pois o professor acaba por ter menos tempo disponível para o planejamento das atividades, o que gera impactos negativos aos alunos.


\section{Objetivos} %<o que deve ser feito/entregue>


Devido aos muitos processos sendo resolvidos de forma manual, propõe-se com solução um sistema moderno, arquitetado para ter funcionamento na web e com um módulo mobile, a fim de fornecer informações de forma rápida e eficiente para os professores através de notificações, já que o acesso à internet móvel é comum entre os possíveis usuários do sistema em questão.
O principal requisito para o sistema seria dispor recursos para reduzir o tempo desperdiçado pelos professores durante as atividades extra classe.


\section{Metodologia} %<como será feito | como resolver o problema apontado inicialmente>


%<analise de literatura | design | implementação | validação>
Baseando-se nas tecnologias gratuitas em alta no cenário atual do desenvolvimento web, dispomos de algumas opções eficientes para a implementação da solução. Dentre as possibilidades, considerando a facilidade para futura manutenção e continuidade do projeto, tende-se a optar por uma tecnologia popular. Como linguagem de programação, adota-se o PHP. A escolha é fundamentada de acordo com a pesquisa da RedMonk de 2015, que evidencia o uso das linguagens de programação de acordo com as discussões no StackOverflow e repositórios no GitHub. É possível constatar a popularidade do PHP no cenário atual com o gráfico da pesquisa citada, na qual o PHP é apresentado na terceira colocação, apenas atrás do lider JavaScript e do segundo colocado, o Java.

\begin{figure}
	\label{fig:graficoRedmonk}
	\includegraphics[width=1\textwidth]{img/grafico_redmonk}
	\caption{Ranking das liguagens de programação no Stack Overflow e Github}
\end{figure}


Ainda assim, para compor a interface do dado projeto, também ocorrerá o uso do líder JavaScript de forma intensa, provendo o elo com o as informações gerenciadas pelo PHP.


Entretanto, não seria inteligente desenvolver um sistema completo sem o auxílio de um framework. Dentre os frameworks disponíveis para PHP, hoje o destaque está com o Laravel, que se encontra no topo dentre os mais utilizados no momento. 


A WebHostFace, uma empresa de hospedagem, compilou várias estatísticas para criar um infográfico mostrando os frameworks PHP mais populares de 2015. Utilizando informações sobre os próprios clientes, o Google Trends, estatísticas de repositórios do GitHub e a pesquisa do SitePoint “Best PHP Frameworks 2015”, a WebHostFace elaborou o seguinte infográfico: 

\begin{figure}
	\label{fig:graficoWebhostface}
	\includegraphics[width=1\textwidth]{img/infografico_webhostface}
	\caption{Infográfico da WebhostFace, exibindo a popularidade dos Frameworks PHP em 2015}
\end{figure}

Assim, tem-se a evidência que o Laravel em 2015 teve a maior popularidade em projetos pessoais e tem a maior comunidade entre os concorrentes, o que o torna uma boa escolha para a escrita de um software que será continuado por terceiros.


Para elaborar os recursos de interface e integrar ao back-end PHP do sistema, será adotado o já conhecido AngularJS, ferramenta sólida e conhecida no aspecto em questão. 


Dados coletados via Google Trends, que propõe comparações entre termos pesquisados, revela a popularidade do AngularJs diante de alguns dos principais concorrentes. O gráfico abaixo evidencia o cenário.


%Como mostra a Figura \ref{fig:graficoGoogleTrendsFerramentasFront}. 
\begin{figure}
	\label{fig:graficoGoogleTrendsFerramentasFront}
	\includegraphics[width=1\textwidth]{img/grafico_ferramentas_front}
	\caption{Gráfico do Google Trends exibindo as pesquisas por ferramentas front-end}
\end{figure}


Junto ao Angular JS, será utilizada a agradável tendência de interface do Material Design da Google, que propõe layouts limpos e otimizados já conhecidos pelos usuários de smartphones Android. 


Para a elaboração da plataforma mobile do projeto, será utilizado o Ionic Framework, muito difundido e bastante pesquisado na área, o que fica evidenciado com o gráfico de pesquisbaixo, coletado via Google Trends buscando por frameworks de desenvolvimento híbrido mobile.


\begin{figure}
	\label{fig:graficoGoogleTrendsFerramentasHibridasMobile}
	\includegraphics[width=1\textwidth]{img/grafico_ferramentas_hibridas_mobile}
	\caption{Gráfico do Google Trends exibindo as pesquisas por Frameworks híbridos mobile}
\end{figure}	

Para layout da interface mobile, também será aplicado a tendência do Material Design, a fim de propor uma harmonia entre o módulo web e mobile para os usuários


\section{Resultados Esperados}


Como fruto de um sistema para pós-graduação da UFBA, espera-se que os professores tenham mais recursos para integrar as atividades e também prover melhores condições para acompanhamento da vida acadêmica dos alunos em questão. Também, que os novos colaboradores que entrarem no processo tenham facilidade de compreender o fluxo do setor ao navegar pelo sistema proposto.


\section{Fora de Escopo}


Interação com os alunos devido às complicações para realizar a integração com o sistema empregado na UFBA, gerenciado pela XXXXXX, o que causaria uma inviabilidade no projeto devido à necessidade de entrega do produto ser mais forte que o tempo necessário para executar o processo de obtenção de acesso ao sistema legado para realizar a integração.


\section{Estrutura do Trabalho}


<breve resumo sobre os capítulos do TCC>
\chapter{Referencial Teórico}


Projetar o desenvolvimento de um software requer muito planejamento, pois as falhas iniciais podem custar bastante caro ou até mesmo inviabilizar a continuação de um projeto. Assim, a escolha da arquitetura ideal para a aplicabilidade é essencial na concepção de um produto de software. 
De todo o modo, sempre busca-se fazer mais com menos. Diante de tal filosofia, temos neste capítulo, uma breve discussão sobre alguns elementos de projeto e arquitetura de software, a fim de contextualizar este trabalho de conclusão de curso.
O capítulo corrente é composto por quatro seções. A \ref{sec:saas} trata de Software como serviço, discutindo alguns elementos do contexto que são relevantes para o trabalho proposto. A \ref{sec:reuso}  discute sobre a empregabilidade do reuso de software. A \ref{sec:modularizacao} seção, trata sobre aspectos envolvidos na modularização dos softwares. Por fim, a \ref{sec:apps_web} aborda as aplicações web, discutindo sobre aspectos relevantes sobre a aplicação web que compõe este trabalho.


 \section{Software como serviço}\label{sec:saas}


Segundo La e Chun \citep{La2009Systematic}, o princípio da definição de Software como um Serviço (Sofware as a Service - SaaS) é um serviço complementar para aplicações da computação em nuvem (cloud computing). No entanto, as áreas não se confundem. SaaS deve ser entendido como um mecanismo de suporte às soluções existentes na cloud. Os SaaS existem justamente para maximizar o reuso de serviços repetidos e não centrais em uma aplicação remota.


Como vantagens, diversos fatores podem ser favoráveis para a adoção de um SaaS, como custo e manutenção dentre outros fatores aplicáveis a soluções específicas. Lechesaet al. \citep{LechesaSS11}, quantificam índices sobre os fatores determinantes para adoção ou não de um SaaS voltado para ERP na África do Sul. Os principais fatores determinantes para adoção desse mecanismo de software são sua fluidez quanto à rede e a segurança. Esses fatores estão presentes na aplicação desenvolvida neste trabalho de conclusão de curso.


Devido ao fato de ter um serviço constantemente na nuvem, fica o questionamento sobre a segurança da informação manipulada. Sabe-se que a vulnerabilidade na Web não é restrita ao SaaS, atingindo diversos âmbitos. Rai et al. \cite{journals/corr/RaiSM13} consideram que o avanço da computação em nuvem não é um problema apenas para os serviços Web do ponto de vista da segurança, pois muitos trabalhos na literatura mostram a área como mais um ponto de vulnerabilidade para diversos setores, a exemplo de infraestrutura. No mesmo artigo mencionado de Rai et al. \cite{journals/corr/RaiSM13}, também realizaram-se estudos exploratórios junto a empresas usuárias de serviços em computação em nuvem e consideram que a perspectiva de SaaS também pode fortalecer a segurança nas aplicações de cloud computing, pois o software de autenticação compartilhado por várias aplicações em nuvem, oferece uma melhor padronização e consequente facilidade de prevenção a erros de vulnerabilidade específicas de cada módulo da pesquisa. Esse ponto de vista é fundamenta para a compreensão de SaaS.


A arquitetura de armazenamento de dados de um Saas pode variar de acordo com a necessidade do contexto. Huixin \cite{7586486} descreve possíveis modelagens nesse sentido. Essa abordagem pode ser com um banco de dados único, fazendo com que diferentes clientes compartilhem o mesmo banco, diferindo os dados através de controle de usuário, ou isolando os diferentes clientes através de bancos de dados exclusivos para cada um. Esse fator também pode ser combinado com a arquitetura da aplicação, caso ofereça aplicação única para todos os clientes ou aplicação compartilhada. Diante das possíveis abordagens, a modelagem de dados do software pode ser decidida pela regra de negócio. Este trabalho optou por aplicação única e banco de dados compartilhado.

Devido ao conceito particular de obtenção de software, tanto pela visão do cliente como pela visão do vendedor, é necessário compreender os diversos aspectos que podem ser relevantes ao orçar um Saas. O recente trabalho de T. Kaur et al. \citep{6949281} orienta um modelo para compor o custo de um Saas. O custo total seria composto pelos fatores que dão suporte ao funcionamento do software. Tais fatores incluem infra-estrutura, configurabilidade, customização, parâmetros de QoS (Quality of service) como escalabilidade, disponibilidade, usabilidade, pontualidade e desempenho da resposta, portabilidade, custo total de propriedade e retorno do investimento. Esses fatores caracterizam o custo de forma eficaz, possibilitando ao fornecedor, prover um Serviço de acordo com a exigência do consumidor em vários pacotes de serviços.


O conceito de software como serviço se aplica neste trabalho de conclusão de curso, pois o mesmo estará disponível na web com alta disponibilidade adotando as características apresentadas para qualquer pessoa que desejar usá-lo. Assim, basta se cadastrar, configurá-lo e usar como um serviço, sem preocupação com a manutenção do mesmo.


\section{Reuso de software}\label{sec:reuso} %CRUISE BOOK CAPITULO 2


De acordo com o livro Software pratical reuse \citep{Ezran2002}, o reuso de software é a utilização de qualquer informação que um desenvolvedor pode necessitar no processo de criação de software. O livro de Basili e Rombach \citep{Basili1991} define reutilização de software como o uso de tudo o que está associado a um projeto de conhecimento.
Assim, o objetivo da reutilização de software é reciclar o design, código e outros componentes de um produto de software e assim reduzir o custo, o tempo e melhorar a qualidade do produto.
Segundo Keswani et al. \cite{6783445}, o componente reutilizável de software pode ser qualquer parte de seu desenvolvimento, como um fragmento de código, design, casos de teste, ou até mesmo a especificação de requisitos de uma funcionalidade do software. 

O reuso de software pode ter impacto positivo em diversos aspectos do software, vejamos alguns, conforme apresentados no C.R.U.I.S.E Book \citep{cruise2007} :

\begin{itemize}

\item Qualidade: As correções de erro tornam-se úteis em todos os locais em que ocorreu, padronizando e facilitando a manutenção.

\item Produtividade: O ganho de produtividade é alcançado devido ao menor número de artefatos desenvolvido. Isso resulta em menor esforço de teste e também análise e design, reduzindo custos.

\item Confiabilidade: A utilização de componentes bem testados aumenta a
confiança no software. Além disso, a utilização de um mesmo componente em vários sistemas, aumenta a possibilidade de detecção de erros e reforça a confiança no componente.

\item Redução do Esforço: A reutilização de software proporciona uma redução do tempo de desenvolvimento, o que reduz o tempo necessário para o produto ser disponibilizado no mercado para trazer rentabilidade.

\item Trabalho redundante e tempo de desenvolvimento: Desenvolver um sistema do
zero significa desenvolvimento redundante de muitos componentes, como requisitos, especificações, casos de uso, arquitetura, etc. Isso pode ser evitado quando estes estão disponíveis como componentes reutilizáveis e podem ser compartilhados, resultando em um processo de desenvolvimento otimizado.

\item Documentação: Embora a documentação seja muito importante para a
manutenção de um sistema, muitas vezes é negligenciada. A reutilização de componentes de software reduz a quantidade de documentação a ser escrita, entretanto depende da qualidade do que está escrito. Assim, apenas a estrutura do sistema e os novos artefatos desenvolvidos necessitam ser documentados.

\item Custo de manutenção: Menos defeitos e manutenções são esperados quando tem-se comprovada a qualidade dos componentes utilizados.

\item Tamanho da equipe: É comum haver casos em que a equipe de desenvolvimento sofre sobrecarga. Entretanto, dobrar o tamanho da equipe de desenvolvimento não necessariamente duplica produtividade. Se muitos componentes podem ser reutilizados, é possível desenvolver com equipes menores, levando a melhor comunicação e aumento da produtividade.

\end{itemize}

Apesar dos benefícios da reutilização de software, ela não é suficientemente aproveitada. Existem fatores que influenciam direta ou indiretamente na sua adoção. Esses fatores podem ser de aspecto gerencial, organizacional, econômico, conceitual ou técnico. Veremos a seguir alguns aspectos que podem gerar conflito com a cultura de reuso de software, segundo o C.R.U.I.S.E Book \citep{cruise2007} :
%(Sametinger, 1997). REVER

\begin{itemize}
	
\item Falta de apoio da gestão: Como a reutilização de software gera custos iniciais,
a medida pode não ser amplamente alcançada em uma organização sem o apoio de alto nível de gestão. Os gestores têm de ser informados sobre os custos iniciais e serem convencidos sobre economias futuras.

\item Gerenciamento do Projeto: Gerenciar projetos tradicionais é uma tarefa árdua, principalmente, os que praticam a reutilização de software. Utilizando a técnica em larga escala, tem-se impacto sobre todo o ciclo de vida do software.

\item Estruturas organizacionais inadequadas: As estruturas organizacionais devem
considerar diferentes necessidades que surgem quando a reutilização em larga escala está sendo adotada. Por exemplo, uma equipe particionada pode ser alocada somente para desenvolver, manter e certificar componentes reutilizáveis de software.

\item Incentivos de gestão: É comum a falta de incentivo para deixar os desenvolvedores gastarem tempo elaborando componentes do sistemas. A produtividade é muitas vezes medida apenas no tempo necessário para concluir um projeto. Assim, fazer qualquer trabalho além disso, embora benéfico para a empresa como um todo, diminui o seu sucesso. Mesmo quando os componentes reutilizáveis são utilizados, os benefícios obtidos são uma pequena fração do que poderia ser alcançado caso houvesse reutilização explícita, planejada e organizada.

\item Dificuldade de encontrar software reutilizável: Para reutilizar os componentes, devem existir formas eficientes de busca. Além disso, é importante ter um repositório bem organizado contendo componentes com um eficiente meio de acesso.

\item Não reutilização do software encontrado. O acesso fácil ao software existente
não necessariamente aumentar a reutilização. Os componentes reutilizáveis devem ser cuidadosamente especificados, projetados, implementados e documentados, pois em alguns casos, modificar e adaptar o código  pode ser mais custoso que a programação da funcionalidade necessária a partir do zero.

\item Modificação: É muito difícil encontrar um componente que funcione
exatamente da mesma maneira que queremos. Desta forma, são necessárias modificações e devem existir formas de determinar os seus efeitos sobre o componente.


\end{itemize}


A definição do reuso de software pode ser associada a este trabalho, pois ainda que como um serviço, estaremos adotando uma aplicação única para servir aos mais diversos softwares de uma empresa, por exemplo. Logo, ao adotar o projeto, pode ser interessante aplicar algumas métricas para tomar conhecimento de possíveis vantanjens como as citadas nessa seção, a exemplo de economia de tempo de desenvolvimento, e custo do projeto.


%Outra diretriz importante para a reutilização de software é reduzir o risco na criação de novos softwares. O risco tende a ser bastante reduzido se os componentes que estão sendo reutilizados têm as documentação, interfaces necessárias e devidamente testadas, fatores que contibruem para uma fácil integração.
%De acordo com Keswani et al. \citep{6783445}, para o reuso de software dar retornos apropriados, o processo deve ser sistemático e planejado. Qualquer organização que implemente a reutilização de software deve identificar os melhores métodos e estratégias de reutilização para obter a máxima produtividade. A reutilização de software ajuda a evitar software de engenharia a partir do zero, pois usa módulos de software existentes. A reutilização de software, embora seja uma tarefa difícil, especialmente para softwares antigos sem padrões de projeto, pode melhorar significativamente a produtividade e a qualidade de um produto de software. Embora a reutilização de software não seja um novo campo, ela pode dar grandes retornos em curto período de tempo.


\section{Modularização}\label{sec:modularizacao} %artigo de claudio pagina 222 introdução


%A modularidade vem desempenhando um papel predominante estágios emergentes das disciplinas de arquitetura de software [13]. Engenheiros de software consideram modularidade como princípio base na comparação entre arquiteturas alternativas  e arquitetura degeneração [9]. De fato, os engenheiros de software são incentivados a arquitecturas, baseando-se numa multiplicidade de mecanismos de modularidade disponíveis em: 
%(i) Linguagens de descrição de arquitetura (ADLs), como ACME [8], 
%(ii) catálogos de arquitetônicos [2, 13], e 
%(iii) conhecem bem princípios de alto nível, como interfaces de componentes estreitos, acoplamento arquitectónico reduzido e semelhantes.


Conforme é frisado por Wickramaarachchi e Lai \citep{7062705}, o conceito de modularização na indústria de software tem uma longa história e tem sido utilizado para melhorar o processo de desenvolvimento de software em diferentes estágios. Os principais conceitos por trás da modularização do software foram introduzidos por pesquisadores pioneiros há quarenta anos, com uma notável contribuição feita por Melvin Conway e David Parnas, que tem representação notável na engenharia de software.


Modularizar um software é um bom padrão a ser adotado. Segundo Wickramaarachchi e Lai \citep{7062705}, a modularização é importante na identificação de dependências e reduz as dificuldades diante de uma possível necessidade de grandes alterações. De uma perspectiva da engenharia de software, uma modularização geralmente tem várias vantagens, tais como: tornar a complexidade do software mais gerenciável, facilitar o trabalho paralelo e tornar o software mais maleável para acomodar o futuro incerto que um software pode ter. O objetivo final da modularização do software é aumentar a produtividade ea qualidade do software. Tal conceito encontra-se bastante difundido e estái incorporado em linguagens de programação e ferramentas de software. O presente trabalho favorece ao uso da modularização de um software e até mesmo pode ser considerado um módulo a ser acoplado a qualquer software, mediante a compatibilidade.


Ao realizar a adoção do trabalho proposto, fica bastante evidente a "responsabilidade" do mesmo no escopo do projeto. Assim, é possível tratá-lo como um módulo do projeto que o usa, mesmo que seja consumido como um serviço.


\section{Aplicações web}\label{sec:apps_web}


A popularidade das soluções Web aumentou exponencialmente na última década e todos os dias cresce o número de pessoas usuárias desse tipo de software. E seguindo um padrão próprio, Kumar et al. \citep{7813710} sugerem que para o desenvolvimento web, deve-se manter a prática eficaz de produzir diagramas UML. A abordagem baseada na web oferece uma maneira fácil e eficaz para gerenciar e controlar o processo de desenvolvimento por meio de artefatos de modelagem. Tal abordagem pode ser usada quando há uma exigência de lidar com mudanças muito rápidas e grandes em requisitos de forma muito eficaz em muito menos tempo, gerando assim um menor impacto. 


Para atender à fomentada demanda de aplicativos web, é necessário adotar métodos de desenvolvimentos que sejam ágeis, eficientes e de fácil manutenção. Yu Ping et al. \cite{1372143} propõem o uso do modelo MVC (Model, View e Controller) no desenvolvimento para softwares web. O modelo apresentado tornou-se um padrão popular e divide o software em camadas com propósito definido, tornando-o de mais fácil manutenção.


O Ajax (Asynchronous Javascript and XML) revolucionou a web. Conforme demonstrado por Yuping \citep{6845605}, ao usar a tecnologia Ajax, podemos enriquecer a experiência do usuário em aplicações baseadas em navegador de internet, e fornecer uma variedade de aplicações interativas para atender às necessidade de humanização das aplicações.
Os aplicativos Ajax em execução no navegador se comunicam com um servidor Web de forma assíncrona e atualizam apenas uma parte da página.


No artigo de Tesarik et al. \citep{4664349}, temos que o design de software SPA (Single page application) é uma maneira interessante de criar um software Web numa uma única página. Essa solução de página única sem navegação funciona apenas com base em técnicas dinâmicas e assíncronas, como o AJAX (citado no parágrafo anterior). No entanto, esta abordagem coloca o desenvolvedor antes de alguns desafios substanciais.
Para projetar a interface do usuário que mostra as informações, é importante projetar corretamente a tela para manipular os dados do aplicativo numa única página. O design da página deve ser elaborado para maximizar a decomposição da página em componentes distintos que encapsulam os principais casos de uso.
Também recomenda-se que para elaborar os artefatos visuais, explore os mais ricos recursos na implementação, como HTML5, JavaScript, Ajax, CSS3, e outras tecnologias que se apliquem. Entretanto, desenvolver uma rica interface com o uso de diversas tecnologias/frameworks pode ocasionar um esforço maior para explorar as possibilidades.
Tais componentes da interface de uma aplicação SPA normalmente são alimentados mediante o consumo de uma API Rest via requisições AJAX. Assim, é possível particionar as responsabilidades de processamento do software entre cliente e servidor.


Como é possível denotar no trabalho de Salvadori e Siqueire \citep{7195633}, REST é uma arquitetura muito popular para integração de aplicativos web, que permite compartilhar e reutilizar informações através de sistemas. Aplicações de grande escala baseadas em no REST deve ser implementadas utilizando estratégias e mecanismos para produzir sistemas que sejam fáceis de desenvolver, reutilizar e manter. As interfaces de integração fornecidas por esses sistemas, chamadas de Web API, têm uma influência importante nas características da implementação resultante, pois a sua forma de resposta deve ser casada com a interface que o consome.
 
 
Os elementos comentados nessa seção encontram-se presentes neste trabalho de conclusão de curso. Juntos, montam a estrutura tecnológica necessária aliada à arquitetura adotada. Esses elementos seguem tendências atuais dos softwares web, fazendo com que esse trabalho esteja composto por tecnologias modernas que estão com boa aceitação no mercado.

% %% RiSE Latex Template - version 0.5
%%
%% RiSE's latex template for thesis and dissertations
%% http://risetemplate.sourceforge.net
%%
%% (c) 2012 Yguaratã Cerqueira Cavalcanti (yguarata@gmail.com)
%%          Vinicius Cardoso Garcia (vinicius.garcia@gmail.com)
%%
%% This document was initially based on UFPEThesis template, from Paulo Gustavo
%% S. Fonseca.
%%
%% ACKNOWLEDGEMENTS
%%
%% We would like to thanks the RiSE's researchers community, the 
%% students from Federal University of Pernambuco, and other users that have
%% been contributing to this projects with comments and patches.
%%
%% GENERAL INSTRUCTIONS
%%
%% We strongly recommend you to compile your documents using pdflatex command.
%% It is also recommend use the texlipse plugin for Eclipse to edit your documents.
%%
%% Options for \documentclass command:
%%         * Idiom
%%           pt   - Portguese (default)
%%           en   - English
%%
%%         * Text type
%%           bsc  - B.Sc. Thesis
%%           msc  - M.Sc. Thesis (default)
%%           qual - PHD qualification (not tested yet)
%%           prop - PHD proposal (not tested yet)
%%           phd  - PHD thesis
%%
%%         * Media
%%           scr  - to eletronic version (PDF) / see the users guide
%%
%%         * Pagination
%%           oneside - unique face press
%%           twoside - two faces press
%%
%%		   * Line spacing
%%           singlespacing  - the same as using \linespread{1}
%%           onehalfspacing - the same as using \linespread{1.3}
%%           doublespacing  - the same as using \linespread{1.6}
%%
%% Reference commands. Use the following commands to make references in your
%% text:
%%          \figref  -- for Figure reference
%%          \tabref  -- for Table reference
%%          \eqnref  -- for equation reference
%%          \chapref -- for chapter reference
%%          \secref  -- for section reference
%%          \appref  -- for appendix reference
%%          \axiref  -- for axiom reference
%%          \conjref -- for conjecture reference
%%          \defref  -- for definition reference
%%          \lemref  -- for lemma reference
%%          \theoref -- for theorem reference
%%          \corref  -- for corollary reference
%%          \propref -- for proprosition reference
%%          \pgref   -- for page reference
%%
%%          Example: See \chapref{chap:introduction}. It will produce 
%%                   'See Chapter 1', in case of English language.

\documentclass[pt,twoside,onehalfspacing,bsc]{risethesis}

\usepackage[sort,square]{natbib} %package for bibliography
\bibliographystyle{ieeetr} %authordate1

\usepackage[utf8]{inputenc}
\usepackage[brazilian]{babel}
\usepackage[T1]{fontenc}

\usepackage{graphicx}

%% Change the following pdf author attribute name to your name.
\usepackage[linkcolor=blue,citecolor=blue,urlcolor=blue,colorlinks,pdfpagelabels,pdftitle={Bruno Cabral's Bachelor Thesis},pdfauthor={Bruno Cabral}]{hyperref}

\address{SALVADOR}

\universitypt{Universidade Federal da Bahia}
\universityen{Federal University of Bahia}

\departmentpt{Depertamento de Ciência da Computação}
\departmenten{Computer Science Department}

\programpt{Programa Multiinstitucional de Pós-graduação em Ciência da Computação}
\programen{Graduate in Computer Science}

\majorfieldpt{Ciência da Computação}
\majorfielden{Computer Science}

\title{Sistema de apoio à Pós graduação - UFBA}
\date{Outubro/2016}

\author{Victor de Azevedo Nunes}
\adviser{Ivan do Carmo Machado}

\begin{document}

\frontmatter
\frontpage
\presentationpage

\begin{dedicatory}
Eu dedico esta dissertação...
%I dedicate this dissertation to my family, girlfriend, friends and
%professors who gave me all necessary support to get here.
\end{dedicatory}

\acknowledgements
Meus agradecimentos...

\begin{epigraph}[]{Edward V Berard}
Walking on water and developing software from a specification are easy if both are frozen
\end{epigraph}

\resumo
% Escreva seu resumo no arquivo resumo.tex
Este trabalho de conclusão de curso utiliza elementos da engenharia de software para propor um software como serviço a fim de otimizar o processo de construção e manutenção dos softwares. Assim, o objetivo deste SaaS é gerenciar as permissões de acesso de sistemas cliente, provendo o reuso de software e padronizando as soluções. O sistema proposto neste trabalho encontra-se implementado e disponível no Github, e traz fundamentos sobre a arquitetura e tecnologias utilizadas, bem como avaliações sobre possibilidades de utilização do mesmo em diversos ambientes, como Web e mobile.

\begin{keywords}
Software; Reuso; SaaS; Web; PHP
\end{keywords}

\abstract
% Write your abstract in a file called abstract.tex
My abstract...

\begin{keywords}
key words...
\end{keywords}

% Summary (tables of contents)
\tableofcontents

% List of figures
\listoffigures

% List of tables
\listoftables

% List of acronyms
% Acronyms manual: http://linorg.usp.br/CTAN/macros/latex/contrib/acronym/acronym.pdf
\listofacronyms
\begin{acronym}[ACRONYM] 
% Change the word ACRONYM above to change the acronym column width.
% The column width is equals to the width of the word that you put.
% Read the manual about acronym package for more examples:
%   http://linorg.usp.br/CTAN/macros/latex/contrib/acronym/acronym.pdf

\acro{SPA}{Single Page Application}
\acro{JSON}{Javascript Object Notation}
\acro{PHP}{PHP: Hypertext Preprocessor}
\acro{SaaS}{Software as a Service}
\acro{ERP}{Enterprise Resource Planning}
\acro{QoS}{Quality of Service}
\acro{UML}{Unified Modeling Language}
\acro{MVC}{Model-View-Controller}
\acro{Ajax}{Asynchronous Javascript and XML}
\acro{HTML}{HyperText Markup Language}
\acro{CSS}{Cascading Style Sheets}
\acro{API}{Application Programming Interface}
\acro{DOM}{Document Object Model}
\acro{BPMN}{Business Process Model and Notation}
\acro{REST}{Representational State Transfer}

\end{acronym}

% List of listings
%\lstlistoflistings

\mainmatter

\chapter{Introdução}

\section{Motivação}

Organizar os procedimentos de um processo sempre nos traz vantagens. Apesar de no processo de implantação de um sistema, o mesmo burocratizar o processo, com o tempo temos o retorno da dedicação para a inserção dos dados. Com um certo volume de dados, é possível estruturar informações que num processo manual são difíceis de serem enxergadas. Assim, é possível depender menos das pessoas que organizam o processo, pois o legado de informações não estará mais somente na mente de alguns, mas sim documentado nos dados do sistema.

Além de colaborar na organização, também haverá uma grande colaboração no tempo gasto na gestão. Lidar com muitos papéis e confiar na mente humana para guardar informações, não é uma alternativa muito segura devido ao fato que as pessoas sempre estão sujeitas a sair do processo e levar contigo a experiência obtida. Experiência essa que faz com que os procedimentos sejam executados de forma mais eficiente. Entretanto, com um sistema inteligente, é possível auxiliar e tornar mais ágil a execução das tarefas.


\section{Problema}


De acordo com funcionários ligados ao o setor de pós graduação da UFBA, entrevistados a fim de um maior entendimento do cenário, apesar das semelhanças estruturais, a pós graduação gerida de forma diferencia da graduação. FULANO afirma que devido ao fato de não ter a mesma visibilidade, não tem acesso aos mesmos recursos de gestão acadêmica da graduação. O professores não executam somente atividades dentro da sala de aula, também tem diversas outras ocupações no setor. E muitos procedimentos realizados extra classe ainda se encontram sendo realizados de forma manual, estando mais vulnerável ao erro ou até mesmo à violação do processo. Também ocorre um grande desperdício de tempo pelos professores e gestores da área, devido ao diversos processos ainda realizados de forma manual, sem a devida documentação. Segundo FULANO, também entrevistado, esse tempo perdido implica numa redução da eficiência na sala de aula, pois o professor acaba por ter menos tempo disponível para o planejamento das atividades, o que gera impactos negativos aos alunos.


\section{Objetivos} %<o que deve ser feito/entregue>


Devido aos muitos processos sendo resolvidos de forma manual, propõe-se com solução um sistema moderno, arquitetado para ter funcionamento na web e com um módulo mobile, a fim de fornecer informações de forma rápida e eficiente para os professores através de notificações, já que o acesso à internet móvel é comum entre os possíveis usuários do sistema em questão.
O principal requisito para o sistema seria dispor recursos para reduzir o tempo desperdiçado pelos professores durante as atividades extra classe.


\section{Metodologia} %<como será feito | como resolver o problema apontado inicialmente>


%<analise de literatura | design | implementação | validação>
Baseando-se nas tecnologias gratuitas em alta no cenário atual do desenvolvimento web, dispomos de algumas opções eficientes para a implementação da solução. Dentre as possibilidades, considerando a facilidade para futura manutenção e continuidade do projeto, tende-se a optar por uma tecnologia popular. Como linguagem de programação, adota-se o PHP. A escolha é fundamentada de acordo com a pesquisa da RedMonk de 2015, que evidencia o uso das linguagens de programação de acordo com as discussões no StackOverflow e repositórios no GitHub. É possível constatar a popularidade do PHP no cenário atual com o gráfico da pesquisa citada, na qual o PHP é apresentado na terceira colocação, apenas atrás do lider JavaScript e do segundo colocado, o Java.

\begin{figure}
	\label{fig:graficoRedmonk}
	\includegraphics[width=1\textwidth]{img/grafico_redmonk}
	\caption{Ranking das liguagens de programação no Stack Overflow e Github}
\end{figure}


Ainda assim, para compor a interface do dado projeto, também ocorrerá o uso do líder JavaScript de forma intensa, provendo o elo com o as informações gerenciadas pelo PHP.


Entretanto, não seria inteligente desenvolver um sistema completo sem o auxílio de um framework. Dentre os frameworks disponíveis para PHP, hoje o destaque está com o Laravel, que se encontra no topo dentre os mais utilizados no momento. 


A WebHostFace, uma empresa de hospedagem, compilou várias estatísticas para criar um infográfico mostrando os frameworks PHP mais populares de 2015. Utilizando informações sobre os próprios clientes, o Google Trends, estatísticas de repositórios do GitHub e a pesquisa do SitePoint “Best PHP Frameworks 2015”, a WebHostFace elaborou o seguinte infográfico: 

\begin{figure}
	\label{fig:graficoWebhostface}
	\includegraphics[width=1\textwidth]{img/infografico_webhostface}
	\caption{Infográfico da WebhostFace, exibindo a popularidade dos Frameworks PHP em 2015}
\end{figure}

Assim, tem-se a evidência que o Laravel em 2015 teve a maior popularidade em projetos pessoais e tem a maior comunidade entre os concorrentes, o que o torna uma boa escolha para a escrita de um software que será continuado por terceiros.


Para elaborar os recursos de interface e integrar ao back-end PHP do sistema, será adotado o já conhecido AngularJS, ferramenta sólida e conhecida no aspecto em questão. 


Dados coletados via Google Trends, que propõe comparações entre termos pesquisados, revela a popularidade do AngularJs diante de alguns dos principais concorrentes. O gráfico abaixo evidencia o cenário.


%Como mostra a Figura \ref{fig:graficoGoogleTrendsFerramentasFront}. 
\begin{figure}
	\label{fig:graficoGoogleTrendsFerramentasFront}
	\includegraphics[width=1\textwidth]{img/grafico_ferramentas_front}
	\caption{Gráfico do Google Trends exibindo as pesquisas por ferramentas front-end}
\end{figure}


Junto ao Angular JS, será utilizada a agradável tendência de interface do Material Design da Google, que propõe layouts limpos e otimizados já conhecidos pelos usuários de smartphones Android. 


Para a elaboração da plataforma mobile do projeto, será utilizado o Ionic Framework, muito difundido e bastante pesquisado na área, o que fica evidenciado com o gráfico de pesquisbaixo, coletado via Google Trends buscando por frameworks de desenvolvimento híbrido mobile.


\begin{figure}
	\label{fig:graficoGoogleTrendsFerramentasHibridasMobile}
	\includegraphics[width=1\textwidth]{img/grafico_ferramentas_hibridas_mobile}
	\caption{Gráfico do Google Trends exibindo as pesquisas por Frameworks híbridos mobile}
\end{figure}	

Para layout da interface mobile, também será aplicado a tendência do Material Design, a fim de propor uma harmonia entre o módulo web e mobile para os usuários


\section{Resultados Esperados}


Como fruto de um sistema para pós-graduação da UFBA, espera-se que os professores tenham mais recursos para integrar as atividades e também prover melhores condições para acompanhamento da vida acadêmica dos alunos em questão. Também, que os novos colaboradores que entrarem no processo tenham facilidade de compreender o fluxo do setor ao navegar pelo sistema proposto.


\section{Fora de Escopo}


Interação com os alunos devido às complicações para realizar a integração com o sistema empregado na UFBA, gerenciado pela XXXXXX, o que causaria uma inviabilidade no projeto devido à necessidade de entrega do produto ser mais forte que o tempo necessário para executar o processo de obtenção de acesso ao sistema legado para realizar a integração.


\section{Estrutura do Trabalho}


<breve resumo sobre os capítulos do TCC>
\chapter{Referencial Teórico}


Projetar o desenvolvimento de um software requer muito planejamento, pois as falhas iniciais podem custar bastante caro ou até mesmo inviabilizar a continuação de um projeto. Assim, a escolha da arquitetura ideal para a aplicabilidade é essencial na concepção de um produto de software. 
De todo o modo, sempre busca-se fazer mais com menos. Diante de tal filosofia, temos neste capítulo, uma breve discussão sobre alguns elementos de projeto e arquitetura de software, a fim de contextualizar este trabalho de conclusão de curso.
O capítulo corrente é composto por quatro seções. A \ref{sec:saas} trata de Software como serviço, discutindo alguns elementos do contexto que são relevantes para o trabalho proposto. A \ref{sec:reuso}  discute sobre a empregabilidade do reuso de software. A \ref{sec:modularizacao} seção, trata sobre aspectos envolvidos na modularização dos softwares. Por fim, a \ref{sec:apps_web} aborda as aplicações web, discutindo sobre aspectos relevantes sobre a aplicação web que compõe este trabalho.


 \section{Software como serviço}\label{sec:saas}


Segundo La e Chun \citep{La2009Systematic}, o princípio da definição de Software como um Serviço (Sofware as a Service - SaaS) é um serviço complementar para aplicações da computação em nuvem (cloud computing). No entanto, as áreas não se confundem. SaaS deve ser entendido como um mecanismo de suporte às soluções existentes na cloud. Os SaaS existem justamente para maximizar o reuso de serviços repetidos e não centrais em uma aplicação remota.


Como vantagens, diversos fatores podem ser favoráveis para a adoção de um SaaS, como custo e manutenção dentre outros fatores aplicáveis a soluções específicas. Lechesaet al. \citep{LechesaSS11}, quantificam índices sobre os fatores determinantes para adoção ou não de um SaaS voltado para ERP na África do Sul. Os principais fatores determinantes para adoção desse mecanismo de software são sua fluidez quanto à rede e a segurança. Esses fatores estão presentes na aplicação desenvolvida neste trabalho de conclusão de curso.


Devido ao fato de ter um serviço constantemente na nuvem, fica o questionamento sobre a segurança da informação manipulada. Sabe-se que a vulnerabilidade na Web não é restrita ao SaaS, atingindo diversos âmbitos. Rai et al. \cite{journals/corr/RaiSM13} consideram que o avanço da computação em nuvem não é um problema apenas para os serviços Web do ponto de vista da segurança, pois muitos trabalhos na literatura mostram a área como mais um ponto de vulnerabilidade para diversos setores, a exemplo de infraestrutura. No mesmo artigo mencionado de Rai et al. \cite{journals/corr/RaiSM13}, também realizaram-se estudos exploratórios junto a empresas usuárias de serviços em computação em nuvem e consideram que a perspectiva de SaaS também pode fortalecer a segurança nas aplicações de cloud computing, pois o software de autenticação compartilhado por várias aplicações em nuvem, oferece uma melhor padronização e consequente facilidade de prevenção a erros de vulnerabilidade específicas de cada módulo da pesquisa. Esse ponto de vista é fundamenta para a compreensão de SaaS.


A arquitetura de armazenamento de dados de um Saas pode variar de acordo com a necessidade do contexto. Huixin \cite{7586486} descreve possíveis modelagens nesse sentido. Essa abordagem pode ser com um banco de dados único, fazendo com que diferentes clientes compartilhem o mesmo banco, diferindo os dados através de controle de usuário, ou isolando os diferentes clientes através de bancos de dados exclusivos para cada um. Esse fator também pode ser combinado com a arquitetura da aplicação, caso ofereça aplicação única para todos os clientes ou aplicação compartilhada. Diante das possíveis abordagens, a modelagem de dados do software pode ser decidida pela regra de negócio. Este trabalho optou por aplicação única e banco de dados compartilhado.

Devido ao conceito particular de obtenção de software, tanto pela visão do cliente como pela visão do vendedor, é necessário compreender os diversos aspectos que podem ser relevantes ao orçar um Saas. O recente trabalho de T. Kaur et al. \citep{6949281} orienta um modelo para compor o custo de um Saas. O custo total seria composto pelos fatores que dão suporte ao funcionamento do software. Tais fatores incluem infra-estrutura, configurabilidade, customização, parâmetros de QoS (Quality of service) como escalabilidade, disponibilidade, usabilidade, pontualidade e desempenho da resposta, portabilidade, custo total de propriedade e retorno do investimento. Esses fatores caracterizam o custo de forma eficaz, possibilitando ao fornecedor, prover um Serviço de acordo com a exigência do consumidor em vários pacotes de serviços.


O conceito de software como serviço se aplica neste trabalho de conclusão de curso, pois o mesmo estará disponível na web com alta disponibilidade adotando as características apresentadas para qualquer pessoa que desejar usá-lo. Assim, basta se cadastrar, configurá-lo e usar como um serviço, sem preocupação com a manutenção do mesmo.


\section{Reuso de software}\label{sec:reuso} %CRUISE BOOK CAPITULO 2


De acordo com o livro Software pratical reuse \citep{Ezran2002}, o reuso de software é a utilização de qualquer informação que um desenvolvedor pode necessitar no processo de criação de software. O livro de Basili e Rombach \citep{Basili1991} define reutilização de software como o uso de tudo o que está associado a um projeto de conhecimento.
Assim, o objetivo da reutilização de software é reciclar o design, código e outros componentes de um produto de software e assim reduzir o custo, o tempo e melhorar a qualidade do produto.
Segundo Keswani et al. \cite{6783445}, o componente reutilizável de software pode ser qualquer parte de seu desenvolvimento, como um fragmento de código, design, casos de teste, ou até mesmo a especificação de requisitos de uma funcionalidade do software. 

O reuso de software pode ter impacto positivo em diversos aspectos do software, vejamos alguns, conforme apresentados no C.R.U.I.S.E Book \citep{cruise2007} :

\begin{itemize}

\item Qualidade: As correções de erro tornam-se úteis em todos os locais em que ocorreu, padronizando e facilitando a manutenção.

\item Produtividade: O ganho de produtividade é alcançado devido ao menor número de artefatos desenvolvido. Isso resulta em menor esforço de teste e também análise e design, reduzindo custos.

\item Confiabilidade: A utilização de componentes bem testados aumenta a
confiança no software. Além disso, a utilização de um mesmo componente em vários sistemas, aumenta a possibilidade de detecção de erros e reforça a confiança no componente.

\item Redução do Esforço: A reutilização de software proporciona uma redução do tempo de desenvolvimento, o que reduz o tempo necessário para o produto ser disponibilizado no mercado para trazer rentabilidade.

\item Trabalho redundante e tempo de desenvolvimento: Desenvolver um sistema do
zero significa desenvolvimento redundante de muitos componentes, como requisitos, especificações, casos de uso, arquitetura, etc. Isso pode ser evitado quando estes estão disponíveis como componentes reutilizáveis e podem ser compartilhados, resultando em um processo de desenvolvimento otimizado.

\item Documentação: Embora a documentação seja muito importante para a
manutenção de um sistema, muitas vezes é negligenciada. A reutilização de componentes de software reduz a quantidade de documentação a ser escrita, entretanto depende da qualidade do que está escrito. Assim, apenas a estrutura do sistema e os novos artefatos desenvolvidos necessitam ser documentados.

\item Custo de manutenção: Menos defeitos e manutenções são esperados quando tem-se comprovada a qualidade dos componentes utilizados.

\item Tamanho da equipe: É comum haver casos em que a equipe de desenvolvimento sofre sobrecarga. Entretanto, dobrar o tamanho da equipe de desenvolvimento não necessariamente duplica produtividade. Se muitos componentes podem ser reutilizados, é possível desenvolver com equipes menores, levando a melhor comunicação e aumento da produtividade.

\end{itemize}

Apesar dos benefícios da reutilização de software, ela não é suficientemente aproveitada. Existem fatores que influenciam direta ou indiretamente na sua adoção. Esses fatores podem ser de aspecto gerencial, organizacional, econômico, conceitual ou técnico. Veremos a seguir alguns aspectos que podem gerar conflito com a cultura de reuso de software, segundo o C.R.U.I.S.E Book \citep{cruise2007} :
%(Sametinger, 1997). REVER

\begin{itemize}
	
\item Falta de apoio da gestão: Como a reutilização de software gera custos iniciais,
a medida pode não ser amplamente alcançada em uma organização sem o apoio de alto nível de gestão. Os gestores têm de ser informados sobre os custos iniciais e serem convencidos sobre economias futuras.

\item Gerenciamento do Projeto: Gerenciar projetos tradicionais é uma tarefa árdua, principalmente, os que praticam a reutilização de software. Utilizando a técnica em larga escala, tem-se impacto sobre todo o ciclo de vida do software.

\item Estruturas organizacionais inadequadas: As estruturas organizacionais devem
considerar diferentes necessidades que surgem quando a reutilização em larga escala está sendo adotada. Por exemplo, uma equipe particionada pode ser alocada somente para desenvolver, manter e certificar componentes reutilizáveis de software.

\item Incentivos de gestão: É comum a falta de incentivo para deixar os desenvolvedores gastarem tempo elaborando componentes do sistemas. A produtividade é muitas vezes medida apenas no tempo necessário para concluir um projeto. Assim, fazer qualquer trabalho além disso, embora benéfico para a empresa como um todo, diminui o seu sucesso. Mesmo quando os componentes reutilizáveis são utilizados, os benefícios obtidos são uma pequena fração do que poderia ser alcançado caso houvesse reutilização explícita, planejada e organizada.

\item Dificuldade de encontrar software reutilizável: Para reutilizar os componentes, devem existir formas eficientes de busca. Além disso, é importante ter um repositório bem organizado contendo componentes com um eficiente meio de acesso.

\item Não reutilização do software encontrado. O acesso fácil ao software existente
não necessariamente aumentar a reutilização. Os componentes reutilizáveis devem ser cuidadosamente especificados, projetados, implementados e documentados, pois em alguns casos, modificar e adaptar o código  pode ser mais custoso que a programação da funcionalidade necessária a partir do zero.

\item Modificação: É muito difícil encontrar um componente que funcione
exatamente da mesma maneira que queremos. Desta forma, são necessárias modificações e devem existir formas de determinar os seus efeitos sobre o componente.


\end{itemize}


A definição do reuso de software pode ser associada a este trabalho, pois ainda que como um serviço, estaremos adotando uma aplicação única para servir aos mais diversos softwares de uma empresa, por exemplo. Logo, ao adotar o projeto, pode ser interessante aplicar algumas métricas para tomar conhecimento de possíveis vantanjens como as citadas nessa seção, a exemplo de economia de tempo de desenvolvimento, e custo do projeto.


%Outra diretriz importante para a reutilização de software é reduzir o risco na criação de novos softwares. O risco tende a ser bastante reduzido se os componentes que estão sendo reutilizados têm as documentação, interfaces necessárias e devidamente testadas, fatores que contibruem para uma fácil integração.
%De acordo com Keswani et al. \citep{6783445}, para o reuso de software dar retornos apropriados, o processo deve ser sistemático e planejado. Qualquer organização que implemente a reutilização de software deve identificar os melhores métodos e estratégias de reutilização para obter a máxima produtividade. A reutilização de software ajuda a evitar software de engenharia a partir do zero, pois usa módulos de software existentes. A reutilização de software, embora seja uma tarefa difícil, especialmente para softwares antigos sem padrões de projeto, pode melhorar significativamente a produtividade e a qualidade de um produto de software. Embora a reutilização de software não seja um novo campo, ela pode dar grandes retornos em curto período de tempo.


\section{Modularização}\label{sec:modularizacao} %artigo de claudio pagina 222 introdução


%A modularidade vem desempenhando um papel predominante estágios emergentes das disciplinas de arquitetura de software [13]. Engenheiros de software consideram modularidade como princípio base na comparação entre arquiteturas alternativas  e arquitetura degeneração [9]. De fato, os engenheiros de software são incentivados a arquitecturas, baseando-se numa multiplicidade de mecanismos de modularidade disponíveis em: 
%(i) Linguagens de descrição de arquitetura (ADLs), como ACME [8], 
%(ii) catálogos de arquitetônicos [2, 13], e 
%(iii) conhecem bem princípios de alto nível, como interfaces de componentes estreitos, acoplamento arquitectónico reduzido e semelhantes.


Conforme é frisado por Wickramaarachchi e Lai \citep{7062705}, o conceito de modularização na indústria de software tem uma longa história e tem sido utilizado para melhorar o processo de desenvolvimento de software em diferentes estágios. Os principais conceitos por trás da modularização do software foram introduzidos por pesquisadores pioneiros há quarenta anos, com uma notável contribuição feita por Melvin Conway e David Parnas, que tem representação notável na engenharia de software.


Modularizar um software é um bom padrão a ser adotado. Segundo Wickramaarachchi e Lai \citep{7062705}, a modularização é importante na identificação de dependências e reduz as dificuldades diante de uma possível necessidade de grandes alterações. De uma perspectiva da engenharia de software, uma modularização geralmente tem várias vantagens, tais como: tornar a complexidade do software mais gerenciável, facilitar o trabalho paralelo e tornar o software mais maleável para acomodar o futuro incerto que um software pode ter. O objetivo final da modularização do software é aumentar a produtividade ea qualidade do software. Tal conceito encontra-se bastante difundido e estái incorporado em linguagens de programação e ferramentas de software. O presente trabalho favorece ao uso da modularização de um software e até mesmo pode ser considerado um módulo a ser acoplado a qualquer software, mediante a compatibilidade.


Ao realizar a adoção do trabalho proposto, fica bastante evidente a "responsabilidade" do mesmo no escopo do projeto. Assim, é possível tratá-lo como um módulo do projeto que o usa, mesmo que seja consumido como um serviço.


\section{Aplicações web}\label{sec:apps_web}


A popularidade das soluções Web aumentou exponencialmente na última década e todos os dias cresce o número de pessoas usuárias desse tipo de software. E seguindo um padrão próprio, Kumar et al. \citep{7813710} sugerem que para o desenvolvimento web, deve-se manter a prática eficaz de produzir diagramas UML. A abordagem baseada na web oferece uma maneira fácil e eficaz para gerenciar e controlar o processo de desenvolvimento por meio de artefatos de modelagem. Tal abordagem pode ser usada quando há uma exigência de lidar com mudanças muito rápidas e grandes em requisitos de forma muito eficaz em muito menos tempo, gerando assim um menor impacto. 


Para atender à fomentada demanda de aplicativos web, é necessário adotar métodos de desenvolvimentos que sejam ágeis, eficientes e de fácil manutenção. Yu Ping et al. \cite{1372143} propõem o uso do modelo MVC (Model, View e Controller) no desenvolvimento para softwares web. O modelo apresentado tornou-se um padrão popular e divide o software em camadas com propósito definido, tornando-o de mais fácil manutenção.


O Ajax (Asynchronous Javascript and XML) revolucionou a web. Conforme demonstrado por Yuping \citep{6845605}, ao usar a tecnologia Ajax, podemos enriquecer a experiência do usuário em aplicações baseadas em navegador de internet, e fornecer uma variedade de aplicações interativas para atender às necessidade de humanização das aplicações.
Os aplicativos Ajax em execução no navegador se comunicam com um servidor Web de forma assíncrona e atualizam apenas uma parte da página.


No artigo de Tesarik et al. \citep{4664349}, temos que o design de software SPA (Single page application) é uma maneira interessante de criar um software Web numa uma única página. Essa solução de página única sem navegação funciona apenas com base em técnicas dinâmicas e assíncronas, como o AJAX (citado no parágrafo anterior). No entanto, esta abordagem coloca o desenvolvedor antes de alguns desafios substanciais.
Para projetar a interface do usuário que mostra as informações, é importante projetar corretamente a tela para manipular os dados do aplicativo numa única página. O design da página deve ser elaborado para maximizar a decomposição da página em componentes distintos que encapsulam os principais casos de uso.
Também recomenda-se que para elaborar os artefatos visuais, explore os mais ricos recursos na implementação, como HTML5, JavaScript, Ajax, CSS3, e outras tecnologias que se apliquem. Entretanto, desenvolver uma rica interface com o uso de diversas tecnologias/frameworks pode ocasionar um esforço maior para explorar as possibilidades.
Tais componentes da interface de uma aplicação SPA normalmente são alimentados mediante o consumo de uma API Rest via requisições AJAX. Assim, é possível particionar as responsabilidades de processamento do software entre cliente e servidor.


Como é possível denotar no trabalho de Salvadori e Siqueire \citep{7195633}, REST é uma arquitetura muito popular para integração de aplicativos web, que permite compartilhar e reutilizar informações através de sistemas. Aplicações de grande escala baseadas em no REST deve ser implementadas utilizando estratégias e mecanismos para produzir sistemas que sejam fáceis de desenvolver, reutilizar e manter. As interfaces de integração fornecidas por esses sistemas, chamadas de Web API, têm uma influência importante nas características da implementação resultante, pois a sua forma de resposta deve ser casada com a interface que o consome.
 
 
Os elementos comentados nessa seção encontram-se presentes neste trabalho de conclusão de curso. Juntos, montam a estrutura tecnológica necessária aliada à arquitetura adotada. Esses elementos seguem tendências atuais dos softwares web, fazendo com que esse trabalho esteja composto por tecnologias modernas que estão com boa aceitação no mercado.

% %% RiSE Latex Template - version 0.5
%%
%% RiSE's latex template for thesis and dissertations
%% http://risetemplate.sourceforge.net
%%
%% (c) 2012 Yguaratã Cerqueira Cavalcanti (yguarata@gmail.com)
%%          Vinicius Cardoso Garcia (vinicius.garcia@gmail.com)
%%
%% This document was initially based on UFPEThesis template, from Paulo Gustavo
%% S. Fonseca.
%%
%% ACKNOWLEDGEMENTS
%%
%% We would like to thanks the RiSE's researchers community, the 
%% students from Federal University of Pernambuco, and other users that have
%% been contributing to this projects with comments and patches.
%%
%% GENERAL INSTRUCTIONS
%%
%% We strongly recommend you to compile your documents using pdflatex command.
%% It is also recommend use the texlipse plugin for Eclipse to edit your documents.
%%
%% Options for \documentclass command:
%%         * Idiom
%%           pt   - Portguese (default)
%%           en   - English
%%
%%         * Text type
%%           bsc  - B.Sc. Thesis
%%           msc  - M.Sc. Thesis (default)
%%           qual - PHD qualification (not tested yet)
%%           prop - PHD proposal (not tested yet)
%%           phd  - PHD thesis
%%
%%         * Media
%%           scr  - to eletronic version (PDF) / see the users guide
%%
%%         * Pagination
%%           oneside - unique face press
%%           twoside - two faces press
%%
%%		   * Line spacing
%%           singlespacing  - the same as using \linespread{1}
%%           onehalfspacing - the same as using \linespread{1.3}
%%           doublespacing  - the same as using \linespread{1.6}
%%
%% Reference commands. Use the following commands to make references in your
%% text:
%%          \figref  -- for Figure reference
%%          \tabref  -- for Table reference
%%          \eqnref  -- for equation reference
%%          \chapref -- for chapter reference
%%          \secref  -- for section reference
%%          \appref  -- for appendix reference
%%          \axiref  -- for axiom reference
%%          \conjref -- for conjecture reference
%%          \defref  -- for definition reference
%%          \lemref  -- for lemma reference
%%          \theoref -- for theorem reference
%%          \corref  -- for corollary reference
%%          \propref -- for proprosition reference
%%          \pgref   -- for page reference
%%
%%          Example: See \chapref{chap:introduction}. It will produce 
%%                   'See Chapter 1', in case of English language.

\documentclass[pt,twoside,onehalfspacing,bsc]{risethesis}

\usepackage[sort,square]{natbib} %package for bibliography
\bibliographystyle{ieeetr} %authordate1

\usepackage[utf8]{inputenc}
\usepackage[brazilian]{babel}
\usepackage[T1]{fontenc}

\usepackage{graphicx}

%% Change the following pdf author attribute name to your name.
\usepackage[linkcolor=blue,citecolor=blue,urlcolor=blue,colorlinks,pdfpagelabels,pdftitle={Bruno Cabral's Bachelor Thesis},pdfauthor={Bruno Cabral}]{hyperref}

\address{SALVADOR}

\universitypt{Universidade Federal da Bahia}
\universityen{Federal University of Bahia}

\departmentpt{Depertamento de Ciência da Computação}
\departmenten{Computer Science Department}

\programpt{Programa Multiinstitucional de Pós-graduação em Ciência da Computação}
\programen{Graduate in Computer Science}

\majorfieldpt{Ciência da Computação}
\majorfielden{Computer Science}

\title{Sistema de apoio à Pós graduação - UFBA}
\date{Outubro/2016}

\author{Victor de Azevedo Nunes}
\adviser{Ivan do Carmo Machado}

\begin{document}

\frontmatter
\frontpage
\presentationpage

\begin{dedicatory}
Eu dedico esta dissertação...
%I dedicate this dissertation to my family, girlfriend, friends and
%professors who gave me all necessary support to get here.
\end{dedicatory}

\acknowledgements
Meus agradecimentos...

\begin{epigraph}[]{Edward V Berard}
Walking on water and developing software from a specification are easy if both are frozen
\end{epigraph}

\resumo
% Escreva seu resumo no arquivo resumo.tex
\input{resumo}

\abstract
% Write your abstract in a file called abstract.tex
\input{abstract}

% Summary (tables of contents)
\tableofcontents

% List of figures
\listoffigures

% List of tables
\listoftables

% List of acronyms
% Acronyms manual: http://linorg.usp.br/CTAN/macros/latex/contrib/acronym/acronym.pdf
\listofacronyms
\input{acronyms}

% List of listings
%\lstlistoflistings

\mainmatter

\include{chapters/intro}
\include{chapters/referencial_teorico}

% \include{chapters/introduction/main}
% \include{chapters/background/main}
% \include{chapters/proposed_solution/main}
% \include{chapters/experiment/main}
% \include{chapters/conclusion/main}

%\bibliographystyle{natbib}
%\addcontentsline{toc}{chapter}{\bibliographytocname}
\bibliography{references}

% Appendix
\clearpage
\addappheadtotoc
\appendix
\appendixpage
% \include{appendix/experiment-instruments}

\end{document}
% %% RiSE Latex Template - version 0.5
%%
%% RiSE's latex template for thesis and dissertations
%% http://risetemplate.sourceforge.net
%%
%% (c) 2012 Yguaratã Cerqueira Cavalcanti (yguarata@gmail.com)
%%          Vinicius Cardoso Garcia (vinicius.garcia@gmail.com)
%%
%% This document was initially based on UFPEThesis template, from Paulo Gustavo
%% S. Fonseca.
%%
%% ACKNOWLEDGEMENTS
%%
%% We would like to thanks the RiSE's researchers community, the 
%% students from Federal University of Pernambuco, and other users that have
%% been contributing to this projects with comments and patches.
%%
%% GENERAL INSTRUCTIONS
%%
%% We strongly recommend you to compile your documents using pdflatex command.
%% It is also recommend use the texlipse plugin for Eclipse to edit your documents.
%%
%% Options for \documentclass command:
%%         * Idiom
%%           pt   - Portguese (default)
%%           en   - English
%%
%%         * Text type
%%           bsc  - B.Sc. Thesis
%%           msc  - M.Sc. Thesis (default)
%%           qual - PHD qualification (not tested yet)
%%           prop - PHD proposal (not tested yet)
%%           phd  - PHD thesis
%%
%%         * Media
%%           scr  - to eletronic version (PDF) / see the users guide
%%
%%         * Pagination
%%           oneside - unique face press
%%           twoside - two faces press
%%
%%		   * Line spacing
%%           singlespacing  - the same as using \linespread{1}
%%           onehalfspacing - the same as using \linespread{1.3}
%%           doublespacing  - the same as using \linespread{1.6}
%%
%% Reference commands. Use the following commands to make references in your
%% text:
%%          \figref  -- for Figure reference
%%          \tabref  -- for Table reference
%%          \eqnref  -- for equation reference
%%          \chapref -- for chapter reference
%%          \secref  -- for section reference
%%          \appref  -- for appendix reference
%%          \axiref  -- for axiom reference
%%          \conjref -- for conjecture reference
%%          \defref  -- for definition reference
%%          \lemref  -- for lemma reference
%%          \theoref -- for theorem reference
%%          \corref  -- for corollary reference
%%          \propref -- for proprosition reference
%%          \pgref   -- for page reference
%%
%%          Example: See \chapref{chap:introduction}. It will produce 
%%                   'See Chapter 1', in case of English language.

\documentclass[pt,twoside,onehalfspacing,bsc]{risethesis}

\usepackage[sort,square]{natbib} %package for bibliography
\bibliographystyle{ieeetr} %authordate1

\usepackage[utf8]{inputenc}
\usepackage[brazilian]{babel}
\usepackage[T1]{fontenc}

\usepackage{graphicx}

%% Change the following pdf author attribute name to your name.
\usepackage[linkcolor=blue,citecolor=blue,urlcolor=blue,colorlinks,pdfpagelabels,pdftitle={Bruno Cabral's Bachelor Thesis},pdfauthor={Bruno Cabral}]{hyperref}

\address{SALVADOR}

\universitypt{Universidade Federal da Bahia}
\universityen{Federal University of Bahia}

\departmentpt{Depertamento de Ciência da Computação}
\departmenten{Computer Science Department}

\programpt{Programa Multiinstitucional de Pós-graduação em Ciência da Computação}
\programen{Graduate in Computer Science}

\majorfieldpt{Ciência da Computação}
\majorfielden{Computer Science}

\title{Sistema de apoio à Pós graduação - UFBA}
\date{Outubro/2016}

\author{Victor de Azevedo Nunes}
\adviser{Ivan do Carmo Machado}

\begin{document}

\frontmatter
\frontpage
\presentationpage

\begin{dedicatory}
Eu dedico esta dissertação...
%I dedicate this dissertation to my family, girlfriend, friends and
%professors who gave me all necessary support to get here.
\end{dedicatory}

\acknowledgements
Meus agradecimentos...

\begin{epigraph}[]{Edward V Berard}
Walking on water and developing software from a specification are easy if both are frozen
\end{epigraph}

\resumo
% Escreva seu resumo no arquivo resumo.tex
\input{resumo}

\abstract
% Write your abstract in a file called abstract.tex
\input{abstract}

% Summary (tables of contents)
\tableofcontents

% List of figures
\listoffigures

% List of tables
\listoftables

% List of acronyms
% Acronyms manual: http://linorg.usp.br/CTAN/macros/latex/contrib/acronym/acronym.pdf
\listofacronyms
\input{acronyms}

% List of listings
%\lstlistoflistings

\mainmatter

\include{chapters/intro}
\include{chapters/referencial_teorico}

% \include{chapters/introduction/main}
% \include{chapters/background/main}
% \include{chapters/proposed_solution/main}
% \include{chapters/experiment/main}
% \include{chapters/conclusion/main}

%\bibliographystyle{natbib}
%\addcontentsline{toc}{chapter}{\bibliographytocname}
\bibliography{references}

% Appendix
\clearpage
\addappheadtotoc
\appendix
\appendixpage
% \include{appendix/experiment-instruments}

\end{document}
% %% RiSE Latex Template - version 0.5
%%
%% RiSE's latex template for thesis and dissertations
%% http://risetemplate.sourceforge.net
%%
%% (c) 2012 Yguaratã Cerqueira Cavalcanti (yguarata@gmail.com)
%%          Vinicius Cardoso Garcia (vinicius.garcia@gmail.com)
%%
%% This document was initially based on UFPEThesis template, from Paulo Gustavo
%% S. Fonseca.
%%
%% ACKNOWLEDGEMENTS
%%
%% We would like to thanks the RiSE's researchers community, the 
%% students from Federal University of Pernambuco, and other users that have
%% been contributing to this projects with comments and patches.
%%
%% GENERAL INSTRUCTIONS
%%
%% We strongly recommend you to compile your documents using pdflatex command.
%% It is also recommend use the texlipse plugin for Eclipse to edit your documents.
%%
%% Options for \documentclass command:
%%         * Idiom
%%           pt   - Portguese (default)
%%           en   - English
%%
%%         * Text type
%%           bsc  - B.Sc. Thesis
%%           msc  - M.Sc. Thesis (default)
%%           qual - PHD qualification (not tested yet)
%%           prop - PHD proposal (not tested yet)
%%           phd  - PHD thesis
%%
%%         * Media
%%           scr  - to eletronic version (PDF) / see the users guide
%%
%%         * Pagination
%%           oneside - unique face press
%%           twoside - two faces press
%%
%%		   * Line spacing
%%           singlespacing  - the same as using \linespread{1}
%%           onehalfspacing - the same as using \linespread{1.3}
%%           doublespacing  - the same as using \linespread{1.6}
%%
%% Reference commands. Use the following commands to make references in your
%% text:
%%          \figref  -- for Figure reference
%%          \tabref  -- for Table reference
%%          \eqnref  -- for equation reference
%%          \chapref -- for chapter reference
%%          \secref  -- for section reference
%%          \appref  -- for appendix reference
%%          \axiref  -- for axiom reference
%%          \conjref -- for conjecture reference
%%          \defref  -- for definition reference
%%          \lemref  -- for lemma reference
%%          \theoref -- for theorem reference
%%          \corref  -- for corollary reference
%%          \propref -- for proprosition reference
%%          \pgref   -- for page reference
%%
%%          Example: See \chapref{chap:introduction}. It will produce 
%%                   'See Chapter 1', in case of English language.

\documentclass[pt,twoside,onehalfspacing,bsc]{risethesis}

\usepackage[sort,square]{natbib} %package for bibliography
\bibliographystyle{ieeetr} %authordate1

\usepackage[utf8]{inputenc}
\usepackage[brazilian]{babel}
\usepackage[T1]{fontenc}

\usepackage{graphicx}

%% Change the following pdf author attribute name to your name.
\usepackage[linkcolor=blue,citecolor=blue,urlcolor=blue,colorlinks,pdfpagelabels,pdftitle={Bruno Cabral's Bachelor Thesis},pdfauthor={Bruno Cabral}]{hyperref}

\address{SALVADOR}

\universitypt{Universidade Federal da Bahia}
\universityen{Federal University of Bahia}

\departmentpt{Depertamento de Ciência da Computação}
\departmenten{Computer Science Department}

\programpt{Programa Multiinstitucional de Pós-graduação em Ciência da Computação}
\programen{Graduate in Computer Science}

\majorfieldpt{Ciência da Computação}
\majorfielden{Computer Science}

\title{Sistema de apoio à Pós graduação - UFBA}
\date{Outubro/2016}

\author{Victor de Azevedo Nunes}
\adviser{Ivan do Carmo Machado}

\begin{document}

\frontmatter
\frontpage
\presentationpage

\begin{dedicatory}
Eu dedico esta dissertação...
%I dedicate this dissertation to my family, girlfriend, friends and
%professors who gave me all necessary support to get here.
\end{dedicatory}

\acknowledgements
Meus agradecimentos...

\begin{epigraph}[]{Edward V Berard}
Walking on water and developing software from a specification are easy if both are frozen
\end{epigraph}

\resumo
% Escreva seu resumo no arquivo resumo.tex
\input{resumo}

\abstract
% Write your abstract in a file called abstract.tex
\input{abstract}

% Summary (tables of contents)
\tableofcontents

% List of figures
\listoffigures

% List of tables
\listoftables

% List of acronyms
% Acronyms manual: http://linorg.usp.br/CTAN/macros/latex/contrib/acronym/acronym.pdf
\listofacronyms
\input{acronyms}

% List of listings
%\lstlistoflistings

\mainmatter

\include{chapters/intro}
\include{chapters/referencial_teorico}

% \include{chapters/introduction/main}
% \include{chapters/background/main}
% \include{chapters/proposed_solution/main}
% \include{chapters/experiment/main}
% \include{chapters/conclusion/main}

%\bibliographystyle{natbib}
%\addcontentsline{toc}{chapter}{\bibliographytocname}
\bibliography{references}

% Appendix
\clearpage
\addappheadtotoc
\appendix
\appendixpage
% \include{appendix/experiment-instruments}

\end{document}
% %% RiSE Latex Template - version 0.5
%%
%% RiSE's latex template for thesis and dissertations
%% http://risetemplate.sourceforge.net
%%
%% (c) 2012 Yguaratã Cerqueira Cavalcanti (yguarata@gmail.com)
%%          Vinicius Cardoso Garcia (vinicius.garcia@gmail.com)
%%
%% This document was initially based on UFPEThesis template, from Paulo Gustavo
%% S. Fonseca.
%%
%% ACKNOWLEDGEMENTS
%%
%% We would like to thanks the RiSE's researchers community, the 
%% students from Federal University of Pernambuco, and other users that have
%% been contributing to this projects with comments and patches.
%%
%% GENERAL INSTRUCTIONS
%%
%% We strongly recommend you to compile your documents using pdflatex command.
%% It is also recommend use the texlipse plugin for Eclipse to edit your documents.
%%
%% Options for \documentclass command:
%%         * Idiom
%%           pt   - Portguese (default)
%%           en   - English
%%
%%         * Text type
%%           bsc  - B.Sc. Thesis
%%           msc  - M.Sc. Thesis (default)
%%           qual - PHD qualification (not tested yet)
%%           prop - PHD proposal (not tested yet)
%%           phd  - PHD thesis
%%
%%         * Media
%%           scr  - to eletronic version (PDF) / see the users guide
%%
%%         * Pagination
%%           oneside - unique face press
%%           twoside - two faces press
%%
%%		   * Line spacing
%%           singlespacing  - the same as using \linespread{1}
%%           onehalfspacing - the same as using \linespread{1.3}
%%           doublespacing  - the same as using \linespread{1.6}
%%
%% Reference commands. Use the following commands to make references in your
%% text:
%%          \figref  -- for Figure reference
%%          \tabref  -- for Table reference
%%          \eqnref  -- for equation reference
%%          \chapref -- for chapter reference
%%          \secref  -- for section reference
%%          \appref  -- for appendix reference
%%          \axiref  -- for axiom reference
%%          \conjref -- for conjecture reference
%%          \defref  -- for definition reference
%%          \lemref  -- for lemma reference
%%          \theoref -- for theorem reference
%%          \corref  -- for corollary reference
%%          \propref -- for proprosition reference
%%          \pgref   -- for page reference
%%
%%          Example: See \chapref{chap:introduction}. It will produce 
%%                   'See Chapter 1', in case of English language.

\documentclass[pt,twoside,onehalfspacing,bsc]{risethesis}

\usepackage[sort,square]{natbib} %package for bibliography
\bibliographystyle{ieeetr} %authordate1

\usepackage[utf8]{inputenc}
\usepackage[brazilian]{babel}
\usepackage[T1]{fontenc}

\usepackage{graphicx}

%% Change the following pdf author attribute name to your name.
\usepackage[linkcolor=blue,citecolor=blue,urlcolor=blue,colorlinks,pdfpagelabels,pdftitle={Bruno Cabral's Bachelor Thesis},pdfauthor={Bruno Cabral}]{hyperref}

\address{SALVADOR}

\universitypt{Universidade Federal da Bahia}
\universityen{Federal University of Bahia}

\departmentpt{Depertamento de Ciência da Computação}
\departmenten{Computer Science Department}

\programpt{Programa Multiinstitucional de Pós-graduação em Ciência da Computação}
\programen{Graduate in Computer Science}

\majorfieldpt{Ciência da Computação}
\majorfielden{Computer Science}

\title{Sistema de apoio à Pós graduação - UFBA}
\date{Outubro/2016}

\author{Victor de Azevedo Nunes}
\adviser{Ivan do Carmo Machado}

\begin{document}

\frontmatter
\frontpage
\presentationpage

\begin{dedicatory}
Eu dedico esta dissertação...
%I dedicate this dissertation to my family, girlfriend, friends and
%professors who gave me all necessary support to get here.
\end{dedicatory}

\acknowledgements
Meus agradecimentos...

\begin{epigraph}[]{Edward V Berard}
Walking on water and developing software from a specification are easy if both are frozen
\end{epigraph}

\resumo
% Escreva seu resumo no arquivo resumo.tex
\input{resumo}

\abstract
% Write your abstract in a file called abstract.tex
\input{abstract}

% Summary (tables of contents)
\tableofcontents

% List of figures
\listoffigures

% List of tables
\listoftables

% List of acronyms
% Acronyms manual: http://linorg.usp.br/CTAN/macros/latex/contrib/acronym/acronym.pdf
\listofacronyms
\input{acronyms}

% List of listings
%\lstlistoflistings

\mainmatter

\include{chapters/intro}
\include{chapters/referencial_teorico}

% \include{chapters/introduction/main}
% \include{chapters/background/main}
% \include{chapters/proposed_solution/main}
% \include{chapters/experiment/main}
% \include{chapters/conclusion/main}

%\bibliographystyle{natbib}
%\addcontentsline{toc}{chapter}{\bibliographytocname}
\bibliography{references}

% Appendix
\clearpage
\addappheadtotoc
\appendix
\appendixpage
% \include{appendix/experiment-instruments}

\end{document}
% %% RiSE Latex Template - version 0.5
%%
%% RiSE's latex template for thesis and dissertations
%% http://risetemplate.sourceforge.net
%%
%% (c) 2012 Yguaratã Cerqueira Cavalcanti (yguarata@gmail.com)
%%          Vinicius Cardoso Garcia (vinicius.garcia@gmail.com)
%%
%% This document was initially based on UFPEThesis template, from Paulo Gustavo
%% S. Fonseca.
%%
%% ACKNOWLEDGEMENTS
%%
%% We would like to thanks the RiSE's researchers community, the 
%% students from Federal University of Pernambuco, and other users that have
%% been contributing to this projects with comments and patches.
%%
%% GENERAL INSTRUCTIONS
%%
%% We strongly recommend you to compile your documents using pdflatex command.
%% It is also recommend use the texlipse plugin for Eclipse to edit your documents.
%%
%% Options for \documentclass command:
%%         * Idiom
%%           pt   - Portguese (default)
%%           en   - English
%%
%%         * Text type
%%           bsc  - B.Sc. Thesis
%%           msc  - M.Sc. Thesis (default)
%%           qual - PHD qualification (not tested yet)
%%           prop - PHD proposal (not tested yet)
%%           phd  - PHD thesis
%%
%%         * Media
%%           scr  - to eletronic version (PDF) / see the users guide
%%
%%         * Pagination
%%           oneside - unique face press
%%           twoside - two faces press
%%
%%		   * Line spacing
%%           singlespacing  - the same as using \linespread{1}
%%           onehalfspacing - the same as using \linespread{1.3}
%%           doublespacing  - the same as using \linespread{1.6}
%%
%% Reference commands. Use the following commands to make references in your
%% text:
%%          \figref  -- for Figure reference
%%          \tabref  -- for Table reference
%%          \eqnref  -- for equation reference
%%          \chapref -- for chapter reference
%%          \secref  -- for section reference
%%          \appref  -- for appendix reference
%%          \axiref  -- for axiom reference
%%          \conjref -- for conjecture reference
%%          \defref  -- for definition reference
%%          \lemref  -- for lemma reference
%%          \theoref -- for theorem reference
%%          \corref  -- for corollary reference
%%          \propref -- for proprosition reference
%%          \pgref   -- for page reference
%%
%%          Example: See \chapref{chap:introduction}. It will produce 
%%                   'See Chapter 1', in case of English language.

\documentclass[pt,twoside,onehalfspacing,bsc]{risethesis}

\usepackage[sort,square]{natbib} %package for bibliography
\bibliographystyle{ieeetr} %authordate1

\usepackage[utf8]{inputenc}
\usepackage[brazilian]{babel}
\usepackage[T1]{fontenc}

\usepackage{graphicx}

%% Change the following pdf author attribute name to your name.
\usepackage[linkcolor=blue,citecolor=blue,urlcolor=blue,colorlinks,pdfpagelabels,pdftitle={Bruno Cabral's Bachelor Thesis},pdfauthor={Bruno Cabral}]{hyperref}

\address{SALVADOR}

\universitypt{Universidade Federal da Bahia}
\universityen{Federal University of Bahia}

\departmentpt{Depertamento de Ciência da Computação}
\departmenten{Computer Science Department}

\programpt{Programa Multiinstitucional de Pós-graduação em Ciência da Computação}
\programen{Graduate in Computer Science}

\majorfieldpt{Ciência da Computação}
\majorfielden{Computer Science}

\title{Sistema de apoio à Pós graduação - UFBA}
\date{Outubro/2016}

\author{Victor de Azevedo Nunes}
\adviser{Ivan do Carmo Machado}

\begin{document}

\frontmatter
\frontpage
\presentationpage

\begin{dedicatory}
Eu dedico esta dissertação...
%I dedicate this dissertation to my family, girlfriend, friends and
%professors who gave me all necessary support to get here.
\end{dedicatory}

\acknowledgements
Meus agradecimentos...

\begin{epigraph}[]{Edward V Berard}
Walking on water and developing software from a specification are easy if both are frozen
\end{epigraph}

\resumo
% Escreva seu resumo no arquivo resumo.tex
\input{resumo}

\abstract
% Write your abstract in a file called abstract.tex
\input{abstract}

% Summary (tables of contents)
\tableofcontents

% List of figures
\listoffigures

% List of tables
\listoftables

% List of acronyms
% Acronyms manual: http://linorg.usp.br/CTAN/macros/latex/contrib/acronym/acronym.pdf
\listofacronyms
\input{acronyms}

% List of listings
%\lstlistoflistings

\mainmatter

\include{chapters/intro}
\include{chapters/referencial_teorico}

% \include{chapters/introduction/main}
% \include{chapters/background/main}
% \include{chapters/proposed_solution/main}
% \include{chapters/experiment/main}
% \include{chapters/conclusion/main}

%\bibliographystyle{natbib}
%\addcontentsline{toc}{chapter}{\bibliographytocname}
\bibliography{references}

% Appendix
\clearpage
\addappheadtotoc
\appendix
\appendixpage
% \include{appendix/experiment-instruments}

\end{document}

%\bibliographystyle{natbib}
%\addcontentsline{toc}{chapter}{\bibliographytocname}
\bibliography{references}

% Appendix
\clearpage
\addappheadtotoc
\appendix
\appendixpage
% \include{appendix/experiment-instruments}

\end{document}
% %% RiSE Latex Template - version 0.5
%%
%% RiSE's latex template for thesis and dissertations
%% http://risetemplate.sourceforge.net
%%
%% (c) 2012 Yguaratã Cerqueira Cavalcanti (yguarata@gmail.com)
%%          Vinicius Cardoso Garcia (vinicius.garcia@gmail.com)
%%
%% This document was initially based on UFPEThesis template, from Paulo Gustavo
%% S. Fonseca.
%%
%% ACKNOWLEDGEMENTS
%%
%% We would like to thanks the RiSE's researchers community, the 
%% students from Federal University of Pernambuco, and other users that have
%% been contributing to this projects with comments and patches.
%%
%% GENERAL INSTRUCTIONS
%%
%% We strongly recommend you to compile your documents using pdflatex command.
%% It is also recommend use the texlipse plugin for Eclipse to edit your documents.
%%
%% Options for \documentclass command:
%%         * Idiom
%%           pt   - Portguese (default)
%%           en   - English
%%
%%         * Text type
%%           bsc  - B.Sc. Thesis
%%           msc  - M.Sc. Thesis (default)
%%           qual - PHD qualification (not tested yet)
%%           prop - PHD proposal (not tested yet)
%%           phd  - PHD thesis
%%
%%         * Media
%%           scr  - to eletronic version (PDF) / see the users guide
%%
%%         * Pagination
%%           oneside - unique face press
%%           twoside - two faces press
%%
%%		   * Line spacing
%%           singlespacing  - the same as using \linespread{1}
%%           onehalfspacing - the same as using \linespread{1.3}
%%           doublespacing  - the same as using \linespread{1.6}
%%
%% Reference commands. Use the following commands to make references in your
%% text:
%%          \figref  -- for Figure reference
%%          \tabref  -- for Table reference
%%          \eqnref  -- for equation reference
%%          \chapref -- for chapter reference
%%          \secref  -- for section reference
%%          \appref  -- for appendix reference
%%          \axiref  -- for axiom reference
%%          \conjref -- for conjecture reference
%%          \defref  -- for definition reference
%%          \lemref  -- for lemma reference
%%          \theoref -- for theorem reference
%%          \corref  -- for corollary reference
%%          \propref -- for proprosition reference
%%          \pgref   -- for page reference
%%
%%          Example: See \chapref{chap:introduction}. It will produce 
%%                   'See Chapter 1', in case of English language.

\documentclass[pt,twoside,onehalfspacing,bsc]{risethesis}

\usepackage[sort,square]{natbib} %package for bibliography
\bibliographystyle{ieeetr} %authordate1

\usepackage[utf8]{inputenc}
\usepackage[brazilian]{babel}
\usepackage[T1]{fontenc}

\usepackage{graphicx}

%% Change the following pdf author attribute name to your name.
\usepackage[linkcolor=blue,citecolor=blue,urlcolor=blue,colorlinks,pdfpagelabels,pdftitle={Bruno Cabral's Bachelor Thesis},pdfauthor={Bruno Cabral}]{hyperref}

\address{SALVADOR}

\universitypt{Universidade Federal da Bahia}
\universityen{Federal University of Bahia}

\departmentpt{Depertamento de Ciência da Computação}
\departmenten{Computer Science Department}

\programpt{Programa Multiinstitucional de Pós-graduação em Ciência da Computação}
\programen{Graduate in Computer Science}

\majorfieldpt{Ciência da Computação}
\majorfielden{Computer Science}

\title{Sistema de apoio à Pós graduação - UFBA}
\date{Outubro/2016}

\author{Victor de Azevedo Nunes}
\adviser{Ivan do Carmo Machado}

\begin{document}

\frontmatter
\frontpage
\presentationpage

\begin{dedicatory}
Eu dedico esta dissertação...
%I dedicate this dissertation to my family, girlfriend, friends and
%professors who gave me all necessary support to get here.
\end{dedicatory}

\acknowledgements
Meus agradecimentos...

\begin{epigraph}[]{Edward V Berard}
Walking on water and developing software from a specification are easy if both are frozen
\end{epigraph}

\resumo
% Escreva seu resumo no arquivo resumo.tex
Este trabalho de conclusão de curso utiliza elementos da engenharia de software para propor um software como serviço a fim de otimizar o processo de construção e manutenção dos softwares. Assim, o objetivo deste SaaS é gerenciar as permissões de acesso de sistemas cliente, provendo o reuso de software e padronizando as soluções. O sistema proposto neste trabalho encontra-se implementado e disponível no Github, e traz fundamentos sobre a arquitetura e tecnologias utilizadas, bem como avaliações sobre possibilidades de utilização do mesmo em diversos ambientes, como Web e mobile.

\begin{keywords}
Software; Reuso; SaaS; Web; PHP
\end{keywords}

\abstract
% Write your abstract in a file called abstract.tex
My abstract...

\begin{keywords}
key words...
\end{keywords}

% Summary (tables of contents)
\tableofcontents

% List of figures
\listoffigures

% List of tables
\listoftables

% List of acronyms
% Acronyms manual: http://linorg.usp.br/CTAN/macros/latex/contrib/acronym/acronym.pdf
\listofacronyms
\begin{acronym}[ACRONYM] 
% Change the word ACRONYM above to change the acronym column width.
% The column width is equals to the width of the word that you put.
% Read the manual about acronym package for more examples:
%   http://linorg.usp.br/CTAN/macros/latex/contrib/acronym/acronym.pdf

\acro{SPA}{Single Page Application}
\acro{JSON}{Javascript Object Notation}
\acro{PHP}{PHP: Hypertext Preprocessor}
\acro{SaaS}{Software as a Service}
\acro{ERP}{Enterprise Resource Planning}
\acro{QoS}{Quality of Service}
\acro{UML}{Unified Modeling Language}
\acro{MVC}{Model-View-Controller}
\acro{Ajax}{Asynchronous Javascript and XML}
\acro{HTML}{HyperText Markup Language}
\acro{CSS}{Cascading Style Sheets}
\acro{API}{Application Programming Interface}
\acro{DOM}{Document Object Model}
\acro{BPMN}{Business Process Model and Notation}
\acro{REST}{Representational State Transfer}

\end{acronym}

% List of listings
%\lstlistoflistings

\mainmatter

\chapter{Introdução}

\section{Motivação}

Organizar os procedimentos de um processo sempre nos traz vantagens. Apesar de no processo de implantação de um sistema, o mesmo burocratizar o processo, com o tempo temos o retorno da dedicação para a inserção dos dados. Com um certo volume de dados, é possível estruturar informações que num processo manual são difíceis de serem enxergadas. Assim, é possível depender menos das pessoas que organizam o processo, pois o legado de informações não estará mais somente na mente de alguns, mas sim documentado nos dados do sistema.

Além de colaborar na organização, também haverá uma grande colaboração no tempo gasto na gestão. Lidar com muitos papéis e confiar na mente humana para guardar informações, não é uma alternativa muito segura devido ao fato que as pessoas sempre estão sujeitas a sair do processo e levar contigo a experiência obtida. Experiência essa que faz com que os procedimentos sejam executados de forma mais eficiente. Entretanto, com um sistema inteligente, é possível auxiliar e tornar mais ágil a execução das tarefas.


\section{Problema}


De acordo com funcionários ligados ao o setor de pós graduação da UFBA, entrevistados a fim de um maior entendimento do cenário, apesar das semelhanças estruturais, a pós graduação gerida de forma diferencia da graduação. FULANO afirma que devido ao fato de não ter a mesma visibilidade, não tem acesso aos mesmos recursos de gestão acadêmica da graduação. O professores não executam somente atividades dentro da sala de aula, também tem diversas outras ocupações no setor. E muitos procedimentos realizados extra classe ainda se encontram sendo realizados de forma manual, estando mais vulnerável ao erro ou até mesmo à violação do processo. Também ocorre um grande desperdício de tempo pelos professores e gestores da área, devido ao diversos processos ainda realizados de forma manual, sem a devida documentação. Segundo FULANO, também entrevistado, esse tempo perdido implica numa redução da eficiência na sala de aula, pois o professor acaba por ter menos tempo disponível para o planejamento das atividades, o que gera impactos negativos aos alunos.


\section{Objetivos} %<o que deve ser feito/entregue>


Devido aos muitos processos sendo resolvidos de forma manual, propõe-se com solução um sistema moderno, arquitetado para ter funcionamento na web e com um módulo mobile, a fim de fornecer informações de forma rápida e eficiente para os professores através de notificações, já que o acesso à internet móvel é comum entre os possíveis usuários do sistema em questão.
O principal requisito para o sistema seria dispor recursos para reduzir o tempo desperdiçado pelos professores durante as atividades extra classe.


\section{Metodologia} %<como será feito | como resolver o problema apontado inicialmente>


%<analise de literatura | design | implementação | validação>
Baseando-se nas tecnologias gratuitas em alta no cenário atual do desenvolvimento web, dispomos de algumas opções eficientes para a implementação da solução. Dentre as possibilidades, considerando a facilidade para futura manutenção e continuidade do projeto, tende-se a optar por uma tecnologia popular. Como linguagem de programação, adota-se o PHP. A escolha é fundamentada de acordo com a pesquisa da RedMonk de 2015, que evidencia o uso das linguagens de programação de acordo com as discussões no StackOverflow e repositórios no GitHub. É possível constatar a popularidade do PHP no cenário atual com o gráfico da pesquisa citada, na qual o PHP é apresentado na terceira colocação, apenas atrás do lider JavaScript e do segundo colocado, o Java.

\begin{figure}
	\label{fig:graficoRedmonk}
	\includegraphics[width=1\textwidth]{img/grafico_redmonk}
	\caption{Ranking das liguagens de programação no Stack Overflow e Github}
\end{figure}


Ainda assim, para compor a interface do dado projeto, também ocorrerá o uso do líder JavaScript de forma intensa, provendo o elo com o as informações gerenciadas pelo PHP.


Entretanto, não seria inteligente desenvolver um sistema completo sem o auxílio de um framework. Dentre os frameworks disponíveis para PHP, hoje o destaque está com o Laravel, que se encontra no topo dentre os mais utilizados no momento. 


A WebHostFace, uma empresa de hospedagem, compilou várias estatísticas para criar um infográfico mostrando os frameworks PHP mais populares de 2015. Utilizando informações sobre os próprios clientes, o Google Trends, estatísticas de repositórios do GitHub e a pesquisa do SitePoint “Best PHP Frameworks 2015”, a WebHostFace elaborou o seguinte infográfico: 

\begin{figure}
	\label{fig:graficoWebhostface}
	\includegraphics[width=1\textwidth]{img/infografico_webhostface}
	\caption{Infográfico da WebhostFace, exibindo a popularidade dos Frameworks PHP em 2015}
\end{figure}

Assim, tem-se a evidência que o Laravel em 2015 teve a maior popularidade em projetos pessoais e tem a maior comunidade entre os concorrentes, o que o torna uma boa escolha para a escrita de um software que será continuado por terceiros.


Para elaborar os recursos de interface e integrar ao back-end PHP do sistema, será adotado o já conhecido AngularJS, ferramenta sólida e conhecida no aspecto em questão. 


Dados coletados via Google Trends, que propõe comparações entre termos pesquisados, revela a popularidade do AngularJs diante de alguns dos principais concorrentes. O gráfico abaixo evidencia o cenário.


%Como mostra a Figura \ref{fig:graficoGoogleTrendsFerramentasFront}. 
\begin{figure}
	\label{fig:graficoGoogleTrendsFerramentasFront}
	\includegraphics[width=1\textwidth]{img/grafico_ferramentas_front}
	\caption{Gráfico do Google Trends exibindo as pesquisas por ferramentas front-end}
\end{figure}


Junto ao Angular JS, será utilizada a agradável tendência de interface do Material Design da Google, que propõe layouts limpos e otimizados já conhecidos pelos usuários de smartphones Android. 


Para a elaboração da plataforma mobile do projeto, será utilizado o Ionic Framework, muito difundido e bastante pesquisado na área, o que fica evidenciado com o gráfico de pesquisbaixo, coletado via Google Trends buscando por frameworks de desenvolvimento híbrido mobile.


\begin{figure}
	\label{fig:graficoGoogleTrendsFerramentasHibridasMobile}
	\includegraphics[width=1\textwidth]{img/grafico_ferramentas_hibridas_mobile}
	\caption{Gráfico do Google Trends exibindo as pesquisas por Frameworks híbridos mobile}
\end{figure}	

Para layout da interface mobile, também será aplicado a tendência do Material Design, a fim de propor uma harmonia entre o módulo web e mobile para os usuários


\section{Resultados Esperados}


Como fruto de um sistema para pós-graduação da UFBA, espera-se que os professores tenham mais recursos para integrar as atividades e também prover melhores condições para acompanhamento da vida acadêmica dos alunos em questão. Também, que os novos colaboradores que entrarem no processo tenham facilidade de compreender o fluxo do setor ao navegar pelo sistema proposto.


\section{Fora de Escopo}


Interação com os alunos devido às complicações para realizar a integração com o sistema empregado na UFBA, gerenciado pela XXXXXX, o que causaria uma inviabilidade no projeto devido à necessidade de entrega do produto ser mais forte que o tempo necessário para executar o processo de obtenção de acesso ao sistema legado para realizar a integração.


\section{Estrutura do Trabalho}


<breve resumo sobre os capítulos do TCC>
\chapter{Referencial Teórico}


Projetar o desenvolvimento de um software requer muito planejamento, pois as falhas iniciais podem custar bastante caro ou até mesmo inviabilizar a continuação de um projeto. Assim, a escolha da arquitetura ideal para a aplicabilidade é essencial na concepção de um produto de software. 
De todo o modo, sempre busca-se fazer mais com menos. Diante de tal filosofia, temos neste capítulo, uma breve discussão sobre alguns elementos de projeto e arquitetura de software, a fim de contextualizar este trabalho de conclusão de curso.
O capítulo corrente é composto por quatro seções. A \ref{sec:saas} trata de Software como serviço, discutindo alguns elementos do contexto que são relevantes para o trabalho proposto. A \ref{sec:reuso}  discute sobre a empregabilidade do reuso de software. A \ref{sec:modularizacao} seção, trata sobre aspectos envolvidos na modularização dos softwares. Por fim, a \ref{sec:apps_web} aborda as aplicações web, discutindo sobre aspectos relevantes sobre a aplicação web que compõe este trabalho.


 \section{Software como serviço}\label{sec:saas}


Segundo La e Chun \citep{La2009Systematic}, o princípio da definição de Software como um Serviço (Sofware as a Service - SaaS) é um serviço complementar para aplicações da computação em nuvem (cloud computing). No entanto, as áreas não se confundem. SaaS deve ser entendido como um mecanismo de suporte às soluções existentes na cloud. Os SaaS existem justamente para maximizar o reuso de serviços repetidos e não centrais em uma aplicação remota.


Como vantagens, diversos fatores podem ser favoráveis para a adoção de um SaaS, como custo e manutenção dentre outros fatores aplicáveis a soluções específicas. Lechesaet al. \citep{LechesaSS11}, quantificam índices sobre os fatores determinantes para adoção ou não de um SaaS voltado para ERP na África do Sul. Os principais fatores determinantes para adoção desse mecanismo de software são sua fluidez quanto à rede e a segurança. Esses fatores estão presentes na aplicação desenvolvida neste trabalho de conclusão de curso.


Devido ao fato de ter um serviço constantemente na nuvem, fica o questionamento sobre a segurança da informação manipulada. Sabe-se que a vulnerabilidade na Web não é restrita ao SaaS, atingindo diversos âmbitos. Rai et al. \cite{journals/corr/RaiSM13} consideram que o avanço da computação em nuvem não é um problema apenas para os serviços Web do ponto de vista da segurança, pois muitos trabalhos na literatura mostram a área como mais um ponto de vulnerabilidade para diversos setores, a exemplo de infraestrutura. No mesmo artigo mencionado de Rai et al. \cite{journals/corr/RaiSM13}, também realizaram-se estudos exploratórios junto a empresas usuárias de serviços em computação em nuvem e consideram que a perspectiva de SaaS também pode fortalecer a segurança nas aplicações de cloud computing, pois o software de autenticação compartilhado por várias aplicações em nuvem, oferece uma melhor padronização e consequente facilidade de prevenção a erros de vulnerabilidade específicas de cada módulo da pesquisa. Esse ponto de vista é fundamenta para a compreensão de SaaS.


A arquitetura de armazenamento de dados de um Saas pode variar de acordo com a necessidade do contexto. Huixin \cite{7586486} descreve possíveis modelagens nesse sentido. Essa abordagem pode ser com um banco de dados único, fazendo com que diferentes clientes compartilhem o mesmo banco, diferindo os dados através de controle de usuário, ou isolando os diferentes clientes através de bancos de dados exclusivos para cada um. Esse fator também pode ser combinado com a arquitetura da aplicação, caso ofereça aplicação única para todos os clientes ou aplicação compartilhada. Diante das possíveis abordagens, a modelagem de dados do software pode ser decidida pela regra de negócio. Este trabalho optou por aplicação única e banco de dados compartilhado.

Devido ao conceito particular de obtenção de software, tanto pela visão do cliente como pela visão do vendedor, é necessário compreender os diversos aspectos que podem ser relevantes ao orçar um Saas. O recente trabalho de T. Kaur et al. \citep{6949281} orienta um modelo para compor o custo de um Saas. O custo total seria composto pelos fatores que dão suporte ao funcionamento do software. Tais fatores incluem infra-estrutura, configurabilidade, customização, parâmetros de QoS (Quality of service) como escalabilidade, disponibilidade, usabilidade, pontualidade e desempenho da resposta, portabilidade, custo total de propriedade e retorno do investimento. Esses fatores caracterizam o custo de forma eficaz, possibilitando ao fornecedor, prover um Serviço de acordo com a exigência do consumidor em vários pacotes de serviços.


O conceito de software como serviço se aplica neste trabalho de conclusão de curso, pois o mesmo estará disponível na web com alta disponibilidade adotando as características apresentadas para qualquer pessoa que desejar usá-lo. Assim, basta se cadastrar, configurá-lo e usar como um serviço, sem preocupação com a manutenção do mesmo.


\section{Reuso de software}\label{sec:reuso} %CRUISE BOOK CAPITULO 2


De acordo com o livro Software pratical reuse \citep{Ezran2002}, o reuso de software é a utilização de qualquer informação que um desenvolvedor pode necessitar no processo de criação de software. O livro de Basili e Rombach \citep{Basili1991} define reutilização de software como o uso de tudo o que está associado a um projeto de conhecimento.
Assim, o objetivo da reutilização de software é reciclar o design, código e outros componentes de um produto de software e assim reduzir o custo, o tempo e melhorar a qualidade do produto.
Segundo Keswani et al. \cite{6783445}, o componente reutilizável de software pode ser qualquer parte de seu desenvolvimento, como um fragmento de código, design, casos de teste, ou até mesmo a especificação de requisitos de uma funcionalidade do software. 

O reuso de software pode ter impacto positivo em diversos aspectos do software, vejamos alguns, conforme apresentados no C.R.U.I.S.E Book \citep{cruise2007} :

\begin{itemize}

\item Qualidade: As correções de erro tornam-se úteis em todos os locais em que ocorreu, padronizando e facilitando a manutenção.

\item Produtividade: O ganho de produtividade é alcançado devido ao menor número de artefatos desenvolvido. Isso resulta em menor esforço de teste e também análise e design, reduzindo custos.

\item Confiabilidade: A utilização de componentes bem testados aumenta a
confiança no software. Além disso, a utilização de um mesmo componente em vários sistemas, aumenta a possibilidade de detecção de erros e reforça a confiança no componente.

\item Redução do Esforço: A reutilização de software proporciona uma redução do tempo de desenvolvimento, o que reduz o tempo necessário para o produto ser disponibilizado no mercado para trazer rentabilidade.

\item Trabalho redundante e tempo de desenvolvimento: Desenvolver um sistema do
zero significa desenvolvimento redundante de muitos componentes, como requisitos, especificações, casos de uso, arquitetura, etc. Isso pode ser evitado quando estes estão disponíveis como componentes reutilizáveis e podem ser compartilhados, resultando em um processo de desenvolvimento otimizado.

\item Documentação: Embora a documentação seja muito importante para a
manutenção de um sistema, muitas vezes é negligenciada. A reutilização de componentes de software reduz a quantidade de documentação a ser escrita, entretanto depende da qualidade do que está escrito. Assim, apenas a estrutura do sistema e os novos artefatos desenvolvidos necessitam ser documentados.

\item Custo de manutenção: Menos defeitos e manutenções são esperados quando tem-se comprovada a qualidade dos componentes utilizados.

\item Tamanho da equipe: É comum haver casos em que a equipe de desenvolvimento sofre sobrecarga. Entretanto, dobrar o tamanho da equipe de desenvolvimento não necessariamente duplica produtividade. Se muitos componentes podem ser reutilizados, é possível desenvolver com equipes menores, levando a melhor comunicação e aumento da produtividade.

\end{itemize}

Apesar dos benefícios da reutilização de software, ela não é suficientemente aproveitada. Existem fatores que influenciam direta ou indiretamente na sua adoção. Esses fatores podem ser de aspecto gerencial, organizacional, econômico, conceitual ou técnico. Veremos a seguir alguns aspectos que podem gerar conflito com a cultura de reuso de software, segundo o C.R.U.I.S.E Book \citep{cruise2007} :
%(Sametinger, 1997). REVER

\begin{itemize}
	
\item Falta de apoio da gestão: Como a reutilização de software gera custos iniciais,
a medida pode não ser amplamente alcançada em uma organização sem o apoio de alto nível de gestão. Os gestores têm de ser informados sobre os custos iniciais e serem convencidos sobre economias futuras.

\item Gerenciamento do Projeto: Gerenciar projetos tradicionais é uma tarefa árdua, principalmente, os que praticam a reutilização de software. Utilizando a técnica em larga escala, tem-se impacto sobre todo o ciclo de vida do software.

\item Estruturas organizacionais inadequadas: As estruturas organizacionais devem
considerar diferentes necessidades que surgem quando a reutilização em larga escala está sendo adotada. Por exemplo, uma equipe particionada pode ser alocada somente para desenvolver, manter e certificar componentes reutilizáveis de software.

\item Incentivos de gestão: É comum a falta de incentivo para deixar os desenvolvedores gastarem tempo elaborando componentes do sistemas. A produtividade é muitas vezes medida apenas no tempo necessário para concluir um projeto. Assim, fazer qualquer trabalho além disso, embora benéfico para a empresa como um todo, diminui o seu sucesso. Mesmo quando os componentes reutilizáveis são utilizados, os benefícios obtidos são uma pequena fração do que poderia ser alcançado caso houvesse reutilização explícita, planejada e organizada.

\item Dificuldade de encontrar software reutilizável: Para reutilizar os componentes, devem existir formas eficientes de busca. Além disso, é importante ter um repositório bem organizado contendo componentes com um eficiente meio de acesso.

\item Não reutilização do software encontrado. O acesso fácil ao software existente
não necessariamente aumentar a reutilização. Os componentes reutilizáveis devem ser cuidadosamente especificados, projetados, implementados e documentados, pois em alguns casos, modificar e adaptar o código  pode ser mais custoso que a programação da funcionalidade necessária a partir do zero.

\item Modificação: É muito difícil encontrar um componente que funcione
exatamente da mesma maneira que queremos. Desta forma, são necessárias modificações e devem existir formas de determinar os seus efeitos sobre o componente.


\end{itemize}


A definição do reuso de software pode ser associada a este trabalho, pois ainda que como um serviço, estaremos adotando uma aplicação única para servir aos mais diversos softwares de uma empresa, por exemplo. Logo, ao adotar o projeto, pode ser interessante aplicar algumas métricas para tomar conhecimento de possíveis vantanjens como as citadas nessa seção, a exemplo de economia de tempo de desenvolvimento, e custo do projeto.


%Outra diretriz importante para a reutilização de software é reduzir o risco na criação de novos softwares. O risco tende a ser bastante reduzido se os componentes que estão sendo reutilizados têm as documentação, interfaces necessárias e devidamente testadas, fatores que contibruem para uma fácil integração.
%De acordo com Keswani et al. \citep{6783445}, para o reuso de software dar retornos apropriados, o processo deve ser sistemático e planejado. Qualquer organização que implemente a reutilização de software deve identificar os melhores métodos e estratégias de reutilização para obter a máxima produtividade. A reutilização de software ajuda a evitar software de engenharia a partir do zero, pois usa módulos de software existentes. A reutilização de software, embora seja uma tarefa difícil, especialmente para softwares antigos sem padrões de projeto, pode melhorar significativamente a produtividade e a qualidade de um produto de software. Embora a reutilização de software não seja um novo campo, ela pode dar grandes retornos em curto período de tempo.


\section{Modularização}\label{sec:modularizacao} %artigo de claudio pagina 222 introdução


%A modularidade vem desempenhando um papel predominante estágios emergentes das disciplinas de arquitetura de software [13]. Engenheiros de software consideram modularidade como princípio base na comparação entre arquiteturas alternativas  e arquitetura degeneração [9]. De fato, os engenheiros de software são incentivados a arquitecturas, baseando-se numa multiplicidade de mecanismos de modularidade disponíveis em: 
%(i) Linguagens de descrição de arquitetura (ADLs), como ACME [8], 
%(ii) catálogos de arquitetônicos [2, 13], e 
%(iii) conhecem bem princípios de alto nível, como interfaces de componentes estreitos, acoplamento arquitectónico reduzido e semelhantes.


Conforme é frisado por Wickramaarachchi e Lai \citep{7062705}, o conceito de modularização na indústria de software tem uma longa história e tem sido utilizado para melhorar o processo de desenvolvimento de software em diferentes estágios. Os principais conceitos por trás da modularização do software foram introduzidos por pesquisadores pioneiros há quarenta anos, com uma notável contribuição feita por Melvin Conway e David Parnas, que tem representação notável na engenharia de software.


Modularizar um software é um bom padrão a ser adotado. Segundo Wickramaarachchi e Lai \citep{7062705}, a modularização é importante na identificação de dependências e reduz as dificuldades diante de uma possível necessidade de grandes alterações. De uma perspectiva da engenharia de software, uma modularização geralmente tem várias vantagens, tais como: tornar a complexidade do software mais gerenciável, facilitar o trabalho paralelo e tornar o software mais maleável para acomodar o futuro incerto que um software pode ter. O objetivo final da modularização do software é aumentar a produtividade ea qualidade do software. Tal conceito encontra-se bastante difundido e estái incorporado em linguagens de programação e ferramentas de software. O presente trabalho favorece ao uso da modularização de um software e até mesmo pode ser considerado um módulo a ser acoplado a qualquer software, mediante a compatibilidade.


Ao realizar a adoção do trabalho proposto, fica bastante evidente a "responsabilidade" do mesmo no escopo do projeto. Assim, é possível tratá-lo como um módulo do projeto que o usa, mesmo que seja consumido como um serviço.


\section{Aplicações web}\label{sec:apps_web}


A popularidade das soluções Web aumentou exponencialmente na última década e todos os dias cresce o número de pessoas usuárias desse tipo de software. E seguindo um padrão próprio, Kumar et al. \citep{7813710} sugerem que para o desenvolvimento web, deve-se manter a prática eficaz de produzir diagramas UML. A abordagem baseada na web oferece uma maneira fácil e eficaz para gerenciar e controlar o processo de desenvolvimento por meio de artefatos de modelagem. Tal abordagem pode ser usada quando há uma exigência de lidar com mudanças muito rápidas e grandes em requisitos de forma muito eficaz em muito menos tempo, gerando assim um menor impacto. 


Para atender à fomentada demanda de aplicativos web, é necessário adotar métodos de desenvolvimentos que sejam ágeis, eficientes e de fácil manutenção. Yu Ping et al. \cite{1372143} propõem o uso do modelo MVC (Model, View e Controller) no desenvolvimento para softwares web. O modelo apresentado tornou-se um padrão popular e divide o software em camadas com propósito definido, tornando-o de mais fácil manutenção.


O Ajax (Asynchronous Javascript and XML) revolucionou a web. Conforme demonstrado por Yuping \citep{6845605}, ao usar a tecnologia Ajax, podemos enriquecer a experiência do usuário em aplicações baseadas em navegador de internet, e fornecer uma variedade de aplicações interativas para atender às necessidade de humanização das aplicações.
Os aplicativos Ajax em execução no navegador se comunicam com um servidor Web de forma assíncrona e atualizam apenas uma parte da página.


No artigo de Tesarik et al. \citep{4664349}, temos que o design de software SPA (Single page application) é uma maneira interessante de criar um software Web numa uma única página. Essa solução de página única sem navegação funciona apenas com base em técnicas dinâmicas e assíncronas, como o AJAX (citado no parágrafo anterior). No entanto, esta abordagem coloca o desenvolvedor antes de alguns desafios substanciais.
Para projetar a interface do usuário que mostra as informações, é importante projetar corretamente a tela para manipular os dados do aplicativo numa única página. O design da página deve ser elaborado para maximizar a decomposição da página em componentes distintos que encapsulam os principais casos de uso.
Também recomenda-se que para elaborar os artefatos visuais, explore os mais ricos recursos na implementação, como HTML5, JavaScript, Ajax, CSS3, e outras tecnologias que se apliquem. Entretanto, desenvolver uma rica interface com o uso de diversas tecnologias/frameworks pode ocasionar um esforço maior para explorar as possibilidades.
Tais componentes da interface de uma aplicação SPA normalmente são alimentados mediante o consumo de uma API Rest via requisições AJAX. Assim, é possível particionar as responsabilidades de processamento do software entre cliente e servidor.


Como é possível denotar no trabalho de Salvadori e Siqueire \citep{7195633}, REST é uma arquitetura muito popular para integração de aplicativos web, que permite compartilhar e reutilizar informações através de sistemas. Aplicações de grande escala baseadas em no REST deve ser implementadas utilizando estratégias e mecanismos para produzir sistemas que sejam fáceis de desenvolver, reutilizar e manter. As interfaces de integração fornecidas por esses sistemas, chamadas de Web API, têm uma influência importante nas características da implementação resultante, pois a sua forma de resposta deve ser casada com a interface que o consome.
 
 
Os elementos comentados nessa seção encontram-se presentes neste trabalho de conclusão de curso. Juntos, montam a estrutura tecnológica necessária aliada à arquitetura adotada. Esses elementos seguem tendências atuais dos softwares web, fazendo com que esse trabalho esteja composto por tecnologias modernas que estão com boa aceitação no mercado.

% %% RiSE Latex Template - version 0.5
%%
%% RiSE's latex template for thesis and dissertations
%% http://risetemplate.sourceforge.net
%%
%% (c) 2012 Yguaratã Cerqueira Cavalcanti (yguarata@gmail.com)
%%          Vinicius Cardoso Garcia (vinicius.garcia@gmail.com)
%%
%% This document was initially based on UFPEThesis template, from Paulo Gustavo
%% S. Fonseca.
%%
%% ACKNOWLEDGEMENTS
%%
%% We would like to thanks the RiSE's researchers community, the 
%% students from Federal University of Pernambuco, and other users that have
%% been contributing to this projects with comments and patches.
%%
%% GENERAL INSTRUCTIONS
%%
%% We strongly recommend you to compile your documents using pdflatex command.
%% It is also recommend use the texlipse plugin for Eclipse to edit your documents.
%%
%% Options for \documentclass command:
%%         * Idiom
%%           pt   - Portguese (default)
%%           en   - English
%%
%%         * Text type
%%           bsc  - B.Sc. Thesis
%%           msc  - M.Sc. Thesis (default)
%%           qual - PHD qualification (not tested yet)
%%           prop - PHD proposal (not tested yet)
%%           phd  - PHD thesis
%%
%%         * Media
%%           scr  - to eletronic version (PDF) / see the users guide
%%
%%         * Pagination
%%           oneside - unique face press
%%           twoside - two faces press
%%
%%		   * Line spacing
%%           singlespacing  - the same as using \linespread{1}
%%           onehalfspacing - the same as using \linespread{1.3}
%%           doublespacing  - the same as using \linespread{1.6}
%%
%% Reference commands. Use the following commands to make references in your
%% text:
%%          \figref  -- for Figure reference
%%          \tabref  -- for Table reference
%%          \eqnref  -- for equation reference
%%          \chapref -- for chapter reference
%%          \secref  -- for section reference
%%          \appref  -- for appendix reference
%%          \axiref  -- for axiom reference
%%          \conjref -- for conjecture reference
%%          \defref  -- for definition reference
%%          \lemref  -- for lemma reference
%%          \theoref -- for theorem reference
%%          \corref  -- for corollary reference
%%          \propref -- for proprosition reference
%%          \pgref   -- for page reference
%%
%%          Example: See \chapref{chap:introduction}. It will produce 
%%                   'See Chapter 1', in case of English language.

\documentclass[pt,twoside,onehalfspacing,bsc]{risethesis}

\usepackage[sort,square]{natbib} %package for bibliography
\bibliographystyle{ieeetr} %authordate1

\usepackage[utf8]{inputenc}
\usepackage[brazilian]{babel}
\usepackage[T1]{fontenc}

\usepackage{graphicx}

%% Change the following pdf author attribute name to your name.
\usepackage[linkcolor=blue,citecolor=blue,urlcolor=blue,colorlinks,pdfpagelabels,pdftitle={Bruno Cabral's Bachelor Thesis},pdfauthor={Bruno Cabral}]{hyperref}

\address{SALVADOR}

\universitypt{Universidade Federal da Bahia}
\universityen{Federal University of Bahia}

\departmentpt{Depertamento de Ciência da Computação}
\departmenten{Computer Science Department}

\programpt{Programa Multiinstitucional de Pós-graduação em Ciência da Computação}
\programen{Graduate in Computer Science}

\majorfieldpt{Ciência da Computação}
\majorfielden{Computer Science}

\title{Sistema de apoio à Pós graduação - UFBA}
\date{Outubro/2016}

\author{Victor de Azevedo Nunes}
\adviser{Ivan do Carmo Machado}

\begin{document}

\frontmatter
\frontpage
\presentationpage

\begin{dedicatory}
Eu dedico esta dissertação...
%I dedicate this dissertation to my family, girlfriend, friends and
%professors who gave me all necessary support to get here.
\end{dedicatory}

\acknowledgements
Meus agradecimentos...

\begin{epigraph}[]{Edward V Berard}
Walking on water and developing software from a specification are easy if both are frozen
\end{epigraph}

\resumo
% Escreva seu resumo no arquivo resumo.tex
\input{resumo}

\abstract
% Write your abstract in a file called abstract.tex
\input{abstract}

% Summary (tables of contents)
\tableofcontents

% List of figures
\listoffigures

% List of tables
\listoftables

% List of acronyms
% Acronyms manual: http://linorg.usp.br/CTAN/macros/latex/contrib/acronym/acronym.pdf
\listofacronyms
\input{acronyms}

% List of listings
%\lstlistoflistings

\mainmatter

\include{chapters/intro}
\include{chapters/referencial_teorico}

% \include{chapters/introduction/main}
% \include{chapters/background/main}
% \include{chapters/proposed_solution/main}
% \include{chapters/experiment/main}
% \include{chapters/conclusion/main}

%\bibliographystyle{natbib}
%\addcontentsline{toc}{chapter}{\bibliographytocname}
\bibliography{references}

% Appendix
\clearpage
\addappheadtotoc
\appendix
\appendixpage
% \include{appendix/experiment-instruments}

\end{document}
% %% RiSE Latex Template - version 0.5
%%
%% RiSE's latex template for thesis and dissertations
%% http://risetemplate.sourceforge.net
%%
%% (c) 2012 Yguaratã Cerqueira Cavalcanti (yguarata@gmail.com)
%%          Vinicius Cardoso Garcia (vinicius.garcia@gmail.com)
%%
%% This document was initially based on UFPEThesis template, from Paulo Gustavo
%% S. Fonseca.
%%
%% ACKNOWLEDGEMENTS
%%
%% We would like to thanks the RiSE's researchers community, the 
%% students from Federal University of Pernambuco, and other users that have
%% been contributing to this projects with comments and patches.
%%
%% GENERAL INSTRUCTIONS
%%
%% We strongly recommend you to compile your documents using pdflatex command.
%% It is also recommend use the texlipse plugin for Eclipse to edit your documents.
%%
%% Options for \documentclass command:
%%         * Idiom
%%           pt   - Portguese (default)
%%           en   - English
%%
%%         * Text type
%%           bsc  - B.Sc. Thesis
%%           msc  - M.Sc. Thesis (default)
%%           qual - PHD qualification (not tested yet)
%%           prop - PHD proposal (not tested yet)
%%           phd  - PHD thesis
%%
%%         * Media
%%           scr  - to eletronic version (PDF) / see the users guide
%%
%%         * Pagination
%%           oneside - unique face press
%%           twoside - two faces press
%%
%%		   * Line spacing
%%           singlespacing  - the same as using \linespread{1}
%%           onehalfspacing - the same as using \linespread{1.3}
%%           doublespacing  - the same as using \linespread{1.6}
%%
%% Reference commands. Use the following commands to make references in your
%% text:
%%          \figref  -- for Figure reference
%%          \tabref  -- for Table reference
%%          \eqnref  -- for equation reference
%%          \chapref -- for chapter reference
%%          \secref  -- for section reference
%%          \appref  -- for appendix reference
%%          \axiref  -- for axiom reference
%%          \conjref -- for conjecture reference
%%          \defref  -- for definition reference
%%          \lemref  -- for lemma reference
%%          \theoref -- for theorem reference
%%          \corref  -- for corollary reference
%%          \propref -- for proprosition reference
%%          \pgref   -- for page reference
%%
%%          Example: See \chapref{chap:introduction}. It will produce 
%%                   'See Chapter 1', in case of English language.

\documentclass[pt,twoside,onehalfspacing,bsc]{risethesis}

\usepackage[sort,square]{natbib} %package for bibliography
\bibliographystyle{ieeetr} %authordate1

\usepackage[utf8]{inputenc}
\usepackage[brazilian]{babel}
\usepackage[T1]{fontenc}

\usepackage{graphicx}

%% Change the following pdf author attribute name to your name.
\usepackage[linkcolor=blue,citecolor=blue,urlcolor=blue,colorlinks,pdfpagelabels,pdftitle={Bruno Cabral's Bachelor Thesis},pdfauthor={Bruno Cabral}]{hyperref}

\address{SALVADOR}

\universitypt{Universidade Federal da Bahia}
\universityen{Federal University of Bahia}

\departmentpt{Depertamento de Ciência da Computação}
\departmenten{Computer Science Department}

\programpt{Programa Multiinstitucional de Pós-graduação em Ciência da Computação}
\programen{Graduate in Computer Science}

\majorfieldpt{Ciência da Computação}
\majorfielden{Computer Science}

\title{Sistema de apoio à Pós graduação - UFBA}
\date{Outubro/2016}

\author{Victor de Azevedo Nunes}
\adviser{Ivan do Carmo Machado}

\begin{document}

\frontmatter
\frontpage
\presentationpage

\begin{dedicatory}
Eu dedico esta dissertação...
%I dedicate this dissertation to my family, girlfriend, friends and
%professors who gave me all necessary support to get here.
\end{dedicatory}

\acknowledgements
Meus agradecimentos...

\begin{epigraph}[]{Edward V Berard}
Walking on water and developing software from a specification are easy if both are frozen
\end{epigraph}

\resumo
% Escreva seu resumo no arquivo resumo.tex
\input{resumo}

\abstract
% Write your abstract in a file called abstract.tex
\input{abstract}

% Summary (tables of contents)
\tableofcontents

% List of figures
\listoffigures

% List of tables
\listoftables

% List of acronyms
% Acronyms manual: http://linorg.usp.br/CTAN/macros/latex/contrib/acronym/acronym.pdf
\listofacronyms
\input{acronyms}

% List of listings
%\lstlistoflistings

\mainmatter

\include{chapters/intro}
\include{chapters/referencial_teorico}

% \include{chapters/introduction/main}
% \include{chapters/background/main}
% \include{chapters/proposed_solution/main}
% \include{chapters/experiment/main}
% \include{chapters/conclusion/main}

%\bibliographystyle{natbib}
%\addcontentsline{toc}{chapter}{\bibliographytocname}
\bibliography{references}

% Appendix
\clearpage
\addappheadtotoc
\appendix
\appendixpage
% \include{appendix/experiment-instruments}

\end{document}
% %% RiSE Latex Template - version 0.5
%%
%% RiSE's latex template for thesis and dissertations
%% http://risetemplate.sourceforge.net
%%
%% (c) 2012 Yguaratã Cerqueira Cavalcanti (yguarata@gmail.com)
%%          Vinicius Cardoso Garcia (vinicius.garcia@gmail.com)
%%
%% This document was initially based on UFPEThesis template, from Paulo Gustavo
%% S. Fonseca.
%%
%% ACKNOWLEDGEMENTS
%%
%% We would like to thanks the RiSE's researchers community, the 
%% students from Federal University of Pernambuco, and other users that have
%% been contributing to this projects with comments and patches.
%%
%% GENERAL INSTRUCTIONS
%%
%% We strongly recommend you to compile your documents using pdflatex command.
%% It is also recommend use the texlipse plugin for Eclipse to edit your documents.
%%
%% Options for \documentclass command:
%%         * Idiom
%%           pt   - Portguese (default)
%%           en   - English
%%
%%         * Text type
%%           bsc  - B.Sc. Thesis
%%           msc  - M.Sc. Thesis (default)
%%           qual - PHD qualification (not tested yet)
%%           prop - PHD proposal (not tested yet)
%%           phd  - PHD thesis
%%
%%         * Media
%%           scr  - to eletronic version (PDF) / see the users guide
%%
%%         * Pagination
%%           oneside - unique face press
%%           twoside - two faces press
%%
%%		   * Line spacing
%%           singlespacing  - the same as using \linespread{1}
%%           onehalfspacing - the same as using \linespread{1.3}
%%           doublespacing  - the same as using \linespread{1.6}
%%
%% Reference commands. Use the following commands to make references in your
%% text:
%%          \figref  -- for Figure reference
%%          \tabref  -- for Table reference
%%          \eqnref  -- for equation reference
%%          \chapref -- for chapter reference
%%          \secref  -- for section reference
%%          \appref  -- for appendix reference
%%          \axiref  -- for axiom reference
%%          \conjref -- for conjecture reference
%%          \defref  -- for definition reference
%%          \lemref  -- for lemma reference
%%          \theoref -- for theorem reference
%%          \corref  -- for corollary reference
%%          \propref -- for proprosition reference
%%          \pgref   -- for page reference
%%
%%          Example: See \chapref{chap:introduction}. It will produce 
%%                   'See Chapter 1', in case of English language.

\documentclass[pt,twoside,onehalfspacing,bsc]{risethesis}

\usepackage[sort,square]{natbib} %package for bibliography
\bibliographystyle{ieeetr} %authordate1

\usepackage[utf8]{inputenc}
\usepackage[brazilian]{babel}
\usepackage[T1]{fontenc}

\usepackage{graphicx}

%% Change the following pdf author attribute name to your name.
\usepackage[linkcolor=blue,citecolor=blue,urlcolor=blue,colorlinks,pdfpagelabels,pdftitle={Bruno Cabral's Bachelor Thesis},pdfauthor={Bruno Cabral}]{hyperref}

\address{SALVADOR}

\universitypt{Universidade Federal da Bahia}
\universityen{Federal University of Bahia}

\departmentpt{Depertamento de Ciência da Computação}
\departmenten{Computer Science Department}

\programpt{Programa Multiinstitucional de Pós-graduação em Ciência da Computação}
\programen{Graduate in Computer Science}

\majorfieldpt{Ciência da Computação}
\majorfielden{Computer Science}

\title{Sistema de apoio à Pós graduação - UFBA}
\date{Outubro/2016}

\author{Victor de Azevedo Nunes}
\adviser{Ivan do Carmo Machado}

\begin{document}

\frontmatter
\frontpage
\presentationpage

\begin{dedicatory}
Eu dedico esta dissertação...
%I dedicate this dissertation to my family, girlfriend, friends and
%professors who gave me all necessary support to get here.
\end{dedicatory}

\acknowledgements
Meus agradecimentos...

\begin{epigraph}[]{Edward V Berard}
Walking on water and developing software from a specification are easy if both are frozen
\end{epigraph}

\resumo
% Escreva seu resumo no arquivo resumo.tex
\input{resumo}

\abstract
% Write your abstract in a file called abstract.tex
\input{abstract}

% Summary (tables of contents)
\tableofcontents

% List of figures
\listoffigures

% List of tables
\listoftables

% List of acronyms
% Acronyms manual: http://linorg.usp.br/CTAN/macros/latex/contrib/acronym/acronym.pdf
\listofacronyms
\input{acronyms}

% List of listings
%\lstlistoflistings

\mainmatter

\include{chapters/intro}
\include{chapters/referencial_teorico}

% \include{chapters/introduction/main}
% \include{chapters/background/main}
% \include{chapters/proposed_solution/main}
% \include{chapters/experiment/main}
% \include{chapters/conclusion/main}

%\bibliographystyle{natbib}
%\addcontentsline{toc}{chapter}{\bibliographytocname}
\bibliography{references}

% Appendix
\clearpage
\addappheadtotoc
\appendix
\appendixpage
% \include{appendix/experiment-instruments}

\end{document}
% %% RiSE Latex Template - version 0.5
%%
%% RiSE's latex template for thesis and dissertations
%% http://risetemplate.sourceforge.net
%%
%% (c) 2012 Yguaratã Cerqueira Cavalcanti (yguarata@gmail.com)
%%          Vinicius Cardoso Garcia (vinicius.garcia@gmail.com)
%%
%% This document was initially based on UFPEThesis template, from Paulo Gustavo
%% S. Fonseca.
%%
%% ACKNOWLEDGEMENTS
%%
%% We would like to thanks the RiSE's researchers community, the 
%% students from Federal University of Pernambuco, and other users that have
%% been contributing to this projects with comments and patches.
%%
%% GENERAL INSTRUCTIONS
%%
%% We strongly recommend you to compile your documents using pdflatex command.
%% It is also recommend use the texlipse plugin for Eclipse to edit your documents.
%%
%% Options for \documentclass command:
%%         * Idiom
%%           pt   - Portguese (default)
%%           en   - English
%%
%%         * Text type
%%           bsc  - B.Sc. Thesis
%%           msc  - M.Sc. Thesis (default)
%%           qual - PHD qualification (not tested yet)
%%           prop - PHD proposal (not tested yet)
%%           phd  - PHD thesis
%%
%%         * Media
%%           scr  - to eletronic version (PDF) / see the users guide
%%
%%         * Pagination
%%           oneside - unique face press
%%           twoside - two faces press
%%
%%		   * Line spacing
%%           singlespacing  - the same as using \linespread{1}
%%           onehalfspacing - the same as using \linespread{1.3}
%%           doublespacing  - the same as using \linespread{1.6}
%%
%% Reference commands. Use the following commands to make references in your
%% text:
%%          \figref  -- for Figure reference
%%          \tabref  -- for Table reference
%%          \eqnref  -- for equation reference
%%          \chapref -- for chapter reference
%%          \secref  -- for section reference
%%          \appref  -- for appendix reference
%%          \axiref  -- for axiom reference
%%          \conjref -- for conjecture reference
%%          \defref  -- for definition reference
%%          \lemref  -- for lemma reference
%%          \theoref -- for theorem reference
%%          \corref  -- for corollary reference
%%          \propref -- for proprosition reference
%%          \pgref   -- for page reference
%%
%%          Example: See \chapref{chap:introduction}. It will produce 
%%                   'See Chapter 1', in case of English language.

\documentclass[pt,twoside,onehalfspacing,bsc]{risethesis}

\usepackage[sort,square]{natbib} %package for bibliography
\bibliographystyle{ieeetr} %authordate1

\usepackage[utf8]{inputenc}
\usepackage[brazilian]{babel}
\usepackage[T1]{fontenc}

\usepackage{graphicx}

%% Change the following pdf author attribute name to your name.
\usepackage[linkcolor=blue,citecolor=blue,urlcolor=blue,colorlinks,pdfpagelabels,pdftitle={Bruno Cabral's Bachelor Thesis},pdfauthor={Bruno Cabral}]{hyperref}

\address{SALVADOR}

\universitypt{Universidade Federal da Bahia}
\universityen{Federal University of Bahia}

\departmentpt{Depertamento de Ciência da Computação}
\departmenten{Computer Science Department}

\programpt{Programa Multiinstitucional de Pós-graduação em Ciência da Computação}
\programen{Graduate in Computer Science}

\majorfieldpt{Ciência da Computação}
\majorfielden{Computer Science}

\title{Sistema de apoio à Pós graduação - UFBA}
\date{Outubro/2016}

\author{Victor de Azevedo Nunes}
\adviser{Ivan do Carmo Machado}

\begin{document}

\frontmatter
\frontpage
\presentationpage

\begin{dedicatory}
Eu dedico esta dissertação...
%I dedicate this dissertation to my family, girlfriend, friends and
%professors who gave me all necessary support to get here.
\end{dedicatory}

\acknowledgements
Meus agradecimentos...

\begin{epigraph}[]{Edward V Berard}
Walking on water and developing software from a specification are easy if both are frozen
\end{epigraph}

\resumo
% Escreva seu resumo no arquivo resumo.tex
\input{resumo}

\abstract
% Write your abstract in a file called abstract.tex
\input{abstract}

% Summary (tables of contents)
\tableofcontents

% List of figures
\listoffigures

% List of tables
\listoftables

% List of acronyms
% Acronyms manual: http://linorg.usp.br/CTAN/macros/latex/contrib/acronym/acronym.pdf
\listofacronyms
\input{acronyms}

% List of listings
%\lstlistoflistings

\mainmatter

\include{chapters/intro}
\include{chapters/referencial_teorico}

% \include{chapters/introduction/main}
% \include{chapters/background/main}
% \include{chapters/proposed_solution/main}
% \include{chapters/experiment/main}
% \include{chapters/conclusion/main}

%\bibliographystyle{natbib}
%\addcontentsline{toc}{chapter}{\bibliographytocname}
\bibliography{references}

% Appendix
\clearpage
\addappheadtotoc
\appendix
\appendixpage
% \include{appendix/experiment-instruments}

\end{document}
% %% RiSE Latex Template - version 0.5
%%
%% RiSE's latex template for thesis and dissertations
%% http://risetemplate.sourceforge.net
%%
%% (c) 2012 Yguaratã Cerqueira Cavalcanti (yguarata@gmail.com)
%%          Vinicius Cardoso Garcia (vinicius.garcia@gmail.com)
%%
%% This document was initially based on UFPEThesis template, from Paulo Gustavo
%% S. Fonseca.
%%
%% ACKNOWLEDGEMENTS
%%
%% We would like to thanks the RiSE's researchers community, the 
%% students from Federal University of Pernambuco, and other users that have
%% been contributing to this projects with comments and patches.
%%
%% GENERAL INSTRUCTIONS
%%
%% We strongly recommend you to compile your documents using pdflatex command.
%% It is also recommend use the texlipse plugin for Eclipse to edit your documents.
%%
%% Options for \documentclass command:
%%         * Idiom
%%           pt   - Portguese (default)
%%           en   - English
%%
%%         * Text type
%%           bsc  - B.Sc. Thesis
%%           msc  - M.Sc. Thesis (default)
%%           qual - PHD qualification (not tested yet)
%%           prop - PHD proposal (not tested yet)
%%           phd  - PHD thesis
%%
%%         * Media
%%           scr  - to eletronic version (PDF) / see the users guide
%%
%%         * Pagination
%%           oneside - unique face press
%%           twoside - two faces press
%%
%%		   * Line spacing
%%           singlespacing  - the same as using \linespread{1}
%%           onehalfspacing - the same as using \linespread{1.3}
%%           doublespacing  - the same as using \linespread{1.6}
%%
%% Reference commands. Use the following commands to make references in your
%% text:
%%          \figref  -- for Figure reference
%%          \tabref  -- for Table reference
%%          \eqnref  -- for equation reference
%%          \chapref -- for chapter reference
%%          \secref  -- for section reference
%%          \appref  -- for appendix reference
%%          \axiref  -- for axiom reference
%%          \conjref -- for conjecture reference
%%          \defref  -- for definition reference
%%          \lemref  -- for lemma reference
%%          \theoref -- for theorem reference
%%          \corref  -- for corollary reference
%%          \propref -- for proprosition reference
%%          \pgref   -- for page reference
%%
%%          Example: See \chapref{chap:introduction}. It will produce 
%%                   'See Chapter 1', in case of English language.

\documentclass[pt,twoside,onehalfspacing,bsc]{risethesis}

\usepackage[sort,square]{natbib} %package for bibliography
\bibliographystyle{ieeetr} %authordate1

\usepackage[utf8]{inputenc}
\usepackage[brazilian]{babel}
\usepackage[T1]{fontenc}

\usepackage{graphicx}

%% Change the following pdf author attribute name to your name.
\usepackage[linkcolor=blue,citecolor=blue,urlcolor=blue,colorlinks,pdfpagelabels,pdftitle={Bruno Cabral's Bachelor Thesis},pdfauthor={Bruno Cabral}]{hyperref}

\address{SALVADOR}

\universitypt{Universidade Federal da Bahia}
\universityen{Federal University of Bahia}

\departmentpt{Depertamento de Ciência da Computação}
\departmenten{Computer Science Department}

\programpt{Programa Multiinstitucional de Pós-graduação em Ciência da Computação}
\programen{Graduate in Computer Science}

\majorfieldpt{Ciência da Computação}
\majorfielden{Computer Science}

\title{Sistema de apoio à Pós graduação - UFBA}
\date{Outubro/2016}

\author{Victor de Azevedo Nunes}
\adviser{Ivan do Carmo Machado}

\begin{document}

\frontmatter
\frontpage
\presentationpage

\begin{dedicatory}
Eu dedico esta dissertação...
%I dedicate this dissertation to my family, girlfriend, friends and
%professors who gave me all necessary support to get here.
\end{dedicatory}

\acknowledgements
Meus agradecimentos...

\begin{epigraph}[]{Edward V Berard}
Walking on water and developing software from a specification are easy if both are frozen
\end{epigraph}

\resumo
% Escreva seu resumo no arquivo resumo.tex
\input{resumo}

\abstract
% Write your abstract in a file called abstract.tex
\input{abstract}

% Summary (tables of contents)
\tableofcontents

% List of figures
\listoffigures

% List of tables
\listoftables

% List of acronyms
% Acronyms manual: http://linorg.usp.br/CTAN/macros/latex/contrib/acronym/acronym.pdf
\listofacronyms
\input{acronyms}

% List of listings
%\lstlistoflistings

\mainmatter

\include{chapters/intro}
\include{chapters/referencial_teorico}

% \include{chapters/introduction/main}
% \include{chapters/background/main}
% \include{chapters/proposed_solution/main}
% \include{chapters/experiment/main}
% \include{chapters/conclusion/main}

%\bibliographystyle{natbib}
%\addcontentsline{toc}{chapter}{\bibliographytocname}
\bibliography{references}

% Appendix
\clearpage
\addappheadtotoc
\appendix
\appendixpage
% \include{appendix/experiment-instruments}

\end{document}

%\bibliographystyle{natbib}
%\addcontentsline{toc}{chapter}{\bibliographytocname}
\bibliography{references}

% Appendix
\clearpage
\addappheadtotoc
\appendix
\appendixpage
% \include{appendix/experiment-instruments}

\end{document}
% %% RiSE Latex Template - version 0.5
%%
%% RiSE's latex template for thesis and dissertations
%% http://risetemplate.sourceforge.net
%%
%% (c) 2012 Yguaratã Cerqueira Cavalcanti (yguarata@gmail.com)
%%          Vinicius Cardoso Garcia (vinicius.garcia@gmail.com)
%%
%% This document was initially based on UFPEThesis template, from Paulo Gustavo
%% S. Fonseca.
%%
%% ACKNOWLEDGEMENTS
%%
%% We would like to thanks the RiSE's researchers community, the 
%% students from Federal University of Pernambuco, and other users that have
%% been contributing to this projects with comments and patches.
%%
%% GENERAL INSTRUCTIONS
%%
%% We strongly recommend you to compile your documents using pdflatex command.
%% It is also recommend use the texlipse plugin for Eclipse to edit your documents.
%%
%% Options for \documentclass command:
%%         * Idiom
%%           pt   - Portguese (default)
%%           en   - English
%%
%%         * Text type
%%           bsc  - B.Sc. Thesis
%%           msc  - M.Sc. Thesis (default)
%%           qual - PHD qualification (not tested yet)
%%           prop - PHD proposal (not tested yet)
%%           phd  - PHD thesis
%%
%%         * Media
%%           scr  - to eletronic version (PDF) / see the users guide
%%
%%         * Pagination
%%           oneside - unique face press
%%           twoside - two faces press
%%
%%		   * Line spacing
%%           singlespacing  - the same as using \linespread{1}
%%           onehalfspacing - the same as using \linespread{1.3}
%%           doublespacing  - the same as using \linespread{1.6}
%%
%% Reference commands. Use the following commands to make references in your
%% text:
%%          \figref  -- for Figure reference
%%          \tabref  -- for Table reference
%%          \eqnref  -- for equation reference
%%          \chapref -- for chapter reference
%%          \secref  -- for section reference
%%          \appref  -- for appendix reference
%%          \axiref  -- for axiom reference
%%          \conjref -- for conjecture reference
%%          \defref  -- for definition reference
%%          \lemref  -- for lemma reference
%%          \theoref -- for theorem reference
%%          \corref  -- for corollary reference
%%          \propref -- for proprosition reference
%%          \pgref   -- for page reference
%%
%%          Example: See \chapref{chap:introduction}. It will produce 
%%                   'See Chapter 1', in case of English language.

\documentclass[pt,twoside,onehalfspacing,bsc]{risethesis}

\usepackage[sort,square]{natbib} %package for bibliography
\bibliographystyle{ieeetr} %authordate1

\usepackage[utf8]{inputenc}
\usepackage[brazilian]{babel}
\usepackage[T1]{fontenc}

\usepackage{graphicx}

%% Change the following pdf author attribute name to your name.
\usepackage[linkcolor=blue,citecolor=blue,urlcolor=blue,colorlinks,pdfpagelabels,pdftitle={Bruno Cabral's Bachelor Thesis},pdfauthor={Bruno Cabral}]{hyperref}

\address{SALVADOR}

\universitypt{Universidade Federal da Bahia}
\universityen{Federal University of Bahia}

\departmentpt{Depertamento de Ciência da Computação}
\departmenten{Computer Science Department}

\programpt{Programa Multiinstitucional de Pós-graduação em Ciência da Computação}
\programen{Graduate in Computer Science}

\majorfieldpt{Ciência da Computação}
\majorfielden{Computer Science}

\title{Sistema de apoio à Pós graduação - UFBA}
\date{Outubro/2016}

\author{Victor de Azevedo Nunes}
\adviser{Ivan do Carmo Machado}

\begin{document}

\frontmatter
\frontpage
\presentationpage

\begin{dedicatory}
Eu dedico esta dissertação...
%I dedicate this dissertation to my family, girlfriend, friends and
%professors who gave me all necessary support to get here.
\end{dedicatory}

\acknowledgements
Meus agradecimentos...

\begin{epigraph}[]{Edward V Berard}
Walking on water and developing software from a specification are easy if both are frozen
\end{epigraph}

\resumo
% Escreva seu resumo no arquivo resumo.tex
Este trabalho de conclusão de curso utiliza elementos da engenharia de software para propor um software como serviço a fim de otimizar o processo de construção e manutenção dos softwares. Assim, o objetivo deste SaaS é gerenciar as permissões de acesso de sistemas cliente, provendo o reuso de software e padronizando as soluções. O sistema proposto neste trabalho encontra-se implementado e disponível no Github, e traz fundamentos sobre a arquitetura e tecnologias utilizadas, bem como avaliações sobre possibilidades de utilização do mesmo em diversos ambientes, como Web e mobile.

\begin{keywords}
Software; Reuso; SaaS; Web; PHP
\end{keywords}

\abstract
% Write your abstract in a file called abstract.tex
My abstract...

\begin{keywords}
key words...
\end{keywords}

% Summary (tables of contents)
\tableofcontents

% List of figures
\listoffigures

% List of tables
\listoftables

% List of acronyms
% Acronyms manual: http://linorg.usp.br/CTAN/macros/latex/contrib/acronym/acronym.pdf
\listofacronyms
\begin{acronym}[ACRONYM] 
% Change the word ACRONYM above to change the acronym column width.
% The column width is equals to the width of the word that you put.
% Read the manual about acronym package for more examples:
%   http://linorg.usp.br/CTAN/macros/latex/contrib/acronym/acronym.pdf

\acro{SPA}{Single Page Application}
\acro{JSON}{Javascript Object Notation}
\acro{PHP}{PHP: Hypertext Preprocessor}
\acro{SaaS}{Software as a Service}
\acro{ERP}{Enterprise Resource Planning}
\acro{QoS}{Quality of Service}
\acro{UML}{Unified Modeling Language}
\acro{MVC}{Model-View-Controller}
\acro{Ajax}{Asynchronous Javascript and XML}
\acro{HTML}{HyperText Markup Language}
\acro{CSS}{Cascading Style Sheets}
\acro{API}{Application Programming Interface}
\acro{DOM}{Document Object Model}
\acro{BPMN}{Business Process Model and Notation}
\acro{REST}{Representational State Transfer}

\end{acronym}

% List of listings
%\lstlistoflistings

\mainmatter

\chapter{Introdução}

\section{Motivação}

Organizar os procedimentos de um processo sempre nos traz vantagens. Apesar de no processo de implantação de um sistema, o mesmo burocratizar o processo, com o tempo temos o retorno da dedicação para a inserção dos dados. Com um certo volume de dados, é possível estruturar informações que num processo manual são difíceis de serem enxergadas. Assim, é possível depender menos das pessoas que organizam o processo, pois o legado de informações não estará mais somente na mente de alguns, mas sim documentado nos dados do sistema.

Além de colaborar na organização, também haverá uma grande colaboração no tempo gasto na gestão. Lidar com muitos papéis e confiar na mente humana para guardar informações, não é uma alternativa muito segura devido ao fato que as pessoas sempre estão sujeitas a sair do processo e levar contigo a experiência obtida. Experiência essa que faz com que os procedimentos sejam executados de forma mais eficiente. Entretanto, com um sistema inteligente, é possível auxiliar e tornar mais ágil a execução das tarefas.


\section{Problema}


De acordo com funcionários ligados ao o setor de pós graduação da UFBA, entrevistados a fim de um maior entendimento do cenário, apesar das semelhanças estruturais, a pós graduação gerida de forma diferencia da graduação. FULANO afirma que devido ao fato de não ter a mesma visibilidade, não tem acesso aos mesmos recursos de gestão acadêmica da graduação. O professores não executam somente atividades dentro da sala de aula, também tem diversas outras ocupações no setor. E muitos procedimentos realizados extra classe ainda se encontram sendo realizados de forma manual, estando mais vulnerável ao erro ou até mesmo à violação do processo. Também ocorre um grande desperdício de tempo pelos professores e gestores da área, devido ao diversos processos ainda realizados de forma manual, sem a devida documentação. Segundo FULANO, também entrevistado, esse tempo perdido implica numa redução da eficiência na sala de aula, pois o professor acaba por ter menos tempo disponível para o planejamento das atividades, o que gera impactos negativos aos alunos.


\section{Objetivos} %<o que deve ser feito/entregue>


Devido aos muitos processos sendo resolvidos de forma manual, propõe-se com solução um sistema moderno, arquitetado para ter funcionamento na web e com um módulo mobile, a fim de fornecer informações de forma rápida e eficiente para os professores através de notificações, já que o acesso à internet móvel é comum entre os possíveis usuários do sistema em questão.
O principal requisito para o sistema seria dispor recursos para reduzir o tempo desperdiçado pelos professores durante as atividades extra classe.


\section{Metodologia} %<como será feito | como resolver o problema apontado inicialmente>


%<analise de literatura | design | implementação | validação>
Baseando-se nas tecnologias gratuitas em alta no cenário atual do desenvolvimento web, dispomos de algumas opções eficientes para a implementação da solução. Dentre as possibilidades, considerando a facilidade para futura manutenção e continuidade do projeto, tende-se a optar por uma tecnologia popular. Como linguagem de programação, adota-se o PHP. A escolha é fundamentada de acordo com a pesquisa da RedMonk de 2015, que evidencia o uso das linguagens de programação de acordo com as discussões no StackOverflow e repositórios no GitHub. É possível constatar a popularidade do PHP no cenário atual com o gráfico da pesquisa citada, na qual o PHP é apresentado na terceira colocação, apenas atrás do lider JavaScript e do segundo colocado, o Java.

\begin{figure}
	\label{fig:graficoRedmonk}
	\includegraphics[width=1\textwidth]{img/grafico_redmonk}
	\caption{Ranking das liguagens de programação no Stack Overflow e Github}
\end{figure}


Ainda assim, para compor a interface do dado projeto, também ocorrerá o uso do líder JavaScript de forma intensa, provendo o elo com o as informações gerenciadas pelo PHP.


Entretanto, não seria inteligente desenvolver um sistema completo sem o auxílio de um framework. Dentre os frameworks disponíveis para PHP, hoje o destaque está com o Laravel, que se encontra no topo dentre os mais utilizados no momento. 


A WebHostFace, uma empresa de hospedagem, compilou várias estatísticas para criar um infográfico mostrando os frameworks PHP mais populares de 2015. Utilizando informações sobre os próprios clientes, o Google Trends, estatísticas de repositórios do GitHub e a pesquisa do SitePoint “Best PHP Frameworks 2015”, a WebHostFace elaborou o seguinte infográfico: 

\begin{figure}
	\label{fig:graficoWebhostface}
	\includegraphics[width=1\textwidth]{img/infografico_webhostface}
	\caption{Infográfico da WebhostFace, exibindo a popularidade dos Frameworks PHP em 2015}
\end{figure}

Assim, tem-se a evidência que o Laravel em 2015 teve a maior popularidade em projetos pessoais e tem a maior comunidade entre os concorrentes, o que o torna uma boa escolha para a escrita de um software que será continuado por terceiros.


Para elaborar os recursos de interface e integrar ao back-end PHP do sistema, será adotado o já conhecido AngularJS, ferramenta sólida e conhecida no aspecto em questão. 


Dados coletados via Google Trends, que propõe comparações entre termos pesquisados, revela a popularidade do AngularJs diante de alguns dos principais concorrentes. O gráfico abaixo evidencia o cenário.


%Como mostra a Figura \ref{fig:graficoGoogleTrendsFerramentasFront}. 
\begin{figure}
	\label{fig:graficoGoogleTrendsFerramentasFront}
	\includegraphics[width=1\textwidth]{img/grafico_ferramentas_front}
	\caption{Gráfico do Google Trends exibindo as pesquisas por ferramentas front-end}
\end{figure}


Junto ao Angular JS, será utilizada a agradável tendência de interface do Material Design da Google, que propõe layouts limpos e otimizados já conhecidos pelos usuários de smartphones Android. 


Para a elaboração da plataforma mobile do projeto, será utilizado o Ionic Framework, muito difundido e bastante pesquisado na área, o que fica evidenciado com o gráfico de pesquisbaixo, coletado via Google Trends buscando por frameworks de desenvolvimento híbrido mobile.


\begin{figure}
	\label{fig:graficoGoogleTrendsFerramentasHibridasMobile}
	\includegraphics[width=1\textwidth]{img/grafico_ferramentas_hibridas_mobile}
	\caption{Gráfico do Google Trends exibindo as pesquisas por Frameworks híbridos mobile}
\end{figure}	

Para layout da interface mobile, também será aplicado a tendência do Material Design, a fim de propor uma harmonia entre o módulo web e mobile para os usuários


\section{Resultados Esperados}


Como fruto de um sistema para pós-graduação da UFBA, espera-se que os professores tenham mais recursos para integrar as atividades e também prover melhores condições para acompanhamento da vida acadêmica dos alunos em questão. Também, que os novos colaboradores que entrarem no processo tenham facilidade de compreender o fluxo do setor ao navegar pelo sistema proposto.


\section{Fora de Escopo}


Interação com os alunos devido às complicações para realizar a integração com o sistema empregado na UFBA, gerenciado pela XXXXXX, o que causaria uma inviabilidade no projeto devido à necessidade de entrega do produto ser mais forte que o tempo necessário para executar o processo de obtenção de acesso ao sistema legado para realizar a integração.


\section{Estrutura do Trabalho}


<breve resumo sobre os capítulos do TCC>
\chapter{Referencial Teórico}


Projetar o desenvolvimento de um software requer muito planejamento, pois as falhas iniciais podem custar bastante caro ou até mesmo inviabilizar a continuação de um projeto. Assim, a escolha da arquitetura ideal para a aplicabilidade é essencial na concepção de um produto de software. 
De todo o modo, sempre busca-se fazer mais com menos. Diante de tal filosofia, temos neste capítulo, uma breve discussão sobre alguns elementos de projeto e arquitetura de software, a fim de contextualizar este trabalho de conclusão de curso.
O capítulo corrente é composto por quatro seções. A \ref{sec:saas} trata de Software como serviço, discutindo alguns elementos do contexto que são relevantes para o trabalho proposto. A \ref{sec:reuso}  discute sobre a empregabilidade do reuso de software. A \ref{sec:modularizacao} seção, trata sobre aspectos envolvidos na modularização dos softwares. Por fim, a \ref{sec:apps_web} aborda as aplicações web, discutindo sobre aspectos relevantes sobre a aplicação web que compõe este trabalho.


 \section{Software como serviço}\label{sec:saas}


Segundo La e Chun \citep{La2009Systematic}, o princípio da definição de Software como um Serviço (Sofware as a Service - SaaS) é um serviço complementar para aplicações da computação em nuvem (cloud computing). No entanto, as áreas não se confundem. SaaS deve ser entendido como um mecanismo de suporte às soluções existentes na cloud. Os SaaS existem justamente para maximizar o reuso de serviços repetidos e não centrais em uma aplicação remota.


Como vantagens, diversos fatores podem ser favoráveis para a adoção de um SaaS, como custo e manutenção dentre outros fatores aplicáveis a soluções específicas. Lechesaet al. \citep{LechesaSS11}, quantificam índices sobre os fatores determinantes para adoção ou não de um SaaS voltado para ERP na África do Sul. Os principais fatores determinantes para adoção desse mecanismo de software são sua fluidez quanto à rede e a segurança. Esses fatores estão presentes na aplicação desenvolvida neste trabalho de conclusão de curso.


Devido ao fato de ter um serviço constantemente na nuvem, fica o questionamento sobre a segurança da informação manipulada. Sabe-se que a vulnerabilidade na Web não é restrita ao SaaS, atingindo diversos âmbitos. Rai et al. \cite{journals/corr/RaiSM13} consideram que o avanço da computação em nuvem não é um problema apenas para os serviços Web do ponto de vista da segurança, pois muitos trabalhos na literatura mostram a área como mais um ponto de vulnerabilidade para diversos setores, a exemplo de infraestrutura. No mesmo artigo mencionado de Rai et al. \cite{journals/corr/RaiSM13}, também realizaram-se estudos exploratórios junto a empresas usuárias de serviços em computação em nuvem e consideram que a perspectiva de SaaS também pode fortalecer a segurança nas aplicações de cloud computing, pois o software de autenticação compartilhado por várias aplicações em nuvem, oferece uma melhor padronização e consequente facilidade de prevenção a erros de vulnerabilidade específicas de cada módulo da pesquisa. Esse ponto de vista é fundamenta para a compreensão de SaaS.


A arquitetura de armazenamento de dados de um Saas pode variar de acordo com a necessidade do contexto. Huixin \cite{7586486} descreve possíveis modelagens nesse sentido. Essa abordagem pode ser com um banco de dados único, fazendo com que diferentes clientes compartilhem o mesmo banco, diferindo os dados através de controle de usuário, ou isolando os diferentes clientes através de bancos de dados exclusivos para cada um. Esse fator também pode ser combinado com a arquitetura da aplicação, caso ofereça aplicação única para todos os clientes ou aplicação compartilhada. Diante das possíveis abordagens, a modelagem de dados do software pode ser decidida pela regra de negócio. Este trabalho optou por aplicação única e banco de dados compartilhado.

Devido ao conceito particular de obtenção de software, tanto pela visão do cliente como pela visão do vendedor, é necessário compreender os diversos aspectos que podem ser relevantes ao orçar um Saas. O recente trabalho de T. Kaur et al. \citep{6949281} orienta um modelo para compor o custo de um Saas. O custo total seria composto pelos fatores que dão suporte ao funcionamento do software. Tais fatores incluem infra-estrutura, configurabilidade, customização, parâmetros de QoS (Quality of service) como escalabilidade, disponibilidade, usabilidade, pontualidade e desempenho da resposta, portabilidade, custo total de propriedade e retorno do investimento. Esses fatores caracterizam o custo de forma eficaz, possibilitando ao fornecedor, prover um Serviço de acordo com a exigência do consumidor em vários pacotes de serviços.


O conceito de software como serviço se aplica neste trabalho de conclusão de curso, pois o mesmo estará disponível na web com alta disponibilidade adotando as características apresentadas para qualquer pessoa que desejar usá-lo. Assim, basta se cadastrar, configurá-lo e usar como um serviço, sem preocupação com a manutenção do mesmo.


\section{Reuso de software}\label{sec:reuso} %CRUISE BOOK CAPITULO 2


De acordo com o livro Software pratical reuse \citep{Ezran2002}, o reuso de software é a utilização de qualquer informação que um desenvolvedor pode necessitar no processo de criação de software. O livro de Basili e Rombach \citep{Basili1991} define reutilização de software como o uso de tudo o que está associado a um projeto de conhecimento.
Assim, o objetivo da reutilização de software é reciclar o design, código e outros componentes de um produto de software e assim reduzir o custo, o tempo e melhorar a qualidade do produto.
Segundo Keswani et al. \cite{6783445}, o componente reutilizável de software pode ser qualquer parte de seu desenvolvimento, como um fragmento de código, design, casos de teste, ou até mesmo a especificação de requisitos de uma funcionalidade do software. 

O reuso de software pode ter impacto positivo em diversos aspectos do software, vejamos alguns, conforme apresentados no C.R.U.I.S.E Book \citep{cruise2007} :

\begin{itemize}

\item Qualidade: As correções de erro tornam-se úteis em todos os locais em que ocorreu, padronizando e facilitando a manutenção.

\item Produtividade: O ganho de produtividade é alcançado devido ao menor número de artefatos desenvolvido. Isso resulta em menor esforço de teste e também análise e design, reduzindo custos.

\item Confiabilidade: A utilização de componentes bem testados aumenta a
confiança no software. Além disso, a utilização de um mesmo componente em vários sistemas, aumenta a possibilidade de detecção de erros e reforça a confiança no componente.

\item Redução do Esforço: A reutilização de software proporciona uma redução do tempo de desenvolvimento, o que reduz o tempo necessário para o produto ser disponibilizado no mercado para trazer rentabilidade.

\item Trabalho redundante e tempo de desenvolvimento: Desenvolver um sistema do
zero significa desenvolvimento redundante de muitos componentes, como requisitos, especificações, casos de uso, arquitetura, etc. Isso pode ser evitado quando estes estão disponíveis como componentes reutilizáveis e podem ser compartilhados, resultando em um processo de desenvolvimento otimizado.

\item Documentação: Embora a documentação seja muito importante para a
manutenção de um sistema, muitas vezes é negligenciada. A reutilização de componentes de software reduz a quantidade de documentação a ser escrita, entretanto depende da qualidade do que está escrito. Assim, apenas a estrutura do sistema e os novos artefatos desenvolvidos necessitam ser documentados.

\item Custo de manutenção: Menos defeitos e manutenções são esperados quando tem-se comprovada a qualidade dos componentes utilizados.

\item Tamanho da equipe: É comum haver casos em que a equipe de desenvolvimento sofre sobrecarga. Entretanto, dobrar o tamanho da equipe de desenvolvimento não necessariamente duplica produtividade. Se muitos componentes podem ser reutilizados, é possível desenvolver com equipes menores, levando a melhor comunicação e aumento da produtividade.

\end{itemize}

Apesar dos benefícios da reutilização de software, ela não é suficientemente aproveitada. Existem fatores que influenciam direta ou indiretamente na sua adoção. Esses fatores podem ser de aspecto gerencial, organizacional, econômico, conceitual ou técnico. Veremos a seguir alguns aspectos que podem gerar conflito com a cultura de reuso de software, segundo o C.R.U.I.S.E Book \citep{cruise2007} :
%(Sametinger, 1997). REVER

\begin{itemize}
	
\item Falta de apoio da gestão: Como a reutilização de software gera custos iniciais,
a medida pode não ser amplamente alcançada em uma organização sem o apoio de alto nível de gestão. Os gestores têm de ser informados sobre os custos iniciais e serem convencidos sobre economias futuras.

\item Gerenciamento do Projeto: Gerenciar projetos tradicionais é uma tarefa árdua, principalmente, os que praticam a reutilização de software. Utilizando a técnica em larga escala, tem-se impacto sobre todo o ciclo de vida do software.

\item Estruturas organizacionais inadequadas: As estruturas organizacionais devem
considerar diferentes necessidades que surgem quando a reutilização em larga escala está sendo adotada. Por exemplo, uma equipe particionada pode ser alocada somente para desenvolver, manter e certificar componentes reutilizáveis de software.

\item Incentivos de gestão: É comum a falta de incentivo para deixar os desenvolvedores gastarem tempo elaborando componentes do sistemas. A produtividade é muitas vezes medida apenas no tempo necessário para concluir um projeto. Assim, fazer qualquer trabalho além disso, embora benéfico para a empresa como um todo, diminui o seu sucesso. Mesmo quando os componentes reutilizáveis são utilizados, os benefícios obtidos são uma pequena fração do que poderia ser alcançado caso houvesse reutilização explícita, planejada e organizada.

\item Dificuldade de encontrar software reutilizável: Para reutilizar os componentes, devem existir formas eficientes de busca. Além disso, é importante ter um repositório bem organizado contendo componentes com um eficiente meio de acesso.

\item Não reutilização do software encontrado. O acesso fácil ao software existente
não necessariamente aumentar a reutilização. Os componentes reutilizáveis devem ser cuidadosamente especificados, projetados, implementados e documentados, pois em alguns casos, modificar e adaptar o código  pode ser mais custoso que a programação da funcionalidade necessária a partir do zero.

\item Modificação: É muito difícil encontrar um componente que funcione
exatamente da mesma maneira que queremos. Desta forma, são necessárias modificações e devem existir formas de determinar os seus efeitos sobre o componente.


\end{itemize}


A definição do reuso de software pode ser associada a este trabalho, pois ainda que como um serviço, estaremos adotando uma aplicação única para servir aos mais diversos softwares de uma empresa, por exemplo. Logo, ao adotar o projeto, pode ser interessante aplicar algumas métricas para tomar conhecimento de possíveis vantanjens como as citadas nessa seção, a exemplo de economia de tempo de desenvolvimento, e custo do projeto.


%Outra diretriz importante para a reutilização de software é reduzir o risco na criação de novos softwares. O risco tende a ser bastante reduzido se os componentes que estão sendo reutilizados têm as documentação, interfaces necessárias e devidamente testadas, fatores que contibruem para uma fácil integração.
%De acordo com Keswani et al. \citep{6783445}, para o reuso de software dar retornos apropriados, o processo deve ser sistemático e planejado. Qualquer organização que implemente a reutilização de software deve identificar os melhores métodos e estratégias de reutilização para obter a máxima produtividade. A reutilização de software ajuda a evitar software de engenharia a partir do zero, pois usa módulos de software existentes. A reutilização de software, embora seja uma tarefa difícil, especialmente para softwares antigos sem padrões de projeto, pode melhorar significativamente a produtividade e a qualidade de um produto de software. Embora a reutilização de software não seja um novo campo, ela pode dar grandes retornos em curto período de tempo.


\section{Modularização}\label{sec:modularizacao} %artigo de claudio pagina 222 introdução


%A modularidade vem desempenhando um papel predominante estágios emergentes das disciplinas de arquitetura de software [13]. Engenheiros de software consideram modularidade como princípio base na comparação entre arquiteturas alternativas  e arquitetura degeneração [9]. De fato, os engenheiros de software são incentivados a arquitecturas, baseando-se numa multiplicidade de mecanismos de modularidade disponíveis em: 
%(i) Linguagens de descrição de arquitetura (ADLs), como ACME [8], 
%(ii) catálogos de arquitetônicos [2, 13], e 
%(iii) conhecem bem princípios de alto nível, como interfaces de componentes estreitos, acoplamento arquitectónico reduzido e semelhantes.


Conforme é frisado por Wickramaarachchi e Lai \citep{7062705}, o conceito de modularização na indústria de software tem uma longa história e tem sido utilizado para melhorar o processo de desenvolvimento de software em diferentes estágios. Os principais conceitos por trás da modularização do software foram introduzidos por pesquisadores pioneiros há quarenta anos, com uma notável contribuição feita por Melvin Conway e David Parnas, que tem representação notável na engenharia de software.


Modularizar um software é um bom padrão a ser adotado. Segundo Wickramaarachchi e Lai \citep{7062705}, a modularização é importante na identificação de dependências e reduz as dificuldades diante de uma possível necessidade de grandes alterações. De uma perspectiva da engenharia de software, uma modularização geralmente tem várias vantagens, tais como: tornar a complexidade do software mais gerenciável, facilitar o trabalho paralelo e tornar o software mais maleável para acomodar o futuro incerto que um software pode ter. O objetivo final da modularização do software é aumentar a produtividade ea qualidade do software. Tal conceito encontra-se bastante difundido e estái incorporado em linguagens de programação e ferramentas de software. O presente trabalho favorece ao uso da modularização de um software e até mesmo pode ser considerado um módulo a ser acoplado a qualquer software, mediante a compatibilidade.


Ao realizar a adoção do trabalho proposto, fica bastante evidente a "responsabilidade" do mesmo no escopo do projeto. Assim, é possível tratá-lo como um módulo do projeto que o usa, mesmo que seja consumido como um serviço.


\section{Aplicações web}\label{sec:apps_web}


A popularidade das soluções Web aumentou exponencialmente na última década e todos os dias cresce o número de pessoas usuárias desse tipo de software. E seguindo um padrão próprio, Kumar et al. \citep{7813710} sugerem que para o desenvolvimento web, deve-se manter a prática eficaz de produzir diagramas UML. A abordagem baseada na web oferece uma maneira fácil e eficaz para gerenciar e controlar o processo de desenvolvimento por meio de artefatos de modelagem. Tal abordagem pode ser usada quando há uma exigência de lidar com mudanças muito rápidas e grandes em requisitos de forma muito eficaz em muito menos tempo, gerando assim um menor impacto. 


Para atender à fomentada demanda de aplicativos web, é necessário adotar métodos de desenvolvimentos que sejam ágeis, eficientes e de fácil manutenção. Yu Ping et al. \cite{1372143} propõem o uso do modelo MVC (Model, View e Controller) no desenvolvimento para softwares web. O modelo apresentado tornou-se um padrão popular e divide o software em camadas com propósito definido, tornando-o de mais fácil manutenção.


O Ajax (Asynchronous Javascript and XML) revolucionou a web. Conforme demonstrado por Yuping \citep{6845605}, ao usar a tecnologia Ajax, podemos enriquecer a experiência do usuário em aplicações baseadas em navegador de internet, e fornecer uma variedade de aplicações interativas para atender às necessidade de humanização das aplicações.
Os aplicativos Ajax em execução no navegador se comunicam com um servidor Web de forma assíncrona e atualizam apenas uma parte da página.


No artigo de Tesarik et al. \citep{4664349}, temos que o design de software SPA (Single page application) é uma maneira interessante de criar um software Web numa uma única página. Essa solução de página única sem navegação funciona apenas com base em técnicas dinâmicas e assíncronas, como o AJAX (citado no parágrafo anterior). No entanto, esta abordagem coloca o desenvolvedor antes de alguns desafios substanciais.
Para projetar a interface do usuário que mostra as informações, é importante projetar corretamente a tela para manipular os dados do aplicativo numa única página. O design da página deve ser elaborado para maximizar a decomposição da página em componentes distintos que encapsulam os principais casos de uso.
Também recomenda-se que para elaborar os artefatos visuais, explore os mais ricos recursos na implementação, como HTML5, JavaScript, Ajax, CSS3, e outras tecnologias que se apliquem. Entretanto, desenvolver uma rica interface com o uso de diversas tecnologias/frameworks pode ocasionar um esforço maior para explorar as possibilidades.
Tais componentes da interface de uma aplicação SPA normalmente são alimentados mediante o consumo de uma API Rest via requisições AJAX. Assim, é possível particionar as responsabilidades de processamento do software entre cliente e servidor.


Como é possível denotar no trabalho de Salvadori e Siqueire \citep{7195633}, REST é uma arquitetura muito popular para integração de aplicativos web, que permite compartilhar e reutilizar informações através de sistemas. Aplicações de grande escala baseadas em no REST deve ser implementadas utilizando estratégias e mecanismos para produzir sistemas que sejam fáceis de desenvolver, reutilizar e manter. As interfaces de integração fornecidas por esses sistemas, chamadas de Web API, têm uma influência importante nas características da implementação resultante, pois a sua forma de resposta deve ser casada com a interface que o consome.
 
 
Os elementos comentados nessa seção encontram-se presentes neste trabalho de conclusão de curso. Juntos, montam a estrutura tecnológica necessária aliada à arquitetura adotada. Esses elementos seguem tendências atuais dos softwares web, fazendo com que esse trabalho esteja composto por tecnologias modernas que estão com boa aceitação no mercado.

% %% RiSE Latex Template - version 0.5
%%
%% RiSE's latex template for thesis and dissertations
%% http://risetemplate.sourceforge.net
%%
%% (c) 2012 Yguaratã Cerqueira Cavalcanti (yguarata@gmail.com)
%%          Vinicius Cardoso Garcia (vinicius.garcia@gmail.com)
%%
%% This document was initially based on UFPEThesis template, from Paulo Gustavo
%% S. Fonseca.
%%
%% ACKNOWLEDGEMENTS
%%
%% We would like to thanks the RiSE's researchers community, the 
%% students from Federal University of Pernambuco, and other users that have
%% been contributing to this projects with comments and patches.
%%
%% GENERAL INSTRUCTIONS
%%
%% We strongly recommend you to compile your documents using pdflatex command.
%% It is also recommend use the texlipse plugin for Eclipse to edit your documents.
%%
%% Options for \documentclass command:
%%         * Idiom
%%           pt   - Portguese (default)
%%           en   - English
%%
%%         * Text type
%%           bsc  - B.Sc. Thesis
%%           msc  - M.Sc. Thesis (default)
%%           qual - PHD qualification (not tested yet)
%%           prop - PHD proposal (not tested yet)
%%           phd  - PHD thesis
%%
%%         * Media
%%           scr  - to eletronic version (PDF) / see the users guide
%%
%%         * Pagination
%%           oneside - unique face press
%%           twoside - two faces press
%%
%%		   * Line spacing
%%           singlespacing  - the same as using \linespread{1}
%%           onehalfspacing - the same as using \linespread{1.3}
%%           doublespacing  - the same as using \linespread{1.6}
%%
%% Reference commands. Use the following commands to make references in your
%% text:
%%          \figref  -- for Figure reference
%%          \tabref  -- for Table reference
%%          \eqnref  -- for equation reference
%%          \chapref -- for chapter reference
%%          \secref  -- for section reference
%%          \appref  -- for appendix reference
%%          \axiref  -- for axiom reference
%%          \conjref -- for conjecture reference
%%          \defref  -- for definition reference
%%          \lemref  -- for lemma reference
%%          \theoref -- for theorem reference
%%          \corref  -- for corollary reference
%%          \propref -- for proprosition reference
%%          \pgref   -- for page reference
%%
%%          Example: See \chapref{chap:introduction}. It will produce 
%%                   'See Chapter 1', in case of English language.

\documentclass[pt,twoside,onehalfspacing,bsc]{risethesis}

\usepackage[sort,square]{natbib} %package for bibliography
\bibliographystyle{ieeetr} %authordate1

\usepackage[utf8]{inputenc}
\usepackage[brazilian]{babel}
\usepackage[T1]{fontenc}

\usepackage{graphicx}

%% Change the following pdf author attribute name to your name.
\usepackage[linkcolor=blue,citecolor=blue,urlcolor=blue,colorlinks,pdfpagelabels,pdftitle={Bruno Cabral's Bachelor Thesis},pdfauthor={Bruno Cabral}]{hyperref}

\address{SALVADOR}

\universitypt{Universidade Federal da Bahia}
\universityen{Federal University of Bahia}

\departmentpt{Depertamento de Ciência da Computação}
\departmenten{Computer Science Department}

\programpt{Programa Multiinstitucional de Pós-graduação em Ciência da Computação}
\programen{Graduate in Computer Science}

\majorfieldpt{Ciência da Computação}
\majorfielden{Computer Science}

\title{Sistema de apoio à Pós graduação - UFBA}
\date{Outubro/2016}

\author{Victor de Azevedo Nunes}
\adviser{Ivan do Carmo Machado}

\begin{document}

\frontmatter
\frontpage
\presentationpage

\begin{dedicatory}
Eu dedico esta dissertação...
%I dedicate this dissertation to my family, girlfriend, friends and
%professors who gave me all necessary support to get here.
\end{dedicatory}

\acknowledgements
Meus agradecimentos...

\begin{epigraph}[]{Edward V Berard}
Walking on water and developing software from a specification are easy if both are frozen
\end{epigraph}

\resumo
% Escreva seu resumo no arquivo resumo.tex
\input{resumo}

\abstract
% Write your abstract in a file called abstract.tex
\input{abstract}

% Summary (tables of contents)
\tableofcontents

% List of figures
\listoffigures

% List of tables
\listoftables

% List of acronyms
% Acronyms manual: http://linorg.usp.br/CTAN/macros/latex/contrib/acronym/acronym.pdf
\listofacronyms
\input{acronyms}

% List of listings
%\lstlistoflistings

\mainmatter

\include{chapters/intro}
\include{chapters/referencial_teorico}

% \include{chapters/introduction/main}
% \include{chapters/background/main}
% \include{chapters/proposed_solution/main}
% \include{chapters/experiment/main}
% \include{chapters/conclusion/main}

%\bibliographystyle{natbib}
%\addcontentsline{toc}{chapter}{\bibliographytocname}
\bibliography{references}

% Appendix
\clearpage
\addappheadtotoc
\appendix
\appendixpage
% \include{appendix/experiment-instruments}

\end{document}
% %% RiSE Latex Template - version 0.5
%%
%% RiSE's latex template for thesis and dissertations
%% http://risetemplate.sourceforge.net
%%
%% (c) 2012 Yguaratã Cerqueira Cavalcanti (yguarata@gmail.com)
%%          Vinicius Cardoso Garcia (vinicius.garcia@gmail.com)
%%
%% This document was initially based on UFPEThesis template, from Paulo Gustavo
%% S. Fonseca.
%%
%% ACKNOWLEDGEMENTS
%%
%% We would like to thanks the RiSE's researchers community, the 
%% students from Federal University of Pernambuco, and other users that have
%% been contributing to this projects with comments and patches.
%%
%% GENERAL INSTRUCTIONS
%%
%% We strongly recommend you to compile your documents using pdflatex command.
%% It is also recommend use the texlipse plugin for Eclipse to edit your documents.
%%
%% Options for \documentclass command:
%%         * Idiom
%%           pt   - Portguese (default)
%%           en   - English
%%
%%         * Text type
%%           bsc  - B.Sc. Thesis
%%           msc  - M.Sc. Thesis (default)
%%           qual - PHD qualification (not tested yet)
%%           prop - PHD proposal (not tested yet)
%%           phd  - PHD thesis
%%
%%         * Media
%%           scr  - to eletronic version (PDF) / see the users guide
%%
%%         * Pagination
%%           oneside - unique face press
%%           twoside - two faces press
%%
%%		   * Line spacing
%%           singlespacing  - the same as using \linespread{1}
%%           onehalfspacing - the same as using \linespread{1.3}
%%           doublespacing  - the same as using \linespread{1.6}
%%
%% Reference commands. Use the following commands to make references in your
%% text:
%%          \figref  -- for Figure reference
%%          \tabref  -- for Table reference
%%          \eqnref  -- for equation reference
%%          \chapref -- for chapter reference
%%          \secref  -- for section reference
%%          \appref  -- for appendix reference
%%          \axiref  -- for axiom reference
%%          \conjref -- for conjecture reference
%%          \defref  -- for definition reference
%%          \lemref  -- for lemma reference
%%          \theoref -- for theorem reference
%%          \corref  -- for corollary reference
%%          \propref -- for proprosition reference
%%          \pgref   -- for page reference
%%
%%          Example: See \chapref{chap:introduction}. It will produce 
%%                   'See Chapter 1', in case of English language.

\documentclass[pt,twoside,onehalfspacing,bsc]{risethesis}

\usepackage[sort,square]{natbib} %package for bibliography
\bibliographystyle{ieeetr} %authordate1

\usepackage[utf8]{inputenc}
\usepackage[brazilian]{babel}
\usepackage[T1]{fontenc}

\usepackage{graphicx}

%% Change the following pdf author attribute name to your name.
\usepackage[linkcolor=blue,citecolor=blue,urlcolor=blue,colorlinks,pdfpagelabels,pdftitle={Bruno Cabral's Bachelor Thesis},pdfauthor={Bruno Cabral}]{hyperref}

\address{SALVADOR}

\universitypt{Universidade Federal da Bahia}
\universityen{Federal University of Bahia}

\departmentpt{Depertamento de Ciência da Computação}
\departmenten{Computer Science Department}

\programpt{Programa Multiinstitucional de Pós-graduação em Ciência da Computação}
\programen{Graduate in Computer Science}

\majorfieldpt{Ciência da Computação}
\majorfielden{Computer Science}

\title{Sistema de apoio à Pós graduação - UFBA}
\date{Outubro/2016}

\author{Victor de Azevedo Nunes}
\adviser{Ivan do Carmo Machado}

\begin{document}

\frontmatter
\frontpage
\presentationpage

\begin{dedicatory}
Eu dedico esta dissertação...
%I dedicate this dissertation to my family, girlfriend, friends and
%professors who gave me all necessary support to get here.
\end{dedicatory}

\acknowledgements
Meus agradecimentos...

\begin{epigraph}[]{Edward V Berard}
Walking on water and developing software from a specification are easy if both are frozen
\end{epigraph}

\resumo
% Escreva seu resumo no arquivo resumo.tex
\input{resumo}

\abstract
% Write your abstract in a file called abstract.tex
\input{abstract}

% Summary (tables of contents)
\tableofcontents

% List of figures
\listoffigures

% List of tables
\listoftables

% List of acronyms
% Acronyms manual: http://linorg.usp.br/CTAN/macros/latex/contrib/acronym/acronym.pdf
\listofacronyms
\input{acronyms}

% List of listings
%\lstlistoflistings

\mainmatter

\include{chapters/intro}
\include{chapters/referencial_teorico}

% \include{chapters/introduction/main}
% \include{chapters/background/main}
% \include{chapters/proposed_solution/main}
% \include{chapters/experiment/main}
% \include{chapters/conclusion/main}

%\bibliographystyle{natbib}
%\addcontentsline{toc}{chapter}{\bibliographytocname}
\bibliography{references}

% Appendix
\clearpage
\addappheadtotoc
\appendix
\appendixpage
% \include{appendix/experiment-instruments}

\end{document}
% %% RiSE Latex Template - version 0.5
%%
%% RiSE's latex template for thesis and dissertations
%% http://risetemplate.sourceforge.net
%%
%% (c) 2012 Yguaratã Cerqueira Cavalcanti (yguarata@gmail.com)
%%          Vinicius Cardoso Garcia (vinicius.garcia@gmail.com)
%%
%% This document was initially based on UFPEThesis template, from Paulo Gustavo
%% S. Fonseca.
%%
%% ACKNOWLEDGEMENTS
%%
%% We would like to thanks the RiSE's researchers community, the 
%% students from Federal University of Pernambuco, and other users that have
%% been contributing to this projects with comments and patches.
%%
%% GENERAL INSTRUCTIONS
%%
%% We strongly recommend you to compile your documents using pdflatex command.
%% It is also recommend use the texlipse plugin for Eclipse to edit your documents.
%%
%% Options for \documentclass command:
%%         * Idiom
%%           pt   - Portguese (default)
%%           en   - English
%%
%%         * Text type
%%           bsc  - B.Sc. Thesis
%%           msc  - M.Sc. Thesis (default)
%%           qual - PHD qualification (not tested yet)
%%           prop - PHD proposal (not tested yet)
%%           phd  - PHD thesis
%%
%%         * Media
%%           scr  - to eletronic version (PDF) / see the users guide
%%
%%         * Pagination
%%           oneside - unique face press
%%           twoside - two faces press
%%
%%		   * Line spacing
%%           singlespacing  - the same as using \linespread{1}
%%           onehalfspacing - the same as using \linespread{1.3}
%%           doublespacing  - the same as using \linespread{1.6}
%%
%% Reference commands. Use the following commands to make references in your
%% text:
%%          \figref  -- for Figure reference
%%          \tabref  -- for Table reference
%%          \eqnref  -- for equation reference
%%          \chapref -- for chapter reference
%%          \secref  -- for section reference
%%          \appref  -- for appendix reference
%%          \axiref  -- for axiom reference
%%          \conjref -- for conjecture reference
%%          \defref  -- for definition reference
%%          \lemref  -- for lemma reference
%%          \theoref -- for theorem reference
%%          \corref  -- for corollary reference
%%          \propref -- for proprosition reference
%%          \pgref   -- for page reference
%%
%%          Example: See \chapref{chap:introduction}. It will produce 
%%                   'See Chapter 1', in case of English language.

\documentclass[pt,twoside,onehalfspacing,bsc]{risethesis}

\usepackage[sort,square]{natbib} %package for bibliography
\bibliographystyle{ieeetr} %authordate1

\usepackage[utf8]{inputenc}
\usepackage[brazilian]{babel}
\usepackage[T1]{fontenc}

\usepackage{graphicx}

%% Change the following pdf author attribute name to your name.
\usepackage[linkcolor=blue,citecolor=blue,urlcolor=blue,colorlinks,pdfpagelabels,pdftitle={Bruno Cabral's Bachelor Thesis},pdfauthor={Bruno Cabral}]{hyperref}

\address{SALVADOR}

\universitypt{Universidade Federal da Bahia}
\universityen{Federal University of Bahia}

\departmentpt{Depertamento de Ciência da Computação}
\departmenten{Computer Science Department}

\programpt{Programa Multiinstitucional de Pós-graduação em Ciência da Computação}
\programen{Graduate in Computer Science}

\majorfieldpt{Ciência da Computação}
\majorfielden{Computer Science}

\title{Sistema de apoio à Pós graduação - UFBA}
\date{Outubro/2016}

\author{Victor de Azevedo Nunes}
\adviser{Ivan do Carmo Machado}

\begin{document}

\frontmatter
\frontpage
\presentationpage

\begin{dedicatory}
Eu dedico esta dissertação...
%I dedicate this dissertation to my family, girlfriend, friends and
%professors who gave me all necessary support to get here.
\end{dedicatory}

\acknowledgements
Meus agradecimentos...

\begin{epigraph}[]{Edward V Berard}
Walking on water and developing software from a specification are easy if both are frozen
\end{epigraph}

\resumo
% Escreva seu resumo no arquivo resumo.tex
\input{resumo}

\abstract
% Write your abstract in a file called abstract.tex
\input{abstract}

% Summary (tables of contents)
\tableofcontents

% List of figures
\listoffigures

% List of tables
\listoftables

% List of acronyms
% Acronyms manual: http://linorg.usp.br/CTAN/macros/latex/contrib/acronym/acronym.pdf
\listofacronyms
\input{acronyms}

% List of listings
%\lstlistoflistings

\mainmatter

\include{chapters/intro}
\include{chapters/referencial_teorico}

% \include{chapters/introduction/main}
% \include{chapters/background/main}
% \include{chapters/proposed_solution/main}
% \include{chapters/experiment/main}
% \include{chapters/conclusion/main}

%\bibliographystyle{natbib}
%\addcontentsline{toc}{chapter}{\bibliographytocname}
\bibliography{references}

% Appendix
\clearpage
\addappheadtotoc
\appendix
\appendixpage
% \include{appendix/experiment-instruments}

\end{document}
% %% RiSE Latex Template - version 0.5
%%
%% RiSE's latex template for thesis and dissertations
%% http://risetemplate.sourceforge.net
%%
%% (c) 2012 Yguaratã Cerqueira Cavalcanti (yguarata@gmail.com)
%%          Vinicius Cardoso Garcia (vinicius.garcia@gmail.com)
%%
%% This document was initially based on UFPEThesis template, from Paulo Gustavo
%% S. Fonseca.
%%
%% ACKNOWLEDGEMENTS
%%
%% We would like to thanks the RiSE's researchers community, the 
%% students from Federal University of Pernambuco, and other users that have
%% been contributing to this projects with comments and patches.
%%
%% GENERAL INSTRUCTIONS
%%
%% We strongly recommend you to compile your documents using pdflatex command.
%% It is also recommend use the texlipse plugin for Eclipse to edit your documents.
%%
%% Options for \documentclass command:
%%         * Idiom
%%           pt   - Portguese (default)
%%           en   - English
%%
%%         * Text type
%%           bsc  - B.Sc. Thesis
%%           msc  - M.Sc. Thesis (default)
%%           qual - PHD qualification (not tested yet)
%%           prop - PHD proposal (not tested yet)
%%           phd  - PHD thesis
%%
%%         * Media
%%           scr  - to eletronic version (PDF) / see the users guide
%%
%%         * Pagination
%%           oneside - unique face press
%%           twoside - two faces press
%%
%%		   * Line spacing
%%           singlespacing  - the same as using \linespread{1}
%%           onehalfspacing - the same as using \linespread{1.3}
%%           doublespacing  - the same as using \linespread{1.6}
%%
%% Reference commands. Use the following commands to make references in your
%% text:
%%          \figref  -- for Figure reference
%%          \tabref  -- for Table reference
%%          \eqnref  -- for equation reference
%%          \chapref -- for chapter reference
%%          \secref  -- for section reference
%%          \appref  -- for appendix reference
%%          \axiref  -- for axiom reference
%%          \conjref -- for conjecture reference
%%          \defref  -- for definition reference
%%          \lemref  -- for lemma reference
%%          \theoref -- for theorem reference
%%          \corref  -- for corollary reference
%%          \propref -- for proprosition reference
%%          \pgref   -- for page reference
%%
%%          Example: See \chapref{chap:introduction}. It will produce 
%%                   'See Chapter 1', in case of English language.

\documentclass[pt,twoside,onehalfspacing,bsc]{risethesis}

\usepackage[sort,square]{natbib} %package for bibliography
\bibliographystyle{ieeetr} %authordate1

\usepackage[utf8]{inputenc}
\usepackage[brazilian]{babel}
\usepackage[T1]{fontenc}

\usepackage{graphicx}

%% Change the following pdf author attribute name to your name.
\usepackage[linkcolor=blue,citecolor=blue,urlcolor=blue,colorlinks,pdfpagelabels,pdftitle={Bruno Cabral's Bachelor Thesis},pdfauthor={Bruno Cabral}]{hyperref}

\address{SALVADOR}

\universitypt{Universidade Federal da Bahia}
\universityen{Federal University of Bahia}

\departmentpt{Depertamento de Ciência da Computação}
\departmenten{Computer Science Department}

\programpt{Programa Multiinstitucional de Pós-graduação em Ciência da Computação}
\programen{Graduate in Computer Science}

\majorfieldpt{Ciência da Computação}
\majorfielden{Computer Science}

\title{Sistema de apoio à Pós graduação - UFBA}
\date{Outubro/2016}

\author{Victor de Azevedo Nunes}
\adviser{Ivan do Carmo Machado}

\begin{document}

\frontmatter
\frontpage
\presentationpage

\begin{dedicatory}
Eu dedico esta dissertação...
%I dedicate this dissertation to my family, girlfriend, friends and
%professors who gave me all necessary support to get here.
\end{dedicatory}

\acknowledgements
Meus agradecimentos...

\begin{epigraph}[]{Edward V Berard}
Walking on water and developing software from a specification are easy if both are frozen
\end{epigraph}

\resumo
% Escreva seu resumo no arquivo resumo.tex
\input{resumo}

\abstract
% Write your abstract in a file called abstract.tex
\input{abstract}

% Summary (tables of contents)
\tableofcontents

% List of figures
\listoffigures

% List of tables
\listoftables

% List of acronyms
% Acronyms manual: http://linorg.usp.br/CTAN/macros/latex/contrib/acronym/acronym.pdf
\listofacronyms
\input{acronyms}

% List of listings
%\lstlistoflistings

\mainmatter

\include{chapters/intro}
\include{chapters/referencial_teorico}

% \include{chapters/introduction/main}
% \include{chapters/background/main}
% \include{chapters/proposed_solution/main}
% \include{chapters/experiment/main}
% \include{chapters/conclusion/main}

%\bibliographystyle{natbib}
%\addcontentsline{toc}{chapter}{\bibliographytocname}
\bibliography{references}

% Appendix
\clearpage
\addappheadtotoc
\appendix
\appendixpage
% \include{appendix/experiment-instruments}

\end{document}
% %% RiSE Latex Template - version 0.5
%%
%% RiSE's latex template for thesis and dissertations
%% http://risetemplate.sourceforge.net
%%
%% (c) 2012 Yguaratã Cerqueira Cavalcanti (yguarata@gmail.com)
%%          Vinicius Cardoso Garcia (vinicius.garcia@gmail.com)
%%
%% This document was initially based on UFPEThesis template, from Paulo Gustavo
%% S. Fonseca.
%%
%% ACKNOWLEDGEMENTS
%%
%% We would like to thanks the RiSE's researchers community, the 
%% students from Federal University of Pernambuco, and other users that have
%% been contributing to this projects with comments and patches.
%%
%% GENERAL INSTRUCTIONS
%%
%% We strongly recommend you to compile your documents using pdflatex command.
%% It is also recommend use the texlipse plugin for Eclipse to edit your documents.
%%
%% Options for \documentclass command:
%%         * Idiom
%%           pt   - Portguese (default)
%%           en   - English
%%
%%         * Text type
%%           bsc  - B.Sc. Thesis
%%           msc  - M.Sc. Thesis (default)
%%           qual - PHD qualification (not tested yet)
%%           prop - PHD proposal (not tested yet)
%%           phd  - PHD thesis
%%
%%         * Media
%%           scr  - to eletronic version (PDF) / see the users guide
%%
%%         * Pagination
%%           oneside - unique face press
%%           twoside - two faces press
%%
%%		   * Line spacing
%%           singlespacing  - the same as using \linespread{1}
%%           onehalfspacing - the same as using \linespread{1.3}
%%           doublespacing  - the same as using \linespread{1.6}
%%
%% Reference commands. Use the following commands to make references in your
%% text:
%%          \figref  -- for Figure reference
%%          \tabref  -- for Table reference
%%          \eqnref  -- for equation reference
%%          \chapref -- for chapter reference
%%          \secref  -- for section reference
%%          \appref  -- for appendix reference
%%          \axiref  -- for axiom reference
%%          \conjref -- for conjecture reference
%%          \defref  -- for definition reference
%%          \lemref  -- for lemma reference
%%          \theoref -- for theorem reference
%%          \corref  -- for corollary reference
%%          \propref -- for proprosition reference
%%          \pgref   -- for page reference
%%
%%          Example: See \chapref{chap:introduction}. It will produce 
%%                   'See Chapter 1', in case of English language.

\documentclass[pt,twoside,onehalfspacing,bsc]{risethesis}

\usepackage[sort,square]{natbib} %package for bibliography
\bibliographystyle{ieeetr} %authordate1

\usepackage[utf8]{inputenc}
\usepackage[brazilian]{babel}
\usepackage[T1]{fontenc}

\usepackage{graphicx}

%% Change the following pdf author attribute name to your name.
\usepackage[linkcolor=blue,citecolor=blue,urlcolor=blue,colorlinks,pdfpagelabels,pdftitle={Bruno Cabral's Bachelor Thesis},pdfauthor={Bruno Cabral}]{hyperref}

\address{SALVADOR}

\universitypt{Universidade Federal da Bahia}
\universityen{Federal University of Bahia}

\departmentpt{Depertamento de Ciência da Computação}
\departmenten{Computer Science Department}

\programpt{Programa Multiinstitucional de Pós-graduação em Ciência da Computação}
\programen{Graduate in Computer Science}

\majorfieldpt{Ciência da Computação}
\majorfielden{Computer Science}

\title{Sistema de apoio à Pós graduação - UFBA}
\date{Outubro/2016}

\author{Victor de Azevedo Nunes}
\adviser{Ivan do Carmo Machado}

\begin{document}

\frontmatter
\frontpage
\presentationpage

\begin{dedicatory}
Eu dedico esta dissertação...
%I dedicate this dissertation to my family, girlfriend, friends and
%professors who gave me all necessary support to get here.
\end{dedicatory}

\acknowledgements
Meus agradecimentos...

\begin{epigraph}[]{Edward V Berard}
Walking on water and developing software from a specification are easy if both are frozen
\end{epigraph}

\resumo
% Escreva seu resumo no arquivo resumo.tex
\input{resumo}

\abstract
% Write your abstract in a file called abstract.tex
\input{abstract}

% Summary (tables of contents)
\tableofcontents

% List of figures
\listoffigures

% List of tables
\listoftables

% List of acronyms
% Acronyms manual: http://linorg.usp.br/CTAN/macros/latex/contrib/acronym/acronym.pdf
\listofacronyms
\input{acronyms}

% List of listings
%\lstlistoflistings

\mainmatter

\include{chapters/intro}
\include{chapters/referencial_teorico}

% \include{chapters/introduction/main}
% \include{chapters/background/main}
% \include{chapters/proposed_solution/main}
% \include{chapters/experiment/main}
% \include{chapters/conclusion/main}

%\bibliographystyle{natbib}
%\addcontentsline{toc}{chapter}{\bibliographytocname}
\bibliography{references}

% Appendix
\clearpage
\addappheadtotoc
\appendix
\appendixpage
% \include{appendix/experiment-instruments}

\end{document}

%\bibliographystyle{natbib}
%\addcontentsline{toc}{chapter}{\bibliographytocname}
\bibliography{references}

% Appendix
\clearpage
\addappheadtotoc
\appendix
\appendixpage
% \include{appendix/experiment-instruments}

\end{document}
% %% RiSE Latex Template - version 0.5
%%
%% RiSE's latex template for thesis and dissertations
%% http://risetemplate.sourceforge.net
%%
%% (c) 2012 Yguaratã Cerqueira Cavalcanti (yguarata@gmail.com)
%%          Vinicius Cardoso Garcia (vinicius.garcia@gmail.com)
%%
%% This document was initially based on UFPEThesis template, from Paulo Gustavo
%% S. Fonseca.
%%
%% ACKNOWLEDGEMENTS
%%
%% We would like to thanks the RiSE's researchers community, the 
%% students from Federal University of Pernambuco, and other users that have
%% been contributing to this projects with comments and patches.
%%
%% GENERAL INSTRUCTIONS
%%
%% We strongly recommend you to compile your documents using pdflatex command.
%% It is also recommend use the texlipse plugin for Eclipse to edit your documents.
%%
%% Options for \documentclass command:
%%         * Idiom
%%           pt   - Portguese (default)
%%           en   - English
%%
%%         * Text type
%%           bsc  - B.Sc. Thesis
%%           msc  - M.Sc. Thesis (default)
%%           qual - PHD qualification (not tested yet)
%%           prop - PHD proposal (not tested yet)
%%           phd  - PHD thesis
%%
%%         * Media
%%           scr  - to eletronic version (PDF) / see the users guide
%%
%%         * Pagination
%%           oneside - unique face press
%%           twoside - two faces press
%%
%%		   * Line spacing
%%           singlespacing  - the same as using \linespread{1}
%%           onehalfspacing - the same as using \linespread{1.3}
%%           doublespacing  - the same as using \linespread{1.6}
%%
%% Reference commands. Use the following commands to make references in your
%% text:
%%          \figref  -- for Figure reference
%%          \tabref  -- for Table reference
%%          \eqnref  -- for equation reference
%%          \chapref -- for chapter reference
%%          \secref  -- for section reference
%%          \appref  -- for appendix reference
%%          \axiref  -- for axiom reference
%%          \conjref -- for conjecture reference
%%          \defref  -- for definition reference
%%          \lemref  -- for lemma reference
%%          \theoref -- for theorem reference
%%          \corref  -- for corollary reference
%%          \propref -- for proprosition reference
%%          \pgref   -- for page reference
%%
%%          Example: See \chapref{chap:introduction}. It will produce 
%%                   'See Chapter 1', in case of English language.

\documentclass[pt,twoside,onehalfspacing,bsc]{risethesis}

\usepackage[sort,square]{natbib} %package for bibliography
\bibliographystyle{ieeetr} %authordate1

\usepackage[utf8]{inputenc}
\usepackage[brazilian]{babel}
\usepackage[T1]{fontenc}

\usepackage{graphicx}

%% Change the following pdf author attribute name to your name.
\usepackage[linkcolor=blue,citecolor=blue,urlcolor=blue,colorlinks,pdfpagelabels,pdftitle={Bruno Cabral's Bachelor Thesis},pdfauthor={Bruno Cabral}]{hyperref}

\address{SALVADOR}

\universitypt{Universidade Federal da Bahia}
\universityen{Federal University of Bahia}

\departmentpt{Depertamento de Ciência da Computação}
\departmenten{Computer Science Department}

\programpt{Programa Multiinstitucional de Pós-graduação em Ciência da Computação}
\programen{Graduate in Computer Science}

\majorfieldpt{Ciência da Computação}
\majorfielden{Computer Science}

\title{Sistema de apoio à Pós graduação - UFBA}
\date{Outubro/2016}

\author{Victor de Azevedo Nunes}
\adviser{Ivan do Carmo Machado}

\begin{document}

\frontmatter
\frontpage
\presentationpage

\begin{dedicatory}
Eu dedico esta dissertação...
%I dedicate this dissertation to my family, girlfriend, friends and
%professors who gave me all necessary support to get here.
\end{dedicatory}

\acknowledgements
Meus agradecimentos...

\begin{epigraph}[]{Edward V Berard}
Walking on water and developing software from a specification are easy if both are frozen
\end{epigraph}

\resumo
% Escreva seu resumo no arquivo resumo.tex
Este trabalho de conclusão de curso utiliza elementos da engenharia de software para propor um software como serviço a fim de otimizar o processo de construção e manutenção dos softwares. Assim, o objetivo deste SaaS é gerenciar as permissões de acesso de sistemas cliente, provendo o reuso de software e padronizando as soluções. O sistema proposto neste trabalho encontra-se implementado e disponível no Github, e traz fundamentos sobre a arquitetura e tecnologias utilizadas, bem como avaliações sobre possibilidades de utilização do mesmo em diversos ambientes, como Web e mobile.

\begin{keywords}
Software; Reuso; SaaS; Web; PHP
\end{keywords}

\abstract
% Write your abstract in a file called abstract.tex
My abstract...

\begin{keywords}
key words...
\end{keywords}

% Summary (tables of contents)
\tableofcontents

% List of figures
\listoffigures

% List of tables
\listoftables

% List of acronyms
% Acronyms manual: http://linorg.usp.br/CTAN/macros/latex/contrib/acronym/acronym.pdf
\listofacronyms
\begin{acronym}[ACRONYM] 
% Change the word ACRONYM above to change the acronym column width.
% The column width is equals to the width of the word that you put.
% Read the manual about acronym package for more examples:
%   http://linorg.usp.br/CTAN/macros/latex/contrib/acronym/acronym.pdf

\acro{SPA}{Single Page Application}
\acro{JSON}{Javascript Object Notation}
\acro{PHP}{PHP: Hypertext Preprocessor}
\acro{SaaS}{Software as a Service}
\acro{ERP}{Enterprise Resource Planning}
\acro{QoS}{Quality of Service}
\acro{UML}{Unified Modeling Language}
\acro{MVC}{Model-View-Controller}
\acro{Ajax}{Asynchronous Javascript and XML}
\acro{HTML}{HyperText Markup Language}
\acro{CSS}{Cascading Style Sheets}
\acro{API}{Application Programming Interface}
\acro{DOM}{Document Object Model}
\acro{BPMN}{Business Process Model and Notation}
\acro{REST}{Representational State Transfer}

\end{acronym}

% List of listings
%\lstlistoflistings

\mainmatter

\chapter{Introdução}

\section{Motivação}

Organizar os procedimentos de um processo sempre nos traz vantagens. Apesar de no processo de implantação de um sistema, o mesmo burocratizar o processo, com o tempo temos o retorno da dedicação para a inserção dos dados. Com um certo volume de dados, é possível estruturar informações que num processo manual são difíceis de serem enxergadas. Assim, é possível depender menos das pessoas que organizam o processo, pois o legado de informações não estará mais somente na mente de alguns, mas sim documentado nos dados do sistema.

Além de colaborar na organização, também haverá uma grande colaboração no tempo gasto na gestão. Lidar com muitos papéis e confiar na mente humana para guardar informações, não é uma alternativa muito segura devido ao fato que as pessoas sempre estão sujeitas a sair do processo e levar contigo a experiência obtida. Experiência essa que faz com que os procedimentos sejam executados de forma mais eficiente. Entretanto, com um sistema inteligente, é possível auxiliar e tornar mais ágil a execução das tarefas.


\section{Problema}


De acordo com funcionários ligados ao o setor de pós graduação da UFBA, entrevistados a fim de um maior entendimento do cenário, apesar das semelhanças estruturais, a pós graduação gerida de forma diferencia da graduação. FULANO afirma que devido ao fato de não ter a mesma visibilidade, não tem acesso aos mesmos recursos de gestão acadêmica da graduação. O professores não executam somente atividades dentro da sala de aula, também tem diversas outras ocupações no setor. E muitos procedimentos realizados extra classe ainda se encontram sendo realizados de forma manual, estando mais vulnerável ao erro ou até mesmo à violação do processo. Também ocorre um grande desperdício de tempo pelos professores e gestores da área, devido ao diversos processos ainda realizados de forma manual, sem a devida documentação. Segundo FULANO, também entrevistado, esse tempo perdido implica numa redução da eficiência na sala de aula, pois o professor acaba por ter menos tempo disponível para o planejamento das atividades, o que gera impactos negativos aos alunos.


\section{Objetivos} %<o que deve ser feito/entregue>


Devido aos muitos processos sendo resolvidos de forma manual, propõe-se com solução um sistema moderno, arquitetado para ter funcionamento na web e com um módulo mobile, a fim de fornecer informações de forma rápida e eficiente para os professores através de notificações, já que o acesso à internet móvel é comum entre os possíveis usuários do sistema em questão.
O principal requisito para o sistema seria dispor recursos para reduzir o tempo desperdiçado pelos professores durante as atividades extra classe.


\section{Metodologia} %<como será feito | como resolver o problema apontado inicialmente>


%<analise de literatura | design | implementação | validação>
Baseando-se nas tecnologias gratuitas em alta no cenário atual do desenvolvimento web, dispomos de algumas opções eficientes para a implementação da solução. Dentre as possibilidades, considerando a facilidade para futura manutenção e continuidade do projeto, tende-se a optar por uma tecnologia popular. Como linguagem de programação, adota-se o PHP. A escolha é fundamentada de acordo com a pesquisa da RedMonk de 2015, que evidencia o uso das linguagens de programação de acordo com as discussões no StackOverflow e repositórios no GitHub. É possível constatar a popularidade do PHP no cenário atual com o gráfico da pesquisa citada, na qual o PHP é apresentado na terceira colocação, apenas atrás do lider JavaScript e do segundo colocado, o Java.

\begin{figure}
	\label{fig:graficoRedmonk}
	\includegraphics[width=1\textwidth]{img/grafico_redmonk}
	\caption{Ranking das liguagens de programação no Stack Overflow e Github}
\end{figure}


Ainda assim, para compor a interface do dado projeto, também ocorrerá o uso do líder JavaScript de forma intensa, provendo o elo com o as informações gerenciadas pelo PHP.


Entretanto, não seria inteligente desenvolver um sistema completo sem o auxílio de um framework. Dentre os frameworks disponíveis para PHP, hoje o destaque está com o Laravel, que se encontra no topo dentre os mais utilizados no momento. 


A WebHostFace, uma empresa de hospedagem, compilou várias estatísticas para criar um infográfico mostrando os frameworks PHP mais populares de 2015. Utilizando informações sobre os próprios clientes, o Google Trends, estatísticas de repositórios do GitHub e a pesquisa do SitePoint “Best PHP Frameworks 2015”, a WebHostFace elaborou o seguinte infográfico: 

\begin{figure}
	\label{fig:graficoWebhostface}
	\includegraphics[width=1\textwidth]{img/infografico_webhostface}
	\caption{Infográfico da WebhostFace, exibindo a popularidade dos Frameworks PHP em 2015}
\end{figure}

Assim, tem-se a evidência que o Laravel em 2015 teve a maior popularidade em projetos pessoais e tem a maior comunidade entre os concorrentes, o que o torna uma boa escolha para a escrita de um software que será continuado por terceiros.


Para elaborar os recursos de interface e integrar ao back-end PHP do sistema, será adotado o já conhecido AngularJS, ferramenta sólida e conhecida no aspecto em questão. 


Dados coletados via Google Trends, que propõe comparações entre termos pesquisados, revela a popularidade do AngularJs diante de alguns dos principais concorrentes. O gráfico abaixo evidencia o cenário.


%Como mostra a Figura \ref{fig:graficoGoogleTrendsFerramentasFront}. 
\begin{figure}
	\label{fig:graficoGoogleTrendsFerramentasFront}
	\includegraphics[width=1\textwidth]{img/grafico_ferramentas_front}
	\caption{Gráfico do Google Trends exibindo as pesquisas por ferramentas front-end}
\end{figure}


Junto ao Angular JS, será utilizada a agradável tendência de interface do Material Design da Google, que propõe layouts limpos e otimizados já conhecidos pelos usuários de smartphones Android. 


Para a elaboração da plataforma mobile do projeto, será utilizado o Ionic Framework, muito difundido e bastante pesquisado na área, o que fica evidenciado com o gráfico de pesquisbaixo, coletado via Google Trends buscando por frameworks de desenvolvimento híbrido mobile.


\begin{figure}
	\label{fig:graficoGoogleTrendsFerramentasHibridasMobile}
	\includegraphics[width=1\textwidth]{img/grafico_ferramentas_hibridas_mobile}
	\caption{Gráfico do Google Trends exibindo as pesquisas por Frameworks híbridos mobile}
\end{figure}	

Para layout da interface mobile, também será aplicado a tendência do Material Design, a fim de propor uma harmonia entre o módulo web e mobile para os usuários


\section{Resultados Esperados}


Como fruto de um sistema para pós-graduação da UFBA, espera-se que os professores tenham mais recursos para integrar as atividades e também prover melhores condições para acompanhamento da vida acadêmica dos alunos em questão. Também, que os novos colaboradores que entrarem no processo tenham facilidade de compreender o fluxo do setor ao navegar pelo sistema proposto.


\section{Fora de Escopo}


Interação com os alunos devido às complicações para realizar a integração com o sistema empregado na UFBA, gerenciado pela XXXXXX, o que causaria uma inviabilidade no projeto devido à necessidade de entrega do produto ser mais forte que o tempo necessário para executar o processo de obtenção de acesso ao sistema legado para realizar a integração.


\section{Estrutura do Trabalho}


<breve resumo sobre os capítulos do TCC>
\chapter{Referencial Teórico}


Projetar o desenvolvimento de um software requer muito planejamento, pois as falhas iniciais podem custar bastante caro ou até mesmo inviabilizar a continuação de um projeto. Assim, a escolha da arquitetura ideal para a aplicabilidade é essencial na concepção de um produto de software. 
De todo o modo, sempre busca-se fazer mais com menos. Diante de tal filosofia, temos neste capítulo, uma breve discussão sobre alguns elementos de projeto e arquitetura de software, a fim de contextualizar este trabalho de conclusão de curso.
O capítulo corrente é composto por quatro seções. A \ref{sec:saas} trata de Software como serviço, discutindo alguns elementos do contexto que são relevantes para o trabalho proposto. A \ref{sec:reuso}  discute sobre a empregabilidade do reuso de software. A \ref{sec:modularizacao} seção, trata sobre aspectos envolvidos na modularização dos softwares. Por fim, a \ref{sec:apps_web} aborda as aplicações web, discutindo sobre aspectos relevantes sobre a aplicação web que compõe este trabalho.


 \section{Software como serviço}\label{sec:saas}


Segundo La e Chun \citep{La2009Systematic}, o princípio da definição de Software como um Serviço (Sofware as a Service - SaaS) é um serviço complementar para aplicações da computação em nuvem (cloud computing). No entanto, as áreas não se confundem. SaaS deve ser entendido como um mecanismo de suporte às soluções existentes na cloud. Os SaaS existem justamente para maximizar o reuso de serviços repetidos e não centrais em uma aplicação remota.


Como vantagens, diversos fatores podem ser favoráveis para a adoção de um SaaS, como custo e manutenção dentre outros fatores aplicáveis a soluções específicas. Lechesaet al. \citep{LechesaSS11}, quantificam índices sobre os fatores determinantes para adoção ou não de um SaaS voltado para ERP na África do Sul. Os principais fatores determinantes para adoção desse mecanismo de software são sua fluidez quanto à rede e a segurança. Esses fatores estão presentes na aplicação desenvolvida neste trabalho de conclusão de curso.


Devido ao fato de ter um serviço constantemente na nuvem, fica o questionamento sobre a segurança da informação manipulada. Sabe-se que a vulnerabilidade na Web não é restrita ao SaaS, atingindo diversos âmbitos. Rai et al. \cite{journals/corr/RaiSM13} consideram que o avanço da computação em nuvem não é um problema apenas para os serviços Web do ponto de vista da segurança, pois muitos trabalhos na literatura mostram a área como mais um ponto de vulnerabilidade para diversos setores, a exemplo de infraestrutura. No mesmo artigo mencionado de Rai et al. \cite{journals/corr/RaiSM13}, também realizaram-se estudos exploratórios junto a empresas usuárias de serviços em computação em nuvem e consideram que a perspectiva de SaaS também pode fortalecer a segurança nas aplicações de cloud computing, pois o software de autenticação compartilhado por várias aplicações em nuvem, oferece uma melhor padronização e consequente facilidade de prevenção a erros de vulnerabilidade específicas de cada módulo da pesquisa. Esse ponto de vista é fundamenta para a compreensão de SaaS.


A arquitetura de armazenamento de dados de um Saas pode variar de acordo com a necessidade do contexto. Huixin \cite{7586486} descreve possíveis modelagens nesse sentido. Essa abordagem pode ser com um banco de dados único, fazendo com que diferentes clientes compartilhem o mesmo banco, diferindo os dados através de controle de usuário, ou isolando os diferentes clientes através de bancos de dados exclusivos para cada um. Esse fator também pode ser combinado com a arquitetura da aplicação, caso ofereça aplicação única para todos os clientes ou aplicação compartilhada. Diante das possíveis abordagens, a modelagem de dados do software pode ser decidida pela regra de negócio. Este trabalho optou por aplicação única e banco de dados compartilhado.

Devido ao conceito particular de obtenção de software, tanto pela visão do cliente como pela visão do vendedor, é necessário compreender os diversos aspectos que podem ser relevantes ao orçar um Saas. O recente trabalho de T. Kaur et al. \citep{6949281} orienta um modelo para compor o custo de um Saas. O custo total seria composto pelos fatores que dão suporte ao funcionamento do software. Tais fatores incluem infra-estrutura, configurabilidade, customização, parâmetros de QoS (Quality of service) como escalabilidade, disponibilidade, usabilidade, pontualidade e desempenho da resposta, portabilidade, custo total de propriedade e retorno do investimento. Esses fatores caracterizam o custo de forma eficaz, possibilitando ao fornecedor, prover um Serviço de acordo com a exigência do consumidor em vários pacotes de serviços.


O conceito de software como serviço se aplica neste trabalho de conclusão de curso, pois o mesmo estará disponível na web com alta disponibilidade adotando as características apresentadas para qualquer pessoa que desejar usá-lo. Assim, basta se cadastrar, configurá-lo e usar como um serviço, sem preocupação com a manutenção do mesmo.


\section{Reuso de software}\label{sec:reuso} %CRUISE BOOK CAPITULO 2


De acordo com o livro Software pratical reuse \citep{Ezran2002}, o reuso de software é a utilização de qualquer informação que um desenvolvedor pode necessitar no processo de criação de software. O livro de Basili e Rombach \citep{Basili1991} define reutilização de software como o uso de tudo o que está associado a um projeto de conhecimento.
Assim, o objetivo da reutilização de software é reciclar o design, código e outros componentes de um produto de software e assim reduzir o custo, o tempo e melhorar a qualidade do produto.
Segundo Keswani et al. \cite{6783445}, o componente reutilizável de software pode ser qualquer parte de seu desenvolvimento, como um fragmento de código, design, casos de teste, ou até mesmo a especificação de requisitos de uma funcionalidade do software. 

O reuso de software pode ter impacto positivo em diversos aspectos do software, vejamos alguns, conforme apresentados no C.R.U.I.S.E Book \citep{cruise2007} :

\begin{itemize}

\item Qualidade: As correções de erro tornam-se úteis em todos os locais em que ocorreu, padronizando e facilitando a manutenção.

\item Produtividade: O ganho de produtividade é alcançado devido ao menor número de artefatos desenvolvido. Isso resulta em menor esforço de teste e também análise e design, reduzindo custos.

\item Confiabilidade: A utilização de componentes bem testados aumenta a
confiança no software. Além disso, a utilização de um mesmo componente em vários sistemas, aumenta a possibilidade de detecção de erros e reforça a confiança no componente.

\item Redução do Esforço: A reutilização de software proporciona uma redução do tempo de desenvolvimento, o que reduz o tempo necessário para o produto ser disponibilizado no mercado para trazer rentabilidade.

\item Trabalho redundante e tempo de desenvolvimento: Desenvolver um sistema do
zero significa desenvolvimento redundante de muitos componentes, como requisitos, especificações, casos de uso, arquitetura, etc. Isso pode ser evitado quando estes estão disponíveis como componentes reutilizáveis e podem ser compartilhados, resultando em um processo de desenvolvimento otimizado.

\item Documentação: Embora a documentação seja muito importante para a
manutenção de um sistema, muitas vezes é negligenciada. A reutilização de componentes de software reduz a quantidade de documentação a ser escrita, entretanto depende da qualidade do que está escrito. Assim, apenas a estrutura do sistema e os novos artefatos desenvolvidos necessitam ser documentados.

\item Custo de manutenção: Menos defeitos e manutenções são esperados quando tem-se comprovada a qualidade dos componentes utilizados.

\item Tamanho da equipe: É comum haver casos em que a equipe de desenvolvimento sofre sobrecarga. Entretanto, dobrar o tamanho da equipe de desenvolvimento não necessariamente duplica produtividade. Se muitos componentes podem ser reutilizados, é possível desenvolver com equipes menores, levando a melhor comunicação e aumento da produtividade.

\end{itemize}

Apesar dos benefícios da reutilização de software, ela não é suficientemente aproveitada. Existem fatores que influenciam direta ou indiretamente na sua adoção. Esses fatores podem ser de aspecto gerencial, organizacional, econômico, conceitual ou técnico. Veremos a seguir alguns aspectos que podem gerar conflito com a cultura de reuso de software, segundo o C.R.U.I.S.E Book \citep{cruise2007} :
%(Sametinger, 1997). REVER

\begin{itemize}
	
\item Falta de apoio da gestão: Como a reutilização de software gera custos iniciais,
a medida pode não ser amplamente alcançada em uma organização sem o apoio de alto nível de gestão. Os gestores têm de ser informados sobre os custos iniciais e serem convencidos sobre economias futuras.

\item Gerenciamento do Projeto: Gerenciar projetos tradicionais é uma tarefa árdua, principalmente, os que praticam a reutilização de software. Utilizando a técnica em larga escala, tem-se impacto sobre todo o ciclo de vida do software.

\item Estruturas organizacionais inadequadas: As estruturas organizacionais devem
considerar diferentes necessidades que surgem quando a reutilização em larga escala está sendo adotada. Por exemplo, uma equipe particionada pode ser alocada somente para desenvolver, manter e certificar componentes reutilizáveis de software.

\item Incentivos de gestão: É comum a falta de incentivo para deixar os desenvolvedores gastarem tempo elaborando componentes do sistemas. A produtividade é muitas vezes medida apenas no tempo necessário para concluir um projeto. Assim, fazer qualquer trabalho além disso, embora benéfico para a empresa como um todo, diminui o seu sucesso. Mesmo quando os componentes reutilizáveis são utilizados, os benefícios obtidos são uma pequena fração do que poderia ser alcançado caso houvesse reutilização explícita, planejada e organizada.

\item Dificuldade de encontrar software reutilizável: Para reutilizar os componentes, devem existir formas eficientes de busca. Além disso, é importante ter um repositório bem organizado contendo componentes com um eficiente meio de acesso.

\item Não reutilização do software encontrado. O acesso fácil ao software existente
não necessariamente aumentar a reutilização. Os componentes reutilizáveis devem ser cuidadosamente especificados, projetados, implementados e documentados, pois em alguns casos, modificar e adaptar o código  pode ser mais custoso que a programação da funcionalidade necessária a partir do zero.

\item Modificação: É muito difícil encontrar um componente que funcione
exatamente da mesma maneira que queremos. Desta forma, são necessárias modificações e devem existir formas de determinar os seus efeitos sobre o componente.


\end{itemize}


A definição do reuso de software pode ser associada a este trabalho, pois ainda que como um serviço, estaremos adotando uma aplicação única para servir aos mais diversos softwares de uma empresa, por exemplo. Logo, ao adotar o projeto, pode ser interessante aplicar algumas métricas para tomar conhecimento de possíveis vantanjens como as citadas nessa seção, a exemplo de economia de tempo de desenvolvimento, e custo do projeto.


%Outra diretriz importante para a reutilização de software é reduzir o risco na criação de novos softwares. O risco tende a ser bastante reduzido se os componentes que estão sendo reutilizados têm as documentação, interfaces necessárias e devidamente testadas, fatores que contibruem para uma fácil integração.
%De acordo com Keswani et al. \citep{6783445}, para o reuso de software dar retornos apropriados, o processo deve ser sistemático e planejado. Qualquer organização que implemente a reutilização de software deve identificar os melhores métodos e estratégias de reutilização para obter a máxima produtividade. A reutilização de software ajuda a evitar software de engenharia a partir do zero, pois usa módulos de software existentes. A reutilização de software, embora seja uma tarefa difícil, especialmente para softwares antigos sem padrões de projeto, pode melhorar significativamente a produtividade e a qualidade de um produto de software. Embora a reutilização de software não seja um novo campo, ela pode dar grandes retornos em curto período de tempo.


\section{Modularização}\label{sec:modularizacao} %artigo de claudio pagina 222 introdução


%A modularidade vem desempenhando um papel predominante estágios emergentes das disciplinas de arquitetura de software [13]. Engenheiros de software consideram modularidade como princípio base na comparação entre arquiteturas alternativas  e arquitetura degeneração [9]. De fato, os engenheiros de software são incentivados a arquitecturas, baseando-se numa multiplicidade de mecanismos de modularidade disponíveis em: 
%(i) Linguagens de descrição de arquitetura (ADLs), como ACME [8], 
%(ii) catálogos de arquitetônicos [2, 13], e 
%(iii) conhecem bem princípios de alto nível, como interfaces de componentes estreitos, acoplamento arquitectónico reduzido e semelhantes.


Conforme é frisado por Wickramaarachchi e Lai \citep{7062705}, o conceito de modularização na indústria de software tem uma longa história e tem sido utilizado para melhorar o processo de desenvolvimento de software em diferentes estágios. Os principais conceitos por trás da modularização do software foram introduzidos por pesquisadores pioneiros há quarenta anos, com uma notável contribuição feita por Melvin Conway e David Parnas, que tem representação notável na engenharia de software.


Modularizar um software é um bom padrão a ser adotado. Segundo Wickramaarachchi e Lai \citep{7062705}, a modularização é importante na identificação de dependências e reduz as dificuldades diante de uma possível necessidade de grandes alterações. De uma perspectiva da engenharia de software, uma modularização geralmente tem várias vantagens, tais como: tornar a complexidade do software mais gerenciável, facilitar o trabalho paralelo e tornar o software mais maleável para acomodar o futuro incerto que um software pode ter. O objetivo final da modularização do software é aumentar a produtividade ea qualidade do software. Tal conceito encontra-se bastante difundido e estái incorporado em linguagens de programação e ferramentas de software. O presente trabalho favorece ao uso da modularização de um software e até mesmo pode ser considerado um módulo a ser acoplado a qualquer software, mediante a compatibilidade.


Ao realizar a adoção do trabalho proposto, fica bastante evidente a "responsabilidade" do mesmo no escopo do projeto. Assim, é possível tratá-lo como um módulo do projeto que o usa, mesmo que seja consumido como um serviço.


\section{Aplicações web}\label{sec:apps_web}


A popularidade das soluções Web aumentou exponencialmente na última década e todos os dias cresce o número de pessoas usuárias desse tipo de software. E seguindo um padrão próprio, Kumar et al. \citep{7813710} sugerem que para o desenvolvimento web, deve-se manter a prática eficaz de produzir diagramas UML. A abordagem baseada na web oferece uma maneira fácil e eficaz para gerenciar e controlar o processo de desenvolvimento por meio de artefatos de modelagem. Tal abordagem pode ser usada quando há uma exigência de lidar com mudanças muito rápidas e grandes em requisitos de forma muito eficaz em muito menos tempo, gerando assim um menor impacto. 


Para atender à fomentada demanda de aplicativos web, é necessário adotar métodos de desenvolvimentos que sejam ágeis, eficientes e de fácil manutenção. Yu Ping et al. \cite{1372143} propõem o uso do modelo MVC (Model, View e Controller) no desenvolvimento para softwares web. O modelo apresentado tornou-se um padrão popular e divide o software em camadas com propósito definido, tornando-o de mais fácil manutenção.


O Ajax (Asynchronous Javascript and XML) revolucionou a web. Conforme demonstrado por Yuping \citep{6845605}, ao usar a tecnologia Ajax, podemos enriquecer a experiência do usuário em aplicações baseadas em navegador de internet, e fornecer uma variedade de aplicações interativas para atender às necessidade de humanização das aplicações.
Os aplicativos Ajax em execução no navegador se comunicam com um servidor Web de forma assíncrona e atualizam apenas uma parte da página.


No artigo de Tesarik et al. \citep{4664349}, temos que o design de software SPA (Single page application) é uma maneira interessante de criar um software Web numa uma única página. Essa solução de página única sem navegação funciona apenas com base em técnicas dinâmicas e assíncronas, como o AJAX (citado no parágrafo anterior). No entanto, esta abordagem coloca o desenvolvedor antes de alguns desafios substanciais.
Para projetar a interface do usuário que mostra as informações, é importante projetar corretamente a tela para manipular os dados do aplicativo numa única página. O design da página deve ser elaborado para maximizar a decomposição da página em componentes distintos que encapsulam os principais casos de uso.
Também recomenda-se que para elaborar os artefatos visuais, explore os mais ricos recursos na implementação, como HTML5, JavaScript, Ajax, CSS3, e outras tecnologias que se apliquem. Entretanto, desenvolver uma rica interface com o uso de diversas tecnologias/frameworks pode ocasionar um esforço maior para explorar as possibilidades.
Tais componentes da interface de uma aplicação SPA normalmente são alimentados mediante o consumo de uma API Rest via requisições AJAX. Assim, é possível particionar as responsabilidades de processamento do software entre cliente e servidor.


Como é possível denotar no trabalho de Salvadori e Siqueire \citep{7195633}, REST é uma arquitetura muito popular para integração de aplicativos web, que permite compartilhar e reutilizar informações através de sistemas. Aplicações de grande escala baseadas em no REST deve ser implementadas utilizando estratégias e mecanismos para produzir sistemas que sejam fáceis de desenvolver, reutilizar e manter. As interfaces de integração fornecidas por esses sistemas, chamadas de Web API, têm uma influência importante nas características da implementação resultante, pois a sua forma de resposta deve ser casada com a interface que o consome.
 
 
Os elementos comentados nessa seção encontram-se presentes neste trabalho de conclusão de curso. Juntos, montam a estrutura tecnológica necessária aliada à arquitetura adotada. Esses elementos seguem tendências atuais dos softwares web, fazendo com que esse trabalho esteja composto por tecnologias modernas que estão com boa aceitação no mercado.

% %% RiSE Latex Template - version 0.5
%%
%% RiSE's latex template for thesis and dissertations
%% http://risetemplate.sourceforge.net
%%
%% (c) 2012 Yguaratã Cerqueira Cavalcanti (yguarata@gmail.com)
%%          Vinicius Cardoso Garcia (vinicius.garcia@gmail.com)
%%
%% This document was initially based on UFPEThesis template, from Paulo Gustavo
%% S. Fonseca.
%%
%% ACKNOWLEDGEMENTS
%%
%% We would like to thanks the RiSE's researchers community, the 
%% students from Federal University of Pernambuco, and other users that have
%% been contributing to this projects with comments and patches.
%%
%% GENERAL INSTRUCTIONS
%%
%% We strongly recommend you to compile your documents using pdflatex command.
%% It is also recommend use the texlipse plugin for Eclipse to edit your documents.
%%
%% Options for \documentclass command:
%%         * Idiom
%%           pt   - Portguese (default)
%%           en   - English
%%
%%         * Text type
%%           bsc  - B.Sc. Thesis
%%           msc  - M.Sc. Thesis (default)
%%           qual - PHD qualification (not tested yet)
%%           prop - PHD proposal (not tested yet)
%%           phd  - PHD thesis
%%
%%         * Media
%%           scr  - to eletronic version (PDF) / see the users guide
%%
%%         * Pagination
%%           oneside - unique face press
%%           twoside - two faces press
%%
%%		   * Line spacing
%%           singlespacing  - the same as using \linespread{1}
%%           onehalfspacing - the same as using \linespread{1.3}
%%           doublespacing  - the same as using \linespread{1.6}
%%
%% Reference commands. Use the following commands to make references in your
%% text:
%%          \figref  -- for Figure reference
%%          \tabref  -- for Table reference
%%          \eqnref  -- for equation reference
%%          \chapref -- for chapter reference
%%          \secref  -- for section reference
%%          \appref  -- for appendix reference
%%          \axiref  -- for axiom reference
%%          \conjref -- for conjecture reference
%%          \defref  -- for definition reference
%%          \lemref  -- for lemma reference
%%          \theoref -- for theorem reference
%%          \corref  -- for corollary reference
%%          \propref -- for proprosition reference
%%          \pgref   -- for page reference
%%
%%          Example: See \chapref{chap:introduction}. It will produce 
%%                   'See Chapter 1', in case of English language.

\documentclass[pt,twoside,onehalfspacing,bsc]{risethesis}

\usepackage[sort,square]{natbib} %package for bibliography
\bibliographystyle{ieeetr} %authordate1

\usepackage[utf8]{inputenc}
\usepackage[brazilian]{babel}
\usepackage[T1]{fontenc}

\usepackage{graphicx}

%% Change the following pdf author attribute name to your name.
\usepackage[linkcolor=blue,citecolor=blue,urlcolor=blue,colorlinks,pdfpagelabels,pdftitle={Bruno Cabral's Bachelor Thesis},pdfauthor={Bruno Cabral}]{hyperref}

\address{SALVADOR}

\universitypt{Universidade Federal da Bahia}
\universityen{Federal University of Bahia}

\departmentpt{Depertamento de Ciência da Computação}
\departmenten{Computer Science Department}

\programpt{Programa Multiinstitucional de Pós-graduação em Ciência da Computação}
\programen{Graduate in Computer Science}

\majorfieldpt{Ciência da Computação}
\majorfielden{Computer Science}

\title{Sistema de apoio à Pós graduação - UFBA}
\date{Outubro/2016}

\author{Victor de Azevedo Nunes}
\adviser{Ivan do Carmo Machado}

\begin{document}

\frontmatter
\frontpage
\presentationpage

\begin{dedicatory}
Eu dedico esta dissertação...
%I dedicate this dissertation to my family, girlfriend, friends and
%professors who gave me all necessary support to get here.
\end{dedicatory}

\acknowledgements
Meus agradecimentos...

\begin{epigraph}[]{Edward V Berard}
Walking on water and developing software from a specification are easy if both are frozen
\end{epigraph}

\resumo
% Escreva seu resumo no arquivo resumo.tex
\input{resumo}

\abstract
% Write your abstract in a file called abstract.tex
\input{abstract}

% Summary (tables of contents)
\tableofcontents

% List of figures
\listoffigures

% List of tables
\listoftables

% List of acronyms
% Acronyms manual: http://linorg.usp.br/CTAN/macros/latex/contrib/acronym/acronym.pdf
\listofacronyms
\input{acronyms}

% List of listings
%\lstlistoflistings

\mainmatter

\include{chapters/intro}
\include{chapters/referencial_teorico}

% \include{chapters/introduction/main}
% \include{chapters/background/main}
% \include{chapters/proposed_solution/main}
% \include{chapters/experiment/main}
% \include{chapters/conclusion/main}

%\bibliographystyle{natbib}
%\addcontentsline{toc}{chapter}{\bibliographytocname}
\bibliography{references}

% Appendix
\clearpage
\addappheadtotoc
\appendix
\appendixpage
% \include{appendix/experiment-instruments}

\end{document}
% %% RiSE Latex Template - version 0.5
%%
%% RiSE's latex template for thesis and dissertations
%% http://risetemplate.sourceforge.net
%%
%% (c) 2012 Yguaratã Cerqueira Cavalcanti (yguarata@gmail.com)
%%          Vinicius Cardoso Garcia (vinicius.garcia@gmail.com)
%%
%% This document was initially based on UFPEThesis template, from Paulo Gustavo
%% S. Fonseca.
%%
%% ACKNOWLEDGEMENTS
%%
%% We would like to thanks the RiSE's researchers community, the 
%% students from Federal University of Pernambuco, and other users that have
%% been contributing to this projects with comments and patches.
%%
%% GENERAL INSTRUCTIONS
%%
%% We strongly recommend you to compile your documents using pdflatex command.
%% It is also recommend use the texlipse plugin for Eclipse to edit your documents.
%%
%% Options for \documentclass command:
%%         * Idiom
%%           pt   - Portguese (default)
%%           en   - English
%%
%%         * Text type
%%           bsc  - B.Sc. Thesis
%%           msc  - M.Sc. Thesis (default)
%%           qual - PHD qualification (not tested yet)
%%           prop - PHD proposal (not tested yet)
%%           phd  - PHD thesis
%%
%%         * Media
%%           scr  - to eletronic version (PDF) / see the users guide
%%
%%         * Pagination
%%           oneside - unique face press
%%           twoside - two faces press
%%
%%		   * Line spacing
%%           singlespacing  - the same as using \linespread{1}
%%           onehalfspacing - the same as using \linespread{1.3}
%%           doublespacing  - the same as using \linespread{1.6}
%%
%% Reference commands. Use the following commands to make references in your
%% text:
%%          \figref  -- for Figure reference
%%          \tabref  -- for Table reference
%%          \eqnref  -- for equation reference
%%          \chapref -- for chapter reference
%%          \secref  -- for section reference
%%          \appref  -- for appendix reference
%%          \axiref  -- for axiom reference
%%          \conjref -- for conjecture reference
%%          \defref  -- for definition reference
%%          \lemref  -- for lemma reference
%%          \theoref -- for theorem reference
%%          \corref  -- for corollary reference
%%          \propref -- for proprosition reference
%%          \pgref   -- for page reference
%%
%%          Example: See \chapref{chap:introduction}. It will produce 
%%                   'See Chapter 1', in case of English language.

\documentclass[pt,twoside,onehalfspacing,bsc]{risethesis}

\usepackage[sort,square]{natbib} %package for bibliography
\bibliographystyle{ieeetr} %authordate1

\usepackage[utf8]{inputenc}
\usepackage[brazilian]{babel}
\usepackage[T1]{fontenc}

\usepackage{graphicx}

%% Change the following pdf author attribute name to your name.
\usepackage[linkcolor=blue,citecolor=blue,urlcolor=blue,colorlinks,pdfpagelabels,pdftitle={Bruno Cabral's Bachelor Thesis},pdfauthor={Bruno Cabral}]{hyperref}

\address{SALVADOR}

\universitypt{Universidade Federal da Bahia}
\universityen{Federal University of Bahia}

\departmentpt{Depertamento de Ciência da Computação}
\departmenten{Computer Science Department}

\programpt{Programa Multiinstitucional de Pós-graduação em Ciência da Computação}
\programen{Graduate in Computer Science}

\majorfieldpt{Ciência da Computação}
\majorfielden{Computer Science}

\title{Sistema de apoio à Pós graduação - UFBA}
\date{Outubro/2016}

\author{Victor de Azevedo Nunes}
\adviser{Ivan do Carmo Machado}

\begin{document}

\frontmatter
\frontpage
\presentationpage

\begin{dedicatory}
Eu dedico esta dissertação...
%I dedicate this dissertation to my family, girlfriend, friends and
%professors who gave me all necessary support to get here.
\end{dedicatory}

\acknowledgements
Meus agradecimentos...

\begin{epigraph}[]{Edward V Berard}
Walking on water and developing software from a specification are easy if both are frozen
\end{epigraph}

\resumo
% Escreva seu resumo no arquivo resumo.tex
\input{resumo}

\abstract
% Write your abstract in a file called abstract.tex
\input{abstract}

% Summary (tables of contents)
\tableofcontents

% List of figures
\listoffigures

% List of tables
\listoftables

% List of acronyms
% Acronyms manual: http://linorg.usp.br/CTAN/macros/latex/contrib/acronym/acronym.pdf
\listofacronyms
\input{acronyms}

% List of listings
%\lstlistoflistings

\mainmatter

\include{chapters/intro}
\include{chapters/referencial_teorico}

% \include{chapters/introduction/main}
% \include{chapters/background/main}
% \include{chapters/proposed_solution/main}
% \include{chapters/experiment/main}
% \include{chapters/conclusion/main}

%\bibliographystyle{natbib}
%\addcontentsline{toc}{chapter}{\bibliographytocname}
\bibliography{references}

% Appendix
\clearpage
\addappheadtotoc
\appendix
\appendixpage
% \include{appendix/experiment-instruments}

\end{document}
% %% RiSE Latex Template - version 0.5
%%
%% RiSE's latex template for thesis and dissertations
%% http://risetemplate.sourceforge.net
%%
%% (c) 2012 Yguaratã Cerqueira Cavalcanti (yguarata@gmail.com)
%%          Vinicius Cardoso Garcia (vinicius.garcia@gmail.com)
%%
%% This document was initially based on UFPEThesis template, from Paulo Gustavo
%% S. Fonseca.
%%
%% ACKNOWLEDGEMENTS
%%
%% We would like to thanks the RiSE's researchers community, the 
%% students from Federal University of Pernambuco, and other users that have
%% been contributing to this projects with comments and patches.
%%
%% GENERAL INSTRUCTIONS
%%
%% We strongly recommend you to compile your documents using pdflatex command.
%% It is also recommend use the texlipse plugin for Eclipse to edit your documents.
%%
%% Options for \documentclass command:
%%         * Idiom
%%           pt   - Portguese (default)
%%           en   - English
%%
%%         * Text type
%%           bsc  - B.Sc. Thesis
%%           msc  - M.Sc. Thesis (default)
%%           qual - PHD qualification (not tested yet)
%%           prop - PHD proposal (not tested yet)
%%           phd  - PHD thesis
%%
%%         * Media
%%           scr  - to eletronic version (PDF) / see the users guide
%%
%%         * Pagination
%%           oneside - unique face press
%%           twoside - two faces press
%%
%%		   * Line spacing
%%           singlespacing  - the same as using \linespread{1}
%%           onehalfspacing - the same as using \linespread{1.3}
%%           doublespacing  - the same as using \linespread{1.6}
%%
%% Reference commands. Use the following commands to make references in your
%% text:
%%          \figref  -- for Figure reference
%%          \tabref  -- for Table reference
%%          \eqnref  -- for equation reference
%%          \chapref -- for chapter reference
%%          \secref  -- for section reference
%%          \appref  -- for appendix reference
%%          \axiref  -- for axiom reference
%%          \conjref -- for conjecture reference
%%          \defref  -- for definition reference
%%          \lemref  -- for lemma reference
%%          \theoref -- for theorem reference
%%          \corref  -- for corollary reference
%%          \propref -- for proprosition reference
%%          \pgref   -- for page reference
%%
%%          Example: See \chapref{chap:introduction}. It will produce 
%%                   'See Chapter 1', in case of English language.

\documentclass[pt,twoside,onehalfspacing,bsc]{risethesis}

\usepackage[sort,square]{natbib} %package for bibliography
\bibliographystyle{ieeetr} %authordate1

\usepackage[utf8]{inputenc}
\usepackage[brazilian]{babel}
\usepackage[T1]{fontenc}

\usepackage{graphicx}

%% Change the following pdf author attribute name to your name.
\usepackage[linkcolor=blue,citecolor=blue,urlcolor=blue,colorlinks,pdfpagelabels,pdftitle={Bruno Cabral's Bachelor Thesis},pdfauthor={Bruno Cabral}]{hyperref}

\address{SALVADOR}

\universitypt{Universidade Federal da Bahia}
\universityen{Federal University of Bahia}

\departmentpt{Depertamento de Ciência da Computação}
\departmenten{Computer Science Department}

\programpt{Programa Multiinstitucional de Pós-graduação em Ciência da Computação}
\programen{Graduate in Computer Science}

\majorfieldpt{Ciência da Computação}
\majorfielden{Computer Science}

\title{Sistema de apoio à Pós graduação - UFBA}
\date{Outubro/2016}

\author{Victor de Azevedo Nunes}
\adviser{Ivan do Carmo Machado}

\begin{document}

\frontmatter
\frontpage
\presentationpage

\begin{dedicatory}
Eu dedico esta dissertação...
%I dedicate this dissertation to my family, girlfriend, friends and
%professors who gave me all necessary support to get here.
\end{dedicatory}

\acknowledgements
Meus agradecimentos...

\begin{epigraph}[]{Edward V Berard}
Walking on water and developing software from a specification are easy if both are frozen
\end{epigraph}

\resumo
% Escreva seu resumo no arquivo resumo.tex
\input{resumo}

\abstract
% Write your abstract in a file called abstract.tex
\input{abstract}

% Summary (tables of contents)
\tableofcontents

% List of figures
\listoffigures

% List of tables
\listoftables

% List of acronyms
% Acronyms manual: http://linorg.usp.br/CTAN/macros/latex/contrib/acronym/acronym.pdf
\listofacronyms
\input{acronyms}

% List of listings
%\lstlistoflistings

\mainmatter

\include{chapters/intro}
\include{chapters/referencial_teorico}

% \include{chapters/introduction/main}
% \include{chapters/background/main}
% \include{chapters/proposed_solution/main}
% \include{chapters/experiment/main}
% \include{chapters/conclusion/main}

%\bibliographystyle{natbib}
%\addcontentsline{toc}{chapter}{\bibliographytocname}
\bibliography{references}

% Appendix
\clearpage
\addappheadtotoc
\appendix
\appendixpage
% \include{appendix/experiment-instruments}

\end{document}
% %% RiSE Latex Template - version 0.5
%%
%% RiSE's latex template for thesis and dissertations
%% http://risetemplate.sourceforge.net
%%
%% (c) 2012 Yguaratã Cerqueira Cavalcanti (yguarata@gmail.com)
%%          Vinicius Cardoso Garcia (vinicius.garcia@gmail.com)
%%
%% This document was initially based on UFPEThesis template, from Paulo Gustavo
%% S. Fonseca.
%%
%% ACKNOWLEDGEMENTS
%%
%% We would like to thanks the RiSE's researchers community, the 
%% students from Federal University of Pernambuco, and other users that have
%% been contributing to this projects with comments and patches.
%%
%% GENERAL INSTRUCTIONS
%%
%% We strongly recommend you to compile your documents using pdflatex command.
%% It is also recommend use the texlipse plugin for Eclipse to edit your documents.
%%
%% Options for \documentclass command:
%%         * Idiom
%%           pt   - Portguese (default)
%%           en   - English
%%
%%         * Text type
%%           bsc  - B.Sc. Thesis
%%           msc  - M.Sc. Thesis (default)
%%           qual - PHD qualification (not tested yet)
%%           prop - PHD proposal (not tested yet)
%%           phd  - PHD thesis
%%
%%         * Media
%%           scr  - to eletronic version (PDF) / see the users guide
%%
%%         * Pagination
%%           oneside - unique face press
%%           twoside - two faces press
%%
%%		   * Line spacing
%%           singlespacing  - the same as using \linespread{1}
%%           onehalfspacing - the same as using \linespread{1.3}
%%           doublespacing  - the same as using \linespread{1.6}
%%
%% Reference commands. Use the following commands to make references in your
%% text:
%%          \figref  -- for Figure reference
%%          \tabref  -- for Table reference
%%          \eqnref  -- for equation reference
%%          \chapref -- for chapter reference
%%          \secref  -- for section reference
%%          \appref  -- for appendix reference
%%          \axiref  -- for axiom reference
%%          \conjref -- for conjecture reference
%%          \defref  -- for definition reference
%%          \lemref  -- for lemma reference
%%          \theoref -- for theorem reference
%%          \corref  -- for corollary reference
%%          \propref -- for proprosition reference
%%          \pgref   -- for page reference
%%
%%          Example: See \chapref{chap:introduction}. It will produce 
%%                   'See Chapter 1', in case of English language.

\documentclass[pt,twoside,onehalfspacing,bsc]{risethesis}

\usepackage[sort,square]{natbib} %package for bibliography
\bibliographystyle{ieeetr} %authordate1

\usepackage[utf8]{inputenc}
\usepackage[brazilian]{babel}
\usepackage[T1]{fontenc}

\usepackage{graphicx}

%% Change the following pdf author attribute name to your name.
\usepackage[linkcolor=blue,citecolor=blue,urlcolor=blue,colorlinks,pdfpagelabels,pdftitle={Bruno Cabral's Bachelor Thesis},pdfauthor={Bruno Cabral}]{hyperref}

\address{SALVADOR}

\universitypt{Universidade Federal da Bahia}
\universityen{Federal University of Bahia}

\departmentpt{Depertamento de Ciência da Computação}
\departmenten{Computer Science Department}

\programpt{Programa Multiinstitucional de Pós-graduação em Ciência da Computação}
\programen{Graduate in Computer Science}

\majorfieldpt{Ciência da Computação}
\majorfielden{Computer Science}

\title{Sistema de apoio à Pós graduação - UFBA}
\date{Outubro/2016}

\author{Victor de Azevedo Nunes}
\adviser{Ivan do Carmo Machado}

\begin{document}

\frontmatter
\frontpage
\presentationpage

\begin{dedicatory}
Eu dedico esta dissertação...
%I dedicate this dissertation to my family, girlfriend, friends and
%professors who gave me all necessary support to get here.
\end{dedicatory}

\acknowledgements
Meus agradecimentos...

\begin{epigraph}[]{Edward V Berard}
Walking on water and developing software from a specification are easy if both are frozen
\end{epigraph}

\resumo
% Escreva seu resumo no arquivo resumo.tex
\input{resumo}

\abstract
% Write your abstract in a file called abstract.tex
\input{abstract}

% Summary (tables of contents)
\tableofcontents

% List of figures
\listoffigures

% List of tables
\listoftables

% List of acronyms
% Acronyms manual: http://linorg.usp.br/CTAN/macros/latex/contrib/acronym/acronym.pdf
\listofacronyms
\input{acronyms}

% List of listings
%\lstlistoflistings

\mainmatter

\include{chapters/intro}
\include{chapters/referencial_teorico}

% \include{chapters/introduction/main}
% \include{chapters/background/main}
% \include{chapters/proposed_solution/main}
% \include{chapters/experiment/main}
% \include{chapters/conclusion/main}

%\bibliographystyle{natbib}
%\addcontentsline{toc}{chapter}{\bibliographytocname}
\bibliography{references}

% Appendix
\clearpage
\addappheadtotoc
\appendix
\appendixpage
% \include{appendix/experiment-instruments}

\end{document}
% %% RiSE Latex Template - version 0.5
%%
%% RiSE's latex template for thesis and dissertations
%% http://risetemplate.sourceforge.net
%%
%% (c) 2012 Yguaratã Cerqueira Cavalcanti (yguarata@gmail.com)
%%          Vinicius Cardoso Garcia (vinicius.garcia@gmail.com)
%%
%% This document was initially based on UFPEThesis template, from Paulo Gustavo
%% S. Fonseca.
%%
%% ACKNOWLEDGEMENTS
%%
%% We would like to thanks the RiSE's researchers community, the 
%% students from Federal University of Pernambuco, and other users that have
%% been contributing to this projects with comments and patches.
%%
%% GENERAL INSTRUCTIONS
%%
%% We strongly recommend you to compile your documents using pdflatex command.
%% It is also recommend use the texlipse plugin for Eclipse to edit your documents.
%%
%% Options for \documentclass command:
%%         * Idiom
%%           pt   - Portguese (default)
%%           en   - English
%%
%%         * Text type
%%           bsc  - B.Sc. Thesis
%%           msc  - M.Sc. Thesis (default)
%%           qual - PHD qualification (not tested yet)
%%           prop - PHD proposal (not tested yet)
%%           phd  - PHD thesis
%%
%%         * Media
%%           scr  - to eletronic version (PDF) / see the users guide
%%
%%         * Pagination
%%           oneside - unique face press
%%           twoside - two faces press
%%
%%		   * Line spacing
%%           singlespacing  - the same as using \linespread{1}
%%           onehalfspacing - the same as using \linespread{1.3}
%%           doublespacing  - the same as using \linespread{1.6}
%%
%% Reference commands. Use the following commands to make references in your
%% text:
%%          \figref  -- for Figure reference
%%          \tabref  -- for Table reference
%%          \eqnref  -- for equation reference
%%          \chapref -- for chapter reference
%%          \secref  -- for section reference
%%          \appref  -- for appendix reference
%%          \axiref  -- for axiom reference
%%          \conjref -- for conjecture reference
%%          \defref  -- for definition reference
%%          \lemref  -- for lemma reference
%%          \theoref -- for theorem reference
%%          \corref  -- for corollary reference
%%          \propref -- for proprosition reference
%%          \pgref   -- for page reference
%%
%%          Example: See \chapref{chap:introduction}. It will produce 
%%                   'See Chapter 1', in case of English language.

\documentclass[pt,twoside,onehalfspacing,bsc]{risethesis}

\usepackage[sort,square]{natbib} %package for bibliography
\bibliographystyle{ieeetr} %authordate1

\usepackage[utf8]{inputenc}
\usepackage[brazilian]{babel}
\usepackage[T1]{fontenc}

\usepackage{graphicx}

%% Change the following pdf author attribute name to your name.
\usepackage[linkcolor=blue,citecolor=blue,urlcolor=blue,colorlinks,pdfpagelabels,pdftitle={Bruno Cabral's Bachelor Thesis},pdfauthor={Bruno Cabral}]{hyperref}

\address{SALVADOR}

\universitypt{Universidade Federal da Bahia}
\universityen{Federal University of Bahia}

\departmentpt{Depertamento de Ciência da Computação}
\departmenten{Computer Science Department}

\programpt{Programa Multiinstitucional de Pós-graduação em Ciência da Computação}
\programen{Graduate in Computer Science}

\majorfieldpt{Ciência da Computação}
\majorfielden{Computer Science}

\title{Sistema de apoio à Pós graduação - UFBA}
\date{Outubro/2016}

\author{Victor de Azevedo Nunes}
\adviser{Ivan do Carmo Machado}

\begin{document}

\frontmatter
\frontpage
\presentationpage

\begin{dedicatory}
Eu dedico esta dissertação...
%I dedicate this dissertation to my family, girlfriend, friends and
%professors who gave me all necessary support to get here.
\end{dedicatory}

\acknowledgements
Meus agradecimentos...

\begin{epigraph}[]{Edward V Berard}
Walking on water and developing software from a specification are easy if both are frozen
\end{epigraph}

\resumo
% Escreva seu resumo no arquivo resumo.tex
\input{resumo}

\abstract
% Write your abstract in a file called abstract.tex
\input{abstract}

% Summary (tables of contents)
\tableofcontents

% List of figures
\listoffigures

% List of tables
\listoftables

% List of acronyms
% Acronyms manual: http://linorg.usp.br/CTAN/macros/latex/contrib/acronym/acronym.pdf
\listofacronyms
\input{acronyms}

% List of listings
%\lstlistoflistings

\mainmatter

\include{chapters/intro}
\include{chapters/referencial_teorico}

% \include{chapters/introduction/main}
% \include{chapters/background/main}
% \include{chapters/proposed_solution/main}
% \include{chapters/experiment/main}
% \include{chapters/conclusion/main}

%\bibliographystyle{natbib}
%\addcontentsline{toc}{chapter}{\bibliographytocname}
\bibliography{references}

% Appendix
\clearpage
\addappheadtotoc
\appendix
\appendixpage
% \include{appendix/experiment-instruments}

\end{document}

%\bibliographystyle{natbib}
%\addcontentsline{toc}{chapter}{\bibliographytocname}
\bibliography{references}

% Appendix
\clearpage
\addappheadtotoc
\appendix
\appendixpage
% \include{appendix/experiment-instruments}

\end{document}
% %% RiSE Latex Template - version 0.5
%%
%% RiSE's latex template for thesis and dissertations
%% http://risetemplate.sourceforge.net
%%
%% (c) 2012 Yguaratã Cerqueira Cavalcanti (yguarata@gmail.com)
%%          Vinicius Cardoso Garcia (vinicius.garcia@gmail.com)
%%
%% This document was initially based on UFPEThesis template, from Paulo Gustavo
%% S. Fonseca.
%%
%% ACKNOWLEDGEMENTS
%%
%% We would like to thanks the RiSE's researchers community, the 
%% students from Federal University of Pernambuco, and other users that have
%% been contributing to this projects with comments and patches.
%%
%% GENERAL INSTRUCTIONS
%%
%% We strongly recommend you to compile your documents using pdflatex command.
%% It is also recommend use the texlipse plugin for Eclipse to edit your documents.
%%
%% Options for \documentclass command:
%%         * Idiom
%%           pt   - Portguese (default)
%%           en   - English
%%
%%         * Text type
%%           bsc  - B.Sc. Thesis
%%           msc  - M.Sc. Thesis (default)
%%           qual - PHD qualification (not tested yet)
%%           prop - PHD proposal (not tested yet)
%%           phd  - PHD thesis
%%
%%         * Media
%%           scr  - to eletronic version (PDF) / see the users guide
%%
%%         * Pagination
%%           oneside - unique face press
%%           twoside - two faces press
%%
%%		   * Line spacing
%%           singlespacing  - the same as using \linespread{1}
%%           onehalfspacing - the same as using \linespread{1.3}
%%           doublespacing  - the same as using \linespread{1.6}
%%
%% Reference commands. Use the following commands to make references in your
%% text:
%%          \figref  -- for Figure reference
%%          \tabref  -- for Table reference
%%          \eqnref  -- for equation reference
%%          \chapref -- for chapter reference
%%          \secref  -- for section reference
%%          \appref  -- for appendix reference
%%          \axiref  -- for axiom reference
%%          \conjref -- for conjecture reference
%%          \defref  -- for definition reference
%%          \lemref  -- for lemma reference
%%          \theoref -- for theorem reference
%%          \corref  -- for corollary reference
%%          \propref -- for proprosition reference
%%          \pgref   -- for page reference
%%
%%          Example: See \chapref{chap:introduction}. It will produce 
%%                   'See Chapter 1', in case of English language.

\documentclass[pt,twoside,onehalfspacing,bsc]{risethesis}

\usepackage[sort,square]{natbib} %package for bibliography
\bibliographystyle{ieeetr} %authordate1

\usepackage[utf8]{inputenc}
\usepackage[brazilian]{babel}
\usepackage[T1]{fontenc}

\usepackage{graphicx}

%% Change the following pdf author attribute name to your name.
\usepackage[linkcolor=blue,citecolor=blue,urlcolor=blue,colorlinks,pdfpagelabels,pdftitle={Bruno Cabral's Bachelor Thesis},pdfauthor={Bruno Cabral}]{hyperref}

\address{SALVADOR}

\universitypt{Universidade Federal da Bahia}
\universityen{Federal University of Bahia}

\departmentpt{Depertamento de Ciência da Computação}
\departmenten{Computer Science Department}

\programpt{Programa Multiinstitucional de Pós-graduação em Ciência da Computação}
\programen{Graduate in Computer Science}

\majorfieldpt{Ciência da Computação}
\majorfielden{Computer Science}

\title{Sistema de apoio à Pós graduação - UFBA}
\date{Outubro/2016}

\author{Victor de Azevedo Nunes}
\adviser{Ivan do Carmo Machado}

\begin{document}

\frontmatter
\frontpage
\presentationpage

\begin{dedicatory}
Eu dedico esta dissertação...
%I dedicate this dissertation to my family, girlfriend, friends and
%professors who gave me all necessary support to get here.
\end{dedicatory}

\acknowledgements
Meus agradecimentos...

\begin{epigraph}[]{Edward V Berard}
Walking on water and developing software from a specification are easy if both are frozen
\end{epigraph}

\resumo
% Escreva seu resumo no arquivo resumo.tex
Este trabalho de conclusão de curso utiliza elementos da engenharia de software para propor um software como serviço a fim de otimizar o processo de construção e manutenção dos softwares. Assim, o objetivo deste SaaS é gerenciar as permissões de acesso de sistemas cliente, provendo o reuso de software e padronizando as soluções. O sistema proposto neste trabalho encontra-se implementado e disponível no Github, e traz fundamentos sobre a arquitetura e tecnologias utilizadas, bem como avaliações sobre possibilidades de utilização do mesmo em diversos ambientes, como Web e mobile.

\begin{keywords}
Software; Reuso; SaaS; Web; PHP
\end{keywords}

\abstract
% Write your abstract in a file called abstract.tex
My abstract...

\begin{keywords}
key words...
\end{keywords}

% Summary (tables of contents)
\tableofcontents

% List of figures
\listoffigures

% List of tables
\listoftables

% List of acronyms
% Acronyms manual: http://linorg.usp.br/CTAN/macros/latex/contrib/acronym/acronym.pdf
\listofacronyms
\begin{acronym}[ACRONYM] 
% Change the word ACRONYM above to change the acronym column width.
% The column width is equals to the width of the word that you put.
% Read the manual about acronym package for more examples:
%   http://linorg.usp.br/CTAN/macros/latex/contrib/acronym/acronym.pdf

\acro{SPA}{Single Page Application}
\acro{JSON}{Javascript Object Notation}
\acro{PHP}{PHP: Hypertext Preprocessor}
\acro{SaaS}{Software as a Service}
\acro{ERP}{Enterprise Resource Planning}
\acro{QoS}{Quality of Service}
\acro{UML}{Unified Modeling Language}
\acro{MVC}{Model-View-Controller}
\acro{Ajax}{Asynchronous Javascript and XML}
\acro{HTML}{HyperText Markup Language}
\acro{CSS}{Cascading Style Sheets}
\acro{API}{Application Programming Interface}
\acro{DOM}{Document Object Model}
\acro{BPMN}{Business Process Model and Notation}
\acro{REST}{Representational State Transfer}

\end{acronym}

% List of listings
%\lstlistoflistings

\mainmatter

\chapter{Introdução}

\section{Motivação}

Organizar os procedimentos de um processo sempre nos traz vantagens. Apesar de no processo de implantação de um sistema, o mesmo burocratizar o processo, com o tempo temos o retorno da dedicação para a inserção dos dados. Com um certo volume de dados, é possível estruturar informações que num processo manual são difíceis de serem enxergadas. Assim, é possível depender menos das pessoas que organizam o processo, pois o legado de informações não estará mais somente na mente de alguns, mas sim documentado nos dados do sistema.

Além de colaborar na organização, também haverá uma grande colaboração no tempo gasto na gestão. Lidar com muitos papéis e confiar na mente humana para guardar informações, não é uma alternativa muito segura devido ao fato que as pessoas sempre estão sujeitas a sair do processo e levar contigo a experiência obtida. Experiência essa que faz com que os procedimentos sejam executados de forma mais eficiente. Entretanto, com um sistema inteligente, é possível auxiliar e tornar mais ágil a execução das tarefas.


\section{Problema}


De acordo com funcionários ligados ao o setor de pós graduação da UFBA, entrevistados a fim de um maior entendimento do cenário, apesar das semelhanças estruturais, a pós graduação gerida de forma diferencia da graduação. FULANO afirma que devido ao fato de não ter a mesma visibilidade, não tem acesso aos mesmos recursos de gestão acadêmica da graduação. O professores não executam somente atividades dentro da sala de aula, também tem diversas outras ocupações no setor. E muitos procedimentos realizados extra classe ainda se encontram sendo realizados de forma manual, estando mais vulnerável ao erro ou até mesmo à violação do processo. Também ocorre um grande desperdício de tempo pelos professores e gestores da área, devido ao diversos processos ainda realizados de forma manual, sem a devida documentação. Segundo FULANO, também entrevistado, esse tempo perdido implica numa redução da eficiência na sala de aula, pois o professor acaba por ter menos tempo disponível para o planejamento das atividades, o que gera impactos negativos aos alunos.


\section{Objetivos} %<o que deve ser feito/entregue>


Devido aos muitos processos sendo resolvidos de forma manual, propõe-se com solução um sistema moderno, arquitetado para ter funcionamento na web e com um módulo mobile, a fim de fornecer informações de forma rápida e eficiente para os professores através de notificações, já que o acesso à internet móvel é comum entre os possíveis usuários do sistema em questão.
O principal requisito para o sistema seria dispor recursos para reduzir o tempo desperdiçado pelos professores durante as atividades extra classe.


\section{Metodologia} %<como será feito | como resolver o problema apontado inicialmente>


%<analise de literatura | design | implementação | validação>
Baseando-se nas tecnologias gratuitas em alta no cenário atual do desenvolvimento web, dispomos de algumas opções eficientes para a implementação da solução. Dentre as possibilidades, considerando a facilidade para futura manutenção e continuidade do projeto, tende-se a optar por uma tecnologia popular. Como linguagem de programação, adota-se o PHP. A escolha é fundamentada de acordo com a pesquisa da RedMonk de 2015, que evidencia o uso das linguagens de programação de acordo com as discussões no StackOverflow e repositórios no GitHub. É possível constatar a popularidade do PHP no cenário atual com o gráfico da pesquisa citada, na qual o PHP é apresentado na terceira colocação, apenas atrás do lider JavaScript e do segundo colocado, o Java.

\begin{figure}
	\label{fig:graficoRedmonk}
	\includegraphics[width=1\textwidth]{img/grafico_redmonk}
	\caption{Ranking das liguagens de programação no Stack Overflow e Github}
\end{figure}


Ainda assim, para compor a interface do dado projeto, também ocorrerá o uso do líder JavaScript de forma intensa, provendo o elo com o as informações gerenciadas pelo PHP.


Entretanto, não seria inteligente desenvolver um sistema completo sem o auxílio de um framework. Dentre os frameworks disponíveis para PHP, hoje o destaque está com o Laravel, que se encontra no topo dentre os mais utilizados no momento. 


A WebHostFace, uma empresa de hospedagem, compilou várias estatísticas para criar um infográfico mostrando os frameworks PHP mais populares de 2015. Utilizando informações sobre os próprios clientes, o Google Trends, estatísticas de repositórios do GitHub e a pesquisa do SitePoint “Best PHP Frameworks 2015”, a WebHostFace elaborou o seguinte infográfico: 

\begin{figure}
	\label{fig:graficoWebhostface}
	\includegraphics[width=1\textwidth]{img/infografico_webhostface}
	\caption{Infográfico da WebhostFace, exibindo a popularidade dos Frameworks PHP em 2015}
\end{figure}

Assim, tem-se a evidência que o Laravel em 2015 teve a maior popularidade em projetos pessoais e tem a maior comunidade entre os concorrentes, o que o torna uma boa escolha para a escrita de um software que será continuado por terceiros.


Para elaborar os recursos de interface e integrar ao back-end PHP do sistema, será adotado o já conhecido AngularJS, ferramenta sólida e conhecida no aspecto em questão. 


Dados coletados via Google Trends, que propõe comparações entre termos pesquisados, revela a popularidade do AngularJs diante de alguns dos principais concorrentes. O gráfico abaixo evidencia o cenário.


%Como mostra a Figura \ref{fig:graficoGoogleTrendsFerramentasFront}. 
\begin{figure}
	\label{fig:graficoGoogleTrendsFerramentasFront}
	\includegraphics[width=1\textwidth]{img/grafico_ferramentas_front}
	\caption{Gráfico do Google Trends exibindo as pesquisas por ferramentas front-end}
\end{figure}


Junto ao Angular JS, será utilizada a agradável tendência de interface do Material Design da Google, que propõe layouts limpos e otimizados já conhecidos pelos usuários de smartphones Android. 


Para a elaboração da plataforma mobile do projeto, será utilizado o Ionic Framework, muito difundido e bastante pesquisado na área, o que fica evidenciado com o gráfico de pesquisbaixo, coletado via Google Trends buscando por frameworks de desenvolvimento híbrido mobile.


\begin{figure}
	\label{fig:graficoGoogleTrendsFerramentasHibridasMobile}
	\includegraphics[width=1\textwidth]{img/grafico_ferramentas_hibridas_mobile}
	\caption{Gráfico do Google Trends exibindo as pesquisas por Frameworks híbridos mobile}
\end{figure}	

Para layout da interface mobile, também será aplicado a tendência do Material Design, a fim de propor uma harmonia entre o módulo web e mobile para os usuários


\section{Resultados Esperados}


Como fruto de um sistema para pós-graduação da UFBA, espera-se que os professores tenham mais recursos para integrar as atividades e também prover melhores condições para acompanhamento da vida acadêmica dos alunos em questão. Também, que os novos colaboradores que entrarem no processo tenham facilidade de compreender o fluxo do setor ao navegar pelo sistema proposto.


\section{Fora de Escopo}


Interação com os alunos devido às complicações para realizar a integração com o sistema empregado na UFBA, gerenciado pela XXXXXX, o que causaria uma inviabilidade no projeto devido à necessidade de entrega do produto ser mais forte que o tempo necessário para executar o processo de obtenção de acesso ao sistema legado para realizar a integração.


\section{Estrutura do Trabalho}


<breve resumo sobre os capítulos do TCC>
\chapter{Referencial Teórico}


Projetar o desenvolvimento de um software requer muito planejamento, pois as falhas iniciais podem custar bastante caro ou até mesmo inviabilizar a continuação de um projeto. Assim, a escolha da arquitetura ideal para a aplicabilidade é essencial na concepção de um produto de software. 
De todo o modo, sempre busca-se fazer mais com menos. Diante de tal filosofia, temos neste capítulo, uma breve discussão sobre alguns elementos de projeto e arquitetura de software, a fim de contextualizar este trabalho de conclusão de curso.
O capítulo corrente é composto por quatro seções. A \ref{sec:saas} trata de Software como serviço, discutindo alguns elementos do contexto que são relevantes para o trabalho proposto. A \ref{sec:reuso}  discute sobre a empregabilidade do reuso de software. A \ref{sec:modularizacao} seção, trata sobre aspectos envolvidos na modularização dos softwares. Por fim, a \ref{sec:apps_web} aborda as aplicações web, discutindo sobre aspectos relevantes sobre a aplicação web que compõe este trabalho.


 \section{Software como serviço}\label{sec:saas}


Segundo La e Chun \citep{La2009Systematic}, o princípio da definição de Software como um Serviço (Sofware as a Service - SaaS) é um serviço complementar para aplicações da computação em nuvem (cloud computing). No entanto, as áreas não se confundem. SaaS deve ser entendido como um mecanismo de suporte às soluções existentes na cloud. Os SaaS existem justamente para maximizar o reuso de serviços repetidos e não centrais em uma aplicação remota.


Como vantagens, diversos fatores podem ser favoráveis para a adoção de um SaaS, como custo e manutenção dentre outros fatores aplicáveis a soluções específicas. Lechesaet al. \citep{LechesaSS11}, quantificam índices sobre os fatores determinantes para adoção ou não de um SaaS voltado para ERP na África do Sul. Os principais fatores determinantes para adoção desse mecanismo de software são sua fluidez quanto à rede e a segurança. Esses fatores estão presentes na aplicação desenvolvida neste trabalho de conclusão de curso.


Devido ao fato de ter um serviço constantemente na nuvem, fica o questionamento sobre a segurança da informação manipulada. Sabe-se que a vulnerabilidade na Web não é restrita ao SaaS, atingindo diversos âmbitos. Rai et al. \cite{journals/corr/RaiSM13} consideram que o avanço da computação em nuvem não é um problema apenas para os serviços Web do ponto de vista da segurança, pois muitos trabalhos na literatura mostram a área como mais um ponto de vulnerabilidade para diversos setores, a exemplo de infraestrutura. No mesmo artigo mencionado de Rai et al. \cite{journals/corr/RaiSM13}, também realizaram-se estudos exploratórios junto a empresas usuárias de serviços em computação em nuvem e consideram que a perspectiva de SaaS também pode fortalecer a segurança nas aplicações de cloud computing, pois o software de autenticação compartilhado por várias aplicações em nuvem, oferece uma melhor padronização e consequente facilidade de prevenção a erros de vulnerabilidade específicas de cada módulo da pesquisa. Esse ponto de vista é fundamenta para a compreensão de SaaS.


A arquitetura de armazenamento de dados de um Saas pode variar de acordo com a necessidade do contexto. Huixin \cite{7586486} descreve possíveis modelagens nesse sentido. Essa abordagem pode ser com um banco de dados único, fazendo com que diferentes clientes compartilhem o mesmo banco, diferindo os dados através de controle de usuário, ou isolando os diferentes clientes através de bancos de dados exclusivos para cada um. Esse fator também pode ser combinado com a arquitetura da aplicação, caso ofereça aplicação única para todos os clientes ou aplicação compartilhada. Diante das possíveis abordagens, a modelagem de dados do software pode ser decidida pela regra de negócio. Este trabalho optou por aplicação única e banco de dados compartilhado.

Devido ao conceito particular de obtenção de software, tanto pela visão do cliente como pela visão do vendedor, é necessário compreender os diversos aspectos que podem ser relevantes ao orçar um Saas. O recente trabalho de T. Kaur et al. \citep{6949281} orienta um modelo para compor o custo de um Saas. O custo total seria composto pelos fatores que dão suporte ao funcionamento do software. Tais fatores incluem infra-estrutura, configurabilidade, customização, parâmetros de QoS (Quality of service) como escalabilidade, disponibilidade, usabilidade, pontualidade e desempenho da resposta, portabilidade, custo total de propriedade e retorno do investimento. Esses fatores caracterizam o custo de forma eficaz, possibilitando ao fornecedor, prover um Serviço de acordo com a exigência do consumidor em vários pacotes de serviços.


O conceito de software como serviço se aplica neste trabalho de conclusão de curso, pois o mesmo estará disponível na web com alta disponibilidade adotando as características apresentadas para qualquer pessoa que desejar usá-lo. Assim, basta se cadastrar, configurá-lo e usar como um serviço, sem preocupação com a manutenção do mesmo.


\section{Reuso de software}\label{sec:reuso} %CRUISE BOOK CAPITULO 2


De acordo com o livro Software pratical reuse \citep{Ezran2002}, o reuso de software é a utilização de qualquer informação que um desenvolvedor pode necessitar no processo de criação de software. O livro de Basili e Rombach \citep{Basili1991} define reutilização de software como o uso de tudo o que está associado a um projeto de conhecimento.
Assim, o objetivo da reutilização de software é reciclar o design, código e outros componentes de um produto de software e assim reduzir o custo, o tempo e melhorar a qualidade do produto.
Segundo Keswani et al. \cite{6783445}, o componente reutilizável de software pode ser qualquer parte de seu desenvolvimento, como um fragmento de código, design, casos de teste, ou até mesmo a especificação de requisitos de uma funcionalidade do software. 

O reuso de software pode ter impacto positivo em diversos aspectos do software, vejamos alguns, conforme apresentados no C.R.U.I.S.E Book \citep{cruise2007} :

\begin{itemize}

\item Qualidade: As correções de erro tornam-se úteis em todos os locais em que ocorreu, padronizando e facilitando a manutenção.

\item Produtividade: O ganho de produtividade é alcançado devido ao menor número de artefatos desenvolvido. Isso resulta em menor esforço de teste e também análise e design, reduzindo custos.

\item Confiabilidade: A utilização de componentes bem testados aumenta a
confiança no software. Além disso, a utilização de um mesmo componente em vários sistemas, aumenta a possibilidade de detecção de erros e reforça a confiança no componente.

\item Redução do Esforço: A reutilização de software proporciona uma redução do tempo de desenvolvimento, o que reduz o tempo necessário para o produto ser disponibilizado no mercado para trazer rentabilidade.

\item Trabalho redundante e tempo de desenvolvimento: Desenvolver um sistema do
zero significa desenvolvimento redundante de muitos componentes, como requisitos, especificações, casos de uso, arquitetura, etc. Isso pode ser evitado quando estes estão disponíveis como componentes reutilizáveis e podem ser compartilhados, resultando em um processo de desenvolvimento otimizado.

\item Documentação: Embora a documentação seja muito importante para a
manutenção de um sistema, muitas vezes é negligenciada. A reutilização de componentes de software reduz a quantidade de documentação a ser escrita, entretanto depende da qualidade do que está escrito. Assim, apenas a estrutura do sistema e os novos artefatos desenvolvidos necessitam ser documentados.

\item Custo de manutenção: Menos defeitos e manutenções são esperados quando tem-se comprovada a qualidade dos componentes utilizados.

\item Tamanho da equipe: É comum haver casos em que a equipe de desenvolvimento sofre sobrecarga. Entretanto, dobrar o tamanho da equipe de desenvolvimento não necessariamente duplica produtividade. Se muitos componentes podem ser reutilizados, é possível desenvolver com equipes menores, levando a melhor comunicação e aumento da produtividade.

\end{itemize}

Apesar dos benefícios da reutilização de software, ela não é suficientemente aproveitada. Existem fatores que influenciam direta ou indiretamente na sua adoção. Esses fatores podem ser de aspecto gerencial, organizacional, econômico, conceitual ou técnico. Veremos a seguir alguns aspectos que podem gerar conflito com a cultura de reuso de software, segundo o C.R.U.I.S.E Book \citep{cruise2007} :
%(Sametinger, 1997). REVER

\begin{itemize}
	
\item Falta de apoio da gestão: Como a reutilização de software gera custos iniciais,
a medida pode não ser amplamente alcançada em uma organização sem o apoio de alto nível de gestão. Os gestores têm de ser informados sobre os custos iniciais e serem convencidos sobre economias futuras.

\item Gerenciamento do Projeto: Gerenciar projetos tradicionais é uma tarefa árdua, principalmente, os que praticam a reutilização de software. Utilizando a técnica em larga escala, tem-se impacto sobre todo o ciclo de vida do software.

\item Estruturas organizacionais inadequadas: As estruturas organizacionais devem
considerar diferentes necessidades que surgem quando a reutilização em larga escala está sendo adotada. Por exemplo, uma equipe particionada pode ser alocada somente para desenvolver, manter e certificar componentes reutilizáveis de software.

\item Incentivos de gestão: É comum a falta de incentivo para deixar os desenvolvedores gastarem tempo elaborando componentes do sistemas. A produtividade é muitas vezes medida apenas no tempo necessário para concluir um projeto. Assim, fazer qualquer trabalho além disso, embora benéfico para a empresa como um todo, diminui o seu sucesso. Mesmo quando os componentes reutilizáveis são utilizados, os benefícios obtidos são uma pequena fração do que poderia ser alcançado caso houvesse reutilização explícita, planejada e organizada.

\item Dificuldade de encontrar software reutilizável: Para reutilizar os componentes, devem existir formas eficientes de busca. Além disso, é importante ter um repositório bem organizado contendo componentes com um eficiente meio de acesso.

\item Não reutilização do software encontrado. O acesso fácil ao software existente
não necessariamente aumentar a reutilização. Os componentes reutilizáveis devem ser cuidadosamente especificados, projetados, implementados e documentados, pois em alguns casos, modificar e adaptar o código  pode ser mais custoso que a programação da funcionalidade necessária a partir do zero.

\item Modificação: É muito difícil encontrar um componente que funcione
exatamente da mesma maneira que queremos. Desta forma, são necessárias modificações e devem existir formas de determinar os seus efeitos sobre o componente.


\end{itemize}


A definição do reuso de software pode ser associada a este trabalho, pois ainda que como um serviço, estaremos adotando uma aplicação única para servir aos mais diversos softwares de uma empresa, por exemplo. Logo, ao adotar o projeto, pode ser interessante aplicar algumas métricas para tomar conhecimento de possíveis vantanjens como as citadas nessa seção, a exemplo de economia de tempo de desenvolvimento, e custo do projeto.


%Outra diretriz importante para a reutilização de software é reduzir o risco na criação de novos softwares. O risco tende a ser bastante reduzido se os componentes que estão sendo reutilizados têm as documentação, interfaces necessárias e devidamente testadas, fatores que contibruem para uma fácil integração.
%De acordo com Keswani et al. \citep{6783445}, para o reuso de software dar retornos apropriados, o processo deve ser sistemático e planejado. Qualquer organização que implemente a reutilização de software deve identificar os melhores métodos e estratégias de reutilização para obter a máxima produtividade. A reutilização de software ajuda a evitar software de engenharia a partir do zero, pois usa módulos de software existentes. A reutilização de software, embora seja uma tarefa difícil, especialmente para softwares antigos sem padrões de projeto, pode melhorar significativamente a produtividade e a qualidade de um produto de software. Embora a reutilização de software não seja um novo campo, ela pode dar grandes retornos em curto período de tempo.


\section{Modularização}\label{sec:modularizacao} %artigo de claudio pagina 222 introdução


%A modularidade vem desempenhando um papel predominante estágios emergentes das disciplinas de arquitetura de software [13]. Engenheiros de software consideram modularidade como princípio base na comparação entre arquiteturas alternativas  e arquitetura degeneração [9]. De fato, os engenheiros de software são incentivados a arquitecturas, baseando-se numa multiplicidade de mecanismos de modularidade disponíveis em: 
%(i) Linguagens de descrição de arquitetura (ADLs), como ACME [8], 
%(ii) catálogos de arquitetônicos [2, 13], e 
%(iii) conhecem bem princípios de alto nível, como interfaces de componentes estreitos, acoplamento arquitectónico reduzido e semelhantes.


Conforme é frisado por Wickramaarachchi e Lai \citep{7062705}, o conceito de modularização na indústria de software tem uma longa história e tem sido utilizado para melhorar o processo de desenvolvimento de software em diferentes estágios. Os principais conceitos por trás da modularização do software foram introduzidos por pesquisadores pioneiros há quarenta anos, com uma notável contribuição feita por Melvin Conway e David Parnas, que tem representação notável na engenharia de software.


Modularizar um software é um bom padrão a ser adotado. Segundo Wickramaarachchi e Lai \citep{7062705}, a modularização é importante na identificação de dependências e reduz as dificuldades diante de uma possível necessidade de grandes alterações. De uma perspectiva da engenharia de software, uma modularização geralmente tem várias vantagens, tais como: tornar a complexidade do software mais gerenciável, facilitar o trabalho paralelo e tornar o software mais maleável para acomodar o futuro incerto que um software pode ter. O objetivo final da modularização do software é aumentar a produtividade ea qualidade do software. Tal conceito encontra-se bastante difundido e estái incorporado em linguagens de programação e ferramentas de software. O presente trabalho favorece ao uso da modularização de um software e até mesmo pode ser considerado um módulo a ser acoplado a qualquer software, mediante a compatibilidade.


Ao realizar a adoção do trabalho proposto, fica bastante evidente a "responsabilidade" do mesmo no escopo do projeto. Assim, é possível tratá-lo como um módulo do projeto que o usa, mesmo que seja consumido como um serviço.


\section{Aplicações web}\label{sec:apps_web}


A popularidade das soluções Web aumentou exponencialmente na última década e todos os dias cresce o número de pessoas usuárias desse tipo de software. E seguindo um padrão próprio, Kumar et al. \citep{7813710} sugerem que para o desenvolvimento web, deve-se manter a prática eficaz de produzir diagramas UML. A abordagem baseada na web oferece uma maneira fácil e eficaz para gerenciar e controlar o processo de desenvolvimento por meio de artefatos de modelagem. Tal abordagem pode ser usada quando há uma exigência de lidar com mudanças muito rápidas e grandes em requisitos de forma muito eficaz em muito menos tempo, gerando assim um menor impacto. 


Para atender à fomentada demanda de aplicativos web, é necessário adotar métodos de desenvolvimentos que sejam ágeis, eficientes e de fácil manutenção. Yu Ping et al. \cite{1372143} propõem o uso do modelo MVC (Model, View e Controller) no desenvolvimento para softwares web. O modelo apresentado tornou-se um padrão popular e divide o software em camadas com propósito definido, tornando-o de mais fácil manutenção.


O Ajax (Asynchronous Javascript and XML) revolucionou a web. Conforme demonstrado por Yuping \citep{6845605}, ao usar a tecnologia Ajax, podemos enriquecer a experiência do usuário em aplicações baseadas em navegador de internet, e fornecer uma variedade de aplicações interativas para atender às necessidade de humanização das aplicações.
Os aplicativos Ajax em execução no navegador se comunicam com um servidor Web de forma assíncrona e atualizam apenas uma parte da página.


No artigo de Tesarik et al. \citep{4664349}, temos que o design de software SPA (Single page application) é uma maneira interessante de criar um software Web numa uma única página. Essa solução de página única sem navegação funciona apenas com base em técnicas dinâmicas e assíncronas, como o AJAX (citado no parágrafo anterior). No entanto, esta abordagem coloca o desenvolvedor antes de alguns desafios substanciais.
Para projetar a interface do usuário que mostra as informações, é importante projetar corretamente a tela para manipular os dados do aplicativo numa única página. O design da página deve ser elaborado para maximizar a decomposição da página em componentes distintos que encapsulam os principais casos de uso.
Também recomenda-se que para elaborar os artefatos visuais, explore os mais ricos recursos na implementação, como HTML5, JavaScript, Ajax, CSS3, e outras tecnologias que se apliquem. Entretanto, desenvolver uma rica interface com o uso de diversas tecnologias/frameworks pode ocasionar um esforço maior para explorar as possibilidades.
Tais componentes da interface de uma aplicação SPA normalmente são alimentados mediante o consumo de uma API Rest via requisições AJAX. Assim, é possível particionar as responsabilidades de processamento do software entre cliente e servidor.


Como é possível denotar no trabalho de Salvadori e Siqueire \citep{7195633}, REST é uma arquitetura muito popular para integração de aplicativos web, que permite compartilhar e reutilizar informações através de sistemas. Aplicações de grande escala baseadas em no REST deve ser implementadas utilizando estratégias e mecanismos para produzir sistemas que sejam fáceis de desenvolver, reutilizar e manter. As interfaces de integração fornecidas por esses sistemas, chamadas de Web API, têm uma influência importante nas características da implementação resultante, pois a sua forma de resposta deve ser casada com a interface que o consome.
 
 
Os elementos comentados nessa seção encontram-se presentes neste trabalho de conclusão de curso. Juntos, montam a estrutura tecnológica necessária aliada à arquitetura adotada. Esses elementos seguem tendências atuais dos softwares web, fazendo com que esse trabalho esteja composto por tecnologias modernas que estão com boa aceitação no mercado.

% %% RiSE Latex Template - version 0.5
%%
%% RiSE's latex template for thesis and dissertations
%% http://risetemplate.sourceforge.net
%%
%% (c) 2012 Yguaratã Cerqueira Cavalcanti (yguarata@gmail.com)
%%          Vinicius Cardoso Garcia (vinicius.garcia@gmail.com)
%%
%% This document was initially based on UFPEThesis template, from Paulo Gustavo
%% S. Fonseca.
%%
%% ACKNOWLEDGEMENTS
%%
%% We would like to thanks the RiSE's researchers community, the 
%% students from Federal University of Pernambuco, and other users that have
%% been contributing to this projects with comments and patches.
%%
%% GENERAL INSTRUCTIONS
%%
%% We strongly recommend you to compile your documents using pdflatex command.
%% It is also recommend use the texlipse plugin for Eclipse to edit your documents.
%%
%% Options for \documentclass command:
%%         * Idiom
%%           pt   - Portguese (default)
%%           en   - English
%%
%%         * Text type
%%           bsc  - B.Sc. Thesis
%%           msc  - M.Sc. Thesis (default)
%%           qual - PHD qualification (not tested yet)
%%           prop - PHD proposal (not tested yet)
%%           phd  - PHD thesis
%%
%%         * Media
%%           scr  - to eletronic version (PDF) / see the users guide
%%
%%         * Pagination
%%           oneside - unique face press
%%           twoside - two faces press
%%
%%		   * Line spacing
%%           singlespacing  - the same as using \linespread{1}
%%           onehalfspacing - the same as using \linespread{1.3}
%%           doublespacing  - the same as using \linespread{1.6}
%%
%% Reference commands. Use the following commands to make references in your
%% text:
%%          \figref  -- for Figure reference
%%          \tabref  -- for Table reference
%%          \eqnref  -- for equation reference
%%          \chapref -- for chapter reference
%%          \secref  -- for section reference
%%          \appref  -- for appendix reference
%%          \axiref  -- for axiom reference
%%          \conjref -- for conjecture reference
%%          \defref  -- for definition reference
%%          \lemref  -- for lemma reference
%%          \theoref -- for theorem reference
%%          \corref  -- for corollary reference
%%          \propref -- for proprosition reference
%%          \pgref   -- for page reference
%%
%%          Example: See \chapref{chap:introduction}. It will produce 
%%                   'See Chapter 1', in case of English language.

\documentclass[pt,twoside,onehalfspacing,bsc]{risethesis}

\usepackage[sort,square]{natbib} %package for bibliography
\bibliographystyle{ieeetr} %authordate1

\usepackage[utf8]{inputenc}
\usepackage[brazilian]{babel}
\usepackage[T1]{fontenc}

\usepackage{graphicx}

%% Change the following pdf author attribute name to your name.
\usepackage[linkcolor=blue,citecolor=blue,urlcolor=blue,colorlinks,pdfpagelabels,pdftitle={Bruno Cabral's Bachelor Thesis},pdfauthor={Bruno Cabral}]{hyperref}

\address{SALVADOR}

\universitypt{Universidade Federal da Bahia}
\universityen{Federal University of Bahia}

\departmentpt{Depertamento de Ciência da Computação}
\departmenten{Computer Science Department}

\programpt{Programa Multiinstitucional de Pós-graduação em Ciência da Computação}
\programen{Graduate in Computer Science}

\majorfieldpt{Ciência da Computação}
\majorfielden{Computer Science}

\title{Sistema de apoio à Pós graduação - UFBA}
\date{Outubro/2016}

\author{Victor de Azevedo Nunes}
\adviser{Ivan do Carmo Machado}

\begin{document}

\frontmatter
\frontpage
\presentationpage

\begin{dedicatory}
Eu dedico esta dissertação...
%I dedicate this dissertation to my family, girlfriend, friends and
%professors who gave me all necessary support to get here.
\end{dedicatory}

\acknowledgements
Meus agradecimentos...

\begin{epigraph}[]{Edward V Berard}
Walking on water and developing software from a specification are easy if both are frozen
\end{epigraph}

\resumo
% Escreva seu resumo no arquivo resumo.tex
\input{resumo}

\abstract
% Write your abstract in a file called abstract.tex
\input{abstract}

% Summary (tables of contents)
\tableofcontents

% List of figures
\listoffigures

% List of tables
\listoftables

% List of acronyms
% Acronyms manual: http://linorg.usp.br/CTAN/macros/latex/contrib/acronym/acronym.pdf
\listofacronyms
\input{acronyms}

% List of listings
%\lstlistoflistings

\mainmatter

\include{chapters/intro}
\include{chapters/referencial_teorico}

% \include{chapters/introduction/main}
% \include{chapters/background/main}
% \include{chapters/proposed_solution/main}
% \include{chapters/experiment/main}
% \include{chapters/conclusion/main}

%\bibliographystyle{natbib}
%\addcontentsline{toc}{chapter}{\bibliographytocname}
\bibliography{references}

% Appendix
\clearpage
\addappheadtotoc
\appendix
\appendixpage
% \include{appendix/experiment-instruments}

\end{document}
% %% RiSE Latex Template - version 0.5
%%
%% RiSE's latex template for thesis and dissertations
%% http://risetemplate.sourceforge.net
%%
%% (c) 2012 Yguaratã Cerqueira Cavalcanti (yguarata@gmail.com)
%%          Vinicius Cardoso Garcia (vinicius.garcia@gmail.com)
%%
%% This document was initially based on UFPEThesis template, from Paulo Gustavo
%% S. Fonseca.
%%
%% ACKNOWLEDGEMENTS
%%
%% We would like to thanks the RiSE's researchers community, the 
%% students from Federal University of Pernambuco, and other users that have
%% been contributing to this projects with comments and patches.
%%
%% GENERAL INSTRUCTIONS
%%
%% We strongly recommend you to compile your documents using pdflatex command.
%% It is also recommend use the texlipse plugin for Eclipse to edit your documents.
%%
%% Options for \documentclass command:
%%         * Idiom
%%           pt   - Portguese (default)
%%           en   - English
%%
%%         * Text type
%%           bsc  - B.Sc. Thesis
%%           msc  - M.Sc. Thesis (default)
%%           qual - PHD qualification (not tested yet)
%%           prop - PHD proposal (not tested yet)
%%           phd  - PHD thesis
%%
%%         * Media
%%           scr  - to eletronic version (PDF) / see the users guide
%%
%%         * Pagination
%%           oneside - unique face press
%%           twoside - two faces press
%%
%%		   * Line spacing
%%           singlespacing  - the same as using \linespread{1}
%%           onehalfspacing - the same as using \linespread{1.3}
%%           doublespacing  - the same as using \linespread{1.6}
%%
%% Reference commands. Use the following commands to make references in your
%% text:
%%          \figref  -- for Figure reference
%%          \tabref  -- for Table reference
%%          \eqnref  -- for equation reference
%%          \chapref -- for chapter reference
%%          \secref  -- for section reference
%%          \appref  -- for appendix reference
%%          \axiref  -- for axiom reference
%%          \conjref -- for conjecture reference
%%          \defref  -- for definition reference
%%          \lemref  -- for lemma reference
%%          \theoref -- for theorem reference
%%          \corref  -- for corollary reference
%%          \propref -- for proprosition reference
%%          \pgref   -- for page reference
%%
%%          Example: See \chapref{chap:introduction}. It will produce 
%%                   'See Chapter 1', in case of English language.

\documentclass[pt,twoside,onehalfspacing,bsc]{risethesis}

\usepackage[sort,square]{natbib} %package for bibliography
\bibliographystyle{ieeetr} %authordate1

\usepackage[utf8]{inputenc}
\usepackage[brazilian]{babel}
\usepackage[T1]{fontenc}

\usepackage{graphicx}

%% Change the following pdf author attribute name to your name.
\usepackage[linkcolor=blue,citecolor=blue,urlcolor=blue,colorlinks,pdfpagelabels,pdftitle={Bruno Cabral's Bachelor Thesis},pdfauthor={Bruno Cabral}]{hyperref}

\address{SALVADOR}

\universitypt{Universidade Federal da Bahia}
\universityen{Federal University of Bahia}

\departmentpt{Depertamento de Ciência da Computação}
\departmenten{Computer Science Department}

\programpt{Programa Multiinstitucional de Pós-graduação em Ciência da Computação}
\programen{Graduate in Computer Science}

\majorfieldpt{Ciência da Computação}
\majorfielden{Computer Science}

\title{Sistema de apoio à Pós graduação - UFBA}
\date{Outubro/2016}

\author{Victor de Azevedo Nunes}
\adviser{Ivan do Carmo Machado}

\begin{document}

\frontmatter
\frontpage
\presentationpage

\begin{dedicatory}
Eu dedico esta dissertação...
%I dedicate this dissertation to my family, girlfriend, friends and
%professors who gave me all necessary support to get here.
\end{dedicatory}

\acknowledgements
Meus agradecimentos...

\begin{epigraph}[]{Edward V Berard}
Walking on water and developing software from a specification are easy if both are frozen
\end{epigraph}

\resumo
% Escreva seu resumo no arquivo resumo.tex
\input{resumo}

\abstract
% Write your abstract in a file called abstract.tex
\input{abstract}

% Summary (tables of contents)
\tableofcontents

% List of figures
\listoffigures

% List of tables
\listoftables

% List of acronyms
% Acronyms manual: http://linorg.usp.br/CTAN/macros/latex/contrib/acronym/acronym.pdf
\listofacronyms
\input{acronyms}

% List of listings
%\lstlistoflistings

\mainmatter

\include{chapters/intro}
\include{chapters/referencial_teorico}

% \include{chapters/introduction/main}
% \include{chapters/background/main}
% \include{chapters/proposed_solution/main}
% \include{chapters/experiment/main}
% \include{chapters/conclusion/main}

%\bibliographystyle{natbib}
%\addcontentsline{toc}{chapter}{\bibliographytocname}
\bibliography{references}

% Appendix
\clearpage
\addappheadtotoc
\appendix
\appendixpage
% \include{appendix/experiment-instruments}

\end{document}
% %% RiSE Latex Template - version 0.5
%%
%% RiSE's latex template for thesis and dissertations
%% http://risetemplate.sourceforge.net
%%
%% (c) 2012 Yguaratã Cerqueira Cavalcanti (yguarata@gmail.com)
%%          Vinicius Cardoso Garcia (vinicius.garcia@gmail.com)
%%
%% This document was initially based on UFPEThesis template, from Paulo Gustavo
%% S. Fonseca.
%%
%% ACKNOWLEDGEMENTS
%%
%% We would like to thanks the RiSE's researchers community, the 
%% students from Federal University of Pernambuco, and other users that have
%% been contributing to this projects with comments and patches.
%%
%% GENERAL INSTRUCTIONS
%%
%% We strongly recommend you to compile your documents using pdflatex command.
%% It is also recommend use the texlipse plugin for Eclipse to edit your documents.
%%
%% Options for \documentclass command:
%%         * Idiom
%%           pt   - Portguese (default)
%%           en   - English
%%
%%         * Text type
%%           bsc  - B.Sc. Thesis
%%           msc  - M.Sc. Thesis (default)
%%           qual - PHD qualification (not tested yet)
%%           prop - PHD proposal (not tested yet)
%%           phd  - PHD thesis
%%
%%         * Media
%%           scr  - to eletronic version (PDF) / see the users guide
%%
%%         * Pagination
%%           oneside - unique face press
%%           twoside - two faces press
%%
%%		   * Line spacing
%%           singlespacing  - the same as using \linespread{1}
%%           onehalfspacing - the same as using \linespread{1.3}
%%           doublespacing  - the same as using \linespread{1.6}
%%
%% Reference commands. Use the following commands to make references in your
%% text:
%%          \figref  -- for Figure reference
%%          \tabref  -- for Table reference
%%          \eqnref  -- for equation reference
%%          \chapref -- for chapter reference
%%          \secref  -- for section reference
%%          \appref  -- for appendix reference
%%          \axiref  -- for axiom reference
%%          \conjref -- for conjecture reference
%%          \defref  -- for definition reference
%%          \lemref  -- for lemma reference
%%          \theoref -- for theorem reference
%%          \corref  -- for corollary reference
%%          \propref -- for proprosition reference
%%          \pgref   -- for page reference
%%
%%          Example: See \chapref{chap:introduction}. It will produce 
%%                   'See Chapter 1', in case of English language.

\documentclass[pt,twoside,onehalfspacing,bsc]{risethesis}

\usepackage[sort,square]{natbib} %package for bibliography
\bibliographystyle{ieeetr} %authordate1

\usepackage[utf8]{inputenc}
\usepackage[brazilian]{babel}
\usepackage[T1]{fontenc}

\usepackage{graphicx}

%% Change the following pdf author attribute name to your name.
\usepackage[linkcolor=blue,citecolor=blue,urlcolor=blue,colorlinks,pdfpagelabels,pdftitle={Bruno Cabral's Bachelor Thesis},pdfauthor={Bruno Cabral}]{hyperref}

\address{SALVADOR}

\universitypt{Universidade Federal da Bahia}
\universityen{Federal University of Bahia}

\departmentpt{Depertamento de Ciência da Computação}
\departmenten{Computer Science Department}

\programpt{Programa Multiinstitucional de Pós-graduação em Ciência da Computação}
\programen{Graduate in Computer Science}

\majorfieldpt{Ciência da Computação}
\majorfielden{Computer Science}

\title{Sistema de apoio à Pós graduação - UFBA}
\date{Outubro/2016}

\author{Victor de Azevedo Nunes}
\adviser{Ivan do Carmo Machado}

\begin{document}

\frontmatter
\frontpage
\presentationpage

\begin{dedicatory}
Eu dedico esta dissertação...
%I dedicate this dissertation to my family, girlfriend, friends and
%professors who gave me all necessary support to get here.
\end{dedicatory}

\acknowledgements
Meus agradecimentos...

\begin{epigraph}[]{Edward V Berard}
Walking on water and developing software from a specification are easy if both are frozen
\end{epigraph}

\resumo
% Escreva seu resumo no arquivo resumo.tex
\input{resumo}

\abstract
% Write your abstract in a file called abstract.tex
\input{abstract}

% Summary (tables of contents)
\tableofcontents

% List of figures
\listoffigures

% List of tables
\listoftables

% List of acronyms
% Acronyms manual: http://linorg.usp.br/CTAN/macros/latex/contrib/acronym/acronym.pdf
\listofacronyms
\input{acronyms}

% List of listings
%\lstlistoflistings

\mainmatter

\include{chapters/intro}
\include{chapters/referencial_teorico}

% \include{chapters/introduction/main}
% \include{chapters/background/main}
% \include{chapters/proposed_solution/main}
% \include{chapters/experiment/main}
% \include{chapters/conclusion/main}

%\bibliographystyle{natbib}
%\addcontentsline{toc}{chapter}{\bibliographytocname}
\bibliography{references}

% Appendix
\clearpage
\addappheadtotoc
\appendix
\appendixpage
% \include{appendix/experiment-instruments}

\end{document}
% %% RiSE Latex Template - version 0.5
%%
%% RiSE's latex template for thesis and dissertations
%% http://risetemplate.sourceforge.net
%%
%% (c) 2012 Yguaratã Cerqueira Cavalcanti (yguarata@gmail.com)
%%          Vinicius Cardoso Garcia (vinicius.garcia@gmail.com)
%%
%% This document was initially based on UFPEThesis template, from Paulo Gustavo
%% S. Fonseca.
%%
%% ACKNOWLEDGEMENTS
%%
%% We would like to thanks the RiSE's researchers community, the 
%% students from Federal University of Pernambuco, and other users that have
%% been contributing to this projects with comments and patches.
%%
%% GENERAL INSTRUCTIONS
%%
%% We strongly recommend you to compile your documents using pdflatex command.
%% It is also recommend use the texlipse plugin for Eclipse to edit your documents.
%%
%% Options for \documentclass command:
%%         * Idiom
%%           pt   - Portguese (default)
%%           en   - English
%%
%%         * Text type
%%           bsc  - B.Sc. Thesis
%%           msc  - M.Sc. Thesis (default)
%%           qual - PHD qualification (not tested yet)
%%           prop - PHD proposal (not tested yet)
%%           phd  - PHD thesis
%%
%%         * Media
%%           scr  - to eletronic version (PDF) / see the users guide
%%
%%         * Pagination
%%           oneside - unique face press
%%           twoside - two faces press
%%
%%		   * Line spacing
%%           singlespacing  - the same as using \linespread{1}
%%           onehalfspacing - the same as using \linespread{1.3}
%%           doublespacing  - the same as using \linespread{1.6}
%%
%% Reference commands. Use the following commands to make references in your
%% text:
%%          \figref  -- for Figure reference
%%          \tabref  -- for Table reference
%%          \eqnref  -- for equation reference
%%          \chapref -- for chapter reference
%%          \secref  -- for section reference
%%          \appref  -- for appendix reference
%%          \axiref  -- for axiom reference
%%          \conjref -- for conjecture reference
%%          \defref  -- for definition reference
%%          \lemref  -- for lemma reference
%%          \theoref -- for theorem reference
%%          \corref  -- for corollary reference
%%          \propref -- for proprosition reference
%%          \pgref   -- for page reference
%%
%%          Example: See \chapref{chap:introduction}. It will produce 
%%                   'See Chapter 1', in case of English language.

\documentclass[pt,twoside,onehalfspacing,bsc]{risethesis}

\usepackage[sort,square]{natbib} %package for bibliography
\bibliographystyle{ieeetr} %authordate1

\usepackage[utf8]{inputenc}
\usepackage[brazilian]{babel}
\usepackage[T1]{fontenc}

\usepackage{graphicx}

%% Change the following pdf author attribute name to your name.
\usepackage[linkcolor=blue,citecolor=blue,urlcolor=blue,colorlinks,pdfpagelabels,pdftitle={Bruno Cabral's Bachelor Thesis},pdfauthor={Bruno Cabral}]{hyperref}

\address{SALVADOR}

\universitypt{Universidade Federal da Bahia}
\universityen{Federal University of Bahia}

\departmentpt{Depertamento de Ciência da Computação}
\departmenten{Computer Science Department}

\programpt{Programa Multiinstitucional de Pós-graduação em Ciência da Computação}
\programen{Graduate in Computer Science}

\majorfieldpt{Ciência da Computação}
\majorfielden{Computer Science}

\title{Sistema de apoio à Pós graduação - UFBA}
\date{Outubro/2016}

\author{Victor de Azevedo Nunes}
\adviser{Ivan do Carmo Machado}

\begin{document}

\frontmatter
\frontpage
\presentationpage

\begin{dedicatory}
Eu dedico esta dissertação...
%I dedicate this dissertation to my family, girlfriend, friends and
%professors who gave me all necessary support to get here.
\end{dedicatory}

\acknowledgements
Meus agradecimentos...

\begin{epigraph}[]{Edward V Berard}
Walking on water and developing software from a specification are easy if both are frozen
\end{epigraph}

\resumo
% Escreva seu resumo no arquivo resumo.tex
\input{resumo}

\abstract
% Write your abstract in a file called abstract.tex
\input{abstract}

% Summary (tables of contents)
\tableofcontents

% List of figures
\listoffigures

% List of tables
\listoftables

% List of acronyms
% Acronyms manual: http://linorg.usp.br/CTAN/macros/latex/contrib/acronym/acronym.pdf
\listofacronyms
\input{acronyms}

% List of listings
%\lstlistoflistings

\mainmatter

\include{chapters/intro}
\include{chapters/referencial_teorico}

% \include{chapters/introduction/main}
% \include{chapters/background/main}
% \include{chapters/proposed_solution/main}
% \include{chapters/experiment/main}
% \include{chapters/conclusion/main}

%\bibliographystyle{natbib}
%\addcontentsline{toc}{chapter}{\bibliographytocname}
\bibliography{references}

% Appendix
\clearpage
\addappheadtotoc
\appendix
\appendixpage
% \include{appendix/experiment-instruments}

\end{document}
% %% RiSE Latex Template - version 0.5
%%
%% RiSE's latex template for thesis and dissertations
%% http://risetemplate.sourceforge.net
%%
%% (c) 2012 Yguaratã Cerqueira Cavalcanti (yguarata@gmail.com)
%%          Vinicius Cardoso Garcia (vinicius.garcia@gmail.com)
%%
%% This document was initially based on UFPEThesis template, from Paulo Gustavo
%% S. Fonseca.
%%
%% ACKNOWLEDGEMENTS
%%
%% We would like to thanks the RiSE's researchers community, the 
%% students from Federal University of Pernambuco, and other users that have
%% been contributing to this projects with comments and patches.
%%
%% GENERAL INSTRUCTIONS
%%
%% We strongly recommend you to compile your documents using pdflatex command.
%% It is also recommend use the texlipse plugin for Eclipse to edit your documents.
%%
%% Options for \documentclass command:
%%         * Idiom
%%           pt   - Portguese (default)
%%           en   - English
%%
%%         * Text type
%%           bsc  - B.Sc. Thesis
%%           msc  - M.Sc. Thesis (default)
%%           qual - PHD qualification (not tested yet)
%%           prop - PHD proposal (not tested yet)
%%           phd  - PHD thesis
%%
%%         * Media
%%           scr  - to eletronic version (PDF) / see the users guide
%%
%%         * Pagination
%%           oneside - unique face press
%%           twoside - two faces press
%%
%%		   * Line spacing
%%           singlespacing  - the same as using \linespread{1}
%%           onehalfspacing - the same as using \linespread{1.3}
%%           doublespacing  - the same as using \linespread{1.6}
%%
%% Reference commands. Use the following commands to make references in your
%% text:
%%          \figref  -- for Figure reference
%%          \tabref  -- for Table reference
%%          \eqnref  -- for equation reference
%%          \chapref -- for chapter reference
%%          \secref  -- for section reference
%%          \appref  -- for appendix reference
%%          \axiref  -- for axiom reference
%%          \conjref -- for conjecture reference
%%          \defref  -- for definition reference
%%          \lemref  -- for lemma reference
%%          \theoref -- for theorem reference
%%          \corref  -- for corollary reference
%%          \propref -- for proprosition reference
%%          \pgref   -- for page reference
%%
%%          Example: See \chapref{chap:introduction}. It will produce 
%%                   'See Chapter 1', in case of English language.

\documentclass[pt,twoside,onehalfspacing,bsc]{risethesis}

\usepackage[sort,square]{natbib} %package for bibliography
\bibliographystyle{ieeetr} %authordate1

\usepackage[utf8]{inputenc}
\usepackage[brazilian]{babel}
\usepackage[T1]{fontenc}

\usepackage{graphicx}

%% Change the following pdf author attribute name to your name.
\usepackage[linkcolor=blue,citecolor=blue,urlcolor=blue,colorlinks,pdfpagelabels,pdftitle={Bruno Cabral's Bachelor Thesis},pdfauthor={Bruno Cabral}]{hyperref}

\address{SALVADOR}

\universitypt{Universidade Federal da Bahia}
\universityen{Federal University of Bahia}

\departmentpt{Depertamento de Ciência da Computação}
\departmenten{Computer Science Department}

\programpt{Programa Multiinstitucional de Pós-graduação em Ciência da Computação}
\programen{Graduate in Computer Science}

\majorfieldpt{Ciência da Computação}
\majorfielden{Computer Science}

\title{Sistema de apoio à Pós graduação - UFBA}
\date{Outubro/2016}

\author{Victor de Azevedo Nunes}
\adviser{Ivan do Carmo Machado}

\begin{document}

\frontmatter
\frontpage
\presentationpage

\begin{dedicatory}
Eu dedico esta dissertação...
%I dedicate this dissertation to my family, girlfriend, friends and
%professors who gave me all necessary support to get here.
\end{dedicatory}

\acknowledgements
Meus agradecimentos...

\begin{epigraph}[]{Edward V Berard}
Walking on water and developing software from a specification are easy if both are frozen
\end{epigraph}

\resumo
% Escreva seu resumo no arquivo resumo.tex
\input{resumo}

\abstract
% Write your abstract in a file called abstract.tex
\input{abstract}

% Summary (tables of contents)
\tableofcontents

% List of figures
\listoffigures

% List of tables
\listoftables

% List of acronyms
% Acronyms manual: http://linorg.usp.br/CTAN/macros/latex/contrib/acronym/acronym.pdf
\listofacronyms
\input{acronyms}

% List of listings
%\lstlistoflistings

\mainmatter

\include{chapters/intro}
\include{chapters/referencial_teorico}

% \include{chapters/introduction/main}
% \include{chapters/background/main}
% \include{chapters/proposed_solution/main}
% \include{chapters/experiment/main}
% \include{chapters/conclusion/main}

%\bibliographystyle{natbib}
%\addcontentsline{toc}{chapter}{\bibliographytocname}
\bibliography{references}

% Appendix
\clearpage
\addappheadtotoc
\appendix
\appendixpage
% \include{appendix/experiment-instruments}

\end{document}

%\bibliographystyle{natbib}
%\addcontentsline{toc}{chapter}{\bibliographytocname}
\bibliography{references}

% Appendix
\clearpage
\addappheadtotoc
\appendix
\appendixpage
% \include{appendix/experiment-instruments}

\end{document}

%\bibliographystyle{natbib}
%\addcontentsline{toc}{chapter}{\bibliographytocname}
\bibliography{references}

% Appendix
\clearpage
\addappheadtotoc
\appendix
\appendixpage
% \include{appendix/experiment-instruments}

\end{document}
% %% RiSE Latex Template - version 0.5
%%
%% RiSE's latex template for thesis and dissertations
%% http://risetemplate.sourceforge.net
%%
%% (c) 2012 Yguaratã Cerqueira Cavalcanti (yguarata@gmail.com)
%%          Vinicius Cardoso Garcia (vinicius.garcia@gmail.com)
%%
%% This document was initially based on UFPEThesis template, from Paulo Gustavo
%% S. Fonseca.
%%
%% ACKNOWLEDGEMENTS
%%
%% We would like to thanks the RiSE's researchers community, the 
%% students from Federal University of Pernambuco, and other users that have
%% been contributing to this projects with comments and patches.
%%
%% GENERAL INSTRUCTIONS
%%
%% We strongly recommend you to compile your documents using pdflatex command.
%% It is also recommend use the texlipse plugin for Eclipse to edit your documents.
%%
%% Options for \documentclass command:
%%         * Idiom
%%           pt   - Portguese (default)
%%           en   - English
%%
%%         * Text type
%%           bsc  - B.Sc. Thesis
%%           msc  - M.Sc. Thesis (default)
%%           qual - PHD qualification (not tested yet)
%%           prop - PHD proposal (not tested yet)
%%           phd  - PHD thesis
%%
%%         * Media
%%           scr  - to eletronic version (PDF) / see the users guide
%%
%%         * Pagination
%%           oneside - unique face press
%%           twoside - two faces press
%%
%%		   * Line spacing
%%           singlespacing  - the same as using \linespread{1}
%%           onehalfspacing - the same as using \linespread{1.3}
%%           doublespacing  - the same as using \linespread{1.6}
%%
%% Reference commands. Use the following commands to make references in your
%% text:
%%          \figref  -- for Figure reference
%%          \tabref  -- for Table reference
%%          \eqnref  -- for equation reference
%%          \chapref -- for chapter reference
%%          \secref  -- for section reference
%%          \appref  -- for appendix reference
%%          \axiref  -- for axiom reference
%%          \conjref -- for conjecture reference
%%          \defref  -- for definition reference
%%          \lemref  -- for lemma reference
%%          \theoref -- for theorem reference
%%          \corref  -- for corollary reference
%%          \propref -- for proprosition reference
%%          \pgref   -- for page reference
%%
%%          Example: See \chapref{chap:introduction}. It will produce 
%%                   'See Chapter 1', in case of English language.

\documentclass[pt,twoside,onehalfspacing,bsc]{risethesis}

\usepackage[sort,square]{natbib} %package for bibliography
\bibliographystyle{ieeetr} %authordate1

\usepackage[utf8]{inputenc}
\usepackage[brazilian]{babel}
\usepackage[T1]{fontenc}

\usepackage{graphicx}

%% Change the following pdf author attribute name to your name.
\usepackage[linkcolor=blue,citecolor=blue,urlcolor=blue,colorlinks,pdfpagelabels,pdftitle={Bruno Cabral's Bachelor Thesis},pdfauthor={Bruno Cabral}]{hyperref}

\address{SALVADOR}

\universitypt{Universidade Federal da Bahia}
\universityen{Federal University of Bahia}

\departmentpt{Depertamento de Ciência da Computação}
\departmenten{Computer Science Department}

\programpt{Programa Multiinstitucional de Pós-graduação em Ciência da Computação}
\programen{Graduate in Computer Science}

\majorfieldpt{Ciência da Computação}
\majorfielden{Computer Science}

\title{Sistema de apoio à Pós graduação - UFBA}
\date{Outubro/2016}

\author{Victor de Azevedo Nunes}
\adviser{Ivan do Carmo Machado}

\begin{document}

\frontmatter
\frontpage
\presentationpage

\begin{dedicatory}
Eu dedico esta dissertação...
%I dedicate this dissertation to my family, girlfriend, friends and
%professors who gave me all necessary support to get here.
\end{dedicatory}

\acknowledgements
Meus agradecimentos...

\begin{epigraph}[]{Edward V Berard}
Walking on water and developing software from a specification are easy if both are frozen
\end{epigraph}

\resumo
% Escreva seu resumo no arquivo resumo.tex
Este trabalho de conclusão de curso utiliza elementos da engenharia de software para propor um software como serviço a fim de otimizar o processo de construção e manutenção dos softwares. Assim, o objetivo deste SaaS é gerenciar as permissões de acesso de sistemas cliente, provendo o reuso de software e padronizando as soluções. O sistema proposto neste trabalho encontra-se implementado e disponível no Github, e traz fundamentos sobre a arquitetura e tecnologias utilizadas, bem como avaliações sobre possibilidades de utilização do mesmo em diversos ambientes, como Web e mobile.

\begin{keywords}
Software; Reuso; SaaS; Web; PHP
\end{keywords}

\abstract
% Write your abstract in a file called abstract.tex
My abstract...

\begin{keywords}
key words...
\end{keywords}

% Summary (tables of contents)
\tableofcontents

% List of figures
\listoffigures

% List of tables
\listoftables

% List of acronyms
% Acronyms manual: http://linorg.usp.br/CTAN/macros/latex/contrib/acronym/acronym.pdf
\listofacronyms
\begin{acronym}[ACRONYM] 
% Change the word ACRONYM above to change the acronym column width.
% The column width is equals to the width of the word that you put.
% Read the manual about acronym package for more examples:
%   http://linorg.usp.br/CTAN/macros/latex/contrib/acronym/acronym.pdf

\acro{SPA}{Single Page Application}
\acro{JSON}{Javascript Object Notation}
\acro{PHP}{PHP: Hypertext Preprocessor}
\acro{SaaS}{Software as a Service}
\acro{ERP}{Enterprise Resource Planning}
\acro{QoS}{Quality of Service}
\acro{UML}{Unified Modeling Language}
\acro{MVC}{Model-View-Controller}
\acro{Ajax}{Asynchronous Javascript and XML}
\acro{HTML}{HyperText Markup Language}
\acro{CSS}{Cascading Style Sheets}
\acro{API}{Application Programming Interface}
\acro{DOM}{Document Object Model}
\acro{BPMN}{Business Process Model and Notation}
\acro{REST}{Representational State Transfer}

\end{acronym}

% List of listings
%\lstlistoflistings

\mainmatter

\chapter{Introdução}

\section{Motivação}

Organizar os procedimentos de um processo sempre nos traz vantagens. Apesar de no processo de implantação de um sistema, o mesmo burocratizar o processo, com o tempo temos o retorno da dedicação para a inserção dos dados. Com um certo volume de dados, é possível estruturar informações que num processo manual são difíceis de serem enxergadas. Assim, é possível depender menos das pessoas que organizam o processo, pois o legado de informações não estará mais somente na mente de alguns, mas sim documentado nos dados do sistema.

Além de colaborar na organização, também haverá uma grande colaboração no tempo gasto na gestão. Lidar com muitos papéis e confiar na mente humana para guardar informações, não é uma alternativa muito segura devido ao fato que as pessoas sempre estão sujeitas a sair do processo e levar contigo a experiência obtida. Experiência essa que faz com que os procedimentos sejam executados de forma mais eficiente. Entretanto, com um sistema inteligente, é possível auxiliar e tornar mais ágil a execução das tarefas.


\section{Problema}


De acordo com funcionários ligados ao o setor de pós graduação da UFBA, entrevistados a fim de um maior entendimento do cenário, apesar das semelhanças estruturais, a pós graduação gerida de forma diferencia da graduação. FULANO afirma que devido ao fato de não ter a mesma visibilidade, não tem acesso aos mesmos recursos de gestão acadêmica da graduação. O professores não executam somente atividades dentro da sala de aula, também tem diversas outras ocupações no setor. E muitos procedimentos realizados extra classe ainda se encontram sendo realizados de forma manual, estando mais vulnerável ao erro ou até mesmo à violação do processo. Também ocorre um grande desperdício de tempo pelos professores e gestores da área, devido ao diversos processos ainda realizados de forma manual, sem a devida documentação. Segundo FULANO, também entrevistado, esse tempo perdido implica numa redução da eficiência na sala de aula, pois o professor acaba por ter menos tempo disponível para o planejamento das atividades, o que gera impactos negativos aos alunos.


\section{Objetivos} %<o que deve ser feito/entregue>


Devido aos muitos processos sendo resolvidos de forma manual, propõe-se com solução um sistema moderno, arquitetado para ter funcionamento na web e com um módulo mobile, a fim de fornecer informações de forma rápida e eficiente para os professores através de notificações, já que o acesso à internet móvel é comum entre os possíveis usuários do sistema em questão.
O principal requisito para o sistema seria dispor recursos para reduzir o tempo desperdiçado pelos professores durante as atividades extra classe.


\section{Metodologia} %<como será feito | como resolver o problema apontado inicialmente>


%<analise de literatura | design | implementação | validação>
Baseando-se nas tecnologias gratuitas em alta no cenário atual do desenvolvimento web, dispomos de algumas opções eficientes para a implementação da solução. Dentre as possibilidades, considerando a facilidade para futura manutenção e continuidade do projeto, tende-se a optar por uma tecnologia popular. Como linguagem de programação, adota-se o PHP. A escolha é fundamentada de acordo com a pesquisa da RedMonk de 2015, que evidencia o uso das linguagens de programação de acordo com as discussões no StackOverflow e repositórios no GitHub. É possível constatar a popularidade do PHP no cenário atual com o gráfico da pesquisa citada, na qual o PHP é apresentado na terceira colocação, apenas atrás do lider JavaScript e do segundo colocado, o Java.

\begin{figure}
	\label{fig:graficoRedmonk}
	\includegraphics[width=1\textwidth]{img/grafico_redmonk}
	\caption{Ranking das liguagens de programação no Stack Overflow e Github}
\end{figure}


Ainda assim, para compor a interface do dado projeto, também ocorrerá o uso do líder JavaScript de forma intensa, provendo o elo com o as informações gerenciadas pelo PHP.


Entretanto, não seria inteligente desenvolver um sistema completo sem o auxílio de um framework. Dentre os frameworks disponíveis para PHP, hoje o destaque está com o Laravel, que se encontra no topo dentre os mais utilizados no momento. 


A WebHostFace, uma empresa de hospedagem, compilou várias estatísticas para criar um infográfico mostrando os frameworks PHP mais populares de 2015. Utilizando informações sobre os próprios clientes, o Google Trends, estatísticas de repositórios do GitHub e a pesquisa do SitePoint “Best PHP Frameworks 2015”, a WebHostFace elaborou o seguinte infográfico: 

\begin{figure}
	\label{fig:graficoWebhostface}
	\includegraphics[width=1\textwidth]{img/infografico_webhostface}
	\caption{Infográfico da WebhostFace, exibindo a popularidade dos Frameworks PHP em 2015}
\end{figure}

Assim, tem-se a evidência que o Laravel em 2015 teve a maior popularidade em projetos pessoais e tem a maior comunidade entre os concorrentes, o que o torna uma boa escolha para a escrita de um software que será continuado por terceiros.


Para elaborar os recursos de interface e integrar ao back-end PHP do sistema, será adotado o já conhecido AngularJS, ferramenta sólida e conhecida no aspecto em questão. 


Dados coletados via Google Trends, que propõe comparações entre termos pesquisados, revela a popularidade do AngularJs diante de alguns dos principais concorrentes. O gráfico abaixo evidencia o cenário.


%Como mostra a Figura \ref{fig:graficoGoogleTrendsFerramentasFront}. 
\begin{figure}
	\label{fig:graficoGoogleTrendsFerramentasFront}
	\includegraphics[width=1\textwidth]{img/grafico_ferramentas_front}
	\caption{Gráfico do Google Trends exibindo as pesquisas por ferramentas front-end}
\end{figure}


Junto ao Angular JS, será utilizada a agradável tendência de interface do Material Design da Google, que propõe layouts limpos e otimizados já conhecidos pelos usuários de smartphones Android. 


Para a elaboração da plataforma mobile do projeto, será utilizado o Ionic Framework, muito difundido e bastante pesquisado na área, o que fica evidenciado com o gráfico de pesquisbaixo, coletado via Google Trends buscando por frameworks de desenvolvimento híbrido mobile.


\begin{figure}
	\label{fig:graficoGoogleTrendsFerramentasHibridasMobile}
	\includegraphics[width=1\textwidth]{img/grafico_ferramentas_hibridas_mobile}
	\caption{Gráfico do Google Trends exibindo as pesquisas por Frameworks híbridos mobile}
\end{figure}	

Para layout da interface mobile, também será aplicado a tendência do Material Design, a fim de propor uma harmonia entre o módulo web e mobile para os usuários


\section{Resultados Esperados}


Como fruto de um sistema para pós-graduação da UFBA, espera-se que os professores tenham mais recursos para integrar as atividades e também prover melhores condições para acompanhamento da vida acadêmica dos alunos em questão. Também, que os novos colaboradores que entrarem no processo tenham facilidade de compreender o fluxo do setor ao navegar pelo sistema proposto.


\section{Fora de Escopo}


Interação com os alunos devido às complicações para realizar a integração com o sistema empregado na UFBA, gerenciado pela XXXXXX, o que causaria uma inviabilidade no projeto devido à necessidade de entrega do produto ser mais forte que o tempo necessário para executar o processo de obtenção de acesso ao sistema legado para realizar a integração.


\section{Estrutura do Trabalho}


<breve resumo sobre os capítulos do TCC>
\chapter{Referencial Teórico}


Projetar o desenvolvimento de um software requer muito planejamento, pois as falhas iniciais podem custar bastante caro ou até mesmo inviabilizar a continuação de um projeto. Assim, a escolha da arquitetura ideal para a aplicabilidade é essencial na concepção de um produto de software. 
De todo o modo, sempre busca-se fazer mais com menos. Diante de tal filosofia, temos neste capítulo, uma breve discussão sobre alguns elementos de projeto e arquitetura de software, a fim de contextualizar este trabalho de conclusão de curso.
O capítulo corrente é composto por quatro seções. A \ref{sec:saas} trata de Software como serviço, discutindo alguns elementos do contexto que são relevantes para o trabalho proposto. A \ref{sec:reuso}  discute sobre a empregabilidade do reuso de software. A \ref{sec:modularizacao} seção, trata sobre aspectos envolvidos na modularização dos softwares. Por fim, a \ref{sec:apps_web} aborda as aplicações web, discutindo sobre aspectos relevantes sobre a aplicação web que compõe este trabalho.


 \section{Software como serviço}\label{sec:saas}


Segundo La e Chun \citep{La2009Systematic}, o princípio da definição de Software como um Serviço (Sofware as a Service - SaaS) é um serviço complementar para aplicações da computação em nuvem (cloud computing). No entanto, as áreas não se confundem. SaaS deve ser entendido como um mecanismo de suporte às soluções existentes na cloud. Os SaaS existem justamente para maximizar o reuso de serviços repetidos e não centrais em uma aplicação remota.


Como vantagens, diversos fatores podem ser favoráveis para a adoção de um SaaS, como custo e manutenção dentre outros fatores aplicáveis a soluções específicas. Lechesaet al. \citep{LechesaSS11}, quantificam índices sobre os fatores determinantes para adoção ou não de um SaaS voltado para ERP na África do Sul. Os principais fatores determinantes para adoção desse mecanismo de software são sua fluidez quanto à rede e a segurança. Esses fatores estão presentes na aplicação desenvolvida neste trabalho de conclusão de curso.


Devido ao fato de ter um serviço constantemente na nuvem, fica o questionamento sobre a segurança da informação manipulada. Sabe-se que a vulnerabilidade na Web não é restrita ao SaaS, atingindo diversos âmbitos. Rai et al. \cite{journals/corr/RaiSM13} consideram que o avanço da computação em nuvem não é um problema apenas para os serviços Web do ponto de vista da segurança, pois muitos trabalhos na literatura mostram a área como mais um ponto de vulnerabilidade para diversos setores, a exemplo de infraestrutura. No mesmo artigo mencionado de Rai et al. \cite{journals/corr/RaiSM13}, também realizaram-se estudos exploratórios junto a empresas usuárias de serviços em computação em nuvem e consideram que a perspectiva de SaaS também pode fortalecer a segurança nas aplicações de cloud computing, pois o software de autenticação compartilhado por várias aplicações em nuvem, oferece uma melhor padronização e consequente facilidade de prevenção a erros de vulnerabilidade específicas de cada módulo da pesquisa. Esse ponto de vista é fundamenta para a compreensão de SaaS.


A arquitetura de armazenamento de dados de um Saas pode variar de acordo com a necessidade do contexto. Huixin \cite{7586486} descreve possíveis modelagens nesse sentido. Essa abordagem pode ser com um banco de dados único, fazendo com que diferentes clientes compartilhem o mesmo banco, diferindo os dados através de controle de usuário, ou isolando os diferentes clientes através de bancos de dados exclusivos para cada um. Esse fator também pode ser combinado com a arquitetura da aplicação, caso ofereça aplicação única para todos os clientes ou aplicação compartilhada. Diante das possíveis abordagens, a modelagem de dados do software pode ser decidida pela regra de negócio. Este trabalho optou por aplicação única e banco de dados compartilhado.

Devido ao conceito particular de obtenção de software, tanto pela visão do cliente como pela visão do vendedor, é necessário compreender os diversos aspectos que podem ser relevantes ao orçar um Saas. O recente trabalho de T. Kaur et al. \citep{6949281} orienta um modelo para compor o custo de um Saas. O custo total seria composto pelos fatores que dão suporte ao funcionamento do software. Tais fatores incluem infra-estrutura, configurabilidade, customização, parâmetros de QoS (Quality of service) como escalabilidade, disponibilidade, usabilidade, pontualidade e desempenho da resposta, portabilidade, custo total de propriedade e retorno do investimento. Esses fatores caracterizam o custo de forma eficaz, possibilitando ao fornecedor, prover um Serviço de acordo com a exigência do consumidor em vários pacotes de serviços.


O conceito de software como serviço se aplica neste trabalho de conclusão de curso, pois o mesmo estará disponível na web com alta disponibilidade adotando as características apresentadas para qualquer pessoa que desejar usá-lo. Assim, basta se cadastrar, configurá-lo e usar como um serviço, sem preocupação com a manutenção do mesmo.


\section{Reuso de software}\label{sec:reuso} %CRUISE BOOK CAPITULO 2


De acordo com o livro Software pratical reuse \citep{Ezran2002}, o reuso de software é a utilização de qualquer informação que um desenvolvedor pode necessitar no processo de criação de software. O livro de Basili e Rombach \citep{Basili1991} define reutilização de software como o uso de tudo o que está associado a um projeto de conhecimento.
Assim, o objetivo da reutilização de software é reciclar o design, código e outros componentes de um produto de software e assim reduzir o custo, o tempo e melhorar a qualidade do produto.
Segundo Keswani et al. \cite{6783445}, o componente reutilizável de software pode ser qualquer parte de seu desenvolvimento, como um fragmento de código, design, casos de teste, ou até mesmo a especificação de requisitos de uma funcionalidade do software. 

O reuso de software pode ter impacto positivo em diversos aspectos do software, vejamos alguns, conforme apresentados no C.R.U.I.S.E Book \citep{cruise2007} :

\begin{itemize}

\item Qualidade: As correções de erro tornam-se úteis em todos os locais em que ocorreu, padronizando e facilitando a manutenção.

\item Produtividade: O ganho de produtividade é alcançado devido ao menor número de artefatos desenvolvido. Isso resulta em menor esforço de teste e também análise e design, reduzindo custos.

\item Confiabilidade: A utilização de componentes bem testados aumenta a
confiança no software. Além disso, a utilização de um mesmo componente em vários sistemas, aumenta a possibilidade de detecção de erros e reforça a confiança no componente.

\item Redução do Esforço: A reutilização de software proporciona uma redução do tempo de desenvolvimento, o que reduz o tempo necessário para o produto ser disponibilizado no mercado para trazer rentabilidade.

\item Trabalho redundante e tempo de desenvolvimento: Desenvolver um sistema do
zero significa desenvolvimento redundante de muitos componentes, como requisitos, especificações, casos de uso, arquitetura, etc. Isso pode ser evitado quando estes estão disponíveis como componentes reutilizáveis e podem ser compartilhados, resultando em um processo de desenvolvimento otimizado.

\item Documentação: Embora a documentação seja muito importante para a
manutenção de um sistema, muitas vezes é negligenciada. A reutilização de componentes de software reduz a quantidade de documentação a ser escrita, entretanto depende da qualidade do que está escrito. Assim, apenas a estrutura do sistema e os novos artefatos desenvolvidos necessitam ser documentados.

\item Custo de manutenção: Menos defeitos e manutenções são esperados quando tem-se comprovada a qualidade dos componentes utilizados.

\item Tamanho da equipe: É comum haver casos em que a equipe de desenvolvimento sofre sobrecarga. Entretanto, dobrar o tamanho da equipe de desenvolvimento não necessariamente duplica produtividade. Se muitos componentes podem ser reutilizados, é possível desenvolver com equipes menores, levando a melhor comunicação e aumento da produtividade.

\end{itemize}

Apesar dos benefícios da reutilização de software, ela não é suficientemente aproveitada. Existem fatores que influenciam direta ou indiretamente na sua adoção. Esses fatores podem ser de aspecto gerencial, organizacional, econômico, conceitual ou técnico. Veremos a seguir alguns aspectos que podem gerar conflito com a cultura de reuso de software, segundo o C.R.U.I.S.E Book \citep{cruise2007} :
%(Sametinger, 1997). REVER

\begin{itemize}
	
\item Falta de apoio da gestão: Como a reutilização de software gera custos iniciais,
a medida pode não ser amplamente alcançada em uma organização sem o apoio de alto nível de gestão. Os gestores têm de ser informados sobre os custos iniciais e serem convencidos sobre economias futuras.

\item Gerenciamento do Projeto: Gerenciar projetos tradicionais é uma tarefa árdua, principalmente, os que praticam a reutilização de software. Utilizando a técnica em larga escala, tem-se impacto sobre todo o ciclo de vida do software.

\item Estruturas organizacionais inadequadas: As estruturas organizacionais devem
considerar diferentes necessidades que surgem quando a reutilização em larga escala está sendo adotada. Por exemplo, uma equipe particionada pode ser alocada somente para desenvolver, manter e certificar componentes reutilizáveis de software.

\item Incentivos de gestão: É comum a falta de incentivo para deixar os desenvolvedores gastarem tempo elaborando componentes do sistemas. A produtividade é muitas vezes medida apenas no tempo necessário para concluir um projeto. Assim, fazer qualquer trabalho além disso, embora benéfico para a empresa como um todo, diminui o seu sucesso. Mesmo quando os componentes reutilizáveis são utilizados, os benefícios obtidos são uma pequena fração do que poderia ser alcançado caso houvesse reutilização explícita, planejada e organizada.

\item Dificuldade de encontrar software reutilizável: Para reutilizar os componentes, devem existir formas eficientes de busca. Além disso, é importante ter um repositório bem organizado contendo componentes com um eficiente meio de acesso.

\item Não reutilização do software encontrado. O acesso fácil ao software existente
não necessariamente aumentar a reutilização. Os componentes reutilizáveis devem ser cuidadosamente especificados, projetados, implementados e documentados, pois em alguns casos, modificar e adaptar o código  pode ser mais custoso que a programação da funcionalidade necessária a partir do zero.

\item Modificação: É muito difícil encontrar um componente que funcione
exatamente da mesma maneira que queremos. Desta forma, são necessárias modificações e devem existir formas de determinar os seus efeitos sobre o componente.


\end{itemize}


A definição do reuso de software pode ser associada a este trabalho, pois ainda que como um serviço, estaremos adotando uma aplicação única para servir aos mais diversos softwares de uma empresa, por exemplo. Logo, ao adotar o projeto, pode ser interessante aplicar algumas métricas para tomar conhecimento de possíveis vantanjens como as citadas nessa seção, a exemplo de economia de tempo de desenvolvimento, e custo do projeto.


%Outra diretriz importante para a reutilização de software é reduzir o risco na criação de novos softwares. O risco tende a ser bastante reduzido se os componentes que estão sendo reutilizados têm as documentação, interfaces necessárias e devidamente testadas, fatores que contibruem para uma fácil integração.
%De acordo com Keswani et al. \citep{6783445}, para o reuso de software dar retornos apropriados, o processo deve ser sistemático e planejado. Qualquer organização que implemente a reutilização de software deve identificar os melhores métodos e estratégias de reutilização para obter a máxima produtividade. A reutilização de software ajuda a evitar software de engenharia a partir do zero, pois usa módulos de software existentes. A reutilização de software, embora seja uma tarefa difícil, especialmente para softwares antigos sem padrões de projeto, pode melhorar significativamente a produtividade e a qualidade de um produto de software. Embora a reutilização de software não seja um novo campo, ela pode dar grandes retornos em curto período de tempo.


\section{Modularização}\label{sec:modularizacao} %artigo de claudio pagina 222 introdução


%A modularidade vem desempenhando um papel predominante estágios emergentes das disciplinas de arquitetura de software [13]. Engenheiros de software consideram modularidade como princípio base na comparação entre arquiteturas alternativas  e arquitetura degeneração [9]. De fato, os engenheiros de software são incentivados a arquitecturas, baseando-se numa multiplicidade de mecanismos de modularidade disponíveis em: 
%(i) Linguagens de descrição de arquitetura (ADLs), como ACME [8], 
%(ii) catálogos de arquitetônicos [2, 13], e 
%(iii) conhecem bem princípios de alto nível, como interfaces de componentes estreitos, acoplamento arquitectónico reduzido e semelhantes.


Conforme é frisado por Wickramaarachchi e Lai \citep{7062705}, o conceito de modularização na indústria de software tem uma longa história e tem sido utilizado para melhorar o processo de desenvolvimento de software em diferentes estágios. Os principais conceitos por trás da modularização do software foram introduzidos por pesquisadores pioneiros há quarenta anos, com uma notável contribuição feita por Melvin Conway e David Parnas, que tem representação notável na engenharia de software.


Modularizar um software é um bom padrão a ser adotado. Segundo Wickramaarachchi e Lai \citep{7062705}, a modularização é importante na identificação de dependências e reduz as dificuldades diante de uma possível necessidade de grandes alterações. De uma perspectiva da engenharia de software, uma modularização geralmente tem várias vantagens, tais como: tornar a complexidade do software mais gerenciável, facilitar o trabalho paralelo e tornar o software mais maleável para acomodar o futuro incerto que um software pode ter. O objetivo final da modularização do software é aumentar a produtividade ea qualidade do software. Tal conceito encontra-se bastante difundido e estái incorporado em linguagens de programação e ferramentas de software. O presente trabalho favorece ao uso da modularização de um software e até mesmo pode ser considerado um módulo a ser acoplado a qualquer software, mediante a compatibilidade.


Ao realizar a adoção do trabalho proposto, fica bastante evidente a "responsabilidade" do mesmo no escopo do projeto. Assim, é possível tratá-lo como um módulo do projeto que o usa, mesmo que seja consumido como um serviço.


\section{Aplicações web}\label{sec:apps_web}


A popularidade das soluções Web aumentou exponencialmente na última década e todos os dias cresce o número de pessoas usuárias desse tipo de software. E seguindo um padrão próprio, Kumar et al. \citep{7813710} sugerem que para o desenvolvimento web, deve-se manter a prática eficaz de produzir diagramas UML. A abordagem baseada na web oferece uma maneira fácil e eficaz para gerenciar e controlar o processo de desenvolvimento por meio de artefatos de modelagem. Tal abordagem pode ser usada quando há uma exigência de lidar com mudanças muito rápidas e grandes em requisitos de forma muito eficaz em muito menos tempo, gerando assim um menor impacto. 


Para atender à fomentada demanda de aplicativos web, é necessário adotar métodos de desenvolvimentos que sejam ágeis, eficientes e de fácil manutenção. Yu Ping et al. \cite{1372143} propõem o uso do modelo MVC (Model, View e Controller) no desenvolvimento para softwares web. O modelo apresentado tornou-se um padrão popular e divide o software em camadas com propósito definido, tornando-o de mais fácil manutenção.


O Ajax (Asynchronous Javascript and XML) revolucionou a web. Conforme demonstrado por Yuping \citep{6845605}, ao usar a tecnologia Ajax, podemos enriquecer a experiência do usuário em aplicações baseadas em navegador de internet, e fornecer uma variedade de aplicações interativas para atender às necessidade de humanização das aplicações.
Os aplicativos Ajax em execução no navegador se comunicam com um servidor Web de forma assíncrona e atualizam apenas uma parte da página.


No artigo de Tesarik et al. \citep{4664349}, temos que o design de software SPA (Single page application) é uma maneira interessante de criar um software Web numa uma única página. Essa solução de página única sem navegação funciona apenas com base em técnicas dinâmicas e assíncronas, como o AJAX (citado no parágrafo anterior). No entanto, esta abordagem coloca o desenvolvedor antes de alguns desafios substanciais.
Para projetar a interface do usuário que mostra as informações, é importante projetar corretamente a tela para manipular os dados do aplicativo numa única página. O design da página deve ser elaborado para maximizar a decomposição da página em componentes distintos que encapsulam os principais casos de uso.
Também recomenda-se que para elaborar os artefatos visuais, explore os mais ricos recursos na implementação, como HTML5, JavaScript, Ajax, CSS3, e outras tecnologias que se apliquem. Entretanto, desenvolver uma rica interface com o uso de diversas tecnologias/frameworks pode ocasionar um esforço maior para explorar as possibilidades.
Tais componentes da interface de uma aplicação SPA normalmente são alimentados mediante o consumo de uma API Rest via requisições AJAX. Assim, é possível particionar as responsabilidades de processamento do software entre cliente e servidor.


Como é possível denotar no trabalho de Salvadori e Siqueire \citep{7195633}, REST é uma arquitetura muito popular para integração de aplicativos web, que permite compartilhar e reutilizar informações através de sistemas. Aplicações de grande escala baseadas em no REST deve ser implementadas utilizando estratégias e mecanismos para produzir sistemas que sejam fáceis de desenvolver, reutilizar e manter. As interfaces de integração fornecidas por esses sistemas, chamadas de Web API, têm uma influência importante nas características da implementação resultante, pois a sua forma de resposta deve ser casada com a interface que o consome.
 
 
Os elementos comentados nessa seção encontram-se presentes neste trabalho de conclusão de curso. Juntos, montam a estrutura tecnológica necessária aliada à arquitetura adotada. Esses elementos seguem tendências atuais dos softwares web, fazendo com que esse trabalho esteja composto por tecnologias modernas que estão com boa aceitação no mercado.

% %% RiSE Latex Template - version 0.5
%%
%% RiSE's latex template for thesis and dissertations
%% http://risetemplate.sourceforge.net
%%
%% (c) 2012 Yguaratã Cerqueira Cavalcanti (yguarata@gmail.com)
%%          Vinicius Cardoso Garcia (vinicius.garcia@gmail.com)
%%
%% This document was initially based on UFPEThesis template, from Paulo Gustavo
%% S. Fonseca.
%%
%% ACKNOWLEDGEMENTS
%%
%% We would like to thanks the RiSE's researchers community, the 
%% students from Federal University of Pernambuco, and other users that have
%% been contributing to this projects with comments and patches.
%%
%% GENERAL INSTRUCTIONS
%%
%% We strongly recommend you to compile your documents using pdflatex command.
%% It is also recommend use the texlipse plugin for Eclipse to edit your documents.
%%
%% Options for \documentclass command:
%%         * Idiom
%%           pt   - Portguese (default)
%%           en   - English
%%
%%         * Text type
%%           bsc  - B.Sc. Thesis
%%           msc  - M.Sc. Thesis (default)
%%           qual - PHD qualification (not tested yet)
%%           prop - PHD proposal (not tested yet)
%%           phd  - PHD thesis
%%
%%         * Media
%%           scr  - to eletronic version (PDF) / see the users guide
%%
%%         * Pagination
%%           oneside - unique face press
%%           twoside - two faces press
%%
%%		   * Line spacing
%%           singlespacing  - the same as using \linespread{1}
%%           onehalfspacing - the same as using \linespread{1.3}
%%           doublespacing  - the same as using \linespread{1.6}
%%
%% Reference commands. Use the following commands to make references in your
%% text:
%%          \figref  -- for Figure reference
%%          \tabref  -- for Table reference
%%          \eqnref  -- for equation reference
%%          \chapref -- for chapter reference
%%          \secref  -- for section reference
%%          \appref  -- for appendix reference
%%          \axiref  -- for axiom reference
%%          \conjref -- for conjecture reference
%%          \defref  -- for definition reference
%%          \lemref  -- for lemma reference
%%          \theoref -- for theorem reference
%%          \corref  -- for corollary reference
%%          \propref -- for proprosition reference
%%          \pgref   -- for page reference
%%
%%          Example: See \chapref{chap:introduction}. It will produce 
%%                   'See Chapter 1', in case of English language.

\documentclass[pt,twoside,onehalfspacing,bsc]{risethesis}

\usepackage[sort,square]{natbib} %package for bibliography
\bibliographystyle{ieeetr} %authordate1

\usepackage[utf8]{inputenc}
\usepackage[brazilian]{babel}
\usepackage[T1]{fontenc}

\usepackage{graphicx}

%% Change the following pdf author attribute name to your name.
\usepackage[linkcolor=blue,citecolor=blue,urlcolor=blue,colorlinks,pdfpagelabels,pdftitle={Bruno Cabral's Bachelor Thesis},pdfauthor={Bruno Cabral}]{hyperref}

\address{SALVADOR}

\universitypt{Universidade Federal da Bahia}
\universityen{Federal University of Bahia}

\departmentpt{Depertamento de Ciência da Computação}
\departmenten{Computer Science Department}

\programpt{Programa Multiinstitucional de Pós-graduação em Ciência da Computação}
\programen{Graduate in Computer Science}

\majorfieldpt{Ciência da Computação}
\majorfielden{Computer Science}

\title{Sistema de apoio à Pós graduação - UFBA}
\date{Outubro/2016}

\author{Victor de Azevedo Nunes}
\adviser{Ivan do Carmo Machado}

\begin{document}

\frontmatter
\frontpage
\presentationpage

\begin{dedicatory}
Eu dedico esta dissertação...
%I dedicate this dissertation to my family, girlfriend, friends and
%professors who gave me all necessary support to get here.
\end{dedicatory}

\acknowledgements
Meus agradecimentos...

\begin{epigraph}[]{Edward V Berard}
Walking on water and developing software from a specification are easy if both are frozen
\end{epigraph}

\resumo
% Escreva seu resumo no arquivo resumo.tex
Este trabalho de conclusão de curso utiliza elementos da engenharia de software para propor um software como serviço a fim de otimizar o processo de construção e manutenção dos softwares. Assim, o objetivo deste SaaS é gerenciar as permissões de acesso de sistemas cliente, provendo o reuso de software e padronizando as soluções. O sistema proposto neste trabalho encontra-se implementado e disponível no Github, e traz fundamentos sobre a arquitetura e tecnologias utilizadas, bem como avaliações sobre possibilidades de utilização do mesmo em diversos ambientes, como Web e mobile.

\begin{keywords}
Software; Reuso; SaaS; Web; PHP
\end{keywords}

\abstract
% Write your abstract in a file called abstract.tex
My abstract...

\begin{keywords}
key words...
\end{keywords}

% Summary (tables of contents)
\tableofcontents

% List of figures
\listoffigures

% List of tables
\listoftables

% List of acronyms
% Acronyms manual: http://linorg.usp.br/CTAN/macros/latex/contrib/acronym/acronym.pdf
\listofacronyms
\begin{acronym}[ACRONYM] 
% Change the word ACRONYM above to change the acronym column width.
% The column width is equals to the width of the word that you put.
% Read the manual about acronym package for more examples:
%   http://linorg.usp.br/CTAN/macros/latex/contrib/acronym/acronym.pdf

\acro{SPA}{Single Page Application}
\acro{JSON}{Javascript Object Notation}
\acro{PHP}{PHP: Hypertext Preprocessor}
\acro{SaaS}{Software as a Service}
\acro{ERP}{Enterprise Resource Planning}
\acro{QoS}{Quality of Service}
\acro{UML}{Unified Modeling Language}
\acro{MVC}{Model-View-Controller}
\acro{Ajax}{Asynchronous Javascript and XML}
\acro{HTML}{HyperText Markup Language}
\acro{CSS}{Cascading Style Sheets}
\acro{API}{Application Programming Interface}
\acro{DOM}{Document Object Model}
\acro{BPMN}{Business Process Model and Notation}
\acro{REST}{Representational State Transfer}

\end{acronym}

% List of listings
%\lstlistoflistings

\mainmatter

\chapter{Introdução}

\section{Motivação}

Organizar os procedimentos de um processo sempre nos traz vantagens. Apesar de no processo de implantação de um sistema, o mesmo burocratizar o processo, com o tempo temos o retorno da dedicação para a inserção dos dados. Com um certo volume de dados, é possível estruturar informações que num processo manual são difíceis de serem enxergadas. Assim, é possível depender menos das pessoas que organizam o processo, pois o legado de informações não estará mais somente na mente de alguns, mas sim documentado nos dados do sistema.

Além de colaborar na organização, também haverá uma grande colaboração no tempo gasto na gestão. Lidar com muitos papéis e confiar na mente humana para guardar informações, não é uma alternativa muito segura devido ao fato que as pessoas sempre estão sujeitas a sair do processo e levar contigo a experiência obtida. Experiência essa que faz com que os procedimentos sejam executados de forma mais eficiente. Entretanto, com um sistema inteligente, é possível auxiliar e tornar mais ágil a execução das tarefas.


\section{Problema}


De acordo com funcionários ligados ao o setor de pós graduação da UFBA, entrevistados a fim de um maior entendimento do cenário, apesar das semelhanças estruturais, a pós graduação gerida de forma diferencia da graduação. FULANO afirma que devido ao fato de não ter a mesma visibilidade, não tem acesso aos mesmos recursos de gestão acadêmica da graduação. O professores não executam somente atividades dentro da sala de aula, também tem diversas outras ocupações no setor. E muitos procedimentos realizados extra classe ainda se encontram sendo realizados de forma manual, estando mais vulnerável ao erro ou até mesmo à violação do processo. Também ocorre um grande desperdício de tempo pelos professores e gestores da área, devido ao diversos processos ainda realizados de forma manual, sem a devida documentação. Segundo FULANO, também entrevistado, esse tempo perdido implica numa redução da eficiência na sala de aula, pois o professor acaba por ter menos tempo disponível para o planejamento das atividades, o que gera impactos negativos aos alunos.


\section{Objetivos} %<o que deve ser feito/entregue>


Devido aos muitos processos sendo resolvidos de forma manual, propõe-se com solução um sistema moderno, arquitetado para ter funcionamento na web e com um módulo mobile, a fim de fornecer informações de forma rápida e eficiente para os professores através de notificações, já que o acesso à internet móvel é comum entre os possíveis usuários do sistema em questão.
O principal requisito para o sistema seria dispor recursos para reduzir o tempo desperdiçado pelos professores durante as atividades extra classe.


\section{Metodologia} %<como será feito | como resolver o problema apontado inicialmente>


%<analise de literatura | design | implementação | validação>
Baseando-se nas tecnologias gratuitas em alta no cenário atual do desenvolvimento web, dispomos de algumas opções eficientes para a implementação da solução. Dentre as possibilidades, considerando a facilidade para futura manutenção e continuidade do projeto, tende-se a optar por uma tecnologia popular. Como linguagem de programação, adota-se o PHP. A escolha é fundamentada de acordo com a pesquisa da RedMonk de 2015, que evidencia o uso das linguagens de programação de acordo com as discussões no StackOverflow e repositórios no GitHub. É possível constatar a popularidade do PHP no cenário atual com o gráfico da pesquisa citada, na qual o PHP é apresentado na terceira colocação, apenas atrás do lider JavaScript e do segundo colocado, o Java.

\begin{figure}
	\label{fig:graficoRedmonk}
	\includegraphics[width=1\textwidth]{img/grafico_redmonk}
	\caption{Ranking das liguagens de programação no Stack Overflow e Github}
\end{figure}


Ainda assim, para compor a interface do dado projeto, também ocorrerá o uso do líder JavaScript de forma intensa, provendo o elo com o as informações gerenciadas pelo PHP.


Entretanto, não seria inteligente desenvolver um sistema completo sem o auxílio de um framework. Dentre os frameworks disponíveis para PHP, hoje o destaque está com o Laravel, que se encontra no topo dentre os mais utilizados no momento. 


A WebHostFace, uma empresa de hospedagem, compilou várias estatísticas para criar um infográfico mostrando os frameworks PHP mais populares de 2015. Utilizando informações sobre os próprios clientes, o Google Trends, estatísticas de repositórios do GitHub e a pesquisa do SitePoint “Best PHP Frameworks 2015”, a WebHostFace elaborou o seguinte infográfico: 

\begin{figure}
	\label{fig:graficoWebhostface}
	\includegraphics[width=1\textwidth]{img/infografico_webhostface}
	\caption{Infográfico da WebhostFace, exibindo a popularidade dos Frameworks PHP em 2015}
\end{figure}

Assim, tem-se a evidência que o Laravel em 2015 teve a maior popularidade em projetos pessoais e tem a maior comunidade entre os concorrentes, o que o torna uma boa escolha para a escrita de um software que será continuado por terceiros.


Para elaborar os recursos de interface e integrar ao back-end PHP do sistema, será adotado o já conhecido AngularJS, ferramenta sólida e conhecida no aspecto em questão. 


Dados coletados via Google Trends, que propõe comparações entre termos pesquisados, revela a popularidade do AngularJs diante de alguns dos principais concorrentes. O gráfico abaixo evidencia o cenário.


%Como mostra a Figura \ref{fig:graficoGoogleTrendsFerramentasFront}. 
\begin{figure}
	\label{fig:graficoGoogleTrendsFerramentasFront}
	\includegraphics[width=1\textwidth]{img/grafico_ferramentas_front}
	\caption{Gráfico do Google Trends exibindo as pesquisas por ferramentas front-end}
\end{figure}


Junto ao Angular JS, será utilizada a agradável tendência de interface do Material Design da Google, que propõe layouts limpos e otimizados já conhecidos pelos usuários de smartphones Android. 


Para a elaboração da plataforma mobile do projeto, será utilizado o Ionic Framework, muito difundido e bastante pesquisado na área, o que fica evidenciado com o gráfico de pesquisbaixo, coletado via Google Trends buscando por frameworks de desenvolvimento híbrido mobile.


\begin{figure}
	\label{fig:graficoGoogleTrendsFerramentasHibridasMobile}
	\includegraphics[width=1\textwidth]{img/grafico_ferramentas_hibridas_mobile}
	\caption{Gráfico do Google Trends exibindo as pesquisas por Frameworks híbridos mobile}
\end{figure}	

Para layout da interface mobile, também será aplicado a tendência do Material Design, a fim de propor uma harmonia entre o módulo web e mobile para os usuários


\section{Resultados Esperados}


Como fruto de um sistema para pós-graduação da UFBA, espera-se que os professores tenham mais recursos para integrar as atividades e também prover melhores condições para acompanhamento da vida acadêmica dos alunos em questão. Também, que os novos colaboradores que entrarem no processo tenham facilidade de compreender o fluxo do setor ao navegar pelo sistema proposto.


\section{Fora de Escopo}


Interação com os alunos devido às complicações para realizar a integração com o sistema empregado na UFBA, gerenciado pela XXXXXX, o que causaria uma inviabilidade no projeto devido à necessidade de entrega do produto ser mais forte que o tempo necessário para executar o processo de obtenção de acesso ao sistema legado para realizar a integração.


\section{Estrutura do Trabalho}


<breve resumo sobre os capítulos do TCC>
\chapter{Referencial Teórico}


Projetar o desenvolvimento de um software requer muito planejamento, pois as falhas iniciais podem custar bastante caro ou até mesmo inviabilizar a continuação de um projeto. Assim, a escolha da arquitetura ideal para a aplicabilidade é essencial na concepção de um produto de software. 
De todo o modo, sempre busca-se fazer mais com menos. Diante de tal filosofia, temos neste capítulo, uma breve discussão sobre alguns elementos de projeto e arquitetura de software, a fim de contextualizar este trabalho de conclusão de curso.
O capítulo corrente é composto por quatro seções. A \ref{sec:saas} trata de Software como serviço, discutindo alguns elementos do contexto que são relevantes para o trabalho proposto. A \ref{sec:reuso}  discute sobre a empregabilidade do reuso de software. A \ref{sec:modularizacao} seção, trata sobre aspectos envolvidos na modularização dos softwares. Por fim, a \ref{sec:apps_web} aborda as aplicações web, discutindo sobre aspectos relevantes sobre a aplicação web que compõe este trabalho.


 \section{Software como serviço}\label{sec:saas}


Segundo La e Chun \citep{La2009Systematic}, o princípio da definição de Software como um Serviço (Sofware as a Service - SaaS) é um serviço complementar para aplicações da computação em nuvem (cloud computing). No entanto, as áreas não se confundem. SaaS deve ser entendido como um mecanismo de suporte às soluções existentes na cloud. Os SaaS existem justamente para maximizar o reuso de serviços repetidos e não centrais em uma aplicação remota.


Como vantagens, diversos fatores podem ser favoráveis para a adoção de um SaaS, como custo e manutenção dentre outros fatores aplicáveis a soluções específicas. Lechesaet al. \citep{LechesaSS11}, quantificam índices sobre os fatores determinantes para adoção ou não de um SaaS voltado para ERP na África do Sul. Os principais fatores determinantes para adoção desse mecanismo de software são sua fluidez quanto à rede e a segurança. Esses fatores estão presentes na aplicação desenvolvida neste trabalho de conclusão de curso.


Devido ao fato de ter um serviço constantemente na nuvem, fica o questionamento sobre a segurança da informação manipulada. Sabe-se que a vulnerabilidade na Web não é restrita ao SaaS, atingindo diversos âmbitos. Rai et al. \cite{journals/corr/RaiSM13} consideram que o avanço da computação em nuvem não é um problema apenas para os serviços Web do ponto de vista da segurança, pois muitos trabalhos na literatura mostram a área como mais um ponto de vulnerabilidade para diversos setores, a exemplo de infraestrutura. No mesmo artigo mencionado de Rai et al. \cite{journals/corr/RaiSM13}, também realizaram-se estudos exploratórios junto a empresas usuárias de serviços em computação em nuvem e consideram que a perspectiva de SaaS também pode fortalecer a segurança nas aplicações de cloud computing, pois o software de autenticação compartilhado por várias aplicações em nuvem, oferece uma melhor padronização e consequente facilidade de prevenção a erros de vulnerabilidade específicas de cada módulo da pesquisa. Esse ponto de vista é fundamenta para a compreensão de SaaS.


A arquitetura de armazenamento de dados de um Saas pode variar de acordo com a necessidade do contexto. Huixin \cite{7586486} descreve possíveis modelagens nesse sentido. Essa abordagem pode ser com um banco de dados único, fazendo com que diferentes clientes compartilhem o mesmo banco, diferindo os dados através de controle de usuário, ou isolando os diferentes clientes através de bancos de dados exclusivos para cada um. Esse fator também pode ser combinado com a arquitetura da aplicação, caso ofereça aplicação única para todos os clientes ou aplicação compartilhada. Diante das possíveis abordagens, a modelagem de dados do software pode ser decidida pela regra de negócio. Este trabalho optou por aplicação única e banco de dados compartilhado.

Devido ao conceito particular de obtenção de software, tanto pela visão do cliente como pela visão do vendedor, é necessário compreender os diversos aspectos que podem ser relevantes ao orçar um Saas. O recente trabalho de T. Kaur et al. \citep{6949281} orienta um modelo para compor o custo de um Saas. O custo total seria composto pelos fatores que dão suporte ao funcionamento do software. Tais fatores incluem infra-estrutura, configurabilidade, customização, parâmetros de QoS (Quality of service) como escalabilidade, disponibilidade, usabilidade, pontualidade e desempenho da resposta, portabilidade, custo total de propriedade e retorno do investimento. Esses fatores caracterizam o custo de forma eficaz, possibilitando ao fornecedor, prover um Serviço de acordo com a exigência do consumidor em vários pacotes de serviços.


O conceito de software como serviço se aplica neste trabalho de conclusão de curso, pois o mesmo estará disponível na web com alta disponibilidade adotando as características apresentadas para qualquer pessoa que desejar usá-lo. Assim, basta se cadastrar, configurá-lo e usar como um serviço, sem preocupação com a manutenção do mesmo.


\section{Reuso de software}\label{sec:reuso} %CRUISE BOOK CAPITULO 2


De acordo com o livro Software pratical reuse \citep{Ezran2002}, o reuso de software é a utilização de qualquer informação que um desenvolvedor pode necessitar no processo de criação de software. O livro de Basili e Rombach \citep{Basili1991} define reutilização de software como o uso de tudo o que está associado a um projeto de conhecimento.
Assim, o objetivo da reutilização de software é reciclar o design, código e outros componentes de um produto de software e assim reduzir o custo, o tempo e melhorar a qualidade do produto.
Segundo Keswani et al. \cite{6783445}, o componente reutilizável de software pode ser qualquer parte de seu desenvolvimento, como um fragmento de código, design, casos de teste, ou até mesmo a especificação de requisitos de uma funcionalidade do software. 

O reuso de software pode ter impacto positivo em diversos aspectos do software, vejamos alguns, conforme apresentados no C.R.U.I.S.E Book \citep{cruise2007} :

\begin{itemize}

\item Qualidade: As correções de erro tornam-se úteis em todos os locais em que ocorreu, padronizando e facilitando a manutenção.

\item Produtividade: O ganho de produtividade é alcançado devido ao menor número de artefatos desenvolvido. Isso resulta em menor esforço de teste e também análise e design, reduzindo custos.

\item Confiabilidade: A utilização de componentes bem testados aumenta a
confiança no software. Além disso, a utilização de um mesmo componente em vários sistemas, aumenta a possibilidade de detecção de erros e reforça a confiança no componente.

\item Redução do Esforço: A reutilização de software proporciona uma redução do tempo de desenvolvimento, o que reduz o tempo necessário para o produto ser disponibilizado no mercado para trazer rentabilidade.

\item Trabalho redundante e tempo de desenvolvimento: Desenvolver um sistema do
zero significa desenvolvimento redundante de muitos componentes, como requisitos, especificações, casos de uso, arquitetura, etc. Isso pode ser evitado quando estes estão disponíveis como componentes reutilizáveis e podem ser compartilhados, resultando em um processo de desenvolvimento otimizado.

\item Documentação: Embora a documentação seja muito importante para a
manutenção de um sistema, muitas vezes é negligenciada. A reutilização de componentes de software reduz a quantidade de documentação a ser escrita, entretanto depende da qualidade do que está escrito. Assim, apenas a estrutura do sistema e os novos artefatos desenvolvidos necessitam ser documentados.

\item Custo de manutenção: Menos defeitos e manutenções são esperados quando tem-se comprovada a qualidade dos componentes utilizados.

\item Tamanho da equipe: É comum haver casos em que a equipe de desenvolvimento sofre sobrecarga. Entretanto, dobrar o tamanho da equipe de desenvolvimento não necessariamente duplica produtividade. Se muitos componentes podem ser reutilizados, é possível desenvolver com equipes menores, levando a melhor comunicação e aumento da produtividade.

\end{itemize}

Apesar dos benefícios da reutilização de software, ela não é suficientemente aproveitada. Existem fatores que influenciam direta ou indiretamente na sua adoção. Esses fatores podem ser de aspecto gerencial, organizacional, econômico, conceitual ou técnico. Veremos a seguir alguns aspectos que podem gerar conflito com a cultura de reuso de software, segundo o C.R.U.I.S.E Book \citep{cruise2007} :
%(Sametinger, 1997). REVER

\begin{itemize}
	
\item Falta de apoio da gestão: Como a reutilização de software gera custos iniciais,
a medida pode não ser amplamente alcançada em uma organização sem o apoio de alto nível de gestão. Os gestores têm de ser informados sobre os custos iniciais e serem convencidos sobre economias futuras.

\item Gerenciamento do Projeto: Gerenciar projetos tradicionais é uma tarefa árdua, principalmente, os que praticam a reutilização de software. Utilizando a técnica em larga escala, tem-se impacto sobre todo o ciclo de vida do software.

\item Estruturas organizacionais inadequadas: As estruturas organizacionais devem
considerar diferentes necessidades que surgem quando a reutilização em larga escala está sendo adotada. Por exemplo, uma equipe particionada pode ser alocada somente para desenvolver, manter e certificar componentes reutilizáveis de software.

\item Incentivos de gestão: É comum a falta de incentivo para deixar os desenvolvedores gastarem tempo elaborando componentes do sistemas. A produtividade é muitas vezes medida apenas no tempo necessário para concluir um projeto. Assim, fazer qualquer trabalho além disso, embora benéfico para a empresa como um todo, diminui o seu sucesso. Mesmo quando os componentes reutilizáveis são utilizados, os benefícios obtidos são uma pequena fração do que poderia ser alcançado caso houvesse reutilização explícita, planejada e organizada.

\item Dificuldade de encontrar software reutilizável: Para reutilizar os componentes, devem existir formas eficientes de busca. Além disso, é importante ter um repositório bem organizado contendo componentes com um eficiente meio de acesso.

\item Não reutilização do software encontrado. O acesso fácil ao software existente
não necessariamente aumentar a reutilização. Os componentes reutilizáveis devem ser cuidadosamente especificados, projetados, implementados e documentados, pois em alguns casos, modificar e adaptar o código  pode ser mais custoso que a programação da funcionalidade necessária a partir do zero.

\item Modificação: É muito difícil encontrar um componente que funcione
exatamente da mesma maneira que queremos. Desta forma, são necessárias modificações e devem existir formas de determinar os seus efeitos sobre o componente.


\end{itemize}


A definição do reuso de software pode ser associada a este trabalho, pois ainda que como um serviço, estaremos adotando uma aplicação única para servir aos mais diversos softwares de uma empresa, por exemplo. Logo, ao adotar o projeto, pode ser interessante aplicar algumas métricas para tomar conhecimento de possíveis vantanjens como as citadas nessa seção, a exemplo de economia de tempo de desenvolvimento, e custo do projeto.


%Outra diretriz importante para a reutilização de software é reduzir o risco na criação de novos softwares. O risco tende a ser bastante reduzido se os componentes que estão sendo reutilizados têm as documentação, interfaces necessárias e devidamente testadas, fatores que contibruem para uma fácil integração.
%De acordo com Keswani et al. \citep{6783445}, para o reuso de software dar retornos apropriados, o processo deve ser sistemático e planejado. Qualquer organização que implemente a reutilização de software deve identificar os melhores métodos e estratégias de reutilização para obter a máxima produtividade. A reutilização de software ajuda a evitar software de engenharia a partir do zero, pois usa módulos de software existentes. A reutilização de software, embora seja uma tarefa difícil, especialmente para softwares antigos sem padrões de projeto, pode melhorar significativamente a produtividade e a qualidade de um produto de software. Embora a reutilização de software não seja um novo campo, ela pode dar grandes retornos em curto período de tempo.


\section{Modularização}\label{sec:modularizacao} %artigo de claudio pagina 222 introdução


%A modularidade vem desempenhando um papel predominante estágios emergentes das disciplinas de arquitetura de software [13]. Engenheiros de software consideram modularidade como princípio base na comparação entre arquiteturas alternativas  e arquitetura degeneração [9]. De fato, os engenheiros de software são incentivados a arquitecturas, baseando-se numa multiplicidade de mecanismos de modularidade disponíveis em: 
%(i) Linguagens de descrição de arquitetura (ADLs), como ACME [8], 
%(ii) catálogos de arquitetônicos [2, 13], e 
%(iii) conhecem bem princípios de alto nível, como interfaces de componentes estreitos, acoplamento arquitectónico reduzido e semelhantes.


Conforme é frisado por Wickramaarachchi e Lai \citep{7062705}, o conceito de modularização na indústria de software tem uma longa história e tem sido utilizado para melhorar o processo de desenvolvimento de software em diferentes estágios. Os principais conceitos por trás da modularização do software foram introduzidos por pesquisadores pioneiros há quarenta anos, com uma notável contribuição feita por Melvin Conway e David Parnas, que tem representação notável na engenharia de software.


Modularizar um software é um bom padrão a ser adotado. Segundo Wickramaarachchi e Lai \citep{7062705}, a modularização é importante na identificação de dependências e reduz as dificuldades diante de uma possível necessidade de grandes alterações. De uma perspectiva da engenharia de software, uma modularização geralmente tem várias vantagens, tais como: tornar a complexidade do software mais gerenciável, facilitar o trabalho paralelo e tornar o software mais maleável para acomodar o futuro incerto que um software pode ter. O objetivo final da modularização do software é aumentar a produtividade ea qualidade do software. Tal conceito encontra-se bastante difundido e estái incorporado em linguagens de programação e ferramentas de software. O presente trabalho favorece ao uso da modularização de um software e até mesmo pode ser considerado um módulo a ser acoplado a qualquer software, mediante a compatibilidade.


Ao realizar a adoção do trabalho proposto, fica bastante evidente a "responsabilidade" do mesmo no escopo do projeto. Assim, é possível tratá-lo como um módulo do projeto que o usa, mesmo que seja consumido como um serviço.


\section{Aplicações web}\label{sec:apps_web}


A popularidade das soluções Web aumentou exponencialmente na última década e todos os dias cresce o número de pessoas usuárias desse tipo de software. E seguindo um padrão próprio, Kumar et al. \citep{7813710} sugerem que para o desenvolvimento web, deve-se manter a prática eficaz de produzir diagramas UML. A abordagem baseada na web oferece uma maneira fácil e eficaz para gerenciar e controlar o processo de desenvolvimento por meio de artefatos de modelagem. Tal abordagem pode ser usada quando há uma exigência de lidar com mudanças muito rápidas e grandes em requisitos de forma muito eficaz em muito menos tempo, gerando assim um menor impacto. 


Para atender à fomentada demanda de aplicativos web, é necessário adotar métodos de desenvolvimentos que sejam ágeis, eficientes e de fácil manutenção. Yu Ping et al. \cite{1372143} propõem o uso do modelo MVC (Model, View e Controller) no desenvolvimento para softwares web. O modelo apresentado tornou-se um padrão popular e divide o software em camadas com propósito definido, tornando-o de mais fácil manutenção.


O Ajax (Asynchronous Javascript and XML) revolucionou a web. Conforme demonstrado por Yuping \citep{6845605}, ao usar a tecnologia Ajax, podemos enriquecer a experiência do usuário em aplicações baseadas em navegador de internet, e fornecer uma variedade de aplicações interativas para atender às necessidade de humanização das aplicações.
Os aplicativos Ajax em execução no navegador se comunicam com um servidor Web de forma assíncrona e atualizam apenas uma parte da página.


No artigo de Tesarik et al. \citep{4664349}, temos que o design de software SPA (Single page application) é uma maneira interessante de criar um software Web numa uma única página. Essa solução de página única sem navegação funciona apenas com base em técnicas dinâmicas e assíncronas, como o AJAX (citado no parágrafo anterior). No entanto, esta abordagem coloca o desenvolvedor antes de alguns desafios substanciais.
Para projetar a interface do usuário que mostra as informações, é importante projetar corretamente a tela para manipular os dados do aplicativo numa única página. O design da página deve ser elaborado para maximizar a decomposição da página em componentes distintos que encapsulam os principais casos de uso.
Também recomenda-se que para elaborar os artefatos visuais, explore os mais ricos recursos na implementação, como HTML5, JavaScript, Ajax, CSS3, e outras tecnologias que se apliquem. Entretanto, desenvolver uma rica interface com o uso de diversas tecnologias/frameworks pode ocasionar um esforço maior para explorar as possibilidades.
Tais componentes da interface de uma aplicação SPA normalmente são alimentados mediante o consumo de uma API Rest via requisições AJAX. Assim, é possível particionar as responsabilidades de processamento do software entre cliente e servidor.


Como é possível denotar no trabalho de Salvadori e Siqueire \citep{7195633}, REST é uma arquitetura muito popular para integração de aplicativos web, que permite compartilhar e reutilizar informações através de sistemas. Aplicações de grande escala baseadas em no REST deve ser implementadas utilizando estratégias e mecanismos para produzir sistemas que sejam fáceis de desenvolver, reutilizar e manter. As interfaces de integração fornecidas por esses sistemas, chamadas de Web API, têm uma influência importante nas características da implementação resultante, pois a sua forma de resposta deve ser casada com a interface que o consome.
 
 
Os elementos comentados nessa seção encontram-se presentes neste trabalho de conclusão de curso. Juntos, montam a estrutura tecnológica necessária aliada à arquitetura adotada. Esses elementos seguem tendências atuais dos softwares web, fazendo com que esse trabalho esteja composto por tecnologias modernas que estão com boa aceitação no mercado.

% %% RiSE Latex Template - version 0.5
%%
%% RiSE's latex template for thesis and dissertations
%% http://risetemplate.sourceforge.net
%%
%% (c) 2012 Yguaratã Cerqueira Cavalcanti (yguarata@gmail.com)
%%          Vinicius Cardoso Garcia (vinicius.garcia@gmail.com)
%%
%% This document was initially based on UFPEThesis template, from Paulo Gustavo
%% S. Fonseca.
%%
%% ACKNOWLEDGEMENTS
%%
%% We would like to thanks the RiSE's researchers community, the 
%% students from Federal University of Pernambuco, and other users that have
%% been contributing to this projects with comments and patches.
%%
%% GENERAL INSTRUCTIONS
%%
%% We strongly recommend you to compile your documents using pdflatex command.
%% It is also recommend use the texlipse plugin for Eclipse to edit your documents.
%%
%% Options for \documentclass command:
%%         * Idiom
%%           pt   - Portguese (default)
%%           en   - English
%%
%%         * Text type
%%           bsc  - B.Sc. Thesis
%%           msc  - M.Sc. Thesis (default)
%%           qual - PHD qualification (not tested yet)
%%           prop - PHD proposal (not tested yet)
%%           phd  - PHD thesis
%%
%%         * Media
%%           scr  - to eletronic version (PDF) / see the users guide
%%
%%         * Pagination
%%           oneside - unique face press
%%           twoside - two faces press
%%
%%		   * Line spacing
%%           singlespacing  - the same as using \linespread{1}
%%           onehalfspacing - the same as using \linespread{1.3}
%%           doublespacing  - the same as using \linespread{1.6}
%%
%% Reference commands. Use the following commands to make references in your
%% text:
%%          \figref  -- for Figure reference
%%          \tabref  -- for Table reference
%%          \eqnref  -- for equation reference
%%          \chapref -- for chapter reference
%%          \secref  -- for section reference
%%          \appref  -- for appendix reference
%%          \axiref  -- for axiom reference
%%          \conjref -- for conjecture reference
%%          \defref  -- for definition reference
%%          \lemref  -- for lemma reference
%%          \theoref -- for theorem reference
%%          \corref  -- for corollary reference
%%          \propref -- for proprosition reference
%%          \pgref   -- for page reference
%%
%%          Example: See \chapref{chap:introduction}. It will produce 
%%                   'See Chapter 1', in case of English language.

\documentclass[pt,twoside,onehalfspacing,bsc]{risethesis}

\usepackage[sort,square]{natbib} %package for bibliography
\bibliographystyle{ieeetr} %authordate1

\usepackage[utf8]{inputenc}
\usepackage[brazilian]{babel}
\usepackage[T1]{fontenc}

\usepackage{graphicx}

%% Change the following pdf author attribute name to your name.
\usepackage[linkcolor=blue,citecolor=blue,urlcolor=blue,colorlinks,pdfpagelabels,pdftitle={Bruno Cabral's Bachelor Thesis},pdfauthor={Bruno Cabral}]{hyperref}

\address{SALVADOR}

\universitypt{Universidade Federal da Bahia}
\universityen{Federal University of Bahia}

\departmentpt{Depertamento de Ciência da Computação}
\departmenten{Computer Science Department}

\programpt{Programa Multiinstitucional de Pós-graduação em Ciência da Computação}
\programen{Graduate in Computer Science}

\majorfieldpt{Ciência da Computação}
\majorfielden{Computer Science}

\title{Sistema de apoio à Pós graduação - UFBA}
\date{Outubro/2016}

\author{Victor de Azevedo Nunes}
\adviser{Ivan do Carmo Machado}

\begin{document}

\frontmatter
\frontpage
\presentationpage

\begin{dedicatory}
Eu dedico esta dissertação...
%I dedicate this dissertation to my family, girlfriend, friends and
%professors who gave me all necessary support to get here.
\end{dedicatory}

\acknowledgements
Meus agradecimentos...

\begin{epigraph}[]{Edward V Berard}
Walking on water and developing software from a specification are easy if both are frozen
\end{epigraph}

\resumo
% Escreva seu resumo no arquivo resumo.tex
\input{resumo}

\abstract
% Write your abstract in a file called abstract.tex
\input{abstract}

% Summary (tables of contents)
\tableofcontents

% List of figures
\listoffigures

% List of tables
\listoftables

% List of acronyms
% Acronyms manual: http://linorg.usp.br/CTAN/macros/latex/contrib/acronym/acronym.pdf
\listofacronyms
\input{acronyms}

% List of listings
%\lstlistoflistings

\mainmatter

\include{chapters/intro}
\include{chapters/referencial_teorico}

% \include{chapters/introduction/main}
% \include{chapters/background/main}
% \include{chapters/proposed_solution/main}
% \include{chapters/experiment/main}
% \include{chapters/conclusion/main}

%\bibliographystyle{natbib}
%\addcontentsline{toc}{chapter}{\bibliographytocname}
\bibliography{references}

% Appendix
\clearpage
\addappheadtotoc
\appendix
\appendixpage
% \include{appendix/experiment-instruments}

\end{document}
% %% RiSE Latex Template - version 0.5
%%
%% RiSE's latex template for thesis and dissertations
%% http://risetemplate.sourceforge.net
%%
%% (c) 2012 Yguaratã Cerqueira Cavalcanti (yguarata@gmail.com)
%%          Vinicius Cardoso Garcia (vinicius.garcia@gmail.com)
%%
%% This document was initially based on UFPEThesis template, from Paulo Gustavo
%% S. Fonseca.
%%
%% ACKNOWLEDGEMENTS
%%
%% We would like to thanks the RiSE's researchers community, the 
%% students from Federal University of Pernambuco, and other users that have
%% been contributing to this projects with comments and patches.
%%
%% GENERAL INSTRUCTIONS
%%
%% We strongly recommend you to compile your documents using pdflatex command.
%% It is also recommend use the texlipse plugin for Eclipse to edit your documents.
%%
%% Options for \documentclass command:
%%         * Idiom
%%           pt   - Portguese (default)
%%           en   - English
%%
%%         * Text type
%%           bsc  - B.Sc. Thesis
%%           msc  - M.Sc. Thesis (default)
%%           qual - PHD qualification (not tested yet)
%%           prop - PHD proposal (not tested yet)
%%           phd  - PHD thesis
%%
%%         * Media
%%           scr  - to eletronic version (PDF) / see the users guide
%%
%%         * Pagination
%%           oneside - unique face press
%%           twoside - two faces press
%%
%%		   * Line spacing
%%           singlespacing  - the same as using \linespread{1}
%%           onehalfspacing - the same as using \linespread{1.3}
%%           doublespacing  - the same as using \linespread{1.6}
%%
%% Reference commands. Use the following commands to make references in your
%% text:
%%          \figref  -- for Figure reference
%%          \tabref  -- for Table reference
%%          \eqnref  -- for equation reference
%%          \chapref -- for chapter reference
%%          \secref  -- for section reference
%%          \appref  -- for appendix reference
%%          \axiref  -- for axiom reference
%%          \conjref -- for conjecture reference
%%          \defref  -- for definition reference
%%          \lemref  -- for lemma reference
%%          \theoref -- for theorem reference
%%          \corref  -- for corollary reference
%%          \propref -- for proprosition reference
%%          \pgref   -- for page reference
%%
%%          Example: See \chapref{chap:introduction}. It will produce 
%%                   'See Chapter 1', in case of English language.

\documentclass[pt,twoside,onehalfspacing,bsc]{risethesis}

\usepackage[sort,square]{natbib} %package for bibliography
\bibliographystyle{ieeetr} %authordate1

\usepackage[utf8]{inputenc}
\usepackage[brazilian]{babel}
\usepackage[T1]{fontenc}

\usepackage{graphicx}

%% Change the following pdf author attribute name to your name.
\usepackage[linkcolor=blue,citecolor=blue,urlcolor=blue,colorlinks,pdfpagelabels,pdftitle={Bruno Cabral's Bachelor Thesis},pdfauthor={Bruno Cabral}]{hyperref}

\address{SALVADOR}

\universitypt{Universidade Federal da Bahia}
\universityen{Federal University of Bahia}

\departmentpt{Depertamento de Ciência da Computação}
\departmenten{Computer Science Department}

\programpt{Programa Multiinstitucional de Pós-graduação em Ciência da Computação}
\programen{Graduate in Computer Science}

\majorfieldpt{Ciência da Computação}
\majorfielden{Computer Science}

\title{Sistema de apoio à Pós graduação - UFBA}
\date{Outubro/2016}

\author{Victor de Azevedo Nunes}
\adviser{Ivan do Carmo Machado}

\begin{document}

\frontmatter
\frontpage
\presentationpage

\begin{dedicatory}
Eu dedico esta dissertação...
%I dedicate this dissertation to my family, girlfriend, friends and
%professors who gave me all necessary support to get here.
\end{dedicatory}

\acknowledgements
Meus agradecimentos...

\begin{epigraph}[]{Edward V Berard}
Walking on water and developing software from a specification are easy if both are frozen
\end{epigraph}

\resumo
% Escreva seu resumo no arquivo resumo.tex
\input{resumo}

\abstract
% Write your abstract in a file called abstract.tex
\input{abstract}

% Summary (tables of contents)
\tableofcontents

% List of figures
\listoffigures

% List of tables
\listoftables

% List of acronyms
% Acronyms manual: http://linorg.usp.br/CTAN/macros/latex/contrib/acronym/acronym.pdf
\listofacronyms
\input{acronyms}

% List of listings
%\lstlistoflistings

\mainmatter

\include{chapters/intro}
\include{chapters/referencial_teorico}

% \include{chapters/introduction/main}
% \include{chapters/background/main}
% \include{chapters/proposed_solution/main}
% \include{chapters/experiment/main}
% \include{chapters/conclusion/main}

%\bibliographystyle{natbib}
%\addcontentsline{toc}{chapter}{\bibliographytocname}
\bibliography{references}

% Appendix
\clearpage
\addappheadtotoc
\appendix
\appendixpage
% \include{appendix/experiment-instruments}

\end{document}
% %% RiSE Latex Template - version 0.5
%%
%% RiSE's latex template for thesis and dissertations
%% http://risetemplate.sourceforge.net
%%
%% (c) 2012 Yguaratã Cerqueira Cavalcanti (yguarata@gmail.com)
%%          Vinicius Cardoso Garcia (vinicius.garcia@gmail.com)
%%
%% This document was initially based on UFPEThesis template, from Paulo Gustavo
%% S. Fonseca.
%%
%% ACKNOWLEDGEMENTS
%%
%% We would like to thanks the RiSE's researchers community, the 
%% students from Federal University of Pernambuco, and other users that have
%% been contributing to this projects with comments and patches.
%%
%% GENERAL INSTRUCTIONS
%%
%% We strongly recommend you to compile your documents using pdflatex command.
%% It is also recommend use the texlipse plugin for Eclipse to edit your documents.
%%
%% Options for \documentclass command:
%%         * Idiom
%%           pt   - Portguese (default)
%%           en   - English
%%
%%         * Text type
%%           bsc  - B.Sc. Thesis
%%           msc  - M.Sc. Thesis (default)
%%           qual - PHD qualification (not tested yet)
%%           prop - PHD proposal (not tested yet)
%%           phd  - PHD thesis
%%
%%         * Media
%%           scr  - to eletronic version (PDF) / see the users guide
%%
%%         * Pagination
%%           oneside - unique face press
%%           twoside - two faces press
%%
%%		   * Line spacing
%%           singlespacing  - the same as using \linespread{1}
%%           onehalfspacing - the same as using \linespread{1.3}
%%           doublespacing  - the same as using \linespread{1.6}
%%
%% Reference commands. Use the following commands to make references in your
%% text:
%%          \figref  -- for Figure reference
%%          \tabref  -- for Table reference
%%          \eqnref  -- for equation reference
%%          \chapref -- for chapter reference
%%          \secref  -- for section reference
%%          \appref  -- for appendix reference
%%          \axiref  -- for axiom reference
%%          \conjref -- for conjecture reference
%%          \defref  -- for definition reference
%%          \lemref  -- for lemma reference
%%          \theoref -- for theorem reference
%%          \corref  -- for corollary reference
%%          \propref -- for proprosition reference
%%          \pgref   -- for page reference
%%
%%          Example: See \chapref{chap:introduction}. It will produce 
%%                   'See Chapter 1', in case of English language.

\documentclass[pt,twoside,onehalfspacing,bsc]{risethesis}

\usepackage[sort,square]{natbib} %package for bibliography
\bibliographystyle{ieeetr} %authordate1

\usepackage[utf8]{inputenc}
\usepackage[brazilian]{babel}
\usepackage[T1]{fontenc}

\usepackage{graphicx}

%% Change the following pdf author attribute name to your name.
\usepackage[linkcolor=blue,citecolor=blue,urlcolor=blue,colorlinks,pdfpagelabels,pdftitle={Bruno Cabral's Bachelor Thesis},pdfauthor={Bruno Cabral}]{hyperref}

\address{SALVADOR}

\universitypt{Universidade Federal da Bahia}
\universityen{Federal University of Bahia}

\departmentpt{Depertamento de Ciência da Computação}
\departmenten{Computer Science Department}

\programpt{Programa Multiinstitucional de Pós-graduação em Ciência da Computação}
\programen{Graduate in Computer Science}

\majorfieldpt{Ciência da Computação}
\majorfielden{Computer Science}

\title{Sistema de apoio à Pós graduação - UFBA}
\date{Outubro/2016}

\author{Victor de Azevedo Nunes}
\adviser{Ivan do Carmo Machado}

\begin{document}

\frontmatter
\frontpage
\presentationpage

\begin{dedicatory}
Eu dedico esta dissertação...
%I dedicate this dissertation to my family, girlfriend, friends and
%professors who gave me all necessary support to get here.
\end{dedicatory}

\acknowledgements
Meus agradecimentos...

\begin{epigraph}[]{Edward V Berard}
Walking on water and developing software from a specification are easy if both are frozen
\end{epigraph}

\resumo
% Escreva seu resumo no arquivo resumo.tex
\input{resumo}

\abstract
% Write your abstract in a file called abstract.tex
\input{abstract}

% Summary (tables of contents)
\tableofcontents

% List of figures
\listoffigures

% List of tables
\listoftables

% List of acronyms
% Acronyms manual: http://linorg.usp.br/CTAN/macros/latex/contrib/acronym/acronym.pdf
\listofacronyms
\input{acronyms}

% List of listings
%\lstlistoflistings

\mainmatter

\include{chapters/intro}
\include{chapters/referencial_teorico}

% \include{chapters/introduction/main}
% \include{chapters/background/main}
% \include{chapters/proposed_solution/main}
% \include{chapters/experiment/main}
% \include{chapters/conclusion/main}

%\bibliographystyle{natbib}
%\addcontentsline{toc}{chapter}{\bibliographytocname}
\bibliography{references}

% Appendix
\clearpage
\addappheadtotoc
\appendix
\appendixpage
% \include{appendix/experiment-instruments}

\end{document}
% %% RiSE Latex Template - version 0.5
%%
%% RiSE's latex template for thesis and dissertations
%% http://risetemplate.sourceforge.net
%%
%% (c) 2012 Yguaratã Cerqueira Cavalcanti (yguarata@gmail.com)
%%          Vinicius Cardoso Garcia (vinicius.garcia@gmail.com)
%%
%% This document was initially based on UFPEThesis template, from Paulo Gustavo
%% S. Fonseca.
%%
%% ACKNOWLEDGEMENTS
%%
%% We would like to thanks the RiSE's researchers community, the 
%% students from Federal University of Pernambuco, and other users that have
%% been contributing to this projects with comments and patches.
%%
%% GENERAL INSTRUCTIONS
%%
%% We strongly recommend you to compile your documents using pdflatex command.
%% It is also recommend use the texlipse plugin for Eclipse to edit your documents.
%%
%% Options for \documentclass command:
%%         * Idiom
%%           pt   - Portguese (default)
%%           en   - English
%%
%%         * Text type
%%           bsc  - B.Sc. Thesis
%%           msc  - M.Sc. Thesis (default)
%%           qual - PHD qualification (not tested yet)
%%           prop - PHD proposal (not tested yet)
%%           phd  - PHD thesis
%%
%%         * Media
%%           scr  - to eletronic version (PDF) / see the users guide
%%
%%         * Pagination
%%           oneside - unique face press
%%           twoside - two faces press
%%
%%		   * Line spacing
%%           singlespacing  - the same as using \linespread{1}
%%           onehalfspacing - the same as using \linespread{1.3}
%%           doublespacing  - the same as using \linespread{1.6}
%%
%% Reference commands. Use the following commands to make references in your
%% text:
%%          \figref  -- for Figure reference
%%          \tabref  -- for Table reference
%%          \eqnref  -- for equation reference
%%          \chapref -- for chapter reference
%%          \secref  -- for section reference
%%          \appref  -- for appendix reference
%%          \axiref  -- for axiom reference
%%          \conjref -- for conjecture reference
%%          \defref  -- for definition reference
%%          \lemref  -- for lemma reference
%%          \theoref -- for theorem reference
%%          \corref  -- for corollary reference
%%          \propref -- for proprosition reference
%%          \pgref   -- for page reference
%%
%%          Example: See \chapref{chap:introduction}. It will produce 
%%                   'See Chapter 1', in case of English language.

\documentclass[pt,twoside,onehalfspacing,bsc]{risethesis}

\usepackage[sort,square]{natbib} %package for bibliography
\bibliographystyle{ieeetr} %authordate1

\usepackage[utf8]{inputenc}
\usepackage[brazilian]{babel}
\usepackage[T1]{fontenc}

\usepackage{graphicx}

%% Change the following pdf author attribute name to your name.
\usepackage[linkcolor=blue,citecolor=blue,urlcolor=blue,colorlinks,pdfpagelabels,pdftitle={Bruno Cabral's Bachelor Thesis},pdfauthor={Bruno Cabral}]{hyperref}

\address{SALVADOR}

\universitypt{Universidade Federal da Bahia}
\universityen{Federal University of Bahia}

\departmentpt{Depertamento de Ciência da Computação}
\departmenten{Computer Science Department}

\programpt{Programa Multiinstitucional de Pós-graduação em Ciência da Computação}
\programen{Graduate in Computer Science}

\majorfieldpt{Ciência da Computação}
\majorfielden{Computer Science}

\title{Sistema de apoio à Pós graduação - UFBA}
\date{Outubro/2016}

\author{Victor de Azevedo Nunes}
\adviser{Ivan do Carmo Machado}

\begin{document}

\frontmatter
\frontpage
\presentationpage

\begin{dedicatory}
Eu dedico esta dissertação...
%I dedicate this dissertation to my family, girlfriend, friends and
%professors who gave me all necessary support to get here.
\end{dedicatory}

\acknowledgements
Meus agradecimentos...

\begin{epigraph}[]{Edward V Berard}
Walking on water and developing software from a specification are easy if both are frozen
\end{epigraph}

\resumo
% Escreva seu resumo no arquivo resumo.tex
\input{resumo}

\abstract
% Write your abstract in a file called abstract.tex
\input{abstract}

% Summary (tables of contents)
\tableofcontents

% List of figures
\listoffigures

% List of tables
\listoftables

% List of acronyms
% Acronyms manual: http://linorg.usp.br/CTAN/macros/latex/contrib/acronym/acronym.pdf
\listofacronyms
\input{acronyms}

% List of listings
%\lstlistoflistings

\mainmatter

\include{chapters/intro}
\include{chapters/referencial_teorico}

% \include{chapters/introduction/main}
% \include{chapters/background/main}
% \include{chapters/proposed_solution/main}
% \include{chapters/experiment/main}
% \include{chapters/conclusion/main}

%\bibliographystyle{natbib}
%\addcontentsline{toc}{chapter}{\bibliographytocname}
\bibliography{references}

% Appendix
\clearpage
\addappheadtotoc
\appendix
\appendixpage
% \include{appendix/experiment-instruments}

\end{document}
% %% RiSE Latex Template - version 0.5
%%
%% RiSE's latex template for thesis and dissertations
%% http://risetemplate.sourceforge.net
%%
%% (c) 2012 Yguaratã Cerqueira Cavalcanti (yguarata@gmail.com)
%%          Vinicius Cardoso Garcia (vinicius.garcia@gmail.com)
%%
%% This document was initially based on UFPEThesis template, from Paulo Gustavo
%% S. Fonseca.
%%
%% ACKNOWLEDGEMENTS
%%
%% We would like to thanks the RiSE's researchers community, the 
%% students from Federal University of Pernambuco, and other users that have
%% been contributing to this projects with comments and patches.
%%
%% GENERAL INSTRUCTIONS
%%
%% We strongly recommend you to compile your documents using pdflatex command.
%% It is also recommend use the texlipse plugin for Eclipse to edit your documents.
%%
%% Options for \documentclass command:
%%         * Idiom
%%           pt   - Portguese (default)
%%           en   - English
%%
%%         * Text type
%%           bsc  - B.Sc. Thesis
%%           msc  - M.Sc. Thesis (default)
%%           qual - PHD qualification (not tested yet)
%%           prop - PHD proposal (not tested yet)
%%           phd  - PHD thesis
%%
%%         * Media
%%           scr  - to eletronic version (PDF) / see the users guide
%%
%%         * Pagination
%%           oneside - unique face press
%%           twoside - two faces press
%%
%%		   * Line spacing
%%           singlespacing  - the same as using \linespread{1}
%%           onehalfspacing - the same as using \linespread{1.3}
%%           doublespacing  - the same as using \linespread{1.6}
%%
%% Reference commands. Use the following commands to make references in your
%% text:
%%          \figref  -- for Figure reference
%%          \tabref  -- for Table reference
%%          \eqnref  -- for equation reference
%%          \chapref -- for chapter reference
%%          \secref  -- for section reference
%%          \appref  -- for appendix reference
%%          \axiref  -- for axiom reference
%%          \conjref -- for conjecture reference
%%          \defref  -- for definition reference
%%          \lemref  -- for lemma reference
%%          \theoref -- for theorem reference
%%          \corref  -- for corollary reference
%%          \propref -- for proprosition reference
%%          \pgref   -- for page reference
%%
%%          Example: See \chapref{chap:introduction}. It will produce 
%%                   'See Chapter 1', in case of English language.

\documentclass[pt,twoside,onehalfspacing,bsc]{risethesis}

\usepackage[sort,square]{natbib} %package for bibliography
\bibliographystyle{ieeetr} %authordate1

\usepackage[utf8]{inputenc}
\usepackage[brazilian]{babel}
\usepackage[T1]{fontenc}

\usepackage{graphicx}

%% Change the following pdf author attribute name to your name.
\usepackage[linkcolor=blue,citecolor=blue,urlcolor=blue,colorlinks,pdfpagelabels,pdftitle={Bruno Cabral's Bachelor Thesis},pdfauthor={Bruno Cabral}]{hyperref}

\address{SALVADOR}

\universitypt{Universidade Federal da Bahia}
\universityen{Federal University of Bahia}

\departmentpt{Depertamento de Ciência da Computação}
\departmenten{Computer Science Department}

\programpt{Programa Multiinstitucional de Pós-graduação em Ciência da Computação}
\programen{Graduate in Computer Science}

\majorfieldpt{Ciência da Computação}
\majorfielden{Computer Science}

\title{Sistema de apoio à Pós graduação - UFBA}
\date{Outubro/2016}

\author{Victor de Azevedo Nunes}
\adviser{Ivan do Carmo Machado}

\begin{document}

\frontmatter
\frontpage
\presentationpage

\begin{dedicatory}
Eu dedico esta dissertação...
%I dedicate this dissertation to my family, girlfriend, friends and
%professors who gave me all necessary support to get here.
\end{dedicatory}

\acknowledgements
Meus agradecimentos...

\begin{epigraph}[]{Edward V Berard}
Walking on water and developing software from a specification are easy if both are frozen
\end{epigraph}

\resumo
% Escreva seu resumo no arquivo resumo.tex
\input{resumo}

\abstract
% Write your abstract in a file called abstract.tex
\input{abstract}

% Summary (tables of contents)
\tableofcontents

% List of figures
\listoffigures

% List of tables
\listoftables

% List of acronyms
% Acronyms manual: http://linorg.usp.br/CTAN/macros/latex/contrib/acronym/acronym.pdf
\listofacronyms
\input{acronyms}

% List of listings
%\lstlistoflistings

\mainmatter

\include{chapters/intro}
\include{chapters/referencial_teorico}

% \include{chapters/introduction/main}
% \include{chapters/background/main}
% \include{chapters/proposed_solution/main}
% \include{chapters/experiment/main}
% \include{chapters/conclusion/main}

%\bibliographystyle{natbib}
%\addcontentsline{toc}{chapter}{\bibliographytocname}
\bibliography{references}

% Appendix
\clearpage
\addappheadtotoc
\appendix
\appendixpage
% \include{appendix/experiment-instruments}

\end{document}

%\bibliographystyle{natbib}
%\addcontentsline{toc}{chapter}{\bibliographytocname}
\bibliography{references}

% Appendix
\clearpage
\addappheadtotoc
\appendix
\appendixpage
% \include{appendix/experiment-instruments}

\end{document}
% %% RiSE Latex Template - version 0.5
%%
%% RiSE's latex template for thesis and dissertations
%% http://risetemplate.sourceforge.net
%%
%% (c) 2012 Yguaratã Cerqueira Cavalcanti (yguarata@gmail.com)
%%          Vinicius Cardoso Garcia (vinicius.garcia@gmail.com)
%%
%% This document was initially based on UFPEThesis template, from Paulo Gustavo
%% S. Fonseca.
%%
%% ACKNOWLEDGEMENTS
%%
%% We would like to thanks the RiSE's researchers community, the 
%% students from Federal University of Pernambuco, and other users that have
%% been contributing to this projects with comments and patches.
%%
%% GENERAL INSTRUCTIONS
%%
%% We strongly recommend you to compile your documents using pdflatex command.
%% It is also recommend use the texlipse plugin for Eclipse to edit your documents.
%%
%% Options for \documentclass command:
%%         * Idiom
%%           pt   - Portguese (default)
%%           en   - English
%%
%%         * Text type
%%           bsc  - B.Sc. Thesis
%%           msc  - M.Sc. Thesis (default)
%%           qual - PHD qualification (not tested yet)
%%           prop - PHD proposal (not tested yet)
%%           phd  - PHD thesis
%%
%%         * Media
%%           scr  - to eletronic version (PDF) / see the users guide
%%
%%         * Pagination
%%           oneside - unique face press
%%           twoside - two faces press
%%
%%		   * Line spacing
%%           singlespacing  - the same as using \linespread{1}
%%           onehalfspacing - the same as using \linespread{1.3}
%%           doublespacing  - the same as using \linespread{1.6}
%%
%% Reference commands. Use the following commands to make references in your
%% text:
%%          \figref  -- for Figure reference
%%          \tabref  -- for Table reference
%%          \eqnref  -- for equation reference
%%          \chapref -- for chapter reference
%%          \secref  -- for section reference
%%          \appref  -- for appendix reference
%%          \axiref  -- for axiom reference
%%          \conjref -- for conjecture reference
%%          \defref  -- for definition reference
%%          \lemref  -- for lemma reference
%%          \theoref -- for theorem reference
%%          \corref  -- for corollary reference
%%          \propref -- for proprosition reference
%%          \pgref   -- for page reference
%%
%%          Example: See \chapref{chap:introduction}. It will produce 
%%                   'See Chapter 1', in case of English language.

\documentclass[pt,twoside,onehalfspacing,bsc]{risethesis}

\usepackage[sort,square]{natbib} %package for bibliography
\bibliographystyle{ieeetr} %authordate1

\usepackage[utf8]{inputenc}
\usepackage[brazilian]{babel}
\usepackage[T1]{fontenc}

\usepackage{graphicx}

%% Change the following pdf author attribute name to your name.
\usepackage[linkcolor=blue,citecolor=blue,urlcolor=blue,colorlinks,pdfpagelabels,pdftitle={Bruno Cabral's Bachelor Thesis},pdfauthor={Bruno Cabral}]{hyperref}

\address{SALVADOR}

\universitypt{Universidade Federal da Bahia}
\universityen{Federal University of Bahia}

\departmentpt{Depertamento de Ciência da Computação}
\departmenten{Computer Science Department}

\programpt{Programa Multiinstitucional de Pós-graduação em Ciência da Computação}
\programen{Graduate in Computer Science}

\majorfieldpt{Ciência da Computação}
\majorfielden{Computer Science}

\title{Sistema de apoio à Pós graduação - UFBA}
\date{Outubro/2016}

\author{Victor de Azevedo Nunes}
\adviser{Ivan do Carmo Machado}

\begin{document}

\frontmatter
\frontpage
\presentationpage

\begin{dedicatory}
Eu dedico esta dissertação...
%I dedicate this dissertation to my family, girlfriend, friends and
%professors who gave me all necessary support to get here.
\end{dedicatory}

\acknowledgements
Meus agradecimentos...

\begin{epigraph}[]{Edward V Berard}
Walking on water and developing software from a specification are easy if both are frozen
\end{epigraph}

\resumo
% Escreva seu resumo no arquivo resumo.tex
Este trabalho de conclusão de curso utiliza elementos da engenharia de software para propor um software como serviço a fim de otimizar o processo de construção e manutenção dos softwares. Assim, o objetivo deste SaaS é gerenciar as permissões de acesso de sistemas cliente, provendo o reuso de software e padronizando as soluções. O sistema proposto neste trabalho encontra-se implementado e disponível no Github, e traz fundamentos sobre a arquitetura e tecnologias utilizadas, bem como avaliações sobre possibilidades de utilização do mesmo em diversos ambientes, como Web e mobile.

\begin{keywords}
Software; Reuso; SaaS; Web; PHP
\end{keywords}

\abstract
% Write your abstract in a file called abstract.tex
My abstract...

\begin{keywords}
key words...
\end{keywords}

% Summary (tables of contents)
\tableofcontents

% List of figures
\listoffigures

% List of tables
\listoftables

% List of acronyms
% Acronyms manual: http://linorg.usp.br/CTAN/macros/latex/contrib/acronym/acronym.pdf
\listofacronyms
\begin{acronym}[ACRONYM] 
% Change the word ACRONYM above to change the acronym column width.
% The column width is equals to the width of the word that you put.
% Read the manual about acronym package for more examples:
%   http://linorg.usp.br/CTAN/macros/latex/contrib/acronym/acronym.pdf

\acro{SPA}{Single Page Application}
\acro{JSON}{Javascript Object Notation}
\acro{PHP}{PHP: Hypertext Preprocessor}
\acro{SaaS}{Software as a Service}
\acro{ERP}{Enterprise Resource Planning}
\acro{QoS}{Quality of Service}
\acro{UML}{Unified Modeling Language}
\acro{MVC}{Model-View-Controller}
\acro{Ajax}{Asynchronous Javascript and XML}
\acro{HTML}{HyperText Markup Language}
\acro{CSS}{Cascading Style Sheets}
\acro{API}{Application Programming Interface}
\acro{DOM}{Document Object Model}
\acro{BPMN}{Business Process Model and Notation}
\acro{REST}{Representational State Transfer}

\end{acronym}

% List of listings
%\lstlistoflistings

\mainmatter

\chapter{Introdução}

\section{Motivação}

Organizar os procedimentos de um processo sempre nos traz vantagens. Apesar de no processo de implantação de um sistema, o mesmo burocratizar o processo, com o tempo temos o retorno da dedicação para a inserção dos dados. Com um certo volume de dados, é possível estruturar informações que num processo manual são difíceis de serem enxergadas. Assim, é possível depender menos das pessoas que organizam o processo, pois o legado de informações não estará mais somente na mente de alguns, mas sim documentado nos dados do sistema.

Além de colaborar na organização, também haverá uma grande colaboração no tempo gasto na gestão. Lidar com muitos papéis e confiar na mente humana para guardar informações, não é uma alternativa muito segura devido ao fato que as pessoas sempre estão sujeitas a sair do processo e levar contigo a experiência obtida. Experiência essa que faz com que os procedimentos sejam executados de forma mais eficiente. Entretanto, com um sistema inteligente, é possível auxiliar e tornar mais ágil a execução das tarefas.


\section{Problema}


De acordo com funcionários ligados ao o setor de pós graduação da UFBA, entrevistados a fim de um maior entendimento do cenário, apesar das semelhanças estruturais, a pós graduação gerida de forma diferencia da graduação. FULANO afirma que devido ao fato de não ter a mesma visibilidade, não tem acesso aos mesmos recursos de gestão acadêmica da graduação. O professores não executam somente atividades dentro da sala de aula, também tem diversas outras ocupações no setor. E muitos procedimentos realizados extra classe ainda se encontram sendo realizados de forma manual, estando mais vulnerável ao erro ou até mesmo à violação do processo. Também ocorre um grande desperdício de tempo pelos professores e gestores da área, devido ao diversos processos ainda realizados de forma manual, sem a devida documentação. Segundo FULANO, também entrevistado, esse tempo perdido implica numa redução da eficiência na sala de aula, pois o professor acaba por ter menos tempo disponível para o planejamento das atividades, o que gera impactos negativos aos alunos.


\section{Objetivos} %<o que deve ser feito/entregue>


Devido aos muitos processos sendo resolvidos de forma manual, propõe-se com solução um sistema moderno, arquitetado para ter funcionamento na web e com um módulo mobile, a fim de fornecer informações de forma rápida e eficiente para os professores através de notificações, já que o acesso à internet móvel é comum entre os possíveis usuários do sistema em questão.
O principal requisito para o sistema seria dispor recursos para reduzir o tempo desperdiçado pelos professores durante as atividades extra classe.


\section{Metodologia} %<como será feito | como resolver o problema apontado inicialmente>


%<analise de literatura | design | implementação | validação>
Baseando-se nas tecnologias gratuitas em alta no cenário atual do desenvolvimento web, dispomos de algumas opções eficientes para a implementação da solução. Dentre as possibilidades, considerando a facilidade para futura manutenção e continuidade do projeto, tende-se a optar por uma tecnologia popular. Como linguagem de programação, adota-se o PHP. A escolha é fundamentada de acordo com a pesquisa da RedMonk de 2015, que evidencia o uso das linguagens de programação de acordo com as discussões no StackOverflow e repositórios no GitHub. É possível constatar a popularidade do PHP no cenário atual com o gráfico da pesquisa citada, na qual o PHP é apresentado na terceira colocação, apenas atrás do lider JavaScript e do segundo colocado, o Java.

\begin{figure}
	\label{fig:graficoRedmonk}
	\includegraphics[width=1\textwidth]{img/grafico_redmonk}
	\caption{Ranking das liguagens de programação no Stack Overflow e Github}
\end{figure}


Ainda assim, para compor a interface do dado projeto, também ocorrerá o uso do líder JavaScript de forma intensa, provendo o elo com o as informações gerenciadas pelo PHP.


Entretanto, não seria inteligente desenvolver um sistema completo sem o auxílio de um framework. Dentre os frameworks disponíveis para PHP, hoje o destaque está com o Laravel, que se encontra no topo dentre os mais utilizados no momento. 


A WebHostFace, uma empresa de hospedagem, compilou várias estatísticas para criar um infográfico mostrando os frameworks PHP mais populares de 2015. Utilizando informações sobre os próprios clientes, o Google Trends, estatísticas de repositórios do GitHub e a pesquisa do SitePoint “Best PHP Frameworks 2015”, a WebHostFace elaborou o seguinte infográfico: 

\begin{figure}
	\label{fig:graficoWebhostface}
	\includegraphics[width=1\textwidth]{img/infografico_webhostface}
	\caption{Infográfico da WebhostFace, exibindo a popularidade dos Frameworks PHP em 2015}
\end{figure}

Assim, tem-se a evidência que o Laravel em 2015 teve a maior popularidade em projetos pessoais e tem a maior comunidade entre os concorrentes, o que o torna uma boa escolha para a escrita de um software que será continuado por terceiros.


Para elaborar os recursos de interface e integrar ao back-end PHP do sistema, será adotado o já conhecido AngularJS, ferramenta sólida e conhecida no aspecto em questão. 


Dados coletados via Google Trends, que propõe comparações entre termos pesquisados, revela a popularidade do AngularJs diante de alguns dos principais concorrentes. O gráfico abaixo evidencia o cenário.


%Como mostra a Figura \ref{fig:graficoGoogleTrendsFerramentasFront}. 
\begin{figure}
	\label{fig:graficoGoogleTrendsFerramentasFront}
	\includegraphics[width=1\textwidth]{img/grafico_ferramentas_front}
	\caption{Gráfico do Google Trends exibindo as pesquisas por ferramentas front-end}
\end{figure}


Junto ao Angular JS, será utilizada a agradável tendência de interface do Material Design da Google, que propõe layouts limpos e otimizados já conhecidos pelos usuários de smartphones Android. 


Para a elaboração da plataforma mobile do projeto, será utilizado o Ionic Framework, muito difundido e bastante pesquisado na área, o que fica evidenciado com o gráfico de pesquisbaixo, coletado via Google Trends buscando por frameworks de desenvolvimento híbrido mobile.


\begin{figure}
	\label{fig:graficoGoogleTrendsFerramentasHibridasMobile}
	\includegraphics[width=1\textwidth]{img/grafico_ferramentas_hibridas_mobile}
	\caption{Gráfico do Google Trends exibindo as pesquisas por Frameworks híbridos mobile}
\end{figure}	

Para layout da interface mobile, também será aplicado a tendência do Material Design, a fim de propor uma harmonia entre o módulo web e mobile para os usuários


\section{Resultados Esperados}


Como fruto de um sistema para pós-graduação da UFBA, espera-se que os professores tenham mais recursos para integrar as atividades e também prover melhores condições para acompanhamento da vida acadêmica dos alunos em questão. Também, que os novos colaboradores que entrarem no processo tenham facilidade de compreender o fluxo do setor ao navegar pelo sistema proposto.


\section{Fora de Escopo}


Interação com os alunos devido às complicações para realizar a integração com o sistema empregado na UFBA, gerenciado pela XXXXXX, o que causaria uma inviabilidade no projeto devido à necessidade de entrega do produto ser mais forte que o tempo necessário para executar o processo de obtenção de acesso ao sistema legado para realizar a integração.


\section{Estrutura do Trabalho}


<breve resumo sobre os capítulos do TCC>
\chapter{Referencial Teórico}


Projetar o desenvolvimento de um software requer muito planejamento, pois as falhas iniciais podem custar bastante caro ou até mesmo inviabilizar a continuação de um projeto. Assim, a escolha da arquitetura ideal para a aplicabilidade é essencial na concepção de um produto de software. 
De todo o modo, sempre busca-se fazer mais com menos. Diante de tal filosofia, temos neste capítulo, uma breve discussão sobre alguns elementos de projeto e arquitetura de software, a fim de contextualizar este trabalho de conclusão de curso.
O capítulo corrente é composto por quatro seções. A \ref{sec:saas} trata de Software como serviço, discutindo alguns elementos do contexto que são relevantes para o trabalho proposto. A \ref{sec:reuso}  discute sobre a empregabilidade do reuso de software. A \ref{sec:modularizacao} seção, trata sobre aspectos envolvidos na modularização dos softwares. Por fim, a \ref{sec:apps_web} aborda as aplicações web, discutindo sobre aspectos relevantes sobre a aplicação web que compõe este trabalho.


 \section{Software como serviço}\label{sec:saas}


Segundo La e Chun \citep{La2009Systematic}, o princípio da definição de Software como um Serviço (Sofware as a Service - SaaS) é um serviço complementar para aplicações da computação em nuvem (cloud computing). No entanto, as áreas não se confundem. SaaS deve ser entendido como um mecanismo de suporte às soluções existentes na cloud. Os SaaS existem justamente para maximizar o reuso de serviços repetidos e não centrais em uma aplicação remota.


Como vantagens, diversos fatores podem ser favoráveis para a adoção de um SaaS, como custo e manutenção dentre outros fatores aplicáveis a soluções específicas. Lechesaet al. \citep{LechesaSS11}, quantificam índices sobre os fatores determinantes para adoção ou não de um SaaS voltado para ERP na África do Sul. Os principais fatores determinantes para adoção desse mecanismo de software são sua fluidez quanto à rede e a segurança. Esses fatores estão presentes na aplicação desenvolvida neste trabalho de conclusão de curso.


Devido ao fato de ter um serviço constantemente na nuvem, fica o questionamento sobre a segurança da informação manipulada. Sabe-se que a vulnerabilidade na Web não é restrita ao SaaS, atingindo diversos âmbitos. Rai et al. \cite{journals/corr/RaiSM13} consideram que o avanço da computação em nuvem não é um problema apenas para os serviços Web do ponto de vista da segurança, pois muitos trabalhos na literatura mostram a área como mais um ponto de vulnerabilidade para diversos setores, a exemplo de infraestrutura. No mesmo artigo mencionado de Rai et al. \cite{journals/corr/RaiSM13}, também realizaram-se estudos exploratórios junto a empresas usuárias de serviços em computação em nuvem e consideram que a perspectiva de SaaS também pode fortalecer a segurança nas aplicações de cloud computing, pois o software de autenticação compartilhado por várias aplicações em nuvem, oferece uma melhor padronização e consequente facilidade de prevenção a erros de vulnerabilidade específicas de cada módulo da pesquisa. Esse ponto de vista é fundamenta para a compreensão de SaaS.


A arquitetura de armazenamento de dados de um Saas pode variar de acordo com a necessidade do contexto. Huixin \cite{7586486} descreve possíveis modelagens nesse sentido. Essa abordagem pode ser com um banco de dados único, fazendo com que diferentes clientes compartilhem o mesmo banco, diferindo os dados através de controle de usuário, ou isolando os diferentes clientes através de bancos de dados exclusivos para cada um. Esse fator também pode ser combinado com a arquitetura da aplicação, caso ofereça aplicação única para todos os clientes ou aplicação compartilhada. Diante das possíveis abordagens, a modelagem de dados do software pode ser decidida pela regra de negócio. Este trabalho optou por aplicação única e banco de dados compartilhado.

Devido ao conceito particular de obtenção de software, tanto pela visão do cliente como pela visão do vendedor, é necessário compreender os diversos aspectos que podem ser relevantes ao orçar um Saas. O recente trabalho de T. Kaur et al. \citep{6949281} orienta um modelo para compor o custo de um Saas. O custo total seria composto pelos fatores que dão suporte ao funcionamento do software. Tais fatores incluem infra-estrutura, configurabilidade, customização, parâmetros de QoS (Quality of service) como escalabilidade, disponibilidade, usabilidade, pontualidade e desempenho da resposta, portabilidade, custo total de propriedade e retorno do investimento. Esses fatores caracterizam o custo de forma eficaz, possibilitando ao fornecedor, prover um Serviço de acordo com a exigência do consumidor em vários pacotes de serviços.


O conceito de software como serviço se aplica neste trabalho de conclusão de curso, pois o mesmo estará disponível na web com alta disponibilidade adotando as características apresentadas para qualquer pessoa que desejar usá-lo. Assim, basta se cadastrar, configurá-lo e usar como um serviço, sem preocupação com a manutenção do mesmo.


\section{Reuso de software}\label{sec:reuso} %CRUISE BOOK CAPITULO 2


De acordo com o livro Software pratical reuse \citep{Ezran2002}, o reuso de software é a utilização de qualquer informação que um desenvolvedor pode necessitar no processo de criação de software. O livro de Basili e Rombach \citep{Basili1991} define reutilização de software como o uso de tudo o que está associado a um projeto de conhecimento.
Assim, o objetivo da reutilização de software é reciclar o design, código e outros componentes de um produto de software e assim reduzir o custo, o tempo e melhorar a qualidade do produto.
Segundo Keswani et al. \cite{6783445}, o componente reutilizável de software pode ser qualquer parte de seu desenvolvimento, como um fragmento de código, design, casos de teste, ou até mesmo a especificação de requisitos de uma funcionalidade do software. 

O reuso de software pode ter impacto positivo em diversos aspectos do software, vejamos alguns, conforme apresentados no C.R.U.I.S.E Book \citep{cruise2007} :

\begin{itemize}

\item Qualidade: As correções de erro tornam-se úteis em todos os locais em que ocorreu, padronizando e facilitando a manutenção.

\item Produtividade: O ganho de produtividade é alcançado devido ao menor número de artefatos desenvolvido. Isso resulta em menor esforço de teste e também análise e design, reduzindo custos.

\item Confiabilidade: A utilização de componentes bem testados aumenta a
confiança no software. Além disso, a utilização de um mesmo componente em vários sistemas, aumenta a possibilidade de detecção de erros e reforça a confiança no componente.

\item Redução do Esforço: A reutilização de software proporciona uma redução do tempo de desenvolvimento, o que reduz o tempo necessário para o produto ser disponibilizado no mercado para trazer rentabilidade.

\item Trabalho redundante e tempo de desenvolvimento: Desenvolver um sistema do
zero significa desenvolvimento redundante de muitos componentes, como requisitos, especificações, casos de uso, arquitetura, etc. Isso pode ser evitado quando estes estão disponíveis como componentes reutilizáveis e podem ser compartilhados, resultando em um processo de desenvolvimento otimizado.

\item Documentação: Embora a documentação seja muito importante para a
manutenção de um sistema, muitas vezes é negligenciada. A reutilização de componentes de software reduz a quantidade de documentação a ser escrita, entretanto depende da qualidade do que está escrito. Assim, apenas a estrutura do sistema e os novos artefatos desenvolvidos necessitam ser documentados.

\item Custo de manutenção: Menos defeitos e manutenções são esperados quando tem-se comprovada a qualidade dos componentes utilizados.

\item Tamanho da equipe: É comum haver casos em que a equipe de desenvolvimento sofre sobrecarga. Entretanto, dobrar o tamanho da equipe de desenvolvimento não necessariamente duplica produtividade. Se muitos componentes podem ser reutilizados, é possível desenvolver com equipes menores, levando a melhor comunicação e aumento da produtividade.

\end{itemize}

Apesar dos benefícios da reutilização de software, ela não é suficientemente aproveitada. Existem fatores que influenciam direta ou indiretamente na sua adoção. Esses fatores podem ser de aspecto gerencial, organizacional, econômico, conceitual ou técnico. Veremos a seguir alguns aspectos que podem gerar conflito com a cultura de reuso de software, segundo o C.R.U.I.S.E Book \citep{cruise2007} :
%(Sametinger, 1997). REVER

\begin{itemize}
	
\item Falta de apoio da gestão: Como a reutilização de software gera custos iniciais,
a medida pode não ser amplamente alcançada em uma organização sem o apoio de alto nível de gestão. Os gestores têm de ser informados sobre os custos iniciais e serem convencidos sobre economias futuras.

\item Gerenciamento do Projeto: Gerenciar projetos tradicionais é uma tarefa árdua, principalmente, os que praticam a reutilização de software. Utilizando a técnica em larga escala, tem-se impacto sobre todo o ciclo de vida do software.

\item Estruturas organizacionais inadequadas: As estruturas organizacionais devem
considerar diferentes necessidades que surgem quando a reutilização em larga escala está sendo adotada. Por exemplo, uma equipe particionada pode ser alocada somente para desenvolver, manter e certificar componentes reutilizáveis de software.

\item Incentivos de gestão: É comum a falta de incentivo para deixar os desenvolvedores gastarem tempo elaborando componentes do sistemas. A produtividade é muitas vezes medida apenas no tempo necessário para concluir um projeto. Assim, fazer qualquer trabalho além disso, embora benéfico para a empresa como um todo, diminui o seu sucesso. Mesmo quando os componentes reutilizáveis são utilizados, os benefícios obtidos são uma pequena fração do que poderia ser alcançado caso houvesse reutilização explícita, planejada e organizada.

\item Dificuldade de encontrar software reutilizável: Para reutilizar os componentes, devem existir formas eficientes de busca. Além disso, é importante ter um repositório bem organizado contendo componentes com um eficiente meio de acesso.

\item Não reutilização do software encontrado. O acesso fácil ao software existente
não necessariamente aumentar a reutilização. Os componentes reutilizáveis devem ser cuidadosamente especificados, projetados, implementados e documentados, pois em alguns casos, modificar e adaptar o código  pode ser mais custoso que a programação da funcionalidade necessária a partir do zero.

\item Modificação: É muito difícil encontrar um componente que funcione
exatamente da mesma maneira que queremos. Desta forma, são necessárias modificações e devem existir formas de determinar os seus efeitos sobre o componente.


\end{itemize}


A definição do reuso de software pode ser associada a este trabalho, pois ainda que como um serviço, estaremos adotando uma aplicação única para servir aos mais diversos softwares de uma empresa, por exemplo. Logo, ao adotar o projeto, pode ser interessante aplicar algumas métricas para tomar conhecimento de possíveis vantanjens como as citadas nessa seção, a exemplo de economia de tempo de desenvolvimento, e custo do projeto.


%Outra diretriz importante para a reutilização de software é reduzir o risco na criação de novos softwares. O risco tende a ser bastante reduzido se os componentes que estão sendo reutilizados têm as documentação, interfaces necessárias e devidamente testadas, fatores que contibruem para uma fácil integração.
%De acordo com Keswani et al. \citep{6783445}, para o reuso de software dar retornos apropriados, o processo deve ser sistemático e planejado. Qualquer organização que implemente a reutilização de software deve identificar os melhores métodos e estratégias de reutilização para obter a máxima produtividade. A reutilização de software ajuda a evitar software de engenharia a partir do zero, pois usa módulos de software existentes. A reutilização de software, embora seja uma tarefa difícil, especialmente para softwares antigos sem padrões de projeto, pode melhorar significativamente a produtividade e a qualidade de um produto de software. Embora a reutilização de software não seja um novo campo, ela pode dar grandes retornos em curto período de tempo.


\section{Modularização}\label{sec:modularizacao} %artigo de claudio pagina 222 introdução


%A modularidade vem desempenhando um papel predominante estágios emergentes das disciplinas de arquitetura de software [13]. Engenheiros de software consideram modularidade como princípio base na comparação entre arquiteturas alternativas  e arquitetura degeneração [9]. De fato, os engenheiros de software são incentivados a arquitecturas, baseando-se numa multiplicidade de mecanismos de modularidade disponíveis em: 
%(i) Linguagens de descrição de arquitetura (ADLs), como ACME [8], 
%(ii) catálogos de arquitetônicos [2, 13], e 
%(iii) conhecem bem princípios de alto nível, como interfaces de componentes estreitos, acoplamento arquitectónico reduzido e semelhantes.


Conforme é frisado por Wickramaarachchi e Lai \citep{7062705}, o conceito de modularização na indústria de software tem uma longa história e tem sido utilizado para melhorar o processo de desenvolvimento de software em diferentes estágios. Os principais conceitos por trás da modularização do software foram introduzidos por pesquisadores pioneiros há quarenta anos, com uma notável contribuição feita por Melvin Conway e David Parnas, que tem representação notável na engenharia de software.


Modularizar um software é um bom padrão a ser adotado. Segundo Wickramaarachchi e Lai \citep{7062705}, a modularização é importante na identificação de dependências e reduz as dificuldades diante de uma possível necessidade de grandes alterações. De uma perspectiva da engenharia de software, uma modularização geralmente tem várias vantagens, tais como: tornar a complexidade do software mais gerenciável, facilitar o trabalho paralelo e tornar o software mais maleável para acomodar o futuro incerto que um software pode ter. O objetivo final da modularização do software é aumentar a produtividade ea qualidade do software. Tal conceito encontra-se bastante difundido e estái incorporado em linguagens de programação e ferramentas de software. O presente trabalho favorece ao uso da modularização de um software e até mesmo pode ser considerado um módulo a ser acoplado a qualquer software, mediante a compatibilidade.


Ao realizar a adoção do trabalho proposto, fica bastante evidente a "responsabilidade" do mesmo no escopo do projeto. Assim, é possível tratá-lo como um módulo do projeto que o usa, mesmo que seja consumido como um serviço.


\section{Aplicações web}\label{sec:apps_web}


A popularidade das soluções Web aumentou exponencialmente na última década e todos os dias cresce o número de pessoas usuárias desse tipo de software. E seguindo um padrão próprio, Kumar et al. \citep{7813710} sugerem que para o desenvolvimento web, deve-se manter a prática eficaz de produzir diagramas UML. A abordagem baseada na web oferece uma maneira fácil e eficaz para gerenciar e controlar o processo de desenvolvimento por meio de artefatos de modelagem. Tal abordagem pode ser usada quando há uma exigência de lidar com mudanças muito rápidas e grandes em requisitos de forma muito eficaz em muito menos tempo, gerando assim um menor impacto. 


Para atender à fomentada demanda de aplicativos web, é necessário adotar métodos de desenvolvimentos que sejam ágeis, eficientes e de fácil manutenção. Yu Ping et al. \cite{1372143} propõem o uso do modelo MVC (Model, View e Controller) no desenvolvimento para softwares web. O modelo apresentado tornou-se um padrão popular e divide o software em camadas com propósito definido, tornando-o de mais fácil manutenção.


O Ajax (Asynchronous Javascript and XML) revolucionou a web. Conforme demonstrado por Yuping \citep{6845605}, ao usar a tecnologia Ajax, podemos enriquecer a experiência do usuário em aplicações baseadas em navegador de internet, e fornecer uma variedade de aplicações interativas para atender às necessidade de humanização das aplicações.
Os aplicativos Ajax em execução no navegador se comunicam com um servidor Web de forma assíncrona e atualizam apenas uma parte da página.


No artigo de Tesarik et al. \citep{4664349}, temos que o design de software SPA (Single page application) é uma maneira interessante de criar um software Web numa uma única página. Essa solução de página única sem navegação funciona apenas com base em técnicas dinâmicas e assíncronas, como o AJAX (citado no parágrafo anterior). No entanto, esta abordagem coloca o desenvolvedor antes de alguns desafios substanciais.
Para projetar a interface do usuário que mostra as informações, é importante projetar corretamente a tela para manipular os dados do aplicativo numa única página. O design da página deve ser elaborado para maximizar a decomposição da página em componentes distintos que encapsulam os principais casos de uso.
Também recomenda-se que para elaborar os artefatos visuais, explore os mais ricos recursos na implementação, como HTML5, JavaScript, Ajax, CSS3, e outras tecnologias que se apliquem. Entretanto, desenvolver uma rica interface com o uso de diversas tecnologias/frameworks pode ocasionar um esforço maior para explorar as possibilidades.
Tais componentes da interface de uma aplicação SPA normalmente são alimentados mediante o consumo de uma API Rest via requisições AJAX. Assim, é possível particionar as responsabilidades de processamento do software entre cliente e servidor.


Como é possível denotar no trabalho de Salvadori e Siqueire \citep{7195633}, REST é uma arquitetura muito popular para integração de aplicativos web, que permite compartilhar e reutilizar informações através de sistemas. Aplicações de grande escala baseadas em no REST deve ser implementadas utilizando estratégias e mecanismos para produzir sistemas que sejam fáceis de desenvolver, reutilizar e manter. As interfaces de integração fornecidas por esses sistemas, chamadas de Web API, têm uma influência importante nas características da implementação resultante, pois a sua forma de resposta deve ser casada com a interface que o consome.
 
 
Os elementos comentados nessa seção encontram-se presentes neste trabalho de conclusão de curso. Juntos, montam a estrutura tecnológica necessária aliada à arquitetura adotada. Esses elementos seguem tendências atuais dos softwares web, fazendo com que esse trabalho esteja composto por tecnologias modernas que estão com boa aceitação no mercado.

% %% RiSE Latex Template - version 0.5
%%
%% RiSE's latex template for thesis and dissertations
%% http://risetemplate.sourceforge.net
%%
%% (c) 2012 Yguaratã Cerqueira Cavalcanti (yguarata@gmail.com)
%%          Vinicius Cardoso Garcia (vinicius.garcia@gmail.com)
%%
%% This document was initially based on UFPEThesis template, from Paulo Gustavo
%% S. Fonseca.
%%
%% ACKNOWLEDGEMENTS
%%
%% We would like to thanks the RiSE's researchers community, the 
%% students from Federal University of Pernambuco, and other users that have
%% been contributing to this projects with comments and patches.
%%
%% GENERAL INSTRUCTIONS
%%
%% We strongly recommend you to compile your documents using pdflatex command.
%% It is also recommend use the texlipse plugin for Eclipse to edit your documents.
%%
%% Options for \documentclass command:
%%         * Idiom
%%           pt   - Portguese (default)
%%           en   - English
%%
%%         * Text type
%%           bsc  - B.Sc. Thesis
%%           msc  - M.Sc. Thesis (default)
%%           qual - PHD qualification (not tested yet)
%%           prop - PHD proposal (not tested yet)
%%           phd  - PHD thesis
%%
%%         * Media
%%           scr  - to eletronic version (PDF) / see the users guide
%%
%%         * Pagination
%%           oneside - unique face press
%%           twoside - two faces press
%%
%%		   * Line spacing
%%           singlespacing  - the same as using \linespread{1}
%%           onehalfspacing - the same as using \linespread{1.3}
%%           doublespacing  - the same as using \linespread{1.6}
%%
%% Reference commands. Use the following commands to make references in your
%% text:
%%          \figref  -- for Figure reference
%%          \tabref  -- for Table reference
%%          \eqnref  -- for equation reference
%%          \chapref -- for chapter reference
%%          \secref  -- for section reference
%%          \appref  -- for appendix reference
%%          \axiref  -- for axiom reference
%%          \conjref -- for conjecture reference
%%          \defref  -- for definition reference
%%          \lemref  -- for lemma reference
%%          \theoref -- for theorem reference
%%          \corref  -- for corollary reference
%%          \propref -- for proprosition reference
%%          \pgref   -- for page reference
%%
%%          Example: See \chapref{chap:introduction}. It will produce 
%%                   'See Chapter 1', in case of English language.

\documentclass[pt,twoside,onehalfspacing,bsc]{risethesis}

\usepackage[sort,square]{natbib} %package for bibliography
\bibliographystyle{ieeetr} %authordate1

\usepackage[utf8]{inputenc}
\usepackage[brazilian]{babel}
\usepackage[T1]{fontenc}

\usepackage{graphicx}

%% Change the following pdf author attribute name to your name.
\usepackage[linkcolor=blue,citecolor=blue,urlcolor=blue,colorlinks,pdfpagelabels,pdftitle={Bruno Cabral's Bachelor Thesis},pdfauthor={Bruno Cabral}]{hyperref}

\address{SALVADOR}

\universitypt{Universidade Federal da Bahia}
\universityen{Federal University of Bahia}

\departmentpt{Depertamento de Ciência da Computação}
\departmenten{Computer Science Department}

\programpt{Programa Multiinstitucional de Pós-graduação em Ciência da Computação}
\programen{Graduate in Computer Science}

\majorfieldpt{Ciência da Computação}
\majorfielden{Computer Science}

\title{Sistema de apoio à Pós graduação - UFBA}
\date{Outubro/2016}

\author{Victor de Azevedo Nunes}
\adviser{Ivan do Carmo Machado}

\begin{document}

\frontmatter
\frontpage
\presentationpage

\begin{dedicatory}
Eu dedico esta dissertação...
%I dedicate this dissertation to my family, girlfriend, friends and
%professors who gave me all necessary support to get here.
\end{dedicatory}

\acknowledgements
Meus agradecimentos...

\begin{epigraph}[]{Edward V Berard}
Walking on water and developing software from a specification are easy if both are frozen
\end{epigraph}

\resumo
% Escreva seu resumo no arquivo resumo.tex
\input{resumo}

\abstract
% Write your abstract in a file called abstract.tex
\input{abstract}

% Summary (tables of contents)
\tableofcontents

% List of figures
\listoffigures

% List of tables
\listoftables

% List of acronyms
% Acronyms manual: http://linorg.usp.br/CTAN/macros/latex/contrib/acronym/acronym.pdf
\listofacronyms
\input{acronyms}

% List of listings
%\lstlistoflistings

\mainmatter

\include{chapters/intro}
\include{chapters/referencial_teorico}

% \include{chapters/introduction/main}
% \include{chapters/background/main}
% \include{chapters/proposed_solution/main}
% \include{chapters/experiment/main}
% \include{chapters/conclusion/main}

%\bibliographystyle{natbib}
%\addcontentsline{toc}{chapter}{\bibliographytocname}
\bibliography{references}

% Appendix
\clearpage
\addappheadtotoc
\appendix
\appendixpage
% \include{appendix/experiment-instruments}

\end{document}
% %% RiSE Latex Template - version 0.5
%%
%% RiSE's latex template for thesis and dissertations
%% http://risetemplate.sourceforge.net
%%
%% (c) 2012 Yguaratã Cerqueira Cavalcanti (yguarata@gmail.com)
%%          Vinicius Cardoso Garcia (vinicius.garcia@gmail.com)
%%
%% This document was initially based on UFPEThesis template, from Paulo Gustavo
%% S. Fonseca.
%%
%% ACKNOWLEDGEMENTS
%%
%% We would like to thanks the RiSE's researchers community, the 
%% students from Federal University of Pernambuco, and other users that have
%% been contributing to this projects with comments and patches.
%%
%% GENERAL INSTRUCTIONS
%%
%% We strongly recommend you to compile your documents using pdflatex command.
%% It is also recommend use the texlipse plugin for Eclipse to edit your documents.
%%
%% Options for \documentclass command:
%%         * Idiom
%%           pt   - Portguese (default)
%%           en   - English
%%
%%         * Text type
%%           bsc  - B.Sc. Thesis
%%           msc  - M.Sc. Thesis (default)
%%           qual - PHD qualification (not tested yet)
%%           prop - PHD proposal (not tested yet)
%%           phd  - PHD thesis
%%
%%         * Media
%%           scr  - to eletronic version (PDF) / see the users guide
%%
%%         * Pagination
%%           oneside - unique face press
%%           twoside - two faces press
%%
%%		   * Line spacing
%%           singlespacing  - the same as using \linespread{1}
%%           onehalfspacing - the same as using \linespread{1.3}
%%           doublespacing  - the same as using \linespread{1.6}
%%
%% Reference commands. Use the following commands to make references in your
%% text:
%%          \figref  -- for Figure reference
%%          \tabref  -- for Table reference
%%          \eqnref  -- for equation reference
%%          \chapref -- for chapter reference
%%          \secref  -- for section reference
%%          \appref  -- for appendix reference
%%          \axiref  -- for axiom reference
%%          \conjref -- for conjecture reference
%%          \defref  -- for definition reference
%%          \lemref  -- for lemma reference
%%          \theoref -- for theorem reference
%%          \corref  -- for corollary reference
%%          \propref -- for proprosition reference
%%          \pgref   -- for page reference
%%
%%          Example: See \chapref{chap:introduction}. It will produce 
%%                   'See Chapter 1', in case of English language.

\documentclass[pt,twoside,onehalfspacing,bsc]{risethesis}

\usepackage[sort,square]{natbib} %package for bibliography
\bibliographystyle{ieeetr} %authordate1

\usepackage[utf8]{inputenc}
\usepackage[brazilian]{babel}
\usepackage[T1]{fontenc}

\usepackage{graphicx}

%% Change the following pdf author attribute name to your name.
\usepackage[linkcolor=blue,citecolor=blue,urlcolor=blue,colorlinks,pdfpagelabels,pdftitle={Bruno Cabral's Bachelor Thesis},pdfauthor={Bruno Cabral}]{hyperref}

\address{SALVADOR}

\universitypt{Universidade Federal da Bahia}
\universityen{Federal University of Bahia}

\departmentpt{Depertamento de Ciência da Computação}
\departmenten{Computer Science Department}

\programpt{Programa Multiinstitucional de Pós-graduação em Ciência da Computação}
\programen{Graduate in Computer Science}

\majorfieldpt{Ciência da Computação}
\majorfielden{Computer Science}

\title{Sistema de apoio à Pós graduação - UFBA}
\date{Outubro/2016}

\author{Victor de Azevedo Nunes}
\adviser{Ivan do Carmo Machado}

\begin{document}

\frontmatter
\frontpage
\presentationpage

\begin{dedicatory}
Eu dedico esta dissertação...
%I dedicate this dissertation to my family, girlfriend, friends and
%professors who gave me all necessary support to get here.
\end{dedicatory}

\acknowledgements
Meus agradecimentos...

\begin{epigraph}[]{Edward V Berard}
Walking on water and developing software from a specification are easy if both are frozen
\end{epigraph}

\resumo
% Escreva seu resumo no arquivo resumo.tex
\input{resumo}

\abstract
% Write your abstract in a file called abstract.tex
\input{abstract}

% Summary (tables of contents)
\tableofcontents

% List of figures
\listoffigures

% List of tables
\listoftables

% List of acronyms
% Acronyms manual: http://linorg.usp.br/CTAN/macros/latex/contrib/acronym/acronym.pdf
\listofacronyms
\input{acronyms}

% List of listings
%\lstlistoflistings

\mainmatter

\include{chapters/intro}
\include{chapters/referencial_teorico}

% \include{chapters/introduction/main}
% \include{chapters/background/main}
% \include{chapters/proposed_solution/main}
% \include{chapters/experiment/main}
% \include{chapters/conclusion/main}

%\bibliographystyle{natbib}
%\addcontentsline{toc}{chapter}{\bibliographytocname}
\bibliography{references}

% Appendix
\clearpage
\addappheadtotoc
\appendix
\appendixpage
% \include{appendix/experiment-instruments}

\end{document}
% %% RiSE Latex Template - version 0.5
%%
%% RiSE's latex template for thesis and dissertations
%% http://risetemplate.sourceforge.net
%%
%% (c) 2012 Yguaratã Cerqueira Cavalcanti (yguarata@gmail.com)
%%          Vinicius Cardoso Garcia (vinicius.garcia@gmail.com)
%%
%% This document was initially based on UFPEThesis template, from Paulo Gustavo
%% S. Fonseca.
%%
%% ACKNOWLEDGEMENTS
%%
%% We would like to thanks the RiSE's researchers community, the 
%% students from Federal University of Pernambuco, and other users that have
%% been contributing to this projects with comments and patches.
%%
%% GENERAL INSTRUCTIONS
%%
%% We strongly recommend you to compile your documents using pdflatex command.
%% It is also recommend use the texlipse plugin for Eclipse to edit your documents.
%%
%% Options for \documentclass command:
%%         * Idiom
%%           pt   - Portguese (default)
%%           en   - English
%%
%%         * Text type
%%           bsc  - B.Sc. Thesis
%%           msc  - M.Sc. Thesis (default)
%%           qual - PHD qualification (not tested yet)
%%           prop - PHD proposal (not tested yet)
%%           phd  - PHD thesis
%%
%%         * Media
%%           scr  - to eletronic version (PDF) / see the users guide
%%
%%         * Pagination
%%           oneside - unique face press
%%           twoside - two faces press
%%
%%		   * Line spacing
%%           singlespacing  - the same as using \linespread{1}
%%           onehalfspacing - the same as using \linespread{1.3}
%%           doublespacing  - the same as using \linespread{1.6}
%%
%% Reference commands. Use the following commands to make references in your
%% text:
%%          \figref  -- for Figure reference
%%          \tabref  -- for Table reference
%%          \eqnref  -- for equation reference
%%          \chapref -- for chapter reference
%%          \secref  -- for section reference
%%          \appref  -- for appendix reference
%%          \axiref  -- for axiom reference
%%          \conjref -- for conjecture reference
%%          \defref  -- for definition reference
%%          \lemref  -- for lemma reference
%%          \theoref -- for theorem reference
%%          \corref  -- for corollary reference
%%          \propref -- for proprosition reference
%%          \pgref   -- for page reference
%%
%%          Example: See \chapref{chap:introduction}. It will produce 
%%                   'See Chapter 1', in case of English language.

\documentclass[pt,twoside,onehalfspacing,bsc]{risethesis}

\usepackage[sort,square]{natbib} %package for bibliography
\bibliographystyle{ieeetr} %authordate1

\usepackage[utf8]{inputenc}
\usepackage[brazilian]{babel}
\usepackage[T1]{fontenc}

\usepackage{graphicx}

%% Change the following pdf author attribute name to your name.
\usepackage[linkcolor=blue,citecolor=blue,urlcolor=blue,colorlinks,pdfpagelabels,pdftitle={Bruno Cabral's Bachelor Thesis},pdfauthor={Bruno Cabral}]{hyperref}

\address{SALVADOR}

\universitypt{Universidade Federal da Bahia}
\universityen{Federal University of Bahia}

\departmentpt{Depertamento de Ciência da Computação}
\departmenten{Computer Science Department}

\programpt{Programa Multiinstitucional de Pós-graduação em Ciência da Computação}
\programen{Graduate in Computer Science}

\majorfieldpt{Ciência da Computação}
\majorfielden{Computer Science}

\title{Sistema de apoio à Pós graduação - UFBA}
\date{Outubro/2016}

\author{Victor de Azevedo Nunes}
\adviser{Ivan do Carmo Machado}

\begin{document}

\frontmatter
\frontpage
\presentationpage

\begin{dedicatory}
Eu dedico esta dissertação...
%I dedicate this dissertation to my family, girlfriend, friends and
%professors who gave me all necessary support to get here.
\end{dedicatory}

\acknowledgements
Meus agradecimentos...

\begin{epigraph}[]{Edward V Berard}
Walking on water and developing software from a specification are easy if both are frozen
\end{epigraph}

\resumo
% Escreva seu resumo no arquivo resumo.tex
\input{resumo}

\abstract
% Write your abstract in a file called abstract.tex
\input{abstract}

% Summary (tables of contents)
\tableofcontents

% List of figures
\listoffigures

% List of tables
\listoftables

% List of acronyms
% Acronyms manual: http://linorg.usp.br/CTAN/macros/latex/contrib/acronym/acronym.pdf
\listofacronyms
\input{acronyms}

% List of listings
%\lstlistoflistings

\mainmatter

\include{chapters/intro}
\include{chapters/referencial_teorico}

% \include{chapters/introduction/main}
% \include{chapters/background/main}
% \include{chapters/proposed_solution/main}
% \include{chapters/experiment/main}
% \include{chapters/conclusion/main}

%\bibliographystyle{natbib}
%\addcontentsline{toc}{chapter}{\bibliographytocname}
\bibliography{references}

% Appendix
\clearpage
\addappheadtotoc
\appendix
\appendixpage
% \include{appendix/experiment-instruments}

\end{document}
% %% RiSE Latex Template - version 0.5
%%
%% RiSE's latex template for thesis and dissertations
%% http://risetemplate.sourceforge.net
%%
%% (c) 2012 Yguaratã Cerqueira Cavalcanti (yguarata@gmail.com)
%%          Vinicius Cardoso Garcia (vinicius.garcia@gmail.com)
%%
%% This document was initially based on UFPEThesis template, from Paulo Gustavo
%% S. Fonseca.
%%
%% ACKNOWLEDGEMENTS
%%
%% We would like to thanks the RiSE's researchers community, the 
%% students from Federal University of Pernambuco, and other users that have
%% been contributing to this projects with comments and patches.
%%
%% GENERAL INSTRUCTIONS
%%
%% We strongly recommend you to compile your documents using pdflatex command.
%% It is also recommend use the texlipse plugin for Eclipse to edit your documents.
%%
%% Options for \documentclass command:
%%         * Idiom
%%           pt   - Portguese (default)
%%           en   - English
%%
%%         * Text type
%%           bsc  - B.Sc. Thesis
%%           msc  - M.Sc. Thesis (default)
%%           qual - PHD qualification (not tested yet)
%%           prop - PHD proposal (not tested yet)
%%           phd  - PHD thesis
%%
%%         * Media
%%           scr  - to eletronic version (PDF) / see the users guide
%%
%%         * Pagination
%%           oneside - unique face press
%%           twoside - two faces press
%%
%%		   * Line spacing
%%           singlespacing  - the same as using \linespread{1}
%%           onehalfspacing - the same as using \linespread{1.3}
%%           doublespacing  - the same as using \linespread{1.6}
%%
%% Reference commands. Use the following commands to make references in your
%% text:
%%          \figref  -- for Figure reference
%%          \tabref  -- for Table reference
%%          \eqnref  -- for equation reference
%%          \chapref -- for chapter reference
%%          \secref  -- for section reference
%%          \appref  -- for appendix reference
%%          \axiref  -- for axiom reference
%%          \conjref -- for conjecture reference
%%          \defref  -- for definition reference
%%          \lemref  -- for lemma reference
%%          \theoref -- for theorem reference
%%          \corref  -- for corollary reference
%%          \propref -- for proprosition reference
%%          \pgref   -- for page reference
%%
%%          Example: See \chapref{chap:introduction}. It will produce 
%%                   'See Chapter 1', in case of English language.

\documentclass[pt,twoside,onehalfspacing,bsc]{risethesis}

\usepackage[sort,square]{natbib} %package for bibliography
\bibliographystyle{ieeetr} %authordate1

\usepackage[utf8]{inputenc}
\usepackage[brazilian]{babel}
\usepackage[T1]{fontenc}

\usepackage{graphicx}

%% Change the following pdf author attribute name to your name.
\usepackage[linkcolor=blue,citecolor=blue,urlcolor=blue,colorlinks,pdfpagelabels,pdftitle={Bruno Cabral's Bachelor Thesis},pdfauthor={Bruno Cabral}]{hyperref}

\address{SALVADOR}

\universitypt{Universidade Federal da Bahia}
\universityen{Federal University of Bahia}

\departmentpt{Depertamento de Ciência da Computação}
\departmenten{Computer Science Department}

\programpt{Programa Multiinstitucional de Pós-graduação em Ciência da Computação}
\programen{Graduate in Computer Science}

\majorfieldpt{Ciência da Computação}
\majorfielden{Computer Science}

\title{Sistema de apoio à Pós graduação - UFBA}
\date{Outubro/2016}

\author{Victor de Azevedo Nunes}
\adviser{Ivan do Carmo Machado}

\begin{document}

\frontmatter
\frontpage
\presentationpage

\begin{dedicatory}
Eu dedico esta dissertação...
%I dedicate this dissertation to my family, girlfriend, friends and
%professors who gave me all necessary support to get here.
\end{dedicatory}

\acknowledgements
Meus agradecimentos...

\begin{epigraph}[]{Edward V Berard}
Walking on water and developing software from a specification are easy if both are frozen
\end{epigraph}

\resumo
% Escreva seu resumo no arquivo resumo.tex
\input{resumo}

\abstract
% Write your abstract in a file called abstract.tex
\input{abstract}

% Summary (tables of contents)
\tableofcontents

% List of figures
\listoffigures

% List of tables
\listoftables

% List of acronyms
% Acronyms manual: http://linorg.usp.br/CTAN/macros/latex/contrib/acronym/acronym.pdf
\listofacronyms
\input{acronyms}

% List of listings
%\lstlistoflistings

\mainmatter

\include{chapters/intro}
\include{chapters/referencial_teorico}

% \include{chapters/introduction/main}
% \include{chapters/background/main}
% \include{chapters/proposed_solution/main}
% \include{chapters/experiment/main}
% \include{chapters/conclusion/main}

%\bibliographystyle{natbib}
%\addcontentsline{toc}{chapter}{\bibliographytocname}
\bibliography{references}

% Appendix
\clearpage
\addappheadtotoc
\appendix
\appendixpage
% \include{appendix/experiment-instruments}

\end{document}
% %% RiSE Latex Template - version 0.5
%%
%% RiSE's latex template for thesis and dissertations
%% http://risetemplate.sourceforge.net
%%
%% (c) 2012 Yguaratã Cerqueira Cavalcanti (yguarata@gmail.com)
%%          Vinicius Cardoso Garcia (vinicius.garcia@gmail.com)
%%
%% This document was initially based on UFPEThesis template, from Paulo Gustavo
%% S. Fonseca.
%%
%% ACKNOWLEDGEMENTS
%%
%% We would like to thanks the RiSE's researchers community, the 
%% students from Federal University of Pernambuco, and other users that have
%% been contributing to this projects with comments and patches.
%%
%% GENERAL INSTRUCTIONS
%%
%% We strongly recommend you to compile your documents using pdflatex command.
%% It is also recommend use the texlipse plugin for Eclipse to edit your documents.
%%
%% Options for \documentclass command:
%%         * Idiom
%%           pt   - Portguese (default)
%%           en   - English
%%
%%         * Text type
%%           bsc  - B.Sc. Thesis
%%           msc  - M.Sc. Thesis (default)
%%           qual - PHD qualification (not tested yet)
%%           prop - PHD proposal (not tested yet)
%%           phd  - PHD thesis
%%
%%         * Media
%%           scr  - to eletronic version (PDF) / see the users guide
%%
%%         * Pagination
%%           oneside - unique face press
%%           twoside - two faces press
%%
%%		   * Line spacing
%%           singlespacing  - the same as using \linespread{1}
%%           onehalfspacing - the same as using \linespread{1.3}
%%           doublespacing  - the same as using \linespread{1.6}
%%
%% Reference commands. Use the following commands to make references in your
%% text:
%%          \figref  -- for Figure reference
%%          \tabref  -- for Table reference
%%          \eqnref  -- for equation reference
%%          \chapref -- for chapter reference
%%          \secref  -- for section reference
%%          \appref  -- for appendix reference
%%          \axiref  -- for axiom reference
%%          \conjref -- for conjecture reference
%%          \defref  -- for definition reference
%%          \lemref  -- for lemma reference
%%          \theoref -- for theorem reference
%%          \corref  -- for corollary reference
%%          \propref -- for proprosition reference
%%          \pgref   -- for page reference
%%
%%          Example: See \chapref{chap:introduction}. It will produce 
%%                   'See Chapter 1', in case of English language.

\documentclass[pt,twoside,onehalfspacing,bsc]{risethesis}

\usepackage[sort,square]{natbib} %package for bibliography
\bibliographystyle{ieeetr} %authordate1

\usepackage[utf8]{inputenc}
\usepackage[brazilian]{babel}
\usepackage[T1]{fontenc}

\usepackage{graphicx}

%% Change the following pdf author attribute name to your name.
\usepackage[linkcolor=blue,citecolor=blue,urlcolor=blue,colorlinks,pdfpagelabels,pdftitle={Bruno Cabral's Bachelor Thesis},pdfauthor={Bruno Cabral}]{hyperref}

\address{SALVADOR}

\universitypt{Universidade Federal da Bahia}
\universityen{Federal University of Bahia}

\departmentpt{Depertamento de Ciência da Computação}
\departmenten{Computer Science Department}

\programpt{Programa Multiinstitucional de Pós-graduação em Ciência da Computação}
\programen{Graduate in Computer Science}

\majorfieldpt{Ciência da Computação}
\majorfielden{Computer Science}

\title{Sistema de apoio à Pós graduação - UFBA}
\date{Outubro/2016}

\author{Victor de Azevedo Nunes}
\adviser{Ivan do Carmo Machado}

\begin{document}

\frontmatter
\frontpage
\presentationpage

\begin{dedicatory}
Eu dedico esta dissertação...
%I dedicate this dissertation to my family, girlfriend, friends and
%professors who gave me all necessary support to get here.
\end{dedicatory}

\acknowledgements
Meus agradecimentos...

\begin{epigraph}[]{Edward V Berard}
Walking on water and developing software from a specification are easy if both are frozen
\end{epigraph}

\resumo
% Escreva seu resumo no arquivo resumo.tex
\input{resumo}

\abstract
% Write your abstract in a file called abstract.tex
\input{abstract}

% Summary (tables of contents)
\tableofcontents

% List of figures
\listoffigures

% List of tables
\listoftables

% List of acronyms
% Acronyms manual: http://linorg.usp.br/CTAN/macros/latex/contrib/acronym/acronym.pdf
\listofacronyms
\input{acronyms}

% List of listings
%\lstlistoflistings

\mainmatter

\include{chapters/intro}
\include{chapters/referencial_teorico}

% \include{chapters/introduction/main}
% \include{chapters/background/main}
% \include{chapters/proposed_solution/main}
% \include{chapters/experiment/main}
% \include{chapters/conclusion/main}

%\bibliographystyle{natbib}
%\addcontentsline{toc}{chapter}{\bibliographytocname}
\bibliography{references}

% Appendix
\clearpage
\addappheadtotoc
\appendix
\appendixpage
% \include{appendix/experiment-instruments}

\end{document}

%\bibliographystyle{natbib}
%\addcontentsline{toc}{chapter}{\bibliographytocname}
\bibliography{references}

% Appendix
\clearpage
\addappheadtotoc
\appendix
\appendixpage
% \include{appendix/experiment-instruments}

\end{document}
% %% RiSE Latex Template - version 0.5
%%
%% RiSE's latex template for thesis and dissertations
%% http://risetemplate.sourceforge.net
%%
%% (c) 2012 Yguaratã Cerqueira Cavalcanti (yguarata@gmail.com)
%%          Vinicius Cardoso Garcia (vinicius.garcia@gmail.com)
%%
%% This document was initially based on UFPEThesis template, from Paulo Gustavo
%% S. Fonseca.
%%
%% ACKNOWLEDGEMENTS
%%
%% We would like to thanks the RiSE's researchers community, the 
%% students from Federal University of Pernambuco, and other users that have
%% been contributing to this projects with comments and patches.
%%
%% GENERAL INSTRUCTIONS
%%
%% We strongly recommend you to compile your documents using pdflatex command.
%% It is also recommend use the texlipse plugin for Eclipse to edit your documents.
%%
%% Options for \documentclass command:
%%         * Idiom
%%           pt   - Portguese (default)
%%           en   - English
%%
%%         * Text type
%%           bsc  - B.Sc. Thesis
%%           msc  - M.Sc. Thesis (default)
%%           qual - PHD qualification (not tested yet)
%%           prop - PHD proposal (not tested yet)
%%           phd  - PHD thesis
%%
%%         * Media
%%           scr  - to eletronic version (PDF) / see the users guide
%%
%%         * Pagination
%%           oneside - unique face press
%%           twoside - two faces press
%%
%%		   * Line spacing
%%           singlespacing  - the same as using \linespread{1}
%%           onehalfspacing - the same as using \linespread{1.3}
%%           doublespacing  - the same as using \linespread{1.6}
%%
%% Reference commands. Use the following commands to make references in your
%% text:
%%          \figref  -- for Figure reference
%%          \tabref  -- for Table reference
%%          \eqnref  -- for equation reference
%%          \chapref -- for chapter reference
%%          \secref  -- for section reference
%%          \appref  -- for appendix reference
%%          \axiref  -- for axiom reference
%%          \conjref -- for conjecture reference
%%          \defref  -- for definition reference
%%          \lemref  -- for lemma reference
%%          \theoref -- for theorem reference
%%          \corref  -- for corollary reference
%%          \propref -- for proprosition reference
%%          \pgref   -- for page reference
%%
%%          Example: See \chapref{chap:introduction}. It will produce 
%%                   'See Chapter 1', in case of English language.

\documentclass[pt,twoside,onehalfspacing,bsc]{risethesis}

\usepackage[sort,square]{natbib} %package for bibliography
\bibliographystyle{ieeetr} %authordate1

\usepackage[utf8]{inputenc}
\usepackage[brazilian]{babel}
\usepackage[T1]{fontenc}

\usepackage{graphicx}

%% Change the following pdf author attribute name to your name.
\usepackage[linkcolor=blue,citecolor=blue,urlcolor=blue,colorlinks,pdfpagelabels,pdftitle={Bruno Cabral's Bachelor Thesis},pdfauthor={Bruno Cabral}]{hyperref}

\address{SALVADOR}

\universitypt{Universidade Federal da Bahia}
\universityen{Federal University of Bahia}

\departmentpt{Depertamento de Ciência da Computação}
\departmenten{Computer Science Department}

\programpt{Programa Multiinstitucional de Pós-graduação em Ciência da Computação}
\programen{Graduate in Computer Science}

\majorfieldpt{Ciência da Computação}
\majorfielden{Computer Science}

\title{Sistema de apoio à Pós graduação - UFBA}
\date{Outubro/2016}

\author{Victor de Azevedo Nunes}
\adviser{Ivan do Carmo Machado}

\begin{document}

\frontmatter
\frontpage
\presentationpage

\begin{dedicatory}
Eu dedico esta dissertação...
%I dedicate this dissertation to my family, girlfriend, friends and
%professors who gave me all necessary support to get here.
\end{dedicatory}

\acknowledgements
Meus agradecimentos...

\begin{epigraph}[]{Edward V Berard}
Walking on water and developing software from a specification are easy if both are frozen
\end{epigraph}

\resumo
% Escreva seu resumo no arquivo resumo.tex
Este trabalho de conclusão de curso utiliza elementos da engenharia de software para propor um software como serviço a fim de otimizar o processo de construção e manutenção dos softwares. Assim, o objetivo deste SaaS é gerenciar as permissões de acesso de sistemas cliente, provendo o reuso de software e padronizando as soluções. O sistema proposto neste trabalho encontra-se implementado e disponível no Github, e traz fundamentos sobre a arquitetura e tecnologias utilizadas, bem como avaliações sobre possibilidades de utilização do mesmo em diversos ambientes, como Web e mobile.

\begin{keywords}
Software; Reuso; SaaS; Web; PHP
\end{keywords}

\abstract
% Write your abstract in a file called abstract.tex
My abstract...

\begin{keywords}
key words...
\end{keywords}

% Summary (tables of contents)
\tableofcontents

% List of figures
\listoffigures

% List of tables
\listoftables

% List of acronyms
% Acronyms manual: http://linorg.usp.br/CTAN/macros/latex/contrib/acronym/acronym.pdf
\listofacronyms
\begin{acronym}[ACRONYM] 
% Change the word ACRONYM above to change the acronym column width.
% The column width is equals to the width of the word that you put.
% Read the manual about acronym package for more examples:
%   http://linorg.usp.br/CTAN/macros/latex/contrib/acronym/acronym.pdf

\acro{SPA}{Single Page Application}
\acro{JSON}{Javascript Object Notation}
\acro{PHP}{PHP: Hypertext Preprocessor}
\acro{SaaS}{Software as a Service}
\acro{ERP}{Enterprise Resource Planning}
\acro{QoS}{Quality of Service}
\acro{UML}{Unified Modeling Language}
\acro{MVC}{Model-View-Controller}
\acro{Ajax}{Asynchronous Javascript and XML}
\acro{HTML}{HyperText Markup Language}
\acro{CSS}{Cascading Style Sheets}
\acro{API}{Application Programming Interface}
\acro{DOM}{Document Object Model}
\acro{BPMN}{Business Process Model and Notation}
\acro{REST}{Representational State Transfer}

\end{acronym}

% List of listings
%\lstlistoflistings

\mainmatter

\chapter{Introdução}

\section{Motivação}

Organizar os procedimentos de um processo sempre nos traz vantagens. Apesar de no processo de implantação de um sistema, o mesmo burocratizar o processo, com o tempo temos o retorno da dedicação para a inserção dos dados. Com um certo volume de dados, é possível estruturar informações que num processo manual são difíceis de serem enxergadas. Assim, é possível depender menos das pessoas que organizam o processo, pois o legado de informações não estará mais somente na mente de alguns, mas sim documentado nos dados do sistema.

Além de colaborar na organização, também haverá uma grande colaboração no tempo gasto na gestão. Lidar com muitos papéis e confiar na mente humana para guardar informações, não é uma alternativa muito segura devido ao fato que as pessoas sempre estão sujeitas a sair do processo e levar contigo a experiência obtida. Experiência essa que faz com que os procedimentos sejam executados de forma mais eficiente. Entretanto, com um sistema inteligente, é possível auxiliar e tornar mais ágil a execução das tarefas.


\section{Problema}


De acordo com funcionários ligados ao o setor de pós graduação da UFBA, entrevistados a fim de um maior entendimento do cenário, apesar das semelhanças estruturais, a pós graduação gerida de forma diferencia da graduação. FULANO afirma que devido ao fato de não ter a mesma visibilidade, não tem acesso aos mesmos recursos de gestão acadêmica da graduação. O professores não executam somente atividades dentro da sala de aula, também tem diversas outras ocupações no setor. E muitos procedimentos realizados extra classe ainda se encontram sendo realizados de forma manual, estando mais vulnerável ao erro ou até mesmo à violação do processo. Também ocorre um grande desperdício de tempo pelos professores e gestores da área, devido ao diversos processos ainda realizados de forma manual, sem a devida documentação. Segundo FULANO, também entrevistado, esse tempo perdido implica numa redução da eficiência na sala de aula, pois o professor acaba por ter menos tempo disponível para o planejamento das atividades, o que gera impactos negativos aos alunos.


\section{Objetivos} %<o que deve ser feito/entregue>


Devido aos muitos processos sendo resolvidos de forma manual, propõe-se com solução um sistema moderno, arquitetado para ter funcionamento na web e com um módulo mobile, a fim de fornecer informações de forma rápida e eficiente para os professores através de notificações, já que o acesso à internet móvel é comum entre os possíveis usuários do sistema em questão.
O principal requisito para o sistema seria dispor recursos para reduzir o tempo desperdiçado pelos professores durante as atividades extra classe.


\section{Metodologia} %<como será feito | como resolver o problema apontado inicialmente>


%<analise de literatura | design | implementação | validação>
Baseando-se nas tecnologias gratuitas em alta no cenário atual do desenvolvimento web, dispomos de algumas opções eficientes para a implementação da solução. Dentre as possibilidades, considerando a facilidade para futura manutenção e continuidade do projeto, tende-se a optar por uma tecnologia popular. Como linguagem de programação, adota-se o PHP. A escolha é fundamentada de acordo com a pesquisa da RedMonk de 2015, que evidencia o uso das linguagens de programação de acordo com as discussões no StackOverflow e repositórios no GitHub. É possível constatar a popularidade do PHP no cenário atual com o gráfico da pesquisa citada, na qual o PHP é apresentado na terceira colocação, apenas atrás do lider JavaScript e do segundo colocado, o Java.

\begin{figure}
	\label{fig:graficoRedmonk}
	\includegraphics[width=1\textwidth]{img/grafico_redmonk}
	\caption{Ranking das liguagens de programação no Stack Overflow e Github}
\end{figure}


Ainda assim, para compor a interface do dado projeto, também ocorrerá o uso do líder JavaScript de forma intensa, provendo o elo com o as informações gerenciadas pelo PHP.


Entretanto, não seria inteligente desenvolver um sistema completo sem o auxílio de um framework. Dentre os frameworks disponíveis para PHP, hoje o destaque está com o Laravel, que se encontra no topo dentre os mais utilizados no momento. 


A WebHostFace, uma empresa de hospedagem, compilou várias estatísticas para criar um infográfico mostrando os frameworks PHP mais populares de 2015. Utilizando informações sobre os próprios clientes, o Google Trends, estatísticas de repositórios do GitHub e a pesquisa do SitePoint “Best PHP Frameworks 2015”, a WebHostFace elaborou o seguinte infográfico: 

\begin{figure}
	\label{fig:graficoWebhostface}
	\includegraphics[width=1\textwidth]{img/infografico_webhostface}
	\caption{Infográfico da WebhostFace, exibindo a popularidade dos Frameworks PHP em 2015}
\end{figure}

Assim, tem-se a evidência que o Laravel em 2015 teve a maior popularidade em projetos pessoais e tem a maior comunidade entre os concorrentes, o que o torna uma boa escolha para a escrita de um software que será continuado por terceiros.


Para elaborar os recursos de interface e integrar ao back-end PHP do sistema, será adotado o já conhecido AngularJS, ferramenta sólida e conhecida no aspecto em questão. 


Dados coletados via Google Trends, que propõe comparações entre termos pesquisados, revela a popularidade do AngularJs diante de alguns dos principais concorrentes. O gráfico abaixo evidencia o cenário.


%Como mostra a Figura \ref{fig:graficoGoogleTrendsFerramentasFront}. 
\begin{figure}
	\label{fig:graficoGoogleTrendsFerramentasFront}
	\includegraphics[width=1\textwidth]{img/grafico_ferramentas_front}
	\caption{Gráfico do Google Trends exibindo as pesquisas por ferramentas front-end}
\end{figure}


Junto ao Angular JS, será utilizada a agradável tendência de interface do Material Design da Google, que propõe layouts limpos e otimizados já conhecidos pelos usuários de smartphones Android. 


Para a elaboração da plataforma mobile do projeto, será utilizado o Ionic Framework, muito difundido e bastante pesquisado na área, o que fica evidenciado com o gráfico de pesquisbaixo, coletado via Google Trends buscando por frameworks de desenvolvimento híbrido mobile.


\begin{figure}
	\label{fig:graficoGoogleTrendsFerramentasHibridasMobile}
	\includegraphics[width=1\textwidth]{img/grafico_ferramentas_hibridas_mobile}
	\caption{Gráfico do Google Trends exibindo as pesquisas por Frameworks híbridos mobile}
\end{figure}	

Para layout da interface mobile, também será aplicado a tendência do Material Design, a fim de propor uma harmonia entre o módulo web e mobile para os usuários


\section{Resultados Esperados}


Como fruto de um sistema para pós-graduação da UFBA, espera-se que os professores tenham mais recursos para integrar as atividades e também prover melhores condições para acompanhamento da vida acadêmica dos alunos em questão. Também, que os novos colaboradores que entrarem no processo tenham facilidade de compreender o fluxo do setor ao navegar pelo sistema proposto.


\section{Fora de Escopo}


Interação com os alunos devido às complicações para realizar a integração com o sistema empregado na UFBA, gerenciado pela XXXXXX, o que causaria uma inviabilidade no projeto devido à necessidade de entrega do produto ser mais forte que o tempo necessário para executar o processo de obtenção de acesso ao sistema legado para realizar a integração.


\section{Estrutura do Trabalho}


<breve resumo sobre os capítulos do TCC>
\chapter{Referencial Teórico}


Projetar o desenvolvimento de um software requer muito planejamento, pois as falhas iniciais podem custar bastante caro ou até mesmo inviabilizar a continuação de um projeto. Assim, a escolha da arquitetura ideal para a aplicabilidade é essencial na concepção de um produto de software. 
De todo o modo, sempre busca-se fazer mais com menos. Diante de tal filosofia, temos neste capítulo, uma breve discussão sobre alguns elementos de projeto e arquitetura de software, a fim de contextualizar este trabalho de conclusão de curso.
O capítulo corrente é composto por quatro seções. A \ref{sec:saas} trata de Software como serviço, discutindo alguns elementos do contexto que são relevantes para o trabalho proposto. A \ref{sec:reuso}  discute sobre a empregabilidade do reuso de software. A \ref{sec:modularizacao} seção, trata sobre aspectos envolvidos na modularização dos softwares. Por fim, a \ref{sec:apps_web} aborda as aplicações web, discutindo sobre aspectos relevantes sobre a aplicação web que compõe este trabalho.


 \section{Software como serviço}\label{sec:saas}


Segundo La e Chun \citep{La2009Systematic}, o princípio da definição de Software como um Serviço (Sofware as a Service - SaaS) é um serviço complementar para aplicações da computação em nuvem (cloud computing). No entanto, as áreas não se confundem. SaaS deve ser entendido como um mecanismo de suporte às soluções existentes na cloud. Os SaaS existem justamente para maximizar o reuso de serviços repetidos e não centrais em uma aplicação remota.


Como vantagens, diversos fatores podem ser favoráveis para a adoção de um SaaS, como custo e manutenção dentre outros fatores aplicáveis a soluções específicas. Lechesaet al. \citep{LechesaSS11}, quantificam índices sobre os fatores determinantes para adoção ou não de um SaaS voltado para ERP na África do Sul. Os principais fatores determinantes para adoção desse mecanismo de software são sua fluidez quanto à rede e a segurança. Esses fatores estão presentes na aplicação desenvolvida neste trabalho de conclusão de curso.


Devido ao fato de ter um serviço constantemente na nuvem, fica o questionamento sobre a segurança da informação manipulada. Sabe-se que a vulnerabilidade na Web não é restrita ao SaaS, atingindo diversos âmbitos. Rai et al. \cite{journals/corr/RaiSM13} consideram que o avanço da computação em nuvem não é um problema apenas para os serviços Web do ponto de vista da segurança, pois muitos trabalhos na literatura mostram a área como mais um ponto de vulnerabilidade para diversos setores, a exemplo de infraestrutura. No mesmo artigo mencionado de Rai et al. \cite{journals/corr/RaiSM13}, também realizaram-se estudos exploratórios junto a empresas usuárias de serviços em computação em nuvem e consideram que a perspectiva de SaaS também pode fortalecer a segurança nas aplicações de cloud computing, pois o software de autenticação compartilhado por várias aplicações em nuvem, oferece uma melhor padronização e consequente facilidade de prevenção a erros de vulnerabilidade específicas de cada módulo da pesquisa. Esse ponto de vista é fundamenta para a compreensão de SaaS.


A arquitetura de armazenamento de dados de um Saas pode variar de acordo com a necessidade do contexto. Huixin \cite{7586486} descreve possíveis modelagens nesse sentido. Essa abordagem pode ser com um banco de dados único, fazendo com que diferentes clientes compartilhem o mesmo banco, diferindo os dados através de controle de usuário, ou isolando os diferentes clientes através de bancos de dados exclusivos para cada um. Esse fator também pode ser combinado com a arquitetura da aplicação, caso ofereça aplicação única para todos os clientes ou aplicação compartilhada. Diante das possíveis abordagens, a modelagem de dados do software pode ser decidida pela regra de negócio. Este trabalho optou por aplicação única e banco de dados compartilhado.

Devido ao conceito particular de obtenção de software, tanto pela visão do cliente como pela visão do vendedor, é necessário compreender os diversos aspectos que podem ser relevantes ao orçar um Saas. O recente trabalho de T. Kaur et al. \citep{6949281} orienta um modelo para compor o custo de um Saas. O custo total seria composto pelos fatores que dão suporte ao funcionamento do software. Tais fatores incluem infra-estrutura, configurabilidade, customização, parâmetros de QoS (Quality of service) como escalabilidade, disponibilidade, usabilidade, pontualidade e desempenho da resposta, portabilidade, custo total de propriedade e retorno do investimento. Esses fatores caracterizam o custo de forma eficaz, possibilitando ao fornecedor, prover um Serviço de acordo com a exigência do consumidor em vários pacotes de serviços.


O conceito de software como serviço se aplica neste trabalho de conclusão de curso, pois o mesmo estará disponível na web com alta disponibilidade adotando as características apresentadas para qualquer pessoa que desejar usá-lo. Assim, basta se cadastrar, configurá-lo e usar como um serviço, sem preocupação com a manutenção do mesmo.


\section{Reuso de software}\label{sec:reuso} %CRUISE BOOK CAPITULO 2


De acordo com o livro Software pratical reuse \citep{Ezran2002}, o reuso de software é a utilização de qualquer informação que um desenvolvedor pode necessitar no processo de criação de software. O livro de Basili e Rombach \citep{Basili1991} define reutilização de software como o uso de tudo o que está associado a um projeto de conhecimento.
Assim, o objetivo da reutilização de software é reciclar o design, código e outros componentes de um produto de software e assim reduzir o custo, o tempo e melhorar a qualidade do produto.
Segundo Keswani et al. \cite{6783445}, o componente reutilizável de software pode ser qualquer parte de seu desenvolvimento, como um fragmento de código, design, casos de teste, ou até mesmo a especificação de requisitos de uma funcionalidade do software. 

O reuso de software pode ter impacto positivo em diversos aspectos do software, vejamos alguns, conforme apresentados no C.R.U.I.S.E Book \citep{cruise2007} :

\begin{itemize}

\item Qualidade: As correções de erro tornam-se úteis em todos os locais em que ocorreu, padronizando e facilitando a manutenção.

\item Produtividade: O ganho de produtividade é alcançado devido ao menor número de artefatos desenvolvido. Isso resulta em menor esforço de teste e também análise e design, reduzindo custos.

\item Confiabilidade: A utilização de componentes bem testados aumenta a
confiança no software. Além disso, a utilização de um mesmo componente em vários sistemas, aumenta a possibilidade de detecção de erros e reforça a confiança no componente.

\item Redução do Esforço: A reutilização de software proporciona uma redução do tempo de desenvolvimento, o que reduz o tempo necessário para o produto ser disponibilizado no mercado para trazer rentabilidade.

\item Trabalho redundante e tempo de desenvolvimento: Desenvolver um sistema do
zero significa desenvolvimento redundante de muitos componentes, como requisitos, especificações, casos de uso, arquitetura, etc. Isso pode ser evitado quando estes estão disponíveis como componentes reutilizáveis e podem ser compartilhados, resultando em um processo de desenvolvimento otimizado.

\item Documentação: Embora a documentação seja muito importante para a
manutenção de um sistema, muitas vezes é negligenciada. A reutilização de componentes de software reduz a quantidade de documentação a ser escrita, entretanto depende da qualidade do que está escrito. Assim, apenas a estrutura do sistema e os novos artefatos desenvolvidos necessitam ser documentados.

\item Custo de manutenção: Menos defeitos e manutenções são esperados quando tem-se comprovada a qualidade dos componentes utilizados.

\item Tamanho da equipe: É comum haver casos em que a equipe de desenvolvimento sofre sobrecarga. Entretanto, dobrar o tamanho da equipe de desenvolvimento não necessariamente duplica produtividade. Se muitos componentes podem ser reutilizados, é possível desenvolver com equipes menores, levando a melhor comunicação e aumento da produtividade.

\end{itemize}

Apesar dos benefícios da reutilização de software, ela não é suficientemente aproveitada. Existem fatores que influenciam direta ou indiretamente na sua adoção. Esses fatores podem ser de aspecto gerencial, organizacional, econômico, conceitual ou técnico. Veremos a seguir alguns aspectos que podem gerar conflito com a cultura de reuso de software, segundo o C.R.U.I.S.E Book \citep{cruise2007} :
%(Sametinger, 1997). REVER

\begin{itemize}
	
\item Falta de apoio da gestão: Como a reutilização de software gera custos iniciais,
a medida pode não ser amplamente alcançada em uma organização sem o apoio de alto nível de gestão. Os gestores têm de ser informados sobre os custos iniciais e serem convencidos sobre economias futuras.

\item Gerenciamento do Projeto: Gerenciar projetos tradicionais é uma tarefa árdua, principalmente, os que praticam a reutilização de software. Utilizando a técnica em larga escala, tem-se impacto sobre todo o ciclo de vida do software.

\item Estruturas organizacionais inadequadas: As estruturas organizacionais devem
considerar diferentes necessidades que surgem quando a reutilização em larga escala está sendo adotada. Por exemplo, uma equipe particionada pode ser alocada somente para desenvolver, manter e certificar componentes reutilizáveis de software.

\item Incentivos de gestão: É comum a falta de incentivo para deixar os desenvolvedores gastarem tempo elaborando componentes do sistemas. A produtividade é muitas vezes medida apenas no tempo necessário para concluir um projeto. Assim, fazer qualquer trabalho além disso, embora benéfico para a empresa como um todo, diminui o seu sucesso. Mesmo quando os componentes reutilizáveis são utilizados, os benefícios obtidos são uma pequena fração do que poderia ser alcançado caso houvesse reutilização explícita, planejada e organizada.

\item Dificuldade de encontrar software reutilizável: Para reutilizar os componentes, devem existir formas eficientes de busca. Além disso, é importante ter um repositório bem organizado contendo componentes com um eficiente meio de acesso.

\item Não reutilização do software encontrado. O acesso fácil ao software existente
não necessariamente aumentar a reutilização. Os componentes reutilizáveis devem ser cuidadosamente especificados, projetados, implementados e documentados, pois em alguns casos, modificar e adaptar o código  pode ser mais custoso que a programação da funcionalidade necessária a partir do zero.

\item Modificação: É muito difícil encontrar um componente que funcione
exatamente da mesma maneira que queremos. Desta forma, são necessárias modificações e devem existir formas de determinar os seus efeitos sobre o componente.


\end{itemize}


A definição do reuso de software pode ser associada a este trabalho, pois ainda que como um serviço, estaremos adotando uma aplicação única para servir aos mais diversos softwares de uma empresa, por exemplo. Logo, ao adotar o projeto, pode ser interessante aplicar algumas métricas para tomar conhecimento de possíveis vantanjens como as citadas nessa seção, a exemplo de economia de tempo de desenvolvimento, e custo do projeto.


%Outra diretriz importante para a reutilização de software é reduzir o risco na criação de novos softwares. O risco tende a ser bastante reduzido se os componentes que estão sendo reutilizados têm as documentação, interfaces necessárias e devidamente testadas, fatores que contibruem para uma fácil integração.
%De acordo com Keswani et al. \citep{6783445}, para o reuso de software dar retornos apropriados, o processo deve ser sistemático e planejado. Qualquer organização que implemente a reutilização de software deve identificar os melhores métodos e estratégias de reutilização para obter a máxima produtividade. A reutilização de software ajuda a evitar software de engenharia a partir do zero, pois usa módulos de software existentes. A reutilização de software, embora seja uma tarefa difícil, especialmente para softwares antigos sem padrões de projeto, pode melhorar significativamente a produtividade e a qualidade de um produto de software. Embora a reutilização de software não seja um novo campo, ela pode dar grandes retornos em curto período de tempo.


\section{Modularização}\label{sec:modularizacao} %artigo de claudio pagina 222 introdução


%A modularidade vem desempenhando um papel predominante estágios emergentes das disciplinas de arquitetura de software [13]. Engenheiros de software consideram modularidade como princípio base na comparação entre arquiteturas alternativas  e arquitetura degeneração [9]. De fato, os engenheiros de software são incentivados a arquitecturas, baseando-se numa multiplicidade de mecanismos de modularidade disponíveis em: 
%(i) Linguagens de descrição de arquitetura (ADLs), como ACME [8], 
%(ii) catálogos de arquitetônicos [2, 13], e 
%(iii) conhecem bem princípios de alto nível, como interfaces de componentes estreitos, acoplamento arquitectónico reduzido e semelhantes.


Conforme é frisado por Wickramaarachchi e Lai \citep{7062705}, o conceito de modularização na indústria de software tem uma longa história e tem sido utilizado para melhorar o processo de desenvolvimento de software em diferentes estágios. Os principais conceitos por trás da modularização do software foram introduzidos por pesquisadores pioneiros há quarenta anos, com uma notável contribuição feita por Melvin Conway e David Parnas, que tem representação notável na engenharia de software.


Modularizar um software é um bom padrão a ser adotado. Segundo Wickramaarachchi e Lai \citep{7062705}, a modularização é importante na identificação de dependências e reduz as dificuldades diante de uma possível necessidade de grandes alterações. De uma perspectiva da engenharia de software, uma modularização geralmente tem várias vantagens, tais como: tornar a complexidade do software mais gerenciável, facilitar o trabalho paralelo e tornar o software mais maleável para acomodar o futuro incerto que um software pode ter. O objetivo final da modularização do software é aumentar a produtividade ea qualidade do software. Tal conceito encontra-se bastante difundido e estái incorporado em linguagens de programação e ferramentas de software. O presente trabalho favorece ao uso da modularização de um software e até mesmo pode ser considerado um módulo a ser acoplado a qualquer software, mediante a compatibilidade.


Ao realizar a adoção do trabalho proposto, fica bastante evidente a "responsabilidade" do mesmo no escopo do projeto. Assim, é possível tratá-lo como um módulo do projeto que o usa, mesmo que seja consumido como um serviço.


\section{Aplicações web}\label{sec:apps_web}


A popularidade das soluções Web aumentou exponencialmente na última década e todos os dias cresce o número de pessoas usuárias desse tipo de software. E seguindo um padrão próprio, Kumar et al. \citep{7813710} sugerem que para o desenvolvimento web, deve-se manter a prática eficaz de produzir diagramas UML. A abordagem baseada na web oferece uma maneira fácil e eficaz para gerenciar e controlar o processo de desenvolvimento por meio de artefatos de modelagem. Tal abordagem pode ser usada quando há uma exigência de lidar com mudanças muito rápidas e grandes em requisitos de forma muito eficaz em muito menos tempo, gerando assim um menor impacto. 


Para atender à fomentada demanda de aplicativos web, é necessário adotar métodos de desenvolvimentos que sejam ágeis, eficientes e de fácil manutenção. Yu Ping et al. \cite{1372143} propõem o uso do modelo MVC (Model, View e Controller) no desenvolvimento para softwares web. O modelo apresentado tornou-se um padrão popular e divide o software em camadas com propósito definido, tornando-o de mais fácil manutenção.


O Ajax (Asynchronous Javascript and XML) revolucionou a web. Conforme demonstrado por Yuping \citep{6845605}, ao usar a tecnologia Ajax, podemos enriquecer a experiência do usuário em aplicações baseadas em navegador de internet, e fornecer uma variedade de aplicações interativas para atender às necessidade de humanização das aplicações.
Os aplicativos Ajax em execução no navegador se comunicam com um servidor Web de forma assíncrona e atualizam apenas uma parte da página.


No artigo de Tesarik et al. \citep{4664349}, temos que o design de software SPA (Single page application) é uma maneira interessante de criar um software Web numa uma única página. Essa solução de página única sem navegação funciona apenas com base em técnicas dinâmicas e assíncronas, como o AJAX (citado no parágrafo anterior). No entanto, esta abordagem coloca o desenvolvedor antes de alguns desafios substanciais.
Para projetar a interface do usuário que mostra as informações, é importante projetar corretamente a tela para manipular os dados do aplicativo numa única página. O design da página deve ser elaborado para maximizar a decomposição da página em componentes distintos que encapsulam os principais casos de uso.
Também recomenda-se que para elaborar os artefatos visuais, explore os mais ricos recursos na implementação, como HTML5, JavaScript, Ajax, CSS3, e outras tecnologias que se apliquem. Entretanto, desenvolver uma rica interface com o uso de diversas tecnologias/frameworks pode ocasionar um esforço maior para explorar as possibilidades.
Tais componentes da interface de uma aplicação SPA normalmente são alimentados mediante o consumo de uma API Rest via requisições AJAX. Assim, é possível particionar as responsabilidades de processamento do software entre cliente e servidor.


Como é possível denotar no trabalho de Salvadori e Siqueire \citep{7195633}, REST é uma arquitetura muito popular para integração de aplicativos web, que permite compartilhar e reutilizar informações através de sistemas. Aplicações de grande escala baseadas em no REST deve ser implementadas utilizando estratégias e mecanismos para produzir sistemas que sejam fáceis de desenvolver, reutilizar e manter. As interfaces de integração fornecidas por esses sistemas, chamadas de Web API, têm uma influência importante nas características da implementação resultante, pois a sua forma de resposta deve ser casada com a interface que o consome.
 
 
Os elementos comentados nessa seção encontram-se presentes neste trabalho de conclusão de curso. Juntos, montam a estrutura tecnológica necessária aliada à arquitetura adotada. Esses elementos seguem tendências atuais dos softwares web, fazendo com que esse trabalho esteja composto por tecnologias modernas que estão com boa aceitação no mercado.

% %% RiSE Latex Template - version 0.5
%%
%% RiSE's latex template for thesis and dissertations
%% http://risetemplate.sourceforge.net
%%
%% (c) 2012 Yguaratã Cerqueira Cavalcanti (yguarata@gmail.com)
%%          Vinicius Cardoso Garcia (vinicius.garcia@gmail.com)
%%
%% This document was initially based on UFPEThesis template, from Paulo Gustavo
%% S. Fonseca.
%%
%% ACKNOWLEDGEMENTS
%%
%% We would like to thanks the RiSE's researchers community, the 
%% students from Federal University of Pernambuco, and other users that have
%% been contributing to this projects with comments and patches.
%%
%% GENERAL INSTRUCTIONS
%%
%% We strongly recommend you to compile your documents using pdflatex command.
%% It is also recommend use the texlipse plugin for Eclipse to edit your documents.
%%
%% Options for \documentclass command:
%%         * Idiom
%%           pt   - Portguese (default)
%%           en   - English
%%
%%         * Text type
%%           bsc  - B.Sc. Thesis
%%           msc  - M.Sc. Thesis (default)
%%           qual - PHD qualification (not tested yet)
%%           prop - PHD proposal (not tested yet)
%%           phd  - PHD thesis
%%
%%         * Media
%%           scr  - to eletronic version (PDF) / see the users guide
%%
%%         * Pagination
%%           oneside - unique face press
%%           twoside - two faces press
%%
%%		   * Line spacing
%%           singlespacing  - the same as using \linespread{1}
%%           onehalfspacing - the same as using \linespread{1.3}
%%           doublespacing  - the same as using \linespread{1.6}
%%
%% Reference commands. Use the following commands to make references in your
%% text:
%%          \figref  -- for Figure reference
%%          \tabref  -- for Table reference
%%          \eqnref  -- for equation reference
%%          \chapref -- for chapter reference
%%          \secref  -- for section reference
%%          \appref  -- for appendix reference
%%          \axiref  -- for axiom reference
%%          \conjref -- for conjecture reference
%%          \defref  -- for definition reference
%%          \lemref  -- for lemma reference
%%          \theoref -- for theorem reference
%%          \corref  -- for corollary reference
%%          \propref -- for proprosition reference
%%          \pgref   -- for page reference
%%
%%          Example: See \chapref{chap:introduction}. It will produce 
%%                   'See Chapter 1', in case of English language.

\documentclass[pt,twoside,onehalfspacing,bsc]{risethesis}

\usepackage[sort,square]{natbib} %package for bibliography
\bibliographystyle{ieeetr} %authordate1

\usepackage[utf8]{inputenc}
\usepackage[brazilian]{babel}
\usepackage[T1]{fontenc}

\usepackage{graphicx}

%% Change the following pdf author attribute name to your name.
\usepackage[linkcolor=blue,citecolor=blue,urlcolor=blue,colorlinks,pdfpagelabels,pdftitle={Bruno Cabral's Bachelor Thesis},pdfauthor={Bruno Cabral}]{hyperref}

\address{SALVADOR}

\universitypt{Universidade Federal da Bahia}
\universityen{Federal University of Bahia}

\departmentpt{Depertamento de Ciência da Computação}
\departmenten{Computer Science Department}

\programpt{Programa Multiinstitucional de Pós-graduação em Ciência da Computação}
\programen{Graduate in Computer Science}

\majorfieldpt{Ciência da Computação}
\majorfielden{Computer Science}

\title{Sistema de apoio à Pós graduação - UFBA}
\date{Outubro/2016}

\author{Victor de Azevedo Nunes}
\adviser{Ivan do Carmo Machado}

\begin{document}

\frontmatter
\frontpage
\presentationpage

\begin{dedicatory}
Eu dedico esta dissertação...
%I dedicate this dissertation to my family, girlfriend, friends and
%professors who gave me all necessary support to get here.
\end{dedicatory}

\acknowledgements
Meus agradecimentos...

\begin{epigraph}[]{Edward V Berard}
Walking on water and developing software from a specification are easy if both are frozen
\end{epigraph}

\resumo
% Escreva seu resumo no arquivo resumo.tex
\input{resumo}

\abstract
% Write your abstract in a file called abstract.tex
\input{abstract}

% Summary (tables of contents)
\tableofcontents

% List of figures
\listoffigures

% List of tables
\listoftables

% List of acronyms
% Acronyms manual: http://linorg.usp.br/CTAN/macros/latex/contrib/acronym/acronym.pdf
\listofacronyms
\input{acronyms}

% List of listings
%\lstlistoflistings

\mainmatter

\include{chapters/intro}
\include{chapters/referencial_teorico}

% \include{chapters/introduction/main}
% \include{chapters/background/main}
% \include{chapters/proposed_solution/main}
% \include{chapters/experiment/main}
% \include{chapters/conclusion/main}

%\bibliographystyle{natbib}
%\addcontentsline{toc}{chapter}{\bibliographytocname}
\bibliography{references}

% Appendix
\clearpage
\addappheadtotoc
\appendix
\appendixpage
% \include{appendix/experiment-instruments}

\end{document}
% %% RiSE Latex Template - version 0.5
%%
%% RiSE's latex template for thesis and dissertations
%% http://risetemplate.sourceforge.net
%%
%% (c) 2012 Yguaratã Cerqueira Cavalcanti (yguarata@gmail.com)
%%          Vinicius Cardoso Garcia (vinicius.garcia@gmail.com)
%%
%% This document was initially based on UFPEThesis template, from Paulo Gustavo
%% S. Fonseca.
%%
%% ACKNOWLEDGEMENTS
%%
%% We would like to thanks the RiSE's researchers community, the 
%% students from Federal University of Pernambuco, and other users that have
%% been contributing to this projects with comments and patches.
%%
%% GENERAL INSTRUCTIONS
%%
%% We strongly recommend you to compile your documents using pdflatex command.
%% It is also recommend use the texlipse plugin for Eclipse to edit your documents.
%%
%% Options for \documentclass command:
%%         * Idiom
%%           pt   - Portguese (default)
%%           en   - English
%%
%%         * Text type
%%           bsc  - B.Sc. Thesis
%%           msc  - M.Sc. Thesis (default)
%%           qual - PHD qualification (not tested yet)
%%           prop - PHD proposal (not tested yet)
%%           phd  - PHD thesis
%%
%%         * Media
%%           scr  - to eletronic version (PDF) / see the users guide
%%
%%         * Pagination
%%           oneside - unique face press
%%           twoside - two faces press
%%
%%		   * Line spacing
%%           singlespacing  - the same as using \linespread{1}
%%           onehalfspacing - the same as using \linespread{1.3}
%%           doublespacing  - the same as using \linespread{1.6}
%%
%% Reference commands. Use the following commands to make references in your
%% text:
%%          \figref  -- for Figure reference
%%          \tabref  -- for Table reference
%%          \eqnref  -- for equation reference
%%          \chapref -- for chapter reference
%%          \secref  -- for section reference
%%          \appref  -- for appendix reference
%%          \axiref  -- for axiom reference
%%          \conjref -- for conjecture reference
%%          \defref  -- for definition reference
%%          \lemref  -- for lemma reference
%%          \theoref -- for theorem reference
%%          \corref  -- for corollary reference
%%          \propref -- for proprosition reference
%%          \pgref   -- for page reference
%%
%%          Example: See \chapref{chap:introduction}. It will produce 
%%                   'See Chapter 1', in case of English language.

\documentclass[pt,twoside,onehalfspacing,bsc]{risethesis}

\usepackage[sort,square]{natbib} %package for bibliography
\bibliographystyle{ieeetr} %authordate1

\usepackage[utf8]{inputenc}
\usepackage[brazilian]{babel}
\usepackage[T1]{fontenc}

\usepackage{graphicx}

%% Change the following pdf author attribute name to your name.
\usepackage[linkcolor=blue,citecolor=blue,urlcolor=blue,colorlinks,pdfpagelabels,pdftitle={Bruno Cabral's Bachelor Thesis},pdfauthor={Bruno Cabral}]{hyperref}

\address{SALVADOR}

\universitypt{Universidade Federal da Bahia}
\universityen{Federal University of Bahia}

\departmentpt{Depertamento de Ciência da Computação}
\departmenten{Computer Science Department}

\programpt{Programa Multiinstitucional de Pós-graduação em Ciência da Computação}
\programen{Graduate in Computer Science}

\majorfieldpt{Ciência da Computação}
\majorfielden{Computer Science}

\title{Sistema de apoio à Pós graduação - UFBA}
\date{Outubro/2016}

\author{Victor de Azevedo Nunes}
\adviser{Ivan do Carmo Machado}

\begin{document}

\frontmatter
\frontpage
\presentationpage

\begin{dedicatory}
Eu dedico esta dissertação...
%I dedicate this dissertation to my family, girlfriend, friends and
%professors who gave me all necessary support to get here.
\end{dedicatory}

\acknowledgements
Meus agradecimentos...

\begin{epigraph}[]{Edward V Berard}
Walking on water and developing software from a specification are easy if both are frozen
\end{epigraph}

\resumo
% Escreva seu resumo no arquivo resumo.tex
\input{resumo}

\abstract
% Write your abstract in a file called abstract.tex
\input{abstract}

% Summary (tables of contents)
\tableofcontents

% List of figures
\listoffigures

% List of tables
\listoftables

% List of acronyms
% Acronyms manual: http://linorg.usp.br/CTAN/macros/latex/contrib/acronym/acronym.pdf
\listofacronyms
\input{acronyms}

% List of listings
%\lstlistoflistings

\mainmatter

\include{chapters/intro}
\include{chapters/referencial_teorico}

% \include{chapters/introduction/main}
% \include{chapters/background/main}
% \include{chapters/proposed_solution/main}
% \include{chapters/experiment/main}
% \include{chapters/conclusion/main}

%\bibliographystyle{natbib}
%\addcontentsline{toc}{chapter}{\bibliographytocname}
\bibliography{references}

% Appendix
\clearpage
\addappheadtotoc
\appendix
\appendixpage
% \include{appendix/experiment-instruments}

\end{document}
% %% RiSE Latex Template - version 0.5
%%
%% RiSE's latex template for thesis and dissertations
%% http://risetemplate.sourceforge.net
%%
%% (c) 2012 Yguaratã Cerqueira Cavalcanti (yguarata@gmail.com)
%%          Vinicius Cardoso Garcia (vinicius.garcia@gmail.com)
%%
%% This document was initially based on UFPEThesis template, from Paulo Gustavo
%% S. Fonseca.
%%
%% ACKNOWLEDGEMENTS
%%
%% We would like to thanks the RiSE's researchers community, the 
%% students from Federal University of Pernambuco, and other users that have
%% been contributing to this projects with comments and patches.
%%
%% GENERAL INSTRUCTIONS
%%
%% We strongly recommend you to compile your documents using pdflatex command.
%% It is also recommend use the texlipse plugin for Eclipse to edit your documents.
%%
%% Options for \documentclass command:
%%         * Idiom
%%           pt   - Portguese (default)
%%           en   - English
%%
%%         * Text type
%%           bsc  - B.Sc. Thesis
%%           msc  - M.Sc. Thesis (default)
%%           qual - PHD qualification (not tested yet)
%%           prop - PHD proposal (not tested yet)
%%           phd  - PHD thesis
%%
%%         * Media
%%           scr  - to eletronic version (PDF) / see the users guide
%%
%%         * Pagination
%%           oneside - unique face press
%%           twoside - two faces press
%%
%%		   * Line spacing
%%           singlespacing  - the same as using \linespread{1}
%%           onehalfspacing - the same as using \linespread{1.3}
%%           doublespacing  - the same as using \linespread{1.6}
%%
%% Reference commands. Use the following commands to make references in your
%% text:
%%          \figref  -- for Figure reference
%%          \tabref  -- for Table reference
%%          \eqnref  -- for equation reference
%%          \chapref -- for chapter reference
%%          \secref  -- for section reference
%%          \appref  -- for appendix reference
%%          \axiref  -- for axiom reference
%%          \conjref -- for conjecture reference
%%          \defref  -- for definition reference
%%          \lemref  -- for lemma reference
%%          \theoref -- for theorem reference
%%          \corref  -- for corollary reference
%%          \propref -- for proprosition reference
%%          \pgref   -- for page reference
%%
%%          Example: See \chapref{chap:introduction}. It will produce 
%%                   'See Chapter 1', in case of English language.

\documentclass[pt,twoside,onehalfspacing,bsc]{risethesis}

\usepackage[sort,square]{natbib} %package for bibliography
\bibliographystyle{ieeetr} %authordate1

\usepackage[utf8]{inputenc}
\usepackage[brazilian]{babel}
\usepackage[T1]{fontenc}

\usepackage{graphicx}

%% Change the following pdf author attribute name to your name.
\usepackage[linkcolor=blue,citecolor=blue,urlcolor=blue,colorlinks,pdfpagelabels,pdftitle={Bruno Cabral's Bachelor Thesis},pdfauthor={Bruno Cabral}]{hyperref}

\address{SALVADOR}

\universitypt{Universidade Federal da Bahia}
\universityen{Federal University of Bahia}

\departmentpt{Depertamento de Ciência da Computação}
\departmenten{Computer Science Department}

\programpt{Programa Multiinstitucional de Pós-graduação em Ciência da Computação}
\programen{Graduate in Computer Science}

\majorfieldpt{Ciência da Computação}
\majorfielden{Computer Science}

\title{Sistema de apoio à Pós graduação - UFBA}
\date{Outubro/2016}

\author{Victor de Azevedo Nunes}
\adviser{Ivan do Carmo Machado}

\begin{document}

\frontmatter
\frontpage
\presentationpage

\begin{dedicatory}
Eu dedico esta dissertação...
%I dedicate this dissertation to my family, girlfriend, friends and
%professors who gave me all necessary support to get here.
\end{dedicatory}

\acknowledgements
Meus agradecimentos...

\begin{epigraph}[]{Edward V Berard}
Walking on water and developing software from a specification are easy if both are frozen
\end{epigraph}

\resumo
% Escreva seu resumo no arquivo resumo.tex
\input{resumo}

\abstract
% Write your abstract in a file called abstract.tex
\input{abstract}

% Summary (tables of contents)
\tableofcontents

% List of figures
\listoffigures

% List of tables
\listoftables

% List of acronyms
% Acronyms manual: http://linorg.usp.br/CTAN/macros/latex/contrib/acronym/acronym.pdf
\listofacronyms
\input{acronyms}

% List of listings
%\lstlistoflistings

\mainmatter

\include{chapters/intro}
\include{chapters/referencial_teorico}

% \include{chapters/introduction/main}
% \include{chapters/background/main}
% \include{chapters/proposed_solution/main}
% \include{chapters/experiment/main}
% \include{chapters/conclusion/main}

%\bibliographystyle{natbib}
%\addcontentsline{toc}{chapter}{\bibliographytocname}
\bibliography{references}

% Appendix
\clearpage
\addappheadtotoc
\appendix
\appendixpage
% \include{appendix/experiment-instruments}

\end{document}
% %% RiSE Latex Template - version 0.5
%%
%% RiSE's latex template for thesis and dissertations
%% http://risetemplate.sourceforge.net
%%
%% (c) 2012 Yguaratã Cerqueira Cavalcanti (yguarata@gmail.com)
%%          Vinicius Cardoso Garcia (vinicius.garcia@gmail.com)
%%
%% This document was initially based on UFPEThesis template, from Paulo Gustavo
%% S. Fonseca.
%%
%% ACKNOWLEDGEMENTS
%%
%% We would like to thanks the RiSE's researchers community, the 
%% students from Federal University of Pernambuco, and other users that have
%% been contributing to this projects with comments and patches.
%%
%% GENERAL INSTRUCTIONS
%%
%% We strongly recommend you to compile your documents using pdflatex command.
%% It is also recommend use the texlipse plugin for Eclipse to edit your documents.
%%
%% Options for \documentclass command:
%%         * Idiom
%%           pt   - Portguese (default)
%%           en   - English
%%
%%         * Text type
%%           bsc  - B.Sc. Thesis
%%           msc  - M.Sc. Thesis (default)
%%           qual - PHD qualification (not tested yet)
%%           prop - PHD proposal (not tested yet)
%%           phd  - PHD thesis
%%
%%         * Media
%%           scr  - to eletronic version (PDF) / see the users guide
%%
%%         * Pagination
%%           oneside - unique face press
%%           twoside - two faces press
%%
%%		   * Line spacing
%%           singlespacing  - the same as using \linespread{1}
%%           onehalfspacing - the same as using \linespread{1.3}
%%           doublespacing  - the same as using \linespread{1.6}
%%
%% Reference commands. Use the following commands to make references in your
%% text:
%%          \figref  -- for Figure reference
%%          \tabref  -- for Table reference
%%          \eqnref  -- for equation reference
%%          \chapref -- for chapter reference
%%          \secref  -- for section reference
%%          \appref  -- for appendix reference
%%          \axiref  -- for axiom reference
%%          \conjref -- for conjecture reference
%%          \defref  -- for definition reference
%%          \lemref  -- for lemma reference
%%          \theoref -- for theorem reference
%%          \corref  -- for corollary reference
%%          \propref -- for proprosition reference
%%          \pgref   -- for page reference
%%
%%          Example: See \chapref{chap:introduction}. It will produce 
%%                   'See Chapter 1', in case of English language.

\documentclass[pt,twoside,onehalfspacing,bsc]{risethesis}

\usepackage[sort,square]{natbib} %package for bibliography
\bibliographystyle{ieeetr} %authordate1

\usepackage[utf8]{inputenc}
\usepackage[brazilian]{babel}
\usepackage[T1]{fontenc}

\usepackage{graphicx}

%% Change the following pdf author attribute name to your name.
\usepackage[linkcolor=blue,citecolor=blue,urlcolor=blue,colorlinks,pdfpagelabels,pdftitle={Bruno Cabral's Bachelor Thesis},pdfauthor={Bruno Cabral}]{hyperref}

\address{SALVADOR}

\universitypt{Universidade Federal da Bahia}
\universityen{Federal University of Bahia}

\departmentpt{Depertamento de Ciência da Computação}
\departmenten{Computer Science Department}

\programpt{Programa Multiinstitucional de Pós-graduação em Ciência da Computação}
\programen{Graduate in Computer Science}

\majorfieldpt{Ciência da Computação}
\majorfielden{Computer Science}

\title{Sistema de apoio à Pós graduação - UFBA}
\date{Outubro/2016}

\author{Victor de Azevedo Nunes}
\adviser{Ivan do Carmo Machado}

\begin{document}

\frontmatter
\frontpage
\presentationpage

\begin{dedicatory}
Eu dedico esta dissertação...
%I dedicate this dissertation to my family, girlfriend, friends and
%professors who gave me all necessary support to get here.
\end{dedicatory}

\acknowledgements
Meus agradecimentos...

\begin{epigraph}[]{Edward V Berard}
Walking on water and developing software from a specification are easy if both are frozen
\end{epigraph}

\resumo
% Escreva seu resumo no arquivo resumo.tex
\input{resumo}

\abstract
% Write your abstract in a file called abstract.tex
\input{abstract}

% Summary (tables of contents)
\tableofcontents

% List of figures
\listoffigures

% List of tables
\listoftables

% List of acronyms
% Acronyms manual: http://linorg.usp.br/CTAN/macros/latex/contrib/acronym/acronym.pdf
\listofacronyms
\input{acronyms}

% List of listings
%\lstlistoflistings

\mainmatter

\include{chapters/intro}
\include{chapters/referencial_teorico}

% \include{chapters/introduction/main}
% \include{chapters/background/main}
% \include{chapters/proposed_solution/main}
% \include{chapters/experiment/main}
% \include{chapters/conclusion/main}

%\bibliographystyle{natbib}
%\addcontentsline{toc}{chapter}{\bibliographytocname}
\bibliography{references}

% Appendix
\clearpage
\addappheadtotoc
\appendix
\appendixpage
% \include{appendix/experiment-instruments}

\end{document}
% %% RiSE Latex Template - version 0.5
%%
%% RiSE's latex template for thesis and dissertations
%% http://risetemplate.sourceforge.net
%%
%% (c) 2012 Yguaratã Cerqueira Cavalcanti (yguarata@gmail.com)
%%          Vinicius Cardoso Garcia (vinicius.garcia@gmail.com)
%%
%% This document was initially based on UFPEThesis template, from Paulo Gustavo
%% S. Fonseca.
%%
%% ACKNOWLEDGEMENTS
%%
%% We would like to thanks the RiSE's researchers community, the 
%% students from Federal University of Pernambuco, and other users that have
%% been contributing to this projects with comments and patches.
%%
%% GENERAL INSTRUCTIONS
%%
%% We strongly recommend you to compile your documents using pdflatex command.
%% It is also recommend use the texlipse plugin for Eclipse to edit your documents.
%%
%% Options for \documentclass command:
%%         * Idiom
%%           pt   - Portguese (default)
%%           en   - English
%%
%%         * Text type
%%           bsc  - B.Sc. Thesis
%%           msc  - M.Sc. Thesis (default)
%%           qual - PHD qualification (not tested yet)
%%           prop - PHD proposal (not tested yet)
%%           phd  - PHD thesis
%%
%%         * Media
%%           scr  - to eletronic version (PDF) / see the users guide
%%
%%         * Pagination
%%           oneside - unique face press
%%           twoside - two faces press
%%
%%		   * Line spacing
%%           singlespacing  - the same as using \linespread{1}
%%           onehalfspacing - the same as using \linespread{1.3}
%%           doublespacing  - the same as using \linespread{1.6}
%%
%% Reference commands. Use the following commands to make references in your
%% text:
%%          \figref  -- for Figure reference
%%          \tabref  -- for Table reference
%%          \eqnref  -- for equation reference
%%          \chapref -- for chapter reference
%%          \secref  -- for section reference
%%          \appref  -- for appendix reference
%%          \axiref  -- for axiom reference
%%          \conjref -- for conjecture reference
%%          \defref  -- for definition reference
%%          \lemref  -- for lemma reference
%%          \theoref -- for theorem reference
%%          \corref  -- for corollary reference
%%          \propref -- for proprosition reference
%%          \pgref   -- for page reference
%%
%%          Example: See \chapref{chap:introduction}. It will produce 
%%                   'See Chapter 1', in case of English language.

\documentclass[pt,twoside,onehalfspacing,bsc]{risethesis}

\usepackage[sort,square]{natbib} %package for bibliography
\bibliographystyle{ieeetr} %authordate1

\usepackage[utf8]{inputenc}
\usepackage[brazilian]{babel}
\usepackage[T1]{fontenc}

\usepackage{graphicx}

%% Change the following pdf author attribute name to your name.
\usepackage[linkcolor=blue,citecolor=blue,urlcolor=blue,colorlinks,pdfpagelabels,pdftitle={Bruno Cabral's Bachelor Thesis},pdfauthor={Bruno Cabral}]{hyperref}

\address{SALVADOR}

\universitypt{Universidade Federal da Bahia}
\universityen{Federal University of Bahia}

\departmentpt{Depertamento de Ciência da Computação}
\departmenten{Computer Science Department}

\programpt{Programa Multiinstitucional de Pós-graduação em Ciência da Computação}
\programen{Graduate in Computer Science}

\majorfieldpt{Ciência da Computação}
\majorfielden{Computer Science}

\title{Sistema de apoio à Pós graduação - UFBA}
\date{Outubro/2016}

\author{Victor de Azevedo Nunes}
\adviser{Ivan do Carmo Machado}

\begin{document}

\frontmatter
\frontpage
\presentationpage

\begin{dedicatory}
Eu dedico esta dissertação...
%I dedicate this dissertation to my family, girlfriend, friends and
%professors who gave me all necessary support to get here.
\end{dedicatory}

\acknowledgements
Meus agradecimentos...

\begin{epigraph}[]{Edward V Berard}
Walking on water and developing software from a specification are easy if both are frozen
\end{epigraph}

\resumo
% Escreva seu resumo no arquivo resumo.tex
\input{resumo}

\abstract
% Write your abstract in a file called abstract.tex
\input{abstract}

% Summary (tables of contents)
\tableofcontents

% List of figures
\listoffigures

% List of tables
\listoftables

% List of acronyms
% Acronyms manual: http://linorg.usp.br/CTAN/macros/latex/contrib/acronym/acronym.pdf
\listofacronyms
\input{acronyms}

% List of listings
%\lstlistoflistings

\mainmatter

\include{chapters/intro}
\include{chapters/referencial_teorico}

% \include{chapters/introduction/main}
% \include{chapters/background/main}
% \include{chapters/proposed_solution/main}
% \include{chapters/experiment/main}
% \include{chapters/conclusion/main}

%\bibliographystyle{natbib}
%\addcontentsline{toc}{chapter}{\bibliographytocname}
\bibliography{references}

% Appendix
\clearpage
\addappheadtotoc
\appendix
\appendixpage
% \include{appendix/experiment-instruments}

\end{document}

%\bibliographystyle{natbib}
%\addcontentsline{toc}{chapter}{\bibliographytocname}
\bibliography{references}

% Appendix
\clearpage
\addappheadtotoc
\appendix
\appendixpage
% \include{appendix/experiment-instruments}

\end{document}
% %% RiSE Latex Template - version 0.5
%%
%% RiSE's latex template for thesis and dissertations
%% http://risetemplate.sourceforge.net
%%
%% (c) 2012 Yguaratã Cerqueira Cavalcanti (yguarata@gmail.com)
%%          Vinicius Cardoso Garcia (vinicius.garcia@gmail.com)
%%
%% This document was initially based on UFPEThesis template, from Paulo Gustavo
%% S. Fonseca.
%%
%% ACKNOWLEDGEMENTS
%%
%% We would like to thanks the RiSE's researchers community, the 
%% students from Federal University of Pernambuco, and other users that have
%% been contributing to this projects with comments and patches.
%%
%% GENERAL INSTRUCTIONS
%%
%% We strongly recommend you to compile your documents using pdflatex command.
%% It is also recommend use the texlipse plugin for Eclipse to edit your documents.
%%
%% Options for \documentclass command:
%%         * Idiom
%%           pt   - Portguese (default)
%%           en   - English
%%
%%         * Text type
%%           bsc  - B.Sc. Thesis
%%           msc  - M.Sc. Thesis (default)
%%           qual - PHD qualification (not tested yet)
%%           prop - PHD proposal (not tested yet)
%%           phd  - PHD thesis
%%
%%         * Media
%%           scr  - to eletronic version (PDF) / see the users guide
%%
%%         * Pagination
%%           oneside - unique face press
%%           twoside - two faces press
%%
%%		   * Line spacing
%%           singlespacing  - the same as using \linespread{1}
%%           onehalfspacing - the same as using \linespread{1.3}
%%           doublespacing  - the same as using \linespread{1.6}
%%
%% Reference commands. Use the following commands to make references in your
%% text:
%%          \figref  -- for Figure reference
%%          \tabref  -- for Table reference
%%          \eqnref  -- for equation reference
%%          \chapref -- for chapter reference
%%          \secref  -- for section reference
%%          \appref  -- for appendix reference
%%          \axiref  -- for axiom reference
%%          \conjref -- for conjecture reference
%%          \defref  -- for definition reference
%%          \lemref  -- for lemma reference
%%          \theoref -- for theorem reference
%%          \corref  -- for corollary reference
%%          \propref -- for proprosition reference
%%          \pgref   -- for page reference
%%
%%          Example: See \chapref{chap:introduction}. It will produce 
%%                   'See Chapter 1', in case of English language.

\documentclass[pt,twoside,onehalfspacing,bsc]{risethesis}

\usepackage[sort,square]{natbib} %package for bibliography
\bibliographystyle{ieeetr} %authordate1

\usepackage[utf8]{inputenc}
\usepackage[brazilian]{babel}
\usepackage[T1]{fontenc}

\usepackage{graphicx}

%% Change the following pdf author attribute name to your name.
\usepackage[linkcolor=blue,citecolor=blue,urlcolor=blue,colorlinks,pdfpagelabels,pdftitle={Bruno Cabral's Bachelor Thesis},pdfauthor={Bruno Cabral}]{hyperref}

\address{SALVADOR}

\universitypt{Universidade Federal da Bahia}
\universityen{Federal University of Bahia}

\departmentpt{Depertamento de Ciência da Computação}
\departmenten{Computer Science Department}

\programpt{Programa Multiinstitucional de Pós-graduação em Ciência da Computação}
\programen{Graduate in Computer Science}

\majorfieldpt{Ciência da Computação}
\majorfielden{Computer Science}

\title{Sistema de apoio à Pós graduação - UFBA}
\date{Outubro/2016}

\author{Victor de Azevedo Nunes}
\adviser{Ivan do Carmo Machado}

\begin{document}

\frontmatter
\frontpage
\presentationpage

\begin{dedicatory}
Eu dedico esta dissertação...
%I dedicate this dissertation to my family, girlfriend, friends and
%professors who gave me all necessary support to get here.
\end{dedicatory}

\acknowledgements
Meus agradecimentos...

\begin{epigraph}[]{Edward V Berard}
Walking on water and developing software from a specification are easy if both are frozen
\end{epigraph}

\resumo
% Escreva seu resumo no arquivo resumo.tex
Este trabalho de conclusão de curso utiliza elementos da engenharia de software para propor um software como serviço a fim de otimizar o processo de construção e manutenção dos softwares. Assim, o objetivo deste SaaS é gerenciar as permissões de acesso de sistemas cliente, provendo o reuso de software e padronizando as soluções. O sistema proposto neste trabalho encontra-se implementado e disponível no Github, e traz fundamentos sobre a arquitetura e tecnologias utilizadas, bem como avaliações sobre possibilidades de utilização do mesmo em diversos ambientes, como Web e mobile.

\begin{keywords}
Software; Reuso; SaaS; Web; PHP
\end{keywords}

\abstract
% Write your abstract in a file called abstract.tex
My abstract...

\begin{keywords}
key words...
\end{keywords}

% Summary (tables of contents)
\tableofcontents

% List of figures
\listoffigures

% List of tables
\listoftables

% List of acronyms
% Acronyms manual: http://linorg.usp.br/CTAN/macros/latex/contrib/acronym/acronym.pdf
\listofacronyms
\begin{acronym}[ACRONYM] 
% Change the word ACRONYM above to change the acronym column width.
% The column width is equals to the width of the word that you put.
% Read the manual about acronym package for more examples:
%   http://linorg.usp.br/CTAN/macros/latex/contrib/acronym/acronym.pdf

\acro{SPA}{Single Page Application}
\acro{JSON}{Javascript Object Notation}
\acro{PHP}{PHP: Hypertext Preprocessor}
\acro{SaaS}{Software as a Service}
\acro{ERP}{Enterprise Resource Planning}
\acro{QoS}{Quality of Service}
\acro{UML}{Unified Modeling Language}
\acro{MVC}{Model-View-Controller}
\acro{Ajax}{Asynchronous Javascript and XML}
\acro{HTML}{HyperText Markup Language}
\acro{CSS}{Cascading Style Sheets}
\acro{API}{Application Programming Interface}
\acro{DOM}{Document Object Model}
\acro{BPMN}{Business Process Model and Notation}
\acro{REST}{Representational State Transfer}

\end{acronym}

% List of listings
%\lstlistoflistings

\mainmatter

\chapter{Introdução}

\section{Motivação}

Organizar os procedimentos de um processo sempre nos traz vantagens. Apesar de no processo de implantação de um sistema, o mesmo burocratizar o processo, com o tempo temos o retorno da dedicação para a inserção dos dados. Com um certo volume de dados, é possível estruturar informações que num processo manual são difíceis de serem enxergadas. Assim, é possível depender menos das pessoas que organizam o processo, pois o legado de informações não estará mais somente na mente de alguns, mas sim documentado nos dados do sistema.

Além de colaborar na organização, também haverá uma grande colaboração no tempo gasto na gestão. Lidar com muitos papéis e confiar na mente humana para guardar informações, não é uma alternativa muito segura devido ao fato que as pessoas sempre estão sujeitas a sair do processo e levar contigo a experiência obtida. Experiência essa que faz com que os procedimentos sejam executados de forma mais eficiente. Entretanto, com um sistema inteligente, é possível auxiliar e tornar mais ágil a execução das tarefas.


\section{Problema}


De acordo com funcionários ligados ao o setor de pós graduação da UFBA, entrevistados a fim de um maior entendimento do cenário, apesar das semelhanças estruturais, a pós graduação gerida de forma diferencia da graduação. FULANO afirma que devido ao fato de não ter a mesma visibilidade, não tem acesso aos mesmos recursos de gestão acadêmica da graduação. O professores não executam somente atividades dentro da sala de aula, também tem diversas outras ocupações no setor. E muitos procedimentos realizados extra classe ainda se encontram sendo realizados de forma manual, estando mais vulnerável ao erro ou até mesmo à violação do processo. Também ocorre um grande desperdício de tempo pelos professores e gestores da área, devido ao diversos processos ainda realizados de forma manual, sem a devida documentação. Segundo FULANO, também entrevistado, esse tempo perdido implica numa redução da eficiência na sala de aula, pois o professor acaba por ter menos tempo disponível para o planejamento das atividades, o que gera impactos negativos aos alunos.


\section{Objetivos} %<o que deve ser feito/entregue>


Devido aos muitos processos sendo resolvidos de forma manual, propõe-se com solução um sistema moderno, arquitetado para ter funcionamento na web e com um módulo mobile, a fim de fornecer informações de forma rápida e eficiente para os professores através de notificações, já que o acesso à internet móvel é comum entre os possíveis usuários do sistema em questão.
O principal requisito para o sistema seria dispor recursos para reduzir o tempo desperdiçado pelos professores durante as atividades extra classe.


\section{Metodologia} %<como será feito | como resolver o problema apontado inicialmente>


%<analise de literatura | design | implementação | validação>
Baseando-se nas tecnologias gratuitas em alta no cenário atual do desenvolvimento web, dispomos de algumas opções eficientes para a implementação da solução. Dentre as possibilidades, considerando a facilidade para futura manutenção e continuidade do projeto, tende-se a optar por uma tecnologia popular. Como linguagem de programação, adota-se o PHP. A escolha é fundamentada de acordo com a pesquisa da RedMonk de 2015, que evidencia o uso das linguagens de programação de acordo com as discussões no StackOverflow e repositórios no GitHub. É possível constatar a popularidade do PHP no cenário atual com o gráfico da pesquisa citada, na qual o PHP é apresentado na terceira colocação, apenas atrás do lider JavaScript e do segundo colocado, o Java.

\begin{figure}
	\label{fig:graficoRedmonk}
	\includegraphics[width=1\textwidth]{img/grafico_redmonk}
	\caption{Ranking das liguagens de programação no Stack Overflow e Github}
\end{figure}


Ainda assim, para compor a interface do dado projeto, também ocorrerá o uso do líder JavaScript de forma intensa, provendo o elo com o as informações gerenciadas pelo PHP.


Entretanto, não seria inteligente desenvolver um sistema completo sem o auxílio de um framework. Dentre os frameworks disponíveis para PHP, hoje o destaque está com o Laravel, que se encontra no topo dentre os mais utilizados no momento. 


A WebHostFace, uma empresa de hospedagem, compilou várias estatísticas para criar um infográfico mostrando os frameworks PHP mais populares de 2015. Utilizando informações sobre os próprios clientes, o Google Trends, estatísticas de repositórios do GitHub e a pesquisa do SitePoint “Best PHP Frameworks 2015”, a WebHostFace elaborou o seguinte infográfico: 

\begin{figure}
	\label{fig:graficoWebhostface}
	\includegraphics[width=1\textwidth]{img/infografico_webhostface}
	\caption{Infográfico da WebhostFace, exibindo a popularidade dos Frameworks PHP em 2015}
\end{figure}

Assim, tem-se a evidência que o Laravel em 2015 teve a maior popularidade em projetos pessoais e tem a maior comunidade entre os concorrentes, o que o torna uma boa escolha para a escrita de um software que será continuado por terceiros.


Para elaborar os recursos de interface e integrar ao back-end PHP do sistema, será adotado o já conhecido AngularJS, ferramenta sólida e conhecida no aspecto em questão. 


Dados coletados via Google Trends, que propõe comparações entre termos pesquisados, revela a popularidade do AngularJs diante de alguns dos principais concorrentes. O gráfico abaixo evidencia o cenário.


%Como mostra a Figura \ref{fig:graficoGoogleTrendsFerramentasFront}. 
\begin{figure}
	\label{fig:graficoGoogleTrendsFerramentasFront}
	\includegraphics[width=1\textwidth]{img/grafico_ferramentas_front}
	\caption{Gráfico do Google Trends exibindo as pesquisas por ferramentas front-end}
\end{figure}


Junto ao Angular JS, será utilizada a agradável tendência de interface do Material Design da Google, que propõe layouts limpos e otimizados já conhecidos pelos usuários de smartphones Android. 


Para a elaboração da plataforma mobile do projeto, será utilizado o Ionic Framework, muito difundido e bastante pesquisado na área, o que fica evidenciado com o gráfico de pesquisbaixo, coletado via Google Trends buscando por frameworks de desenvolvimento híbrido mobile.


\begin{figure}
	\label{fig:graficoGoogleTrendsFerramentasHibridasMobile}
	\includegraphics[width=1\textwidth]{img/grafico_ferramentas_hibridas_mobile}
	\caption{Gráfico do Google Trends exibindo as pesquisas por Frameworks híbridos mobile}
\end{figure}	

Para layout da interface mobile, também será aplicado a tendência do Material Design, a fim de propor uma harmonia entre o módulo web e mobile para os usuários


\section{Resultados Esperados}


Como fruto de um sistema para pós-graduação da UFBA, espera-se que os professores tenham mais recursos para integrar as atividades e também prover melhores condições para acompanhamento da vida acadêmica dos alunos em questão. Também, que os novos colaboradores que entrarem no processo tenham facilidade de compreender o fluxo do setor ao navegar pelo sistema proposto.


\section{Fora de Escopo}


Interação com os alunos devido às complicações para realizar a integração com o sistema empregado na UFBA, gerenciado pela XXXXXX, o que causaria uma inviabilidade no projeto devido à necessidade de entrega do produto ser mais forte que o tempo necessário para executar o processo de obtenção de acesso ao sistema legado para realizar a integração.


\section{Estrutura do Trabalho}


<breve resumo sobre os capítulos do TCC>
\chapter{Referencial Teórico}


Projetar o desenvolvimento de um software requer muito planejamento, pois as falhas iniciais podem custar bastante caro ou até mesmo inviabilizar a continuação de um projeto. Assim, a escolha da arquitetura ideal para a aplicabilidade é essencial na concepção de um produto de software. 
De todo o modo, sempre busca-se fazer mais com menos. Diante de tal filosofia, temos neste capítulo, uma breve discussão sobre alguns elementos de projeto e arquitetura de software, a fim de contextualizar este trabalho de conclusão de curso.
O capítulo corrente é composto por quatro seções. A \ref{sec:saas} trata de Software como serviço, discutindo alguns elementos do contexto que são relevantes para o trabalho proposto. A \ref{sec:reuso}  discute sobre a empregabilidade do reuso de software. A \ref{sec:modularizacao} seção, trata sobre aspectos envolvidos na modularização dos softwares. Por fim, a \ref{sec:apps_web} aborda as aplicações web, discutindo sobre aspectos relevantes sobre a aplicação web que compõe este trabalho.


 \section{Software como serviço}\label{sec:saas}


Segundo La e Chun \citep{La2009Systematic}, o princípio da definição de Software como um Serviço (Sofware as a Service - SaaS) é um serviço complementar para aplicações da computação em nuvem (cloud computing). No entanto, as áreas não se confundem. SaaS deve ser entendido como um mecanismo de suporte às soluções existentes na cloud. Os SaaS existem justamente para maximizar o reuso de serviços repetidos e não centrais em uma aplicação remota.


Como vantagens, diversos fatores podem ser favoráveis para a adoção de um SaaS, como custo e manutenção dentre outros fatores aplicáveis a soluções específicas. Lechesaet al. \citep{LechesaSS11}, quantificam índices sobre os fatores determinantes para adoção ou não de um SaaS voltado para ERP na África do Sul. Os principais fatores determinantes para adoção desse mecanismo de software são sua fluidez quanto à rede e a segurança. Esses fatores estão presentes na aplicação desenvolvida neste trabalho de conclusão de curso.


Devido ao fato de ter um serviço constantemente na nuvem, fica o questionamento sobre a segurança da informação manipulada. Sabe-se que a vulnerabilidade na Web não é restrita ao SaaS, atingindo diversos âmbitos. Rai et al. \cite{journals/corr/RaiSM13} consideram que o avanço da computação em nuvem não é um problema apenas para os serviços Web do ponto de vista da segurança, pois muitos trabalhos na literatura mostram a área como mais um ponto de vulnerabilidade para diversos setores, a exemplo de infraestrutura. No mesmo artigo mencionado de Rai et al. \cite{journals/corr/RaiSM13}, também realizaram-se estudos exploratórios junto a empresas usuárias de serviços em computação em nuvem e consideram que a perspectiva de SaaS também pode fortalecer a segurança nas aplicações de cloud computing, pois o software de autenticação compartilhado por várias aplicações em nuvem, oferece uma melhor padronização e consequente facilidade de prevenção a erros de vulnerabilidade específicas de cada módulo da pesquisa. Esse ponto de vista é fundamenta para a compreensão de SaaS.


A arquitetura de armazenamento de dados de um Saas pode variar de acordo com a necessidade do contexto. Huixin \cite{7586486} descreve possíveis modelagens nesse sentido. Essa abordagem pode ser com um banco de dados único, fazendo com que diferentes clientes compartilhem o mesmo banco, diferindo os dados através de controle de usuário, ou isolando os diferentes clientes através de bancos de dados exclusivos para cada um. Esse fator também pode ser combinado com a arquitetura da aplicação, caso ofereça aplicação única para todos os clientes ou aplicação compartilhada. Diante das possíveis abordagens, a modelagem de dados do software pode ser decidida pela regra de negócio. Este trabalho optou por aplicação única e banco de dados compartilhado.

Devido ao conceito particular de obtenção de software, tanto pela visão do cliente como pela visão do vendedor, é necessário compreender os diversos aspectos que podem ser relevantes ao orçar um Saas. O recente trabalho de T. Kaur et al. \citep{6949281} orienta um modelo para compor o custo de um Saas. O custo total seria composto pelos fatores que dão suporte ao funcionamento do software. Tais fatores incluem infra-estrutura, configurabilidade, customização, parâmetros de QoS (Quality of service) como escalabilidade, disponibilidade, usabilidade, pontualidade e desempenho da resposta, portabilidade, custo total de propriedade e retorno do investimento. Esses fatores caracterizam o custo de forma eficaz, possibilitando ao fornecedor, prover um Serviço de acordo com a exigência do consumidor em vários pacotes de serviços.


O conceito de software como serviço se aplica neste trabalho de conclusão de curso, pois o mesmo estará disponível na web com alta disponibilidade adotando as características apresentadas para qualquer pessoa que desejar usá-lo. Assim, basta se cadastrar, configurá-lo e usar como um serviço, sem preocupação com a manutenção do mesmo.


\section{Reuso de software}\label{sec:reuso} %CRUISE BOOK CAPITULO 2


De acordo com o livro Software pratical reuse \citep{Ezran2002}, o reuso de software é a utilização de qualquer informação que um desenvolvedor pode necessitar no processo de criação de software. O livro de Basili e Rombach \citep{Basili1991} define reutilização de software como o uso de tudo o que está associado a um projeto de conhecimento.
Assim, o objetivo da reutilização de software é reciclar o design, código e outros componentes de um produto de software e assim reduzir o custo, o tempo e melhorar a qualidade do produto.
Segundo Keswani et al. \cite{6783445}, o componente reutilizável de software pode ser qualquer parte de seu desenvolvimento, como um fragmento de código, design, casos de teste, ou até mesmo a especificação de requisitos de uma funcionalidade do software. 

O reuso de software pode ter impacto positivo em diversos aspectos do software, vejamos alguns, conforme apresentados no C.R.U.I.S.E Book \citep{cruise2007} :

\begin{itemize}

\item Qualidade: As correções de erro tornam-se úteis em todos os locais em que ocorreu, padronizando e facilitando a manutenção.

\item Produtividade: O ganho de produtividade é alcançado devido ao menor número de artefatos desenvolvido. Isso resulta em menor esforço de teste e também análise e design, reduzindo custos.

\item Confiabilidade: A utilização de componentes bem testados aumenta a
confiança no software. Além disso, a utilização de um mesmo componente em vários sistemas, aumenta a possibilidade de detecção de erros e reforça a confiança no componente.

\item Redução do Esforço: A reutilização de software proporciona uma redução do tempo de desenvolvimento, o que reduz o tempo necessário para o produto ser disponibilizado no mercado para trazer rentabilidade.

\item Trabalho redundante e tempo de desenvolvimento: Desenvolver um sistema do
zero significa desenvolvimento redundante de muitos componentes, como requisitos, especificações, casos de uso, arquitetura, etc. Isso pode ser evitado quando estes estão disponíveis como componentes reutilizáveis e podem ser compartilhados, resultando em um processo de desenvolvimento otimizado.

\item Documentação: Embora a documentação seja muito importante para a
manutenção de um sistema, muitas vezes é negligenciada. A reutilização de componentes de software reduz a quantidade de documentação a ser escrita, entretanto depende da qualidade do que está escrito. Assim, apenas a estrutura do sistema e os novos artefatos desenvolvidos necessitam ser documentados.

\item Custo de manutenção: Menos defeitos e manutenções são esperados quando tem-se comprovada a qualidade dos componentes utilizados.

\item Tamanho da equipe: É comum haver casos em que a equipe de desenvolvimento sofre sobrecarga. Entretanto, dobrar o tamanho da equipe de desenvolvimento não necessariamente duplica produtividade. Se muitos componentes podem ser reutilizados, é possível desenvolver com equipes menores, levando a melhor comunicação e aumento da produtividade.

\end{itemize}

Apesar dos benefícios da reutilização de software, ela não é suficientemente aproveitada. Existem fatores que influenciam direta ou indiretamente na sua adoção. Esses fatores podem ser de aspecto gerencial, organizacional, econômico, conceitual ou técnico. Veremos a seguir alguns aspectos que podem gerar conflito com a cultura de reuso de software, segundo o C.R.U.I.S.E Book \citep{cruise2007} :
%(Sametinger, 1997). REVER

\begin{itemize}
	
\item Falta de apoio da gestão: Como a reutilização de software gera custos iniciais,
a medida pode não ser amplamente alcançada em uma organização sem o apoio de alto nível de gestão. Os gestores têm de ser informados sobre os custos iniciais e serem convencidos sobre economias futuras.

\item Gerenciamento do Projeto: Gerenciar projetos tradicionais é uma tarefa árdua, principalmente, os que praticam a reutilização de software. Utilizando a técnica em larga escala, tem-se impacto sobre todo o ciclo de vida do software.

\item Estruturas organizacionais inadequadas: As estruturas organizacionais devem
considerar diferentes necessidades que surgem quando a reutilização em larga escala está sendo adotada. Por exemplo, uma equipe particionada pode ser alocada somente para desenvolver, manter e certificar componentes reutilizáveis de software.

\item Incentivos de gestão: É comum a falta de incentivo para deixar os desenvolvedores gastarem tempo elaborando componentes do sistemas. A produtividade é muitas vezes medida apenas no tempo necessário para concluir um projeto. Assim, fazer qualquer trabalho além disso, embora benéfico para a empresa como um todo, diminui o seu sucesso. Mesmo quando os componentes reutilizáveis são utilizados, os benefícios obtidos são uma pequena fração do que poderia ser alcançado caso houvesse reutilização explícita, planejada e organizada.

\item Dificuldade de encontrar software reutilizável: Para reutilizar os componentes, devem existir formas eficientes de busca. Além disso, é importante ter um repositório bem organizado contendo componentes com um eficiente meio de acesso.

\item Não reutilização do software encontrado. O acesso fácil ao software existente
não necessariamente aumentar a reutilização. Os componentes reutilizáveis devem ser cuidadosamente especificados, projetados, implementados e documentados, pois em alguns casos, modificar e adaptar o código  pode ser mais custoso que a programação da funcionalidade necessária a partir do zero.

\item Modificação: É muito difícil encontrar um componente que funcione
exatamente da mesma maneira que queremos. Desta forma, são necessárias modificações e devem existir formas de determinar os seus efeitos sobre o componente.


\end{itemize}


A definição do reuso de software pode ser associada a este trabalho, pois ainda que como um serviço, estaremos adotando uma aplicação única para servir aos mais diversos softwares de uma empresa, por exemplo. Logo, ao adotar o projeto, pode ser interessante aplicar algumas métricas para tomar conhecimento de possíveis vantanjens como as citadas nessa seção, a exemplo de economia de tempo de desenvolvimento, e custo do projeto.


%Outra diretriz importante para a reutilização de software é reduzir o risco na criação de novos softwares. O risco tende a ser bastante reduzido se os componentes que estão sendo reutilizados têm as documentação, interfaces necessárias e devidamente testadas, fatores que contibruem para uma fácil integração.
%De acordo com Keswani et al. \citep{6783445}, para o reuso de software dar retornos apropriados, o processo deve ser sistemático e planejado. Qualquer organização que implemente a reutilização de software deve identificar os melhores métodos e estratégias de reutilização para obter a máxima produtividade. A reutilização de software ajuda a evitar software de engenharia a partir do zero, pois usa módulos de software existentes. A reutilização de software, embora seja uma tarefa difícil, especialmente para softwares antigos sem padrões de projeto, pode melhorar significativamente a produtividade e a qualidade de um produto de software. Embora a reutilização de software não seja um novo campo, ela pode dar grandes retornos em curto período de tempo.


\section{Modularização}\label{sec:modularizacao} %artigo de claudio pagina 222 introdução


%A modularidade vem desempenhando um papel predominante estágios emergentes das disciplinas de arquitetura de software [13]. Engenheiros de software consideram modularidade como princípio base na comparação entre arquiteturas alternativas  e arquitetura degeneração [9]. De fato, os engenheiros de software são incentivados a arquitecturas, baseando-se numa multiplicidade de mecanismos de modularidade disponíveis em: 
%(i) Linguagens de descrição de arquitetura (ADLs), como ACME [8], 
%(ii) catálogos de arquitetônicos [2, 13], e 
%(iii) conhecem bem princípios de alto nível, como interfaces de componentes estreitos, acoplamento arquitectónico reduzido e semelhantes.


Conforme é frisado por Wickramaarachchi e Lai \citep{7062705}, o conceito de modularização na indústria de software tem uma longa história e tem sido utilizado para melhorar o processo de desenvolvimento de software em diferentes estágios. Os principais conceitos por trás da modularização do software foram introduzidos por pesquisadores pioneiros há quarenta anos, com uma notável contribuição feita por Melvin Conway e David Parnas, que tem representação notável na engenharia de software.


Modularizar um software é um bom padrão a ser adotado. Segundo Wickramaarachchi e Lai \citep{7062705}, a modularização é importante na identificação de dependências e reduz as dificuldades diante de uma possível necessidade de grandes alterações. De uma perspectiva da engenharia de software, uma modularização geralmente tem várias vantagens, tais como: tornar a complexidade do software mais gerenciável, facilitar o trabalho paralelo e tornar o software mais maleável para acomodar o futuro incerto que um software pode ter. O objetivo final da modularização do software é aumentar a produtividade ea qualidade do software. Tal conceito encontra-se bastante difundido e estái incorporado em linguagens de programação e ferramentas de software. O presente trabalho favorece ao uso da modularização de um software e até mesmo pode ser considerado um módulo a ser acoplado a qualquer software, mediante a compatibilidade.


Ao realizar a adoção do trabalho proposto, fica bastante evidente a "responsabilidade" do mesmo no escopo do projeto. Assim, é possível tratá-lo como um módulo do projeto que o usa, mesmo que seja consumido como um serviço.


\section{Aplicações web}\label{sec:apps_web}


A popularidade das soluções Web aumentou exponencialmente na última década e todos os dias cresce o número de pessoas usuárias desse tipo de software. E seguindo um padrão próprio, Kumar et al. \citep{7813710} sugerem que para o desenvolvimento web, deve-se manter a prática eficaz de produzir diagramas UML. A abordagem baseada na web oferece uma maneira fácil e eficaz para gerenciar e controlar o processo de desenvolvimento por meio de artefatos de modelagem. Tal abordagem pode ser usada quando há uma exigência de lidar com mudanças muito rápidas e grandes em requisitos de forma muito eficaz em muito menos tempo, gerando assim um menor impacto. 


Para atender à fomentada demanda de aplicativos web, é necessário adotar métodos de desenvolvimentos que sejam ágeis, eficientes e de fácil manutenção. Yu Ping et al. \cite{1372143} propõem o uso do modelo MVC (Model, View e Controller) no desenvolvimento para softwares web. O modelo apresentado tornou-se um padrão popular e divide o software em camadas com propósito definido, tornando-o de mais fácil manutenção.


O Ajax (Asynchronous Javascript and XML) revolucionou a web. Conforme demonstrado por Yuping \citep{6845605}, ao usar a tecnologia Ajax, podemos enriquecer a experiência do usuário em aplicações baseadas em navegador de internet, e fornecer uma variedade de aplicações interativas para atender às necessidade de humanização das aplicações.
Os aplicativos Ajax em execução no navegador se comunicam com um servidor Web de forma assíncrona e atualizam apenas uma parte da página.


No artigo de Tesarik et al. \citep{4664349}, temos que o design de software SPA (Single page application) é uma maneira interessante de criar um software Web numa uma única página. Essa solução de página única sem navegação funciona apenas com base em técnicas dinâmicas e assíncronas, como o AJAX (citado no parágrafo anterior). No entanto, esta abordagem coloca o desenvolvedor antes de alguns desafios substanciais.
Para projetar a interface do usuário que mostra as informações, é importante projetar corretamente a tela para manipular os dados do aplicativo numa única página. O design da página deve ser elaborado para maximizar a decomposição da página em componentes distintos que encapsulam os principais casos de uso.
Também recomenda-se que para elaborar os artefatos visuais, explore os mais ricos recursos na implementação, como HTML5, JavaScript, Ajax, CSS3, e outras tecnologias que se apliquem. Entretanto, desenvolver uma rica interface com o uso de diversas tecnologias/frameworks pode ocasionar um esforço maior para explorar as possibilidades.
Tais componentes da interface de uma aplicação SPA normalmente são alimentados mediante o consumo de uma API Rest via requisições AJAX. Assim, é possível particionar as responsabilidades de processamento do software entre cliente e servidor.


Como é possível denotar no trabalho de Salvadori e Siqueire \citep{7195633}, REST é uma arquitetura muito popular para integração de aplicativos web, que permite compartilhar e reutilizar informações através de sistemas. Aplicações de grande escala baseadas em no REST deve ser implementadas utilizando estratégias e mecanismos para produzir sistemas que sejam fáceis de desenvolver, reutilizar e manter. As interfaces de integração fornecidas por esses sistemas, chamadas de Web API, têm uma influência importante nas características da implementação resultante, pois a sua forma de resposta deve ser casada com a interface que o consome.
 
 
Os elementos comentados nessa seção encontram-se presentes neste trabalho de conclusão de curso. Juntos, montam a estrutura tecnológica necessária aliada à arquitetura adotada. Esses elementos seguem tendências atuais dos softwares web, fazendo com que esse trabalho esteja composto por tecnologias modernas que estão com boa aceitação no mercado.

% %% RiSE Latex Template - version 0.5
%%
%% RiSE's latex template for thesis and dissertations
%% http://risetemplate.sourceforge.net
%%
%% (c) 2012 Yguaratã Cerqueira Cavalcanti (yguarata@gmail.com)
%%          Vinicius Cardoso Garcia (vinicius.garcia@gmail.com)
%%
%% This document was initially based on UFPEThesis template, from Paulo Gustavo
%% S. Fonseca.
%%
%% ACKNOWLEDGEMENTS
%%
%% We would like to thanks the RiSE's researchers community, the 
%% students from Federal University of Pernambuco, and other users that have
%% been contributing to this projects with comments and patches.
%%
%% GENERAL INSTRUCTIONS
%%
%% We strongly recommend you to compile your documents using pdflatex command.
%% It is also recommend use the texlipse plugin for Eclipse to edit your documents.
%%
%% Options for \documentclass command:
%%         * Idiom
%%           pt   - Portguese (default)
%%           en   - English
%%
%%         * Text type
%%           bsc  - B.Sc. Thesis
%%           msc  - M.Sc. Thesis (default)
%%           qual - PHD qualification (not tested yet)
%%           prop - PHD proposal (not tested yet)
%%           phd  - PHD thesis
%%
%%         * Media
%%           scr  - to eletronic version (PDF) / see the users guide
%%
%%         * Pagination
%%           oneside - unique face press
%%           twoside - two faces press
%%
%%		   * Line spacing
%%           singlespacing  - the same as using \linespread{1}
%%           onehalfspacing - the same as using \linespread{1.3}
%%           doublespacing  - the same as using \linespread{1.6}
%%
%% Reference commands. Use the following commands to make references in your
%% text:
%%          \figref  -- for Figure reference
%%          \tabref  -- for Table reference
%%          \eqnref  -- for equation reference
%%          \chapref -- for chapter reference
%%          \secref  -- for section reference
%%          \appref  -- for appendix reference
%%          \axiref  -- for axiom reference
%%          \conjref -- for conjecture reference
%%          \defref  -- for definition reference
%%          \lemref  -- for lemma reference
%%          \theoref -- for theorem reference
%%          \corref  -- for corollary reference
%%          \propref -- for proprosition reference
%%          \pgref   -- for page reference
%%
%%          Example: See \chapref{chap:introduction}. It will produce 
%%                   'See Chapter 1', in case of English language.

\documentclass[pt,twoside,onehalfspacing,bsc]{risethesis}

\usepackage[sort,square]{natbib} %package for bibliography
\bibliographystyle{ieeetr} %authordate1

\usepackage[utf8]{inputenc}
\usepackage[brazilian]{babel}
\usepackage[T1]{fontenc}

\usepackage{graphicx}

%% Change the following pdf author attribute name to your name.
\usepackage[linkcolor=blue,citecolor=blue,urlcolor=blue,colorlinks,pdfpagelabels,pdftitle={Bruno Cabral's Bachelor Thesis},pdfauthor={Bruno Cabral}]{hyperref}

\address{SALVADOR}

\universitypt{Universidade Federal da Bahia}
\universityen{Federal University of Bahia}

\departmentpt{Depertamento de Ciência da Computação}
\departmenten{Computer Science Department}

\programpt{Programa Multiinstitucional de Pós-graduação em Ciência da Computação}
\programen{Graduate in Computer Science}

\majorfieldpt{Ciência da Computação}
\majorfielden{Computer Science}

\title{Sistema de apoio à Pós graduação - UFBA}
\date{Outubro/2016}

\author{Victor de Azevedo Nunes}
\adviser{Ivan do Carmo Machado}

\begin{document}

\frontmatter
\frontpage
\presentationpage

\begin{dedicatory}
Eu dedico esta dissertação...
%I dedicate this dissertation to my family, girlfriend, friends and
%professors who gave me all necessary support to get here.
\end{dedicatory}

\acknowledgements
Meus agradecimentos...

\begin{epigraph}[]{Edward V Berard}
Walking on water and developing software from a specification are easy if both are frozen
\end{epigraph}

\resumo
% Escreva seu resumo no arquivo resumo.tex
\input{resumo}

\abstract
% Write your abstract in a file called abstract.tex
\input{abstract}

% Summary (tables of contents)
\tableofcontents

% List of figures
\listoffigures

% List of tables
\listoftables

% List of acronyms
% Acronyms manual: http://linorg.usp.br/CTAN/macros/latex/contrib/acronym/acronym.pdf
\listofacronyms
\input{acronyms}

% List of listings
%\lstlistoflistings

\mainmatter

\include{chapters/intro}
\include{chapters/referencial_teorico}

% \include{chapters/introduction/main}
% \include{chapters/background/main}
% \include{chapters/proposed_solution/main}
% \include{chapters/experiment/main}
% \include{chapters/conclusion/main}

%\bibliographystyle{natbib}
%\addcontentsline{toc}{chapter}{\bibliographytocname}
\bibliography{references}

% Appendix
\clearpage
\addappheadtotoc
\appendix
\appendixpage
% \include{appendix/experiment-instruments}

\end{document}
% %% RiSE Latex Template - version 0.5
%%
%% RiSE's latex template for thesis and dissertations
%% http://risetemplate.sourceforge.net
%%
%% (c) 2012 Yguaratã Cerqueira Cavalcanti (yguarata@gmail.com)
%%          Vinicius Cardoso Garcia (vinicius.garcia@gmail.com)
%%
%% This document was initially based on UFPEThesis template, from Paulo Gustavo
%% S. Fonseca.
%%
%% ACKNOWLEDGEMENTS
%%
%% We would like to thanks the RiSE's researchers community, the 
%% students from Federal University of Pernambuco, and other users that have
%% been contributing to this projects with comments and patches.
%%
%% GENERAL INSTRUCTIONS
%%
%% We strongly recommend you to compile your documents using pdflatex command.
%% It is also recommend use the texlipse plugin for Eclipse to edit your documents.
%%
%% Options for \documentclass command:
%%         * Idiom
%%           pt   - Portguese (default)
%%           en   - English
%%
%%         * Text type
%%           bsc  - B.Sc. Thesis
%%           msc  - M.Sc. Thesis (default)
%%           qual - PHD qualification (not tested yet)
%%           prop - PHD proposal (not tested yet)
%%           phd  - PHD thesis
%%
%%         * Media
%%           scr  - to eletronic version (PDF) / see the users guide
%%
%%         * Pagination
%%           oneside - unique face press
%%           twoside - two faces press
%%
%%		   * Line spacing
%%           singlespacing  - the same as using \linespread{1}
%%           onehalfspacing - the same as using \linespread{1.3}
%%           doublespacing  - the same as using \linespread{1.6}
%%
%% Reference commands. Use the following commands to make references in your
%% text:
%%          \figref  -- for Figure reference
%%          \tabref  -- for Table reference
%%          \eqnref  -- for equation reference
%%          \chapref -- for chapter reference
%%          \secref  -- for section reference
%%          \appref  -- for appendix reference
%%          \axiref  -- for axiom reference
%%          \conjref -- for conjecture reference
%%          \defref  -- for definition reference
%%          \lemref  -- for lemma reference
%%          \theoref -- for theorem reference
%%          \corref  -- for corollary reference
%%          \propref -- for proprosition reference
%%          \pgref   -- for page reference
%%
%%          Example: See \chapref{chap:introduction}. It will produce 
%%                   'See Chapter 1', in case of English language.

\documentclass[pt,twoside,onehalfspacing,bsc]{risethesis}

\usepackage[sort,square]{natbib} %package for bibliography
\bibliographystyle{ieeetr} %authordate1

\usepackage[utf8]{inputenc}
\usepackage[brazilian]{babel}
\usepackage[T1]{fontenc}

\usepackage{graphicx}

%% Change the following pdf author attribute name to your name.
\usepackage[linkcolor=blue,citecolor=blue,urlcolor=blue,colorlinks,pdfpagelabels,pdftitle={Bruno Cabral's Bachelor Thesis},pdfauthor={Bruno Cabral}]{hyperref}

\address{SALVADOR}

\universitypt{Universidade Federal da Bahia}
\universityen{Federal University of Bahia}

\departmentpt{Depertamento de Ciência da Computação}
\departmenten{Computer Science Department}

\programpt{Programa Multiinstitucional de Pós-graduação em Ciência da Computação}
\programen{Graduate in Computer Science}

\majorfieldpt{Ciência da Computação}
\majorfielden{Computer Science}

\title{Sistema de apoio à Pós graduação - UFBA}
\date{Outubro/2016}

\author{Victor de Azevedo Nunes}
\adviser{Ivan do Carmo Machado}

\begin{document}

\frontmatter
\frontpage
\presentationpage

\begin{dedicatory}
Eu dedico esta dissertação...
%I dedicate this dissertation to my family, girlfriend, friends and
%professors who gave me all necessary support to get here.
\end{dedicatory}

\acknowledgements
Meus agradecimentos...

\begin{epigraph}[]{Edward V Berard}
Walking on water and developing software from a specification are easy if both are frozen
\end{epigraph}

\resumo
% Escreva seu resumo no arquivo resumo.tex
\input{resumo}

\abstract
% Write your abstract in a file called abstract.tex
\input{abstract}

% Summary (tables of contents)
\tableofcontents

% List of figures
\listoffigures

% List of tables
\listoftables

% List of acronyms
% Acronyms manual: http://linorg.usp.br/CTAN/macros/latex/contrib/acronym/acronym.pdf
\listofacronyms
\input{acronyms}

% List of listings
%\lstlistoflistings

\mainmatter

\include{chapters/intro}
\include{chapters/referencial_teorico}

% \include{chapters/introduction/main}
% \include{chapters/background/main}
% \include{chapters/proposed_solution/main}
% \include{chapters/experiment/main}
% \include{chapters/conclusion/main}

%\bibliographystyle{natbib}
%\addcontentsline{toc}{chapter}{\bibliographytocname}
\bibliography{references}

% Appendix
\clearpage
\addappheadtotoc
\appendix
\appendixpage
% \include{appendix/experiment-instruments}

\end{document}
% %% RiSE Latex Template - version 0.5
%%
%% RiSE's latex template for thesis and dissertations
%% http://risetemplate.sourceforge.net
%%
%% (c) 2012 Yguaratã Cerqueira Cavalcanti (yguarata@gmail.com)
%%          Vinicius Cardoso Garcia (vinicius.garcia@gmail.com)
%%
%% This document was initially based on UFPEThesis template, from Paulo Gustavo
%% S. Fonseca.
%%
%% ACKNOWLEDGEMENTS
%%
%% We would like to thanks the RiSE's researchers community, the 
%% students from Federal University of Pernambuco, and other users that have
%% been contributing to this projects with comments and patches.
%%
%% GENERAL INSTRUCTIONS
%%
%% We strongly recommend you to compile your documents using pdflatex command.
%% It is also recommend use the texlipse plugin for Eclipse to edit your documents.
%%
%% Options for \documentclass command:
%%         * Idiom
%%           pt   - Portguese (default)
%%           en   - English
%%
%%         * Text type
%%           bsc  - B.Sc. Thesis
%%           msc  - M.Sc. Thesis (default)
%%           qual - PHD qualification (not tested yet)
%%           prop - PHD proposal (not tested yet)
%%           phd  - PHD thesis
%%
%%         * Media
%%           scr  - to eletronic version (PDF) / see the users guide
%%
%%         * Pagination
%%           oneside - unique face press
%%           twoside - two faces press
%%
%%		   * Line spacing
%%           singlespacing  - the same as using \linespread{1}
%%           onehalfspacing - the same as using \linespread{1.3}
%%           doublespacing  - the same as using \linespread{1.6}
%%
%% Reference commands. Use the following commands to make references in your
%% text:
%%          \figref  -- for Figure reference
%%          \tabref  -- for Table reference
%%          \eqnref  -- for equation reference
%%          \chapref -- for chapter reference
%%          \secref  -- for section reference
%%          \appref  -- for appendix reference
%%          \axiref  -- for axiom reference
%%          \conjref -- for conjecture reference
%%          \defref  -- for definition reference
%%          \lemref  -- for lemma reference
%%          \theoref -- for theorem reference
%%          \corref  -- for corollary reference
%%          \propref -- for proprosition reference
%%          \pgref   -- for page reference
%%
%%          Example: See \chapref{chap:introduction}. It will produce 
%%                   'See Chapter 1', in case of English language.

\documentclass[pt,twoside,onehalfspacing,bsc]{risethesis}

\usepackage[sort,square]{natbib} %package for bibliography
\bibliographystyle{ieeetr} %authordate1

\usepackage[utf8]{inputenc}
\usepackage[brazilian]{babel}
\usepackage[T1]{fontenc}

\usepackage{graphicx}

%% Change the following pdf author attribute name to your name.
\usepackage[linkcolor=blue,citecolor=blue,urlcolor=blue,colorlinks,pdfpagelabels,pdftitle={Bruno Cabral's Bachelor Thesis},pdfauthor={Bruno Cabral}]{hyperref}

\address{SALVADOR}

\universitypt{Universidade Federal da Bahia}
\universityen{Federal University of Bahia}

\departmentpt{Depertamento de Ciência da Computação}
\departmenten{Computer Science Department}

\programpt{Programa Multiinstitucional de Pós-graduação em Ciência da Computação}
\programen{Graduate in Computer Science}

\majorfieldpt{Ciência da Computação}
\majorfielden{Computer Science}

\title{Sistema de apoio à Pós graduação - UFBA}
\date{Outubro/2016}

\author{Victor de Azevedo Nunes}
\adviser{Ivan do Carmo Machado}

\begin{document}

\frontmatter
\frontpage
\presentationpage

\begin{dedicatory}
Eu dedico esta dissertação...
%I dedicate this dissertation to my family, girlfriend, friends and
%professors who gave me all necessary support to get here.
\end{dedicatory}

\acknowledgements
Meus agradecimentos...

\begin{epigraph}[]{Edward V Berard}
Walking on water and developing software from a specification are easy if both are frozen
\end{epigraph}

\resumo
% Escreva seu resumo no arquivo resumo.tex
\input{resumo}

\abstract
% Write your abstract in a file called abstract.tex
\input{abstract}

% Summary (tables of contents)
\tableofcontents

% List of figures
\listoffigures

% List of tables
\listoftables

% List of acronyms
% Acronyms manual: http://linorg.usp.br/CTAN/macros/latex/contrib/acronym/acronym.pdf
\listofacronyms
\input{acronyms}

% List of listings
%\lstlistoflistings

\mainmatter

\include{chapters/intro}
\include{chapters/referencial_teorico}

% \include{chapters/introduction/main}
% \include{chapters/background/main}
% \include{chapters/proposed_solution/main}
% \include{chapters/experiment/main}
% \include{chapters/conclusion/main}

%\bibliographystyle{natbib}
%\addcontentsline{toc}{chapter}{\bibliographytocname}
\bibliography{references}

% Appendix
\clearpage
\addappheadtotoc
\appendix
\appendixpage
% \include{appendix/experiment-instruments}

\end{document}
% %% RiSE Latex Template - version 0.5
%%
%% RiSE's latex template for thesis and dissertations
%% http://risetemplate.sourceforge.net
%%
%% (c) 2012 Yguaratã Cerqueira Cavalcanti (yguarata@gmail.com)
%%          Vinicius Cardoso Garcia (vinicius.garcia@gmail.com)
%%
%% This document was initially based on UFPEThesis template, from Paulo Gustavo
%% S. Fonseca.
%%
%% ACKNOWLEDGEMENTS
%%
%% We would like to thanks the RiSE's researchers community, the 
%% students from Federal University of Pernambuco, and other users that have
%% been contributing to this projects with comments and patches.
%%
%% GENERAL INSTRUCTIONS
%%
%% We strongly recommend you to compile your documents using pdflatex command.
%% It is also recommend use the texlipse plugin for Eclipse to edit your documents.
%%
%% Options for \documentclass command:
%%         * Idiom
%%           pt   - Portguese (default)
%%           en   - English
%%
%%         * Text type
%%           bsc  - B.Sc. Thesis
%%           msc  - M.Sc. Thesis (default)
%%           qual - PHD qualification (not tested yet)
%%           prop - PHD proposal (not tested yet)
%%           phd  - PHD thesis
%%
%%         * Media
%%           scr  - to eletronic version (PDF) / see the users guide
%%
%%         * Pagination
%%           oneside - unique face press
%%           twoside - two faces press
%%
%%		   * Line spacing
%%           singlespacing  - the same as using \linespread{1}
%%           onehalfspacing - the same as using \linespread{1.3}
%%           doublespacing  - the same as using \linespread{1.6}
%%
%% Reference commands. Use the following commands to make references in your
%% text:
%%          \figref  -- for Figure reference
%%          \tabref  -- for Table reference
%%          \eqnref  -- for equation reference
%%          \chapref -- for chapter reference
%%          \secref  -- for section reference
%%          \appref  -- for appendix reference
%%          \axiref  -- for axiom reference
%%          \conjref -- for conjecture reference
%%          \defref  -- for definition reference
%%          \lemref  -- for lemma reference
%%          \theoref -- for theorem reference
%%          \corref  -- for corollary reference
%%          \propref -- for proprosition reference
%%          \pgref   -- for page reference
%%
%%          Example: See \chapref{chap:introduction}. It will produce 
%%                   'See Chapter 1', in case of English language.

\documentclass[pt,twoside,onehalfspacing,bsc]{risethesis}

\usepackage[sort,square]{natbib} %package for bibliography
\bibliographystyle{ieeetr} %authordate1

\usepackage[utf8]{inputenc}
\usepackage[brazilian]{babel}
\usepackage[T1]{fontenc}

\usepackage{graphicx}

%% Change the following pdf author attribute name to your name.
\usepackage[linkcolor=blue,citecolor=blue,urlcolor=blue,colorlinks,pdfpagelabels,pdftitle={Bruno Cabral's Bachelor Thesis},pdfauthor={Bruno Cabral}]{hyperref}

\address{SALVADOR}

\universitypt{Universidade Federal da Bahia}
\universityen{Federal University of Bahia}

\departmentpt{Depertamento de Ciência da Computação}
\departmenten{Computer Science Department}

\programpt{Programa Multiinstitucional de Pós-graduação em Ciência da Computação}
\programen{Graduate in Computer Science}

\majorfieldpt{Ciência da Computação}
\majorfielden{Computer Science}

\title{Sistema de apoio à Pós graduação - UFBA}
\date{Outubro/2016}

\author{Victor de Azevedo Nunes}
\adviser{Ivan do Carmo Machado}

\begin{document}

\frontmatter
\frontpage
\presentationpage

\begin{dedicatory}
Eu dedico esta dissertação...
%I dedicate this dissertation to my family, girlfriend, friends and
%professors who gave me all necessary support to get here.
\end{dedicatory}

\acknowledgements
Meus agradecimentos...

\begin{epigraph}[]{Edward V Berard}
Walking on water and developing software from a specification are easy if both are frozen
\end{epigraph}

\resumo
% Escreva seu resumo no arquivo resumo.tex
\input{resumo}

\abstract
% Write your abstract in a file called abstract.tex
\input{abstract}

% Summary (tables of contents)
\tableofcontents

% List of figures
\listoffigures

% List of tables
\listoftables

% List of acronyms
% Acronyms manual: http://linorg.usp.br/CTAN/macros/latex/contrib/acronym/acronym.pdf
\listofacronyms
\input{acronyms}

% List of listings
%\lstlistoflistings

\mainmatter

\include{chapters/intro}
\include{chapters/referencial_teorico}

% \include{chapters/introduction/main}
% \include{chapters/background/main}
% \include{chapters/proposed_solution/main}
% \include{chapters/experiment/main}
% \include{chapters/conclusion/main}

%\bibliographystyle{natbib}
%\addcontentsline{toc}{chapter}{\bibliographytocname}
\bibliography{references}

% Appendix
\clearpage
\addappheadtotoc
\appendix
\appendixpage
% \include{appendix/experiment-instruments}

\end{document}
% %% RiSE Latex Template - version 0.5
%%
%% RiSE's latex template for thesis and dissertations
%% http://risetemplate.sourceforge.net
%%
%% (c) 2012 Yguaratã Cerqueira Cavalcanti (yguarata@gmail.com)
%%          Vinicius Cardoso Garcia (vinicius.garcia@gmail.com)
%%
%% This document was initially based on UFPEThesis template, from Paulo Gustavo
%% S. Fonseca.
%%
%% ACKNOWLEDGEMENTS
%%
%% We would like to thanks the RiSE's researchers community, the 
%% students from Federal University of Pernambuco, and other users that have
%% been contributing to this projects with comments and patches.
%%
%% GENERAL INSTRUCTIONS
%%
%% We strongly recommend you to compile your documents using pdflatex command.
%% It is also recommend use the texlipse plugin for Eclipse to edit your documents.
%%
%% Options for \documentclass command:
%%         * Idiom
%%           pt   - Portguese (default)
%%           en   - English
%%
%%         * Text type
%%           bsc  - B.Sc. Thesis
%%           msc  - M.Sc. Thesis (default)
%%           qual - PHD qualification (not tested yet)
%%           prop - PHD proposal (not tested yet)
%%           phd  - PHD thesis
%%
%%         * Media
%%           scr  - to eletronic version (PDF) / see the users guide
%%
%%         * Pagination
%%           oneside - unique face press
%%           twoside - two faces press
%%
%%		   * Line spacing
%%           singlespacing  - the same as using \linespread{1}
%%           onehalfspacing - the same as using \linespread{1.3}
%%           doublespacing  - the same as using \linespread{1.6}
%%
%% Reference commands. Use the following commands to make references in your
%% text:
%%          \figref  -- for Figure reference
%%          \tabref  -- for Table reference
%%          \eqnref  -- for equation reference
%%          \chapref -- for chapter reference
%%          \secref  -- for section reference
%%          \appref  -- for appendix reference
%%          \axiref  -- for axiom reference
%%          \conjref -- for conjecture reference
%%          \defref  -- for definition reference
%%          \lemref  -- for lemma reference
%%          \theoref -- for theorem reference
%%          \corref  -- for corollary reference
%%          \propref -- for proprosition reference
%%          \pgref   -- for page reference
%%
%%          Example: See \chapref{chap:introduction}. It will produce 
%%                   'See Chapter 1', in case of English language.

\documentclass[pt,twoside,onehalfspacing,bsc]{risethesis}

\usepackage[sort,square]{natbib} %package for bibliography
\bibliographystyle{ieeetr} %authordate1

\usepackage[utf8]{inputenc}
\usepackage[brazilian]{babel}
\usepackage[T1]{fontenc}

\usepackage{graphicx}

%% Change the following pdf author attribute name to your name.
\usepackage[linkcolor=blue,citecolor=blue,urlcolor=blue,colorlinks,pdfpagelabels,pdftitle={Bruno Cabral's Bachelor Thesis},pdfauthor={Bruno Cabral}]{hyperref}

\address{SALVADOR}

\universitypt{Universidade Federal da Bahia}
\universityen{Federal University of Bahia}

\departmentpt{Depertamento de Ciência da Computação}
\departmenten{Computer Science Department}

\programpt{Programa Multiinstitucional de Pós-graduação em Ciência da Computação}
\programen{Graduate in Computer Science}

\majorfieldpt{Ciência da Computação}
\majorfielden{Computer Science}

\title{Sistema de apoio à Pós graduação - UFBA}
\date{Outubro/2016}

\author{Victor de Azevedo Nunes}
\adviser{Ivan do Carmo Machado}

\begin{document}

\frontmatter
\frontpage
\presentationpage

\begin{dedicatory}
Eu dedico esta dissertação...
%I dedicate this dissertation to my family, girlfriend, friends and
%professors who gave me all necessary support to get here.
\end{dedicatory}

\acknowledgements
Meus agradecimentos...

\begin{epigraph}[]{Edward V Berard}
Walking on water and developing software from a specification are easy if both are frozen
\end{epigraph}

\resumo
% Escreva seu resumo no arquivo resumo.tex
\input{resumo}

\abstract
% Write your abstract in a file called abstract.tex
\input{abstract}

% Summary (tables of contents)
\tableofcontents

% List of figures
\listoffigures

% List of tables
\listoftables

% List of acronyms
% Acronyms manual: http://linorg.usp.br/CTAN/macros/latex/contrib/acronym/acronym.pdf
\listofacronyms
\input{acronyms}

% List of listings
%\lstlistoflistings

\mainmatter

\include{chapters/intro}
\include{chapters/referencial_teorico}

% \include{chapters/introduction/main}
% \include{chapters/background/main}
% \include{chapters/proposed_solution/main}
% \include{chapters/experiment/main}
% \include{chapters/conclusion/main}

%\bibliographystyle{natbib}
%\addcontentsline{toc}{chapter}{\bibliographytocname}
\bibliography{references}

% Appendix
\clearpage
\addappheadtotoc
\appendix
\appendixpage
% \include{appendix/experiment-instruments}

\end{document}

%\bibliographystyle{natbib}
%\addcontentsline{toc}{chapter}{\bibliographytocname}
\bibliography{references}

% Appendix
\clearpage
\addappheadtotoc
\appendix
\appendixpage
% \include{appendix/experiment-instruments}

\end{document}
% %% RiSE Latex Template - version 0.5
%%
%% RiSE's latex template for thesis and dissertations
%% http://risetemplate.sourceforge.net
%%
%% (c) 2012 Yguaratã Cerqueira Cavalcanti (yguarata@gmail.com)
%%          Vinicius Cardoso Garcia (vinicius.garcia@gmail.com)
%%
%% This document was initially based on UFPEThesis template, from Paulo Gustavo
%% S. Fonseca.
%%
%% ACKNOWLEDGEMENTS
%%
%% We would like to thanks the RiSE's researchers community, the 
%% students from Federal University of Pernambuco, and other users that have
%% been contributing to this projects with comments and patches.
%%
%% GENERAL INSTRUCTIONS
%%
%% We strongly recommend you to compile your documents using pdflatex command.
%% It is also recommend use the texlipse plugin for Eclipse to edit your documents.
%%
%% Options for \documentclass command:
%%         * Idiom
%%           pt   - Portguese (default)
%%           en   - English
%%
%%         * Text type
%%           bsc  - B.Sc. Thesis
%%           msc  - M.Sc. Thesis (default)
%%           qual - PHD qualification (not tested yet)
%%           prop - PHD proposal (not tested yet)
%%           phd  - PHD thesis
%%
%%         * Media
%%           scr  - to eletronic version (PDF) / see the users guide
%%
%%         * Pagination
%%           oneside - unique face press
%%           twoside - two faces press
%%
%%		   * Line spacing
%%           singlespacing  - the same as using \linespread{1}
%%           onehalfspacing - the same as using \linespread{1.3}
%%           doublespacing  - the same as using \linespread{1.6}
%%
%% Reference commands. Use the following commands to make references in your
%% text:
%%          \figref  -- for Figure reference
%%          \tabref  -- for Table reference
%%          \eqnref  -- for equation reference
%%          \chapref -- for chapter reference
%%          \secref  -- for section reference
%%          \appref  -- for appendix reference
%%          \axiref  -- for axiom reference
%%          \conjref -- for conjecture reference
%%          \defref  -- for definition reference
%%          \lemref  -- for lemma reference
%%          \theoref -- for theorem reference
%%          \corref  -- for corollary reference
%%          \propref -- for proprosition reference
%%          \pgref   -- for page reference
%%
%%          Example: See \chapref{chap:introduction}. It will produce 
%%                   'See Chapter 1', in case of English language.

\documentclass[pt,twoside,onehalfspacing,bsc]{risethesis}

\usepackage[sort,square]{natbib} %package for bibliography
\bibliographystyle{ieeetr} %authordate1

\usepackage[utf8]{inputenc}
\usepackage[brazilian]{babel}
\usepackage[T1]{fontenc}

\usepackage{graphicx}

%% Change the following pdf author attribute name to your name.
\usepackage[linkcolor=blue,citecolor=blue,urlcolor=blue,colorlinks,pdfpagelabels,pdftitle={Bruno Cabral's Bachelor Thesis},pdfauthor={Bruno Cabral}]{hyperref}

\address{SALVADOR}

\universitypt{Universidade Federal da Bahia}
\universityen{Federal University of Bahia}

\departmentpt{Depertamento de Ciência da Computação}
\departmenten{Computer Science Department}

\programpt{Programa Multiinstitucional de Pós-graduação em Ciência da Computação}
\programen{Graduate in Computer Science}

\majorfieldpt{Ciência da Computação}
\majorfielden{Computer Science}

\title{Sistema de apoio à Pós graduação - UFBA}
\date{Outubro/2016}

\author{Victor de Azevedo Nunes}
\adviser{Ivan do Carmo Machado}

\begin{document}

\frontmatter
\frontpage
\presentationpage

\begin{dedicatory}
Eu dedico esta dissertação...
%I dedicate this dissertation to my family, girlfriend, friends and
%professors who gave me all necessary support to get here.
\end{dedicatory}

\acknowledgements
Meus agradecimentos...

\begin{epigraph}[]{Edward V Berard}
Walking on water and developing software from a specification are easy if both are frozen
\end{epigraph}

\resumo
% Escreva seu resumo no arquivo resumo.tex
Este trabalho de conclusão de curso utiliza elementos da engenharia de software para propor um software como serviço a fim de otimizar o processo de construção e manutenção dos softwares. Assim, o objetivo deste SaaS é gerenciar as permissões de acesso de sistemas cliente, provendo o reuso de software e padronizando as soluções. O sistema proposto neste trabalho encontra-se implementado e disponível no Github, e traz fundamentos sobre a arquitetura e tecnologias utilizadas, bem como avaliações sobre possibilidades de utilização do mesmo em diversos ambientes, como Web e mobile.

\begin{keywords}
Software; Reuso; SaaS; Web; PHP
\end{keywords}

\abstract
% Write your abstract in a file called abstract.tex
My abstract...

\begin{keywords}
key words...
\end{keywords}

% Summary (tables of contents)
\tableofcontents

% List of figures
\listoffigures

% List of tables
\listoftables

% List of acronyms
% Acronyms manual: http://linorg.usp.br/CTAN/macros/latex/contrib/acronym/acronym.pdf
\listofacronyms
\begin{acronym}[ACRONYM] 
% Change the word ACRONYM above to change the acronym column width.
% The column width is equals to the width of the word that you put.
% Read the manual about acronym package for more examples:
%   http://linorg.usp.br/CTAN/macros/latex/contrib/acronym/acronym.pdf

\acro{SPA}{Single Page Application}
\acro{JSON}{Javascript Object Notation}
\acro{PHP}{PHP: Hypertext Preprocessor}
\acro{SaaS}{Software as a Service}
\acro{ERP}{Enterprise Resource Planning}
\acro{QoS}{Quality of Service}
\acro{UML}{Unified Modeling Language}
\acro{MVC}{Model-View-Controller}
\acro{Ajax}{Asynchronous Javascript and XML}
\acro{HTML}{HyperText Markup Language}
\acro{CSS}{Cascading Style Sheets}
\acro{API}{Application Programming Interface}
\acro{DOM}{Document Object Model}
\acro{BPMN}{Business Process Model and Notation}
\acro{REST}{Representational State Transfer}

\end{acronym}

% List of listings
%\lstlistoflistings

\mainmatter

\chapter{Introdução}

\section{Motivação}

Organizar os procedimentos de um processo sempre nos traz vantagens. Apesar de no processo de implantação de um sistema, o mesmo burocratizar o processo, com o tempo temos o retorno da dedicação para a inserção dos dados. Com um certo volume de dados, é possível estruturar informações que num processo manual são difíceis de serem enxergadas. Assim, é possível depender menos das pessoas que organizam o processo, pois o legado de informações não estará mais somente na mente de alguns, mas sim documentado nos dados do sistema.

Além de colaborar na organização, também haverá uma grande colaboração no tempo gasto na gestão. Lidar com muitos papéis e confiar na mente humana para guardar informações, não é uma alternativa muito segura devido ao fato que as pessoas sempre estão sujeitas a sair do processo e levar contigo a experiência obtida. Experiência essa que faz com que os procedimentos sejam executados de forma mais eficiente. Entretanto, com um sistema inteligente, é possível auxiliar e tornar mais ágil a execução das tarefas.


\section{Problema}


De acordo com funcionários ligados ao o setor de pós graduação da UFBA, entrevistados a fim de um maior entendimento do cenário, apesar das semelhanças estruturais, a pós graduação gerida de forma diferencia da graduação. FULANO afirma que devido ao fato de não ter a mesma visibilidade, não tem acesso aos mesmos recursos de gestão acadêmica da graduação. O professores não executam somente atividades dentro da sala de aula, também tem diversas outras ocupações no setor. E muitos procedimentos realizados extra classe ainda se encontram sendo realizados de forma manual, estando mais vulnerável ao erro ou até mesmo à violação do processo. Também ocorre um grande desperdício de tempo pelos professores e gestores da área, devido ao diversos processos ainda realizados de forma manual, sem a devida documentação. Segundo FULANO, também entrevistado, esse tempo perdido implica numa redução da eficiência na sala de aula, pois o professor acaba por ter menos tempo disponível para o planejamento das atividades, o que gera impactos negativos aos alunos.


\section{Objetivos} %<o que deve ser feito/entregue>


Devido aos muitos processos sendo resolvidos de forma manual, propõe-se com solução um sistema moderno, arquitetado para ter funcionamento na web e com um módulo mobile, a fim de fornecer informações de forma rápida e eficiente para os professores através de notificações, já que o acesso à internet móvel é comum entre os possíveis usuários do sistema em questão.
O principal requisito para o sistema seria dispor recursos para reduzir o tempo desperdiçado pelos professores durante as atividades extra classe.


\section{Metodologia} %<como será feito | como resolver o problema apontado inicialmente>


%<analise de literatura | design | implementação | validação>
Baseando-se nas tecnologias gratuitas em alta no cenário atual do desenvolvimento web, dispomos de algumas opções eficientes para a implementação da solução. Dentre as possibilidades, considerando a facilidade para futura manutenção e continuidade do projeto, tende-se a optar por uma tecnologia popular. Como linguagem de programação, adota-se o PHP. A escolha é fundamentada de acordo com a pesquisa da RedMonk de 2015, que evidencia o uso das linguagens de programação de acordo com as discussões no StackOverflow e repositórios no GitHub. É possível constatar a popularidade do PHP no cenário atual com o gráfico da pesquisa citada, na qual o PHP é apresentado na terceira colocação, apenas atrás do lider JavaScript e do segundo colocado, o Java.

\begin{figure}
	\label{fig:graficoRedmonk}
	\includegraphics[width=1\textwidth]{img/grafico_redmonk}
	\caption{Ranking das liguagens de programação no Stack Overflow e Github}
\end{figure}


Ainda assim, para compor a interface do dado projeto, também ocorrerá o uso do líder JavaScript de forma intensa, provendo o elo com o as informações gerenciadas pelo PHP.


Entretanto, não seria inteligente desenvolver um sistema completo sem o auxílio de um framework. Dentre os frameworks disponíveis para PHP, hoje o destaque está com o Laravel, que se encontra no topo dentre os mais utilizados no momento. 


A WebHostFace, uma empresa de hospedagem, compilou várias estatísticas para criar um infográfico mostrando os frameworks PHP mais populares de 2015. Utilizando informações sobre os próprios clientes, o Google Trends, estatísticas de repositórios do GitHub e a pesquisa do SitePoint “Best PHP Frameworks 2015”, a WebHostFace elaborou o seguinte infográfico: 

\begin{figure}
	\label{fig:graficoWebhostface}
	\includegraphics[width=1\textwidth]{img/infografico_webhostface}
	\caption{Infográfico da WebhostFace, exibindo a popularidade dos Frameworks PHP em 2015}
\end{figure}

Assim, tem-se a evidência que o Laravel em 2015 teve a maior popularidade em projetos pessoais e tem a maior comunidade entre os concorrentes, o que o torna uma boa escolha para a escrita de um software que será continuado por terceiros.


Para elaborar os recursos de interface e integrar ao back-end PHP do sistema, será adotado o já conhecido AngularJS, ferramenta sólida e conhecida no aspecto em questão. 


Dados coletados via Google Trends, que propõe comparações entre termos pesquisados, revela a popularidade do AngularJs diante de alguns dos principais concorrentes. O gráfico abaixo evidencia o cenário.


%Como mostra a Figura \ref{fig:graficoGoogleTrendsFerramentasFront}. 
\begin{figure}
	\label{fig:graficoGoogleTrendsFerramentasFront}
	\includegraphics[width=1\textwidth]{img/grafico_ferramentas_front}
	\caption{Gráfico do Google Trends exibindo as pesquisas por ferramentas front-end}
\end{figure}


Junto ao Angular JS, será utilizada a agradável tendência de interface do Material Design da Google, que propõe layouts limpos e otimizados já conhecidos pelos usuários de smartphones Android. 


Para a elaboração da plataforma mobile do projeto, será utilizado o Ionic Framework, muito difundido e bastante pesquisado na área, o que fica evidenciado com o gráfico de pesquisbaixo, coletado via Google Trends buscando por frameworks de desenvolvimento híbrido mobile.


\begin{figure}
	\label{fig:graficoGoogleTrendsFerramentasHibridasMobile}
	\includegraphics[width=1\textwidth]{img/grafico_ferramentas_hibridas_mobile}
	\caption{Gráfico do Google Trends exibindo as pesquisas por Frameworks híbridos mobile}
\end{figure}	

Para layout da interface mobile, também será aplicado a tendência do Material Design, a fim de propor uma harmonia entre o módulo web e mobile para os usuários


\section{Resultados Esperados}


Como fruto de um sistema para pós-graduação da UFBA, espera-se que os professores tenham mais recursos para integrar as atividades e também prover melhores condições para acompanhamento da vida acadêmica dos alunos em questão. Também, que os novos colaboradores que entrarem no processo tenham facilidade de compreender o fluxo do setor ao navegar pelo sistema proposto.


\section{Fora de Escopo}


Interação com os alunos devido às complicações para realizar a integração com o sistema empregado na UFBA, gerenciado pela XXXXXX, o que causaria uma inviabilidade no projeto devido à necessidade de entrega do produto ser mais forte que o tempo necessário para executar o processo de obtenção de acesso ao sistema legado para realizar a integração.


\section{Estrutura do Trabalho}


<breve resumo sobre os capítulos do TCC>
\chapter{Referencial Teórico}


Projetar o desenvolvimento de um software requer muito planejamento, pois as falhas iniciais podem custar bastante caro ou até mesmo inviabilizar a continuação de um projeto. Assim, a escolha da arquitetura ideal para a aplicabilidade é essencial na concepção de um produto de software. 
De todo o modo, sempre busca-se fazer mais com menos. Diante de tal filosofia, temos neste capítulo, uma breve discussão sobre alguns elementos de projeto e arquitetura de software, a fim de contextualizar este trabalho de conclusão de curso.
O capítulo corrente é composto por quatro seções. A \ref{sec:saas} trata de Software como serviço, discutindo alguns elementos do contexto que são relevantes para o trabalho proposto. A \ref{sec:reuso}  discute sobre a empregabilidade do reuso de software. A \ref{sec:modularizacao} seção, trata sobre aspectos envolvidos na modularização dos softwares. Por fim, a \ref{sec:apps_web} aborda as aplicações web, discutindo sobre aspectos relevantes sobre a aplicação web que compõe este trabalho.


 \section{Software como serviço}\label{sec:saas}


Segundo La e Chun \citep{La2009Systematic}, o princípio da definição de Software como um Serviço (Sofware as a Service - SaaS) é um serviço complementar para aplicações da computação em nuvem (cloud computing). No entanto, as áreas não se confundem. SaaS deve ser entendido como um mecanismo de suporte às soluções existentes na cloud. Os SaaS existem justamente para maximizar o reuso de serviços repetidos e não centrais em uma aplicação remota.


Como vantagens, diversos fatores podem ser favoráveis para a adoção de um SaaS, como custo e manutenção dentre outros fatores aplicáveis a soluções específicas. Lechesaet al. \citep{LechesaSS11}, quantificam índices sobre os fatores determinantes para adoção ou não de um SaaS voltado para ERP na África do Sul. Os principais fatores determinantes para adoção desse mecanismo de software são sua fluidez quanto à rede e a segurança. Esses fatores estão presentes na aplicação desenvolvida neste trabalho de conclusão de curso.


Devido ao fato de ter um serviço constantemente na nuvem, fica o questionamento sobre a segurança da informação manipulada. Sabe-se que a vulnerabilidade na Web não é restrita ao SaaS, atingindo diversos âmbitos. Rai et al. \cite{journals/corr/RaiSM13} consideram que o avanço da computação em nuvem não é um problema apenas para os serviços Web do ponto de vista da segurança, pois muitos trabalhos na literatura mostram a área como mais um ponto de vulnerabilidade para diversos setores, a exemplo de infraestrutura. No mesmo artigo mencionado de Rai et al. \cite{journals/corr/RaiSM13}, também realizaram-se estudos exploratórios junto a empresas usuárias de serviços em computação em nuvem e consideram que a perspectiva de SaaS também pode fortalecer a segurança nas aplicações de cloud computing, pois o software de autenticação compartilhado por várias aplicações em nuvem, oferece uma melhor padronização e consequente facilidade de prevenção a erros de vulnerabilidade específicas de cada módulo da pesquisa. Esse ponto de vista é fundamenta para a compreensão de SaaS.


A arquitetura de armazenamento de dados de um Saas pode variar de acordo com a necessidade do contexto. Huixin \cite{7586486} descreve possíveis modelagens nesse sentido. Essa abordagem pode ser com um banco de dados único, fazendo com que diferentes clientes compartilhem o mesmo banco, diferindo os dados através de controle de usuário, ou isolando os diferentes clientes através de bancos de dados exclusivos para cada um. Esse fator também pode ser combinado com a arquitetura da aplicação, caso ofereça aplicação única para todos os clientes ou aplicação compartilhada. Diante das possíveis abordagens, a modelagem de dados do software pode ser decidida pela regra de negócio. Este trabalho optou por aplicação única e banco de dados compartilhado.

Devido ao conceito particular de obtenção de software, tanto pela visão do cliente como pela visão do vendedor, é necessário compreender os diversos aspectos que podem ser relevantes ao orçar um Saas. O recente trabalho de T. Kaur et al. \citep{6949281} orienta um modelo para compor o custo de um Saas. O custo total seria composto pelos fatores que dão suporte ao funcionamento do software. Tais fatores incluem infra-estrutura, configurabilidade, customização, parâmetros de QoS (Quality of service) como escalabilidade, disponibilidade, usabilidade, pontualidade e desempenho da resposta, portabilidade, custo total de propriedade e retorno do investimento. Esses fatores caracterizam o custo de forma eficaz, possibilitando ao fornecedor, prover um Serviço de acordo com a exigência do consumidor em vários pacotes de serviços.


O conceito de software como serviço se aplica neste trabalho de conclusão de curso, pois o mesmo estará disponível na web com alta disponibilidade adotando as características apresentadas para qualquer pessoa que desejar usá-lo. Assim, basta se cadastrar, configurá-lo e usar como um serviço, sem preocupação com a manutenção do mesmo.


\section{Reuso de software}\label{sec:reuso} %CRUISE BOOK CAPITULO 2


De acordo com o livro Software pratical reuse \citep{Ezran2002}, o reuso de software é a utilização de qualquer informação que um desenvolvedor pode necessitar no processo de criação de software. O livro de Basili e Rombach \citep{Basili1991} define reutilização de software como o uso de tudo o que está associado a um projeto de conhecimento.
Assim, o objetivo da reutilização de software é reciclar o design, código e outros componentes de um produto de software e assim reduzir o custo, o tempo e melhorar a qualidade do produto.
Segundo Keswani et al. \cite{6783445}, o componente reutilizável de software pode ser qualquer parte de seu desenvolvimento, como um fragmento de código, design, casos de teste, ou até mesmo a especificação de requisitos de uma funcionalidade do software. 

O reuso de software pode ter impacto positivo em diversos aspectos do software, vejamos alguns, conforme apresentados no C.R.U.I.S.E Book \citep{cruise2007} :

\begin{itemize}

\item Qualidade: As correções de erro tornam-se úteis em todos os locais em que ocorreu, padronizando e facilitando a manutenção.

\item Produtividade: O ganho de produtividade é alcançado devido ao menor número de artefatos desenvolvido. Isso resulta em menor esforço de teste e também análise e design, reduzindo custos.

\item Confiabilidade: A utilização de componentes bem testados aumenta a
confiança no software. Além disso, a utilização de um mesmo componente em vários sistemas, aumenta a possibilidade de detecção de erros e reforça a confiança no componente.

\item Redução do Esforço: A reutilização de software proporciona uma redução do tempo de desenvolvimento, o que reduz o tempo necessário para o produto ser disponibilizado no mercado para trazer rentabilidade.

\item Trabalho redundante e tempo de desenvolvimento: Desenvolver um sistema do
zero significa desenvolvimento redundante de muitos componentes, como requisitos, especificações, casos de uso, arquitetura, etc. Isso pode ser evitado quando estes estão disponíveis como componentes reutilizáveis e podem ser compartilhados, resultando em um processo de desenvolvimento otimizado.

\item Documentação: Embora a documentação seja muito importante para a
manutenção de um sistema, muitas vezes é negligenciada. A reutilização de componentes de software reduz a quantidade de documentação a ser escrita, entretanto depende da qualidade do que está escrito. Assim, apenas a estrutura do sistema e os novos artefatos desenvolvidos necessitam ser documentados.

\item Custo de manutenção: Menos defeitos e manutenções são esperados quando tem-se comprovada a qualidade dos componentes utilizados.

\item Tamanho da equipe: É comum haver casos em que a equipe de desenvolvimento sofre sobrecarga. Entretanto, dobrar o tamanho da equipe de desenvolvimento não necessariamente duplica produtividade. Se muitos componentes podem ser reutilizados, é possível desenvolver com equipes menores, levando a melhor comunicação e aumento da produtividade.

\end{itemize}

Apesar dos benefícios da reutilização de software, ela não é suficientemente aproveitada. Existem fatores que influenciam direta ou indiretamente na sua adoção. Esses fatores podem ser de aspecto gerencial, organizacional, econômico, conceitual ou técnico. Veremos a seguir alguns aspectos que podem gerar conflito com a cultura de reuso de software, segundo o C.R.U.I.S.E Book \citep{cruise2007} :
%(Sametinger, 1997). REVER

\begin{itemize}
	
\item Falta de apoio da gestão: Como a reutilização de software gera custos iniciais,
a medida pode não ser amplamente alcançada em uma organização sem o apoio de alto nível de gestão. Os gestores têm de ser informados sobre os custos iniciais e serem convencidos sobre economias futuras.

\item Gerenciamento do Projeto: Gerenciar projetos tradicionais é uma tarefa árdua, principalmente, os que praticam a reutilização de software. Utilizando a técnica em larga escala, tem-se impacto sobre todo o ciclo de vida do software.

\item Estruturas organizacionais inadequadas: As estruturas organizacionais devem
considerar diferentes necessidades que surgem quando a reutilização em larga escala está sendo adotada. Por exemplo, uma equipe particionada pode ser alocada somente para desenvolver, manter e certificar componentes reutilizáveis de software.

\item Incentivos de gestão: É comum a falta de incentivo para deixar os desenvolvedores gastarem tempo elaborando componentes do sistemas. A produtividade é muitas vezes medida apenas no tempo necessário para concluir um projeto. Assim, fazer qualquer trabalho além disso, embora benéfico para a empresa como um todo, diminui o seu sucesso. Mesmo quando os componentes reutilizáveis são utilizados, os benefícios obtidos são uma pequena fração do que poderia ser alcançado caso houvesse reutilização explícita, planejada e organizada.

\item Dificuldade de encontrar software reutilizável: Para reutilizar os componentes, devem existir formas eficientes de busca. Além disso, é importante ter um repositório bem organizado contendo componentes com um eficiente meio de acesso.

\item Não reutilização do software encontrado. O acesso fácil ao software existente
não necessariamente aumentar a reutilização. Os componentes reutilizáveis devem ser cuidadosamente especificados, projetados, implementados e documentados, pois em alguns casos, modificar e adaptar o código  pode ser mais custoso que a programação da funcionalidade necessária a partir do zero.

\item Modificação: É muito difícil encontrar um componente que funcione
exatamente da mesma maneira que queremos. Desta forma, são necessárias modificações e devem existir formas de determinar os seus efeitos sobre o componente.


\end{itemize}


A definição do reuso de software pode ser associada a este trabalho, pois ainda que como um serviço, estaremos adotando uma aplicação única para servir aos mais diversos softwares de uma empresa, por exemplo. Logo, ao adotar o projeto, pode ser interessante aplicar algumas métricas para tomar conhecimento de possíveis vantanjens como as citadas nessa seção, a exemplo de economia de tempo de desenvolvimento, e custo do projeto.


%Outra diretriz importante para a reutilização de software é reduzir o risco na criação de novos softwares. O risco tende a ser bastante reduzido se os componentes que estão sendo reutilizados têm as documentação, interfaces necessárias e devidamente testadas, fatores que contibruem para uma fácil integração.
%De acordo com Keswani et al. \citep{6783445}, para o reuso de software dar retornos apropriados, o processo deve ser sistemático e planejado. Qualquer organização que implemente a reutilização de software deve identificar os melhores métodos e estratégias de reutilização para obter a máxima produtividade. A reutilização de software ajuda a evitar software de engenharia a partir do zero, pois usa módulos de software existentes. A reutilização de software, embora seja uma tarefa difícil, especialmente para softwares antigos sem padrões de projeto, pode melhorar significativamente a produtividade e a qualidade de um produto de software. Embora a reutilização de software não seja um novo campo, ela pode dar grandes retornos em curto período de tempo.


\section{Modularização}\label{sec:modularizacao} %artigo de claudio pagina 222 introdução


%A modularidade vem desempenhando um papel predominante estágios emergentes das disciplinas de arquitetura de software [13]. Engenheiros de software consideram modularidade como princípio base na comparação entre arquiteturas alternativas  e arquitetura degeneração [9]. De fato, os engenheiros de software são incentivados a arquitecturas, baseando-se numa multiplicidade de mecanismos de modularidade disponíveis em: 
%(i) Linguagens de descrição de arquitetura (ADLs), como ACME [8], 
%(ii) catálogos de arquitetônicos [2, 13], e 
%(iii) conhecem bem princípios de alto nível, como interfaces de componentes estreitos, acoplamento arquitectónico reduzido e semelhantes.


Conforme é frisado por Wickramaarachchi e Lai \citep{7062705}, o conceito de modularização na indústria de software tem uma longa história e tem sido utilizado para melhorar o processo de desenvolvimento de software em diferentes estágios. Os principais conceitos por trás da modularização do software foram introduzidos por pesquisadores pioneiros há quarenta anos, com uma notável contribuição feita por Melvin Conway e David Parnas, que tem representação notável na engenharia de software.


Modularizar um software é um bom padrão a ser adotado. Segundo Wickramaarachchi e Lai \citep{7062705}, a modularização é importante na identificação de dependências e reduz as dificuldades diante de uma possível necessidade de grandes alterações. De uma perspectiva da engenharia de software, uma modularização geralmente tem várias vantagens, tais como: tornar a complexidade do software mais gerenciável, facilitar o trabalho paralelo e tornar o software mais maleável para acomodar o futuro incerto que um software pode ter. O objetivo final da modularização do software é aumentar a produtividade ea qualidade do software. Tal conceito encontra-se bastante difundido e estái incorporado em linguagens de programação e ferramentas de software. O presente trabalho favorece ao uso da modularização de um software e até mesmo pode ser considerado um módulo a ser acoplado a qualquer software, mediante a compatibilidade.


Ao realizar a adoção do trabalho proposto, fica bastante evidente a "responsabilidade" do mesmo no escopo do projeto. Assim, é possível tratá-lo como um módulo do projeto que o usa, mesmo que seja consumido como um serviço.


\section{Aplicações web}\label{sec:apps_web}


A popularidade das soluções Web aumentou exponencialmente na última década e todos os dias cresce o número de pessoas usuárias desse tipo de software. E seguindo um padrão próprio, Kumar et al. \citep{7813710} sugerem que para o desenvolvimento web, deve-se manter a prática eficaz de produzir diagramas UML. A abordagem baseada na web oferece uma maneira fácil e eficaz para gerenciar e controlar o processo de desenvolvimento por meio de artefatos de modelagem. Tal abordagem pode ser usada quando há uma exigência de lidar com mudanças muito rápidas e grandes em requisitos de forma muito eficaz em muito menos tempo, gerando assim um menor impacto. 


Para atender à fomentada demanda de aplicativos web, é necessário adotar métodos de desenvolvimentos que sejam ágeis, eficientes e de fácil manutenção. Yu Ping et al. \cite{1372143} propõem o uso do modelo MVC (Model, View e Controller) no desenvolvimento para softwares web. O modelo apresentado tornou-se um padrão popular e divide o software em camadas com propósito definido, tornando-o de mais fácil manutenção.


O Ajax (Asynchronous Javascript and XML) revolucionou a web. Conforme demonstrado por Yuping \citep{6845605}, ao usar a tecnologia Ajax, podemos enriquecer a experiência do usuário em aplicações baseadas em navegador de internet, e fornecer uma variedade de aplicações interativas para atender às necessidade de humanização das aplicações.
Os aplicativos Ajax em execução no navegador se comunicam com um servidor Web de forma assíncrona e atualizam apenas uma parte da página.


No artigo de Tesarik et al. \citep{4664349}, temos que o design de software SPA (Single page application) é uma maneira interessante de criar um software Web numa uma única página. Essa solução de página única sem navegação funciona apenas com base em técnicas dinâmicas e assíncronas, como o AJAX (citado no parágrafo anterior). No entanto, esta abordagem coloca o desenvolvedor antes de alguns desafios substanciais.
Para projetar a interface do usuário que mostra as informações, é importante projetar corretamente a tela para manipular os dados do aplicativo numa única página. O design da página deve ser elaborado para maximizar a decomposição da página em componentes distintos que encapsulam os principais casos de uso.
Também recomenda-se que para elaborar os artefatos visuais, explore os mais ricos recursos na implementação, como HTML5, JavaScript, Ajax, CSS3, e outras tecnologias que se apliquem. Entretanto, desenvolver uma rica interface com o uso de diversas tecnologias/frameworks pode ocasionar um esforço maior para explorar as possibilidades.
Tais componentes da interface de uma aplicação SPA normalmente são alimentados mediante o consumo de uma API Rest via requisições AJAX. Assim, é possível particionar as responsabilidades de processamento do software entre cliente e servidor.


Como é possível denotar no trabalho de Salvadori e Siqueire \citep{7195633}, REST é uma arquitetura muito popular para integração de aplicativos web, que permite compartilhar e reutilizar informações através de sistemas. Aplicações de grande escala baseadas em no REST deve ser implementadas utilizando estratégias e mecanismos para produzir sistemas que sejam fáceis de desenvolver, reutilizar e manter. As interfaces de integração fornecidas por esses sistemas, chamadas de Web API, têm uma influência importante nas características da implementação resultante, pois a sua forma de resposta deve ser casada com a interface que o consome.
 
 
Os elementos comentados nessa seção encontram-se presentes neste trabalho de conclusão de curso. Juntos, montam a estrutura tecnológica necessária aliada à arquitetura adotada. Esses elementos seguem tendências atuais dos softwares web, fazendo com que esse trabalho esteja composto por tecnologias modernas que estão com boa aceitação no mercado.

% %% RiSE Latex Template - version 0.5
%%
%% RiSE's latex template for thesis and dissertations
%% http://risetemplate.sourceforge.net
%%
%% (c) 2012 Yguaratã Cerqueira Cavalcanti (yguarata@gmail.com)
%%          Vinicius Cardoso Garcia (vinicius.garcia@gmail.com)
%%
%% This document was initially based on UFPEThesis template, from Paulo Gustavo
%% S. Fonseca.
%%
%% ACKNOWLEDGEMENTS
%%
%% We would like to thanks the RiSE's researchers community, the 
%% students from Federal University of Pernambuco, and other users that have
%% been contributing to this projects with comments and patches.
%%
%% GENERAL INSTRUCTIONS
%%
%% We strongly recommend you to compile your documents using pdflatex command.
%% It is also recommend use the texlipse plugin for Eclipse to edit your documents.
%%
%% Options for \documentclass command:
%%         * Idiom
%%           pt   - Portguese (default)
%%           en   - English
%%
%%         * Text type
%%           bsc  - B.Sc. Thesis
%%           msc  - M.Sc. Thesis (default)
%%           qual - PHD qualification (not tested yet)
%%           prop - PHD proposal (not tested yet)
%%           phd  - PHD thesis
%%
%%         * Media
%%           scr  - to eletronic version (PDF) / see the users guide
%%
%%         * Pagination
%%           oneside - unique face press
%%           twoside - two faces press
%%
%%		   * Line spacing
%%           singlespacing  - the same as using \linespread{1}
%%           onehalfspacing - the same as using \linespread{1.3}
%%           doublespacing  - the same as using \linespread{1.6}
%%
%% Reference commands. Use the following commands to make references in your
%% text:
%%          \figref  -- for Figure reference
%%          \tabref  -- for Table reference
%%          \eqnref  -- for equation reference
%%          \chapref -- for chapter reference
%%          \secref  -- for section reference
%%          \appref  -- for appendix reference
%%          \axiref  -- for axiom reference
%%          \conjref -- for conjecture reference
%%          \defref  -- for definition reference
%%          \lemref  -- for lemma reference
%%          \theoref -- for theorem reference
%%          \corref  -- for corollary reference
%%          \propref -- for proprosition reference
%%          \pgref   -- for page reference
%%
%%          Example: See \chapref{chap:introduction}. It will produce 
%%                   'See Chapter 1', in case of English language.

\documentclass[pt,twoside,onehalfspacing,bsc]{risethesis}

\usepackage[sort,square]{natbib} %package for bibliography
\bibliographystyle{ieeetr} %authordate1

\usepackage[utf8]{inputenc}
\usepackage[brazilian]{babel}
\usepackage[T1]{fontenc}

\usepackage{graphicx}

%% Change the following pdf author attribute name to your name.
\usepackage[linkcolor=blue,citecolor=blue,urlcolor=blue,colorlinks,pdfpagelabels,pdftitle={Bruno Cabral's Bachelor Thesis},pdfauthor={Bruno Cabral}]{hyperref}

\address{SALVADOR}

\universitypt{Universidade Federal da Bahia}
\universityen{Federal University of Bahia}

\departmentpt{Depertamento de Ciência da Computação}
\departmenten{Computer Science Department}

\programpt{Programa Multiinstitucional de Pós-graduação em Ciência da Computação}
\programen{Graduate in Computer Science}

\majorfieldpt{Ciência da Computação}
\majorfielden{Computer Science}

\title{Sistema de apoio à Pós graduação - UFBA}
\date{Outubro/2016}

\author{Victor de Azevedo Nunes}
\adviser{Ivan do Carmo Machado}

\begin{document}

\frontmatter
\frontpage
\presentationpage

\begin{dedicatory}
Eu dedico esta dissertação...
%I dedicate this dissertation to my family, girlfriend, friends and
%professors who gave me all necessary support to get here.
\end{dedicatory}

\acknowledgements
Meus agradecimentos...

\begin{epigraph}[]{Edward V Berard}
Walking on water and developing software from a specification are easy if both are frozen
\end{epigraph}

\resumo
% Escreva seu resumo no arquivo resumo.tex
\input{resumo}

\abstract
% Write your abstract in a file called abstract.tex
\input{abstract}

% Summary (tables of contents)
\tableofcontents

% List of figures
\listoffigures

% List of tables
\listoftables

% List of acronyms
% Acronyms manual: http://linorg.usp.br/CTAN/macros/latex/contrib/acronym/acronym.pdf
\listofacronyms
\input{acronyms}

% List of listings
%\lstlistoflistings

\mainmatter

\include{chapters/intro}
\include{chapters/referencial_teorico}

% \include{chapters/introduction/main}
% \include{chapters/background/main}
% \include{chapters/proposed_solution/main}
% \include{chapters/experiment/main}
% \include{chapters/conclusion/main}

%\bibliographystyle{natbib}
%\addcontentsline{toc}{chapter}{\bibliographytocname}
\bibliography{references}

% Appendix
\clearpage
\addappheadtotoc
\appendix
\appendixpage
% \include{appendix/experiment-instruments}

\end{document}
% %% RiSE Latex Template - version 0.5
%%
%% RiSE's latex template for thesis and dissertations
%% http://risetemplate.sourceforge.net
%%
%% (c) 2012 Yguaratã Cerqueira Cavalcanti (yguarata@gmail.com)
%%          Vinicius Cardoso Garcia (vinicius.garcia@gmail.com)
%%
%% This document was initially based on UFPEThesis template, from Paulo Gustavo
%% S. Fonseca.
%%
%% ACKNOWLEDGEMENTS
%%
%% We would like to thanks the RiSE's researchers community, the 
%% students from Federal University of Pernambuco, and other users that have
%% been contributing to this projects with comments and patches.
%%
%% GENERAL INSTRUCTIONS
%%
%% We strongly recommend you to compile your documents using pdflatex command.
%% It is also recommend use the texlipse plugin for Eclipse to edit your documents.
%%
%% Options for \documentclass command:
%%         * Idiom
%%           pt   - Portguese (default)
%%           en   - English
%%
%%         * Text type
%%           bsc  - B.Sc. Thesis
%%           msc  - M.Sc. Thesis (default)
%%           qual - PHD qualification (not tested yet)
%%           prop - PHD proposal (not tested yet)
%%           phd  - PHD thesis
%%
%%         * Media
%%           scr  - to eletronic version (PDF) / see the users guide
%%
%%         * Pagination
%%           oneside - unique face press
%%           twoside - two faces press
%%
%%		   * Line spacing
%%           singlespacing  - the same as using \linespread{1}
%%           onehalfspacing - the same as using \linespread{1.3}
%%           doublespacing  - the same as using \linespread{1.6}
%%
%% Reference commands. Use the following commands to make references in your
%% text:
%%          \figref  -- for Figure reference
%%          \tabref  -- for Table reference
%%          \eqnref  -- for equation reference
%%          \chapref -- for chapter reference
%%          \secref  -- for section reference
%%          \appref  -- for appendix reference
%%          \axiref  -- for axiom reference
%%          \conjref -- for conjecture reference
%%          \defref  -- for definition reference
%%          \lemref  -- for lemma reference
%%          \theoref -- for theorem reference
%%          \corref  -- for corollary reference
%%          \propref -- for proprosition reference
%%          \pgref   -- for page reference
%%
%%          Example: See \chapref{chap:introduction}. It will produce 
%%                   'See Chapter 1', in case of English language.

\documentclass[pt,twoside,onehalfspacing,bsc]{risethesis}

\usepackage[sort,square]{natbib} %package for bibliography
\bibliographystyle{ieeetr} %authordate1

\usepackage[utf8]{inputenc}
\usepackage[brazilian]{babel}
\usepackage[T1]{fontenc}

\usepackage{graphicx}

%% Change the following pdf author attribute name to your name.
\usepackage[linkcolor=blue,citecolor=blue,urlcolor=blue,colorlinks,pdfpagelabels,pdftitle={Bruno Cabral's Bachelor Thesis},pdfauthor={Bruno Cabral}]{hyperref}

\address{SALVADOR}

\universitypt{Universidade Federal da Bahia}
\universityen{Federal University of Bahia}

\departmentpt{Depertamento de Ciência da Computação}
\departmenten{Computer Science Department}

\programpt{Programa Multiinstitucional de Pós-graduação em Ciência da Computação}
\programen{Graduate in Computer Science}

\majorfieldpt{Ciência da Computação}
\majorfielden{Computer Science}

\title{Sistema de apoio à Pós graduação - UFBA}
\date{Outubro/2016}

\author{Victor de Azevedo Nunes}
\adviser{Ivan do Carmo Machado}

\begin{document}

\frontmatter
\frontpage
\presentationpage

\begin{dedicatory}
Eu dedico esta dissertação...
%I dedicate this dissertation to my family, girlfriend, friends and
%professors who gave me all necessary support to get here.
\end{dedicatory}

\acknowledgements
Meus agradecimentos...

\begin{epigraph}[]{Edward V Berard}
Walking on water and developing software from a specification are easy if both are frozen
\end{epigraph}

\resumo
% Escreva seu resumo no arquivo resumo.tex
\input{resumo}

\abstract
% Write your abstract in a file called abstract.tex
\input{abstract}

% Summary (tables of contents)
\tableofcontents

% List of figures
\listoffigures

% List of tables
\listoftables

% List of acronyms
% Acronyms manual: http://linorg.usp.br/CTAN/macros/latex/contrib/acronym/acronym.pdf
\listofacronyms
\input{acronyms}

% List of listings
%\lstlistoflistings

\mainmatter

\include{chapters/intro}
\include{chapters/referencial_teorico}

% \include{chapters/introduction/main}
% \include{chapters/background/main}
% \include{chapters/proposed_solution/main}
% \include{chapters/experiment/main}
% \include{chapters/conclusion/main}

%\bibliographystyle{natbib}
%\addcontentsline{toc}{chapter}{\bibliographytocname}
\bibliography{references}

% Appendix
\clearpage
\addappheadtotoc
\appendix
\appendixpage
% \include{appendix/experiment-instruments}

\end{document}
% %% RiSE Latex Template - version 0.5
%%
%% RiSE's latex template for thesis and dissertations
%% http://risetemplate.sourceforge.net
%%
%% (c) 2012 Yguaratã Cerqueira Cavalcanti (yguarata@gmail.com)
%%          Vinicius Cardoso Garcia (vinicius.garcia@gmail.com)
%%
%% This document was initially based on UFPEThesis template, from Paulo Gustavo
%% S. Fonseca.
%%
%% ACKNOWLEDGEMENTS
%%
%% We would like to thanks the RiSE's researchers community, the 
%% students from Federal University of Pernambuco, and other users that have
%% been contributing to this projects with comments and patches.
%%
%% GENERAL INSTRUCTIONS
%%
%% We strongly recommend you to compile your documents using pdflatex command.
%% It is also recommend use the texlipse plugin for Eclipse to edit your documents.
%%
%% Options for \documentclass command:
%%         * Idiom
%%           pt   - Portguese (default)
%%           en   - English
%%
%%         * Text type
%%           bsc  - B.Sc. Thesis
%%           msc  - M.Sc. Thesis (default)
%%           qual - PHD qualification (not tested yet)
%%           prop - PHD proposal (not tested yet)
%%           phd  - PHD thesis
%%
%%         * Media
%%           scr  - to eletronic version (PDF) / see the users guide
%%
%%         * Pagination
%%           oneside - unique face press
%%           twoside - two faces press
%%
%%		   * Line spacing
%%           singlespacing  - the same as using \linespread{1}
%%           onehalfspacing - the same as using \linespread{1.3}
%%           doublespacing  - the same as using \linespread{1.6}
%%
%% Reference commands. Use the following commands to make references in your
%% text:
%%          \figref  -- for Figure reference
%%          \tabref  -- for Table reference
%%          \eqnref  -- for equation reference
%%          \chapref -- for chapter reference
%%          \secref  -- for section reference
%%          \appref  -- for appendix reference
%%          \axiref  -- for axiom reference
%%          \conjref -- for conjecture reference
%%          \defref  -- for definition reference
%%          \lemref  -- for lemma reference
%%          \theoref -- for theorem reference
%%          \corref  -- for corollary reference
%%          \propref -- for proprosition reference
%%          \pgref   -- for page reference
%%
%%          Example: See \chapref{chap:introduction}. It will produce 
%%                   'See Chapter 1', in case of English language.

\documentclass[pt,twoside,onehalfspacing,bsc]{risethesis}

\usepackage[sort,square]{natbib} %package for bibliography
\bibliographystyle{ieeetr} %authordate1

\usepackage[utf8]{inputenc}
\usepackage[brazilian]{babel}
\usepackage[T1]{fontenc}

\usepackage{graphicx}

%% Change the following pdf author attribute name to your name.
\usepackage[linkcolor=blue,citecolor=blue,urlcolor=blue,colorlinks,pdfpagelabels,pdftitle={Bruno Cabral's Bachelor Thesis},pdfauthor={Bruno Cabral}]{hyperref}

\address{SALVADOR}

\universitypt{Universidade Federal da Bahia}
\universityen{Federal University of Bahia}

\departmentpt{Depertamento de Ciência da Computação}
\departmenten{Computer Science Department}

\programpt{Programa Multiinstitucional de Pós-graduação em Ciência da Computação}
\programen{Graduate in Computer Science}

\majorfieldpt{Ciência da Computação}
\majorfielden{Computer Science}

\title{Sistema de apoio à Pós graduação - UFBA}
\date{Outubro/2016}

\author{Victor de Azevedo Nunes}
\adviser{Ivan do Carmo Machado}

\begin{document}

\frontmatter
\frontpage
\presentationpage

\begin{dedicatory}
Eu dedico esta dissertação...
%I dedicate this dissertation to my family, girlfriend, friends and
%professors who gave me all necessary support to get here.
\end{dedicatory}

\acknowledgements
Meus agradecimentos...

\begin{epigraph}[]{Edward V Berard}
Walking on water and developing software from a specification are easy if both are frozen
\end{epigraph}

\resumo
% Escreva seu resumo no arquivo resumo.tex
\input{resumo}

\abstract
% Write your abstract in a file called abstract.tex
\input{abstract}

% Summary (tables of contents)
\tableofcontents

% List of figures
\listoffigures

% List of tables
\listoftables

% List of acronyms
% Acronyms manual: http://linorg.usp.br/CTAN/macros/latex/contrib/acronym/acronym.pdf
\listofacronyms
\input{acronyms}

% List of listings
%\lstlistoflistings

\mainmatter

\include{chapters/intro}
\include{chapters/referencial_teorico}

% \include{chapters/introduction/main}
% \include{chapters/background/main}
% \include{chapters/proposed_solution/main}
% \include{chapters/experiment/main}
% \include{chapters/conclusion/main}

%\bibliographystyle{natbib}
%\addcontentsline{toc}{chapter}{\bibliographytocname}
\bibliography{references}

% Appendix
\clearpage
\addappheadtotoc
\appendix
\appendixpage
% \include{appendix/experiment-instruments}

\end{document}
% %% RiSE Latex Template - version 0.5
%%
%% RiSE's latex template for thesis and dissertations
%% http://risetemplate.sourceforge.net
%%
%% (c) 2012 Yguaratã Cerqueira Cavalcanti (yguarata@gmail.com)
%%          Vinicius Cardoso Garcia (vinicius.garcia@gmail.com)
%%
%% This document was initially based on UFPEThesis template, from Paulo Gustavo
%% S. Fonseca.
%%
%% ACKNOWLEDGEMENTS
%%
%% We would like to thanks the RiSE's researchers community, the 
%% students from Federal University of Pernambuco, and other users that have
%% been contributing to this projects with comments and patches.
%%
%% GENERAL INSTRUCTIONS
%%
%% We strongly recommend you to compile your documents using pdflatex command.
%% It is also recommend use the texlipse plugin for Eclipse to edit your documents.
%%
%% Options for \documentclass command:
%%         * Idiom
%%           pt   - Portguese (default)
%%           en   - English
%%
%%         * Text type
%%           bsc  - B.Sc. Thesis
%%           msc  - M.Sc. Thesis (default)
%%           qual - PHD qualification (not tested yet)
%%           prop - PHD proposal (not tested yet)
%%           phd  - PHD thesis
%%
%%         * Media
%%           scr  - to eletronic version (PDF) / see the users guide
%%
%%         * Pagination
%%           oneside - unique face press
%%           twoside - two faces press
%%
%%		   * Line spacing
%%           singlespacing  - the same as using \linespread{1}
%%           onehalfspacing - the same as using \linespread{1.3}
%%           doublespacing  - the same as using \linespread{1.6}
%%
%% Reference commands. Use the following commands to make references in your
%% text:
%%          \figref  -- for Figure reference
%%          \tabref  -- for Table reference
%%          \eqnref  -- for equation reference
%%          \chapref -- for chapter reference
%%          \secref  -- for section reference
%%          \appref  -- for appendix reference
%%          \axiref  -- for axiom reference
%%          \conjref -- for conjecture reference
%%          \defref  -- for definition reference
%%          \lemref  -- for lemma reference
%%          \theoref -- for theorem reference
%%          \corref  -- for corollary reference
%%          \propref -- for proprosition reference
%%          \pgref   -- for page reference
%%
%%          Example: See \chapref{chap:introduction}. It will produce 
%%                   'See Chapter 1', in case of English language.

\documentclass[pt,twoside,onehalfspacing,bsc]{risethesis}

\usepackage[sort,square]{natbib} %package for bibliography
\bibliographystyle{ieeetr} %authordate1

\usepackage[utf8]{inputenc}
\usepackage[brazilian]{babel}
\usepackage[T1]{fontenc}

\usepackage{graphicx}

%% Change the following pdf author attribute name to your name.
\usepackage[linkcolor=blue,citecolor=blue,urlcolor=blue,colorlinks,pdfpagelabels,pdftitle={Bruno Cabral's Bachelor Thesis},pdfauthor={Bruno Cabral}]{hyperref}

\address{SALVADOR}

\universitypt{Universidade Federal da Bahia}
\universityen{Federal University of Bahia}

\departmentpt{Depertamento de Ciência da Computação}
\departmenten{Computer Science Department}

\programpt{Programa Multiinstitucional de Pós-graduação em Ciência da Computação}
\programen{Graduate in Computer Science}

\majorfieldpt{Ciência da Computação}
\majorfielden{Computer Science}

\title{Sistema de apoio à Pós graduação - UFBA}
\date{Outubro/2016}

\author{Victor de Azevedo Nunes}
\adviser{Ivan do Carmo Machado}

\begin{document}

\frontmatter
\frontpage
\presentationpage

\begin{dedicatory}
Eu dedico esta dissertação...
%I dedicate this dissertation to my family, girlfriend, friends and
%professors who gave me all necessary support to get here.
\end{dedicatory}

\acknowledgements
Meus agradecimentos...

\begin{epigraph}[]{Edward V Berard}
Walking on water and developing software from a specification are easy if both are frozen
\end{epigraph}

\resumo
% Escreva seu resumo no arquivo resumo.tex
\input{resumo}

\abstract
% Write your abstract in a file called abstract.tex
\input{abstract}

% Summary (tables of contents)
\tableofcontents

% List of figures
\listoffigures

% List of tables
\listoftables

% List of acronyms
% Acronyms manual: http://linorg.usp.br/CTAN/macros/latex/contrib/acronym/acronym.pdf
\listofacronyms
\input{acronyms}

% List of listings
%\lstlistoflistings

\mainmatter

\include{chapters/intro}
\include{chapters/referencial_teorico}

% \include{chapters/introduction/main}
% \include{chapters/background/main}
% \include{chapters/proposed_solution/main}
% \include{chapters/experiment/main}
% \include{chapters/conclusion/main}

%\bibliographystyle{natbib}
%\addcontentsline{toc}{chapter}{\bibliographytocname}
\bibliography{references}

% Appendix
\clearpage
\addappheadtotoc
\appendix
\appendixpage
% \include{appendix/experiment-instruments}

\end{document}
% %% RiSE Latex Template - version 0.5
%%
%% RiSE's latex template for thesis and dissertations
%% http://risetemplate.sourceforge.net
%%
%% (c) 2012 Yguaratã Cerqueira Cavalcanti (yguarata@gmail.com)
%%          Vinicius Cardoso Garcia (vinicius.garcia@gmail.com)
%%
%% This document was initially based on UFPEThesis template, from Paulo Gustavo
%% S. Fonseca.
%%
%% ACKNOWLEDGEMENTS
%%
%% We would like to thanks the RiSE's researchers community, the 
%% students from Federal University of Pernambuco, and other users that have
%% been contributing to this projects with comments and patches.
%%
%% GENERAL INSTRUCTIONS
%%
%% We strongly recommend you to compile your documents using pdflatex command.
%% It is also recommend use the texlipse plugin for Eclipse to edit your documents.
%%
%% Options for \documentclass command:
%%         * Idiom
%%           pt   - Portguese (default)
%%           en   - English
%%
%%         * Text type
%%           bsc  - B.Sc. Thesis
%%           msc  - M.Sc. Thesis (default)
%%           qual - PHD qualification (not tested yet)
%%           prop - PHD proposal (not tested yet)
%%           phd  - PHD thesis
%%
%%         * Media
%%           scr  - to eletronic version (PDF) / see the users guide
%%
%%         * Pagination
%%           oneside - unique face press
%%           twoside - two faces press
%%
%%		   * Line spacing
%%           singlespacing  - the same as using \linespread{1}
%%           onehalfspacing - the same as using \linespread{1.3}
%%           doublespacing  - the same as using \linespread{1.6}
%%
%% Reference commands. Use the following commands to make references in your
%% text:
%%          \figref  -- for Figure reference
%%          \tabref  -- for Table reference
%%          \eqnref  -- for equation reference
%%          \chapref -- for chapter reference
%%          \secref  -- for section reference
%%          \appref  -- for appendix reference
%%          \axiref  -- for axiom reference
%%          \conjref -- for conjecture reference
%%          \defref  -- for definition reference
%%          \lemref  -- for lemma reference
%%          \theoref -- for theorem reference
%%          \corref  -- for corollary reference
%%          \propref -- for proprosition reference
%%          \pgref   -- for page reference
%%
%%          Example: See \chapref{chap:introduction}. It will produce 
%%                   'See Chapter 1', in case of English language.

\documentclass[pt,twoside,onehalfspacing,bsc]{risethesis}

\usepackage[sort,square]{natbib} %package for bibliography
\bibliographystyle{ieeetr} %authordate1

\usepackage[utf8]{inputenc}
\usepackage[brazilian]{babel}
\usepackage[T1]{fontenc}

\usepackage{graphicx}

%% Change the following pdf author attribute name to your name.
\usepackage[linkcolor=blue,citecolor=blue,urlcolor=blue,colorlinks,pdfpagelabels,pdftitle={Bruno Cabral's Bachelor Thesis},pdfauthor={Bruno Cabral}]{hyperref}

\address{SALVADOR}

\universitypt{Universidade Federal da Bahia}
\universityen{Federal University of Bahia}

\departmentpt{Depertamento de Ciência da Computação}
\departmenten{Computer Science Department}

\programpt{Programa Multiinstitucional de Pós-graduação em Ciência da Computação}
\programen{Graduate in Computer Science}

\majorfieldpt{Ciência da Computação}
\majorfielden{Computer Science}

\title{Sistema de apoio à Pós graduação - UFBA}
\date{Outubro/2016}

\author{Victor de Azevedo Nunes}
\adviser{Ivan do Carmo Machado}

\begin{document}

\frontmatter
\frontpage
\presentationpage

\begin{dedicatory}
Eu dedico esta dissertação...
%I dedicate this dissertation to my family, girlfriend, friends and
%professors who gave me all necessary support to get here.
\end{dedicatory}

\acknowledgements
Meus agradecimentos...

\begin{epigraph}[]{Edward V Berard}
Walking on water and developing software from a specification are easy if both are frozen
\end{epigraph}

\resumo
% Escreva seu resumo no arquivo resumo.tex
\input{resumo}

\abstract
% Write your abstract in a file called abstract.tex
\input{abstract}

% Summary (tables of contents)
\tableofcontents

% List of figures
\listoffigures

% List of tables
\listoftables

% List of acronyms
% Acronyms manual: http://linorg.usp.br/CTAN/macros/latex/contrib/acronym/acronym.pdf
\listofacronyms
\input{acronyms}

% List of listings
%\lstlistoflistings

\mainmatter

\include{chapters/intro}
\include{chapters/referencial_teorico}

% \include{chapters/introduction/main}
% \include{chapters/background/main}
% \include{chapters/proposed_solution/main}
% \include{chapters/experiment/main}
% \include{chapters/conclusion/main}

%\bibliographystyle{natbib}
%\addcontentsline{toc}{chapter}{\bibliographytocname}
\bibliography{references}

% Appendix
\clearpage
\addappheadtotoc
\appendix
\appendixpage
% \include{appendix/experiment-instruments}

\end{document}

%\bibliographystyle{natbib}
%\addcontentsline{toc}{chapter}{\bibliographytocname}
\bibliography{references}

% Appendix
\clearpage
\addappheadtotoc
\appendix
\appendixpage
% \include{appendix/experiment-instruments}

\end{document}

%\bibliographystyle{natbib}
%\addcontentsline{toc}{chapter}{\bibliographytocname}
\bibliography{references}

% Appendix
\clearpage
\addappheadtotoc
\appendix
\appendixpage
% \include{appendix/experiment-instruments}

\end{document}
% %% RiSE Latex Template - version 0.5
%%
%% RiSE's latex template for thesis and dissertations
%% http://risetemplate.sourceforge.net
%%
%% (c) 2012 Yguaratã Cerqueira Cavalcanti (yguarata@gmail.com)
%%          Vinicius Cardoso Garcia (vinicius.garcia@gmail.com)
%%
%% This document was initially based on UFPEThesis template, from Paulo Gustavo
%% S. Fonseca.
%%
%% ACKNOWLEDGEMENTS
%%
%% We would like to thanks the RiSE's researchers community, the 
%% students from Federal University of Pernambuco, and other users that have
%% been contributing to this projects with comments and patches.
%%
%% GENERAL INSTRUCTIONS
%%
%% We strongly recommend you to compile your documents using pdflatex command.
%% It is also recommend use the texlipse plugin for Eclipse to edit your documents.
%%
%% Options for \documentclass command:
%%         * Idiom
%%           pt   - Portguese (default)
%%           en   - English
%%
%%         * Text type
%%           bsc  - B.Sc. Thesis
%%           msc  - M.Sc. Thesis (default)
%%           qual - PHD qualification (not tested yet)
%%           prop - PHD proposal (not tested yet)
%%           phd  - PHD thesis
%%
%%         * Media
%%           scr  - to eletronic version (PDF) / see the users guide
%%
%%         * Pagination
%%           oneside - unique face press
%%           twoside - two faces press
%%
%%		   * Line spacing
%%           singlespacing  - the same as using \linespread{1}
%%           onehalfspacing - the same as using \linespread{1.3}
%%           doublespacing  - the same as using \linespread{1.6}
%%
%% Reference commands. Use the following commands to make references in your
%% text:
%%          \figref  -- for Figure reference
%%          \tabref  -- for Table reference
%%          \eqnref  -- for equation reference
%%          \chapref -- for chapter reference
%%          \secref  -- for section reference
%%          \appref  -- for appendix reference
%%          \axiref  -- for axiom reference
%%          \conjref -- for conjecture reference
%%          \defref  -- for definition reference
%%          \lemref  -- for lemma reference
%%          \theoref -- for theorem reference
%%          \corref  -- for corollary reference
%%          \propref -- for proprosition reference
%%          \pgref   -- for page reference
%%
%%          Example: See \chapref{chap:introduction}. It will produce 
%%                   'See Chapter 1', in case of English language.

\documentclass[pt,twoside,onehalfspacing,bsc]{risethesis}

\usepackage[sort,square]{natbib} %package for bibliography
\bibliographystyle{ieeetr} %authordate1

\usepackage[utf8]{inputenc}
\usepackage[brazilian]{babel}
\usepackage[T1]{fontenc}

\usepackage{graphicx}

%% Change the following pdf author attribute name to your name.
\usepackage[linkcolor=blue,citecolor=blue,urlcolor=blue,colorlinks,pdfpagelabels,pdftitle={Bruno Cabral's Bachelor Thesis},pdfauthor={Bruno Cabral}]{hyperref}

\address{SALVADOR}

\universitypt{Universidade Federal da Bahia}
\universityen{Federal University of Bahia}

\departmentpt{Depertamento de Ciência da Computação}
\departmenten{Computer Science Department}

\programpt{Programa Multiinstitucional de Pós-graduação em Ciência da Computação}
\programen{Graduate in Computer Science}

\majorfieldpt{Ciência da Computação}
\majorfielden{Computer Science}

\title{Sistema de apoio à Pós graduação - UFBA}
\date{Outubro/2016}

\author{Victor de Azevedo Nunes}
\adviser{Ivan do Carmo Machado}

\begin{document}

\frontmatter
\frontpage
\presentationpage

\begin{dedicatory}
Eu dedico esta dissertação...
%I dedicate this dissertation to my family, girlfriend, friends and
%professors who gave me all necessary support to get here.
\end{dedicatory}

\acknowledgements
Meus agradecimentos...

\begin{epigraph}[]{Edward V Berard}
Walking on water and developing software from a specification are easy if both are frozen
\end{epigraph}

\resumo
% Escreva seu resumo no arquivo resumo.tex
Este trabalho de conclusão de curso utiliza elementos da engenharia de software para propor um software como serviço a fim de otimizar o processo de construção e manutenção dos softwares. Assim, o objetivo deste SaaS é gerenciar as permissões de acesso de sistemas cliente, provendo o reuso de software e padronizando as soluções. O sistema proposto neste trabalho encontra-se implementado e disponível no Github, e traz fundamentos sobre a arquitetura e tecnologias utilizadas, bem como avaliações sobre possibilidades de utilização do mesmo em diversos ambientes, como Web e mobile.

\begin{keywords}
Software; Reuso; SaaS; Web; PHP
\end{keywords}

\abstract
% Write your abstract in a file called abstract.tex
My abstract...

\begin{keywords}
key words...
\end{keywords}

% Summary (tables of contents)
\tableofcontents

% List of figures
\listoffigures

% List of tables
\listoftables

% List of acronyms
% Acronyms manual: http://linorg.usp.br/CTAN/macros/latex/contrib/acronym/acronym.pdf
\listofacronyms
\begin{acronym}[ACRONYM] 
% Change the word ACRONYM above to change the acronym column width.
% The column width is equals to the width of the word that you put.
% Read the manual about acronym package for more examples:
%   http://linorg.usp.br/CTAN/macros/latex/contrib/acronym/acronym.pdf

\acro{SPA}{Single Page Application}
\acro{JSON}{Javascript Object Notation}
\acro{PHP}{PHP: Hypertext Preprocessor}
\acro{SaaS}{Software as a Service}
\acro{ERP}{Enterprise Resource Planning}
\acro{QoS}{Quality of Service}
\acro{UML}{Unified Modeling Language}
\acro{MVC}{Model-View-Controller}
\acro{Ajax}{Asynchronous Javascript and XML}
\acro{HTML}{HyperText Markup Language}
\acro{CSS}{Cascading Style Sheets}
\acro{API}{Application Programming Interface}
\acro{DOM}{Document Object Model}
\acro{BPMN}{Business Process Model and Notation}
\acro{REST}{Representational State Transfer}

\end{acronym}

% List of listings
%\lstlistoflistings

\mainmatter

\chapter{Introdução}

\section{Motivação}

Organizar os procedimentos de um processo sempre nos traz vantagens. Apesar de no processo de implantação de um sistema, o mesmo burocratizar o processo, com o tempo temos o retorno da dedicação para a inserção dos dados. Com um certo volume de dados, é possível estruturar informações que num processo manual são difíceis de serem enxergadas. Assim, é possível depender menos das pessoas que organizam o processo, pois o legado de informações não estará mais somente na mente de alguns, mas sim documentado nos dados do sistema.

Além de colaborar na organização, também haverá uma grande colaboração no tempo gasto na gestão. Lidar com muitos papéis e confiar na mente humana para guardar informações, não é uma alternativa muito segura devido ao fato que as pessoas sempre estão sujeitas a sair do processo e levar contigo a experiência obtida. Experiência essa que faz com que os procedimentos sejam executados de forma mais eficiente. Entretanto, com um sistema inteligente, é possível auxiliar e tornar mais ágil a execução das tarefas.


\section{Problema}


De acordo com funcionários ligados ao o setor de pós graduação da UFBA, entrevistados a fim de um maior entendimento do cenário, apesar das semelhanças estruturais, a pós graduação gerida de forma diferencia da graduação. FULANO afirma que devido ao fato de não ter a mesma visibilidade, não tem acesso aos mesmos recursos de gestão acadêmica da graduação. O professores não executam somente atividades dentro da sala de aula, também tem diversas outras ocupações no setor. E muitos procedimentos realizados extra classe ainda se encontram sendo realizados de forma manual, estando mais vulnerável ao erro ou até mesmo à violação do processo. Também ocorre um grande desperdício de tempo pelos professores e gestores da área, devido ao diversos processos ainda realizados de forma manual, sem a devida documentação. Segundo FULANO, também entrevistado, esse tempo perdido implica numa redução da eficiência na sala de aula, pois o professor acaba por ter menos tempo disponível para o planejamento das atividades, o que gera impactos negativos aos alunos.


\section{Objetivos} %<o que deve ser feito/entregue>


Devido aos muitos processos sendo resolvidos de forma manual, propõe-se com solução um sistema moderno, arquitetado para ter funcionamento na web e com um módulo mobile, a fim de fornecer informações de forma rápida e eficiente para os professores através de notificações, já que o acesso à internet móvel é comum entre os possíveis usuários do sistema em questão.
O principal requisito para o sistema seria dispor recursos para reduzir o tempo desperdiçado pelos professores durante as atividades extra classe.


\section{Metodologia} %<como será feito | como resolver o problema apontado inicialmente>


%<analise de literatura | design | implementação | validação>
Baseando-se nas tecnologias gratuitas em alta no cenário atual do desenvolvimento web, dispomos de algumas opções eficientes para a implementação da solução. Dentre as possibilidades, considerando a facilidade para futura manutenção e continuidade do projeto, tende-se a optar por uma tecnologia popular. Como linguagem de programação, adota-se o PHP. A escolha é fundamentada de acordo com a pesquisa da RedMonk de 2015, que evidencia o uso das linguagens de programação de acordo com as discussões no StackOverflow e repositórios no GitHub. É possível constatar a popularidade do PHP no cenário atual com o gráfico da pesquisa citada, na qual o PHP é apresentado na terceira colocação, apenas atrás do lider JavaScript e do segundo colocado, o Java.

\begin{figure}
	\label{fig:graficoRedmonk}
	\includegraphics[width=1\textwidth]{img/grafico_redmonk}
	\caption{Ranking das liguagens de programação no Stack Overflow e Github}
\end{figure}


Ainda assim, para compor a interface do dado projeto, também ocorrerá o uso do líder JavaScript de forma intensa, provendo o elo com o as informações gerenciadas pelo PHP.


Entretanto, não seria inteligente desenvolver um sistema completo sem o auxílio de um framework. Dentre os frameworks disponíveis para PHP, hoje o destaque está com o Laravel, que se encontra no topo dentre os mais utilizados no momento. 


A WebHostFace, uma empresa de hospedagem, compilou várias estatísticas para criar um infográfico mostrando os frameworks PHP mais populares de 2015. Utilizando informações sobre os próprios clientes, o Google Trends, estatísticas de repositórios do GitHub e a pesquisa do SitePoint “Best PHP Frameworks 2015”, a WebHostFace elaborou o seguinte infográfico: 

\begin{figure}
	\label{fig:graficoWebhostface}
	\includegraphics[width=1\textwidth]{img/infografico_webhostface}
	\caption{Infográfico da WebhostFace, exibindo a popularidade dos Frameworks PHP em 2015}
\end{figure}

Assim, tem-se a evidência que o Laravel em 2015 teve a maior popularidade em projetos pessoais e tem a maior comunidade entre os concorrentes, o que o torna uma boa escolha para a escrita de um software que será continuado por terceiros.


Para elaborar os recursos de interface e integrar ao back-end PHP do sistema, será adotado o já conhecido AngularJS, ferramenta sólida e conhecida no aspecto em questão. 


Dados coletados via Google Trends, que propõe comparações entre termos pesquisados, revela a popularidade do AngularJs diante de alguns dos principais concorrentes. O gráfico abaixo evidencia o cenário.


%Como mostra a Figura \ref{fig:graficoGoogleTrendsFerramentasFront}. 
\begin{figure}
	\label{fig:graficoGoogleTrendsFerramentasFront}
	\includegraphics[width=1\textwidth]{img/grafico_ferramentas_front}
	\caption{Gráfico do Google Trends exibindo as pesquisas por ferramentas front-end}
\end{figure}


Junto ao Angular JS, será utilizada a agradável tendência de interface do Material Design da Google, que propõe layouts limpos e otimizados já conhecidos pelos usuários de smartphones Android. 


Para a elaboração da plataforma mobile do projeto, será utilizado o Ionic Framework, muito difundido e bastante pesquisado na área, o que fica evidenciado com o gráfico de pesquisbaixo, coletado via Google Trends buscando por frameworks de desenvolvimento híbrido mobile.


\begin{figure}
	\label{fig:graficoGoogleTrendsFerramentasHibridasMobile}
	\includegraphics[width=1\textwidth]{img/grafico_ferramentas_hibridas_mobile}
	\caption{Gráfico do Google Trends exibindo as pesquisas por Frameworks híbridos mobile}
\end{figure}	

Para layout da interface mobile, também será aplicado a tendência do Material Design, a fim de propor uma harmonia entre o módulo web e mobile para os usuários


\section{Resultados Esperados}


Como fruto de um sistema para pós-graduação da UFBA, espera-se que os professores tenham mais recursos para integrar as atividades e também prover melhores condições para acompanhamento da vida acadêmica dos alunos em questão. Também, que os novos colaboradores que entrarem no processo tenham facilidade de compreender o fluxo do setor ao navegar pelo sistema proposto.


\section{Fora de Escopo}


Interação com os alunos devido às complicações para realizar a integração com o sistema empregado na UFBA, gerenciado pela XXXXXX, o que causaria uma inviabilidade no projeto devido à necessidade de entrega do produto ser mais forte que o tempo necessário para executar o processo de obtenção de acesso ao sistema legado para realizar a integração.


\section{Estrutura do Trabalho}


<breve resumo sobre os capítulos do TCC>
\chapter{Referencial Teórico}


Projetar o desenvolvimento de um software requer muito planejamento, pois as falhas iniciais podem custar bastante caro ou até mesmo inviabilizar a continuação de um projeto. Assim, a escolha da arquitetura ideal para a aplicabilidade é essencial na concepção de um produto de software. 
De todo o modo, sempre busca-se fazer mais com menos. Diante de tal filosofia, temos neste capítulo, uma breve discussão sobre alguns elementos de projeto e arquitetura de software, a fim de contextualizar este trabalho de conclusão de curso.
O capítulo corrente é composto por quatro seções. A \ref{sec:saas} trata de Software como serviço, discutindo alguns elementos do contexto que são relevantes para o trabalho proposto. A \ref{sec:reuso}  discute sobre a empregabilidade do reuso de software. A \ref{sec:modularizacao} seção, trata sobre aspectos envolvidos na modularização dos softwares. Por fim, a \ref{sec:apps_web} aborda as aplicações web, discutindo sobre aspectos relevantes sobre a aplicação web que compõe este trabalho.


 \section{Software como serviço}\label{sec:saas}


Segundo La e Chun \citep{La2009Systematic}, o princípio da definição de Software como um Serviço (Sofware as a Service - SaaS) é um serviço complementar para aplicações da computação em nuvem (cloud computing). No entanto, as áreas não se confundem. SaaS deve ser entendido como um mecanismo de suporte às soluções existentes na cloud. Os SaaS existem justamente para maximizar o reuso de serviços repetidos e não centrais em uma aplicação remota.


Como vantagens, diversos fatores podem ser favoráveis para a adoção de um SaaS, como custo e manutenção dentre outros fatores aplicáveis a soluções específicas. Lechesaet al. \citep{LechesaSS11}, quantificam índices sobre os fatores determinantes para adoção ou não de um SaaS voltado para ERP na África do Sul. Os principais fatores determinantes para adoção desse mecanismo de software são sua fluidez quanto à rede e a segurança. Esses fatores estão presentes na aplicação desenvolvida neste trabalho de conclusão de curso.


Devido ao fato de ter um serviço constantemente na nuvem, fica o questionamento sobre a segurança da informação manipulada. Sabe-se que a vulnerabilidade na Web não é restrita ao SaaS, atingindo diversos âmbitos. Rai et al. \cite{journals/corr/RaiSM13} consideram que o avanço da computação em nuvem não é um problema apenas para os serviços Web do ponto de vista da segurança, pois muitos trabalhos na literatura mostram a área como mais um ponto de vulnerabilidade para diversos setores, a exemplo de infraestrutura. No mesmo artigo mencionado de Rai et al. \cite{journals/corr/RaiSM13}, também realizaram-se estudos exploratórios junto a empresas usuárias de serviços em computação em nuvem e consideram que a perspectiva de SaaS também pode fortalecer a segurança nas aplicações de cloud computing, pois o software de autenticação compartilhado por várias aplicações em nuvem, oferece uma melhor padronização e consequente facilidade de prevenção a erros de vulnerabilidade específicas de cada módulo da pesquisa. Esse ponto de vista é fundamenta para a compreensão de SaaS.


A arquitetura de armazenamento de dados de um Saas pode variar de acordo com a necessidade do contexto. Huixin \cite{7586486} descreve possíveis modelagens nesse sentido. Essa abordagem pode ser com um banco de dados único, fazendo com que diferentes clientes compartilhem o mesmo banco, diferindo os dados através de controle de usuário, ou isolando os diferentes clientes através de bancos de dados exclusivos para cada um. Esse fator também pode ser combinado com a arquitetura da aplicação, caso ofereça aplicação única para todos os clientes ou aplicação compartilhada. Diante das possíveis abordagens, a modelagem de dados do software pode ser decidida pela regra de negócio. Este trabalho optou por aplicação única e banco de dados compartilhado.

Devido ao conceito particular de obtenção de software, tanto pela visão do cliente como pela visão do vendedor, é necessário compreender os diversos aspectos que podem ser relevantes ao orçar um Saas. O recente trabalho de T. Kaur et al. \citep{6949281} orienta um modelo para compor o custo de um Saas. O custo total seria composto pelos fatores que dão suporte ao funcionamento do software. Tais fatores incluem infra-estrutura, configurabilidade, customização, parâmetros de QoS (Quality of service) como escalabilidade, disponibilidade, usabilidade, pontualidade e desempenho da resposta, portabilidade, custo total de propriedade e retorno do investimento. Esses fatores caracterizam o custo de forma eficaz, possibilitando ao fornecedor, prover um Serviço de acordo com a exigência do consumidor em vários pacotes de serviços.


O conceito de software como serviço se aplica neste trabalho de conclusão de curso, pois o mesmo estará disponível na web com alta disponibilidade adotando as características apresentadas para qualquer pessoa que desejar usá-lo. Assim, basta se cadastrar, configurá-lo e usar como um serviço, sem preocupação com a manutenção do mesmo.


\section{Reuso de software}\label{sec:reuso} %CRUISE BOOK CAPITULO 2


De acordo com o livro Software pratical reuse \citep{Ezran2002}, o reuso de software é a utilização de qualquer informação que um desenvolvedor pode necessitar no processo de criação de software. O livro de Basili e Rombach \citep{Basili1991} define reutilização de software como o uso de tudo o que está associado a um projeto de conhecimento.
Assim, o objetivo da reutilização de software é reciclar o design, código e outros componentes de um produto de software e assim reduzir o custo, o tempo e melhorar a qualidade do produto.
Segundo Keswani et al. \cite{6783445}, o componente reutilizável de software pode ser qualquer parte de seu desenvolvimento, como um fragmento de código, design, casos de teste, ou até mesmo a especificação de requisitos de uma funcionalidade do software. 

O reuso de software pode ter impacto positivo em diversos aspectos do software, vejamos alguns, conforme apresentados no C.R.U.I.S.E Book \citep{cruise2007} :

\begin{itemize}

\item Qualidade: As correções de erro tornam-se úteis em todos os locais em que ocorreu, padronizando e facilitando a manutenção.

\item Produtividade: O ganho de produtividade é alcançado devido ao menor número de artefatos desenvolvido. Isso resulta em menor esforço de teste e também análise e design, reduzindo custos.

\item Confiabilidade: A utilização de componentes bem testados aumenta a
confiança no software. Além disso, a utilização de um mesmo componente em vários sistemas, aumenta a possibilidade de detecção de erros e reforça a confiança no componente.

\item Redução do Esforço: A reutilização de software proporciona uma redução do tempo de desenvolvimento, o que reduz o tempo necessário para o produto ser disponibilizado no mercado para trazer rentabilidade.

\item Trabalho redundante e tempo de desenvolvimento: Desenvolver um sistema do
zero significa desenvolvimento redundante de muitos componentes, como requisitos, especificações, casos de uso, arquitetura, etc. Isso pode ser evitado quando estes estão disponíveis como componentes reutilizáveis e podem ser compartilhados, resultando em um processo de desenvolvimento otimizado.

\item Documentação: Embora a documentação seja muito importante para a
manutenção de um sistema, muitas vezes é negligenciada. A reutilização de componentes de software reduz a quantidade de documentação a ser escrita, entretanto depende da qualidade do que está escrito. Assim, apenas a estrutura do sistema e os novos artefatos desenvolvidos necessitam ser documentados.

\item Custo de manutenção: Menos defeitos e manutenções são esperados quando tem-se comprovada a qualidade dos componentes utilizados.

\item Tamanho da equipe: É comum haver casos em que a equipe de desenvolvimento sofre sobrecarga. Entretanto, dobrar o tamanho da equipe de desenvolvimento não necessariamente duplica produtividade. Se muitos componentes podem ser reutilizados, é possível desenvolver com equipes menores, levando a melhor comunicação e aumento da produtividade.

\end{itemize}

Apesar dos benefícios da reutilização de software, ela não é suficientemente aproveitada. Existem fatores que influenciam direta ou indiretamente na sua adoção. Esses fatores podem ser de aspecto gerencial, organizacional, econômico, conceitual ou técnico. Veremos a seguir alguns aspectos que podem gerar conflito com a cultura de reuso de software, segundo o C.R.U.I.S.E Book \citep{cruise2007} :
%(Sametinger, 1997). REVER

\begin{itemize}
	
\item Falta de apoio da gestão: Como a reutilização de software gera custos iniciais,
a medida pode não ser amplamente alcançada em uma organização sem o apoio de alto nível de gestão. Os gestores têm de ser informados sobre os custos iniciais e serem convencidos sobre economias futuras.

\item Gerenciamento do Projeto: Gerenciar projetos tradicionais é uma tarefa árdua, principalmente, os que praticam a reutilização de software. Utilizando a técnica em larga escala, tem-se impacto sobre todo o ciclo de vida do software.

\item Estruturas organizacionais inadequadas: As estruturas organizacionais devem
considerar diferentes necessidades que surgem quando a reutilização em larga escala está sendo adotada. Por exemplo, uma equipe particionada pode ser alocada somente para desenvolver, manter e certificar componentes reutilizáveis de software.

\item Incentivos de gestão: É comum a falta de incentivo para deixar os desenvolvedores gastarem tempo elaborando componentes do sistemas. A produtividade é muitas vezes medida apenas no tempo necessário para concluir um projeto. Assim, fazer qualquer trabalho além disso, embora benéfico para a empresa como um todo, diminui o seu sucesso. Mesmo quando os componentes reutilizáveis são utilizados, os benefícios obtidos são uma pequena fração do que poderia ser alcançado caso houvesse reutilização explícita, planejada e organizada.

\item Dificuldade de encontrar software reutilizável: Para reutilizar os componentes, devem existir formas eficientes de busca. Além disso, é importante ter um repositório bem organizado contendo componentes com um eficiente meio de acesso.

\item Não reutilização do software encontrado. O acesso fácil ao software existente
não necessariamente aumentar a reutilização. Os componentes reutilizáveis devem ser cuidadosamente especificados, projetados, implementados e documentados, pois em alguns casos, modificar e adaptar o código  pode ser mais custoso que a programação da funcionalidade necessária a partir do zero.

\item Modificação: É muito difícil encontrar um componente que funcione
exatamente da mesma maneira que queremos. Desta forma, são necessárias modificações e devem existir formas de determinar os seus efeitos sobre o componente.


\end{itemize}


A definição do reuso de software pode ser associada a este trabalho, pois ainda que como um serviço, estaremos adotando uma aplicação única para servir aos mais diversos softwares de uma empresa, por exemplo. Logo, ao adotar o projeto, pode ser interessante aplicar algumas métricas para tomar conhecimento de possíveis vantanjens como as citadas nessa seção, a exemplo de economia de tempo de desenvolvimento, e custo do projeto.


%Outra diretriz importante para a reutilização de software é reduzir o risco na criação de novos softwares. O risco tende a ser bastante reduzido se os componentes que estão sendo reutilizados têm as documentação, interfaces necessárias e devidamente testadas, fatores que contibruem para uma fácil integração.
%De acordo com Keswani et al. \citep{6783445}, para o reuso de software dar retornos apropriados, o processo deve ser sistemático e planejado. Qualquer organização que implemente a reutilização de software deve identificar os melhores métodos e estratégias de reutilização para obter a máxima produtividade. A reutilização de software ajuda a evitar software de engenharia a partir do zero, pois usa módulos de software existentes. A reutilização de software, embora seja uma tarefa difícil, especialmente para softwares antigos sem padrões de projeto, pode melhorar significativamente a produtividade e a qualidade de um produto de software. Embora a reutilização de software não seja um novo campo, ela pode dar grandes retornos em curto período de tempo.


\section{Modularização}\label{sec:modularizacao} %artigo de claudio pagina 222 introdução


%A modularidade vem desempenhando um papel predominante estágios emergentes das disciplinas de arquitetura de software [13]. Engenheiros de software consideram modularidade como princípio base na comparação entre arquiteturas alternativas  e arquitetura degeneração [9]. De fato, os engenheiros de software são incentivados a arquitecturas, baseando-se numa multiplicidade de mecanismos de modularidade disponíveis em: 
%(i) Linguagens de descrição de arquitetura (ADLs), como ACME [8], 
%(ii) catálogos de arquitetônicos [2, 13], e 
%(iii) conhecem bem princípios de alto nível, como interfaces de componentes estreitos, acoplamento arquitectónico reduzido e semelhantes.


Conforme é frisado por Wickramaarachchi e Lai \citep{7062705}, o conceito de modularização na indústria de software tem uma longa história e tem sido utilizado para melhorar o processo de desenvolvimento de software em diferentes estágios. Os principais conceitos por trás da modularização do software foram introduzidos por pesquisadores pioneiros há quarenta anos, com uma notável contribuição feita por Melvin Conway e David Parnas, que tem representação notável na engenharia de software.


Modularizar um software é um bom padrão a ser adotado. Segundo Wickramaarachchi e Lai \citep{7062705}, a modularização é importante na identificação de dependências e reduz as dificuldades diante de uma possível necessidade de grandes alterações. De uma perspectiva da engenharia de software, uma modularização geralmente tem várias vantagens, tais como: tornar a complexidade do software mais gerenciável, facilitar o trabalho paralelo e tornar o software mais maleável para acomodar o futuro incerto que um software pode ter. O objetivo final da modularização do software é aumentar a produtividade ea qualidade do software. Tal conceito encontra-se bastante difundido e estái incorporado em linguagens de programação e ferramentas de software. O presente trabalho favorece ao uso da modularização de um software e até mesmo pode ser considerado um módulo a ser acoplado a qualquer software, mediante a compatibilidade.


Ao realizar a adoção do trabalho proposto, fica bastante evidente a "responsabilidade" do mesmo no escopo do projeto. Assim, é possível tratá-lo como um módulo do projeto que o usa, mesmo que seja consumido como um serviço.


\section{Aplicações web}\label{sec:apps_web}


A popularidade das soluções Web aumentou exponencialmente na última década e todos os dias cresce o número de pessoas usuárias desse tipo de software. E seguindo um padrão próprio, Kumar et al. \citep{7813710} sugerem que para o desenvolvimento web, deve-se manter a prática eficaz de produzir diagramas UML. A abordagem baseada na web oferece uma maneira fácil e eficaz para gerenciar e controlar o processo de desenvolvimento por meio de artefatos de modelagem. Tal abordagem pode ser usada quando há uma exigência de lidar com mudanças muito rápidas e grandes em requisitos de forma muito eficaz em muito menos tempo, gerando assim um menor impacto. 


Para atender à fomentada demanda de aplicativos web, é necessário adotar métodos de desenvolvimentos que sejam ágeis, eficientes e de fácil manutenção. Yu Ping et al. \cite{1372143} propõem o uso do modelo MVC (Model, View e Controller) no desenvolvimento para softwares web. O modelo apresentado tornou-se um padrão popular e divide o software em camadas com propósito definido, tornando-o de mais fácil manutenção.


O Ajax (Asynchronous Javascript and XML) revolucionou a web. Conforme demonstrado por Yuping \citep{6845605}, ao usar a tecnologia Ajax, podemos enriquecer a experiência do usuário em aplicações baseadas em navegador de internet, e fornecer uma variedade de aplicações interativas para atender às necessidade de humanização das aplicações.
Os aplicativos Ajax em execução no navegador se comunicam com um servidor Web de forma assíncrona e atualizam apenas uma parte da página.


No artigo de Tesarik et al. \citep{4664349}, temos que o design de software SPA (Single page application) é uma maneira interessante de criar um software Web numa uma única página. Essa solução de página única sem navegação funciona apenas com base em técnicas dinâmicas e assíncronas, como o AJAX (citado no parágrafo anterior). No entanto, esta abordagem coloca o desenvolvedor antes de alguns desafios substanciais.
Para projetar a interface do usuário que mostra as informações, é importante projetar corretamente a tela para manipular os dados do aplicativo numa única página. O design da página deve ser elaborado para maximizar a decomposição da página em componentes distintos que encapsulam os principais casos de uso.
Também recomenda-se que para elaborar os artefatos visuais, explore os mais ricos recursos na implementação, como HTML5, JavaScript, Ajax, CSS3, e outras tecnologias que se apliquem. Entretanto, desenvolver uma rica interface com o uso de diversas tecnologias/frameworks pode ocasionar um esforço maior para explorar as possibilidades.
Tais componentes da interface de uma aplicação SPA normalmente são alimentados mediante o consumo de uma API Rest via requisições AJAX. Assim, é possível particionar as responsabilidades de processamento do software entre cliente e servidor.


Como é possível denotar no trabalho de Salvadori e Siqueire \citep{7195633}, REST é uma arquitetura muito popular para integração de aplicativos web, que permite compartilhar e reutilizar informações através de sistemas. Aplicações de grande escala baseadas em no REST deve ser implementadas utilizando estratégias e mecanismos para produzir sistemas que sejam fáceis de desenvolver, reutilizar e manter. As interfaces de integração fornecidas por esses sistemas, chamadas de Web API, têm uma influência importante nas características da implementação resultante, pois a sua forma de resposta deve ser casada com a interface que o consome.
 
 
Os elementos comentados nessa seção encontram-se presentes neste trabalho de conclusão de curso. Juntos, montam a estrutura tecnológica necessária aliada à arquitetura adotada. Esses elementos seguem tendências atuais dos softwares web, fazendo com que esse trabalho esteja composto por tecnologias modernas que estão com boa aceitação no mercado.

% %% RiSE Latex Template - version 0.5
%%
%% RiSE's latex template for thesis and dissertations
%% http://risetemplate.sourceforge.net
%%
%% (c) 2012 Yguaratã Cerqueira Cavalcanti (yguarata@gmail.com)
%%          Vinicius Cardoso Garcia (vinicius.garcia@gmail.com)
%%
%% This document was initially based on UFPEThesis template, from Paulo Gustavo
%% S. Fonseca.
%%
%% ACKNOWLEDGEMENTS
%%
%% We would like to thanks the RiSE's researchers community, the 
%% students from Federal University of Pernambuco, and other users that have
%% been contributing to this projects with comments and patches.
%%
%% GENERAL INSTRUCTIONS
%%
%% We strongly recommend you to compile your documents using pdflatex command.
%% It is also recommend use the texlipse plugin for Eclipse to edit your documents.
%%
%% Options for \documentclass command:
%%         * Idiom
%%           pt   - Portguese (default)
%%           en   - English
%%
%%         * Text type
%%           bsc  - B.Sc. Thesis
%%           msc  - M.Sc. Thesis (default)
%%           qual - PHD qualification (not tested yet)
%%           prop - PHD proposal (not tested yet)
%%           phd  - PHD thesis
%%
%%         * Media
%%           scr  - to eletronic version (PDF) / see the users guide
%%
%%         * Pagination
%%           oneside - unique face press
%%           twoside - two faces press
%%
%%		   * Line spacing
%%           singlespacing  - the same as using \linespread{1}
%%           onehalfspacing - the same as using \linespread{1.3}
%%           doublespacing  - the same as using \linespread{1.6}
%%
%% Reference commands. Use the following commands to make references in your
%% text:
%%          \figref  -- for Figure reference
%%          \tabref  -- for Table reference
%%          \eqnref  -- for equation reference
%%          \chapref -- for chapter reference
%%          \secref  -- for section reference
%%          \appref  -- for appendix reference
%%          \axiref  -- for axiom reference
%%          \conjref -- for conjecture reference
%%          \defref  -- for definition reference
%%          \lemref  -- for lemma reference
%%          \theoref -- for theorem reference
%%          \corref  -- for corollary reference
%%          \propref -- for proprosition reference
%%          \pgref   -- for page reference
%%
%%          Example: See \chapref{chap:introduction}. It will produce 
%%                   'See Chapter 1', in case of English language.

\documentclass[pt,twoside,onehalfspacing,bsc]{risethesis}

\usepackage[sort,square]{natbib} %package for bibliography
\bibliographystyle{ieeetr} %authordate1

\usepackage[utf8]{inputenc}
\usepackage[brazilian]{babel}
\usepackage[T1]{fontenc}

\usepackage{graphicx}

%% Change the following pdf author attribute name to your name.
\usepackage[linkcolor=blue,citecolor=blue,urlcolor=blue,colorlinks,pdfpagelabels,pdftitle={Bruno Cabral's Bachelor Thesis},pdfauthor={Bruno Cabral}]{hyperref}

\address{SALVADOR}

\universitypt{Universidade Federal da Bahia}
\universityen{Federal University of Bahia}

\departmentpt{Depertamento de Ciência da Computação}
\departmenten{Computer Science Department}

\programpt{Programa Multiinstitucional de Pós-graduação em Ciência da Computação}
\programen{Graduate in Computer Science}

\majorfieldpt{Ciência da Computação}
\majorfielden{Computer Science}

\title{Sistema de apoio à Pós graduação - UFBA}
\date{Outubro/2016}

\author{Victor de Azevedo Nunes}
\adviser{Ivan do Carmo Machado}

\begin{document}

\frontmatter
\frontpage
\presentationpage

\begin{dedicatory}
Eu dedico esta dissertação...
%I dedicate this dissertation to my family, girlfriend, friends and
%professors who gave me all necessary support to get here.
\end{dedicatory}

\acknowledgements
Meus agradecimentos...

\begin{epigraph}[]{Edward V Berard}
Walking on water and developing software from a specification are easy if both are frozen
\end{epigraph}

\resumo
% Escreva seu resumo no arquivo resumo.tex
Este trabalho de conclusão de curso utiliza elementos da engenharia de software para propor um software como serviço a fim de otimizar o processo de construção e manutenção dos softwares. Assim, o objetivo deste SaaS é gerenciar as permissões de acesso de sistemas cliente, provendo o reuso de software e padronizando as soluções. O sistema proposto neste trabalho encontra-se implementado e disponível no Github, e traz fundamentos sobre a arquitetura e tecnologias utilizadas, bem como avaliações sobre possibilidades de utilização do mesmo em diversos ambientes, como Web e mobile.

\begin{keywords}
Software; Reuso; SaaS; Web; PHP
\end{keywords}

\abstract
% Write your abstract in a file called abstract.tex
My abstract...

\begin{keywords}
key words...
\end{keywords}

% Summary (tables of contents)
\tableofcontents

% List of figures
\listoffigures

% List of tables
\listoftables

% List of acronyms
% Acronyms manual: http://linorg.usp.br/CTAN/macros/latex/contrib/acronym/acronym.pdf
\listofacronyms
\begin{acronym}[ACRONYM] 
% Change the word ACRONYM above to change the acronym column width.
% The column width is equals to the width of the word that you put.
% Read the manual about acronym package for more examples:
%   http://linorg.usp.br/CTAN/macros/latex/contrib/acronym/acronym.pdf

\acro{SPA}{Single Page Application}
\acro{JSON}{Javascript Object Notation}
\acro{PHP}{PHP: Hypertext Preprocessor}
\acro{SaaS}{Software as a Service}
\acro{ERP}{Enterprise Resource Planning}
\acro{QoS}{Quality of Service}
\acro{UML}{Unified Modeling Language}
\acro{MVC}{Model-View-Controller}
\acro{Ajax}{Asynchronous Javascript and XML}
\acro{HTML}{HyperText Markup Language}
\acro{CSS}{Cascading Style Sheets}
\acro{API}{Application Programming Interface}
\acro{DOM}{Document Object Model}
\acro{BPMN}{Business Process Model and Notation}
\acro{REST}{Representational State Transfer}

\end{acronym}

% List of listings
%\lstlistoflistings

\mainmatter

\chapter{Introdução}

\section{Motivação}

Organizar os procedimentos de um processo sempre nos traz vantagens. Apesar de no processo de implantação de um sistema, o mesmo burocratizar o processo, com o tempo temos o retorno da dedicação para a inserção dos dados. Com um certo volume de dados, é possível estruturar informações que num processo manual são difíceis de serem enxergadas. Assim, é possível depender menos das pessoas que organizam o processo, pois o legado de informações não estará mais somente na mente de alguns, mas sim documentado nos dados do sistema.

Além de colaborar na organização, também haverá uma grande colaboração no tempo gasto na gestão. Lidar com muitos papéis e confiar na mente humana para guardar informações, não é uma alternativa muito segura devido ao fato que as pessoas sempre estão sujeitas a sair do processo e levar contigo a experiência obtida. Experiência essa que faz com que os procedimentos sejam executados de forma mais eficiente. Entretanto, com um sistema inteligente, é possível auxiliar e tornar mais ágil a execução das tarefas.


\section{Problema}


De acordo com funcionários ligados ao o setor de pós graduação da UFBA, entrevistados a fim de um maior entendimento do cenário, apesar das semelhanças estruturais, a pós graduação gerida de forma diferencia da graduação. FULANO afirma que devido ao fato de não ter a mesma visibilidade, não tem acesso aos mesmos recursos de gestão acadêmica da graduação. O professores não executam somente atividades dentro da sala de aula, também tem diversas outras ocupações no setor. E muitos procedimentos realizados extra classe ainda se encontram sendo realizados de forma manual, estando mais vulnerável ao erro ou até mesmo à violação do processo. Também ocorre um grande desperdício de tempo pelos professores e gestores da área, devido ao diversos processos ainda realizados de forma manual, sem a devida documentação. Segundo FULANO, também entrevistado, esse tempo perdido implica numa redução da eficiência na sala de aula, pois o professor acaba por ter menos tempo disponível para o planejamento das atividades, o que gera impactos negativos aos alunos.


\section{Objetivos} %<o que deve ser feito/entregue>


Devido aos muitos processos sendo resolvidos de forma manual, propõe-se com solução um sistema moderno, arquitetado para ter funcionamento na web e com um módulo mobile, a fim de fornecer informações de forma rápida e eficiente para os professores através de notificações, já que o acesso à internet móvel é comum entre os possíveis usuários do sistema em questão.
O principal requisito para o sistema seria dispor recursos para reduzir o tempo desperdiçado pelos professores durante as atividades extra classe.


\section{Metodologia} %<como será feito | como resolver o problema apontado inicialmente>


%<analise de literatura | design | implementação | validação>
Baseando-se nas tecnologias gratuitas em alta no cenário atual do desenvolvimento web, dispomos de algumas opções eficientes para a implementação da solução. Dentre as possibilidades, considerando a facilidade para futura manutenção e continuidade do projeto, tende-se a optar por uma tecnologia popular. Como linguagem de programação, adota-se o PHP. A escolha é fundamentada de acordo com a pesquisa da RedMonk de 2015, que evidencia o uso das linguagens de programação de acordo com as discussões no StackOverflow e repositórios no GitHub. É possível constatar a popularidade do PHP no cenário atual com o gráfico da pesquisa citada, na qual o PHP é apresentado na terceira colocação, apenas atrás do lider JavaScript e do segundo colocado, o Java.

\begin{figure}
	\label{fig:graficoRedmonk}
	\includegraphics[width=1\textwidth]{img/grafico_redmonk}
	\caption{Ranking das liguagens de programação no Stack Overflow e Github}
\end{figure}


Ainda assim, para compor a interface do dado projeto, também ocorrerá o uso do líder JavaScript de forma intensa, provendo o elo com o as informações gerenciadas pelo PHP.


Entretanto, não seria inteligente desenvolver um sistema completo sem o auxílio de um framework. Dentre os frameworks disponíveis para PHP, hoje o destaque está com o Laravel, que se encontra no topo dentre os mais utilizados no momento. 


A WebHostFace, uma empresa de hospedagem, compilou várias estatísticas para criar um infográfico mostrando os frameworks PHP mais populares de 2015. Utilizando informações sobre os próprios clientes, o Google Trends, estatísticas de repositórios do GitHub e a pesquisa do SitePoint “Best PHP Frameworks 2015”, a WebHostFace elaborou o seguinte infográfico: 

\begin{figure}
	\label{fig:graficoWebhostface}
	\includegraphics[width=1\textwidth]{img/infografico_webhostface}
	\caption{Infográfico da WebhostFace, exibindo a popularidade dos Frameworks PHP em 2015}
\end{figure}

Assim, tem-se a evidência que o Laravel em 2015 teve a maior popularidade em projetos pessoais e tem a maior comunidade entre os concorrentes, o que o torna uma boa escolha para a escrita de um software que será continuado por terceiros.


Para elaborar os recursos de interface e integrar ao back-end PHP do sistema, será adotado o já conhecido AngularJS, ferramenta sólida e conhecida no aspecto em questão. 


Dados coletados via Google Trends, que propõe comparações entre termos pesquisados, revela a popularidade do AngularJs diante de alguns dos principais concorrentes. O gráfico abaixo evidencia o cenário.


%Como mostra a Figura \ref{fig:graficoGoogleTrendsFerramentasFront}. 
\begin{figure}
	\label{fig:graficoGoogleTrendsFerramentasFront}
	\includegraphics[width=1\textwidth]{img/grafico_ferramentas_front}
	\caption{Gráfico do Google Trends exibindo as pesquisas por ferramentas front-end}
\end{figure}


Junto ao Angular JS, será utilizada a agradável tendência de interface do Material Design da Google, que propõe layouts limpos e otimizados já conhecidos pelos usuários de smartphones Android. 


Para a elaboração da plataforma mobile do projeto, será utilizado o Ionic Framework, muito difundido e bastante pesquisado na área, o que fica evidenciado com o gráfico de pesquisbaixo, coletado via Google Trends buscando por frameworks de desenvolvimento híbrido mobile.


\begin{figure}
	\label{fig:graficoGoogleTrendsFerramentasHibridasMobile}
	\includegraphics[width=1\textwidth]{img/grafico_ferramentas_hibridas_mobile}
	\caption{Gráfico do Google Trends exibindo as pesquisas por Frameworks híbridos mobile}
\end{figure}	

Para layout da interface mobile, também será aplicado a tendência do Material Design, a fim de propor uma harmonia entre o módulo web e mobile para os usuários


\section{Resultados Esperados}


Como fruto de um sistema para pós-graduação da UFBA, espera-se que os professores tenham mais recursos para integrar as atividades e também prover melhores condições para acompanhamento da vida acadêmica dos alunos em questão. Também, que os novos colaboradores que entrarem no processo tenham facilidade de compreender o fluxo do setor ao navegar pelo sistema proposto.


\section{Fora de Escopo}


Interação com os alunos devido às complicações para realizar a integração com o sistema empregado na UFBA, gerenciado pela XXXXXX, o que causaria uma inviabilidade no projeto devido à necessidade de entrega do produto ser mais forte que o tempo necessário para executar o processo de obtenção de acesso ao sistema legado para realizar a integração.


\section{Estrutura do Trabalho}


<breve resumo sobre os capítulos do TCC>
\chapter{Referencial Teórico}


Projetar o desenvolvimento de um software requer muito planejamento, pois as falhas iniciais podem custar bastante caro ou até mesmo inviabilizar a continuação de um projeto. Assim, a escolha da arquitetura ideal para a aplicabilidade é essencial na concepção de um produto de software. 
De todo o modo, sempre busca-se fazer mais com menos. Diante de tal filosofia, temos neste capítulo, uma breve discussão sobre alguns elementos de projeto e arquitetura de software, a fim de contextualizar este trabalho de conclusão de curso.
O capítulo corrente é composto por quatro seções. A \ref{sec:saas} trata de Software como serviço, discutindo alguns elementos do contexto que são relevantes para o trabalho proposto. A \ref{sec:reuso}  discute sobre a empregabilidade do reuso de software. A \ref{sec:modularizacao} seção, trata sobre aspectos envolvidos na modularização dos softwares. Por fim, a \ref{sec:apps_web} aborda as aplicações web, discutindo sobre aspectos relevantes sobre a aplicação web que compõe este trabalho.


 \section{Software como serviço}\label{sec:saas}


Segundo La e Chun \citep{La2009Systematic}, o princípio da definição de Software como um Serviço (Sofware as a Service - SaaS) é um serviço complementar para aplicações da computação em nuvem (cloud computing). No entanto, as áreas não se confundem. SaaS deve ser entendido como um mecanismo de suporte às soluções existentes na cloud. Os SaaS existem justamente para maximizar o reuso de serviços repetidos e não centrais em uma aplicação remota.


Como vantagens, diversos fatores podem ser favoráveis para a adoção de um SaaS, como custo e manutenção dentre outros fatores aplicáveis a soluções específicas. Lechesaet al. \citep{LechesaSS11}, quantificam índices sobre os fatores determinantes para adoção ou não de um SaaS voltado para ERP na África do Sul. Os principais fatores determinantes para adoção desse mecanismo de software são sua fluidez quanto à rede e a segurança. Esses fatores estão presentes na aplicação desenvolvida neste trabalho de conclusão de curso.


Devido ao fato de ter um serviço constantemente na nuvem, fica o questionamento sobre a segurança da informação manipulada. Sabe-se que a vulnerabilidade na Web não é restrita ao SaaS, atingindo diversos âmbitos. Rai et al. \cite{journals/corr/RaiSM13} consideram que o avanço da computação em nuvem não é um problema apenas para os serviços Web do ponto de vista da segurança, pois muitos trabalhos na literatura mostram a área como mais um ponto de vulnerabilidade para diversos setores, a exemplo de infraestrutura. No mesmo artigo mencionado de Rai et al. \cite{journals/corr/RaiSM13}, também realizaram-se estudos exploratórios junto a empresas usuárias de serviços em computação em nuvem e consideram que a perspectiva de SaaS também pode fortalecer a segurança nas aplicações de cloud computing, pois o software de autenticação compartilhado por várias aplicações em nuvem, oferece uma melhor padronização e consequente facilidade de prevenção a erros de vulnerabilidade específicas de cada módulo da pesquisa. Esse ponto de vista é fundamenta para a compreensão de SaaS.


A arquitetura de armazenamento de dados de um Saas pode variar de acordo com a necessidade do contexto. Huixin \cite{7586486} descreve possíveis modelagens nesse sentido. Essa abordagem pode ser com um banco de dados único, fazendo com que diferentes clientes compartilhem o mesmo banco, diferindo os dados através de controle de usuário, ou isolando os diferentes clientes através de bancos de dados exclusivos para cada um. Esse fator também pode ser combinado com a arquitetura da aplicação, caso ofereça aplicação única para todos os clientes ou aplicação compartilhada. Diante das possíveis abordagens, a modelagem de dados do software pode ser decidida pela regra de negócio. Este trabalho optou por aplicação única e banco de dados compartilhado.

Devido ao conceito particular de obtenção de software, tanto pela visão do cliente como pela visão do vendedor, é necessário compreender os diversos aspectos que podem ser relevantes ao orçar um Saas. O recente trabalho de T. Kaur et al. \citep{6949281} orienta um modelo para compor o custo de um Saas. O custo total seria composto pelos fatores que dão suporte ao funcionamento do software. Tais fatores incluem infra-estrutura, configurabilidade, customização, parâmetros de QoS (Quality of service) como escalabilidade, disponibilidade, usabilidade, pontualidade e desempenho da resposta, portabilidade, custo total de propriedade e retorno do investimento. Esses fatores caracterizam o custo de forma eficaz, possibilitando ao fornecedor, prover um Serviço de acordo com a exigência do consumidor em vários pacotes de serviços.


O conceito de software como serviço se aplica neste trabalho de conclusão de curso, pois o mesmo estará disponível na web com alta disponibilidade adotando as características apresentadas para qualquer pessoa que desejar usá-lo. Assim, basta se cadastrar, configurá-lo e usar como um serviço, sem preocupação com a manutenção do mesmo.


\section{Reuso de software}\label{sec:reuso} %CRUISE BOOK CAPITULO 2


De acordo com o livro Software pratical reuse \citep{Ezran2002}, o reuso de software é a utilização de qualquer informação que um desenvolvedor pode necessitar no processo de criação de software. O livro de Basili e Rombach \citep{Basili1991} define reutilização de software como o uso de tudo o que está associado a um projeto de conhecimento.
Assim, o objetivo da reutilização de software é reciclar o design, código e outros componentes de um produto de software e assim reduzir o custo, o tempo e melhorar a qualidade do produto.
Segundo Keswani et al. \cite{6783445}, o componente reutilizável de software pode ser qualquer parte de seu desenvolvimento, como um fragmento de código, design, casos de teste, ou até mesmo a especificação de requisitos de uma funcionalidade do software. 

O reuso de software pode ter impacto positivo em diversos aspectos do software, vejamos alguns, conforme apresentados no C.R.U.I.S.E Book \citep{cruise2007} :

\begin{itemize}

\item Qualidade: As correções de erro tornam-se úteis em todos os locais em que ocorreu, padronizando e facilitando a manutenção.

\item Produtividade: O ganho de produtividade é alcançado devido ao menor número de artefatos desenvolvido. Isso resulta em menor esforço de teste e também análise e design, reduzindo custos.

\item Confiabilidade: A utilização de componentes bem testados aumenta a
confiança no software. Além disso, a utilização de um mesmo componente em vários sistemas, aumenta a possibilidade de detecção de erros e reforça a confiança no componente.

\item Redução do Esforço: A reutilização de software proporciona uma redução do tempo de desenvolvimento, o que reduz o tempo necessário para o produto ser disponibilizado no mercado para trazer rentabilidade.

\item Trabalho redundante e tempo de desenvolvimento: Desenvolver um sistema do
zero significa desenvolvimento redundante de muitos componentes, como requisitos, especificações, casos de uso, arquitetura, etc. Isso pode ser evitado quando estes estão disponíveis como componentes reutilizáveis e podem ser compartilhados, resultando em um processo de desenvolvimento otimizado.

\item Documentação: Embora a documentação seja muito importante para a
manutenção de um sistema, muitas vezes é negligenciada. A reutilização de componentes de software reduz a quantidade de documentação a ser escrita, entretanto depende da qualidade do que está escrito. Assim, apenas a estrutura do sistema e os novos artefatos desenvolvidos necessitam ser documentados.

\item Custo de manutenção: Menos defeitos e manutenções são esperados quando tem-se comprovada a qualidade dos componentes utilizados.

\item Tamanho da equipe: É comum haver casos em que a equipe de desenvolvimento sofre sobrecarga. Entretanto, dobrar o tamanho da equipe de desenvolvimento não necessariamente duplica produtividade. Se muitos componentes podem ser reutilizados, é possível desenvolver com equipes menores, levando a melhor comunicação e aumento da produtividade.

\end{itemize}

Apesar dos benefícios da reutilização de software, ela não é suficientemente aproveitada. Existem fatores que influenciam direta ou indiretamente na sua adoção. Esses fatores podem ser de aspecto gerencial, organizacional, econômico, conceitual ou técnico. Veremos a seguir alguns aspectos que podem gerar conflito com a cultura de reuso de software, segundo o C.R.U.I.S.E Book \citep{cruise2007} :
%(Sametinger, 1997). REVER

\begin{itemize}
	
\item Falta de apoio da gestão: Como a reutilização de software gera custos iniciais,
a medida pode não ser amplamente alcançada em uma organização sem o apoio de alto nível de gestão. Os gestores têm de ser informados sobre os custos iniciais e serem convencidos sobre economias futuras.

\item Gerenciamento do Projeto: Gerenciar projetos tradicionais é uma tarefa árdua, principalmente, os que praticam a reutilização de software. Utilizando a técnica em larga escala, tem-se impacto sobre todo o ciclo de vida do software.

\item Estruturas organizacionais inadequadas: As estruturas organizacionais devem
considerar diferentes necessidades que surgem quando a reutilização em larga escala está sendo adotada. Por exemplo, uma equipe particionada pode ser alocada somente para desenvolver, manter e certificar componentes reutilizáveis de software.

\item Incentivos de gestão: É comum a falta de incentivo para deixar os desenvolvedores gastarem tempo elaborando componentes do sistemas. A produtividade é muitas vezes medida apenas no tempo necessário para concluir um projeto. Assim, fazer qualquer trabalho além disso, embora benéfico para a empresa como um todo, diminui o seu sucesso. Mesmo quando os componentes reutilizáveis são utilizados, os benefícios obtidos são uma pequena fração do que poderia ser alcançado caso houvesse reutilização explícita, planejada e organizada.

\item Dificuldade de encontrar software reutilizável: Para reutilizar os componentes, devem existir formas eficientes de busca. Além disso, é importante ter um repositório bem organizado contendo componentes com um eficiente meio de acesso.

\item Não reutilização do software encontrado. O acesso fácil ao software existente
não necessariamente aumentar a reutilização. Os componentes reutilizáveis devem ser cuidadosamente especificados, projetados, implementados e documentados, pois em alguns casos, modificar e adaptar o código  pode ser mais custoso que a programação da funcionalidade necessária a partir do zero.

\item Modificação: É muito difícil encontrar um componente que funcione
exatamente da mesma maneira que queremos. Desta forma, são necessárias modificações e devem existir formas de determinar os seus efeitos sobre o componente.


\end{itemize}


A definição do reuso de software pode ser associada a este trabalho, pois ainda que como um serviço, estaremos adotando uma aplicação única para servir aos mais diversos softwares de uma empresa, por exemplo. Logo, ao adotar o projeto, pode ser interessante aplicar algumas métricas para tomar conhecimento de possíveis vantanjens como as citadas nessa seção, a exemplo de economia de tempo de desenvolvimento, e custo do projeto.


%Outra diretriz importante para a reutilização de software é reduzir o risco na criação de novos softwares. O risco tende a ser bastante reduzido se os componentes que estão sendo reutilizados têm as documentação, interfaces necessárias e devidamente testadas, fatores que contibruem para uma fácil integração.
%De acordo com Keswani et al. \citep{6783445}, para o reuso de software dar retornos apropriados, o processo deve ser sistemático e planejado. Qualquer organização que implemente a reutilização de software deve identificar os melhores métodos e estratégias de reutilização para obter a máxima produtividade. A reutilização de software ajuda a evitar software de engenharia a partir do zero, pois usa módulos de software existentes. A reutilização de software, embora seja uma tarefa difícil, especialmente para softwares antigos sem padrões de projeto, pode melhorar significativamente a produtividade e a qualidade de um produto de software. Embora a reutilização de software não seja um novo campo, ela pode dar grandes retornos em curto período de tempo.


\section{Modularização}\label{sec:modularizacao} %artigo de claudio pagina 222 introdução


%A modularidade vem desempenhando um papel predominante estágios emergentes das disciplinas de arquitetura de software [13]. Engenheiros de software consideram modularidade como princípio base na comparação entre arquiteturas alternativas  e arquitetura degeneração [9]. De fato, os engenheiros de software são incentivados a arquitecturas, baseando-se numa multiplicidade de mecanismos de modularidade disponíveis em: 
%(i) Linguagens de descrição de arquitetura (ADLs), como ACME [8], 
%(ii) catálogos de arquitetônicos [2, 13], e 
%(iii) conhecem bem princípios de alto nível, como interfaces de componentes estreitos, acoplamento arquitectónico reduzido e semelhantes.


Conforme é frisado por Wickramaarachchi e Lai \citep{7062705}, o conceito de modularização na indústria de software tem uma longa história e tem sido utilizado para melhorar o processo de desenvolvimento de software em diferentes estágios. Os principais conceitos por trás da modularização do software foram introduzidos por pesquisadores pioneiros há quarenta anos, com uma notável contribuição feita por Melvin Conway e David Parnas, que tem representação notável na engenharia de software.


Modularizar um software é um bom padrão a ser adotado. Segundo Wickramaarachchi e Lai \citep{7062705}, a modularização é importante na identificação de dependências e reduz as dificuldades diante de uma possível necessidade de grandes alterações. De uma perspectiva da engenharia de software, uma modularização geralmente tem várias vantagens, tais como: tornar a complexidade do software mais gerenciável, facilitar o trabalho paralelo e tornar o software mais maleável para acomodar o futuro incerto que um software pode ter. O objetivo final da modularização do software é aumentar a produtividade ea qualidade do software. Tal conceito encontra-se bastante difundido e estái incorporado em linguagens de programação e ferramentas de software. O presente trabalho favorece ao uso da modularização de um software e até mesmo pode ser considerado um módulo a ser acoplado a qualquer software, mediante a compatibilidade.


Ao realizar a adoção do trabalho proposto, fica bastante evidente a "responsabilidade" do mesmo no escopo do projeto. Assim, é possível tratá-lo como um módulo do projeto que o usa, mesmo que seja consumido como um serviço.


\section{Aplicações web}\label{sec:apps_web}


A popularidade das soluções Web aumentou exponencialmente na última década e todos os dias cresce o número de pessoas usuárias desse tipo de software. E seguindo um padrão próprio, Kumar et al. \citep{7813710} sugerem que para o desenvolvimento web, deve-se manter a prática eficaz de produzir diagramas UML. A abordagem baseada na web oferece uma maneira fácil e eficaz para gerenciar e controlar o processo de desenvolvimento por meio de artefatos de modelagem. Tal abordagem pode ser usada quando há uma exigência de lidar com mudanças muito rápidas e grandes em requisitos de forma muito eficaz em muito menos tempo, gerando assim um menor impacto. 


Para atender à fomentada demanda de aplicativos web, é necessário adotar métodos de desenvolvimentos que sejam ágeis, eficientes e de fácil manutenção. Yu Ping et al. \cite{1372143} propõem o uso do modelo MVC (Model, View e Controller) no desenvolvimento para softwares web. O modelo apresentado tornou-se um padrão popular e divide o software em camadas com propósito definido, tornando-o de mais fácil manutenção.


O Ajax (Asynchronous Javascript and XML) revolucionou a web. Conforme demonstrado por Yuping \citep{6845605}, ao usar a tecnologia Ajax, podemos enriquecer a experiência do usuário em aplicações baseadas em navegador de internet, e fornecer uma variedade de aplicações interativas para atender às necessidade de humanização das aplicações.
Os aplicativos Ajax em execução no navegador se comunicam com um servidor Web de forma assíncrona e atualizam apenas uma parte da página.


No artigo de Tesarik et al. \citep{4664349}, temos que o design de software SPA (Single page application) é uma maneira interessante de criar um software Web numa uma única página. Essa solução de página única sem navegação funciona apenas com base em técnicas dinâmicas e assíncronas, como o AJAX (citado no parágrafo anterior). No entanto, esta abordagem coloca o desenvolvedor antes de alguns desafios substanciais.
Para projetar a interface do usuário que mostra as informações, é importante projetar corretamente a tela para manipular os dados do aplicativo numa única página. O design da página deve ser elaborado para maximizar a decomposição da página em componentes distintos que encapsulam os principais casos de uso.
Também recomenda-se que para elaborar os artefatos visuais, explore os mais ricos recursos na implementação, como HTML5, JavaScript, Ajax, CSS3, e outras tecnologias que se apliquem. Entretanto, desenvolver uma rica interface com o uso de diversas tecnologias/frameworks pode ocasionar um esforço maior para explorar as possibilidades.
Tais componentes da interface de uma aplicação SPA normalmente são alimentados mediante o consumo de uma API Rest via requisições AJAX. Assim, é possível particionar as responsabilidades de processamento do software entre cliente e servidor.


Como é possível denotar no trabalho de Salvadori e Siqueire \citep{7195633}, REST é uma arquitetura muito popular para integração de aplicativos web, que permite compartilhar e reutilizar informações através de sistemas. Aplicações de grande escala baseadas em no REST deve ser implementadas utilizando estratégias e mecanismos para produzir sistemas que sejam fáceis de desenvolver, reutilizar e manter. As interfaces de integração fornecidas por esses sistemas, chamadas de Web API, têm uma influência importante nas características da implementação resultante, pois a sua forma de resposta deve ser casada com a interface que o consome.
 
 
Os elementos comentados nessa seção encontram-se presentes neste trabalho de conclusão de curso. Juntos, montam a estrutura tecnológica necessária aliada à arquitetura adotada. Esses elementos seguem tendências atuais dos softwares web, fazendo com que esse trabalho esteja composto por tecnologias modernas que estão com boa aceitação no mercado.

% %% RiSE Latex Template - version 0.5
%%
%% RiSE's latex template for thesis and dissertations
%% http://risetemplate.sourceforge.net
%%
%% (c) 2012 Yguaratã Cerqueira Cavalcanti (yguarata@gmail.com)
%%          Vinicius Cardoso Garcia (vinicius.garcia@gmail.com)
%%
%% This document was initially based on UFPEThesis template, from Paulo Gustavo
%% S. Fonseca.
%%
%% ACKNOWLEDGEMENTS
%%
%% We would like to thanks the RiSE's researchers community, the 
%% students from Federal University of Pernambuco, and other users that have
%% been contributing to this projects with comments and patches.
%%
%% GENERAL INSTRUCTIONS
%%
%% We strongly recommend you to compile your documents using pdflatex command.
%% It is also recommend use the texlipse plugin for Eclipse to edit your documents.
%%
%% Options for \documentclass command:
%%         * Idiom
%%           pt   - Portguese (default)
%%           en   - English
%%
%%         * Text type
%%           bsc  - B.Sc. Thesis
%%           msc  - M.Sc. Thesis (default)
%%           qual - PHD qualification (not tested yet)
%%           prop - PHD proposal (not tested yet)
%%           phd  - PHD thesis
%%
%%         * Media
%%           scr  - to eletronic version (PDF) / see the users guide
%%
%%         * Pagination
%%           oneside - unique face press
%%           twoside - two faces press
%%
%%		   * Line spacing
%%           singlespacing  - the same as using \linespread{1}
%%           onehalfspacing - the same as using \linespread{1.3}
%%           doublespacing  - the same as using \linespread{1.6}
%%
%% Reference commands. Use the following commands to make references in your
%% text:
%%          \figref  -- for Figure reference
%%          \tabref  -- for Table reference
%%          \eqnref  -- for equation reference
%%          \chapref -- for chapter reference
%%          \secref  -- for section reference
%%          \appref  -- for appendix reference
%%          \axiref  -- for axiom reference
%%          \conjref -- for conjecture reference
%%          \defref  -- for definition reference
%%          \lemref  -- for lemma reference
%%          \theoref -- for theorem reference
%%          \corref  -- for corollary reference
%%          \propref -- for proprosition reference
%%          \pgref   -- for page reference
%%
%%          Example: See \chapref{chap:introduction}. It will produce 
%%                   'See Chapter 1', in case of English language.

\documentclass[pt,twoside,onehalfspacing,bsc]{risethesis}

\usepackage[sort,square]{natbib} %package for bibliography
\bibliographystyle{ieeetr} %authordate1

\usepackage[utf8]{inputenc}
\usepackage[brazilian]{babel}
\usepackage[T1]{fontenc}

\usepackage{graphicx}

%% Change the following pdf author attribute name to your name.
\usepackage[linkcolor=blue,citecolor=blue,urlcolor=blue,colorlinks,pdfpagelabels,pdftitle={Bruno Cabral's Bachelor Thesis},pdfauthor={Bruno Cabral}]{hyperref}

\address{SALVADOR}

\universitypt{Universidade Federal da Bahia}
\universityen{Federal University of Bahia}

\departmentpt{Depertamento de Ciência da Computação}
\departmenten{Computer Science Department}

\programpt{Programa Multiinstitucional de Pós-graduação em Ciência da Computação}
\programen{Graduate in Computer Science}

\majorfieldpt{Ciência da Computação}
\majorfielden{Computer Science}

\title{Sistema de apoio à Pós graduação - UFBA}
\date{Outubro/2016}

\author{Victor de Azevedo Nunes}
\adviser{Ivan do Carmo Machado}

\begin{document}

\frontmatter
\frontpage
\presentationpage

\begin{dedicatory}
Eu dedico esta dissertação...
%I dedicate this dissertation to my family, girlfriend, friends and
%professors who gave me all necessary support to get here.
\end{dedicatory}

\acknowledgements
Meus agradecimentos...

\begin{epigraph}[]{Edward V Berard}
Walking on water and developing software from a specification are easy if both are frozen
\end{epigraph}

\resumo
% Escreva seu resumo no arquivo resumo.tex
\input{resumo}

\abstract
% Write your abstract in a file called abstract.tex
\input{abstract}

% Summary (tables of contents)
\tableofcontents

% List of figures
\listoffigures

% List of tables
\listoftables

% List of acronyms
% Acronyms manual: http://linorg.usp.br/CTAN/macros/latex/contrib/acronym/acronym.pdf
\listofacronyms
\input{acronyms}

% List of listings
%\lstlistoflistings

\mainmatter

\include{chapters/intro}
\include{chapters/referencial_teorico}

% \include{chapters/introduction/main}
% \include{chapters/background/main}
% \include{chapters/proposed_solution/main}
% \include{chapters/experiment/main}
% \include{chapters/conclusion/main}

%\bibliographystyle{natbib}
%\addcontentsline{toc}{chapter}{\bibliographytocname}
\bibliography{references}

% Appendix
\clearpage
\addappheadtotoc
\appendix
\appendixpage
% \include{appendix/experiment-instruments}

\end{document}
% %% RiSE Latex Template - version 0.5
%%
%% RiSE's latex template for thesis and dissertations
%% http://risetemplate.sourceforge.net
%%
%% (c) 2012 Yguaratã Cerqueira Cavalcanti (yguarata@gmail.com)
%%          Vinicius Cardoso Garcia (vinicius.garcia@gmail.com)
%%
%% This document was initially based on UFPEThesis template, from Paulo Gustavo
%% S. Fonseca.
%%
%% ACKNOWLEDGEMENTS
%%
%% We would like to thanks the RiSE's researchers community, the 
%% students from Federal University of Pernambuco, and other users that have
%% been contributing to this projects with comments and patches.
%%
%% GENERAL INSTRUCTIONS
%%
%% We strongly recommend you to compile your documents using pdflatex command.
%% It is also recommend use the texlipse plugin for Eclipse to edit your documents.
%%
%% Options for \documentclass command:
%%         * Idiom
%%           pt   - Portguese (default)
%%           en   - English
%%
%%         * Text type
%%           bsc  - B.Sc. Thesis
%%           msc  - M.Sc. Thesis (default)
%%           qual - PHD qualification (not tested yet)
%%           prop - PHD proposal (not tested yet)
%%           phd  - PHD thesis
%%
%%         * Media
%%           scr  - to eletronic version (PDF) / see the users guide
%%
%%         * Pagination
%%           oneside - unique face press
%%           twoside - two faces press
%%
%%		   * Line spacing
%%           singlespacing  - the same as using \linespread{1}
%%           onehalfspacing - the same as using \linespread{1.3}
%%           doublespacing  - the same as using \linespread{1.6}
%%
%% Reference commands. Use the following commands to make references in your
%% text:
%%          \figref  -- for Figure reference
%%          \tabref  -- for Table reference
%%          \eqnref  -- for equation reference
%%          \chapref -- for chapter reference
%%          \secref  -- for section reference
%%          \appref  -- for appendix reference
%%          \axiref  -- for axiom reference
%%          \conjref -- for conjecture reference
%%          \defref  -- for definition reference
%%          \lemref  -- for lemma reference
%%          \theoref -- for theorem reference
%%          \corref  -- for corollary reference
%%          \propref -- for proprosition reference
%%          \pgref   -- for page reference
%%
%%          Example: See \chapref{chap:introduction}. It will produce 
%%                   'See Chapter 1', in case of English language.

\documentclass[pt,twoside,onehalfspacing,bsc]{risethesis}

\usepackage[sort,square]{natbib} %package for bibliography
\bibliographystyle{ieeetr} %authordate1

\usepackage[utf8]{inputenc}
\usepackage[brazilian]{babel}
\usepackage[T1]{fontenc}

\usepackage{graphicx}

%% Change the following pdf author attribute name to your name.
\usepackage[linkcolor=blue,citecolor=blue,urlcolor=blue,colorlinks,pdfpagelabels,pdftitle={Bruno Cabral's Bachelor Thesis},pdfauthor={Bruno Cabral}]{hyperref}

\address{SALVADOR}

\universitypt{Universidade Federal da Bahia}
\universityen{Federal University of Bahia}

\departmentpt{Depertamento de Ciência da Computação}
\departmenten{Computer Science Department}

\programpt{Programa Multiinstitucional de Pós-graduação em Ciência da Computação}
\programen{Graduate in Computer Science}

\majorfieldpt{Ciência da Computação}
\majorfielden{Computer Science}

\title{Sistema de apoio à Pós graduação - UFBA}
\date{Outubro/2016}

\author{Victor de Azevedo Nunes}
\adviser{Ivan do Carmo Machado}

\begin{document}

\frontmatter
\frontpage
\presentationpage

\begin{dedicatory}
Eu dedico esta dissertação...
%I dedicate this dissertation to my family, girlfriend, friends and
%professors who gave me all necessary support to get here.
\end{dedicatory}

\acknowledgements
Meus agradecimentos...

\begin{epigraph}[]{Edward V Berard}
Walking on water and developing software from a specification are easy if both are frozen
\end{epigraph}

\resumo
% Escreva seu resumo no arquivo resumo.tex
\input{resumo}

\abstract
% Write your abstract in a file called abstract.tex
\input{abstract}

% Summary (tables of contents)
\tableofcontents

% List of figures
\listoffigures

% List of tables
\listoftables

% List of acronyms
% Acronyms manual: http://linorg.usp.br/CTAN/macros/latex/contrib/acronym/acronym.pdf
\listofacronyms
\input{acronyms}

% List of listings
%\lstlistoflistings

\mainmatter

\include{chapters/intro}
\include{chapters/referencial_teorico}

% \include{chapters/introduction/main}
% \include{chapters/background/main}
% \include{chapters/proposed_solution/main}
% \include{chapters/experiment/main}
% \include{chapters/conclusion/main}

%\bibliographystyle{natbib}
%\addcontentsline{toc}{chapter}{\bibliographytocname}
\bibliography{references}

% Appendix
\clearpage
\addappheadtotoc
\appendix
\appendixpage
% \include{appendix/experiment-instruments}

\end{document}
% %% RiSE Latex Template - version 0.5
%%
%% RiSE's latex template for thesis and dissertations
%% http://risetemplate.sourceforge.net
%%
%% (c) 2012 Yguaratã Cerqueira Cavalcanti (yguarata@gmail.com)
%%          Vinicius Cardoso Garcia (vinicius.garcia@gmail.com)
%%
%% This document was initially based on UFPEThesis template, from Paulo Gustavo
%% S. Fonseca.
%%
%% ACKNOWLEDGEMENTS
%%
%% We would like to thanks the RiSE's researchers community, the 
%% students from Federal University of Pernambuco, and other users that have
%% been contributing to this projects with comments and patches.
%%
%% GENERAL INSTRUCTIONS
%%
%% We strongly recommend you to compile your documents using pdflatex command.
%% It is also recommend use the texlipse plugin for Eclipse to edit your documents.
%%
%% Options for \documentclass command:
%%         * Idiom
%%           pt   - Portguese (default)
%%           en   - English
%%
%%         * Text type
%%           bsc  - B.Sc. Thesis
%%           msc  - M.Sc. Thesis (default)
%%           qual - PHD qualification (not tested yet)
%%           prop - PHD proposal (not tested yet)
%%           phd  - PHD thesis
%%
%%         * Media
%%           scr  - to eletronic version (PDF) / see the users guide
%%
%%         * Pagination
%%           oneside - unique face press
%%           twoside - two faces press
%%
%%		   * Line spacing
%%           singlespacing  - the same as using \linespread{1}
%%           onehalfspacing - the same as using \linespread{1.3}
%%           doublespacing  - the same as using \linespread{1.6}
%%
%% Reference commands. Use the following commands to make references in your
%% text:
%%          \figref  -- for Figure reference
%%          \tabref  -- for Table reference
%%          \eqnref  -- for equation reference
%%          \chapref -- for chapter reference
%%          \secref  -- for section reference
%%          \appref  -- for appendix reference
%%          \axiref  -- for axiom reference
%%          \conjref -- for conjecture reference
%%          \defref  -- for definition reference
%%          \lemref  -- for lemma reference
%%          \theoref -- for theorem reference
%%          \corref  -- for corollary reference
%%          \propref -- for proprosition reference
%%          \pgref   -- for page reference
%%
%%          Example: See \chapref{chap:introduction}. It will produce 
%%                   'See Chapter 1', in case of English language.

\documentclass[pt,twoside,onehalfspacing,bsc]{risethesis}

\usepackage[sort,square]{natbib} %package for bibliography
\bibliographystyle{ieeetr} %authordate1

\usepackage[utf8]{inputenc}
\usepackage[brazilian]{babel}
\usepackage[T1]{fontenc}

\usepackage{graphicx}

%% Change the following pdf author attribute name to your name.
\usepackage[linkcolor=blue,citecolor=blue,urlcolor=blue,colorlinks,pdfpagelabels,pdftitle={Bruno Cabral's Bachelor Thesis},pdfauthor={Bruno Cabral}]{hyperref}

\address{SALVADOR}

\universitypt{Universidade Federal da Bahia}
\universityen{Federal University of Bahia}

\departmentpt{Depertamento de Ciência da Computação}
\departmenten{Computer Science Department}

\programpt{Programa Multiinstitucional de Pós-graduação em Ciência da Computação}
\programen{Graduate in Computer Science}

\majorfieldpt{Ciência da Computação}
\majorfielden{Computer Science}

\title{Sistema de apoio à Pós graduação - UFBA}
\date{Outubro/2016}

\author{Victor de Azevedo Nunes}
\adviser{Ivan do Carmo Machado}

\begin{document}

\frontmatter
\frontpage
\presentationpage

\begin{dedicatory}
Eu dedico esta dissertação...
%I dedicate this dissertation to my family, girlfriend, friends and
%professors who gave me all necessary support to get here.
\end{dedicatory}

\acknowledgements
Meus agradecimentos...

\begin{epigraph}[]{Edward V Berard}
Walking on water and developing software from a specification are easy if both are frozen
\end{epigraph}

\resumo
% Escreva seu resumo no arquivo resumo.tex
\input{resumo}

\abstract
% Write your abstract in a file called abstract.tex
\input{abstract}

% Summary (tables of contents)
\tableofcontents

% List of figures
\listoffigures

% List of tables
\listoftables

% List of acronyms
% Acronyms manual: http://linorg.usp.br/CTAN/macros/latex/contrib/acronym/acronym.pdf
\listofacronyms
\input{acronyms}

% List of listings
%\lstlistoflistings

\mainmatter

\include{chapters/intro}
\include{chapters/referencial_teorico}

% \include{chapters/introduction/main}
% \include{chapters/background/main}
% \include{chapters/proposed_solution/main}
% \include{chapters/experiment/main}
% \include{chapters/conclusion/main}

%\bibliographystyle{natbib}
%\addcontentsline{toc}{chapter}{\bibliographytocname}
\bibliography{references}

% Appendix
\clearpage
\addappheadtotoc
\appendix
\appendixpage
% \include{appendix/experiment-instruments}

\end{document}
% %% RiSE Latex Template - version 0.5
%%
%% RiSE's latex template for thesis and dissertations
%% http://risetemplate.sourceforge.net
%%
%% (c) 2012 Yguaratã Cerqueira Cavalcanti (yguarata@gmail.com)
%%          Vinicius Cardoso Garcia (vinicius.garcia@gmail.com)
%%
%% This document was initially based on UFPEThesis template, from Paulo Gustavo
%% S. Fonseca.
%%
%% ACKNOWLEDGEMENTS
%%
%% We would like to thanks the RiSE's researchers community, the 
%% students from Federal University of Pernambuco, and other users that have
%% been contributing to this projects with comments and patches.
%%
%% GENERAL INSTRUCTIONS
%%
%% We strongly recommend you to compile your documents using pdflatex command.
%% It is also recommend use the texlipse plugin for Eclipse to edit your documents.
%%
%% Options for \documentclass command:
%%         * Idiom
%%           pt   - Portguese (default)
%%           en   - English
%%
%%         * Text type
%%           bsc  - B.Sc. Thesis
%%           msc  - M.Sc. Thesis (default)
%%           qual - PHD qualification (not tested yet)
%%           prop - PHD proposal (not tested yet)
%%           phd  - PHD thesis
%%
%%         * Media
%%           scr  - to eletronic version (PDF) / see the users guide
%%
%%         * Pagination
%%           oneside - unique face press
%%           twoside - two faces press
%%
%%		   * Line spacing
%%           singlespacing  - the same as using \linespread{1}
%%           onehalfspacing - the same as using \linespread{1.3}
%%           doublespacing  - the same as using \linespread{1.6}
%%
%% Reference commands. Use the following commands to make references in your
%% text:
%%          \figref  -- for Figure reference
%%          \tabref  -- for Table reference
%%          \eqnref  -- for equation reference
%%          \chapref -- for chapter reference
%%          \secref  -- for section reference
%%          \appref  -- for appendix reference
%%          \axiref  -- for axiom reference
%%          \conjref -- for conjecture reference
%%          \defref  -- for definition reference
%%          \lemref  -- for lemma reference
%%          \theoref -- for theorem reference
%%          \corref  -- for corollary reference
%%          \propref -- for proprosition reference
%%          \pgref   -- for page reference
%%
%%          Example: See \chapref{chap:introduction}. It will produce 
%%                   'See Chapter 1', in case of English language.

\documentclass[pt,twoside,onehalfspacing,bsc]{risethesis}

\usepackage[sort,square]{natbib} %package for bibliography
\bibliographystyle{ieeetr} %authordate1

\usepackage[utf8]{inputenc}
\usepackage[brazilian]{babel}
\usepackage[T1]{fontenc}

\usepackage{graphicx}

%% Change the following pdf author attribute name to your name.
\usepackage[linkcolor=blue,citecolor=blue,urlcolor=blue,colorlinks,pdfpagelabels,pdftitle={Bruno Cabral's Bachelor Thesis},pdfauthor={Bruno Cabral}]{hyperref}

\address{SALVADOR}

\universitypt{Universidade Federal da Bahia}
\universityen{Federal University of Bahia}

\departmentpt{Depertamento de Ciência da Computação}
\departmenten{Computer Science Department}

\programpt{Programa Multiinstitucional de Pós-graduação em Ciência da Computação}
\programen{Graduate in Computer Science}

\majorfieldpt{Ciência da Computação}
\majorfielden{Computer Science}

\title{Sistema de apoio à Pós graduação - UFBA}
\date{Outubro/2016}

\author{Victor de Azevedo Nunes}
\adviser{Ivan do Carmo Machado}

\begin{document}

\frontmatter
\frontpage
\presentationpage

\begin{dedicatory}
Eu dedico esta dissertação...
%I dedicate this dissertation to my family, girlfriend, friends and
%professors who gave me all necessary support to get here.
\end{dedicatory}

\acknowledgements
Meus agradecimentos...

\begin{epigraph}[]{Edward V Berard}
Walking on water and developing software from a specification are easy if both are frozen
\end{epigraph}

\resumo
% Escreva seu resumo no arquivo resumo.tex
\input{resumo}

\abstract
% Write your abstract in a file called abstract.tex
\input{abstract}

% Summary (tables of contents)
\tableofcontents

% List of figures
\listoffigures

% List of tables
\listoftables

% List of acronyms
% Acronyms manual: http://linorg.usp.br/CTAN/macros/latex/contrib/acronym/acronym.pdf
\listofacronyms
\input{acronyms}

% List of listings
%\lstlistoflistings

\mainmatter

\include{chapters/intro}
\include{chapters/referencial_teorico}

% \include{chapters/introduction/main}
% \include{chapters/background/main}
% \include{chapters/proposed_solution/main}
% \include{chapters/experiment/main}
% \include{chapters/conclusion/main}

%\bibliographystyle{natbib}
%\addcontentsline{toc}{chapter}{\bibliographytocname}
\bibliography{references}

% Appendix
\clearpage
\addappheadtotoc
\appendix
\appendixpage
% \include{appendix/experiment-instruments}

\end{document}
% %% RiSE Latex Template - version 0.5
%%
%% RiSE's latex template for thesis and dissertations
%% http://risetemplate.sourceforge.net
%%
%% (c) 2012 Yguaratã Cerqueira Cavalcanti (yguarata@gmail.com)
%%          Vinicius Cardoso Garcia (vinicius.garcia@gmail.com)
%%
%% This document was initially based on UFPEThesis template, from Paulo Gustavo
%% S. Fonseca.
%%
%% ACKNOWLEDGEMENTS
%%
%% We would like to thanks the RiSE's researchers community, the 
%% students from Federal University of Pernambuco, and other users that have
%% been contributing to this projects with comments and patches.
%%
%% GENERAL INSTRUCTIONS
%%
%% We strongly recommend you to compile your documents using pdflatex command.
%% It is also recommend use the texlipse plugin for Eclipse to edit your documents.
%%
%% Options for \documentclass command:
%%         * Idiom
%%           pt   - Portguese (default)
%%           en   - English
%%
%%         * Text type
%%           bsc  - B.Sc. Thesis
%%           msc  - M.Sc. Thesis (default)
%%           qual - PHD qualification (not tested yet)
%%           prop - PHD proposal (not tested yet)
%%           phd  - PHD thesis
%%
%%         * Media
%%           scr  - to eletronic version (PDF) / see the users guide
%%
%%         * Pagination
%%           oneside - unique face press
%%           twoside - two faces press
%%
%%		   * Line spacing
%%           singlespacing  - the same as using \linespread{1}
%%           onehalfspacing - the same as using \linespread{1.3}
%%           doublespacing  - the same as using \linespread{1.6}
%%
%% Reference commands. Use the following commands to make references in your
%% text:
%%          \figref  -- for Figure reference
%%          \tabref  -- for Table reference
%%          \eqnref  -- for equation reference
%%          \chapref -- for chapter reference
%%          \secref  -- for section reference
%%          \appref  -- for appendix reference
%%          \axiref  -- for axiom reference
%%          \conjref -- for conjecture reference
%%          \defref  -- for definition reference
%%          \lemref  -- for lemma reference
%%          \theoref -- for theorem reference
%%          \corref  -- for corollary reference
%%          \propref -- for proprosition reference
%%          \pgref   -- for page reference
%%
%%          Example: See \chapref{chap:introduction}. It will produce 
%%                   'See Chapter 1', in case of English language.

\documentclass[pt,twoside,onehalfspacing,bsc]{risethesis}

\usepackage[sort,square]{natbib} %package for bibliography
\bibliographystyle{ieeetr} %authordate1

\usepackage[utf8]{inputenc}
\usepackage[brazilian]{babel}
\usepackage[T1]{fontenc}

\usepackage{graphicx}

%% Change the following pdf author attribute name to your name.
\usepackage[linkcolor=blue,citecolor=blue,urlcolor=blue,colorlinks,pdfpagelabels,pdftitle={Bruno Cabral's Bachelor Thesis},pdfauthor={Bruno Cabral}]{hyperref}

\address{SALVADOR}

\universitypt{Universidade Federal da Bahia}
\universityen{Federal University of Bahia}

\departmentpt{Depertamento de Ciência da Computação}
\departmenten{Computer Science Department}

\programpt{Programa Multiinstitucional de Pós-graduação em Ciência da Computação}
\programen{Graduate in Computer Science}

\majorfieldpt{Ciência da Computação}
\majorfielden{Computer Science}

\title{Sistema de apoio à Pós graduação - UFBA}
\date{Outubro/2016}

\author{Victor de Azevedo Nunes}
\adviser{Ivan do Carmo Machado}

\begin{document}

\frontmatter
\frontpage
\presentationpage

\begin{dedicatory}
Eu dedico esta dissertação...
%I dedicate this dissertation to my family, girlfriend, friends and
%professors who gave me all necessary support to get here.
\end{dedicatory}

\acknowledgements
Meus agradecimentos...

\begin{epigraph}[]{Edward V Berard}
Walking on water and developing software from a specification are easy if both are frozen
\end{epigraph}

\resumo
% Escreva seu resumo no arquivo resumo.tex
\input{resumo}

\abstract
% Write your abstract in a file called abstract.tex
\input{abstract}

% Summary (tables of contents)
\tableofcontents

% List of figures
\listoffigures

% List of tables
\listoftables

% List of acronyms
% Acronyms manual: http://linorg.usp.br/CTAN/macros/latex/contrib/acronym/acronym.pdf
\listofacronyms
\input{acronyms}

% List of listings
%\lstlistoflistings

\mainmatter

\include{chapters/intro}
\include{chapters/referencial_teorico}

% \include{chapters/introduction/main}
% \include{chapters/background/main}
% \include{chapters/proposed_solution/main}
% \include{chapters/experiment/main}
% \include{chapters/conclusion/main}

%\bibliographystyle{natbib}
%\addcontentsline{toc}{chapter}{\bibliographytocname}
\bibliography{references}

% Appendix
\clearpage
\addappheadtotoc
\appendix
\appendixpage
% \include{appendix/experiment-instruments}

\end{document}

%\bibliographystyle{natbib}
%\addcontentsline{toc}{chapter}{\bibliographytocname}
\bibliography{references}

% Appendix
\clearpage
\addappheadtotoc
\appendix
\appendixpage
% \include{appendix/experiment-instruments}

\end{document}
% %% RiSE Latex Template - version 0.5
%%
%% RiSE's latex template for thesis and dissertations
%% http://risetemplate.sourceforge.net
%%
%% (c) 2012 Yguaratã Cerqueira Cavalcanti (yguarata@gmail.com)
%%          Vinicius Cardoso Garcia (vinicius.garcia@gmail.com)
%%
%% This document was initially based on UFPEThesis template, from Paulo Gustavo
%% S. Fonseca.
%%
%% ACKNOWLEDGEMENTS
%%
%% We would like to thanks the RiSE's researchers community, the 
%% students from Federal University of Pernambuco, and other users that have
%% been contributing to this projects with comments and patches.
%%
%% GENERAL INSTRUCTIONS
%%
%% We strongly recommend you to compile your documents using pdflatex command.
%% It is also recommend use the texlipse plugin for Eclipse to edit your documents.
%%
%% Options for \documentclass command:
%%         * Idiom
%%           pt   - Portguese (default)
%%           en   - English
%%
%%         * Text type
%%           bsc  - B.Sc. Thesis
%%           msc  - M.Sc. Thesis (default)
%%           qual - PHD qualification (not tested yet)
%%           prop - PHD proposal (not tested yet)
%%           phd  - PHD thesis
%%
%%         * Media
%%           scr  - to eletronic version (PDF) / see the users guide
%%
%%         * Pagination
%%           oneside - unique face press
%%           twoside - two faces press
%%
%%		   * Line spacing
%%           singlespacing  - the same as using \linespread{1}
%%           onehalfspacing - the same as using \linespread{1.3}
%%           doublespacing  - the same as using \linespread{1.6}
%%
%% Reference commands. Use the following commands to make references in your
%% text:
%%          \figref  -- for Figure reference
%%          \tabref  -- for Table reference
%%          \eqnref  -- for equation reference
%%          \chapref -- for chapter reference
%%          \secref  -- for section reference
%%          \appref  -- for appendix reference
%%          \axiref  -- for axiom reference
%%          \conjref -- for conjecture reference
%%          \defref  -- for definition reference
%%          \lemref  -- for lemma reference
%%          \theoref -- for theorem reference
%%          \corref  -- for corollary reference
%%          \propref -- for proprosition reference
%%          \pgref   -- for page reference
%%
%%          Example: See \chapref{chap:introduction}. It will produce 
%%                   'See Chapter 1', in case of English language.

\documentclass[pt,twoside,onehalfspacing,bsc]{risethesis}

\usepackage[sort,square]{natbib} %package for bibliography
\bibliographystyle{ieeetr} %authordate1

\usepackage[utf8]{inputenc}
\usepackage[brazilian]{babel}
\usepackage[T1]{fontenc}

\usepackage{graphicx}

%% Change the following pdf author attribute name to your name.
\usepackage[linkcolor=blue,citecolor=blue,urlcolor=blue,colorlinks,pdfpagelabels,pdftitle={Bruno Cabral's Bachelor Thesis},pdfauthor={Bruno Cabral}]{hyperref}

\address{SALVADOR}

\universitypt{Universidade Federal da Bahia}
\universityen{Federal University of Bahia}

\departmentpt{Depertamento de Ciência da Computação}
\departmenten{Computer Science Department}

\programpt{Programa Multiinstitucional de Pós-graduação em Ciência da Computação}
\programen{Graduate in Computer Science}

\majorfieldpt{Ciência da Computação}
\majorfielden{Computer Science}

\title{Sistema de apoio à Pós graduação - UFBA}
\date{Outubro/2016}

\author{Victor de Azevedo Nunes}
\adviser{Ivan do Carmo Machado}

\begin{document}

\frontmatter
\frontpage
\presentationpage

\begin{dedicatory}
Eu dedico esta dissertação...
%I dedicate this dissertation to my family, girlfriend, friends and
%professors who gave me all necessary support to get here.
\end{dedicatory}

\acknowledgements
Meus agradecimentos...

\begin{epigraph}[]{Edward V Berard}
Walking on water and developing software from a specification are easy if both are frozen
\end{epigraph}

\resumo
% Escreva seu resumo no arquivo resumo.tex
Este trabalho de conclusão de curso utiliza elementos da engenharia de software para propor um software como serviço a fim de otimizar o processo de construção e manutenção dos softwares. Assim, o objetivo deste SaaS é gerenciar as permissões de acesso de sistemas cliente, provendo o reuso de software e padronizando as soluções. O sistema proposto neste trabalho encontra-se implementado e disponível no Github, e traz fundamentos sobre a arquitetura e tecnologias utilizadas, bem como avaliações sobre possibilidades de utilização do mesmo em diversos ambientes, como Web e mobile.

\begin{keywords}
Software; Reuso; SaaS; Web; PHP
\end{keywords}

\abstract
% Write your abstract in a file called abstract.tex
My abstract...

\begin{keywords}
key words...
\end{keywords}

% Summary (tables of contents)
\tableofcontents

% List of figures
\listoffigures

% List of tables
\listoftables

% List of acronyms
% Acronyms manual: http://linorg.usp.br/CTAN/macros/latex/contrib/acronym/acronym.pdf
\listofacronyms
\begin{acronym}[ACRONYM] 
% Change the word ACRONYM above to change the acronym column width.
% The column width is equals to the width of the word that you put.
% Read the manual about acronym package for more examples:
%   http://linorg.usp.br/CTAN/macros/latex/contrib/acronym/acronym.pdf

\acro{SPA}{Single Page Application}
\acro{JSON}{Javascript Object Notation}
\acro{PHP}{PHP: Hypertext Preprocessor}
\acro{SaaS}{Software as a Service}
\acro{ERP}{Enterprise Resource Planning}
\acro{QoS}{Quality of Service}
\acro{UML}{Unified Modeling Language}
\acro{MVC}{Model-View-Controller}
\acro{Ajax}{Asynchronous Javascript and XML}
\acro{HTML}{HyperText Markup Language}
\acro{CSS}{Cascading Style Sheets}
\acro{API}{Application Programming Interface}
\acro{DOM}{Document Object Model}
\acro{BPMN}{Business Process Model and Notation}
\acro{REST}{Representational State Transfer}

\end{acronym}

% List of listings
%\lstlistoflistings

\mainmatter

\chapter{Introdução}

\section{Motivação}

Organizar os procedimentos de um processo sempre nos traz vantagens. Apesar de no processo de implantação de um sistema, o mesmo burocratizar o processo, com o tempo temos o retorno da dedicação para a inserção dos dados. Com um certo volume de dados, é possível estruturar informações que num processo manual são difíceis de serem enxergadas. Assim, é possível depender menos das pessoas que organizam o processo, pois o legado de informações não estará mais somente na mente de alguns, mas sim documentado nos dados do sistema.

Além de colaborar na organização, também haverá uma grande colaboração no tempo gasto na gestão. Lidar com muitos papéis e confiar na mente humana para guardar informações, não é uma alternativa muito segura devido ao fato que as pessoas sempre estão sujeitas a sair do processo e levar contigo a experiência obtida. Experiência essa que faz com que os procedimentos sejam executados de forma mais eficiente. Entretanto, com um sistema inteligente, é possível auxiliar e tornar mais ágil a execução das tarefas.


\section{Problema}


De acordo com funcionários ligados ao o setor de pós graduação da UFBA, entrevistados a fim de um maior entendimento do cenário, apesar das semelhanças estruturais, a pós graduação gerida de forma diferencia da graduação. FULANO afirma que devido ao fato de não ter a mesma visibilidade, não tem acesso aos mesmos recursos de gestão acadêmica da graduação. O professores não executam somente atividades dentro da sala de aula, também tem diversas outras ocupações no setor. E muitos procedimentos realizados extra classe ainda se encontram sendo realizados de forma manual, estando mais vulnerável ao erro ou até mesmo à violação do processo. Também ocorre um grande desperdício de tempo pelos professores e gestores da área, devido ao diversos processos ainda realizados de forma manual, sem a devida documentação. Segundo FULANO, também entrevistado, esse tempo perdido implica numa redução da eficiência na sala de aula, pois o professor acaba por ter menos tempo disponível para o planejamento das atividades, o que gera impactos negativos aos alunos.


\section{Objetivos} %<o que deve ser feito/entregue>


Devido aos muitos processos sendo resolvidos de forma manual, propõe-se com solução um sistema moderno, arquitetado para ter funcionamento na web e com um módulo mobile, a fim de fornecer informações de forma rápida e eficiente para os professores através de notificações, já que o acesso à internet móvel é comum entre os possíveis usuários do sistema em questão.
O principal requisito para o sistema seria dispor recursos para reduzir o tempo desperdiçado pelos professores durante as atividades extra classe.


\section{Metodologia} %<como será feito | como resolver o problema apontado inicialmente>


%<analise de literatura | design | implementação | validação>
Baseando-se nas tecnologias gratuitas em alta no cenário atual do desenvolvimento web, dispomos de algumas opções eficientes para a implementação da solução. Dentre as possibilidades, considerando a facilidade para futura manutenção e continuidade do projeto, tende-se a optar por uma tecnologia popular. Como linguagem de programação, adota-se o PHP. A escolha é fundamentada de acordo com a pesquisa da RedMonk de 2015, que evidencia o uso das linguagens de programação de acordo com as discussões no StackOverflow e repositórios no GitHub. É possível constatar a popularidade do PHP no cenário atual com o gráfico da pesquisa citada, na qual o PHP é apresentado na terceira colocação, apenas atrás do lider JavaScript e do segundo colocado, o Java.

\begin{figure}
	\label{fig:graficoRedmonk}
	\includegraphics[width=1\textwidth]{img/grafico_redmonk}
	\caption{Ranking das liguagens de programação no Stack Overflow e Github}
\end{figure}


Ainda assim, para compor a interface do dado projeto, também ocorrerá o uso do líder JavaScript de forma intensa, provendo o elo com o as informações gerenciadas pelo PHP.


Entretanto, não seria inteligente desenvolver um sistema completo sem o auxílio de um framework. Dentre os frameworks disponíveis para PHP, hoje o destaque está com o Laravel, que se encontra no topo dentre os mais utilizados no momento. 


A WebHostFace, uma empresa de hospedagem, compilou várias estatísticas para criar um infográfico mostrando os frameworks PHP mais populares de 2015. Utilizando informações sobre os próprios clientes, o Google Trends, estatísticas de repositórios do GitHub e a pesquisa do SitePoint “Best PHP Frameworks 2015”, a WebHostFace elaborou o seguinte infográfico: 

\begin{figure}
	\label{fig:graficoWebhostface}
	\includegraphics[width=1\textwidth]{img/infografico_webhostface}
	\caption{Infográfico da WebhostFace, exibindo a popularidade dos Frameworks PHP em 2015}
\end{figure}

Assim, tem-se a evidência que o Laravel em 2015 teve a maior popularidade em projetos pessoais e tem a maior comunidade entre os concorrentes, o que o torna uma boa escolha para a escrita de um software que será continuado por terceiros.


Para elaborar os recursos de interface e integrar ao back-end PHP do sistema, será adotado o já conhecido AngularJS, ferramenta sólida e conhecida no aspecto em questão. 


Dados coletados via Google Trends, que propõe comparações entre termos pesquisados, revela a popularidade do AngularJs diante de alguns dos principais concorrentes. O gráfico abaixo evidencia o cenário.


%Como mostra a Figura \ref{fig:graficoGoogleTrendsFerramentasFront}. 
\begin{figure}
	\label{fig:graficoGoogleTrendsFerramentasFront}
	\includegraphics[width=1\textwidth]{img/grafico_ferramentas_front}
	\caption{Gráfico do Google Trends exibindo as pesquisas por ferramentas front-end}
\end{figure}


Junto ao Angular JS, será utilizada a agradável tendência de interface do Material Design da Google, que propõe layouts limpos e otimizados já conhecidos pelos usuários de smartphones Android. 


Para a elaboração da plataforma mobile do projeto, será utilizado o Ionic Framework, muito difundido e bastante pesquisado na área, o que fica evidenciado com o gráfico de pesquisbaixo, coletado via Google Trends buscando por frameworks de desenvolvimento híbrido mobile.


\begin{figure}
	\label{fig:graficoGoogleTrendsFerramentasHibridasMobile}
	\includegraphics[width=1\textwidth]{img/grafico_ferramentas_hibridas_mobile}
	\caption{Gráfico do Google Trends exibindo as pesquisas por Frameworks híbridos mobile}
\end{figure}	

Para layout da interface mobile, também será aplicado a tendência do Material Design, a fim de propor uma harmonia entre o módulo web e mobile para os usuários


\section{Resultados Esperados}


Como fruto de um sistema para pós-graduação da UFBA, espera-se que os professores tenham mais recursos para integrar as atividades e também prover melhores condições para acompanhamento da vida acadêmica dos alunos em questão. Também, que os novos colaboradores que entrarem no processo tenham facilidade de compreender o fluxo do setor ao navegar pelo sistema proposto.


\section{Fora de Escopo}


Interação com os alunos devido às complicações para realizar a integração com o sistema empregado na UFBA, gerenciado pela XXXXXX, o que causaria uma inviabilidade no projeto devido à necessidade de entrega do produto ser mais forte que o tempo necessário para executar o processo de obtenção de acesso ao sistema legado para realizar a integração.


\section{Estrutura do Trabalho}


<breve resumo sobre os capítulos do TCC>
\chapter{Referencial Teórico}


Projetar o desenvolvimento de um software requer muito planejamento, pois as falhas iniciais podem custar bastante caro ou até mesmo inviabilizar a continuação de um projeto. Assim, a escolha da arquitetura ideal para a aplicabilidade é essencial na concepção de um produto de software. 
De todo o modo, sempre busca-se fazer mais com menos. Diante de tal filosofia, temos neste capítulo, uma breve discussão sobre alguns elementos de projeto e arquitetura de software, a fim de contextualizar este trabalho de conclusão de curso.
O capítulo corrente é composto por quatro seções. A \ref{sec:saas} trata de Software como serviço, discutindo alguns elementos do contexto que são relevantes para o trabalho proposto. A \ref{sec:reuso}  discute sobre a empregabilidade do reuso de software. A \ref{sec:modularizacao} seção, trata sobre aspectos envolvidos na modularização dos softwares. Por fim, a \ref{sec:apps_web} aborda as aplicações web, discutindo sobre aspectos relevantes sobre a aplicação web que compõe este trabalho.


 \section{Software como serviço}\label{sec:saas}


Segundo La e Chun \citep{La2009Systematic}, o princípio da definição de Software como um Serviço (Sofware as a Service - SaaS) é um serviço complementar para aplicações da computação em nuvem (cloud computing). No entanto, as áreas não se confundem. SaaS deve ser entendido como um mecanismo de suporte às soluções existentes na cloud. Os SaaS existem justamente para maximizar o reuso de serviços repetidos e não centrais em uma aplicação remota.


Como vantagens, diversos fatores podem ser favoráveis para a adoção de um SaaS, como custo e manutenção dentre outros fatores aplicáveis a soluções específicas. Lechesaet al. \citep{LechesaSS11}, quantificam índices sobre os fatores determinantes para adoção ou não de um SaaS voltado para ERP na África do Sul. Os principais fatores determinantes para adoção desse mecanismo de software são sua fluidez quanto à rede e a segurança. Esses fatores estão presentes na aplicação desenvolvida neste trabalho de conclusão de curso.


Devido ao fato de ter um serviço constantemente na nuvem, fica o questionamento sobre a segurança da informação manipulada. Sabe-se que a vulnerabilidade na Web não é restrita ao SaaS, atingindo diversos âmbitos. Rai et al. \cite{journals/corr/RaiSM13} consideram que o avanço da computação em nuvem não é um problema apenas para os serviços Web do ponto de vista da segurança, pois muitos trabalhos na literatura mostram a área como mais um ponto de vulnerabilidade para diversos setores, a exemplo de infraestrutura. No mesmo artigo mencionado de Rai et al. \cite{journals/corr/RaiSM13}, também realizaram-se estudos exploratórios junto a empresas usuárias de serviços em computação em nuvem e consideram que a perspectiva de SaaS também pode fortalecer a segurança nas aplicações de cloud computing, pois o software de autenticação compartilhado por várias aplicações em nuvem, oferece uma melhor padronização e consequente facilidade de prevenção a erros de vulnerabilidade específicas de cada módulo da pesquisa. Esse ponto de vista é fundamenta para a compreensão de SaaS.


A arquitetura de armazenamento de dados de um Saas pode variar de acordo com a necessidade do contexto. Huixin \cite{7586486} descreve possíveis modelagens nesse sentido. Essa abordagem pode ser com um banco de dados único, fazendo com que diferentes clientes compartilhem o mesmo banco, diferindo os dados através de controle de usuário, ou isolando os diferentes clientes através de bancos de dados exclusivos para cada um. Esse fator também pode ser combinado com a arquitetura da aplicação, caso ofereça aplicação única para todos os clientes ou aplicação compartilhada. Diante das possíveis abordagens, a modelagem de dados do software pode ser decidida pela regra de negócio. Este trabalho optou por aplicação única e banco de dados compartilhado.

Devido ao conceito particular de obtenção de software, tanto pela visão do cliente como pela visão do vendedor, é necessário compreender os diversos aspectos que podem ser relevantes ao orçar um Saas. O recente trabalho de T. Kaur et al. \citep{6949281} orienta um modelo para compor o custo de um Saas. O custo total seria composto pelos fatores que dão suporte ao funcionamento do software. Tais fatores incluem infra-estrutura, configurabilidade, customização, parâmetros de QoS (Quality of service) como escalabilidade, disponibilidade, usabilidade, pontualidade e desempenho da resposta, portabilidade, custo total de propriedade e retorno do investimento. Esses fatores caracterizam o custo de forma eficaz, possibilitando ao fornecedor, prover um Serviço de acordo com a exigência do consumidor em vários pacotes de serviços.


O conceito de software como serviço se aplica neste trabalho de conclusão de curso, pois o mesmo estará disponível na web com alta disponibilidade adotando as características apresentadas para qualquer pessoa que desejar usá-lo. Assim, basta se cadastrar, configurá-lo e usar como um serviço, sem preocupação com a manutenção do mesmo.


\section{Reuso de software}\label{sec:reuso} %CRUISE BOOK CAPITULO 2


De acordo com o livro Software pratical reuse \citep{Ezran2002}, o reuso de software é a utilização de qualquer informação que um desenvolvedor pode necessitar no processo de criação de software. O livro de Basili e Rombach \citep{Basili1991} define reutilização de software como o uso de tudo o que está associado a um projeto de conhecimento.
Assim, o objetivo da reutilização de software é reciclar o design, código e outros componentes de um produto de software e assim reduzir o custo, o tempo e melhorar a qualidade do produto.
Segundo Keswani et al. \cite{6783445}, o componente reutilizável de software pode ser qualquer parte de seu desenvolvimento, como um fragmento de código, design, casos de teste, ou até mesmo a especificação de requisitos de uma funcionalidade do software. 

O reuso de software pode ter impacto positivo em diversos aspectos do software, vejamos alguns, conforme apresentados no C.R.U.I.S.E Book \citep{cruise2007} :

\begin{itemize}

\item Qualidade: As correções de erro tornam-se úteis em todos os locais em que ocorreu, padronizando e facilitando a manutenção.

\item Produtividade: O ganho de produtividade é alcançado devido ao menor número de artefatos desenvolvido. Isso resulta em menor esforço de teste e também análise e design, reduzindo custos.

\item Confiabilidade: A utilização de componentes bem testados aumenta a
confiança no software. Além disso, a utilização de um mesmo componente em vários sistemas, aumenta a possibilidade de detecção de erros e reforça a confiança no componente.

\item Redução do Esforço: A reutilização de software proporciona uma redução do tempo de desenvolvimento, o que reduz o tempo necessário para o produto ser disponibilizado no mercado para trazer rentabilidade.

\item Trabalho redundante e tempo de desenvolvimento: Desenvolver um sistema do
zero significa desenvolvimento redundante de muitos componentes, como requisitos, especificações, casos de uso, arquitetura, etc. Isso pode ser evitado quando estes estão disponíveis como componentes reutilizáveis e podem ser compartilhados, resultando em um processo de desenvolvimento otimizado.

\item Documentação: Embora a documentação seja muito importante para a
manutenção de um sistema, muitas vezes é negligenciada. A reutilização de componentes de software reduz a quantidade de documentação a ser escrita, entretanto depende da qualidade do que está escrito. Assim, apenas a estrutura do sistema e os novos artefatos desenvolvidos necessitam ser documentados.

\item Custo de manutenção: Menos defeitos e manutenções são esperados quando tem-se comprovada a qualidade dos componentes utilizados.

\item Tamanho da equipe: É comum haver casos em que a equipe de desenvolvimento sofre sobrecarga. Entretanto, dobrar o tamanho da equipe de desenvolvimento não necessariamente duplica produtividade. Se muitos componentes podem ser reutilizados, é possível desenvolver com equipes menores, levando a melhor comunicação e aumento da produtividade.

\end{itemize}

Apesar dos benefícios da reutilização de software, ela não é suficientemente aproveitada. Existem fatores que influenciam direta ou indiretamente na sua adoção. Esses fatores podem ser de aspecto gerencial, organizacional, econômico, conceitual ou técnico. Veremos a seguir alguns aspectos que podem gerar conflito com a cultura de reuso de software, segundo o C.R.U.I.S.E Book \citep{cruise2007} :
%(Sametinger, 1997). REVER

\begin{itemize}
	
\item Falta de apoio da gestão: Como a reutilização de software gera custos iniciais,
a medida pode não ser amplamente alcançada em uma organização sem o apoio de alto nível de gestão. Os gestores têm de ser informados sobre os custos iniciais e serem convencidos sobre economias futuras.

\item Gerenciamento do Projeto: Gerenciar projetos tradicionais é uma tarefa árdua, principalmente, os que praticam a reutilização de software. Utilizando a técnica em larga escala, tem-se impacto sobre todo o ciclo de vida do software.

\item Estruturas organizacionais inadequadas: As estruturas organizacionais devem
considerar diferentes necessidades que surgem quando a reutilização em larga escala está sendo adotada. Por exemplo, uma equipe particionada pode ser alocada somente para desenvolver, manter e certificar componentes reutilizáveis de software.

\item Incentivos de gestão: É comum a falta de incentivo para deixar os desenvolvedores gastarem tempo elaborando componentes do sistemas. A produtividade é muitas vezes medida apenas no tempo necessário para concluir um projeto. Assim, fazer qualquer trabalho além disso, embora benéfico para a empresa como um todo, diminui o seu sucesso. Mesmo quando os componentes reutilizáveis são utilizados, os benefícios obtidos são uma pequena fração do que poderia ser alcançado caso houvesse reutilização explícita, planejada e organizada.

\item Dificuldade de encontrar software reutilizável: Para reutilizar os componentes, devem existir formas eficientes de busca. Além disso, é importante ter um repositório bem organizado contendo componentes com um eficiente meio de acesso.

\item Não reutilização do software encontrado. O acesso fácil ao software existente
não necessariamente aumentar a reutilização. Os componentes reutilizáveis devem ser cuidadosamente especificados, projetados, implementados e documentados, pois em alguns casos, modificar e adaptar o código  pode ser mais custoso que a programação da funcionalidade necessária a partir do zero.

\item Modificação: É muito difícil encontrar um componente que funcione
exatamente da mesma maneira que queremos. Desta forma, são necessárias modificações e devem existir formas de determinar os seus efeitos sobre o componente.


\end{itemize}


A definição do reuso de software pode ser associada a este trabalho, pois ainda que como um serviço, estaremos adotando uma aplicação única para servir aos mais diversos softwares de uma empresa, por exemplo. Logo, ao adotar o projeto, pode ser interessante aplicar algumas métricas para tomar conhecimento de possíveis vantanjens como as citadas nessa seção, a exemplo de economia de tempo de desenvolvimento, e custo do projeto.


%Outra diretriz importante para a reutilização de software é reduzir o risco na criação de novos softwares. O risco tende a ser bastante reduzido se os componentes que estão sendo reutilizados têm as documentação, interfaces necessárias e devidamente testadas, fatores que contibruem para uma fácil integração.
%De acordo com Keswani et al. \citep{6783445}, para o reuso de software dar retornos apropriados, o processo deve ser sistemático e planejado. Qualquer organização que implemente a reutilização de software deve identificar os melhores métodos e estratégias de reutilização para obter a máxima produtividade. A reutilização de software ajuda a evitar software de engenharia a partir do zero, pois usa módulos de software existentes. A reutilização de software, embora seja uma tarefa difícil, especialmente para softwares antigos sem padrões de projeto, pode melhorar significativamente a produtividade e a qualidade de um produto de software. Embora a reutilização de software não seja um novo campo, ela pode dar grandes retornos em curto período de tempo.


\section{Modularização}\label{sec:modularizacao} %artigo de claudio pagina 222 introdução


%A modularidade vem desempenhando um papel predominante estágios emergentes das disciplinas de arquitetura de software [13]. Engenheiros de software consideram modularidade como princípio base na comparação entre arquiteturas alternativas  e arquitetura degeneração [9]. De fato, os engenheiros de software são incentivados a arquitecturas, baseando-se numa multiplicidade de mecanismos de modularidade disponíveis em: 
%(i) Linguagens de descrição de arquitetura (ADLs), como ACME [8], 
%(ii) catálogos de arquitetônicos [2, 13], e 
%(iii) conhecem bem princípios de alto nível, como interfaces de componentes estreitos, acoplamento arquitectónico reduzido e semelhantes.


Conforme é frisado por Wickramaarachchi e Lai \citep{7062705}, o conceito de modularização na indústria de software tem uma longa história e tem sido utilizado para melhorar o processo de desenvolvimento de software em diferentes estágios. Os principais conceitos por trás da modularização do software foram introduzidos por pesquisadores pioneiros há quarenta anos, com uma notável contribuição feita por Melvin Conway e David Parnas, que tem representação notável na engenharia de software.


Modularizar um software é um bom padrão a ser adotado. Segundo Wickramaarachchi e Lai \citep{7062705}, a modularização é importante na identificação de dependências e reduz as dificuldades diante de uma possível necessidade de grandes alterações. De uma perspectiva da engenharia de software, uma modularização geralmente tem várias vantagens, tais como: tornar a complexidade do software mais gerenciável, facilitar o trabalho paralelo e tornar o software mais maleável para acomodar o futuro incerto que um software pode ter. O objetivo final da modularização do software é aumentar a produtividade ea qualidade do software. Tal conceito encontra-se bastante difundido e estái incorporado em linguagens de programação e ferramentas de software. O presente trabalho favorece ao uso da modularização de um software e até mesmo pode ser considerado um módulo a ser acoplado a qualquer software, mediante a compatibilidade.


Ao realizar a adoção do trabalho proposto, fica bastante evidente a "responsabilidade" do mesmo no escopo do projeto. Assim, é possível tratá-lo como um módulo do projeto que o usa, mesmo que seja consumido como um serviço.


\section{Aplicações web}\label{sec:apps_web}


A popularidade das soluções Web aumentou exponencialmente na última década e todos os dias cresce o número de pessoas usuárias desse tipo de software. E seguindo um padrão próprio, Kumar et al. \citep{7813710} sugerem que para o desenvolvimento web, deve-se manter a prática eficaz de produzir diagramas UML. A abordagem baseada na web oferece uma maneira fácil e eficaz para gerenciar e controlar o processo de desenvolvimento por meio de artefatos de modelagem. Tal abordagem pode ser usada quando há uma exigência de lidar com mudanças muito rápidas e grandes em requisitos de forma muito eficaz em muito menos tempo, gerando assim um menor impacto. 


Para atender à fomentada demanda de aplicativos web, é necessário adotar métodos de desenvolvimentos que sejam ágeis, eficientes e de fácil manutenção. Yu Ping et al. \cite{1372143} propõem o uso do modelo MVC (Model, View e Controller) no desenvolvimento para softwares web. O modelo apresentado tornou-se um padrão popular e divide o software em camadas com propósito definido, tornando-o de mais fácil manutenção.


O Ajax (Asynchronous Javascript and XML) revolucionou a web. Conforme demonstrado por Yuping \citep{6845605}, ao usar a tecnologia Ajax, podemos enriquecer a experiência do usuário em aplicações baseadas em navegador de internet, e fornecer uma variedade de aplicações interativas para atender às necessidade de humanização das aplicações.
Os aplicativos Ajax em execução no navegador se comunicam com um servidor Web de forma assíncrona e atualizam apenas uma parte da página.


No artigo de Tesarik et al. \citep{4664349}, temos que o design de software SPA (Single page application) é uma maneira interessante de criar um software Web numa uma única página. Essa solução de página única sem navegação funciona apenas com base em técnicas dinâmicas e assíncronas, como o AJAX (citado no parágrafo anterior). No entanto, esta abordagem coloca o desenvolvedor antes de alguns desafios substanciais.
Para projetar a interface do usuário que mostra as informações, é importante projetar corretamente a tela para manipular os dados do aplicativo numa única página. O design da página deve ser elaborado para maximizar a decomposição da página em componentes distintos que encapsulam os principais casos de uso.
Também recomenda-se que para elaborar os artefatos visuais, explore os mais ricos recursos na implementação, como HTML5, JavaScript, Ajax, CSS3, e outras tecnologias que se apliquem. Entretanto, desenvolver uma rica interface com o uso de diversas tecnologias/frameworks pode ocasionar um esforço maior para explorar as possibilidades.
Tais componentes da interface de uma aplicação SPA normalmente são alimentados mediante o consumo de uma API Rest via requisições AJAX. Assim, é possível particionar as responsabilidades de processamento do software entre cliente e servidor.


Como é possível denotar no trabalho de Salvadori e Siqueire \citep{7195633}, REST é uma arquitetura muito popular para integração de aplicativos web, que permite compartilhar e reutilizar informações através de sistemas. Aplicações de grande escala baseadas em no REST deve ser implementadas utilizando estratégias e mecanismos para produzir sistemas que sejam fáceis de desenvolver, reutilizar e manter. As interfaces de integração fornecidas por esses sistemas, chamadas de Web API, têm uma influência importante nas características da implementação resultante, pois a sua forma de resposta deve ser casada com a interface que o consome.
 
 
Os elementos comentados nessa seção encontram-se presentes neste trabalho de conclusão de curso. Juntos, montam a estrutura tecnológica necessária aliada à arquitetura adotada. Esses elementos seguem tendências atuais dos softwares web, fazendo com que esse trabalho esteja composto por tecnologias modernas que estão com boa aceitação no mercado.

% %% RiSE Latex Template - version 0.5
%%
%% RiSE's latex template for thesis and dissertations
%% http://risetemplate.sourceforge.net
%%
%% (c) 2012 Yguaratã Cerqueira Cavalcanti (yguarata@gmail.com)
%%          Vinicius Cardoso Garcia (vinicius.garcia@gmail.com)
%%
%% This document was initially based on UFPEThesis template, from Paulo Gustavo
%% S. Fonseca.
%%
%% ACKNOWLEDGEMENTS
%%
%% We would like to thanks the RiSE's researchers community, the 
%% students from Federal University of Pernambuco, and other users that have
%% been contributing to this projects with comments and patches.
%%
%% GENERAL INSTRUCTIONS
%%
%% We strongly recommend you to compile your documents using pdflatex command.
%% It is also recommend use the texlipse plugin for Eclipse to edit your documents.
%%
%% Options for \documentclass command:
%%         * Idiom
%%           pt   - Portguese (default)
%%           en   - English
%%
%%         * Text type
%%           bsc  - B.Sc. Thesis
%%           msc  - M.Sc. Thesis (default)
%%           qual - PHD qualification (not tested yet)
%%           prop - PHD proposal (not tested yet)
%%           phd  - PHD thesis
%%
%%         * Media
%%           scr  - to eletronic version (PDF) / see the users guide
%%
%%         * Pagination
%%           oneside - unique face press
%%           twoside - two faces press
%%
%%		   * Line spacing
%%           singlespacing  - the same as using \linespread{1}
%%           onehalfspacing - the same as using \linespread{1.3}
%%           doublespacing  - the same as using \linespread{1.6}
%%
%% Reference commands. Use the following commands to make references in your
%% text:
%%          \figref  -- for Figure reference
%%          \tabref  -- for Table reference
%%          \eqnref  -- for equation reference
%%          \chapref -- for chapter reference
%%          \secref  -- for section reference
%%          \appref  -- for appendix reference
%%          \axiref  -- for axiom reference
%%          \conjref -- for conjecture reference
%%          \defref  -- for definition reference
%%          \lemref  -- for lemma reference
%%          \theoref -- for theorem reference
%%          \corref  -- for corollary reference
%%          \propref -- for proprosition reference
%%          \pgref   -- for page reference
%%
%%          Example: See \chapref{chap:introduction}. It will produce 
%%                   'See Chapter 1', in case of English language.

\documentclass[pt,twoside,onehalfspacing,bsc]{risethesis}

\usepackage[sort,square]{natbib} %package for bibliography
\bibliographystyle{ieeetr} %authordate1

\usepackage[utf8]{inputenc}
\usepackage[brazilian]{babel}
\usepackage[T1]{fontenc}

\usepackage{graphicx}

%% Change the following pdf author attribute name to your name.
\usepackage[linkcolor=blue,citecolor=blue,urlcolor=blue,colorlinks,pdfpagelabels,pdftitle={Bruno Cabral's Bachelor Thesis},pdfauthor={Bruno Cabral}]{hyperref}

\address{SALVADOR}

\universitypt{Universidade Federal da Bahia}
\universityen{Federal University of Bahia}

\departmentpt{Depertamento de Ciência da Computação}
\departmenten{Computer Science Department}

\programpt{Programa Multiinstitucional de Pós-graduação em Ciência da Computação}
\programen{Graduate in Computer Science}

\majorfieldpt{Ciência da Computação}
\majorfielden{Computer Science}

\title{Sistema de apoio à Pós graduação - UFBA}
\date{Outubro/2016}

\author{Victor de Azevedo Nunes}
\adviser{Ivan do Carmo Machado}

\begin{document}

\frontmatter
\frontpage
\presentationpage

\begin{dedicatory}
Eu dedico esta dissertação...
%I dedicate this dissertation to my family, girlfriend, friends and
%professors who gave me all necessary support to get here.
\end{dedicatory}

\acknowledgements
Meus agradecimentos...

\begin{epigraph}[]{Edward V Berard}
Walking on water and developing software from a specification are easy if both are frozen
\end{epigraph}

\resumo
% Escreva seu resumo no arquivo resumo.tex
\input{resumo}

\abstract
% Write your abstract in a file called abstract.tex
\input{abstract}

% Summary (tables of contents)
\tableofcontents

% List of figures
\listoffigures

% List of tables
\listoftables

% List of acronyms
% Acronyms manual: http://linorg.usp.br/CTAN/macros/latex/contrib/acronym/acronym.pdf
\listofacronyms
\input{acronyms}

% List of listings
%\lstlistoflistings

\mainmatter

\include{chapters/intro}
\include{chapters/referencial_teorico}

% \include{chapters/introduction/main}
% \include{chapters/background/main}
% \include{chapters/proposed_solution/main}
% \include{chapters/experiment/main}
% \include{chapters/conclusion/main}

%\bibliographystyle{natbib}
%\addcontentsline{toc}{chapter}{\bibliographytocname}
\bibliography{references}

% Appendix
\clearpage
\addappheadtotoc
\appendix
\appendixpage
% \include{appendix/experiment-instruments}

\end{document}
% %% RiSE Latex Template - version 0.5
%%
%% RiSE's latex template for thesis and dissertations
%% http://risetemplate.sourceforge.net
%%
%% (c) 2012 Yguaratã Cerqueira Cavalcanti (yguarata@gmail.com)
%%          Vinicius Cardoso Garcia (vinicius.garcia@gmail.com)
%%
%% This document was initially based on UFPEThesis template, from Paulo Gustavo
%% S. Fonseca.
%%
%% ACKNOWLEDGEMENTS
%%
%% We would like to thanks the RiSE's researchers community, the 
%% students from Federal University of Pernambuco, and other users that have
%% been contributing to this projects with comments and patches.
%%
%% GENERAL INSTRUCTIONS
%%
%% We strongly recommend you to compile your documents using pdflatex command.
%% It is also recommend use the texlipse plugin for Eclipse to edit your documents.
%%
%% Options for \documentclass command:
%%         * Idiom
%%           pt   - Portguese (default)
%%           en   - English
%%
%%         * Text type
%%           bsc  - B.Sc. Thesis
%%           msc  - M.Sc. Thesis (default)
%%           qual - PHD qualification (not tested yet)
%%           prop - PHD proposal (not tested yet)
%%           phd  - PHD thesis
%%
%%         * Media
%%           scr  - to eletronic version (PDF) / see the users guide
%%
%%         * Pagination
%%           oneside - unique face press
%%           twoside - two faces press
%%
%%		   * Line spacing
%%           singlespacing  - the same as using \linespread{1}
%%           onehalfspacing - the same as using \linespread{1.3}
%%           doublespacing  - the same as using \linespread{1.6}
%%
%% Reference commands. Use the following commands to make references in your
%% text:
%%          \figref  -- for Figure reference
%%          \tabref  -- for Table reference
%%          \eqnref  -- for equation reference
%%          \chapref -- for chapter reference
%%          \secref  -- for section reference
%%          \appref  -- for appendix reference
%%          \axiref  -- for axiom reference
%%          \conjref -- for conjecture reference
%%          \defref  -- for definition reference
%%          \lemref  -- for lemma reference
%%          \theoref -- for theorem reference
%%          \corref  -- for corollary reference
%%          \propref -- for proprosition reference
%%          \pgref   -- for page reference
%%
%%          Example: See \chapref{chap:introduction}. It will produce 
%%                   'See Chapter 1', in case of English language.

\documentclass[pt,twoside,onehalfspacing,bsc]{risethesis}

\usepackage[sort,square]{natbib} %package for bibliography
\bibliographystyle{ieeetr} %authordate1

\usepackage[utf8]{inputenc}
\usepackage[brazilian]{babel}
\usepackage[T1]{fontenc}

\usepackage{graphicx}

%% Change the following pdf author attribute name to your name.
\usepackage[linkcolor=blue,citecolor=blue,urlcolor=blue,colorlinks,pdfpagelabels,pdftitle={Bruno Cabral's Bachelor Thesis},pdfauthor={Bruno Cabral}]{hyperref}

\address{SALVADOR}

\universitypt{Universidade Federal da Bahia}
\universityen{Federal University of Bahia}

\departmentpt{Depertamento de Ciência da Computação}
\departmenten{Computer Science Department}

\programpt{Programa Multiinstitucional de Pós-graduação em Ciência da Computação}
\programen{Graduate in Computer Science}

\majorfieldpt{Ciência da Computação}
\majorfielden{Computer Science}

\title{Sistema de apoio à Pós graduação - UFBA}
\date{Outubro/2016}

\author{Victor de Azevedo Nunes}
\adviser{Ivan do Carmo Machado}

\begin{document}

\frontmatter
\frontpage
\presentationpage

\begin{dedicatory}
Eu dedico esta dissertação...
%I dedicate this dissertation to my family, girlfriend, friends and
%professors who gave me all necessary support to get here.
\end{dedicatory}

\acknowledgements
Meus agradecimentos...

\begin{epigraph}[]{Edward V Berard}
Walking on water and developing software from a specification are easy if both are frozen
\end{epigraph}

\resumo
% Escreva seu resumo no arquivo resumo.tex
\input{resumo}

\abstract
% Write your abstract in a file called abstract.tex
\input{abstract}

% Summary (tables of contents)
\tableofcontents

% List of figures
\listoffigures

% List of tables
\listoftables

% List of acronyms
% Acronyms manual: http://linorg.usp.br/CTAN/macros/latex/contrib/acronym/acronym.pdf
\listofacronyms
\input{acronyms}

% List of listings
%\lstlistoflistings

\mainmatter

\include{chapters/intro}
\include{chapters/referencial_teorico}

% \include{chapters/introduction/main}
% \include{chapters/background/main}
% \include{chapters/proposed_solution/main}
% \include{chapters/experiment/main}
% \include{chapters/conclusion/main}

%\bibliographystyle{natbib}
%\addcontentsline{toc}{chapter}{\bibliographytocname}
\bibliography{references}

% Appendix
\clearpage
\addappheadtotoc
\appendix
\appendixpage
% \include{appendix/experiment-instruments}

\end{document}
% %% RiSE Latex Template - version 0.5
%%
%% RiSE's latex template for thesis and dissertations
%% http://risetemplate.sourceforge.net
%%
%% (c) 2012 Yguaratã Cerqueira Cavalcanti (yguarata@gmail.com)
%%          Vinicius Cardoso Garcia (vinicius.garcia@gmail.com)
%%
%% This document was initially based on UFPEThesis template, from Paulo Gustavo
%% S. Fonseca.
%%
%% ACKNOWLEDGEMENTS
%%
%% We would like to thanks the RiSE's researchers community, the 
%% students from Federal University of Pernambuco, and other users that have
%% been contributing to this projects with comments and patches.
%%
%% GENERAL INSTRUCTIONS
%%
%% We strongly recommend you to compile your documents using pdflatex command.
%% It is also recommend use the texlipse plugin for Eclipse to edit your documents.
%%
%% Options for \documentclass command:
%%         * Idiom
%%           pt   - Portguese (default)
%%           en   - English
%%
%%         * Text type
%%           bsc  - B.Sc. Thesis
%%           msc  - M.Sc. Thesis (default)
%%           qual - PHD qualification (not tested yet)
%%           prop - PHD proposal (not tested yet)
%%           phd  - PHD thesis
%%
%%         * Media
%%           scr  - to eletronic version (PDF) / see the users guide
%%
%%         * Pagination
%%           oneside - unique face press
%%           twoside - two faces press
%%
%%		   * Line spacing
%%           singlespacing  - the same as using \linespread{1}
%%           onehalfspacing - the same as using \linespread{1.3}
%%           doublespacing  - the same as using \linespread{1.6}
%%
%% Reference commands. Use the following commands to make references in your
%% text:
%%          \figref  -- for Figure reference
%%          \tabref  -- for Table reference
%%          \eqnref  -- for equation reference
%%          \chapref -- for chapter reference
%%          \secref  -- for section reference
%%          \appref  -- for appendix reference
%%          \axiref  -- for axiom reference
%%          \conjref -- for conjecture reference
%%          \defref  -- for definition reference
%%          \lemref  -- for lemma reference
%%          \theoref -- for theorem reference
%%          \corref  -- for corollary reference
%%          \propref -- for proprosition reference
%%          \pgref   -- for page reference
%%
%%          Example: See \chapref{chap:introduction}. It will produce 
%%                   'See Chapter 1', in case of English language.

\documentclass[pt,twoside,onehalfspacing,bsc]{risethesis}

\usepackage[sort,square]{natbib} %package for bibliography
\bibliographystyle{ieeetr} %authordate1

\usepackage[utf8]{inputenc}
\usepackage[brazilian]{babel}
\usepackage[T1]{fontenc}

\usepackage{graphicx}

%% Change the following pdf author attribute name to your name.
\usepackage[linkcolor=blue,citecolor=blue,urlcolor=blue,colorlinks,pdfpagelabels,pdftitle={Bruno Cabral's Bachelor Thesis},pdfauthor={Bruno Cabral}]{hyperref}

\address{SALVADOR}

\universitypt{Universidade Federal da Bahia}
\universityen{Federal University of Bahia}

\departmentpt{Depertamento de Ciência da Computação}
\departmenten{Computer Science Department}

\programpt{Programa Multiinstitucional de Pós-graduação em Ciência da Computação}
\programen{Graduate in Computer Science}

\majorfieldpt{Ciência da Computação}
\majorfielden{Computer Science}

\title{Sistema de apoio à Pós graduação - UFBA}
\date{Outubro/2016}

\author{Victor de Azevedo Nunes}
\adviser{Ivan do Carmo Machado}

\begin{document}

\frontmatter
\frontpage
\presentationpage

\begin{dedicatory}
Eu dedico esta dissertação...
%I dedicate this dissertation to my family, girlfriend, friends and
%professors who gave me all necessary support to get here.
\end{dedicatory}

\acknowledgements
Meus agradecimentos...

\begin{epigraph}[]{Edward V Berard}
Walking on water and developing software from a specification are easy if both are frozen
\end{epigraph}

\resumo
% Escreva seu resumo no arquivo resumo.tex
\input{resumo}

\abstract
% Write your abstract in a file called abstract.tex
\input{abstract}

% Summary (tables of contents)
\tableofcontents

% List of figures
\listoffigures

% List of tables
\listoftables

% List of acronyms
% Acronyms manual: http://linorg.usp.br/CTAN/macros/latex/contrib/acronym/acronym.pdf
\listofacronyms
\input{acronyms}

% List of listings
%\lstlistoflistings

\mainmatter

\include{chapters/intro}
\include{chapters/referencial_teorico}

% \include{chapters/introduction/main}
% \include{chapters/background/main}
% \include{chapters/proposed_solution/main}
% \include{chapters/experiment/main}
% \include{chapters/conclusion/main}

%\bibliographystyle{natbib}
%\addcontentsline{toc}{chapter}{\bibliographytocname}
\bibliography{references}

% Appendix
\clearpage
\addappheadtotoc
\appendix
\appendixpage
% \include{appendix/experiment-instruments}

\end{document}
% %% RiSE Latex Template - version 0.5
%%
%% RiSE's latex template for thesis and dissertations
%% http://risetemplate.sourceforge.net
%%
%% (c) 2012 Yguaratã Cerqueira Cavalcanti (yguarata@gmail.com)
%%          Vinicius Cardoso Garcia (vinicius.garcia@gmail.com)
%%
%% This document was initially based on UFPEThesis template, from Paulo Gustavo
%% S. Fonseca.
%%
%% ACKNOWLEDGEMENTS
%%
%% We would like to thanks the RiSE's researchers community, the 
%% students from Federal University of Pernambuco, and other users that have
%% been contributing to this projects with comments and patches.
%%
%% GENERAL INSTRUCTIONS
%%
%% We strongly recommend you to compile your documents using pdflatex command.
%% It is also recommend use the texlipse plugin for Eclipse to edit your documents.
%%
%% Options for \documentclass command:
%%         * Idiom
%%           pt   - Portguese (default)
%%           en   - English
%%
%%         * Text type
%%           bsc  - B.Sc. Thesis
%%           msc  - M.Sc. Thesis (default)
%%           qual - PHD qualification (not tested yet)
%%           prop - PHD proposal (not tested yet)
%%           phd  - PHD thesis
%%
%%         * Media
%%           scr  - to eletronic version (PDF) / see the users guide
%%
%%         * Pagination
%%           oneside - unique face press
%%           twoside - two faces press
%%
%%		   * Line spacing
%%           singlespacing  - the same as using \linespread{1}
%%           onehalfspacing - the same as using \linespread{1.3}
%%           doublespacing  - the same as using \linespread{1.6}
%%
%% Reference commands. Use the following commands to make references in your
%% text:
%%          \figref  -- for Figure reference
%%          \tabref  -- for Table reference
%%          \eqnref  -- for equation reference
%%          \chapref -- for chapter reference
%%          \secref  -- for section reference
%%          \appref  -- for appendix reference
%%          \axiref  -- for axiom reference
%%          \conjref -- for conjecture reference
%%          \defref  -- for definition reference
%%          \lemref  -- for lemma reference
%%          \theoref -- for theorem reference
%%          \corref  -- for corollary reference
%%          \propref -- for proprosition reference
%%          \pgref   -- for page reference
%%
%%          Example: See \chapref{chap:introduction}. It will produce 
%%                   'See Chapter 1', in case of English language.

\documentclass[pt,twoside,onehalfspacing,bsc]{risethesis}

\usepackage[sort,square]{natbib} %package for bibliography
\bibliographystyle{ieeetr} %authordate1

\usepackage[utf8]{inputenc}
\usepackage[brazilian]{babel}
\usepackage[T1]{fontenc}

\usepackage{graphicx}

%% Change the following pdf author attribute name to your name.
\usepackage[linkcolor=blue,citecolor=blue,urlcolor=blue,colorlinks,pdfpagelabels,pdftitle={Bruno Cabral's Bachelor Thesis},pdfauthor={Bruno Cabral}]{hyperref}

\address{SALVADOR}

\universitypt{Universidade Federal da Bahia}
\universityen{Federal University of Bahia}

\departmentpt{Depertamento de Ciência da Computação}
\departmenten{Computer Science Department}

\programpt{Programa Multiinstitucional de Pós-graduação em Ciência da Computação}
\programen{Graduate in Computer Science}

\majorfieldpt{Ciência da Computação}
\majorfielden{Computer Science}

\title{Sistema de apoio à Pós graduação - UFBA}
\date{Outubro/2016}

\author{Victor de Azevedo Nunes}
\adviser{Ivan do Carmo Machado}

\begin{document}

\frontmatter
\frontpage
\presentationpage

\begin{dedicatory}
Eu dedico esta dissertação...
%I dedicate this dissertation to my family, girlfriend, friends and
%professors who gave me all necessary support to get here.
\end{dedicatory}

\acknowledgements
Meus agradecimentos...

\begin{epigraph}[]{Edward V Berard}
Walking on water and developing software from a specification are easy if both are frozen
\end{epigraph}

\resumo
% Escreva seu resumo no arquivo resumo.tex
\input{resumo}

\abstract
% Write your abstract in a file called abstract.tex
\input{abstract}

% Summary (tables of contents)
\tableofcontents

% List of figures
\listoffigures

% List of tables
\listoftables

% List of acronyms
% Acronyms manual: http://linorg.usp.br/CTAN/macros/latex/contrib/acronym/acronym.pdf
\listofacronyms
\input{acronyms}

% List of listings
%\lstlistoflistings

\mainmatter

\include{chapters/intro}
\include{chapters/referencial_teorico}

% \include{chapters/introduction/main}
% \include{chapters/background/main}
% \include{chapters/proposed_solution/main}
% \include{chapters/experiment/main}
% \include{chapters/conclusion/main}

%\bibliographystyle{natbib}
%\addcontentsline{toc}{chapter}{\bibliographytocname}
\bibliography{references}

% Appendix
\clearpage
\addappheadtotoc
\appendix
\appendixpage
% \include{appendix/experiment-instruments}

\end{document}
% %% RiSE Latex Template - version 0.5
%%
%% RiSE's latex template for thesis and dissertations
%% http://risetemplate.sourceforge.net
%%
%% (c) 2012 Yguaratã Cerqueira Cavalcanti (yguarata@gmail.com)
%%          Vinicius Cardoso Garcia (vinicius.garcia@gmail.com)
%%
%% This document was initially based on UFPEThesis template, from Paulo Gustavo
%% S. Fonseca.
%%
%% ACKNOWLEDGEMENTS
%%
%% We would like to thanks the RiSE's researchers community, the 
%% students from Federal University of Pernambuco, and other users that have
%% been contributing to this projects with comments and patches.
%%
%% GENERAL INSTRUCTIONS
%%
%% We strongly recommend you to compile your documents using pdflatex command.
%% It is also recommend use the texlipse plugin for Eclipse to edit your documents.
%%
%% Options for \documentclass command:
%%         * Idiom
%%           pt   - Portguese (default)
%%           en   - English
%%
%%         * Text type
%%           bsc  - B.Sc. Thesis
%%           msc  - M.Sc. Thesis (default)
%%           qual - PHD qualification (not tested yet)
%%           prop - PHD proposal (not tested yet)
%%           phd  - PHD thesis
%%
%%         * Media
%%           scr  - to eletronic version (PDF) / see the users guide
%%
%%         * Pagination
%%           oneside - unique face press
%%           twoside - two faces press
%%
%%		   * Line spacing
%%           singlespacing  - the same as using \linespread{1}
%%           onehalfspacing - the same as using \linespread{1.3}
%%           doublespacing  - the same as using \linespread{1.6}
%%
%% Reference commands. Use the following commands to make references in your
%% text:
%%          \figref  -- for Figure reference
%%          \tabref  -- for Table reference
%%          \eqnref  -- for equation reference
%%          \chapref -- for chapter reference
%%          \secref  -- for section reference
%%          \appref  -- for appendix reference
%%          \axiref  -- for axiom reference
%%          \conjref -- for conjecture reference
%%          \defref  -- for definition reference
%%          \lemref  -- for lemma reference
%%          \theoref -- for theorem reference
%%          \corref  -- for corollary reference
%%          \propref -- for proprosition reference
%%          \pgref   -- for page reference
%%
%%          Example: See \chapref{chap:introduction}. It will produce 
%%                   'See Chapter 1', in case of English language.

\documentclass[pt,twoside,onehalfspacing,bsc]{risethesis}

\usepackage[sort,square]{natbib} %package for bibliography
\bibliographystyle{ieeetr} %authordate1

\usepackage[utf8]{inputenc}
\usepackage[brazilian]{babel}
\usepackage[T1]{fontenc}

\usepackage{graphicx}

%% Change the following pdf author attribute name to your name.
\usepackage[linkcolor=blue,citecolor=blue,urlcolor=blue,colorlinks,pdfpagelabels,pdftitle={Bruno Cabral's Bachelor Thesis},pdfauthor={Bruno Cabral}]{hyperref}

\address{SALVADOR}

\universitypt{Universidade Federal da Bahia}
\universityen{Federal University of Bahia}

\departmentpt{Depertamento de Ciência da Computação}
\departmenten{Computer Science Department}

\programpt{Programa Multiinstitucional de Pós-graduação em Ciência da Computação}
\programen{Graduate in Computer Science}

\majorfieldpt{Ciência da Computação}
\majorfielden{Computer Science}

\title{Sistema de apoio à Pós graduação - UFBA}
\date{Outubro/2016}

\author{Victor de Azevedo Nunes}
\adviser{Ivan do Carmo Machado}

\begin{document}

\frontmatter
\frontpage
\presentationpage

\begin{dedicatory}
Eu dedico esta dissertação...
%I dedicate this dissertation to my family, girlfriend, friends and
%professors who gave me all necessary support to get here.
\end{dedicatory}

\acknowledgements
Meus agradecimentos...

\begin{epigraph}[]{Edward V Berard}
Walking on water and developing software from a specification are easy if both are frozen
\end{epigraph}

\resumo
% Escreva seu resumo no arquivo resumo.tex
\input{resumo}

\abstract
% Write your abstract in a file called abstract.tex
\input{abstract}

% Summary (tables of contents)
\tableofcontents

% List of figures
\listoffigures

% List of tables
\listoftables

% List of acronyms
% Acronyms manual: http://linorg.usp.br/CTAN/macros/latex/contrib/acronym/acronym.pdf
\listofacronyms
\input{acronyms}

% List of listings
%\lstlistoflistings

\mainmatter

\include{chapters/intro}
\include{chapters/referencial_teorico}

% \include{chapters/introduction/main}
% \include{chapters/background/main}
% \include{chapters/proposed_solution/main}
% \include{chapters/experiment/main}
% \include{chapters/conclusion/main}

%\bibliographystyle{natbib}
%\addcontentsline{toc}{chapter}{\bibliographytocname}
\bibliography{references}

% Appendix
\clearpage
\addappheadtotoc
\appendix
\appendixpage
% \include{appendix/experiment-instruments}

\end{document}

%\bibliographystyle{natbib}
%\addcontentsline{toc}{chapter}{\bibliographytocname}
\bibliography{references}

% Appendix
\clearpage
\addappheadtotoc
\appendix
\appendixpage
% \include{appendix/experiment-instruments}

\end{document}
% %% RiSE Latex Template - version 0.5
%%
%% RiSE's latex template for thesis and dissertations
%% http://risetemplate.sourceforge.net
%%
%% (c) 2012 Yguaratã Cerqueira Cavalcanti (yguarata@gmail.com)
%%          Vinicius Cardoso Garcia (vinicius.garcia@gmail.com)
%%
%% This document was initially based on UFPEThesis template, from Paulo Gustavo
%% S. Fonseca.
%%
%% ACKNOWLEDGEMENTS
%%
%% We would like to thanks the RiSE's researchers community, the 
%% students from Federal University of Pernambuco, and other users that have
%% been contributing to this projects with comments and patches.
%%
%% GENERAL INSTRUCTIONS
%%
%% We strongly recommend you to compile your documents using pdflatex command.
%% It is also recommend use the texlipse plugin for Eclipse to edit your documents.
%%
%% Options for \documentclass command:
%%         * Idiom
%%           pt   - Portguese (default)
%%           en   - English
%%
%%         * Text type
%%           bsc  - B.Sc. Thesis
%%           msc  - M.Sc. Thesis (default)
%%           qual - PHD qualification (not tested yet)
%%           prop - PHD proposal (not tested yet)
%%           phd  - PHD thesis
%%
%%         * Media
%%           scr  - to eletronic version (PDF) / see the users guide
%%
%%         * Pagination
%%           oneside - unique face press
%%           twoside - two faces press
%%
%%		   * Line spacing
%%           singlespacing  - the same as using \linespread{1}
%%           onehalfspacing - the same as using \linespread{1.3}
%%           doublespacing  - the same as using \linespread{1.6}
%%
%% Reference commands. Use the following commands to make references in your
%% text:
%%          \figref  -- for Figure reference
%%          \tabref  -- for Table reference
%%          \eqnref  -- for equation reference
%%          \chapref -- for chapter reference
%%          \secref  -- for section reference
%%          \appref  -- for appendix reference
%%          \axiref  -- for axiom reference
%%          \conjref -- for conjecture reference
%%          \defref  -- for definition reference
%%          \lemref  -- for lemma reference
%%          \theoref -- for theorem reference
%%          \corref  -- for corollary reference
%%          \propref -- for proprosition reference
%%          \pgref   -- for page reference
%%
%%          Example: See \chapref{chap:introduction}. It will produce 
%%                   'See Chapter 1', in case of English language.

\documentclass[pt,twoside,onehalfspacing,bsc]{risethesis}

\usepackage[sort,square]{natbib} %package for bibliography
\bibliographystyle{ieeetr} %authordate1

\usepackage[utf8]{inputenc}
\usepackage[brazilian]{babel}
\usepackage[T1]{fontenc}

\usepackage{graphicx}

%% Change the following pdf author attribute name to your name.
\usepackage[linkcolor=blue,citecolor=blue,urlcolor=blue,colorlinks,pdfpagelabels,pdftitle={Bruno Cabral's Bachelor Thesis},pdfauthor={Bruno Cabral}]{hyperref}

\address{SALVADOR}

\universitypt{Universidade Federal da Bahia}
\universityen{Federal University of Bahia}

\departmentpt{Depertamento de Ciência da Computação}
\departmenten{Computer Science Department}

\programpt{Programa Multiinstitucional de Pós-graduação em Ciência da Computação}
\programen{Graduate in Computer Science}

\majorfieldpt{Ciência da Computação}
\majorfielden{Computer Science}

\title{Sistema de apoio à Pós graduação - UFBA}
\date{Outubro/2016}

\author{Victor de Azevedo Nunes}
\adviser{Ivan do Carmo Machado}

\begin{document}

\frontmatter
\frontpage
\presentationpage

\begin{dedicatory}
Eu dedico esta dissertação...
%I dedicate this dissertation to my family, girlfriend, friends and
%professors who gave me all necessary support to get here.
\end{dedicatory}

\acknowledgements
Meus agradecimentos...

\begin{epigraph}[]{Edward V Berard}
Walking on water and developing software from a specification are easy if both are frozen
\end{epigraph}

\resumo
% Escreva seu resumo no arquivo resumo.tex
Este trabalho de conclusão de curso utiliza elementos da engenharia de software para propor um software como serviço a fim de otimizar o processo de construção e manutenção dos softwares. Assim, o objetivo deste SaaS é gerenciar as permissões de acesso de sistemas cliente, provendo o reuso de software e padronizando as soluções. O sistema proposto neste trabalho encontra-se implementado e disponível no Github, e traz fundamentos sobre a arquitetura e tecnologias utilizadas, bem como avaliações sobre possibilidades de utilização do mesmo em diversos ambientes, como Web e mobile.

\begin{keywords}
Software; Reuso; SaaS; Web; PHP
\end{keywords}

\abstract
% Write your abstract in a file called abstract.tex
My abstract...

\begin{keywords}
key words...
\end{keywords}

% Summary (tables of contents)
\tableofcontents

% List of figures
\listoffigures

% List of tables
\listoftables

% List of acronyms
% Acronyms manual: http://linorg.usp.br/CTAN/macros/latex/contrib/acronym/acronym.pdf
\listofacronyms
\begin{acronym}[ACRONYM] 
% Change the word ACRONYM above to change the acronym column width.
% The column width is equals to the width of the word that you put.
% Read the manual about acronym package for more examples:
%   http://linorg.usp.br/CTAN/macros/latex/contrib/acronym/acronym.pdf

\acro{SPA}{Single Page Application}
\acro{JSON}{Javascript Object Notation}
\acro{PHP}{PHP: Hypertext Preprocessor}
\acro{SaaS}{Software as a Service}
\acro{ERP}{Enterprise Resource Planning}
\acro{QoS}{Quality of Service}
\acro{UML}{Unified Modeling Language}
\acro{MVC}{Model-View-Controller}
\acro{Ajax}{Asynchronous Javascript and XML}
\acro{HTML}{HyperText Markup Language}
\acro{CSS}{Cascading Style Sheets}
\acro{API}{Application Programming Interface}
\acro{DOM}{Document Object Model}
\acro{BPMN}{Business Process Model and Notation}
\acro{REST}{Representational State Transfer}

\end{acronym}

% List of listings
%\lstlistoflistings

\mainmatter

\chapter{Introdução}

\section{Motivação}

Organizar os procedimentos de um processo sempre nos traz vantagens. Apesar de no processo de implantação de um sistema, o mesmo burocratizar o processo, com o tempo temos o retorno da dedicação para a inserção dos dados. Com um certo volume de dados, é possível estruturar informações que num processo manual são difíceis de serem enxergadas. Assim, é possível depender menos das pessoas que organizam o processo, pois o legado de informações não estará mais somente na mente de alguns, mas sim documentado nos dados do sistema.

Além de colaborar na organização, também haverá uma grande colaboração no tempo gasto na gestão. Lidar com muitos papéis e confiar na mente humana para guardar informações, não é uma alternativa muito segura devido ao fato que as pessoas sempre estão sujeitas a sair do processo e levar contigo a experiência obtida. Experiência essa que faz com que os procedimentos sejam executados de forma mais eficiente. Entretanto, com um sistema inteligente, é possível auxiliar e tornar mais ágil a execução das tarefas.


\section{Problema}


De acordo com funcionários ligados ao o setor de pós graduação da UFBA, entrevistados a fim de um maior entendimento do cenário, apesar das semelhanças estruturais, a pós graduação gerida de forma diferencia da graduação. FULANO afirma que devido ao fato de não ter a mesma visibilidade, não tem acesso aos mesmos recursos de gestão acadêmica da graduação. O professores não executam somente atividades dentro da sala de aula, também tem diversas outras ocupações no setor. E muitos procedimentos realizados extra classe ainda se encontram sendo realizados de forma manual, estando mais vulnerável ao erro ou até mesmo à violação do processo. Também ocorre um grande desperdício de tempo pelos professores e gestores da área, devido ao diversos processos ainda realizados de forma manual, sem a devida documentação. Segundo FULANO, também entrevistado, esse tempo perdido implica numa redução da eficiência na sala de aula, pois o professor acaba por ter menos tempo disponível para o planejamento das atividades, o que gera impactos negativos aos alunos.


\section{Objetivos} %<o que deve ser feito/entregue>


Devido aos muitos processos sendo resolvidos de forma manual, propõe-se com solução um sistema moderno, arquitetado para ter funcionamento na web e com um módulo mobile, a fim de fornecer informações de forma rápida e eficiente para os professores através de notificações, já que o acesso à internet móvel é comum entre os possíveis usuários do sistema em questão.
O principal requisito para o sistema seria dispor recursos para reduzir o tempo desperdiçado pelos professores durante as atividades extra classe.


\section{Metodologia} %<como será feito | como resolver o problema apontado inicialmente>


%<analise de literatura | design | implementação | validação>
Baseando-se nas tecnologias gratuitas em alta no cenário atual do desenvolvimento web, dispomos de algumas opções eficientes para a implementação da solução. Dentre as possibilidades, considerando a facilidade para futura manutenção e continuidade do projeto, tende-se a optar por uma tecnologia popular. Como linguagem de programação, adota-se o PHP. A escolha é fundamentada de acordo com a pesquisa da RedMonk de 2015, que evidencia o uso das linguagens de programação de acordo com as discussões no StackOverflow e repositórios no GitHub. É possível constatar a popularidade do PHP no cenário atual com o gráfico da pesquisa citada, na qual o PHP é apresentado na terceira colocação, apenas atrás do lider JavaScript e do segundo colocado, o Java.

\begin{figure}
	\label{fig:graficoRedmonk}
	\includegraphics[width=1\textwidth]{img/grafico_redmonk}
	\caption{Ranking das liguagens de programação no Stack Overflow e Github}
\end{figure}


Ainda assim, para compor a interface do dado projeto, também ocorrerá o uso do líder JavaScript de forma intensa, provendo o elo com o as informações gerenciadas pelo PHP.


Entretanto, não seria inteligente desenvolver um sistema completo sem o auxílio de um framework. Dentre os frameworks disponíveis para PHP, hoje o destaque está com o Laravel, que se encontra no topo dentre os mais utilizados no momento. 


A WebHostFace, uma empresa de hospedagem, compilou várias estatísticas para criar um infográfico mostrando os frameworks PHP mais populares de 2015. Utilizando informações sobre os próprios clientes, o Google Trends, estatísticas de repositórios do GitHub e a pesquisa do SitePoint “Best PHP Frameworks 2015”, a WebHostFace elaborou o seguinte infográfico: 

\begin{figure}
	\label{fig:graficoWebhostface}
	\includegraphics[width=1\textwidth]{img/infografico_webhostface}
	\caption{Infográfico da WebhostFace, exibindo a popularidade dos Frameworks PHP em 2015}
\end{figure}

Assim, tem-se a evidência que o Laravel em 2015 teve a maior popularidade em projetos pessoais e tem a maior comunidade entre os concorrentes, o que o torna uma boa escolha para a escrita de um software que será continuado por terceiros.


Para elaborar os recursos de interface e integrar ao back-end PHP do sistema, será adotado o já conhecido AngularJS, ferramenta sólida e conhecida no aspecto em questão. 


Dados coletados via Google Trends, que propõe comparações entre termos pesquisados, revela a popularidade do AngularJs diante de alguns dos principais concorrentes. O gráfico abaixo evidencia o cenário.


%Como mostra a Figura \ref{fig:graficoGoogleTrendsFerramentasFront}. 
\begin{figure}
	\label{fig:graficoGoogleTrendsFerramentasFront}
	\includegraphics[width=1\textwidth]{img/grafico_ferramentas_front}
	\caption{Gráfico do Google Trends exibindo as pesquisas por ferramentas front-end}
\end{figure}


Junto ao Angular JS, será utilizada a agradável tendência de interface do Material Design da Google, que propõe layouts limpos e otimizados já conhecidos pelos usuários de smartphones Android. 


Para a elaboração da plataforma mobile do projeto, será utilizado o Ionic Framework, muito difundido e bastante pesquisado na área, o que fica evidenciado com o gráfico de pesquisbaixo, coletado via Google Trends buscando por frameworks de desenvolvimento híbrido mobile.


\begin{figure}
	\label{fig:graficoGoogleTrendsFerramentasHibridasMobile}
	\includegraphics[width=1\textwidth]{img/grafico_ferramentas_hibridas_mobile}
	\caption{Gráfico do Google Trends exibindo as pesquisas por Frameworks híbridos mobile}
\end{figure}	

Para layout da interface mobile, também será aplicado a tendência do Material Design, a fim de propor uma harmonia entre o módulo web e mobile para os usuários


\section{Resultados Esperados}


Como fruto de um sistema para pós-graduação da UFBA, espera-se que os professores tenham mais recursos para integrar as atividades e também prover melhores condições para acompanhamento da vida acadêmica dos alunos em questão. Também, que os novos colaboradores que entrarem no processo tenham facilidade de compreender o fluxo do setor ao navegar pelo sistema proposto.


\section{Fora de Escopo}


Interação com os alunos devido às complicações para realizar a integração com o sistema empregado na UFBA, gerenciado pela XXXXXX, o que causaria uma inviabilidade no projeto devido à necessidade de entrega do produto ser mais forte que o tempo necessário para executar o processo de obtenção de acesso ao sistema legado para realizar a integração.


\section{Estrutura do Trabalho}


<breve resumo sobre os capítulos do TCC>
\chapter{Referencial Teórico}


Projetar o desenvolvimento de um software requer muito planejamento, pois as falhas iniciais podem custar bastante caro ou até mesmo inviabilizar a continuação de um projeto. Assim, a escolha da arquitetura ideal para a aplicabilidade é essencial na concepção de um produto de software. 
De todo o modo, sempre busca-se fazer mais com menos. Diante de tal filosofia, temos neste capítulo, uma breve discussão sobre alguns elementos de projeto e arquitetura de software, a fim de contextualizar este trabalho de conclusão de curso.
O capítulo corrente é composto por quatro seções. A \ref{sec:saas} trata de Software como serviço, discutindo alguns elementos do contexto que são relevantes para o trabalho proposto. A \ref{sec:reuso}  discute sobre a empregabilidade do reuso de software. A \ref{sec:modularizacao} seção, trata sobre aspectos envolvidos na modularização dos softwares. Por fim, a \ref{sec:apps_web} aborda as aplicações web, discutindo sobre aspectos relevantes sobre a aplicação web que compõe este trabalho.


 \section{Software como serviço}\label{sec:saas}


Segundo La e Chun \citep{La2009Systematic}, o princípio da definição de Software como um Serviço (Sofware as a Service - SaaS) é um serviço complementar para aplicações da computação em nuvem (cloud computing). No entanto, as áreas não se confundem. SaaS deve ser entendido como um mecanismo de suporte às soluções existentes na cloud. Os SaaS existem justamente para maximizar o reuso de serviços repetidos e não centrais em uma aplicação remota.


Como vantagens, diversos fatores podem ser favoráveis para a adoção de um SaaS, como custo e manutenção dentre outros fatores aplicáveis a soluções específicas. Lechesaet al. \citep{LechesaSS11}, quantificam índices sobre os fatores determinantes para adoção ou não de um SaaS voltado para ERP na África do Sul. Os principais fatores determinantes para adoção desse mecanismo de software são sua fluidez quanto à rede e a segurança. Esses fatores estão presentes na aplicação desenvolvida neste trabalho de conclusão de curso.


Devido ao fato de ter um serviço constantemente na nuvem, fica o questionamento sobre a segurança da informação manipulada. Sabe-se que a vulnerabilidade na Web não é restrita ao SaaS, atingindo diversos âmbitos. Rai et al. \cite{journals/corr/RaiSM13} consideram que o avanço da computação em nuvem não é um problema apenas para os serviços Web do ponto de vista da segurança, pois muitos trabalhos na literatura mostram a área como mais um ponto de vulnerabilidade para diversos setores, a exemplo de infraestrutura. No mesmo artigo mencionado de Rai et al. \cite{journals/corr/RaiSM13}, também realizaram-se estudos exploratórios junto a empresas usuárias de serviços em computação em nuvem e consideram que a perspectiva de SaaS também pode fortalecer a segurança nas aplicações de cloud computing, pois o software de autenticação compartilhado por várias aplicações em nuvem, oferece uma melhor padronização e consequente facilidade de prevenção a erros de vulnerabilidade específicas de cada módulo da pesquisa. Esse ponto de vista é fundamenta para a compreensão de SaaS.


A arquitetura de armazenamento de dados de um Saas pode variar de acordo com a necessidade do contexto. Huixin \cite{7586486} descreve possíveis modelagens nesse sentido. Essa abordagem pode ser com um banco de dados único, fazendo com que diferentes clientes compartilhem o mesmo banco, diferindo os dados através de controle de usuário, ou isolando os diferentes clientes através de bancos de dados exclusivos para cada um. Esse fator também pode ser combinado com a arquitetura da aplicação, caso ofereça aplicação única para todos os clientes ou aplicação compartilhada. Diante das possíveis abordagens, a modelagem de dados do software pode ser decidida pela regra de negócio. Este trabalho optou por aplicação única e banco de dados compartilhado.

Devido ao conceito particular de obtenção de software, tanto pela visão do cliente como pela visão do vendedor, é necessário compreender os diversos aspectos que podem ser relevantes ao orçar um Saas. O recente trabalho de T. Kaur et al. \citep{6949281} orienta um modelo para compor o custo de um Saas. O custo total seria composto pelos fatores que dão suporte ao funcionamento do software. Tais fatores incluem infra-estrutura, configurabilidade, customização, parâmetros de QoS (Quality of service) como escalabilidade, disponibilidade, usabilidade, pontualidade e desempenho da resposta, portabilidade, custo total de propriedade e retorno do investimento. Esses fatores caracterizam o custo de forma eficaz, possibilitando ao fornecedor, prover um Serviço de acordo com a exigência do consumidor em vários pacotes de serviços.


O conceito de software como serviço se aplica neste trabalho de conclusão de curso, pois o mesmo estará disponível na web com alta disponibilidade adotando as características apresentadas para qualquer pessoa que desejar usá-lo. Assim, basta se cadastrar, configurá-lo e usar como um serviço, sem preocupação com a manutenção do mesmo.


\section{Reuso de software}\label{sec:reuso} %CRUISE BOOK CAPITULO 2


De acordo com o livro Software pratical reuse \citep{Ezran2002}, o reuso de software é a utilização de qualquer informação que um desenvolvedor pode necessitar no processo de criação de software. O livro de Basili e Rombach \citep{Basili1991} define reutilização de software como o uso de tudo o que está associado a um projeto de conhecimento.
Assim, o objetivo da reutilização de software é reciclar o design, código e outros componentes de um produto de software e assim reduzir o custo, o tempo e melhorar a qualidade do produto.
Segundo Keswani et al. \cite{6783445}, o componente reutilizável de software pode ser qualquer parte de seu desenvolvimento, como um fragmento de código, design, casos de teste, ou até mesmo a especificação de requisitos de uma funcionalidade do software. 

O reuso de software pode ter impacto positivo em diversos aspectos do software, vejamos alguns, conforme apresentados no C.R.U.I.S.E Book \citep{cruise2007} :

\begin{itemize}

\item Qualidade: As correções de erro tornam-se úteis em todos os locais em que ocorreu, padronizando e facilitando a manutenção.

\item Produtividade: O ganho de produtividade é alcançado devido ao menor número de artefatos desenvolvido. Isso resulta em menor esforço de teste e também análise e design, reduzindo custos.

\item Confiabilidade: A utilização de componentes bem testados aumenta a
confiança no software. Além disso, a utilização de um mesmo componente em vários sistemas, aumenta a possibilidade de detecção de erros e reforça a confiança no componente.

\item Redução do Esforço: A reutilização de software proporciona uma redução do tempo de desenvolvimento, o que reduz o tempo necessário para o produto ser disponibilizado no mercado para trazer rentabilidade.

\item Trabalho redundante e tempo de desenvolvimento: Desenvolver um sistema do
zero significa desenvolvimento redundante de muitos componentes, como requisitos, especificações, casos de uso, arquitetura, etc. Isso pode ser evitado quando estes estão disponíveis como componentes reutilizáveis e podem ser compartilhados, resultando em um processo de desenvolvimento otimizado.

\item Documentação: Embora a documentação seja muito importante para a
manutenção de um sistema, muitas vezes é negligenciada. A reutilização de componentes de software reduz a quantidade de documentação a ser escrita, entretanto depende da qualidade do que está escrito. Assim, apenas a estrutura do sistema e os novos artefatos desenvolvidos necessitam ser documentados.

\item Custo de manutenção: Menos defeitos e manutenções são esperados quando tem-se comprovada a qualidade dos componentes utilizados.

\item Tamanho da equipe: É comum haver casos em que a equipe de desenvolvimento sofre sobrecarga. Entretanto, dobrar o tamanho da equipe de desenvolvimento não necessariamente duplica produtividade. Se muitos componentes podem ser reutilizados, é possível desenvolver com equipes menores, levando a melhor comunicação e aumento da produtividade.

\end{itemize}

Apesar dos benefícios da reutilização de software, ela não é suficientemente aproveitada. Existem fatores que influenciam direta ou indiretamente na sua adoção. Esses fatores podem ser de aspecto gerencial, organizacional, econômico, conceitual ou técnico. Veremos a seguir alguns aspectos que podem gerar conflito com a cultura de reuso de software, segundo o C.R.U.I.S.E Book \citep{cruise2007} :
%(Sametinger, 1997). REVER

\begin{itemize}
	
\item Falta de apoio da gestão: Como a reutilização de software gera custos iniciais,
a medida pode não ser amplamente alcançada em uma organização sem o apoio de alto nível de gestão. Os gestores têm de ser informados sobre os custos iniciais e serem convencidos sobre economias futuras.

\item Gerenciamento do Projeto: Gerenciar projetos tradicionais é uma tarefa árdua, principalmente, os que praticam a reutilização de software. Utilizando a técnica em larga escala, tem-se impacto sobre todo o ciclo de vida do software.

\item Estruturas organizacionais inadequadas: As estruturas organizacionais devem
considerar diferentes necessidades que surgem quando a reutilização em larga escala está sendo adotada. Por exemplo, uma equipe particionada pode ser alocada somente para desenvolver, manter e certificar componentes reutilizáveis de software.

\item Incentivos de gestão: É comum a falta de incentivo para deixar os desenvolvedores gastarem tempo elaborando componentes do sistemas. A produtividade é muitas vezes medida apenas no tempo necessário para concluir um projeto. Assim, fazer qualquer trabalho além disso, embora benéfico para a empresa como um todo, diminui o seu sucesso. Mesmo quando os componentes reutilizáveis são utilizados, os benefícios obtidos são uma pequena fração do que poderia ser alcançado caso houvesse reutilização explícita, planejada e organizada.

\item Dificuldade de encontrar software reutilizável: Para reutilizar os componentes, devem existir formas eficientes de busca. Além disso, é importante ter um repositório bem organizado contendo componentes com um eficiente meio de acesso.

\item Não reutilização do software encontrado. O acesso fácil ao software existente
não necessariamente aumentar a reutilização. Os componentes reutilizáveis devem ser cuidadosamente especificados, projetados, implementados e documentados, pois em alguns casos, modificar e adaptar o código  pode ser mais custoso que a programação da funcionalidade necessária a partir do zero.

\item Modificação: É muito difícil encontrar um componente que funcione
exatamente da mesma maneira que queremos. Desta forma, são necessárias modificações e devem existir formas de determinar os seus efeitos sobre o componente.


\end{itemize}


A definição do reuso de software pode ser associada a este trabalho, pois ainda que como um serviço, estaremos adotando uma aplicação única para servir aos mais diversos softwares de uma empresa, por exemplo. Logo, ao adotar o projeto, pode ser interessante aplicar algumas métricas para tomar conhecimento de possíveis vantanjens como as citadas nessa seção, a exemplo de economia de tempo de desenvolvimento, e custo do projeto.


%Outra diretriz importante para a reutilização de software é reduzir o risco na criação de novos softwares. O risco tende a ser bastante reduzido se os componentes que estão sendo reutilizados têm as documentação, interfaces necessárias e devidamente testadas, fatores que contibruem para uma fácil integração.
%De acordo com Keswani et al. \citep{6783445}, para o reuso de software dar retornos apropriados, o processo deve ser sistemático e planejado. Qualquer organização que implemente a reutilização de software deve identificar os melhores métodos e estratégias de reutilização para obter a máxima produtividade. A reutilização de software ajuda a evitar software de engenharia a partir do zero, pois usa módulos de software existentes. A reutilização de software, embora seja uma tarefa difícil, especialmente para softwares antigos sem padrões de projeto, pode melhorar significativamente a produtividade e a qualidade de um produto de software. Embora a reutilização de software não seja um novo campo, ela pode dar grandes retornos em curto período de tempo.


\section{Modularização}\label{sec:modularizacao} %artigo de claudio pagina 222 introdução


%A modularidade vem desempenhando um papel predominante estágios emergentes das disciplinas de arquitetura de software [13]. Engenheiros de software consideram modularidade como princípio base na comparação entre arquiteturas alternativas  e arquitetura degeneração [9]. De fato, os engenheiros de software são incentivados a arquitecturas, baseando-se numa multiplicidade de mecanismos de modularidade disponíveis em: 
%(i) Linguagens de descrição de arquitetura (ADLs), como ACME [8], 
%(ii) catálogos de arquitetônicos [2, 13], e 
%(iii) conhecem bem princípios de alto nível, como interfaces de componentes estreitos, acoplamento arquitectónico reduzido e semelhantes.


Conforme é frisado por Wickramaarachchi e Lai \citep{7062705}, o conceito de modularização na indústria de software tem uma longa história e tem sido utilizado para melhorar o processo de desenvolvimento de software em diferentes estágios. Os principais conceitos por trás da modularização do software foram introduzidos por pesquisadores pioneiros há quarenta anos, com uma notável contribuição feita por Melvin Conway e David Parnas, que tem representação notável na engenharia de software.


Modularizar um software é um bom padrão a ser adotado. Segundo Wickramaarachchi e Lai \citep{7062705}, a modularização é importante na identificação de dependências e reduz as dificuldades diante de uma possível necessidade de grandes alterações. De uma perspectiva da engenharia de software, uma modularização geralmente tem várias vantagens, tais como: tornar a complexidade do software mais gerenciável, facilitar o trabalho paralelo e tornar o software mais maleável para acomodar o futuro incerto que um software pode ter. O objetivo final da modularização do software é aumentar a produtividade ea qualidade do software. Tal conceito encontra-se bastante difundido e estái incorporado em linguagens de programação e ferramentas de software. O presente trabalho favorece ao uso da modularização de um software e até mesmo pode ser considerado um módulo a ser acoplado a qualquer software, mediante a compatibilidade.


Ao realizar a adoção do trabalho proposto, fica bastante evidente a "responsabilidade" do mesmo no escopo do projeto. Assim, é possível tratá-lo como um módulo do projeto que o usa, mesmo que seja consumido como um serviço.


\section{Aplicações web}\label{sec:apps_web}


A popularidade das soluções Web aumentou exponencialmente na última década e todos os dias cresce o número de pessoas usuárias desse tipo de software. E seguindo um padrão próprio, Kumar et al. \citep{7813710} sugerem que para o desenvolvimento web, deve-se manter a prática eficaz de produzir diagramas UML. A abordagem baseada na web oferece uma maneira fácil e eficaz para gerenciar e controlar o processo de desenvolvimento por meio de artefatos de modelagem. Tal abordagem pode ser usada quando há uma exigência de lidar com mudanças muito rápidas e grandes em requisitos de forma muito eficaz em muito menos tempo, gerando assim um menor impacto. 


Para atender à fomentada demanda de aplicativos web, é necessário adotar métodos de desenvolvimentos que sejam ágeis, eficientes e de fácil manutenção. Yu Ping et al. \cite{1372143} propõem o uso do modelo MVC (Model, View e Controller) no desenvolvimento para softwares web. O modelo apresentado tornou-se um padrão popular e divide o software em camadas com propósito definido, tornando-o de mais fácil manutenção.


O Ajax (Asynchronous Javascript and XML) revolucionou a web. Conforme demonstrado por Yuping \citep{6845605}, ao usar a tecnologia Ajax, podemos enriquecer a experiência do usuário em aplicações baseadas em navegador de internet, e fornecer uma variedade de aplicações interativas para atender às necessidade de humanização das aplicações.
Os aplicativos Ajax em execução no navegador se comunicam com um servidor Web de forma assíncrona e atualizam apenas uma parte da página.


No artigo de Tesarik et al. \citep{4664349}, temos que o design de software SPA (Single page application) é uma maneira interessante de criar um software Web numa uma única página. Essa solução de página única sem navegação funciona apenas com base em técnicas dinâmicas e assíncronas, como o AJAX (citado no parágrafo anterior). No entanto, esta abordagem coloca o desenvolvedor antes de alguns desafios substanciais.
Para projetar a interface do usuário que mostra as informações, é importante projetar corretamente a tela para manipular os dados do aplicativo numa única página. O design da página deve ser elaborado para maximizar a decomposição da página em componentes distintos que encapsulam os principais casos de uso.
Também recomenda-se que para elaborar os artefatos visuais, explore os mais ricos recursos na implementação, como HTML5, JavaScript, Ajax, CSS3, e outras tecnologias que se apliquem. Entretanto, desenvolver uma rica interface com o uso de diversas tecnologias/frameworks pode ocasionar um esforço maior para explorar as possibilidades.
Tais componentes da interface de uma aplicação SPA normalmente são alimentados mediante o consumo de uma API Rest via requisições AJAX. Assim, é possível particionar as responsabilidades de processamento do software entre cliente e servidor.


Como é possível denotar no trabalho de Salvadori e Siqueire \citep{7195633}, REST é uma arquitetura muito popular para integração de aplicativos web, que permite compartilhar e reutilizar informações através de sistemas. Aplicações de grande escala baseadas em no REST deve ser implementadas utilizando estratégias e mecanismos para produzir sistemas que sejam fáceis de desenvolver, reutilizar e manter. As interfaces de integração fornecidas por esses sistemas, chamadas de Web API, têm uma influência importante nas características da implementação resultante, pois a sua forma de resposta deve ser casada com a interface que o consome.
 
 
Os elementos comentados nessa seção encontram-se presentes neste trabalho de conclusão de curso. Juntos, montam a estrutura tecnológica necessária aliada à arquitetura adotada. Esses elementos seguem tendências atuais dos softwares web, fazendo com que esse trabalho esteja composto por tecnologias modernas que estão com boa aceitação no mercado.

% %% RiSE Latex Template - version 0.5
%%
%% RiSE's latex template for thesis and dissertations
%% http://risetemplate.sourceforge.net
%%
%% (c) 2012 Yguaratã Cerqueira Cavalcanti (yguarata@gmail.com)
%%          Vinicius Cardoso Garcia (vinicius.garcia@gmail.com)
%%
%% This document was initially based on UFPEThesis template, from Paulo Gustavo
%% S. Fonseca.
%%
%% ACKNOWLEDGEMENTS
%%
%% We would like to thanks the RiSE's researchers community, the 
%% students from Federal University of Pernambuco, and other users that have
%% been contributing to this projects with comments and patches.
%%
%% GENERAL INSTRUCTIONS
%%
%% We strongly recommend you to compile your documents using pdflatex command.
%% It is also recommend use the texlipse plugin for Eclipse to edit your documents.
%%
%% Options for \documentclass command:
%%         * Idiom
%%           pt   - Portguese (default)
%%           en   - English
%%
%%         * Text type
%%           bsc  - B.Sc. Thesis
%%           msc  - M.Sc. Thesis (default)
%%           qual - PHD qualification (not tested yet)
%%           prop - PHD proposal (not tested yet)
%%           phd  - PHD thesis
%%
%%         * Media
%%           scr  - to eletronic version (PDF) / see the users guide
%%
%%         * Pagination
%%           oneside - unique face press
%%           twoside - two faces press
%%
%%		   * Line spacing
%%           singlespacing  - the same as using \linespread{1}
%%           onehalfspacing - the same as using \linespread{1.3}
%%           doublespacing  - the same as using \linespread{1.6}
%%
%% Reference commands. Use the following commands to make references in your
%% text:
%%          \figref  -- for Figure reference
%%          \tabref  -- for Table reference
%%          \eqnref  -- for equation reference
%%          \chapref -- for chapter reference
%%          \secref  -- for section reference
%%          \appref  -- for appendix reference
%%          \axiref  -- for axiom reference
%%          \conjref -- for conjecture reference
%%          \defref  -- for definition reference
%%          \lemref  -- for lemma reference
%%          \theoref -- for theorem reference
%%          \corref  -- for corollary reference
%%          \propref -- for proprosition reference
%%          \pgref   -- for page reference
%%
%%          Example: See \chapref{chap:introduction}. It will produce 
%%                   'See Chapter 1', in case of English language.

\documentclass[pt,twoside,onehalfspacing,bsc]{risethesis}

\usepackage[sort,square]{natbib} %package for bibliography
\bibliographystyle{ieeetr} %authordate1

\usepackage[utf8]{inputenc}
\usepackage[brazilian]{babel}
\usepackage[T1]{fontenc}

\usepackage{graphicx}

%% Change the following pdf author attribute name to your name.
\usepackage[linkcolor=blue,citecolor=blue,urlcolor=blue,colorlinks,pdfpagelabels,pdftitle={Bruno Cabral's Bachelor Thesis},pdfauthor={Bruno Cabral}]{hyperref}

\address{SALVADOR}

\universitypt{Universidade Federal da Bahia}
\universityen{Federal University of Bahia}

\departmentpt{Depertamento de Ciência da Computação}
\departmenten{Computer Science Department}

\programpt{Programa Multiinstitucional de Pós-graduação em Ciência da Computação}
\programen{Graduate in Computer Science}

\majorfieldpt{Ciência da Computação}
\majorfielden{Computer Science}

\title{Sistema de apoio à Pós graduação - UFBA}
\date{Outubro/2016}

\author{Victor de Azevedo Nunes}
\adviser{Ivan do Carmo Machado}

\begin{document}

\frontmatter
\frontpage
\presentationpage

\begin{dedicatory}
Eu dedico esta dissertação...
%I dedicate this dissertation to my family, girlfriend, friends and
%professors who gave me all necessary support to get here.
\end{dedicatory}

\acknowledgements
Meus agradecimentos...

\begin{epigraph}[]{Edward V Berard}
Walking on water and developing software from a specification are easy if both are frozen
\end{epigraph}

\resumo
% Escreva seu resumo no arquivo resumo.tex
\input{resumo}

\abstract
% Write your abstract in a file called abstract.tex
\input{abstract}

% Summary (tables of contents)
\tableofcontents

% List of figures
\listoffigures

% List of tables
\listoftables

% List of acronyms
% Acronyms manual: http://linorg.usp.br/CTAN/macros/latex/contrib/acronym/acronym.pdf
\listofacronyms
\input{acronyms}

% List of listings
%\lstlistoflistings

\mainmatter

\include{chapters/intro}
\include{chapters/referencial_teorico}

% \include{chapters/introduction/main}
% \include{chapters/background/main}
% \include{chapters/proposed_solution/main}
% \include{chapters/experiment/main}
% \include{chapters/conclusion/main}

%\bibliographystyle{natbib}
%\addcontentsline{toc}{chapter}{\bibliographytocname}
\bibliography{references}

% Appendix
\clearpage
\addappheadtotoc
\appendix
\appendixpage
% \include{appendix/experiment-instruments}

\end{document}
% %% RiSE Latex Template - version 0.5
%%
%% RiSE's latex template for thesis and dissertations
%% http://risetemplate.sourceforge.net
%%
%% (c) 2012 Yguaratã Cerqueira Cavalcanti (yguarata@gmail.com)
%%          Vinicius Cardoso Garcia (vinicius.garcia@gmail.com)
%%
%% This document was initially based on UFPEThesis template, from Paulo Gustavo
%% S. Fonseca.
%%
%% ACKNOWLEDGEMENTS
%%
%% We would like to thanks the RiSE's researchers community, the 
%% students from Federal University of Pernambuco, and other users that have
%% been contributing to this projects with comments and patches.
%%
%% GENERAL INSTRUCTIONS
%%
%% We strongly recommend you to compile your documents using pdflatex command.
%% It is also recommend use the texlipse plugin for Eclipse to edit your documents.
%%
%% Options for \documentclass command:
%%         * Idiom
%%           pt   - Portguese (default)
%%           en   - English
%%
%%         * Text type
%%           bsc  - B.Sc. Thesis
%%           msc  - M.Sc. Thesis (default)
%%           qual - PHD qualification (not tested yet)
%%           prop - PHD proposal (not tested yet)
%%           phd  - PHD thesis
%%
%%         * Media
%%           scr  - to eletronic version (PDF) / see the users guide
%%
%%         * Pagination
%%           oneside - unique face press
%%           twoside - two faces press
%%
%%		   * Line spacing
%%           singlespacing  - the same as using \linespread{1}
%%           onehalfspacing - the same as using \linespread{1.3}
%%           doublespacing  - the same as using \linespread{1.6}
%%
%% Reference commands. Use the following commands to make references in your
%% text:
%%          \figref  -- for Figure reference
%%          \tabref  -- for Table reference
%%          \eqnref  -- for equation reference
%%          \chapref -- for chapter reference
%%          \secref  -- for section reference
%%          \appref  -- for appendix reference
%%          \axiref  -- for axiom reference
%%          \conjref -- for conjecture reference
%%          \defref  -- for definition reference
%%          \lemref  -- for lemma reference
%%          \theoref -- for theorem reference
%%          \corref  -- for corollary reference
%%          \propref -- for proprosition reference
%%          \pgref   -- for page reference
%%
%%          Example: See \chapref{chap:introduction}. It will produce 
%%                   'See Chapter 1', in case of English language.

\documentclass[pt,twoside,onehalfspacing,bsc]{risethesis}

\usepackage[sort,square]{natbib} %package for bibliography
\bibliographystyle{ieeetr} %authordate1

\usepackage[utf8]{inputenc}
\usepackage[brazilian]{babel}
\usepackage[T1]{fontenc}

\usepackage{graphicx}

%% Change the following pdf author attribute name to your name.
\usepackage[linkcolor=blue,citecolor=blue,urlcolor=blue,colorlinks,pdfpagelabels,pdftitle={Bruno Cabral's Bachelor Thesis},pdfauthor={Bruno Cabral}]{hyperref}

\address{SALVADOR}

\universitypt{Universidade Federal da Bahia}
\universityen{Federal University of Bahia}

\departmentpt{Depertamento de Ciência da Computação}
\departmenten{Computer Science Department}

\programpt{Programa Multiinstitucional de Pós-graduação em Ciência da Computação}
\programen{Graduate in Computer Science}

\majorfieldpt{Ciência da Computação}
\majorfielden{Computer Science}

\title{Sistema de apoio à Pós graduação - UFBA}
\date{Outubro/2016}

\author{Victor de Azevedo Nunes}
\adviser{Ivan do Carmo Machado}

\begin{document}

\frontmatter
\frontpage
\presentationpage

\begin{dedicatory}
Eu dedico esta dissertação...
%I dedicate this dissertation to my family, girlfriend, friends and
%professors who gave me all necessary support to get here.
\end{dedicatory}

\acknowledgements
Meus agradecimentos...

\begin{epigraph}[]{Edward V Berard}
Walking on water and developing software from a specification are easy if both are frozen
\end{epigraph}

\resumo
% Escreva seu resumo no arquivo resumo.tex
\input{resumo}

\abstract
% Write your abstract in a file called abstract.tex
\input{abstract}

% Summary (tables of contents)
\tableofcontents

% List of figures
\listoffigures

% List of tables
\listoftables

% List of acronyms
% Acronyms manual: http://linorg.usp.br/CTAN/macros/latex/contrib/acronym/acronym.pdf
\listofacronyms
\input{acronyms}

% List of listings
%\lstlistoflistings

\mainmatter

\include{chapters/intro}
\include{chapters/referencial_teorico}

% \include{chapters/introduction/main}
% \include{chapters/background/main}
% \include{chapters/proposed_solution/main}
% \include{chapters/experiment/main}
% \include{chapters/conclusion/main}

%\bibliographystyle{natbib}
%\addcontentsline{toc}{chapter}{\bibliographytocname}
\bibliography{references}

% Appendix
\clearpage
\addappheadtotoc
\appendix
\appendixpage
% \include{appendix/experiment-instruments}

\end{document}
% %% RiSE Latex Template - version 0.5
%%
%% RiSE's latex template for thesis and dissertations
%% http://risetemplate.sourceforge.net
%%
%% (c) 2012 Yguaratã Cerqueira Cavalcanti (yguarata@gmail.com)
%%          Vinicius Cardoso Garcia (vinicius.garcia@gmail.com)
%%
%% This document was initially based on UFPEThesis template, from Paulo Gustavo
%% S. Fonseca.
%%
%% ACKNOWLEDGEMENTS
%%
%% We would like to thanks the RiSE's researchers community, the 
%% students from Federal University of Pernambuco, and other users that have
%% been contributing to this projects with comments and patches.
%%
%% GENERAL INSTRUCTIONS
%%
%% We strongly recommend you to compile your documents using pdflatex command.
%% It is also recommend use the texlipse plugin for Eclipse to edit your documents.
%%
%% Options for \documentclass command:
%%         * Idiom
%%           pt   - Portguese (default)
%%           en   - English
%%
%%         * Text type
%%           bsc  - B.Sc. Thesis
%%           msc  - M.Sc. Thesis (default)
%%           qual - PHD qualification (not tested yet)
%%           prop - PHD proposal (not tested yet)
%%           phd  - PHD thesis
%%
%%         * Media
%%           scr  - to eletronic version (PDF) / see the users guide
%%
%%         * Pagination
%%           oneside - unique face press
%%           twoside - two faces press
%%
%%		   * Line spacing
%%           singlespacing  - the same as using \linespread{1}
%%           onehalfspacing - the same as using \linespread{1.3}
%%           doublespacing  - the same as using \linespread{1.6}
%%
%% Reference commands. Use the following commands to make references in your
%% text:
%%          \figref  -- for Figure reference
%%          \tabref  -- for Table reference
%%          \eqnref  -- for equation reference
%%          \chapref -- for chapter reference
%%          \secref  -- for section reference
%%          \appref  -- for appendix reference
%%          \axiref  -- for axiom reference
%%          \conjref -- for conjecture reference
%%          \defref  -- for definition reference
%%          \lemref  -- for lemma reference
%%          \theoref -- for theorem reference
%%          \corref  -- for corollary reference
%%          \propref -- for proprosition reference
%%          \pgref   -- for page reference
%%
%%          Example: See \chapref{chap:introduction}. It will produce 
%%                   'See Chapter 1', in case of English language.

\documentclass[pt,twoside,onehalfspacing,bsc]{risethesis}

\usepackage[sort,square]{natbib} %package for bibliography
\bibliographystyle{ieeetr} %authordate1

\usepackage[utf8]{inputenc}
\usepackage[brazilian]{babel}
\usepackage[T1]{fontenc}

\usepackage{graphicx}

%% Change the following pdf author attribute name to your name.
\usepackage[linkcolor=blue,citecolor=blue,urlcolor=blue,colorlinks,pdfpagelabels,pdftitle={Bruno Cabral's Bachelor Thesis},pdfauthor={Bruno Cabral}]{hyperref}

\address{SALVADOR}

\universitypt{Universidade Federal da Bahia}
\universityen{Federal University of Bahia}

\departmentpt{Depertamento de Ciência da Computação}
\departmenten{Computer Science Department}

\programpt{Programa Multiinstitucional de Pós-graduação em Ciência da Computação}
\programen{Graduate in Computer Science}

\majorfieldpt{Ciência da Computação}
\majorfielden{Computer Science}

\title{Sistema de apoio à Pós graduação - UFBA}
\date{Outubro/2016}

\author{Victor de Azevedo Nunes}
\adviser{Ivan do Carmo Machado}

\begin{document}

\frontmatter
\frontpage
\presentationpage

\begin{dedicatory}
Eu dedico esta dissertação...
%I dedicate this dissertation to my family, girlfriend, friends and
%professors who gave me all necessary support to get here.
\end{dedicatory}

\acknowledgements
Meus agradecimentos...

\begin{epigraph}[]{Edward V Berard}
Walking on water and developing software from a specification are easy if both are frozen
\end{epigraph}

\resumo
% Escreva seu resumo no arquivo resumo.tex
\input{resumo}

\abstract
% Write your abstract in a file called abstract.tex
\input{abstract}

% Summary (tables of contents)
\tableofcontents

% List of figures
\listoffigures

% List of tables
\listoftables

% List of acronyms
% Acronyms manual: http://linorg.usp.br/CTAN/macros/latex/contrib/acronym/acronym.pdf
\listofacronyms
\input{acronyms}

% List of listings
%\lstlistoflistings

\mainmatter

\include{chapters/intro}
\include{chapters/referencial_teorico}

% \include{chapters/introduction/main}
% \include{chapters/background/main}
% \include{chapters/proposed_solution/main}
% \include{chapters/experiment/main}
% \include{chapters/conclusion/main}

%\bibliographystyle{natbib}
%\addcontentsline{toc}{chapter}{\bibliographytocname}
\bibliography{references}

% Appendix
\clearpage
\addappheadtotoc
\appendix
\appendixpage
% \include{appendix/experiment-instruments}

\end{document}
% %% RiSE Latex Template - version 0.5
%%
%% RiSE's latex template for thesis and dissertations
%% http://risetemplate.sourceforge.net
%%
%% (c) 2012 Yguaratã Cerqueira Cavalcanti (yguarata@gmail.com)
%%          Vinicius Cardoso Garcia (vinicius.garcia@gmail.com)
%%
%% This document was initially based on UFPEThesis template, from Paulo Gustavo
%% S. Fonseca.
%%
%% ACKNOWLEDGEMENTS
%%
%% We would like to thanks the RiSE's researchers community, the 
%% students from Federal University of Pernambuco, and other users that have
%% been contributing to this projects with comments and patches.
%%
%% GENERAL INSTRUCTIONS
%%
%% We strongly recommend you to compile your documents using pdflatex command.
%% It is also recommend use the texlipse plugin for Eclipse to edit your documents.
%%
%% Options for \documentclass command:
%%         * Idiom
%%           pt   - Portguese (default)
%%           en   - English
%%
%%         * Text type
%%           bsc  - B.Sc. Thesis
%%           msc  - M.Sc. Thesis (default)
%%           qual - PHD qualification (not tested yet)
%%           prop - PHD proposal (not tested yet)
%%           phd  - PHD thesis
%%
%%         * Media
%%           scr  - to eletronic version (PDF) / see the users guide
%%
%%         * Pagination
%%           oneside - unique face press
%%           twoside - two faces press
%%
%%		   * Line spacing
%%           singlespacing  - the same as using \linespread{1}
%%           onehalfspacing - the same as using \linespread{1.3}
%%           doublespacing  - the same as using \linespread{1.6}
%%
%% Reference commands. Use the following commands to make references in your
%% text:
%%          \figref  -- for Figure reference
%%          \tabref  -- for Table reference
%%          \eqnref  -- for equation reference
%%          \chapref -- for chapter reference
%%          \secref  -- for section reference
%%          \appref  -- for appendix reference
%%          \axiref  -- for axiom reference
%%          \conjref -- for conjecture reference
%%          \defref  -- for definition reference
%%          \lemref  -- for lemma reference
%%          \theoref -- for theorem reference
%%          \corref  -- for corollary reference
%%          \propref -- for proprosition reference
%%          \pgref   -- for page reference
%%
%%          Example: See \chapref{chap:introduction}. It will produce 
%%                   'See Chapter 1', in case of English language.

\documentclass[pt,twoside,onehalfspacing,bsc]{risethesis}

\usepackage[sort,square]{natbib} %package for bibliography
\bibliographystyle{ieeetr} %authordate1

\usepackage[utf8]{inputenc}
\usepackage[brazilian]{babel}
\usepackage[T1]{fontenc}

\usepackage{graphicx}

%% Change the following pdf author attribute name to your name.
\usepackage[linkcolor=blue,citecolor=blue,urlcolor=blue,colorlinks,pdfpagelabels,pdftitle={Bruno Cabral's Bachelor Thesis},pdfauthor={Bruno Cabral}]{hyperref}

\address{SALVADOR}

\universitypt{Universidade Federal da Bahia}
\universityen{Federal University of Bahia}

\departmentpt{Depertamento de Ciência da Computação}
\departmenten{Computer Science Department}

\programpt{Programa Multiinstitucional de Pós-graduação em Ciência da Computação}
\programen{Graduate in Computer Science}

\majorfieldpt{Ciência da Computação}
\majorfielden{Computer Science}

\title{Sistema de apoio à Pós graduação - UFBA}
\date{Outubro/2016}

\author{Victor de Azevedo Nunes}
\adviser{Ivan do Carmo Machado}

\begin{document}

\frontmatter
\frontpage
\presentationpage

\begin{dedicatory}
Eu dedico esta dissertação...
%I dedicate this dissertation to my family, girlfriend, friends and
%professors who gave me all necessary support to get here.
\end{dedicatory}

\acknowledgements
Meus agradecimentos...

\begin{epigraph}[]{Edward V Berard}
Walking on water and developing software from a specification are easy if both are frozen
\end{epigraph}

\resumo
% Escreva seu resumo no arquivo resumo.tex
\input{resumo}

\abstract
% Write your abstract in a file called abstract.tex
\input{abstract}

% Summary (tables of contents)
\tableofcontents

% List of figures
\listoffigures

% List of tables
\listoftables

% List of acronyms
% Acronyms manual: http://linorg.usp.br/CTAN/macros/latex/contrib/acronym/acronym.pdf
\listofacronyms
\input{acronyms}

% List of listings
%\lstlistoflistings

\mainmatter

\include{chapters/intro}
\include{chapters/referencial_teorico}

% \include{chapters/introduction/main}
% \include{chapters/background/main}
% \include{chapters/proposed_solution/main}
% \include{chapters/experiment/main}
% \include{chapters/conclusion/main}

%\bibliographystyle{natbib}
%\addcontentsline{toc}{chapter}{\bibliographytocname}
\bibliography{references}

% Appendix
\clearpage
\addappheadtotoc
\appendix
\appendixpage
% \include{appendix/experiment-instruments}

\end{document}
% %% RiSE Latex Template - version 0.5
%%
%% RiSE's latex template for thesis and dissertations
%% http://risetemplate.sourceforge.net
%%
%% (c) 2012 Yguaratã Cerqueira Cavalcanti (yguarata@gmail.com)
%%          Vinicius Cardoso Garcia (vinicius.garcia@gmail.com)
%%
%% This document was initially based on UFPEThesis template, from Paulo Gustavo
%% S. Fonseca.
%%
%% ACKNOWLEDGEMENTS
%%
%% We would like to thanks the RiSE's researchers community, the 
%% students from Federal University of Pernambuco, and other users that have
%% been contributing to this projects with comments and patches.
%%
%% GENERAL INSTRUCTIONS
%%
%% We strongly recommend you to compile your documents using pdflatex command.
%% It is also recommend use the texlipse plugin for Eclipse to edit your documents.
%%
%% Options for \documentclass command:
%%         * Idiom
%%           pt   - Portguese (default)
%%           en   - English
%%
%%         * Text type
%%           bsc  - B.Sc. Thesis
%%           msc  - M.Sc. Thesis (default)
%%           qual - PHD qualification (not tested yet)
%%           prop - PHD proposal (not tested yet)
%%           phd  - PHD thesis
%%
%%         * Media
%%           scr  - to eletronic version (PDF) / see the users guide
%%
%%         * Pagination
%%           oneside - unique face press
%%           twoside - two faces press
%%
%%		   * Line spacing
%%           singlespacing  - the same as using \linespread{1}
%%           onehalfspacing - the same as using \linespread{1.3}
%%           doublespacing  - the same as using \linespread{1.6}
%%
%% Reference commands. Use the following commands to make references in your
%% text:
%%          \figref  -- for Figure reference
%%          \tabref  -- for Table reference
%%          \eqnref  -- for equation reference
%%          \chapref -- for chapter reference
%%          \secref  -- for section reference
%%          \appref  -- for appendix reference
%%          \axiref  -- for axiom reference
%%          \conjref -- for conjecture reference
%%          \defref  -- for definition reference
%%          \lemref  -- for lemma reference
%%          \theoref -- for theorem reference
%%          \corref  -- for corollary reference
%%          \propref -- for proprosition reference
%%          \pgref   -- for page reference
%%
%%          Example: See \chapref{chap:introduction}. It will produce 
%%                   'See Chapter 1', in case of English language.

\documentclass[pt,twoside,onehalfspacing,bsc]{risethesis}

\usepackage[sort,square]{natbib} %package for bibliography
\bibliographystyle{ieeetr} %authordate1

\usepackage[utf8]{inputenc}
\usepackage[brazilian]{babel}
\usepackage[T1]{fontenc}

\usepackage{graphicx}

%% Change the following pdf author attribute name to your name.
\usepackage[linkcolor=blue,citecolor=blue,urlcolor=blue,colorlinks,pdfpagelabels,pdftitle={Bruno Cabral's Bachelor Thesis},pdfauthor={Bruno Cabral}]{hyperref}

\address{SALVADOR}

\universitypt{Universidade Federal da Bahia}
\universityen{Federal University of Bahia}

\departmentpt{Depertamento de Ciência da Computação}
\departmenten{Computer Science Department}

\programpt{Programa Multiinstitucional de Pós-graduação em Ciência da Computação}
\programen{Graduate in Computer Science}

\majorfieldpt{Ciência da Computação}
\majorfielden{Computer Science}

\title{Sistema de apoio à Pós graduação - UFBA}
\date{Outubro/2016}

\author{Victor de Azevedo Nunes}
\adviser{Ivan do Carmo Machado}

\begin{document}

\frontmatter
\frontpage
\presentationpage

\begin{dedicatory}
Eu dedico esta dissertação...
%I dedicate this dissertation to my family, girlfriend, friends and
%professors who gave me all necessary support to get here.
\end{dedicatory}

\acknowledgements
Meus agradecimentos...

\begin{epigraph}[]{Edward V Berard}
Walking on water and developing software from a specification are easy if both are frozen
\end{epigraph}

\resumo
% Escreva seu resumo no arquivo resumo.tex
\input{resumo}

\abstract
% Write your abstract in a file called abstract.tex
\input{abstract}

% Summary (tables of contents)
\tableofcontents

% List of figures
\listoffigures

% List of tables
\listoftables

% List of acronyms
% Acronyms manual: http://linorg.usp.br/CTAN/macros/latex/contrib/acronym/acronym.pdf
\listofacronyms
\input{acronyms}

% List of listings
%\lstlistoflistings

\mainmatter

\include{chapters/intro}
\include{chapters/referencial_teorico}

% \include{chapters/introduction/main}
% \include{chapters/background/main}
% \include{chapters/proposed_solution/main}
% \include{chapters/experiment/main}
% \include{chapters/conclusion/main}

%\bibliographystyle{natbib}
%\addcontentsline{toc}{chapter}{\bibliographytocname}
\bibliography{references}

% Appendix
\clearpage
\addappheadtotoc
\appendix
\appendixpage
% \include{appendix/experiment-instruments}

\end{document}

%\bibliographystyle{natbib}
%\addcontentsline{toc}{chapter}{\bibliographytocname}
\bibliography{references}

% Appendix
\clearpage
\addappheadtotoc
\appendix
\appendixpage
% \include{appendix/experiment-instruments}

\end{document}
% %% RiSE Latex Template - version 0.5
%%
%% RiSE's latex template for thesis and dissertations
%% http://risetemplate.sourceforge.net
%%
%% (c) 2012 Yguaratã Cerqueira Cavalcanti (yguarata@gmail.com)
%%          Vinicius Cardoso Garcia (vinicius.garcia@gmail.com)
%%
%% This document was initially based on UFPEThesis template, from Paulo Gustavo
%% S. Fonseca.
%%
%% ACKNOWLEDGEMENTS
%%
%% We would like to thanks the RiSE's researchers community, the 
%% students from Federal University of Pernambuco, and other users that have
%% been contributing to this projects with comments and patches.
%%
%% GENERAL INSTRUCTIONS
%%
%% We strongly recommend you to compile your documents using pdflatex command.
%% It is also recommend use the texlipse plugin for Eclipse to edit your documents.
%%
%% Options for \documentclass command:
%%         * Idiom
%%           pt   - Portguese (default)
%%           en   - English
%%
%%         * Text type
%%           bsc  - B.Sc. Thesis
%%           msc  - M.Sc. Thesis (default)
%%           qual - PHD qualification (not tested yet)
%%           prop - PHD proposal (not tested yet)
%%           phd  - PHD thesis
%%
%%         * Media
%%           scr  - to eletronic version (PDF) / see the users guide
%%
%%         * Pagination
%%           oneside - unique face press
%%           twoside - two faces press
%%
%%		   * Line spacing
%%           singlespacing  - the same as using \linespread{1}
%%           onehalfspacing - the same as using \linespread{1.3}
%%           doublespacing  - the same as using \linespread{1.6}
%%
%% Reference commands. Use the following commands to make references in your
%% text:
%%          \figref  -- for Figure reference
%%          \tabref  -- for Table reference
%%          \eqnref  -- for equation reference
%%          \chapref -- for chapter reference
%%          \secref  -- for section reference
%%          \appref  -- for appendix reference
%%          \axiref  -- for axiom reference
%%          \conjref -- for conjecture reference
%%          \defref  -- for definition reference
%%          \lemref  -- for lemma reference
%%          \theoref -- for theorem reference
%%          \corref  -- for corollary reference
%%          \propref -- for proprosition reference
%%          \pgref   -- for page reference
%%
%%          Example: See \chapref{chap:introduction}. It will produce 
%%                   'See Chapter 1', in case of English language.

\documentclass[pt,twoside,onehalfspacing,bsc]{risethesis}

\usepackage[sort,square]{natbib} %package for bibliography
\bibliographystyle{ieeetr} %authordate1

\usepackage[utf8]{inputenc}
\usepackage[brazilian]{babel}
\usepackage[T1]{fontenc}

\usepackage{graphicx}

%% Change the following pdf author attribute name to your name.
\usepackage[linkcolor=blue,citecolor=blue,urlcolor=blue,colorlinks,pdfpagelabels,pdftitle={Bruno Cabral's Bachelor Thesis},pdfauthor={Bruno Cabral}]{hyperref}

\address{SALVADOR}

\universitypt{Universidade Federal da Bahia}
\universityen{Federal University of Bahia}

\departmentpt{Depertamento de Ciência da Computação}
\departmenten{Computer Science Department}

\programpt{Programa Multiinstitucional de Pós-graduação em Ciência da Computação}
\programen{Graduate in Computer Science}

\majorfieldpt{Ciência da Computação}
\majorfielden{Computer Science}

\title{Sistema de apoio à Pós graduação - UFBA}
\date{Outubro/2016}

\author{Victor de Azevedo Nunes}
\adviser{Ivan do Carmo Machado}

\begin{document}

\frontmatter
\frontpage
\presentationpage

\begin{dedicatory}
Eu dedico esta dissertação...
%I dedicate this dissertation to my family, girlfriend, friends and
%professors who gave me all necessary support to get here.
\end{dedicatory}

\acknowledgements
Meus agradecimentos...

\begin{epigraph}[]{Edward V Berard}
Walking on water and developing software from a specification are easy if both are frozen
\end{epigraph}

\resumo
% Escreva seu resumo no arquivo resumo.tex
Este trabalho de conclusão de curso utiliza elementos da engenharia de software para propor um software como serviço a fim de otimizar o processo de construção e manutenção dos softwares. Assim, o objetivo deste SaaS é gerenciar as permissões de acesso de sistemas cliente, provendo o reuso de software e padronizando as soluções. O sistema proposto neste trabalho encontra-se implementado e disponível no Github, e traz fundamentos sobre a arquitetura e tecnologias utilizadas, bem como avaliações sobre possibilidades de utilização do mesmo em diversos ambientes, como Web e mobile.

\begin{keywords}
Software; Reuso; SaaS; Web; PHP
\end{keywords}

\abstract
% Write your abstract in a file called abstract.tex
My abstract...

\begin{keywords}
key words...
\end{keywords}

% Summary (tables of contents)
\tableofcontents

% List of figures
\listoffigures

% List of tables
\listoftables

% List of acronyms
% Acronyms manual: http://linorg.usp.br/CTAN/macros/latex/contrib/acronym/acronym.pdf
\listofacronyms
\begin{acronym}[ACRONYM] 
% Change the word ACRONYM above to change the acronym column width.
% The column width is equals to the width of the word that you put.
% Read the manual about acronym package for more examples:
%   http://linorg.usp.br/CTAN/macros/latex/contrib/acronym/acronym.pdf

\acro{SPA}{Single Page Application}
\acro{JSON}{Javascript Object Notation}
\acro{PHP}{PHP: Hypertext Preprocessor}
\acro{SaaS}{Software as a Service}
\acro{ERP}{Enterprise Resource Planning}
\acro{QoS}{Quality of Service}
\acro{UML}{Unified Modeling Language}
\acro{MVC}{Model-View-Controller}
\acro{Ajax}{Asynchronous Javascript and XML}
\acro{HTML}{HyperText Markup Language}
\acro{CSS}{Cascading Style Sheets}
\acro{API}{Application Programming Interface}
\acro{DOM}{Document Object Model}
\acro{BPMN}{Business Process Model and Notation}
\acro{REST}{Representational State Transfer}

\end{acronym}

% List of listings
%\lstlistoflistings

\mainmatter

\chapter{Introdução}

\section{Motivação}

Organizar os procedimentos de um processo sempre nos traz vantagens. Apesar de no processo de implantação de um sistema, o mesmo burocratizar o processo, com o tempo temos o retorno da dedicação para a inserção dos dados. Com um certo volume de dados, é possível estruturar informações que num processo manual são difíceis de serem enxergadas. Assim, é possível depender menos das pessoas que organizam o processo, pois o legado de informações não estará mais somente na mente de alguns, mas sim documentado nos dados do sistema.

Além de colaborar na organização, também haverá uma grande colaboração no tempo gasto na gestão. Lidar com muitos papéis e confiar na mente humana para guardar informações, não é uma alternativa muito segura devido ao fato que as pessoas sempre estão sujeitas a sair do processo e levar contigo a experiência obtida. Experiência essa que faz com que os procedimentos sejam executados de forma mais eficiente. Entretanto, com um sistema inteligente, é possível auxiliar e tornar mais ágil a execução das tarefas.


\section{Problema}


De acordo com funcionários ligados ao o setor de pós graduação da UFBA, entrevistados a fim de um maior entendimento do cenário, apesar das semelhanças estruturais, a pós graduação gerida de forma diferencia da graduação. FULANO afirma que devido ao fato de não ter a mesma visibilidade, não tem acesso aos mesmos recursos de gestão acadêmica da graduação. O professores não executam somente atividades dentro da sala de aula, também tem diversas outras ocupações no setor. E muitos procedimentos realizados extra classe ainda se encontram sendo realizados de forma manual, estando mais vulnerável ao erro ou até mesmo à violação do processo. Também ocorre um grande desperdício de tempo pelos professores e gestores da área, devido ao diversos processos ainda realizados de forma manual, sem a devida documentação. Segundo FULANO, também entrevistado, esse tempo perdido implica numa redução da eficiência na sala de aula, pois o professor acaba por ter menos tempo disponível para o planejamento das atividades, o que gera impactos negativos aos alunos.


\section{Objetivos} %<o que deve ser feito/entregue>


Devido aos muitos processos sendo resolvidos de forma manual, propõe-se com solução um sistema moderno, arquitetado para ter funcionamento na web e com um módulo mobile, a fim de fornecer informações de forma rápida e eficiente para os professores através de notificações, já que o acesso à internet móvel é comum entre os possíveis usuários do sistema em questão.
O principal requisito para o sistema seria dispor recursos para reduzir o tempo desperdiçado pelos professores durante as atividades extra classe.


\section{Metodologia} %<como será feito | como resolver o problema apontado inicialmente>


%<analise de literatura | design | implementação | validação>
Baseando-se nas tecnologias gratuitas em alta no cenário atual do desenvolvimento web, dispomos de algumas opções eficientes para a implementação da solução. Dentre as possibilidades, considerando a facilidade para futura manutenção e continuidade do projeto, tende-se a optar por uma tecnologia popular. Como linguagem de programação, adota-se o PHP. A escolha é fundamentada de acordo com a pesquisa da RedMonk de 2015, que evidencia o uso das linguagens de programação de acordo com as discussões no StackOverflow e repositórios no GitHub. É possível constatar a popularidade do PHP no cenário atual com o gráfico da pesquisa citada, na qual o PHP é apresentado na terceira colocação, apenas atrás do lider JavaScript e do segundo colocado, o Java.

\begin{figure}
	\label{fig:graficoRedmonk}
	\includegraphics[width=1\textwidth]{img/grafico_redmonk}
	\caption{Ranking das liguagens de programação no Stack Overflow e Github}
\end{figure}


Ainda assim, para compor a interface do dado projeto, também ocorrerá o uso do líder JavaScript de forma intensa, provendo o elo com o as informações gerenciadas pelo PHP.


Entretanto, não seria inteligente desenvolver um sistema completo sem o auxílio de um framework. Dentre os frameworks disponíveis para PHP, hoje o destaque está com o Laravel, que se encontra no topo dentre os mais utilizados no momento. 


A WebHostFace, uma empresa de hospedagem, compilou várias estatísticas para criar um infográfico mostrando os frameworks PHP mais populares de 2015. Utilizando informações sobre os próprios clientes, o Google Trends, estatísticas de repositórios do GitHub e a pesquisa do SitePoint “Best PHP Frameworks 2015”, a WebHostFace elaborou o seguinte infográfico: 

\begin{figure}
	\label{fig:graficoWebhostface}
	\includegraphics[width=1\textwidth]{img/infografico_webhostface}
	\caption{Infográfico da WebhostFace, exibindo a popularidade dos Frameworks PHP em 2015}
\end{figure}

Assim, tem-se a evidência que o Laravel em 2015 teve a maior popularidade em projetos pessoais e tem a maior comunidade entre os concorrentes, o que o torna uma boa escolha para a escrita de um software que será continuado por terceiros.


Para elaborar os recursos de interface e integrar ao back-end PHP do sistema, será adotado o já conhecido AngularJS, ferramenta sólida e conhecida no aspecto em questão. 


Dados coletados via Google Trends, que propõe comparações entre termos pesquisados, revela a popularidade do AngularJs diante de alguns dos principais concorrentes. O gráfico abaixo evidencia o cenário.


%Como mostra a Figura \ref{fig:graficoGoogleTrendsFerramentasFront}. 
\begin{figure}
	\label{fig:graficoGoogleTrendsFerramentasFront}
	\includegraphics[width=1\textwidth]{img/grafico_ferramentas_front}
	\caption{Gráfico do Google Trends exibindo as pesquisas por ferramentas front-end}
\end{figure}


Junto ao Angular JS, será utilizada a agradável tendência de interface do Material Design da Google, que propõe layouts limpos e otimizados já conhecidos pelos usuários de smartphones Android. 


Para a elaboração da plataforma mobile do projeto, será utilizado o Ionic Framework, muito difundido e bastante pesquisado na área, o que fica evidenciado com o gráfico de pesquisbaixo, coletado via Google Trends buscando por frameworks de desenvolvimento híbrido mobile.


\begin{figure}
	\label{fig:graficoGoogleTrendsFerramentasHibridasMobile}
	\includegraphics[width=1\textwidth]{img/grafico_ferramentas_hibridas_mobile}
	\caption{Gráfico do Google Trends exibindo as pesquisas por Frameworks híbridos mobile}
\end{figure}	

Para layout da interface mobile, também será aplicado a tendência do Material Design, a fim de propor uma harmonia entre o módulo web e mobile para os usuários


\section{Resultados Esperados}


Como fruto de um sistema para pós-graduação da UFBA, espera-se que os professores tenham mais recursos para integrar as atividades e também prover melhores condições para acompanhamento da vida acadêmica dos alunos em questão. Também, que os novos colaboradores que entrarem no processo tenham facilidade de compreender o fluxo do setor ao navegar pelo sistema proposto.


\section{Fora de Escopo}


Interação com os alunos devido às complicações para realizar a integração com o sistema empregado na UFBA, gerenciado pela XXXXXX, o que causaria uma inviabilidade no projeto devido à necessidade de entrega do produto ser mais forte que o tempo necessário para executar o processo de obtenção de acesso ao sistema legado para realizar a integração.


\section{Estrutura do Trabalho}


<breve resumo sobre os capítulos do TCC>
\chapter{Referencial Teórico}


Projetar o desenvolvimento de um software requer muito planejamento, pois as falhas iniciais podem custar bastante caro ou até mesmo inviabilizar a continuação de um projeto. Assim, a escolha da arquitetura ideal para a aplicabilidade é essencial na concepção de um produto de software. 
De todo o modo, sempre busca-se fazer mais com menos. Diante de tal filosofia, temos neste capítulo, uma breve discussão sobre alguns elementos de projeto e arquitetura de software, a fim de contextualizar este trabalho de conclusão de curso.
O capítulo corrente é composto por quatro seções. A \ref{sec:saas} trata de Software como serviço, discutindo alguns elementos do contexto que são relevantes para o trabalho proposto. A \ref{sec:reuso}  discute sobre a empregabilidade do reuso de software. A \ref{sec:modularizacao} seção, trata sobre aspectos envolvidos na modularização dos softwares. Por fim, a \ref{sec:apps_web} aborda as aplicações web, discutindo sobre aspectos relevantes sobre a aplicação web que compõe este trabalho.


 \section{Software como serviço}\label{sec:saas}


Segundo La e Chun \citep{La2009Systematic}, o princípio da definição de Software como um Serviço (Sofware as a Service - SaaS) é um serviço complementar para aplicações da computação em nuvem (cloud computing). No entanto, as áreas não se confundem. SaaS deve ser entendido como um mecanismo de suporte às soluções existentes na cloud. Os SaaS existem justamente para maximizar o reuso de serviços repetidos e não centrais em uma aplicação remota.


Como vantagens, diversos fatores podem ser favoráveis para a adoção de um SaaS, como custo e manutenção dentre outros fatores aplicáveis a soluções específicas. Lechesaet al. \citep{LechesaSS11}, quantificam índices sobre os fatores determinantes para adoção ou não de um SaaS voltado para ERP na África do Sul. Os principais fatores determinantes para adoção desse mecanismo de software são sua fluidez quanto à rede e a segurança. Esses fatores estão presentes na aplicação desenvolvida neste trabalho de conclusão de curso.


Devido ao fato de ter um serviço constantemente na nuvem, fica o questionamento sobre a segurança da informação manipulada. Sabe-se que a vulnerabilidade na Web não é restrita ao SaaS, atingindo diversos âmbitos. Rai et al. \cite{journals/corr/RaiSM13} consideram que o avanço da computação em nuvem não é um problema apenas para os serviços Web do ponto de vista da segurança, pois muitos trabalhos na literatura mostram a área como mais um ponto de vulnerabilidade para diversos setores, a exemplo de infraestrutura. No mesmo artigo mencionado de Rai et al. \cite{journals/corr/RaiSM13}, também realizaram-se estudos exploratórios junto a empresas usuárias de serviços em computação em nuvem e consideram que a perspectiva de SaaS também pode fortalecer a segurança nas aplicações de cloud computing, pois o software de autenticação compartilhado por várias aplicações em nuvem, oferece uma melhor padronização e consequente facilidade de prevenção a erros de vulnerabilidade específicas de cada módulo da pesquisa. Esse ponto de vista é fundamenta para a compreensão de SaaS.


A arquitetura de armazenamento de dados de um Saas pode variar de acordo com a necessidade do contexto. Huixin \cite{7586486} descreve possíveis modelagens nesse sentido. Essa abordagem pode ser com um banco de dados único, fazendo com que diferentes clientes compartilhem o mesmo banco, diferindo os dados através de controle de usuário, ou isolando os diferentes clientes através de bancos de dados exclusivos para cada um. Esse fator também pode ser combinado com a arquitetura da aplicação, caso ofereça aplicação única para todos os clientes ou aplicação compartilhada. Diante das possíveis abordagens, a modelagem de dados do software pode ser decidida pela regra de negócio. Este trabalho optou por aplicação única e banco de dados compartilhado.

Devido ao conceito particular de obtenção de software, tanto pela visão do cliente como pela visão do vendedor, é necessário compreender os diversos aspectos que podem ser relevantes ao orçar um Saas. O recente trabalho de T. Kaur et al. \citep{6949281} orienta um modelo para compor o custo de um Saas. O custo total seria composto pelos fatores que dão suporte ao funcionamento do software. Tais fatores incluem infra-estrutura, configurabilidade, customização, parâmetros de QoS (Quality of service) como escalabilidade, disponibilidade, usabilidade, pontualidade e desempenho da resposta, portabilidade, custo total de propriedade e retorno do investimento. Esses fatores caracterizam o custo de forma eficaz, possibilitando ao fornecedor, prover um Serviço de acordo com a exigência do consumidor em vários pacotes de serviços.


O conceito de software como serviço se aplica neste trabalho de conclusão de curso, pois o mesmo estará disponível na web com alta disponibilidade adotando as características apresentadas para qualquer pessoa que desejar usá-lo. Assim, basta se cadastrar, configurá-lo e usar como um serviço, sem preocupação com a manutenção do mesmo.


\section{Reuso de software}\label{sec:reuso} %CRUISE BOOK CAPITULO 2


De acordo com o livro Software pratical reuse \citep{Ezran2002}, o reuso de software é a utilização de qualquer informação que um desenvolvedor pode necessitar no processo de criação de software. O livro de Basili e Rombach \citep{Basili1991} define reutilização de software como o uso de tudo o que está associado a um projeto de conhecimento.
Assim, o objetivo da reutilização de software é reciclar o design, código e outros componentes de um produto de software e assim reduzir o custo, o tempo e melhorar a qualidade do produto.
Segundo Keswani et al. \cite{6783445}, o componente reutilizável de software pode ser qualquer parte de seu desenvolvimento, como um fragmento de código, design, casos de teste, ou até mesmo a especificação de requisitos de uma funcionalidade do software. 

O reuso de software pode ter impacto positivo em diversos aspectos do software, vejamos alguns, conforme apresentados no C.R.U.I.S.E Book \citep{cruise2007} :

\begin{itemize}

\item Qualidade: As correções de erro tornam-se úteis em todos os locais em que ocorreu, padronizando e facilitando a manutenção.

\item Produtividade: O ganho de produtividade é alcançado devido ao menor número de artefatos desenvolvido. Isso resulta em menor esforço de teste e também análise e design, reduzindo custos.

\item Confiabilidade: A utilização de componentes bem testados aumenta a
confiança no software. Além disso, a utilização de um mesmo componente em vários sistemas, aumenta a possibilidade de detecção de erros e reforça a confiança no componente.

\item Redução do Esforço: A reutilização de software proporciona uma redução do tempo de desenvolvimento, o que reduz o tempo necessário para o produto ser disponibilizado no mercado para trazer rentabilidade.

\item Trabalho redundante e tempo de desenvolvimento: Desenvolver um sistema do
zero significa desenvolvimento redundante de muitos componentes, como requisitos, especificações, casos de uso, arquitetura, etc. Isso pode ser evitado quando estes estão disponíveis como componentes reutilizáveis e podem ser compartilhados, resultando em um processo de desenvolvimento otimizado.

\item Documentação: Embora a documentação seja muito importante para a
manutenção de um sistema, muitas vezes é negligenciada. A reutilização de componentes de software reduz a quantidade de documentação a ser escrita, entretanto depende da qualidade do que está escrito. Assim, apenas a estrutura do sistema e os novos artefatos desenvolvidos necessitam ser documentados.

\item Custo de manutenção: Menos defeitos e manutenções são esperados quando tem-se comprovada a qualidade dos componentes utilizados.

\item Tamanho da equipe: É comum haver casos em que a equipe de desenvolvimento sofre sobrecarga. Entretanto, dobrar o tamanho da equipe de desenvolvimento não necessariamente duplica produtividade. Se muitos componentes podem ser reutilizados, é possível desenvolver com equipes menores, levando a melhor comunicação e aumento da produtividade.

\end{itemize}

Apesar dos benefícios da reutilização de software, ela não é suficientemente aproveitada. Existem fatores que influenciam direta ou indiretamente na sua adoção. Esses fatores podem ser de aspecto gerencial, organizacional, econômico, conceitual ou técnico. Veremos a seguir alguns aspectos que podem gerar conflito com a cultura de reuso de software, segundo o C.R.U.I.S.E Book \citep{cruise2007} :
%(Sametinger, 1997). REVER

\begin{itemize}
	
\item Falta de apoio da gestão: Como a reutilização de software gera custos iniciais,
a medida pode não ser amplamente alcançada em uma organização sem o apoio de alto nível de gestão. Os gestores têm de ser informados sobre os custos iniciais e serem convencidos sobre economias futuras.

\item Gerenciamento do Projeto: Gerenciar projetos tradicionais é uma tarefa árdua, principalmente, os que praticam a reutilização de software. Utilizando a técnica em larga escala, tem-se impacto sobre todo o ciclo de vida do software.

\item Estruturas organizacionais inadequadas: As estruturas organizacionais devem
considerar diferentes necessidades que surgem quando a reutilização em larga escala está sendo adotada. Por exemplo, uma equipe particionada pode ser alocada somente para desenvolver, manter e certificar componentes reutilizáveis de software.

\item Incentivos de gestão: É comum a falta de incentivo para deixar os desenvolvedores gastarem tempo elaborando componentes do sistemas. A produtividade é muitas vezes medida apenas no tempo necessário para concluir um projeto. Assim, fazer qualquer trabalho além disso, embora benéfico para a empresa como um todo, diminui o seu sucesso. Mesmo quando os componentes reutilizáveis são utilizados, os benefícios obtidos são uma pequena fração do que poderia ser alcançado caso houvesse reutilização explícita, planejada e organizada.

\item Dificuldade de encontrar software reutilizável: Para reutilizar os componentes, devem existir formas eficientes de busca. Além disso, é importante ter um repositório bem organizado contendo componentes com um eficiente meio de acesso.

\item Não reutilização do software encontrado. O acesso fácil ao software existente
não necessariamente aumentar a reutilização. Os componentes reutilizáveis devem ser cuidadosamente especificados, projetados, implementados e documentados, pois em alguns casos, modificar e adaptar o código  pode ser mais custoso que a programação da funcionalidade necessária a partir do zero.

\item Modificação: É muito difícil encontrar um componente que funcione
exatamente da mesma maneira que queremos. Desta forma, são necessárias modificações e devem existir formas de determinar os seus efeitos sobre o componente.


\end{itemize}


A definição do reuso de software pode ser associada a este trabalho, pois ainda que como um serviço, estaremos adotando uma aplicação única para servir aos mais diversos softwares de uma empresa, por exemplo. Logo, ao adotar o projeto, pode ser interessante aplicar algumas métricas para tomar conhecimento de possíveis vantanjens como as citadas nessa seção, a exemplo de economia de tempo de desenvolvimento, e custo do projeto.


%Outra diretriz importante para a reutilização de software é reduzir o risco na criação de novos softwares. O risco tende a ser bastante reduzido se os componentes que estão sendo reutilizados têm as documentação, interfaces necessárias e devidamente testadas, fatores que contibruem para uma fácil integração.
%De acordo com Keswani et al. \citep{6783445}, para o reuso de software dar retornos apropriados, o processo deve ser sistemático e planejado. Qualquer organização que implemente a reutilização de software deve identificar os melhores métodos e estratégias de reutilização para obter a máxima produtividade. A reutilização de software ajuda a evitar software de engenharia a partir do zero, pois usa módulos de software existentes. A reutilização de software, embora seja uma tarefa difícil, especialmente para softwares antigos sem padrões de projeto, pode melhorar significativamente a produtividade e a qualidade de um produto de software. Embora a reutilização de software não seja um novo campo, ela pode dar grandes retornos em curto período de tempo.


\section{Modularização}\label{sec:modularizacao} %artigo de claudio pagina 222 introdução


%A modularidade vem desempenhando um papel predominante estágios emergentes das disciplinas de arquitetura de software [13]. Engenheiros de software consideram modularidade como princípio base na comparação entre arquiteturas alternativas  e arquitetura degeneração [9]. De fato, os engenheiros de software são incentivados a arquitecturas, baseando-se numa multiplicidade de mecanismos de modularidade disponíveis em: 
%(i) Linguagens de descrição de arquitetura (ADLs), como ACME [8], 
%(ii) catálogos de arquitetônicos [2, 13], e 
%(iii) conhecem bem princípios de alto nível, como interfaces de componentes estreitos, acoplamento arquitectónico reduzido e semelhantes.


Conforme é frisado por Wickramaarachchi e Lai \citep{7062705}, o conceito de modularização na indústria de software tem uma longa história e tem sido utilizado para melhorar o processo de desenvolvimento de software em diferentes estágios. Os principais conceitos por trás da modularização do software foram introduzidos por pesquisadores pioneiros há quarenta anos, com uma notável contribuição feita por Melvin Conway e David Parnas, que tem representação notável na engenharia de software.


Modularizar um software é um bom padrão a ser adotado. Segundo Wickramaarachchi e Lai \citep{7062705}, a modularização é importante na identificação de dependências e reduz as dificuldades diante de uma possível necessidade de grandes alterações. De uma perspectiva da engenharia de software, uma modularização geralmente tem várias vantagens, tais como: tornar a complexidade do software mais gerenciável, facilitar o trabalho paralelo e tornar o software mais maleável para acomodar o futuro incerto que um software pode ter. O objetivo final da modularização do software é aumentar a produtividade ea qualidade do software. Tal conceito encontra-se bastante difundido e estái incorporado em linguagens de programação e ferramentas de software. O presente trabalho favorece ao uso da modularização de um software e até mesmo pode ser considerado um módulo a ser acoplado a qualquer software, mediante a compatibilidade.


Ao realizar a adoção do trabalho proposto, fica bastante evidente a "responsabilidade" do mesmo no escopo do projeto. Assim, é possível tratá-lo como um módulo do projeto que o usa, mesmo que seja consumido como um serviço.


\section{Aplicações web}\label{sec:apps_web}


A popularidade das soluções Web aumentou exponencialmente na última década e todos os dias cresce o número de pessoas usuárias desse tipo de software. E seguindo um padrão próprio, Kumar et al. \citep{7813710} sugerem que para o desenvolvimento web, deve-se manter a prática eficaz de produzir diagramas UML. A abordagem baseada na web oferece uma maneira fácil e eficaz para gerenciar e controlar o processo de desenvolvimento por meio de artefatos de modelagem. Tal abordagem pode ser usada quando há uma exigência de lidar com mudanças muito rápidas e grandes em requisitos de forma muito eficaz em muito menos tempo, gerando assim um menor impacto. 


Para atender à fomentada demanda de aplicativos web, é necessário adotar métodos de desenvolvimentos que sejam ágeis, eficientes e de fácil manutenção. Yu Ping et al. \cite{1372143} propõem o uso do modelo MVC (Model, View e Controller) no desenvolvimento para softwares web. O modelo apresentado tornou-se um padrão popular e divide o software em camadas com propósito definido, tornando-o de mais fácil manutenção.


O Ajax (Asynchronous Javascript and XML) revolucionou a web. Conforme demonstrado por Yuping \citep{6845605}, ao usar a tecnologia Ajax, podemos enriquecer a experiência do usuário em aplicações baseadas em navegador de internet, e fornecer uma variedade de aplicações interativas para atender às necessidade de humanização das aplicações.
Os aplicativos Ajax em execução no navegador se comunicam com um servidor Web de forma assíncrona e atualizam apenas uma parte da página.


No artigo de Tesarik et al. \citep{4664349}, temos que o design de software SPA (Single page application) é uma maneira interessante de criar um software Web numa uma única página. Essa solução de página única sem navegação funciona apenas com base em técnicas dinâmicas e assíncronas, como o AJAX (citado no parágrafo anterior). No entanto, esta abordagem coloca o desenvolvedor antes de alguns desafios substanciais.
Para projetar a interface do usuário que mostra as informações, é importante projetar corretamente a tela para manipular os dados do aplicativo numa única página. O design da página deve ser elaborado para maximizar a decomposição da página em componentes distintos que encapsulam os principais casos de uso.
Também recomenda-se que para elaborar os artefatos visuais, explore os mais ricos recursos na implementação, como HTML5, JavaScript, Ajax, CSS3, e outras tecnologias que se apliquem. Entretanto, desenvolver uma rica interface com o uso de diversas tecnologias/frameworks pode ocasionar um esforço maior para explorar as possibilidades.
Tais componentes da interface de uma aplicação SPA normalmente são alimentados mediante o consumo de uma API Rest via requisições AJAX. Assim, é possível particionar as responsabilidades de processamento do software entre cliente e servidor.


Como é possível denotar no trabalho de Salvadori e Siqueire \citep{7195633}, REST é uma arquitetura muito popular para integração de aplicativos web, que permite compartilhar e reutilizar informações através de sistemas. Aplicações de grande escala baseadas em no REST deve ser implementadas utilizando estratégias e mecanismos para produzir sistemas que sejam fáceis de desenvolver, reutilizar e manter. As interfaces de integração fornecidas por esses sistemas, chamadas de Web API, têm uma influência importante nas características da implementação resultante, pois a sua forma de resposta deve ser casada com a interface que o consome.
 
 
Os elementos comentados nessa seção encontram-se presentes neste trabalho de conclusão de curso. Juntos, montam a estrutura tecnológica necessária aliada à arquitetura adotada. Esses elementos seguem tendências atuais dos softwares web, fazendo com que esse trabalho esteja composto por tecnologias modernas que estão com boa aceitação no mercado.

% %% RiSE Latex Template - version 0.5
%%
%% RiSE's latex template for thesis and dissertations
%% http://risetemplate.sourceforge.net
%%
%% (c) 2012 Yguaratã Cerqueira Cavalcanti (yguarata@gmail.com)
%%          Vinicius Cardoso Garcia (vinicius.garcia@gmail.com)
%%
%% This document was initially based on UFPEThesis template, from Paulo Gustavo
%% S. Fonseca.
%%
%% ACKNOWLEDGEMENTS
%%
%% We would like to thanks the RiSE's researchers community, the 
%% students from Federal University of Pernambuco, and other users that have
%% been contributing to this projects with comments and patches.
%%
%% GENERAL INSTRUCTIONS
%%
%% We strongly recommend you to compile your documents using pdflatex command.
%% It is also recommend use the texlipse plugin for Eclipse to edit your documents.
%%
%% Options for \documentclass command:
%%         * Idiom
%%           pt   - Portguese (default)
%%           en   - English
%%
%%         * Text type
%%           bsc  - B.Sc. Thesis
%%           msc  - M.Sc. Thesis (default)
%%           qual - PHD qualification (not tested yet)
%%           prop - PHD proposal (not tested yet)
%%           phd  - PHD thesis
%%
%%         * Media
%%           scr  - to eletronic version (PDF) / see the users guide
%%
%%         * Pagination
%%           oneside - unique face press
%%           twoside - two faces press
%%
%%		   * Line spacing
%%           singlespacing  - the same as using \linespread{1}
%%           onehalfspacing - the same as using \linespread{1.3}
%%           doublespacing  - the same as using \linespread{1.6}
%%
%% Reference commands. Use the following commands to make references in your
%% text:
%%          \figref  -- for Figure reference
%%          \tabref  -- for Table reference
%%          \eqnref  -- for equation reference
%%          \chapref -- for chapter reference
%%          \secref  -- for section reference
%%          \appref  -- for appendix reference
%%          \axiref  -- for axiom reference
%%          \conjref -- for conjecture reference
%%          \defref  -- for definition reference
%%          \lemref  -- for lemma reference
%%          \theoref -- for theorem reference
%%          \corref  -- for corollary reference
%%          \propref -- for proprosition reference
%%          \pgref   -- for page reference
%%
%%          Example: See \chapref{chap:introduction}. It will produce 
%%                   'See Chapter 1', in case of English language.

\documentclass[pt,twoside,onehalfspacing,bsc]{risethesis}

\usepackage[sort,square]{natbib} %package for bibliography
\bibliographystyle{ieeetr} %authordate1

\usepackage[utf8]{inputenc}
\usepackage[brazilian]{babel}
\usepackage[T1]{fontenc}

\usepackage{graphicx}

%% Change the following pdf author attribute name to your name.
\usepackage[linkcolor=blue,citecolor=blue,urlcolor=blue,colorlinks,pdfpagelabels,pdftitle={Bruno Cabral's Bachelor Thesis},pdfauthor={Bruno Cabral}]{hyperref}

\address{SALVADOR}

\universitypt{Universidade Federal da Bahia}
\universityen{Federal University of Bahia}

\departmentpt{Depertamento de Ciência da Computação}
\departmenten{Computer Science Department}

\programpt{Programa Multiinstitucional de Pós-graduação em Ciência da Computação}
\programen{Graduate in Computer Science}

\majorfieldpt{Ciência da Computação}
\majorfielden{Computer Science}

\title{Sistema de apoio à Pós graduação - UFBA}
\date{Outubro/2016}

\author{Victor de Azevedo Nunes}
\adviser{Ivan do Carmo Machado}

\begin{document}

\frontmatter
\frontpage
\presentationpage

\begin{dedicatory}
Eu dedico esta dissertação...
%I dedicate this dissertation to my family, girlfriend, friends and
%professors who gave me all necessary support to get here.
\end{dedicatory}

\acknowledgements
Meus agradecimentos...

\begin{epigraph}[]{Edward V Berard}
Walking on water and developing software from a specification are easy if both are frozen
\end{epigraph}

\resumo
% Escreva seu resumo no arquivo resumo.tex
\input{resumo}

\abstract
% Write your abstract in a file called abstract.tex
\input{abstract}

% Summary (tables of contents)
\tableofcontents

% List of figures
\listoffigures

% List of tables
\listoftables

% List of acronyms
% Acronyms manual: http://linorg.usp.br/CTAN/macros/latex/contrib/acronym/acronym.pdf
\listofacronyms
\input{acronyms}

% List of listings
%\lstlistoflistings

\mainmatter

\include{chapters/intro}
\include{chapters/referencial_teorico}

% \include{chapters/introduction/main}
% \include{chapters/background/main}
% \include{chapters/proposed_solution/main}
% \include{chapters/experiment/main}
% \include{chapters/conclusion/main}

%\bibliographystyle{natbib}
%\addcontentsline{toc}{chapter}{\bibliographytocname}
\bibliography{references}

% Appendix
\clearpage
\addappheadtotoc
\appendix
\appendixpage
% \include{appendix/experiment-instruments}

\end{document}
% %% RiSE Latex Template - version 0.5
%%
%% RiSE's latex template for thesis and dissertations
%% http://risetemplate.sourceforge.net
%%
%% (c) 2012 Yguaratã Cerqueira Cavalcanti (yguarata@gmail.com)
%%          Vinicius Cardoso Garcia (vinicius.garcia@gmail.com)
%%
%% This document was initially based on UFPEThesis template, from Paulo Gustavo
%% S. Fonseca.
%%
%% ACKNOWLEDGEMENTS
%%
%% We would like to thanks the RiSE's researchers community, the 
%% students from Federal University of Pernambuco, and other users that have
%% been contributing to this projects with comments and patches.
%%
%% GENERAL INSTRUCTIONS
%%
%% We strongly recommend you to compile your documents using pdflatex command.
%% It is also recommend use the texlipse plugin for Eclipse to edit your documents.
%%
%% Options for \documentclass command:
%%         * Idiom
%%           pt   - Portguese (default)
%%           en   - English
%%
%%         * Text type
%%           bsc  - B.Sc. Thesis
%%           msc  - M.Sc. Thesis (default)
%%           qual - PHD qualification (not tested yet)
%%           prop - PHD proposal (not tested yet)
%%           phd  - PHD thesis
%%
%%         * Media
%%           scr  - to eletronic version (PDF) / see the users guide
%%
%%         * Pagination
%%           oneside - unique face press
%%           twoside - two faces press
%%
%%		   * Line spacing
%%           singlespacing  - the same as using \linespread{1}
%%           onehalfspacing - the same as using \linespread{1.3}
%%           doublespacing  - the same as using \linespread{1.6}
%%
%% Reference commands. Use the following commands to make references in your
%% text:
%%          \figref  -- for Figure reference
%%          \tabref  -- for Table reference
%%          \eqnref  -- for equation reference
%%          \chapref -- for chapter reference
%%          \secref  -- for section reference
%%          \appref  -- for appendix reference
%%          \axiref  -- for axiom reference
%%          \conjref -- for conjecture reference
%%          \defref  -- for definition reference
%%          \lemref  -- for lemma reference
%%          \theoref -- for theorem reference
%%          \corref  -- for corollary reference
%%          \propref -- for proprosition reference
%%          \pgref   -- for page reference
%%
%%          Example: See \chapref{chap:introduction}. It will produce 
%%                   'See Chapter 1', in case of English language.

\documentclass[pt,twoside,onehalfspacing,bsc]{risethesis}

\usepackage[sort,square]{natbib} %package for bibliography
\bibliographystyle{ieeetr} %authordate1

\usepackage[utf8]{inputenc}
\usepackage[brazilian]{babel}
\usepackage[T1]{fontenc}

\usepackage{graphicx}

%% Change the following pdf author attribute name to your name.
\usepackage[linkcolor=blue,citecolor=blue,urlcolor=blue,colorlinks,pdfpagelabels,pdftitle={Bruno Cabral's Bachelor Thesis},pdfauthor={Bruno Cabral}]{hyperref}

\address{SALVADOR}

\universitypt{Universidade Federal da Bahia}
\universityen{Federal University of Bahia}

\departmentpt{Depertamento de Ciência da Computação}
\departmenten{Computer Science Department}

\programpt{Programa Multiinstitucional de Pós-graduação em Ciência da Computação}
\programen{Graduate in Computer Science}

\majorfieldpt{Ciência da Computação}
\majorfielden{Computer Science}

\title{Sistema de apoio à Pós graduação - UFBA}
\date{Outubro/2016}

\author{Victor de Azevedo Nunes}
\adviser{Ivan do Carmo Machado}

\begin{document}

\frontmatter
\frontpage
\presentationpage

\begin{dedicatory}
Eu dedico esta dissertação...
%I dedicate this dissertation to my family, girlfriend, friends and
%professors who gave me all necessary support to get here.
\end{dedicatory}

\acknowledgements
Meus agradecimentos...

\begin{epigraph}[]{Edward V Berard}
Walking on water and developing software from a specification are easy if both are frozen
\end{epigraph}

\resumo
% Escreva seu resumo no arquivo resumo.tex
\input{resumo}

\abstract
% Write your abstract in a file called abstract.tex
\input{abstract}

% Summary (tables of contents)
\tableofcontents

% List of figures
\listoffigures

% List of tables
\listoftables

% List of acronyms
% Acronyms manual: http://linorg.usp.br/CTAN/macros/latex/contrib/acronym/acronym.pdf
\listofacronyms
\input{acronyms}

% List of listings
%\lstlistoflistings

\mainmatter

\include{chapters/intro}
\include{chapters/referencial_teorico}

% \include{chapters/introduction/main}
% \include{chapters/background/main}
% \include{chapters/proposed_solution/main}
% \include{chapters/experiment/main}
% \include{chapters/conclusion/main}

%\bibliographystyle{natbib}
%\addcontentsline{toc}{chapter}{\bibliographytocname}
\bibliography{references}

% Appendix
\clearpage
\addappheadtotoc
\appendix
\appendixpage
% \include{appendix/experiment-instruments}

\end{document}
% %% RiSE Latex Template - version 0.5
%%
%% RiSE's latex template for thesis and dissertations
%% http://risetemplate.sourceforge.net
%%
%% (c) 2012 Yguaratã Cerqueira Cavalcanti (yguarata@gmail.com)
%%          Vinicius Cardoso Garcia (vinicius.garcia@gmail.com)
%%
%% This document was initially based on UFPEThesis template, from Paulo Gustavo
%% S. Fonseca.
%%
%% ACKNOWLEDGEMENTS
%%
%% We would like to thanks the RiSE's researchers community, the 
%% students from Federal University of Pernambuco, and other users that have
%% been contributing to this projects with comments and patches.
%%
%% GENERAL INSTRUCTIONS
%%
%% We strongly recommend you to compile your documents using pdflatex command.
%% It is also recommend use the texlipse plugin for Eclipse to edit your documents.
%%
%% Options for \documentclass command:
%%         * Idiom
%%           pt   - Portguese (default)
%%           en   - English
%%
%%         * Text type
%%           bsc  - B.Sc. Thesis
%%           msc  - M.Sc. Thesis (default)
%%           qual - PHD qualification (not tested yet)
%%           prop - PHD proposal (not tested yet)
%%           phd  - PHD thesis
%%
%%         * Media
%%           scr  - to eletronic version (PDF) / see the users guide
%%
%%         * Pagination
%%           oneside - unique face press
%%           twoside - two faces press
%%
%%		   * Line spacing
%%           singlespacing  - the same as using \linespread{1}
%%           onehalfspacing - the same as using \linespread{1.3}
%%           doublespacing  - the same as using \linespread{1.6}
%%
%% Reference commands. Use the following commands to make references in your
%% text:
%%          \figref  -- for Figure reference
%%          \tabref  -- for Table reference
%%          \eqnref  -- for equation reference
%%          \chapref -- for chapter reference
%%          \secref  -- for section reference
%%          \appref  -- for appendix reference
%%          \axiref  -- for axiom reference
%%          \conjref -- for conjecture reference
%%          \defref  -- for definition reference
%%          \lemref  -- for lemma reference
%%          \theoref -- for theorem reference
%%          \corref  -- for corollary reference
%%          \propref -- for proprosition reference
%%          \pgref   -- for page reference
%%
%%          Example: See \chapref{chap:introduction}. It will produce 
%%                   'See Chapter 1', in case of English language.

\documentclass[pt,twoside,onehalfspacing,bsc]{risethesis}

\usepackage[sort,square]{natbib} %package for bibliography
\bibliographystyle{ieeetr} %authordate1

\usepackage[utf8]{inputenc}
\usepackage[brazilian]{babel}
\usepackage[T1]{fontenc}

\usepackage{graphicx}

%% Change the following pdf author attribute name to your name.
\usepackage[linkcolor=blue,citecolor=blue,urlcolor=blue,colorlinks,pdfpagelabels,pdftitle={Bruno Cabral's Bachelor Thesis},pdfauthor={Bruno Cabral}]{hyperref}

\address{SALVADOR}

\universitypt{Universidade Federal da Bahia}
\universityen{Federal University of Bahia}

\departmentpt{Depertamento de Ciência da Computação}
\departmenten{Computer Science Department}

\programpt{Programa Multiinstitucional de Pós-graduação em Ciência da Computação}
\programen{Graduate in Computer Science}

\majorfieldpt{Ciência da Computação}
\majorfielden{Computer Science}

\title{Sistema de apoio à Pós graduação - UFBA}
\date{Outubro/2016}

\author{Victor de Azevedo Nunes}
\adviser{Ivan do Carmo Machado}

\begin{document}

\frontmatter
\frontpage
\presentationpage

\begin{dedicatory}
Eu dedico esta dissertação...
%I dedicate this dissertation to my family, girlfriend, friends and
%professors who gave me all necessary support to get here.
\end{dedicatory}

\acknowledgements
Meus agradecimentos...

\begin{epigraph}[]{Edward V Berard}
Walking on water and developing software from a specification are easy if both are frozen
\end{epigraph}

\resumo
% Escreva seu resumo no arquivo resumo.tex
\input{resumo}

\abstract
% Write your abstract in a file called abstract.tex
\input{abstract}

% Summary (tables of contents)
\tableofcontents

% List of figures
\listoffigures

% List of tables
\listoftables

% List of acronyms
% Acronyms manual: http://linorg.usp.br/CTAN/macros/latex/contrib/acronym/acronym.pdf
\listofacronyms
\input{acronyms}

% List of listings
%\lstlistoflistings

\mainmatter

\include{chapters/intro}
\include{chapters/referencial_teorico}

% \include{chapters/introduction/main}
% \include{chapters/background/main}
% \include{chapters/proposed_solution/main}
% \include{chapters/experiment/main}
% \include{chapters/conclusion/main}

%\bibliographystyle{natbib}
%\addcontentsline{toc}{chapter}{\bibliographytocname}
\bibliography{references}

% Appendix
\clearpage
\addappheadtotoc
\appendix
\appendixpage
% \include{appendix/experiment-instruments}

\end{document}
% %% RiSE Latex Template - version 0.5
%%
%% RiSE's latex template for thesis and dissertations
%% http://risetemplate.sourceforge.net
%%
%% (c) 2012 Yguaratã Cerqueira Cavalcanti (yguarata@gmail.com)
%%          Vinicius Cardoso Garcia (vinicius.garcia@gmail.com)
%%
%% This document was initially based on UFPEThesis template, from Paulo Gustavo
%% S. Fonseca.
%%
%% ACKNOWLEDGEMENTS
%%
%% We would like to thanks the RiSE's researchers community, the 
%% students from Federal University of Pernambuco, and other users that have
%% been contributing to this projects with comments and patches.
%%
%% GENERAL INSTRUCTIONS
%%
%% We strongly recommend you to compile your documents using pdflatex command.
%% It is also recommend use the texlipse plugin for Eclipse to edit your documents.
%%
%% Options for \documentclass command:
%%         * Idiom
%%           pt   - Portguese (default)
%%           en   - English
%%
%%         * Text type
%%           bsc  - B.Sc. Thesis
%%           msc  - M.Sc. Thesis (default)
%%           qual - PHD qualification (not tested yet)
%%           prop - PHD proposal (not tested yet)
%%           phd  - PHD thesis
%%
%%         * Media
%%           scr  - to eletronic version (PDF) / see the users guide
%%
%%         * Pagination
%%           oneside - unique face press
%%           twoside - two faces press
%%
%%		   * Line spacing
%%           singlespacing  - the same as using \linespread{1}
%%           onehalfspacing - the same as using \linespread{1.3}
%%           doublespacing  - the same as using \linespread{1.6}
%%
%% Reference commands. Use the following commands to make references in your
%% text:
%%          \figref  -- for Figure reference
%%          \tabref  -- for Table reference
%%          \eqnref  -- for equation reference
%%          \chapref -- for chapter reference
%%          \secref  -- for section reference
%%          \appref  -- for appendix reference
%%          \axiref  -- for axiom reference
%%          \conjref -- for conjecture reference
%%          \defref  -- for definition reference
%%          \lemref  -- for lemma reference
%%          \theoref -- for theorem reference
%%          \corref  -- for corollary reference
%%          \propref -- for proprosition reference
%%          \pgref   -- for page reference
%%
%%          Example: See \chapref{chap:introduction}. It will produce 
%%                   'See Chapter 1', in case of English language.

\documentclass[pt,twoside,onehalfspacing,bsc]{risethesis}

\usepackage[sort,square]{natbib} %package for bibliography
\bibliographystyle{ieeetr} %authordate1

\usepackage[utf8]{inputenc}
\usepackage[brazilian]{babel}
\usepackage[T1]{fontenc}

\usepackage{graphicx}

%% Change the following pdf author attribute name to your name.
\usepackage[linkcolor=blue,citecolor=blue,urlcolor=blue,colorlinks,pdfpagelabels,pdftitle={Bruno Cabral's Bachelor Thesis},pdfauthor={Bruno Cabral}]{hyperref}

\address{SALVADOR}

\universitypt{Universidade Federal da Bahia}
\universityen{Federal University of Bahia}

\departmentpt{Depertamento de Ciência da Computação}
\departmenten{Computer Science Department}

\programpt{Programa Multiinstitucional de Pós-graduação em Ciência da Computação}
\programen{Graduate in Computer Science}

\majorfieldpt{Ciência da Computação}
\majorfielden{Computer Science}

\title{Sistema de apoio à Pós graduação - UFBA}
\date{Outubro/2016}

\author{Victor de Azevedo Nunes}
\adviser{Ivan do Carmo Machado}

\begin{document}

\frontmatter
\frontpage
\presentationpage

\begin{dedicatory}
Eu dedico esta dissertação...
%I dedicate this dissertation to my family, girlfriend, friends and
%professors who gave me all necessary support to get here.
\end{dedicatory}

\acknowledgements
Meus agradecimentos...

\begin{epigraph}[]{Edward V Berard}
Walking on water and developing software from a specification are easy if both are frozen
\end{epigraph}

\resumo
% Escreva seu resumo no arquivo resumo.tex
\input{resumo}

\abstract
% Write your abstract in a file called abstract.tex
\input{abstract}

% Summary (tables of contents)
\tableofcontents

% List of figures
\listoffigures

% List of tables
\listoftables

% List of acronyms
% Acronyms manual: http://linorg.usp.br/CTAN/macros/latex/contrib/acronym/acronym.pdf
\listofacronyms
\input{acronyms}

% List of listings
%\lstlistoflistings

\mainmatter

\include{chapters/intro}
\include{chapters/referencial_teorico}

% \include{chapters/introduction/main}
% \include{chapters/background/main}
% \include{chapters/proposed_solution/main}
% \include{chapters/experiment/main}
% \include{chapters/conclusion/main}

%\bibliographystyle{natbib}
%\addcontentsline{toc}{chapter}{\bibliographytocname}
\bibliography{references}

% Appendix
\clearpage
\addappheadtotoc
\appendix
\appendixpage
% \include{appendix/experiment-instruments}

\end{document}
% %% RiSE Latex Template - version 0.5
%%
%% RiSE's latex template for thesis and dissertations
%% http://risetemplate.sourceforge.net
%%
%% (c) 2012 Yguaratã Cerqueira Cavalcanti (yguarata@gmail.com)
%%          Vinicius Cardoso Garcia (vinicius.garcia@gmail.com)
%%
%% This document was initially based on UFPEThesis template, from Paulo Gustavo
%% S. Fonseca.
%%
%% ACKNOWLEDGEMENTS
%%
%% We would like to thanks the RiSE's researchers community, the 
%% students from Federal University of Pernambuco, and other users that have
%% been contributing to this projects with comments and patches.
%%
%% GENERAL INSTRUCTIONS
%%
%% We strongly recommend you to compile your documents using pdflatex command.
%% It is also recommend use the texlipse plugin for Eclipse to edit your documents.
%%
%% Options for \documentclass command:
%%         * Idiom
%%           pt   - Portguese (default)
%%           en   - English
%%
%%         * Text type
%%           bsc  - B.Sc. Thesis
%%           msc  - M.Sc. Thesis (default)
%%           qual - PHD qualification (not tested yet)
%%           prop - PHD proposal (not tested yet)
%%           phd  - PHD thesis
%%
%%         * Media
%%           scr  - to eletronic version (PDF) / see the users guide
%%
%%         * Pagination
%%           oneside - unique face press
%%           twoside - two faces press
%%
%%		   * Line spacing
%%           singlespacing  - the same as using \linespread{1}
%%           onehalfspacing - the same as using \linespread{1.3}
%%           doublespacing  - the same as using \linespread{1.6}
%%
%% Reference commands. Use the following commands to make references in your
%% text:
%%          \figref  -- for Figure reference
%%          \tabref  -- for Table reference
%%          \eqnref  -- for equation reference
%%          \chapref -- for chapter reference
%%          \secref  -- for section reference
%%          \appref  -- for appendix reference
%%          \axiref  -- for axiom reference
%%          \conjref -- for conjecture reference
%%          \defref  -- for definition reference
%%          \lemref  -- for lemma reference
%%          \theoref -- for theorem reference
%%          \corref  -- for corollary reference
%%          \propref -- for proprosition reference
%%          \pgref   -- for page reference
%%
%%          Example: See \chapref{chap:introduction}. It will produce 
%%                   'See Chapter 1', in case of English language.

\documentclass[pt,twoside,onehalfspacing,bsc]{risethesis}

\usepackage[sort,square]{natbib} %package for bibliography
\bibliographystyle{ieeetr} %authordate1

\usepackage[utf8]{inputenc}
\usepackage[brazilian]{babel}
\usepackage[T1]{fontenc}

\usepackage{graphicx}

%% Change the following pdf author attribute name to your name.
\usepackage[linkcolor=blue,citecolor=blue,urlcolor=blue,colorlinks,pdfpagelabels,pdftitle={Bruno Cabral's Bachelor Thesis},pdfauthor={Bruno Cabral}]{hyperref}

\address{SALVADOR}

\universitypt{Universidade Federal da Bahia}
\universityen{Federal University of Bahia}

\departmentpt{Depertamento de Ciência da Computação}
\departmenten{Computer Science Department}

\programpt{Programa Multiinstitucional de Pós-graduação em Ciência da Computação}
\programen{Graduate in Computer Science}

\majorfieldpt{Ciência da Computação}
\majorfielden{Computer Science}

\title{Sistema de apoio à Pós graduação - UFBA}
\date{Outubro/2016}

\author{Victor de Azevedo Nunes}
\adviser{Ivan do Carmo Machado}

\begin{document}

\frontmatter
\frontpage
\presentationpage

\begin{dedicatory}
Eu dedico esta dissertação...
%I dedicate this dissertation to my family, girlfriend, friends and
%professors who gave me all necessary support to get here.
\end{dedicatory}

\acknowledgements
Meus agradecimentos...

\begin{epigraph}[]{Edward V Berard}
Walking on water and developing software from a specification are easy if both are frozen
\end{epigraph}

\resumo
% Escreva seu resumo no arquivo resumo.tex
\input{resumo}

\abstract
% Write your abstract in a file called abstract.tex
\input{abstract}

% Summary (tables of contents)
\tableofcontents

% List of figures
\listoffigures

% List of tables
\listoftables

% List of acronyms
% Acronyms manual: http://linorg.usp.br/CTAN/macros/latex/contrib/acronym/acronym.pdf
\listofacronyms
\input{acronyms}

% List of listings
%\lstlistoflistings

\mainmatter

\include{chapters/intro}
\include{chapters/referencial_teorico}

% \include{chapters/introduction/main}
% \include{chapters/background/main}
% \include{chapters/proposed_solution/main}
% \include{chapters/experiment/main}
% \include{chapters/conclusion/main}

%\bibliographystyle{natbib}
%\addcontentsline{toc}{chapter}{\bibliographytocname}
\bibliography{references}

% Appendix
\clearpage
\addappheadtotoc
\appendix
\appendixpage
% \include{appendix/experiment-instruments}

\end{document}

%\bibliographystyle{natbib}
%\addcontentsline{toc}{chapter}{\bibliographytocname}
\bibliography{references}

% Appendix
\clearpage
\addappheadtotoc
\appendix
\appendixpage
% \include{appendix/experiment-instruments}

\end{document}
% %% RiSE Latex Template - version 0.5
%%
%% RiSE's latex template for thesis and dissertations
%% http://risetemplate.sourceforge.net
%%
%% (c) 2012 Yguaratã Cerqueira Cavalcanti (yguarata@gmail.com)
%%          Vinicius Cardoso Garcia (vinicius.garcia@gmail.com)
%%
%% This document was initially based on UFPEThesis template, from Paulo Gustavo
%% S. Fonseca.
%%
%% ACKNOWLEDGEMENTS
%%
%% We would like to thanks the RiSE's researchers community, the 
%% students from Federal University of Pernambuco, and other users that have
%% been contributing to this projects with comments and patches.
%%
%% GENERAL INSTRUCTIONS
%%
%% We strongly recommend you to compile your documents using pdflatex command.
%% It is also recommend use the texlipse plugin for Eclipse to edit your documents.
%%
%% Options for \documentclass command:
%%         * Idiom
%%           pt   - Portguese (default)
%%           en   - English
%%
%%         * Text type
%%           bsc  - B.Sc. Thesis
%%           msc  - M.Sc. Thesis (default)
%%           qual - PHD qualification (not tested yet)
%%           prop - PHD proposal (not tested yet)
%%           phd  - PHD thesis
%%
%%         * Media
%%           scr  - to eletronic version (PDF) / see the users guide
%%
%%         * Pagination
%%           oneside - unique face press
%%           twoside - two faces press
%%
%%		   * Line spacing
%%           singlespacing  - the same as using \linespread{1}
%%           onehalfspacing - the same as using \linespread{1.3}
%%           doublespacing  - the same as using \linespread{1.6}
%%
%% Reference commands. Use the following commands to make references in your
%% text:
%%          \figref  -- for Figure reference
%%          \tabref  -- for Table reference
%%          \eqnref  -- for equation reference
%%          \chapref -- for chapter reference
%%          \secref  -- for section reference
%%          \appref  -- for appendix reference
%%          \axiref  -- for axiom reference
%%          \conjref -- for conjecture reference
%%          \defref  -- for definition reference
%%          \lemref  -- for lemma reference
%%          \theoref -- for theorem reference
%%          \corref  -- for corollary reference
%%          \propref -- for proprosition reference
%%          \pgref   -- for page reference
%%
%%          Example: See \chapref{chap:introduction}. It will produce 
%%                   'See Chapter 1', in case of English language.

\documentclass[pt,twoside,onehalfspacing,bsc]{risethesis}

\usepackage[sort,square]{natbib} %package for bibliography
\bibliographystyle{ieeetr} %authordate1

\usepackage[utf8]{inputenc}
\usepackage[brazilian]{babel}
\usepackage[T1]{fontenc}

\usepackage{graphicx}

%% Change the following pdf author attribute name to your name.
\usepackage[linkcolor=blue,citecolor=blue,urlcolor=blue,colorlinks,pdfpagelabels,pdftitle={Bruno Cabral's Bachelor Thesis},pdfauthor={Bruno Cabral}]{hyperref}

\address{SALVADOR}

\universitypt{Universidade Federal da Bahia}
\universityen{Federal University of Bahia}

\departmentpt{Depertamento de Ciência da Computação}
\departmenten{Computer Science Department}

\programpt{Programa Multiinstitucional de Pós-graduação em Ciência da Computação}
\programen{Graduate in Computer Science}

\majorfieldpt{Ciência da Computação}
\majorfielden{Computer Science}

\title{Sistema de apoio à Pós graduação - UFBA}
\date{Outubro/2016}

\author{Victor de Azevedo Nunes}
\adviser{Ivan do Carmo Machado}

\begin{document}

\frontmatter
\frontpage
\presentationpage

\begin{dedicatory}
Eu dedico esta dissertação...
%I dedicate this dissertation to my family, girlfriend, friends and
%professors who gave me all necessary support to get here.
\end{dedicatory}

\acknowledgements
Meus agradecimentos...

\begin{epigraph}[]{Edward V Berard}
Walking on water and developing software from a specification are easy if both are frozen
\end{epigraph}

\resumo
% Escreva seu resumo no arquivo resumo.tex
Este trabalho de conclusão de curso utiliza elementos da engenharia de software para propor um software como serviço a fim de otimizar o processo de construção e manutenção dos softwares. Assim, o objetivo deste SaaS é gerenciar as permissões de acesso de sistemas cliente, provendo o reuso de software e padronizando as soluções. O sistema proposto neste trabalho encontra-se implementado e disponível no Github, e traz fundamentos sobre a arquitetura e tecnologias utilizadas, bem como avaliações sobre possibilidades de utilização do mesmo em diversos ambientes, como Web e mobile.

\begin{keywords}
Software; Reuso; SaaS; Web; PHP
\end{keywords}

\abstract
% Write your abstract in a file called abstract.tex
My abstract...

\begin{keywords}
key words...
\end{keywords}

% Summary (tables of contents)
\tableofcontents

% List of figures
\listoffigures

% List of tables
\listoftables

% List of acronyms
% Acronyms manual: http://linorg.usp.br/CTAN/macros/latex/contrib/acronym/acronym.pdf
\listofacronyms
\begin{acronym}[ACRONYM] 
% Change the word ACRONYM above to change the acronym column width.
% The column width is equals to the width of the word that you put.
% Read the manual about acronym package for more examples:
%   http://linorg.usp.br/CTAN/macros/latex/contrib/acronym/acronym.pdf

\acro{SPA}{Single Page Application}
\acro{JSON}{Javascript Object Notation}
\acro{PHP}{PHP: Hypertext Preprocessor}
\acro{SaaS}{Software as a Service}
\acro{ERP}{Enterprise Resource Planning}
\acro{QoS}{Quality of Service}
\acro{UML}{Unified Modeling Language}
\acro{MVC}{Model-View-Controller}
\acro{Ajax}{Asynchronous Javascript and XML}
\acro{HTML}{HyperText Markup Language}
\acro{CSS}{Cascading Style Sheets}
\acro{API}{Application Programming Interface}
\acro{DOM}{Document Object Model}
\acro{BPMN}{Business Process Model and Notation}
\acro{REST}{Representational State Transfer}

\end{acronym}

% List of listings
%\lstlistoflistings

\mainmatter

\chapter{Introdução}

\section{Motivação}

Organizar os procedimentos de um processo sempre nos traz vantagens. Apesar de no processo de implantação de um sistema, o mesmo burocratizar o processo, com o tempo temos o retorno da dedicação para a inserção dos dados. Com um certo volume de dados, é possível estruturar informações que num processo manual são difíceis de serem enxergadas. Assim, é possível depender menos das pessoas que organizam o processo, pois o legado de informações não estará mais somente na mente de alguns, mas sim documentado nos dados do sistema.

Além de colaborar na organização, também haverá uma grande colaboração no tempo gasto na gestão. Lidar com muitos papéis e confiar na mente humana para guardar informações, não é uma alternativa muito segura devido ao fato que as pessoas sempre estão sujeitas a sair do processo e levar contigo a experiência obtida. Experiência essa que faz com que os procedimentos sejam executados de forma mais eficiente. Entretanto, com um sistema inteligente, é possível auxiliar e tornar mais ágil a execução das tarefas.


\section{Problema}


De acordo com funcionários ligados ao o setor de pós graduação da UFBA, entrevistados a fim de um maior entendimento do cenário, apesar das semelhanças estruturais, a pós graduação gerida de forma diferencia da graduação. FULANO afirma que devido ao fato de não ter a mesma visibilidade, não tem acesso aos mesmos recursos de gestão acadêmica da graduação. O professores não executam somente atividades dentro da sala de aula, também tem diversas outras ocupações no setor. E muitos procedimentos realizados extra classe ainda se encontram sendo realizados de forma manual, estando mais vulnerável ao erro ou até mesmo à violação do processo. Também ocorre um grande desperdício de tempo pelos professores e gestores da área, devido ao diversos processos ainda realizados de forma manual, sem a devida documentação. Segundo FULANO, também entrevistado, esse tempo perdido implica numa redução da eficiência na sala de aula, pois o professor acaba por ter menos tempo disponível para o planejamento das atividades, o que gera impactos negativos aos alunos.


\section{Objetivos} %<o que deve ser feito/entregue>


Devido aos muitos processos sendo resolvidos de forma manual, propõe-se com solução um sistema moderno, arquitetado para ter funcionamento na web e com um módulo mobile, a fim de fornecer informações de forma rápida e eficiente para os professores através de notificações, já que o acesso à internet móvel é comum entre os possíveis usuários do sistema em questão.
O principal requisito para o sistema seria dispor recursos para reduzir o tempo desperdiçado pelos professores durante as atividades extra classe.


\section{Metodologia} %<como será feito | como resolver o problema apontado inicialmente>


%<analise de literatura | design | implementação | validação>
Baseando-se nas tecnologias gratuitas em alta no cenário atual do desenvolvimento web, dispomos de algumas opções eficientes para a implementação da solução. Dentre as possibilidades, considerando a facilidade para futura manutenção e continuidade do projeto, tende-se a optar por uma tecnologia popular. Como linguagem de programação, adota-se o PHP. A escolha é fundamentada de acordo com a pesquisa da RedMonk de 2015, que evidencia o uso das linguagens de programação de acordo com as discussões no StackOverflow e repositórios no GitHub. É possível constatar a popularidade do PHP no cenário atual com o gráfico da pesquisa citada, na qual o PHP é apresentado na terceira colocação, apenas atrás do lider JavaScript e do segundo colocado, o Java.

\begin{figure}
	\label{fig:graficoRedmonk}
	\includegraphics[width=1\textwidth]{img/grafico_redmonk}
	\caption{Ranking das liguagens de programação no Stack Overflow e Github}
\end{figure}


Ainda assim, para compor a interface do dado projeto, também ocorrerá o uso do líder JavaScript de forma intensa, provendo o elo com o as informações gerenciadas pelo PHP.


Entretanto, não seria inteligente desenvolver um sistema completo sem o auxílio de um framework. Dentre os frameworks disponíveis para PHP, hoje o destaque está com o Laravel, que se encontra no topo dentre os mais utilizados no momento. 


A WebHostFace, uma empresa de hospedagem, compilou várias estatísticas para criar um infográfico mostrando os frameworks PHP mais populares de 2015. Utilizando informações sobre os próprios clientes, o Google Trends, estatísticas de repositórios do GitHub e a pesquisa do SitePoint “Best PHP Frameworks 2015”, a WebHostFace elaborou o seguinte infográfico: 

\begin{figure}
	\label{fig:graficoWebhostface}
	\includegraphics[width=1\textwidth]{img/infografico_webhostface}
	\caption{Infográfico da WebhostFace, exibindo a popularidade dos Frameworks PHP em 2015}
\end{figure}

Assim, tem-se a evidência que o Laravel em 2015 teve a maior popularidade em projetos pessoais e tem a maior comunidade entre os concorrentes, o que o torna uma boa escolha para a escrita de um software que será continuado por terceiros.


Para elaborar os recursos de interface e integrar ao back-end PHP do sistema, será adotado o já conhecido AngularJS, ferramenta sólida e conhecida no aspecto em questão. 


Dados coletados via Google Trends, que propõe comparações entre termos pesquisados, revela a popularidade do AngularJs diante de alguns dos principais concorrentes. O gráfico abaixo evidencia o cenário.


%Como mostra a Figura \ref{fig:graficoGoogleTrendsFerramentasFront}. 
\begin{figure}
	\label{fig:graficoGoogleTrendsFerramentasFront}
	\includegraphics[width=1\textwidth]{img/grafico_ferramentas_front}
	\caption{Gráfico do Google Trends exibindo as pesquisas por ferramentas front-end}
\end{figure}


Junto ao Angular JS, será utilizada a agradável tendência de interface do Material Design da Google, que propõe layouts limpos e otimizados já conhecidos pelos usuários de smartphones Android. 


Para a elaboração da plataforma mobile do projeto, será utilizado o Ionic Framework, muito difundido e bastante pesquisado na área, o que fica evidenciado com o gráfico de pesquisbaixo, coletado via Google Trends buscando por frameworks de desenvolvimento híbrido mobile.


\begin{figure}
	\label{fig:graficoGoogleTrendsFerramentasHibridasMobile}
	\includegraphics[width=1\textwidth]{img/grafico_ferramentas_hibridas_mobile}
	\caption{Gráfico do Google Trends exibindo as pesquisas por Frameworks híbridos mobile}
\end{figure}	

Para layout da interface mobile, também será aplicado a tendência do Material Design, a fim de propor uma harmonia entre o módulo web e mobile para os usuários


\section{Resultados Esperados}


Como fruto de um sistema para pós-graduação da UFBA, espera-se que os professores tenham mais recursos para integrar as atividades e também prover melhores condições para acompanhamento da vida acadêmica dos alunos em questão. Também, que os novos colaboradores que entrarem no processo tenham facilidade de compreender o fluxo do setor ao navegar pelo sistema proposto.


\section{Fora de Escopo}


Interação com os alunos devido às complicações para realizar a integração com o sistema empregado na UFBA, gerenciado pela XXXXXX, o que causaria uma inviabilidade no projeto devido à necessidade de entrega do produto ser mais forte que o tempo necessário para executar o processo de obtenção de acesso ao sistema legado para realizar a integração.


\section{Estrutura do Trabalho}


<breve resumo sobre os capítulos do TCC>
\chapter{Referencial Teórico}


Projetar o desenvolvimento de um software requer muito planejamento, pois as falhas iniciais podem custar bastante caro ou até mesmo inviabilizar a continuação de um projeto. Assim, a escolha da arquitetura ideal para a aplicabilidade é essencial na concepção de um produto de software. 
De todo o modo, sempre busca-se fazer mais com menos. Diante de tal filosofia, temos neste capítulo, uma breve discussão sobre alguns elementos de projeto e arquitetura de software, a fim de contextualizar este trabalho de conclusão de curso.
O capítulo corrente é composto por quatro seções. A \ref{sec:saas} trata de Software como serviço, discutindo alguns elementos do contexto que são relevantes para o trabalho proposto. A \ref{sec:reuso}  discute sobre a empregabilidade do reuso de software. A \ref{sec:modularizacao} seção, trata sobre aspectos envolvidos na modularização dos softwares. Por fim, a \ref{sec:apps_web} aborda as aplicações web, discutindo sobre aspectos relevantes sobre a aplicação web que compõe este trabalho.


 \section{Software como serviço}\label{sec:saas}


Segundo La e Chun \citep{La2009Systematic}, o princípio da definição de Software como um Serviço (Sofware as a Service - SaaS) é um serviço complementar para aplicações da computação em nuvem (cloud computing). No entanto, as áreas não se confundem. SaaS deve ser entendido como um mecanismo de suporte às soluções existentes na cloud. Os SaaS existem justamente para maximizar o reuso de serviços repetidos e não centrais em uma aplicação remota.


Como vantagens, diversos fatores podem ser favoráveis para a adoção de um SaaS, como custo e manutenção dentre outros fatores aplicáveis a soluções específicas. Lechesaet al. \citep{LechesaSS11}, quantificam índices sobre os fatores determinantes para adoção ou não de um SaaS voltado para ERP na África do Sul. Os principais fatores determinantes para adoção desse mecanismo de software são sua fluidez quanto à rede e a segurança. Esses fatores estão presentes na aplicação desenvolvida neste trabalho de conclusão de curso.


Devido ao fato de ter um serviço constantemente na nuvem, fica o questionamento sobre a segurança da informação manipulada. Sabe-se que a vulnerabilidade na Web não é restrita ao SaaS, atingindo diversos âmbitos. Rai et al. \cite{journals/corr/RaiSM13} consideram que o avanço da computação em nuvem não é um problema apenas para os serviços Web do ponto de vista da segurança, pois muitos trabalhos na literatura mostram a área como mais um ponto de vulnerabilidade para diversos setores, a exemplo de infraestrutura. No mesmo artigo mencionado de Rai et al. \cite{journals/corr/RaiSM13}, também realizaram-se estudos exploratórios junto a empresas usuárias de serviços em computação em nuvem e consideram que a perspectiva de SaaS também pode fortalecer a segurança nas aplicações de cloud computing, pois o software de autenticação compartilhado por várias aplicações em nuvem, oferece uma melhor padronização e consequente facilidade de prevenção a erros de vulnerabilidade específicas de cada módulo da pesquisa. Esse ponto de vista é fundamenta para a compreensão de SaaS.


A arquitetura de armazenamento de dados de um Saas pode variar de acordo com a necessidade do contexto. Huixin \cite{7586486} descreve possíveis modelagens nesse sentido. Essa abordagem pode ser com um banco de dados único, fazendo com que diferentes clientes compartilhem o mesmo banco, diferindo os dados através de controle de usuário, ou isolando os diferentes clientes através de bancos de dados exclusivos para cada um. Esse fator também pode ser combinado com a arquitetura da aplicação, caso ofereça aplicação única para todos os clientes ou aplicação compartilhada. Diante das possíveis abordagens, a modelagem de dados do software pode ser decidida pela regra de negócio. Este trabalho optou por aplicação única e banco de dados compartilhado.

Devido ao conceito particular de obtenção de software, tanto pela visão do cliente como pela visão do vendedor, é necessário compreender os diversos aspectos que podem ser relevantes ao orçar um Saas. O recente trabalho de T. Kaur et al. \citep{6949281} orienta um modelo para compor o custo de um Saas. O custo total seria composto pelos fatores que dão suporte ao funcionamento do software. Tais fatores incluem infra-estrutura, configurabilidade, customização, parâmetros de QoS (Quality of service) como escalabilidade, disponibilidade, usabilidade, pontualidade e desempenho da resposta, portabilidade, custo total de propriedade e retorno do investimento. Esses fatores caracterizam o custo de forma eficaz, possibilitando ao fornecedor, prover um Serviço de acordo com a exigência do consumidor em vários pacotes de serviços.


O conceito de software como serviço se aplica neste trabalho de conclusão de curso, pois o mesmo estará disponível na web com alta disponibilidade adotando as características apresentadas para qualquer pessoa que desejar usá-lo. Assim, basta se cadastrar, configurá-lo e usar como um serviço, sem preocupação com a manutenção do mesmo.


\section{Reuso de software}\label{sec:reuso} %CRUISE BOOK CAPITULO 2


De acordo com o livro Software pratical reuse \citep{Ezran2002}, o reuso de software é a utilização de qualquer informação que um desenvolvedor pode necessitar no processo de criação de software. O livro de Basili e Rombach \citep{Basili1991} define reutilização de software como o uso de tudo o que está associado a um projeto de conhecimento.
Assim, o objetivo da reutilização de software é reciclar o design, código e outros componentes de um produto de software e assim reduzir o custo, o tempo e melhorar a qualidade do produto.
Segundo Keswani et al. \cite{6783445}, o componente reutilizável de software pode ser qualquer parte de seu desenvolvimento, como um fragmento de código, design, casos de teste, ou até mesmo a especificação de requisitos de uma funcionalidade do software. 

O reuso de software pode ter impacto positivo em diversos aspectos do software, vejamos alguns, conforme apresentados no C.R.U.I.S.E Book \citep{cruise2007} :

\begin{itemize}

\item Qualidade: As correções de erro tornam-se úteis em todos os locais em que ocorreu, padronizando e facilitando a manutenção.

\item Produtividade: O ganho de produtividade é alcançado devido ao menor número de artefatos desenvolvido. Isso resulta em menor esforço de teste e também análise e design, reduzindo custos.

\item Confiabilidade: A utilização de componentes bem testados aumenta a
confiança no software. Além disso, a utilização de um mesmo componente em vários sistemas, aumenta a possibilidade de detecção de erros e reforça a confiança no componente.

\item Redução do Esforço: A reutilização de software proporciona uma redução do tempo de desenvolvimento, o que reduz o tempo necessário para o produto ser disponibilizado no mercado para trazer rentabilidade.

\item Trabalho redundante e tempo de desenvolvimento: Desenvolver um sistema do
zero significa desenvolvimento redundante de muitos componentes, como requisitos, especificações, casos de uso, arquitetura, etc. Isso pode ser evitado quando estes estão disponíveis como componentes reutilizáveis e podem ser compartilhados, resultando em um processo de desenvolvimento otimizado.

\item Documentação: Embora a documentação seja muito importante para a
manutenção de um sistema, muitas vezes é negligenciada. A reutilização de componentes de software reduz a quantidade de documentação a ser escrita, entretanto depende da qualidade do que está escrito. Assim, apenas a estrutura do sistema e os novos artefatos desenvolvidos necessitam ser documentados.

\item Custo de manutenção: Menos defeitos e manutenções são esperados quando tem-se comprovada a qualidade dos componentes utilizados.

\item Tamanho da equipe: É comum haver casos em que a equipe de desenvolvimento sofre sobrecarga. Entretanto, dobrar o tamanho da equipe de desenvolvimento não necessariamente duplica produtividade. Se muitos componentes podem ser reutilizados, é possível desenvolver com equipes menores, levando a melhor comunicação e aumento da produtividade.

\end{itemize}

Apesar dos benefícios da reutilização de software, ela não é suficientemente aproveitada. Existem fatores que influenciam direta ou indiretamente na sua adoção. Esses fatores podem ser de aspecto gerencial, organizacional, econômico, conceitual ou técnico. Veremos a seguir alguns aspectos que podem gerar conflito com a cultura de reuso de software, segundo o C.R.U.I.S.E Book \citep{cruise2007} :
%(Sametinger, 1997). REVER

\begin{itemize}
	
\item Falta de apoio da gestão: Como a reutilização de software gera custos iniciais,
a medida pode não ser amplamente alcançada em uma organização sem o apoio de alto nível de gestão. Os gestores têm de ser informados sobre os custos iniciais e serem convencidos sobre economias futuras.

\item Gerenciamento do Projeto: Gerenciar projetos tradicionais é uma tarefa árdua, principalmente, os que praticam a reutilização de software. Utilizando a técnica em larga escala, tem-se impacto sobre todo o ciclo de vida do software.

\item Estruturas organizacionais inadequadas: As estruturas organizacionais devem
considerar diferentes necessidades que surgem quando a reutilização em larga escala está sendo adotada. Por exemplo, uma equipe particionada pode ser alocada somente para desenvolver, manter e certificar componentes reutilizáveis de software.

\item Incentivos de gestão: É comum a falta de incentivo para deixar os desenvolvedores gastarem tempo elaborando componentes do sistemas. A produtividade é muitas vezes medida apenas no tempo necessário para concluir um projeto. Assim, fazer qualquer trabalho além disso, embora benéfico para a empresa como um todo, diminui o seu sucesso. Mesmo quando os componentes reutilizáveis são utilizados, os benefícios obtidos são uma pequena fração do que poderia ser alcançado caso houvesse reutilização explícita, planejada e organizada.

\item Dificuldade de encontrar software reutilizável: Para reutilizar os componentes, devem existir formas eficientes de busca. Além disso, é importante ter um repositório bem organizado contendo componentes com um eficiente meio de acesso.

\item Não reutilização do software encontrado. O acesso fácil ao software existente
não necessariamente aumentar a reutilização. Os componentes reutilizáveis devem ser cuidadosamente especificados, projetados, implementados e documentados, pois em alguns casos, modificar e adaptar o código  pode ser mais custoso que a programação da funcionalidade necessária a partir do zero.

\item Modificação: É muito difícil encontrar um componente que funcione
exatamente da mesma maneira que queremos. Desta forma, são necessárias modificações e devem existir formas de determinar os seus efeitos sobre o componente.


\end{itemize}


A definição do reuso de software pode ser associada a este trabalho, pois ainda que como um serviço, estaremos adotando uma aplicação única para servir aos mais diversos softwares de uma empresa, por exemplo. Logo, ao adotar o projeto, pode ser interessante aplicar algumas métricas para tomar conhecimento de possíveis vantanjens como as citadas nessa seção, a exemplo de economia de tempo de desenvolvimento, e custo do projeto.


%Outra diretriz importante para a reutilização de software é reduzir o risco na criação de novos softwares. O risco tende a ser bastante reduzido se os componentes que estão sendo reutilizados têm as documentação, interfaces necessárias e devidamente testadas, fatores que contibruem para uma fácil integração.
%De acordo com Keswani et al. \citep{6783445}, para o reuso de software dar retornos apropriados, o processo deve ser sistemático e planejado. Qualquer organização que implemente a reutilização de software deve identificar os melhores métodos e estratégias de reutilização para obter a máxima produtividade. A reutilização de software ajuda a evitar software de engenharia a partir do zero, pois usa módulos de software existentes. A reutilização de software, embora seja uma tarefa difícil, especialmente para softwares antigos sem padrões de projeto, pode melhorar significativamente a produtividade e a qualidade de um produto de software. Embora a reutilização de software não seja um novo campo, ela pode dar grandes retornos em curto período de tempo.


\section{Modularização}\label{sec:modularizacao} %artigo de claudio pagina 222 introdução


%A modularidade vem desempenhando um papel predominante estágios emergentes das disciplinas de arquitetura de software [13]. Engenheiros de software consideram modularidade como princípio base na comparação entre arquiteturas alternativas  e arquitetura degeneração [9]. De fato, os engenheiros de software são incentivados a arquitecturas, baseando-se numa multiplicidade de mecanismos de modularidade disponíveis em: 
%(i) Linguagens de descrição de arquitetura (ADLs), como ACME [8], 
%(ii) catálogos de arquitetônicos [2, 13], e 
%(iii) conhecem bem princípios de alto nível, como interfaces de componentes estreitos, acoplamento arquitectónico reduzido e semelhantes.


Conforme é frisado por Wickramaarachchi e Lai \citep{7062705}, o conceito de modularização na indústria de software tem uma longa história e tem sido utilizado para melhorar o processo de desenvolvimento de software em diferentes estágios. Os principais conceitos por trás da modularização do software foram introduzidos por pesquisadores pioneiros há quarenta anos, com uma notável contribuição feita por Melvin Conway e David Parnas, que tem representação notável na engenharia de software.


Modularizar um software é um bom padrão a ser adotado. Segundo Wickramaarachchi e Lai \citep{7062705}, a modularização é importante na identificação de dependências e reduz as dificuldades diante de uma possível necessidade de grandes alterações. De uma perspectiva da engenharia de software, uma modularização geralmente tem várias vantagens, tais como: tornar a complexidade do software mais gerenciável, facilitar o trabalho paralelo e tornar o software mais maleável para acomodar o futuro incerto que um software pode ter. O objetivo final da modularização do software é aumentar a produtividade ea qualidade do software. Tal conceito encontra-se bastante difundido e estái incorporado em linguagens de programação e ferramentas de software. O presente trabalho favorece ao uso da modularização de um software e até mesmo pode ser considerado um módulo a ser acoplado a qualquer software, mediante a compatibilidade.


Ao realizar a adoção do trabalho proposto, fica bastante evidente a "responsabilidade" do mesmo no escopo do projeto. Assim, é possível tratá-lo como um módulo do projeto que o usa, mesmo que seja consumido como um serviço.


\section{Aplicações web}\label{sec:apps_web}


A popularidade das soluções Web aumentou exponencialmente na última década e todos os dias cresce o número de pessoas usuárias desse tipo de software. E seguindo um padrão próprio, Kumar et al. \citep{7813710} sugerem que para o desenvolvimento web, deve-se manter a prática eficaz de produzir diagramas UML. A abordagem baseada na web oferece uma maneira fácil e eficaz para gerenciar e controlar o processo de desenvolvimento por meio de artefatos de modelagem. Tal abordagem pode ser usada quando há uma exigência de lidar com mudanças muito rápidas e grandes em requisitos de forma muito eficaz em muito menos tempo, gerando assim um menor impacto. 


Para atender à fomentada demanda de aplicativos web, é necessário adotar métodos de desenvolvimentos que sejam ágeis, eficientes e de fácil manutenção. Yu Ping et al. \cite{1372143} propõem o uso do modelo MVC (Model, View e Controller) no desenvolvimento para softwares web. O modelo apresentado tornou-se um padrão popular e divide o software em camadas com propósito definido, tornando-o de mais fácil manutenção.


O Ajax (Asynchronous Javascript and XML) revolucionou a web. Conforme demonstrado por Yuping \citep{6845605}, ao usar a tecnologia Ajax, podemos enriquecer a experiência do usuário em aplicações baseadas em navegador de internet, e fornecer uma variedade de aplicações interativas para atender às necessidade de humanização das aplicações.
Os aplicativos Ajax em execução no navegador se comunicam com um servidor Web de forma assíncrona e atualizam apenas uma parte da página.


No artigo de Tesarik et al. \citep{4664349}, temos que o design de software SPA (Single page application) é uma maneira interessante de criar um software Web numa uma única página. Essa solução de página única sem navegação funciona apenas com base em técnicas dinâmicas e assíncronas, como o AJAX (citado no parágrafo anterior). No entanto, esta abordagem coloca o desenvolvedor antes de alguns desafios substanciais.
Para projetar a interface do usuário que mostra as informações, é importante projetar corretamente a tela para manipular os dados do aplicativo numa única página. O design da página deve ser elaborado para maximizar a decomposição da página em componentes distintos que encapsulam os principais casos de uso.
Também recomenda-se que para elaborar os artefatos visuais, explore os mais ricos recursos na implementação, como HTML5, JavaScript, Ajax, CSS3, e outras tecnologias que se apliquem. Entretanto, desenvolver uma rica interface com o uso de diversas tecnologias/frameworks pode ocasionar um esforço maior para explorar as possibilidades.
Tais componentes da interface de uma aplicação SPA normalmente são alimentados mediante o consumo de uma API Rest via requisições AJAX. Assim, é possível particionar as responsabilidades de processamento do software entre cliente e servidor.


Como é possível denotar no trabalho de Salvadori e Siqueire \citep{7195633}, REST é uma arquitetura muito popular para integração de aplicativos web, que permite compartilhar e reutilizar informações através de sistemas. Aplicações de grande escala baseadas em no REST deve ser implementadas utilizando estratégias e mecanismos para produzir sistemas que sejam fáceis de desenvolver, reutilizar e manter. As interfaces de integração fornecidas por esses sistemas, chamadas de Web API, têm uma influência importante nas características da implementação resultante, pois a sua forma de resposta deve ser casada com a interface que o consome.
 
 
Os elementos comentados nessa seção encontram-se presentes neste trabalho de conclusão de curso. Juntos, montam a estrutura tecnológica necessária aliada à arquitetura adotada. Esses elementos seguem tendências atuais dos softwares web, fazendo com que esse trabalho esteja composto por tecnologias modernas que estão com boa aceitação no mercado.

% %% RiSE Latex Template - version 0.5
%%
%% RiSE's latex template for thesis and dissertations
%% http://risetemplate.sourceforge.net
%%
%% (c) 2012 Yguaratã Cerqueira Cavalcanti (yguarata@gmail.com)
%%          Vinicius Cardoso Garcia (vinicius.garcia@gmail.com)
%%
%% This document was initially based on UFPEThesis template, from Paulo Gustavo
%% S. Fonseca.
%%
%% ACKNOWLEDGEMENTS
%%
%% We would like to thanks the RiSE's researchers community, the 
%% students from Federal University of Pernambuco, and other users that have
%% been contributing to this projects with comments and patches.
%%
%% GENERAL INSTRUCTIONS
%%
%% We strongly recommend you to compile your documents using pdflatex command.
%% It is also recommend use the texlipse plugin for Eclipse to edit your documents.
%%
%% Options for \documentclass command:
%%         * Idiom
%%           pt   - Portguese (default)
%%           en   - English
%%
%%         * Text type
%%           bsc  - B.Sc. Thesis
%%           msc  - M.Sc. Thesis (default)
%%           qual - PHD qualification (not tested yet)
%%           prop - PHD proposal (not tested yet)
%%           phd  - PHD thesis
%%
%%         * Media
%%           scr  - to eletronic version (PDF) / see the users guide
%%
%%         * Pagination
%%           oneside - unique face press
%%           twoside - two faces press
%%
%%		   * Line spacing
%%           singlespacing  - the same as using \linespread{1}
%%           onehalfspacing - the same as using \linespread{1.3}
%%           doublespacing  - the same as using \linespread{1.6}
%%
%% Reference commands. Use the following commands to make references in your
%% text:
%%          \figref  -- for Figure reference
%%          \tabref  -- for Table reference
%%          \eqnref  -- for equation reference
%%          \chapref -- for chapter reference
%%          \secref  -- for section reference
%%          \appref  -- for appendix reference
%%          \axiref  -- for axiom reference
%%          \conjref -- for conjecture reference
%%          \defref  -- for definition reference
%%          \lemref  -- for lemma reference
%%          \theoref -- for theorem reference
%%          \corref  -- for corollary reference
%%          \propref -- for proprosition reference
%%          \pgref   -- for page reference
%%
%%          Example: See \chapref{chap:introduction}. It will produce 
%%                   'See Chapter 1', in case of English language.

\documentclass[pt,twoside,onehalfspacing,bsc]{risethesis}

\usepackage[sort,square]{natbib} %package for bibliography
\bibliographystyle{ieeetr} %authordate1

\usepackage[utf8]{inputenc}
\usepackage[brazilian]{babel}
\usepackage[T1]{fontenc}

\usepackage{graphicx}

%% Change the following pdf author attribute name to your name.
\usepackage[linkcolor=blue,citecolor=blue,urlcolor=blue,colorlinks,pdfpagelabels,pdftitle={Bruno Cabral's Bachelor Thesis},pdfauthor={Bruno Cabral}]{hyperref}

\address{SALVADOR}

\universitypt{Universidade Federal da Bahia}
\universityen{Federal University of Bahia}

\departmentpt{Depertamento de Ciência da Computação}
\departmenten{Computer Science Department}

\programpt{Programa Multiinstitucional de Pós-graduação em Ciência da Computação}
\programen{Graduate in Computer Science}

\majorfieldpt{Ciência da Computação}
\majorfielden{Computer Science}

\title{Sistema de apoio à Pós graduação - UFBA}
\date{Outubro/2016}

\author{Victor de Azevedo Nunes}
\adviser{Ivan do Carmo Machado}

\begin{document}

\frontmatter
\frontpage
\presentationpage

\begin{dedicatory}
Eu dedico esta dissertação...
%I dedicate this dissertation to my family, girlfriend, friends and
%professors who gave me all necessary support to get here.
\end{dedicatory}

\acknowledgements
Meus agradecimentos...

\begin{epigraph}[]{Edward V Berard}
Walking on water and developing software from a specification are easy if both are frozen
\end{epigraph}

\resumo
% Escreva seu resumo no arquivo resumo.tex
\input{resumo}

\abstract
% Write your abstract in a file called abstract.tex
\input{abstract}

% Summary (tables of contents)
\tableofcontents

% List of figures
\listoffigures

% List of tables
\listoftables

% List of acronyms
% Acronyms manual: http://linorg.usp.br/CTAN/macros/latex/contrib/acronym/acronym.pdf
\listofacronyms
\input{acronyms}

% List of listings
%\lstlistoflistings

\mainmatter

\include{chapters/intro}
\include{chapters/referencial_teorico}

% \include{chapters/introduction/main}
% \include{chapters/background/main}
% \include{chapters/proposed_solution/main}
% \include{chapters/experiment/main}
% \include{chapters/conclusion/main}

%\bibliographystyle{natbib}
%\addcontentsline{toc}{chapter}{\bibliographytocname}
\bibliography{references}

% Appendix
\clearpage
\addappheadtotoc
\appendix
\appendixpage
% \include{appendix/experiment-instruments}

\end{document}
% %% RiSE Latex Template - version 0.5
%%
%% RiSE's latex template for thesis and dissertations
%% http://risetemplate.sourceforge.net
%%
%% (c) 2012 Yguaratã Cerqueira Cavalcanti (yguarata@gmail.com)
%%          Vinicius Cardoso Garcia (vinicius.garcia@gmail.com)
%%
%% This document was initially based on UFPEThesis template, from Paulo Gustavo
%% S. Fonseca.
%%
%% ACKNOWLEDGEMENTS
%%
%% We would like to thanks the RiSE's researchers community, the 
%% students from Federal University of Pernambuco, and other users that have
%% been contributing to this projects with comments and patches.
%%
%% GENERAL INSTRUCTIONS
%%
%% We strongly recommend you to compile your documents using pdflatex command.
%% It is also recommend use the texlipse plugin for Eclipse to edit your documents.
%%
%% Options for \documentclass command:
%%         * Idiom
%%           pt   - Portguese (default)
%%           en   - English
%%
%%         * Text type
%%           bsc  - B.Sc. Thesis
%%           msc  - M.Sc. Thesis (default)
%%           qual - PHD qualification (not tested yet)
%%           prop - PHD proposal (not tested yet)
%%           phd  - PHD thesis
%%
%%         * Media
%%           scr  - to eletronic version (PDF) / see the users guide
%%
%%         * Pagination
%%           oneside - unique face press
%%           twoside - two faces press
%%
%%		   * Line spacing
%%           singlespacing  - the same as using \linespread{1}
%%           onehalfspacing - the same as using \linespread{1.3}
%%           doublespacing  - the same as using \linespread{1.6}
%%
%% Reference commands. Use the following commands to make references in your
%% text:
%%          \figref  -- for Figure reference
%%          \tabref  -- for Table reference
%%          \eqnref  -- for equation reference
%%          \chapref -- for chapter reference
%%          \secref  -- for section reference
%%          \appref  -- for appendix reference
%%          \axiref  -- for axiom reference
%%          \conjref -- for conjecture reference
%%          \defref  -- for definition reference
%%          \lemref  -- for lemma reference
%%          \theoref -- for theorem reference
%%          \corref  -- for corollary reference
%%          \propref -- for proprosition reference
%%          \pgref   -- for page reference
%%
%%          Example: See \chapref{chap:introduction}. It will produce 
%%                   'See Chapter 1', in case of English language.

\documentclass[pt,twoside,onehalfspacing,bsc]{risethesis}

\usepackage[sort,square]{natbib} %package for bibliography
\bibliographystyle{ieeetr} %authordate1

\usepackage[utf8]{inputenc}
\usepackage[brazilian]{babel}
\usepackage[T1]{fontenc}

\usepackage{graphicx}

%% Change the following pdf author attribute name to your name.
\usepackage[linkcolor=blue,citecolor=blue,urlcolor=blue,colorlinks,pdfpagelabels,pdftitle={Bruno Cabral's Bachelor Thesis},pdfauthor={Bruno Cabral}]{hyperref}

\address{SALVADOR}

\universitypt{Universidade Federal da Bahia}
\universityen{Federal University of Bahia}

\departmentpt{Depertamento de Ciência da Computação}
\departmenten{Computer Science Department}

\programpt{Programa Multiinstitucional de Pós-graduação em Ciência da Computação}
\programen{Graduate in Computer Science}

\majorfieldpt{Ciência da Computação}
\majorfielden{Computer Science}

\title{Sistema de apoio à Pós graduação - UFBA}
\date{Outubro/2016}

\author{Victor de Azevedo Nunes}
\adviser{Ivan do Carmo Machado}

\begin{document}

\frontmatter
\frontpage
\presentationpage

\begin{dedicatory}
Eu dedico esta dissertação...
%I dedicate this dissertation to my family, girlfriend, friends and
%professors who gave me all necessary support to get here.
\end{dedicatory}

\acknowledgements
Meus agradecimentos...

\begin{epigraph}[]{Edward V Berard}
Walking on water and developing software from a specification are easy if both are frozen
\end{epigraph}

\resumo
% Escreva seu resumo no arquivo resumo.tex
\input{resumo}

\abstract
% Write your abstract in a file called abstract.tex
\input{abstract}

% Summary (tables of contents)
\tableofcontents

% List of figures
\listoffigures

% List of tables
\listoftables

% List of acronyms
% Acronyms manual: http://linorg.usp.br/CTAN/macros/latex/contrib/acronym/acronym.pdf
\listofacronyms
\input{acronyms}

% List of listings
%\lstlistoflistings

\mainmatter

\include{chapters/intro}
\include{chapters/referencial_teorico}

% \include{chapters/introduction/main}
% \include{chapters/background/main}
% \include{chapters/proposed_solution/main}
% \include{chapters/experiment/main}
% \include{chapters/conclusion/main}

%\bibliographystyle{natbib}
%\addcontentsline{toc}{chapter}{\bibliographytocname}
\bibliography{references}

% Appendix
\clearpage
\addappheadtotoc
\appendix
\appendixpage
% \include{appendix/experiment-instruments}

\end{document}
% %% RiSE Latex Template - version 0.5
%%
%% RiSE's latex template for thesis and dissertations
%% http://risetemplate.sourceforge.net
%%
%% (c) 2012 Yguaratã Cerqueira Cavalcanti (yguarata@gmail.com)
%%          Vinicius Cardoso Garcia (vinicius.garcia@gmail.com)
%%
%% This document was initially based on UFPEThesis template, from Paulo Gustavo
%% S. Fonseca.
%%
%% ACKNOWLEDGEMENTS
%%
%% We would like to thanks the RiSE's researchers community, the 
%% students from Federal University of Pernambuco, and other users that have
%% been contributing to this projects with comments and patches.
%%
%% GENERAL INSTRUCTIONS
%%
%% We strongly recommend you to compile your documents using pdflatex command.
%% It is also recommend use the texlipse plugin for Eclipse to edit your documents.
%%
%% Options for \documentclass command:
%%         * Idiom
%%           pt   - Portguese (default)
%%           en   - English
%%
%%         * Text type
%%           bsc  - B.Sc. Thesis
%%           msc  - M.Sc. Thesis (default)
%%           qual - PHD qualification (not tested yet)
%%           prop - PHD proposal (not tested yet)
%%           phd  - PHD thesis
%%
%%         * Media
%%           scr  - to eletronic version (PDF) / see the users guide
%%
%%         * Pagination
%%           oneside - unique face press
%%           twoside - two faces press
%%
%%		   * Line spacing
%%           singlespacing  - the same as using \linespread{1}
%%           onehalfspacing - the same as using \linespread{1.3}
%%           doublespacing  - the same as using \linespread{1.6}
%%
%% Reference commands. Use the following commands to make references in your
%% text:
%%          \figref  -- for Figure reference
%%          \tabref  -- for Table reference
%%          \eqnref  -- for equation reference
%%          \chapref -- for chapter reference
%%          \secref  -- for section reference
%%          \appref  -- for appendix reference
%%          \axiref  -- for axiom reference
%%          \conjref -- for conjecture reference
%%          \defref  -- for definition reference
%%          \lemref  -- for lemma reference
%%          \theoref -- for theorem reference
%%          \corref  -- for corollary reference
%%          \propref -- for proprosition reference
%%          \pgref   -- for page reference
%%
%%          Example: See \chapref{chap:introduction}. It will produce 
%%                   'See Chapter 1', in case of English language.

\documentclass[pt,twoside,onehalfspacing,bsc]{risethesis}

\usepackage[sort,square]{natbib} %package for bibliography
\bibliographystyle{ieeetr} %authordate1

\usepackage[utf8]{inputenc}
\usepackage[brazilian]{babel}
\usepackage[T1]{fontenc}

\usepackage{graphicx}

%% Change the following pdf author attribute name to your name.
\usepackage[linkcolor=blue,citecolor=blue,urlcolor=blue,colorlinks,pdfpagelabels,pdftitle={Bruno Cabral's Bachelor Thesis},pdfauthor={Bruno Cabral}]{hyperref}

\address{SALVADOR}

\universitypt{Universidade Federal da Bahia}
\universityen{Federal University of Bahia}

\departmentpt{Depertamento de Ciência da Computação}
\departmenten{Computer Science Department}

\programpt{Programa Multiinstitucional de Pós-graduação em Ciência da Computação}
\programen{Graduate in Computer Science}

\majorfieldpt{Ciência da Computação}
\majorfielden{Computer Science}

\title{Sistema de apoio à Pós graduação - UFBA}
\date{Outubro/2016}

\author{Victor de Azevedo Nunes}
\adviser{Ivan do Carmo Machado}

\begin{document}

\frontmatter
\frontpage
\presentationpage

\begin{dedicatory}
Eu dedico esta dissertação...
%I dedicate this dissertation to my family, girlfriend, friends and
%professors who gave me all necessary support to get here.
\end{dedicatory}

\acknowledgements
Meus agradecimentos...

\begin{epigraph}[]{Edward V Berard}
Walking on water and developing software from a specification are easy if both are frozen
\end{epigraph}

\resumo
% Escreva seu resumo no arquivo resumo.tex
\input{resumo}

\abstract
% Write your abstract in a file called abstract.tex
\input{abstract}

% Summary (tables of contents)
\tableofcontents

% List of figures
\listoffigures

% List of tables
\listoftables

% List of acronyms
% Acronyms manual: http://linorg.usp.br/CTAN/macros/latex/contrib/acronym/acronym.pdf
\listofacronyms
\input{acronyms}

% List of listings
%\lstlistoflistings

\mainmatter

\include{chapters/intro}
\include{chapters/referencial_teorico}

% \include{chapters/introduction/main}
% \include{chapters/background/main}
% \include{chapters/proposed_solution/main}
% \include{chapters/experiment/main}
% \include{chapters/conclusion/main}

%\bibliographystyle{natbib}
%\addcontentsline{toc}{chapter}{\bibliographytocname}
\bibliography{references}

% Appendix
\clearpage
\addappheadtotoc
\appendix
\appendixpage
% \include{appendix/experiment-instruments}

\end{document}
% %% RiSE Latex Template - version 0.5
%%
%% RiSE's latex template for thesis and dissertations
%% http://risetemplate.sourceforge.net
%%
%% (c) 2012 Yguaratã Cerqueira Cavalcanti (yguarata@gmail.com)
%%          Vinicius Cardoso Garcia (vinicius.garcia@gmail.com)
%%
%% This document was initially based on UFPEThesis template, from Paulo Gustavo
%% S. Fonseca.
%%
%% ACKNOWLEDGEMENTS
%%
%% We would like to thanks the RiSE's researchers community, the 
%% students from Federal University of Pernambuco, and other users that have
%% been contributing to this projects with comments and patches.
%%
%% GENERAL INSTRUCTIONS
%%
%% We strongly recommend you to compile your documents using pdflatex command.
%% It is also recommend use the texlipse plugin for Eclipse to edit your documents.
%%
%% Options for \documentclass command:
%%         * Idiom
%%           pt   - Portguese (default)
%%           en   - English
%%
%%         * Text type
%%           bsc  - B.Sc. Thesis
%%           msc  - M.Sc. Thesis (default)
%%           qual - PHD qualification (not tested yet)
%%           prop - PHD proposal (not tested yet)
%%           phd  - PHD thesis
%%
%%         * Media
%%           scr  - to eletronic version (PDF) / see the users guide
%%
%%         * Pagination
%%           oneside - unique face press
%%           twoside - two faces press
%%
%%		   * Line spacing
%%           singlespacing  - the same as using \linespread{1}
%%           onehalfspacing - the same as using \linespread{1.3}
%%           doublespacing  - the same as using \linespread{1.6}
%%
%% Reference commands. Use the following commands to make references in your
%% text:
%%          \figref  -- for Figure reference
%%          \tabref  -- for Table reference
%%          \eqnref  -- for equation reference
%%          \chapref -- for chapter reference
%%          \secref  -- for section reference
%%          \appref  -- for appendix reference
%%          \axiref  -- for axiom reference
%%          \conjref -- for conjecture reference
%%          \defref  -- for definition reference
%%          \lemref  -- for lemma reference
%%          \theoref -- for theorem reference
%%          \corref  -- for corollary reference
%%          \propref -- for proprosition reference
%%          \pgref   -- for page reference
%%
%%          Example: See \chapref{chap:introduction}. It will produce 
%%                   'See Chapter 1', in case of English language.

\documentclass[pt,twoside,onehalfspacing,bsc]{risethesis}

\usepackage[sort,square]{natbib} %package for bibliography
\bibliographystyle{ieeetr} %authordate1

\usepackage[utf8]{inputenc}
\usepackage[brazilian]{babel}
\usepackage[T1]{fontenc}

\usepackage{graphicx}

%% Change the following pdf author attribute name to your name.
\usepackage[linkcolor=blue,citecolor=blue,urlcolor=blue,colorlinks,pdfpagelabels,pdftitle={Bruno Cabral's Bachelor Thesis},pdfauthor={Bruno Cabral}]{hyperref}

\address{SALVADOR}

\universitypt{Universidade Federal da Bahia}
\universityen{Federal University of Bahia}

\departmentpt{Depertamento de Ciência da Computação}
\departmenten{Computer Science Department}

\programpt{Programa Multiinstitucional de Pós-graduação em Ciência da Computação}
\programen{Graduate in Computer Science}

\majorfieldpt{Ciência da Computação}
\majorfielden{Computer Science}

\title{Sistema de apoio à Pós graduação - UFBA}
\date{Outubro/2016}

\author{Victor de Azevedo Nunes}
\adviser{Ivan do Carmo Machado}

\begin{document}

\frontmatter
\frontpage
\presentationpage

\begin{dedicatory}
Eu dedico esta dissertação...
%I dedicate this dissertation to my family, girlfriend, friends and
%professors who gave me all necessary support to get here.
\end{dedicatory}

\acknowledgements
Meus agradecimentos...

\begin{epigraph}[]{Edward V Berard}
Walking on water and developing software from a specification are easy if both are frozen
\end{epigraph}

\resumo
% Escreva seu resumo no arquivo resumo.tex
\input{resumo}

\abstract
% Write your abstract in a file called abstract.tex
\input{abstract}

% Summary (tables of contents)
\tableofcontents

% List of figures
\listoffigures

% List of tables
\listoftables

% List of acronyms
% Acronyms manual: http://linorg.usp.br/CTAN/macros/latex/contrib/acronym/acronym.pdf
\listofacronyms
\input{acronyms}

% List of listings
%\lstlistoflistings

\mainmatter

\include{chapters/intro}
\include{chapters/referencial_teorico}

% \include{chapters/introduction/main}
% \include{chapters/background/main}
% \include{chapters/proposed_solution/main}
% \include{chapters/experiment/main}
% \include{chapters/conclusion/main}

%\bibliographystyle{natbib}
%\addcontentsline{toc}{chapter}{\bibliographytocname}
\bibliography{references}

% Appendix
\clearpage
\addappheadtotoc
\appendix
\appendixpage
% \include{appendix/experiment-instruments}

\end{document}
% %% RiSE Latex Template - version 0.5
%%
%% RiSE's latex template for thesis and dissertations
%% http://risetemplate.sourceforge.net
%%
%% (c) 2012 Yguaratã Cerqueira Cavalcanti (yguarata@gmail.com)
%%          Vinicius Cardoso Garcia (vinicius.garcia@gmail.com)
%%
%% This document was initially based on UFPEThesis template, from Paulo Gustavo
%% S. Fonseca.
%%
%% ACKNOWLEDGEMENTS
%%
%% We would like to thanks the RiSE's researchers community, the 
%% students from Federal University of Pernambuco, and other users that have
%% been contributing to this projects with comments and patches.
%%
%% GENERAL INSTRUCTIONS
%%
%% We strongly recommend you to compile your documents using pdflatex command.
%% It is also recommend use the texlipse plugin for Eclipse to edit your documents.
%%
%% Options for \documentclass command:
%%         * Idiom
%%           pt   - Portguese (default)
%%           en   - English
%%
%%         * Text type
%%           bsc  - B.Sc. Thesis
%%           msc  - M.Sc. Thesis (default)
%%           qual - PHD qualification (not tested yet)
%%           prop - PHD proposal (not tested yet)
%%           phd  - PHD thesis
%%
%%         * Media
%%           scr  - to eletronic version (PDF) / see the users guide
%%
%%         * Pagination
%%           oneside - unique face press
%%           twoside - two faces press
%%
%%		   * Line spacing
%%           singlespacing  - the same as using \linespread{1}
%%           onehalfspacing - the same as using \linespread{1.3}
%%           doublespacing  - the same as using \linespread{1.6}
%%
%% Reference commands. Use the following commands to make references in your
%% text:
%%          \figref  -- for Figure reference
%%          \tabref  -- for Table reference
%%          \eqnref  -- for equation reference
%%          \chapref -- for chapter reference
%%          \secref  -- for section reference
%%          \appref  -- for appendix reference
%%          \axiref  -- for axiom reference
%%          \conjref -- for conjecture reference
%%          \defref  -- for definition reference
%%          \lemref  -- for lemma reference
%%          \theoref -- for theorem reference
%%          \corref  -- for corollary reference
%%          \propref -- for proprosition reference
%%          \pgref   -- for page reference
%%
%%          Example: See \chapref{chap:introduction}. It will produce 
%%                   'See Chapter 1', in case of English language.

\documentclass[pt,twoside,onehalfspacing,bsc]{risethesis}

\usepackage[sort,square]{natbib} %package for bibliography
\bibliographystyle{ieeetr} %authordate1

\usepackage[utf8]{inputenc}
\usepackage[brazilian]{babel}
\usepackage[T1]{fontenc}

\usepackage{graphicx}

%% Change the following pdf author attribute name to your name.
\usepackage[linkcolor=blue,citecolor=blue,urlcolor=blue,colorlinks,pdfpagelabels,pdftitle={Bruno Cabral's Bachelor Thesis},pdfauthor={Bruno Cabral}]{hyperref}

\address{SALVADOR}

\universitypt{Universidade Federal da Bahia}
\universityen{Federal University of Bahia}

\departmentpt{Depertamento de Ciência da Computação}
\departmenten{Computer Science Department}

\programpt{Programa Multiinstitucional de Pós-graduação em Ciência da Computação}
\programen{Graduate in Computer Science}

\majorfieldpt{Ciência da Computação}
\majorfielden{Computer Science}

\title{Sistema de apoio à Pós graduação - UFBA}
\date{Outubro/2016}

\author{Victor de Azevedo Nunes}
\adviser{Ivan do Carmo Machado}

\begin{document}

\frontmatter
\frontpage
\presentationpage

\begin{dedicatory}
Eu dedico esta dissertação...
%I dedicate this dissertation to my family, girlfriend, friends and
%professors who gave me all necessary support to get here.
\end{dedicatory}

\acknowledgements
Meus agradecimentos...

\begin{epigraph}[]{Edward V Berard}
Walking on water and developing software from a specification are easy if both are frozen
\end{epigraph}

\resumo
% Escreva seu resumo no arquivo resumo.tex
\input{resumo}

\abstract
% Write your abstract in a file called abstract.tex
\input{abstract}

% Summary (tables of contents)
\tableofcontents

% List of figures
\listoffigures

% List of tables
\listoftables

% List of acronyms
% Acronyms manual: http://linorg.usp.br/CTAN/macros/latex/contrib/acronym/acronym.pdf
\listofacronyms
\input{acronyms}

% List of listings
%\lstlistoflistings

\mainmatter

\include{chapters/intro}
\include{chapters/referencial_teorico}

% \include{chapters/introduction/main}
% \include{chapters/background/main}
% \include{chapters/proposed_solution/main}
% \include{chapters/experiment/main}
% \include{chapters/conclusion/main}

%\bibliographystyle{natbib}
%\addcontentsline{toc}{chapter}{\bibliographytocname}
\bibliography{references}

% Appendix
\clearpage
\addappheadtotoc
\appendix
\appendixpage
% \include{appendix/experiment-instruments}

\end{document}

%\bibliographystyle{natbib}
%\addcontentsline{toc}{chapter}{\bibliographytocname}
\bibliography{references}

% Appendix
\clearpage
\addappheadtotoc
\appendix
\appendixpage
% \include{appendix/experiment-instruments}

\end{document}

%\bibliographystyle{natbib}
%\addcontentsline{toc}{chapter}{\bibliographytocname}
\bibliography{references}

% Appendix
\clearpage
\addappheadtotoc
\appendix
\appendixpage
% \include{appendix/experiment-instruments}

\end{document}
% %% RiSE Latex Template - version 0.5
%%
%% RiSE's latex template for thesis and dissertations
%% http://risetemplate.sourceforge.net
%%
%% (c) 2012 Yguaratã Cerqueira Cavalcanti (yguarata@gmail.com)
%%          Vinicius Cardoso Garcia (vinicius.garcia@gmail.com)
%%
%% This document was initially based on UFPEThesis template, from Paulo Gustavo
%% S. Fonseca.
%%
%% ACKNOWLEDGEMENTS
%%
%% We would like to thanks the RiSE's researchers community, the 
%% students from Federal University of Pernambuco, and other users that have
%% been contributing to this projects with comments and patches.
%%
%% GENERAL INSTRUCTIONS
%%
%% We strongly recommend you to compile your documents using pdflatex command.
%% It is also recommend use the texlipse plugin for Eclipse to edit your documents.
%%
%% Options for \documentclass command:
%%         * Idiom
%%           pt   - Portguese (default)
%%           en   - English
%%
%%         * Text type
%%           bsc  - B.Sc. Thesis
%%           msc  - M.Sc. Thesis (default)
%%           qual - PHD qualification (not tested yet)
%%           prop - PHD proposal (not tested yet)
%%           phd  - PHD thesis
%%
%%         * Media
%%           scr  - to eletronic version (PDF) / see the users guide
%%
%%         * Pagination
%%           oneside - unique face press
%%           twoside - two faces press
%%
%%		   * Line spacing
%%           singlespacing  - the same as using \linespread{1}
%%           onehalfspacing - the same as using \linespread{1.3}
%%           doublespacing  - the same as using \linespread{1.6}
%%
%% Reference commands. Use the following commands to make references in your
%% text:
%%          \figref  -- for Figure reference
%%          \tabref  -- for Table reference
%%          \eqnref  -- for equation reference
%%          \chapref -- for chapter reference
%%          \secref  -- for section reference
%%          \appref  -- for appendix reference
%%          \axiref  -- for axiom reference
%%          \conjref -- for conjecture reference
%%          \defref  -- for definition reference
%%          \lemref  -- for lemma reference
%%          \theoref -- for theorem reference
%%          \corref  -- for corollary reference
%%          \propref -- for proprosition reference
%%          \pgref   -- for page reference
%%
%%          Example: See \chapref{chap:introduction}. It will produce 
%%                   'See Chapter 1', in case of English language.

\documentclass[pt,twoside,onehalfspacing,bsc]{risethesis}

\usepackage[sort,square]{natbib} %package for bibliography
\bibliographystyle{ieeetr} %authordate1

\usepackage[utf8]{inputenc}
\usepackage[brazilian]{babel}
\usepackage[T1]{fontenc}

\usepackage{graphicx}

%% Change the following pdf author attribute name to your name.
\usepackage[linkcolor=blue,citecolor=blue,urlcolor=blue,colorlinks,pdfpagelabels,pdftitle={Bruno Cabral's Bachelor Thesis},pdfauthor={Bruno Cabral}]{hyperref}

\address{SALVADOR}

\universitypt{Universidade Federal da Bahia}
\universityen{Federal University of Bahia}

\departmentpt{Depertamento de Ciência da Computação}
\departmenten{Computer Science Department}

\programpt{Programa Multiinstitucional de Pós-graduação em Ciência da Computação}
\programen{Graduate in Computer Science}

\majorfieldpt{Ciência da Computação}
\majorfielden{Computer Science}

\title{Sistema de apoio à Pós graduação - UFBA}
\date{Outubro/2016}

\author{Victor de Azevedo Nunes}
\adviser{Ivan do Carmo Machado}

\begin{document}

\frontmatter
\frontpage
\presentationpage

\begin{dedicatory}
Eu dedico esta dissertação...
%I dedicate this dissertation to my family, girlfriend, friends and
%professors who gave me all necessary support to get here.
\end{dedicatory}

\acknowledgements
Meus agradecimentos...

\begin{epigraph}[]{Edward V Berard}
Walking on water and developing software from a specification are easy if both are frozen
\end{epigraph}

\resumo
% Escreva seu resumo no arquivo resumo.tex
Este trabalho de conclusão de curso utiliza elementos da engenharia de software para propor um software como serviço a fim de otimizar o processo de construção e manutenção dos softwares. Assim, o objetivo deste SaaS é gerenciar as permissões de acesso de sistemas cliente, provendo o reuso de software e padronizando as soluções. O sistema proposto neste trabalho encontra-se implementado e disponível no Github, e traz fundamentos sobre a arquitetura e tecnologias utilizadas, bem como avaliações sobre possibilidades de utilização do mesmo em diversos ambientes, como Web e mobile.

\begin{keywords}
Software; Reuso; SaaS; Web; PHP
\end{keywords}

\abstract
% Write your abstract in a file called abstract.tex
My abstract...

\begin{keywords}
key words...
\end{keywords}

% Summary (tables of contents)
\tableofcontents

% List of figures
\listoffigures

% List of tables
\listoftables

% List of acronyms
% Acronyms manual: http://linorg.usp.br/CTAN/macros/latex/contrib/acronym/acronym.pdf
\listofacronyms
\begin{acronym}[ACRONYM] 
% Change the word ACRONYM above to change the acronym column width.
% The column width is equals to the width of the word that you put.
% Read the manual about acronym package for more examples:
%   http://linorg.usp.br/CTAN/macros/latex/contrib/acronym/acronym.pdf

\acro{SPA}{Single Page Application}
\acro{JSON}{Javascript Object Notation}
\acro{PHP}{PHP: Hypertext Preprocessor}
\acro{SaaS}{Software as a Service}
\acro{ERP}{Enterprise Resource Planning}
\acro{QoS}{Quality of Service}
\acro{UML}{Unified Modeling Language}
\acro{MVC}{Model-View-Controller}
\acro{Ajax}{Asynchronous Javascript and XML}
\acro{HTML}{HyperText Markup Language}
\acro{CSS}{Cascading Style Sheets}
\acro{API}{Application Programming Interface}
\acro{DOM}{Document Object Model}
\acro{BPMN}{Business Process Model and Notation}
\acro{REST}{Representational State Transfer}

\end{acronym}

% List of listings
%\lstlistoflistings

\mainmatter

\chapter{Introdução}

\section{Motivação}

Organizar os procedimentos de um processo sempre nos traz vantagens. Apesar de no processo de implantação de um sistema, o mesmo burocratizar o processo, com o tempo temos o retorno da dedicação para a inserção dos dados. Com um certo volume de dados, é possível estruturar informações que num processo manual são difíceis de serem enxergadas. Assim, é possível depender menos das pessoas que organizam o processo, pois o legado de informações não estará mais somente na mente de alguns, mas sim documentado nos dados do sistema.

Além de colaborar na organização, também haverá uma grande colaboração no tempo gasto na gestão. Lidar com muitos papéis e confiar na mente humana para guardar informações, não é uma alternativa muito segura devido ao fato que as pessoas sempre estão sujeitas a sair do processo e levar contigo a experiência obtida. Experiência essa que faz com que os procedimentos sejam executados de forma mais eficiente. Entretanto, com um sistema inteligente, é possível auxiliar e tornar mais ágil a execução das tarefas.


\section{Problema}


De acordo com funcionários ligados ao o setor de pós graduação da UFBA, entrevistados a fim de um maior entendimento do cenário, apesar das semelhanças estruturais, a pós graduação gerida de forma diferencia da graduação. FULANO afirma que devido ao fato de não ter a mesma visibilidade, não tem acesso aos mesmos recursos de gestão acadêmica da graduação. O professores não executam somente atividades dentro da sala de aula, também tem diversas outras ocupações no setor. E muitos procedimentos realizados extra classe ainda se encontram sendo realizados de forma manual, estando mais vulnerável ao erro ou até mesmo à violação do processo. Também ocorre um grande desperdício de tempo pelos professores e gestores da área, devido ao diversos processos ainda realizados de forma manual, sem a devida documentação. Segundo FULANO, também entrevistado, esse tempo perdido implica numa redução da eficiência na sala de aula, pois o professor acaba por ter menos tempo disponível para o planejamento das atividades, o que gera impactos negativos aos alunos.


\section{Objetivos} %<o que deve ser feito/entregue>


Devido aos muitos processos sendo resolvidos de forma manual, propõe-se com solução um sistema moderno, arquitetado para ter funcionamento na web e com um módulo mobile, a fim de fornecer informações de forma rápida e eficiente para os professores através de notificações, já que o acesso à internet móvel é comum entre os possíveis usuários do sistema em questão.
O principal requisito para o sistema seria dispor recursos para reduzir o tempo desperdiçado pelos professores durante as atividades extra classe.


\section{Metodologia} %<como será feito | como resolver o problema apontado inicialmente>


%<analise de literatura | design | implementação | validação>
Baseando-se nas tecnologias gratuitas em alta no cenário atual do desenvolvimento web, dispomos de algumas opções eficientes para a implementação da solução. Dentre as possibilidades, considerando a facilidade para futura manutenção e continuidade do projeto, tende-se a optar por uma tecnologia popular. Como linguagem de programação, adota-se o PHP. A escolha é fundamentada de acordo com a pesquisa da RedMonk de 2015, que evidencia o uso das linguagens de programação de acordo com as discussões no StackOverflow e repositórios no GitHub. É possível constatar a popularidade do PHP no cenário atual com o gráfico da pesquisa citada, na qual o PHP é apresentado na terceira colocação, apenas atrás do lider JavaScript e do segundo colocado, o Java.

\begin{figure}
	\label{fig:graficoRedmonk}
	\includegraphics[width=1\textwidth]{img/grafico_redmonk}
	\caption{Ranking das liguagens de programação no Stack Overflow e Github}
\end{figure}


Ainda assim, para compor a interface do dado projeto, também ocorrerá o uso do líder JavaScript de forma intensa, provendo o elo com o as informações gerenciadas pelo PHP.


Entretanto, não seria inteligente desenvolver um sistema completo sem o auxílio de um framework. Dentre os frameworks disponíveis para PHP, hoje o destaque está com o Laravel, que se encontra no topo dentre os mais utilizados no momento. 


A WebHostFace, uma empresa de hospedagem, compilou várias estatísticas para criar um infográfico mostrando os frameworks PHP mais populares de 2015. Utilizando informações sobre os próprios clientes, o Google Trends, estatísticas de repositórios do GitHub e a pesquisa do SitePoint “Best PHP Frameworks 2015”, a WebHostFace elaborou o seguinte infográfico: 

\begin{figure}
	\label{fig:graficoWebhostface}
	\includegraphics[width=1\textwidth]{img/infografico_webhostface}
	\caption{Infográfico da WebhostFace, exibindo a popularidade dos Frameworks PHP em 2015}
\end{figure}

Assim, tem-se a evidência que o Laravel em 2015 teve a maior popularidade em projetos pessoais e tem a maior comunidade entre os concorrentes, o que o torna uma boa escolha para a escrita de um software que será continuado por terceiros.


Para elaborar os recursos de interface e integrar ao back-end PHP do sistema, será adotado o já conhecido AngularJS, ferramenta sólida e conhecida no aspecto em questão. 


Dados coletados via Google Trends, que propõe comparações entre termos pesquisados, revela a popularidade do AngularJs diante de alguns dos principais concorrentes. O gráfico abaixo evidencia o cenário.


%Como mostra a Figura \ref{fig:graficoGoogleTrendsFerramentasFront}. 
\begin{figure}
	\label{fig:graficoGoogleTrendsFerramentasFront}
	\includegraphics[width=1\textwidth]{img/grafico_ferramentas_front}
	\caption{Gráfico do Google Trends exibindo as pesquisas por ferramentas front-end}
\end{figure}


Junto ao Angular JS, será utilizada a agradável tendência de interface do Material Design da Google, que propõe layouts limpos e otimizados já conhecidos pelos usuários de smartphones Android. 


Para a elaboração da plataforma mobile do projeto, será utilizado o Ionic Framework, muito difundido e bastante pesquisado na área, o que fica evidenciado com o gráfico de pesquisbaixo, coletado via Google Trends buscando por frameworks de desenvolvimento híbrido mobile.


\begin{figure}
	\label{fig:graficoGoogleTrendsFerramentasHibridasMobile}
	\includegraphics[width=1\textwidth]{img/grafico_ferramentas_hibridas_mobile}
	\caption{Gráfico do Google Trends exibindo as pesquisas por Frameworks híbridos mobile}
\end{figure}	

Para layout da interface mobile, também será aplicado a tendência do Material Design, a fim de propor uma harmonia entre o módulo web e mobile para os usuários


\section{Resultados Esperados}


Como fruto de um sistema para pós-graduação da UFBA, espera-se que os professores tenham mais recursos para integrar as atividades e também prover melhores condições para acompanhamento da vida acadêmica dos alunos em questão. Também, que os novos colaboradores que entrarem no processo tenham facilidade de compreender o fluxo do setor ao navegar pelo sistema proposto.


\section{Fora de Escopo}


Interação com os alunos devido às complicações para realizar a integração com o sistema empregado na UFBA, gerenciado pela XXXXXX, o que causaria uma inviabilidade no projeto devido à necessidade de entrega do produto ser mais forte que o tempo necessário para executar o processo de obtenção de acesso ao sistema legado para realizar a integração.


\section{Estrutura do Trabalho}


<breve resumo sobre os capítulos do TCC>
\chapter{Referencial Teórico}


Projetar o desenvolvimento de um software requer muito planejamento, pois as falhas iniciais podem custar bastante caro ou até mesmo inviabilizar a continuação de um projeto. Assim, a escolha da arquitetura ideal para a aplicabilidade é essencial na concepção de um produto de software. 
De todo o modo, sempre busca-se fazer mais com menos. Diante de tal filosofia, temos neste capítulo, uma breve discussão sobre alguns elementos de projeto e arquitetura de software, a fim de contextualizar este trabalho de conclusão de curso.
O capítulo corrente é composto por quatro seções. A \ref{sec:saas} trata de Software como serviço, discutindo alguns elementos do contexto que são relevantes para o trabalho proposto. A \ref{sec:reuso}  discute sobre a empregabilidade do reuso de software. A \ref{sec:modularizacao} seção, trata sobre aspectos envolvidos na modularização dos softwares. Por fim, a \ref{sec:apps_web} aborda as aplicações web, discutindo sobre aspectos relevantes sobre a aplicação web que compõe este trabalho.


 \section{Software como serviço}\label{sec:saas}


Segundo La e Chun \citep{La2009Systematic}, o princípio da definição de Software como um Serviço (Sofware as a Service - SaaS) é um serviço complementar para aplicações da computação em nuvem (cloud computing). No entanto, as áreas não se confundem. SaaS deve ser entendido como um mecanismo de suporte às soluções existentes na cloud. Os SaaS existem justamente para maximizar o reuso de serviços repetidos e não centrais em uma aplicação remota.


Como vantagens, diversos fatores podem ser favoráveis para a adoção de um SaaS, como custo e manutenção dentre outros fatores aplicáveis a soluções específicas. Lechesaet al. \citep{LechesaSS11}, quantificam índices sobre os fatores determinantes para adoção ou não de um SaaS voltado para ERP na África do Sul. Os principais fatores determinantes para adoção desse mecanismo de software são sua fluidez quanto à rede e a segurança. Esses fatores estão presentes na aplicação desenvolvida neste trabalho de conclusão de curso.


Devido ao fato de ter um serviço constantemente na nuvem, fica o questionamento sobre a segurança da informação manipulada. Sabe-se que a vulnerabilidade na Web não é restrita ao SaaS, atingindo diversos âmbitos. Rai et al. \cite{journals/corr/RaiSM13} consideram que o avanço da computação em nuvem não é um problema apenas para os serviços Web do ponto de vista da segurança, pois muitos trabalhos na literatura mostram a área como mais um ponto de vulnerabilidade para diversos setores, a exemplo de infraestrutura. No mesmo artigo mencionado de Rai et al. \cite{journals/corr/RaiSM13}, também realizaram-se estudos exploratórios junto a empresas usuárias de serviços em computação em nuvem e consideram que a perspectiva de SaaS também pode fortalecer a segurança nas aplicações de cloud computing, pois o software de autenticação compartilhado por várias aplicações em nuvem, oferece uma melhor padronização e consequente facilidade de prevenção a erros de vulnerabilidade específicas de cada módulo da pesquisa. Esse ponto de vista é fundamenta para a compreensão de SaaS.


A arquitetura de armazenamento de dados de um Saas pode variar de acordo com a necessidade do contexto. Huixin \cite{7586486} descreve possíveis modelagens nesse sentido. Essa abordagem pode ser com um banco de dados único, fazendo com que diferentes clientes compartilhem o mesmo banco, diferindo os dados através de controle de usuário, ou isolando os diferentes clientes através de bancos de dados exclusivos para cada um. Esse fator também pode ser combinado com a arquitetura da aplicação, caso ofereça aplicação única para todos os clientes ou aplicação compartilhada. Diante das possíveis abordagens, a modelagem de dados do software pode ser decidida pela regra de negócio. Este trabalho optou por aplicação única e banco de dados compartilhado.

Devido ao conceito particular de obtenção de software, tanto pela visão do cliente como pela visão do vendedor, é necessário compreender os diversos aspectos que podem ser relevantes ao orçar um Saas. O recente trabalho de T. Kaur et al. \citep{6949281} orienta um modelo para compor o custo de um Saas. O custo total seria composto pelos fatores que dão suporte ao funcionamento do software. Tais fatores incluem infra-estrutura, configurabilidade, customização, parâmetros de QoS (Quality of service) como escalabilidade, disponibilidade, usabilidade, pontualidade e desempenho da resposta, portabilidade, custo total de propriedade e retorno do investimento. Esses fatores caracterizam o custo de forma eficaz, possibilitando ao fornecedor, prover um Serviço de acordo com a exigência do consumidor em vários pacotes de serviços.


O conceito de software como serviço se aplica neste trabalho de conclusão de curso, pois o mesmo estará disponível na web com alta disponibilidade adotando as características apresentadas para qualquer pessoa que desejar usá-lo. Assim, basta se cadastrar, configurá-lo e usar como um serviço, sem preocupação com a manutenção do mesmo.


\section{Reuso de software}\label{sec:reuso} %CRUISE BOOK CAPITULO 2


De acordo com o livro Software pratical reuse \citep{Ezran2002}, o reuso de software é a utilização de qualquer informação que um desenvolvedor pode necessitar no processo de criação de software. O livro de Basili e Rombach \citep{Basili1991} define reutilização de software como o uso de tudo o que está associado a um projeto de conhecimento.
Assim, o objetivo da reutilização de software é reciclar o design, código e outros componentes de um produto de software e assim reduzir o custo, o tempo e melhorar a qualidade do produto.
Segundo Keswani et al. \cite{6783445}, o componente reutilizável de software pode ser qualquer parte de seu desenvolvimento, como um fragmento de código, design, casos de teste, ou até mesmo a especificação de requisitos de uma funcionalidade do software. 

O reuso de software pode ter impacto positivo em diversos aspectos do software, vejamos alguns, conforme apresentados no C.R.U.I.S.E Book \citep{cruise2007} :

\begin{itemize}

\item Qualidade: As correções de erro tornam-se úteis em todos os locais em que ocorreu, padronizando e facilitando a manutenção.

\item Produtividade: O ganho de produtividade é alcançado devido ao menor número de artefatos desenvolvido. Isso resulta em menor esforço de teste e também análise e design, reduzindo custos.

\item Confiabilidade: A utilização de componentes bem testados aumenta a
confiança no software. Além disso, a utilização de um mesmo componente em vários sistemas, aumenta a possibilidade de detecção de erros e reforça a confiança no componente.

\item Redução do Esforço: A reutilização de software proporciona uma redução do tempo de desenvolvimento, o que reduz o tempo necessário para o produto ser disponibilizado no mercado para trazer rentabilidade.

\item Trabalho redundante e tempo de desenvolvimento: Desenvolver um sistema do
zero significa desenvolvimento redundante de muitos componentes, como requisitos, especificações, casos de uso, arquitetura, etc. Isso pode ser evitado quando estes estão disponíveis como componentes reutilizáveis e podem ser compartilhados, resultando em um processo de desenvolvimento otimizado.

\item Documentação: Embora a documentação seja muito importante para a
manutenção de um sistema, muitas vezes é negligenciada. A reutilização de componentes de software reduz a quantidade de documentação a ser escrita, entretanto depende da qualidade do que está escrito. Assim, apenas a estrutura do sistema e os novos artefatos desenvolvidos necessitam ser documentados.

\item Custo de manutenção: Menos defeitos e manutenções são esperados quando tem-se comprovada a qualidade dos componentes utilizados.

\item Tamanho da equipe: É comum haver casos em que a equipe de desenvolvimento sofre sobrecarga. Entretanto, dobrar o tamanho da equipe de desenvolvimento não necessariamente duplica produtividade. Se muitos componentes podem ser reutilizados, é possível desenvolver com equipes menores, levando a melhor comunicação e aumento da produtividade.

\end{itemize}

Apesar dos benefícios da reutilização de software, ela não é suficientemente aproveitada. Existem fatores que influenciam direta ou indiretamente na sua adoção. Esses fatores podem ser de aspecto gerencial, organizacional, econômico, conceitual ou técnico. Veremos a seguir alguns aspectos que podem gerar conflito com a cultura de reuso de software, segundo o C.R.U.I.S.E Book \citep{cruise2007} :
%(Sametinger, 1997). REVER

\begin{itemize}
	
\item Falta de apoio da gestão: Como a reutilização de software gera custos iniciais,
a medida pode não ser amplamente alcançada em uma organização sem o apoio de alto nível de gestão. Os gestores têm de ser informados sobre os custos iniciais e serem convencidos sobre economias futuras.

\item Gerenciamento do Projeto: Gerenciar projetos tradicionais é uma tarefa árdua, principalmente, os que praticam a reutilização de software. Utilizando a técnica em larga escala, tem-se impacto sobre todo o ciclo de vida do software.

\item Estruturas organizacionais inadequadas: As estruturas organizacionais devem
considerar diferentes necessidades que surgem quando a reutilização em larga escala está sendo adotada. Por exemplo, uma equipe particionada pode ser alocada somente para desenvolver, manter e certificar componentes reutilizáveis de software.

\item Incentivos de gestão: É comum a falta de incentivo para deixar os desenvolvedores gastarem tempo elaborando componentes do sistemas. A produtividade é muitas vezes medida apenas no tempo necessário para concluir um projeto. Assim, fazer qualquer trabalho além disso, embora benéfico para a empresa como um todo, diminui o seu sucesso. Mesmo quando os componentes reutilizáveis são utilizados, os benefícios obtidos são uma pequena fração do que poderia ser alcançado caso houvesse reutilização explícita, planejada e organizada.

\item Dificuldade de encontrar software reutilizável: Para reutilizar os componentes, devem existir formas eficientes de busca. Além disso, é importante ter um repositório bem organizado contendo componentes com um eficiente meio de acesso.

\item Não reutilização do software encontrado. O acesso fácil ao software existente
não necessariamente aumentar a reutilização. Os componentes reutilizáveis devem ser cuidadosamente especificados, projetados, implementados e documentados, pois em alguns casos, modificar e adaptar o código  pode ser mais custoso que a programação da funcionalidade necessária a partir do zero.

\item Modificação: É muito difícil encontrar um componente que funcione
exatamente da mesma maneira que queremos. Desta forma, são necessárias modificações e devem existir formas de determinar os seus efeitos sobre o componente.


\end{itemize}


A definição do reuso de software pode ser associada a este trabalho, pois ainda que como um serviço, estaremos adotando uma aplicação única para servir aos mais diversos softwares de uma empresa, por exemplo. Logo, ao adotar o projeto, pode ser interessante aplicar algumas métricas para tomar conhecimento de possíveis vantanjens como as citadas nessa seção, a exemplo de economia de tempo de desenvolvimento, e custo do projeto.


%Outra diretriz importante para a reutilização de software é reduzir o risco na criação de novos softwares. O risco tende a ser bastante reduzido se os componentes que estão sendo reutilizados têm as documentação, interfaces necessárias e devidamente testadas, fatores que contibruem para uma fácil integração.
%De acordo com Keswani et al. \citep{6783445}, para o reuso de software dar retornos apropriados, o processo deve ser sistemático e planejado. Qualquer organização que implemente a reutilização de software deve identificar os melhores métodos e estratégias de reutilização para obter a máxima produtividade. A reutilização de software ajuda a evitar software de engenharia a partir do zero, pois usa módulos de software existentes. A reutilização de software, embora seja uma tarefa difícil, especialmente para softwares antigos sem padrões de projeto, pode melhorar significativamente a produtividade e a qualidade de um produto de software. Embora a reutilização de software não seja um novo campo, ela pode dar grandes retornos em curto período de tempo.


\section{Modularização}\label{sec:modularizacao} %artigo de claudio pagina 222 introdução


%A modularidade vem desempenhando um papel predominante estágios emergentes das disciplinas de arquitetura de software [13]. Engenheiros de software consideram modularidade como princípio base na comparação entre arquiteturas alternativas  e arquitetura degeneração [9]. De fato, os engenheiros de software são incentivados a arquitecturas, baseando-se numa multiplicidade de mecanismos de modularidade disponíveis em: 
%(i) Linguagens de descrição de arquitetura (ADLs), como ACME [8], 
%(ii) catálogos de arquitetônicos [2, 13], e 
%(iii) conhecem bem princípios de alto nível, como interfaces de componentes estreitos, acoplamento arquitectónico reduzido e semelhantes.


Conforme é frisado por Wickramaarachchi e Lai \citep{7062705}, o conceito de modularização na indústria de software tem uma longa história e tem sido utilizado para melhorar o processo de desenvolvimento de software em diferentes estágios. Os principais conceitos por trás da modularização do software foram introduzidos por pesquisadores pioneiros há quarenta anos, com uma notável contribuição feita por Melvin Conway e David Parnas, que tem representação notável na engenharia de software.


Modularizar um software é um bom padrão a ser adotado. Segundo Wickramaarachchi e Lai \citep{7062705}, a modularização é importante na identificação de dependências e reduz as dificuldades diante de uma possível necessidade de grandes alterações. De uma perspectiva da engenharia de software, uma modularização geralmente tem várias vantagens, tais como: tornar a complexidade do software mais gerenciável, facilitar o trabalho paralelo e tornar o software mais maleável para acomodar o futuro incerto que um software pode ter. O objetivo final da modularização do software é aumentar a produtividade ea qualidade do software. Tal conceito encontra-se bastante difundido e estái incorporado em linguagens de programação e ferramentas de software. O presente trabalho favorece ao uso da modularização de um software e até mesmo pode ser considerado um módulo a ser acoplado a qualquer software, mediante a compatibilidade.


Ao realizar a adoção do trabalho proposto, fica bastante evidente a "responsabilidade" do mesmo no escopo do projeto. Assim, é possível tratá-lo como um módulo do projeto que o usa, mesmo que seja consumido como um serviço.


\section{Aplicações web}\label{sec:apps_web}


A popularidade das soluções Web aumentou exponencialmente na última década e todos os dias cresce o número de pessoas usuárias desse tipo de software. E seguindo um padrão próprio, Kumar et al. \citep{7813710} sugerem que para o desenvolvimento web, deve-se manter a prática eficaz de produzir diagramas UML. A abordagem baseada na web oferece uma maneira fácil e eficaz para gerenciar e controlar o processo de desenvolvimento por meio de artefatos de modelagem. Tal abordagem pode ser usada quando há uma exigência de lidar com mudanças muito rápidas e grandes em requisitos de forma muito eficaz em muito menos tempo, gerando assim um menor impacto. 


Para atender à fomentada demanda de aplicativos web, é necessário adotar métodos de desenvolvimentos que sejam ágeis, eficientes e de fácil manutenção. Yu Ping et al. \cite{1372143} propõem o uso do modelo MVC (Model, View e Controller) no desenvolvimento para softwares web. O modelo apresentado tornou-se um padrão popular e divide o software em camadas com propósito definido, tornando-o de mais fácil manutenção.


O Ajax (Asynchronous Javascript and XML) revolucionou a web. Conforme demonstrado por Yuping \citep{6845605}, ao usar a tecnologia Ajax, podemos enriquecer a experiência do usuário em aplicações baseadas em navegador de internet, e fornecer uma variedade de aplicações interativas para atender às necessidade de humanização das aplicações.
Os aplicativos Ajax em execução no navegador se comunicam com um servidor Web de forma assíncrona e atualizam apenas uma parte da página.


No artigo de Tesarik et al. \citep{4664349}, temos que o design de software SPA (Single page application) é uma maneira interessante de criar um software Web numa uma única página. Essa solução de página única sem navegação funciona apenas com base em técnicas dinâmicas e assíncronas, como o AJAX (citado no parágrafo anterior). No entanto, esta abordagem coloca o desenvolvedor antes de alguns desafios substanciais.
Para projetar a interface do usuário que mostra as informações, é importante projetar corretamente a tela para manipular os dados do aplicativo numa única página. O design da página deve ser elaborado para maximizar a decomposição da página em componentes distintos que encapsulam os principais casos de uso.
Também recomenda-se que para elaborar os artefatos visuais, explore os mais ricos recursos na implementação, como HTML5, JavaScript, Ajax, CSS3, e outras tecnologias que se apliquem. Entretanto, desenvolver uma rica interface com o uso de diversas tecnologias/frameworks pode ocasionar um esforço maior para explorar as possibilidades.
Tais componentes da interface de uma aplicação SPA normalmente são alimentados mediante o consumo de uma API Rest via requisições AJAX. Assim, é possível particionar as responsabilidades de processamento do software entre cliente e servidor.


Como é possível denotar no trabalho de Salvadori e Siqueire \citep{7195633}, REST é uma arquitetura muito popular para integração de aplicativos web, que permite compartilhar e reutilizar informações através de sistemas. Aplicações de grande escala baseadas em no REST deve ser implementadas utilizando estratégias e mecanismos para produzir sistemas que sejam fáceis de desenvolver, reutilizar e manter. As interfaces de integração fornecidas por esses sistemas, chamadas de Web API, têm uma influência importante nas características da implementação resultante, pois a sua forma de resposta deve ser casada com a interface que o consome.
 
 
Os elementos comentados nessa seção encontram-se presentes neste trabalho de conclusão de curso. Juntos, montam a estrutura tecnológica necessária aliada à arquitetura adotada. Esses elementos seguem tendências atuais dos softwares web, fazendo com que esse trabalho esteja composto por tecnologias modernas que estão com boa aceitação no mercado.

% %% RiSE Latex Template - version 0.5
%%
%% RiSE's latex template for thesis and dissertations
%% http://risetemplate.sourceforge.net
%%
%% (c) 2012 Yguaratã Cerqueira Cavalcanti (yguarata@gmail.com)
%%          Vinicius Cardoso Garcia (vinicius.garcia@gmail.com)
%%
%% This document was initially based on UFPEThesis template, from Paulo Gustavo
%% S. Fonseca.
%%
%% ACKNOWLEDGEMENTS
%%
%% We would like to thanks the RiSE's researchers community, the 
%% students from Federal University of Pernambuco, and other users that have
%% been contributing to this projects with comments and patches.
%%
%% GENERAL INSTRUCTIONS
%%
%% We strongly recommend you to compile your documents using pdflatex command.
%% It is also recommend use the texlipse plugin for Eclipse to edit your documents.
%%
%% Options for \documentclass command:
%%         * Idiom
%%           pt   - Portguese (default)
%%           en   - English
%%
%%         * Text type
%%           bsc  - B.Sc. Thesis
%%           msc  - M.Sc. Thesis (default)
%%           qual - PHD qualification (not tested yet)
%%           prop - PHD proposal (not tested yet)
%%           phd  - PHD thesis
%%
%%         * Media
%%           scr  - to eletronic version (PDF) / see the users guide
%%
%%         * Pagination
%%           oneside - unique face press
%%           twoside - two faces press
%%
%%		   * Line spacing
%%           singlespacing  - the same as using \linespread{1}
%%           onehalfspacing - the same as using \linespread{1.3}
%%           doublespacing  - the same as using \linespread{1.6}
%%
%% Reference commands. Use the following commands to make references in your
%% text:
%%          \figref  -- for Figure reference
%%          \tabref  -- for Table reference
%%          \eqnref  -- for equation reference
%%          \chapref -- for chapter reference
%%          \secref  -- for section reference
%%          \appref  -- for appendix reference
%%          \axiref  -- for axiom reference
%%          \conjref -- for conjecture reference
%%          \defref  -- for definition reference
%%          \lemref  -- for lemma reference
%%          \theoref -- for theorem reference
%%          \corref  -- for corollary reference
%%          \propref -- for proprosition reference
%%          \pgref   -- for page reference
%%
%%          Example: See \chapref{chap:introduction}. It will produce 
%%                   'See Chapter 1', in case of English language.

\documentclass[pt,twoside,onehalfspacing,bsc]{risethesis}

\usepackage[sort,square]{natbib} %package for bibliography
\bibliographystyle{ieeetr} %authordate1

\usepackage[utf8]{inputenc}
\usepackage[brazilian]{babel}
\usepackage[T1]{fontenc}

\usepackage{graphicx}

%% Change the following pdf author attribute name to your name.
\usepackage[linkcolor=blue,citecolor=blue,urlcolor=blue,colorlinks,pdfpagelabels,pdftitle={Bruno Cabral's Bachelor Thesis},pdfauthor={Bruno Cabral}]{hyperref}

\address{SALVADOR}

\universitypt{Universidade Federal da Bahia}
\universityen{Federal University of Bahia}

\departmentpt{Depertamento de Ciência da Computação}
\departmenten{Computer Science Department}

\programpt{Programa Multiinstitucional de Pós-graduação em Ciência da Computação}
\programen{Graduate in Computer Science}

\majorfieldpt{Ciência da Computação}
\majorfielden{Computer Science}

\title{Sistema de apoio à Pós graduação - UFBA}
\date{Outubro/2016}

\author{Victor de Azevedo Nunes}
\adviser{Ivan do Carmo Machado}

\begin{document}

\frontmatter
\frontpage
\presentationpage

\begin{dedicatory}
Eu dedico esta dissertação...
%I dedicate this dissertation to my family, girlfriend, friends and
%professors who gave me all necessary support to get here.
\end{dedicatory}

\acknowledgements
Meus agradecimentos...

\begin{epigraph}[]{Edward V Berard}
Walking on water and developing software from a specification are easy if both are frozen
\end{epigraph}

\resumo
% Escreva seu resumo no arquivo resumo.tex
Este trabalho de conclusão de curso utiliza elementos da engenharia de software para propor um software como serviço a fim de otimizar o processo de construção e manutenção dos softwares. Assim, o objetivo deste SaaS é gerenciar as permissões de acesso de sistemas cliente, provendo o reuso de software e padronizando as soluções. O sistema proposto neste trabalho encontra-se implementado e disponível no Github, e traz fundamentos sobre a arquitetura e tecnologias utilizadas, bem como avaliações sobre possibilidades de utilização do mesmo em diversos ambientes, como Web e mobile.

\begin{keywords}
Software; Reuso; SaaS; Web; PHP
\end{keywords}

\abstract
% Write your abstract in a file called abstract.tex
My abstract...

\begin{keywords}
key words...
\end{keywords}

% Summary (tables of contents)
\tableofcontents

% List of figures
\listoffigures

% List of tables
\listoftables

% List of acronyms
% Acronyms manual: http://linorg.usp.br/CTAN/macros/latex/contrib/acronym/acronym.pdf
\listofacronyms
\begin{acronym}[ACRONYM] 
% Change the word ACRONYM above to change the acronym column width.
% The column width is equals to the width of the word that you put.
% Read the manual about acronym package for more examples:
%   http://linorg.usp.br/CTAN/macros/latex/contrib/acronym/acronym.pdf

\acro{SPA}{Single Page Application}
\acro{JSON}{Javascript Object Notation}
\acro{PHP}{PHP: Hypertext Preprocessor}
\acro{SaaS}{Software as a Service}
\acro{ERP}{Enterprise Resource Planning}
\acro{QoS}{Quality of Service}
\acro{UML}{Unified Modeling Language}
\acro{MVC}{Model-View-Controller}
\acro{Ajax}{Asynchronous Javascript and XML}
\acro{HTML}{HyperText Markup Language}
\acro{CSS}{Cascading Style Sheets}
\acro{API}{Application Programming Interface}
\acro{DOM}{Document Object Model}
\acro{BPMN}{Business Process Model and Notation}
\acro{REST}{Representational State Transfer}

\end{acronym}

% List of listings
%\lstlistoflistings

\mainmatter

\chapter{Introdução}

\section{Motivação}

Organizar os procedimentos de um processo sempre nos traz vantagens. Apesar de no processo de implantação de um sistema, o mesmo burocratizar o processo, com o tempo temos o retorno da dedicação para a inserção dos dados. Com um certo volume de dados, é possível estruturar informações que num processo manual são difíceis de serem enxergadas. Assim, é possível depender menos das pessoas que organizam o processo, pois o legado de informações não estará mais somente na mente de alguns, mas sim documentado nos dados do sistema.

Além de colaborar na organização, também haverá uma grande colaboração no tempo gasto na gestão. Lidar com muitos papéis e confiar na mente humana para guardar informações, não é uma alternativa muito segura devido ao fato que as pessoas sempre estão sujeitas a sair do processo e levar contigo a experiência obtida. Experiência essa que faz com que os procedimentos sejam executados de forma mais eficiente. Entretanto, com um sistema inteligente, é possível auxiliar e tornar mais ágil a execução das tarefas.


\section{Problema}


De acordo com funcionários ligados ao o setor de pós graduação da UFBA, entrevistados a fim de um maior entendimento do cenário, apesar das semelhanças estruturais, a pós graduação gerida de forma diferencia da graduação. FULANO afirma que devido ao fato de não ter a mesma visibilidade, não tem acesso aos mesmos recursos de gestão acadêmica da graduação. O professores não executam somente atividades dentro da sala de aula, também tem diversas outras ocupações no setor. E muitos procedimentos realizados extra classe ainda se encontram sendo realizados de forma manual, estando mais vulnerável ao erro ou até mesmo à violação do processo. Também ocorre um grande desperdício de tempo pelos professores e gestores da área, devido ao diversos processos ainda realizados de forma manual, sem a devida documentação. Segundo FULANO, também entrevistado, esse tempo perdido implica numa redução da eficiência na sala de aula, pois o professor acaba por ter menos tempo disponível para o planejamento das atividades, o que gera impactos negativos aos alunos.


\section{Objetivos} %<o que deve ser feito/entregue>


Devido aos muitos processos sendo resolvidos de forma manual, propõe-se com solução um sistema moderno, arquitetado para ter funcionamento na web e com um módulo mobile, a fim de fornecer informações de forma rápida e eficiente para os professores através de notificações, já que o acesso à internet móvel é comum entre os possíveis usuários do sistema em questão.
O principal requisito para o sistema seria dispor recursos para reduzir o tempo desperdiçado pelos professores durante as atividades extra classe.


\section{Metodologia} %<como será feito | como resolver o problema apontado inicialmente>


%<analise de literatura | design | implementação | validação>
Baseando-se nas tecnologias gratuitas em alta no cenário atual do desenvolvimento web, dispomos de algumas opções eficientes para a implementação da solução. Dentre as possibilidades, considerando a facilidade para futura manutenção e continuidade do projeto, tende-se a optar por uma tecnologia popular. Como linguagem de programação, adota-se o PHP. A escolha é fundamentada de acordo com a pesquisa da RedMonk de 2015, que evidencia o uso das linguagens de programação de acordo com as discussões no StackOverflow e repositórios no GitHub. É possível constatar a popularidade do PHP no cenário atual com o gráfico da pesquisa citada, na qual o PHP é apresentado na terceira colocação, apenas atrás do lider JavaScript e do segundo colocado, o Java.

\begin{figure}
	\label{fig:graficoRedmonk}
	\includegraphics[width=1\textwidth]{img/grafico_redmonk}
	\caption{Ranking das liguagens de programação no Stack Overflow e Github}
\end{figure}


Ainda assim, para compor a interface do dado projeto, também ocorrerá o uso do líder JavaScript de forma intensa, provendo o elo com o as informações gerenciadas pelo PHP.


Entretanto, não seria inteligente desenvolver um sistema completo sem o auxílio de um framework. Dentre os frameworks disponíveis para PHP, hoje o destaque está com o Laravel, que se encontra no topo dentre os mais utilizados no momento. 


A WebHostFace, uma empresa de hospedagem, compilou várias estatísticas para criar um infográfico mostrando os frameworks PHP mais populares de 2015. Utilizando informações sobre os próprios clientes, o Google Trends, estatísticas de repositórios do GitHub e a pesquisa do SitePoint “Best PHP Frameworks 2015”, a WebHostFace elaborou o seguinte infográfico: 

\begin{figure}
	\label{fig:graficoWebhostface}
	\includegraphics[width=1\textwidth]{img/infografico_webhostface}
	\caption{Infográfico da WebhostFace, exibindo a popularidade dos Frameworks PHP em 2015}
\end{figure}

Assim, tem-se a evidência que o Laravel em 2015 teve a maior popularidade em projetos pessoais e tem a maior comunidade entre os concorrentes, o que o torna uma boa escolha para a escrita de um software que será continuado por terceiros.


Para elaborar os recursos de interface e integrar ao back-end PHP do sistema, será adotado o já conhecido AngularJS, ferramenta sólida e conhecida no aspecto em questão. 


Dados coletados via Google Trends, que propõe comparações entre termos pesquisados, revela a popularidade do AngularJs diante de alguns dos principais concorrentes. O gráfico abaixo evidencia o cenário.


%Como mostra a Figura \ref{fig:graficoGoogleTrendsFerramentasFront}. 
\begin{figure}
	\label{fig:graficoGoogleTrendsFerramentasFront}
	\includegraphics[width=1\textwidth]{img/grafico_ferramentas_front}
	\caption{Gráfico do Google Trends exibindo as pesquisas por ferramentas front-end}
\end{figure}


Junto ao Angular JS, será utilizada a agradável tendência de interface do Material Design da Google, que propõe layouts limpos e otimizados já conhecidos pelos usuários de smartphones Android. 


Para a elaboração da plataforma mobile do projeto, será utilizado o Ionic Framework, muito difundido e bastante pesquisado na área, o que fica evidenciado com o gráfico de pesquisbaixo, coletado via Google Trends buscando por frameworks de desenvolvimento híbrido mobile.


\begin{figure}
	\label{fig:graficoGoogleTrendsFerramentasHibridasMobile}
	\includegraphics[width=1\textwidth]{img/grafico_ferramentas_hibridas_mobile}
	\caption{Gráfico do Google Trends exibindo as pesquisas por Frameworks híbridos mobile}
\end{figure}	

Para layout da interface mobile, também será aplicado a tendência do Material Design, a fim de propor uma harmonia entre o módulo web e mobile para os usuários


\section{Resultados Esperados}


Como fruto de um sistema para pós-graduação da UFBA, espera-se que os professores tenham mais recursos para integrar as atividades e também prover melhores condições para acompanhamento da vida acadêmica dos alunos em questão. Também, que os novos colaboradores que entrarem no processo tenham facilidade de compreender o fluxo do setor ao navegar pelo sistema proposto.


\section{Fora de Escopo}


Interação com os alunos devido às complicações para realizar a integração com o sistema empregado na UFBA, gerenciado pela XXXXXX, o que causaria uma inviabilidade no projeto devido à necessidade de entrega do produto ser mais forte que o tempo necessário para executar o processo de obtenção de acesso ao sistema legado para realizar a integração.


\section{Estrutura do Trabalho}


<breve resumo sobre os capítulos do TCC>
\chapter{Referencial Teórico}


Projetar o desenvolvimento de um software requer muito planejamento, pois as falhas iniciais podem custar bastante caro ou até mesmo inviabilizar a continuação de um projeto. Assim, a escolha da arquitetura ideal para a aplicabilidade é essencial na concepção de um produto de software. 
De todo o modo, sempre busca-se fazer mais com menos. Diante de tal filosofia, temos neste capítulo, uma breve discussão sobre alguns elementos de projeto e arquitetura de software, a fim de contextualizar este trabalho de conclusão de curso.
O capítulo corrente é composto por quatro seções. A \ref{sec:saas} trata de Software como serviço, discutindo alguns elementos do contexto que são relevantes para o trabalho proposto. A \ref{sec:reuso}  discute sobre a empregabilidade do reuso de software. A \ref{sec:modularizacao} seção, trata sobre aspectos envolvidos na modularização dos softwares. Por fim, a \ref{sec:apps_web} aborda as aplicações web, discutindo sobre aspectos relevantes sobre a aplicação web que compõe este trabalho.


 \section{Software como serviço}\label{sec:saas}


Segundo La e Chun \citep{La2009Systematic}, o princípio da definição de Software como um Serviço (Sofware as a Service - SaaS) é um serviço complementar para aplicações da computação em nuvem (cloud computing). No entanto, as áreas não se confundem. SaaS deve ser entendido como um mecanismo de suporte às soluções existentes na cloud. Os SaaS existem justamente para maximizar o reuso de serviços repetidos e não centrais em uma aplicação remota.


Como vantagens, diversos fatores podem ser favoráveis para a adoção de um SaaS, como custo e manutenção dentre outros fatores aplicáveis a soluções específicas. Lechesaet al. \citep{LechesaSS11}, quantificam índices sobre os fatores determinantes para adoção ou não de um SaaS voltado para ERP na África do Sul. Os principais fatores determinantes para adoção desse mecanismo de software são sua fluidez quanto à rede e a segurança. Esses fatores estão presentes na aplicação desenvolvida neste trabalho de conclusão de curso.


Devido ao fato de ter um serviço constantemente na nuvem, fica o questionamento sobre a segurança da informação manipulada. Sabe-se que a vulnerabilidade na Web não é restrita ao SaaS, atingindo diversos âmbitos. Rai et al. \cite{journals/corr/RaiSM13} consideram que o avanço da computação em nuvem não é um problema apenas para os serviços Web do ponto de vista da segurança, pois muitos trabalhos na literatura mostram a área como mais um ponto de vulnerabilidade para diversos setores, a exemplo de infraestrutura. No mesmo artigo mencionado de Rai et al. \cite{journals/corr/RaiSM13}, também realizaram-se estudos exploratórios junto a empresas usuárias de serviços em computação em nuvem e consideram que a perspectiva de SaaS também pode fortalecer a segurança nas aplicações de cloud computing, pois o software de autenticação compartilhado por várias aplicações em nuvem, oferece uma melhor padronização e consequente facilidade de prevenção a erros de vulnerabilidade específicas de cada módulo da pesquisa. Esse ponto de vista é fundamenta para a compreensão de SaaS.


A arquitetura de armazenamento de dados de um Saas pode variar de acordo com a necessidade do contexto. Huixin \cite{7586486} descreve possíveis modelagens nesse sentido. Essa abordagem pode ser com um banco de dados único, fazendo com que diferentes clientes compartilhem o mesmo banco, diferindo os dados através de controle de usuário, ou isolando os diferentes clientes através de bancos de dados exclusivos para cada um. Esse fator também pode ser combinado com a arquitetura da aplicação, caso ofereça aplicação única para todos os clientes ou aplicação compartilhada. Diante das possíveis abordagens, a modelagem de dados do software pode ser decidida pela regra de negócio. Este trabalho optou por aplicação única e banco de dados compartilhado.

Devido ao conceito particular de obtenção de software, tanto pela visão do cliente como pela visão do vendedor, é necessário compreender os diversos aspectos que podem ser relevantes ao orçar um Saas. O recente trabalho de T. Kaur et al. \citep{6949281} orienta um modelo para compor o custo de um Saas. O custo total seria composto pelos fatores que dão suporte ao funcionamento do software. Tais fatores incluem infra-estrutura, configurabilidade, customização, parâmetros de QoS (Quality of service) como escalabilidade, disponibilidade, usabilidade, pontualidade e desempenho da resposta, portabilidade, custo total de propriedade e retorno do investimento. Esses fatores caracterizam o custo de forma eficaz, possibilitando ao fornecedor, prover um Serviço de acordo com a exigência do consumidor em vários pacotes de serviços.


O conceito de software como serviço se aplica neste trabalho de conclusão de curso, pois o mesmo estará disponível na web com alta disponibilidade adotando as características apresentadas para qualquer pessoa que desejar usá-lo. Assim, basta se cadastrar, configurá-lo e usar como um serviço, sem preocupação com a manutenção do mesmo.


\section{Reuso de software}\label{sec:reuso} %CRUISE BOOK CAPITULO 2


De acordo com o livro Software pratical reuse \citep{Ezran2002}, o reuso de software é a utilização de qualquer informação que um desenvolvedor pode necessitar no processo de criação de software. O livro de Basili e Rombach \citep{Basili1991} define reutilização de software como o uso de tudo o que está associado a um projeto de conhecimento.
Assim, o objetivo da reutilização de software é reciclar o design, código e outros componentes de um produto de software e assim reduzir o custo, o tempo e melhorar a qualidade do produto.
Segundo Keswani et al. \cite{6783445}, o componente reutilizável de software pode ser qualquer parte de seu desenvolvimento, como um fragmento de código, design, casos de teste, ou até mesmo a especificação de requisitos de uma funcionalidade do software. 

O reuso de software pode ter impacto positivo em diversos aspectos do software, vejamos alguns, conforme apresentados no C.R.U.I.S.E Book \citep{cruise2007} :

\begin{itemize}

\item Qualidade: As correções de erro tornam-se úteis em todos os locais em que ocorreu, padronizando e facilitando a manutenção.

\item Produtividade: O ganho de produtividade é alcançado devido ao menor número de artefatos desenvolvido. Isso resulta em menor esforço de teste e também análise e design, reduzindo custos.

\item Confiabilidade: A utilização de componentes bem testados aumenta a
confiança no software. Além disso, a utilização de um mesmo componente em vários sistemas, aumenta a possibilidade de detecção de erros e reforça a confiança no componente.

\item Redução do Esforço: A reutilização de software proporciona uma redução do tempo de desenvolvimento, o que reduz o tempo necessário para o produto ser disponibilizado no mercado para trazer rentabilidade.

\item Trabalho redundante e tempo de desenvolvimento: Desenvolver um sistema do
zero significa desenvolvimento redundante de muitos componentes, como requisitos, especificações, casos de uso, arquitetura, etc. Isso pode ser evitado quando estes estão disponíveis como componentes reutilizáveis e podem ser compartilhados, resultando em um processo de desenvolvimento otimizado.

\item Documentação: Embora a documentação seja muito importante para a
manutenção de um sistema, muitas vezes é negligenciada. A reutilização de componentes de software reduz a quantidade de documentação a ser escrita, entretanto depende da qualidade do que está escrito. Assim, apenas a estrutura do sistema e os novos artefatos desenvolvidos necessitam ser documentados.

\item Custo de manutenção: Menos defeitos e manutenções são esperados quando tem-se comprovada a qualidade dos componentes utilizados.

\item Tamanho da equipe: É comum haver casos em que a equipe de desenvolvimento sofre sobrecarga. Entretanto, dobrar o tamanho da equipe de desenvolvimento não necessariamente duplica produtividade. Se muitos componentes podem ser reutilizados, é possível desenvolver com equipes menores, levando a melhor comunicação e aumento da produtividade.

\end{itemize}

Apesar dos benefícios da reutilização de software, ela não é suficientemente aproveitada. Existem fatores que influenciam direta ou indiretamente na sua adoção. Esses fatores podem ser de aspecto gerencial, organizacional, econômico, conceitual ou técnico. Veremos a seguir alguns aspectos que podem gerar conflito com a cultura de reuso de software, segundo o C.R.U.I.S.E Book \citep{cruise2007} :
%(Sametinger, 1997). REVER

\begin{itemize}
	
\item Falta de apoio da gestão: Como a reutilização de software gera custos iniciais,
a medida pode não ser amplamente alcançada em uma organização sem o apoio de alto nível de gestão. Os gestores têm de ser informados sobre os custos iniciais e serem convencidos sobre economias futuras.

\item Gerenciamento do Projeto: Gerenciar projetos tradicionais é uma tarefa árdua, principalmente, os que praticam a reutilização de software. Utilizando a técnica em larga escala, tem-se impacto sobre todo o ciclo de vida do software.

\item Estruturas organizacionais inadequadas: As estruturas organizacionais devem
considerar diferentes necessidades que surgem quando a reutilização em larga escala está sendo adotada. Por exemplo, uma equipe particionada pode ser alocada somente para desenvolver, manter e certificar componentes reutilizáveis de software.

\item Incentivos de gestão: É comum a falta de incentivo para deixar os desenvolvedores gastarem tempo elaborando componentes do sistemas. A produtividade é muitas vezes medida apenas no tempo necessário para concluir um projeto. Assim, fazer qualquer trabalho além disso, embora benéfico para a empresa como um todo, diminui o seu sucesso. Mesmo quando os componentes reutilizáveis são utilizados, os benefícios obtidos são uma pequena fração do que poderia ser alcançado caso houvesse reutilização explícita, planejada e organizada.

\item Dificuldade de encontrar software reutilizável: Para reutilizar os componentes, devem existir formas eficientes de busca. Além disso, é importante ter um repositório bem organizado contendo componentes com um eficiente meio de acesso.

\item Não reutilização do software encontrado. O acesso fácil ao software existente
não necessariamente aumentar a reutilização. Os componentes reutilizáveis devem ser cuidadosamente especificados, projetados, implementados e documentados, pois em alguns casos, modificar e adaptar o código  pode ser mais custoso que a programação da funcionalidade necessária a partir do zero.

\item Modificação: É muito difícil encontrar um componente que funcione
exatamente da mesma maneira que queremos. Desta forma, são necessárias modificações e devem existir formas de determinar os seus efeitos sobre o componente.


\end{itemize}


A definição do reuso de software pode ser associada a este trabalho, pois ainda que como um serviço, estaremos adotando uma aplicação única para servir aos mais diversos softwares de uma empresa, por exemplo. Logo, ao adotar o projeto, pode ser interessante aplicar algumas métricas para tomar conhecimento de possíveis vantanjens como as citadas nessa seção, a exemplo de economia de tempo de desenvolvimento, e custo do projeto.


%Outra diretriz importante para a reutilização de software é reduzir o risco na criação de novos softwares. O risco tende a ser bastante reduzido se os componentes que estão sendo reutilizados têm as documentação, interfaces necessárias e devidamente testadas, fatores que contibruem para uma fácil integração.
%De acordo com Keswani et al. \citep{6783445}, para o reuso de software dar retornos apropriados, o processo deve ser sistemático e planejado. Qualquer organização que implemente a reutilização de software deve identificar os melhores métodos e estratégias de reutilização para obter a máxima produtividade. A reutilização de software ajuda a evitar software de engenharia a partir do zero, pois usa módulos de software existentes. A reutilização de software, embora seja uma tarefa difícil, especialmente para softwares antigos sem padrões de projeto, pode melhorar significativamente a produtividade e a qualidade de um produto de software. Embora a reutilização de software não seja um novo campo, ela pode dar grandes retornos em curto período de tempo.


\section{Modularização}\label{sec:modularizacao} %artigo de claudio pagina 222 introdução


%A modularidade vem desempenhando um papel predominante estágios emergentes das disciplinas de arquitetura de software [13]. Engenheiros de software consideram modularidade como princípio base na comparação entre arquiteturas alternativas  e arquitetura degeneração [9]. De fato, os engenheiros de software são incentivados a arquitecturas, baseando-se numa multiplicidade de mecanismos de modularidade disponíveis em: 
%(i) Linguagens de descrição de arquitetura (ADLs), como ACME [8], 
%(ii) catálogos de arquitetônicos [2, 13], e 
%(iii) conhecem bem princípios de alto nível, como interfaces de componentes estreitos, acoplamento arquitectónico reduzido e semelhantes.


Conforme é frisado por Wickramaarachchi e Lai \citep{7062705}, o conceito de modularização na indústria de software tem uma longa história e tem sido utilizado para melhorar o processo de desenvolvimento de software em diferentes estágios. Os principais conceitos por trás da modularização do software foram introduzidos por pesquisadores pioneiros há quarenta anos, com uma notável contribuição feita por Melvin Conway e David Parnas, que tem representação notável na engenharia de software.


Modularizar um software é um bom padrão a ser adotado. Segundo Wickramaarachchi e Lai \citep{7062705}, a modularização é importante na identificação de dependências e reduz as dificuldades diante de uma possível necessidade de grandes alterações. De uma perspectiva da engenharia de software, uma modularização geralmente tem várias vantagens, tais como: tornar a complexidade do software mais gerenciável, facilitar o trabalho paralelo e tornar o software mais maleável para acomodar o futuro incerto que um software pode ter. O objetivo final da modularização do software é aumentar a produtividade ea qualidade do software. Tal conceito encontra-se bastante difundido e estái incorporado em linguagens de programação e ferramentas de software. O presente trabalho favorece ao uso da modularização de um software e até mesmo pode ser considerado um módulo a ser acoplado a qualquer software, mediante a compatibilidade.


Ao realizar a adoção do trabalho proposto, fica bastante evidente a "responsabilidade" do mesmo no escopo do projeto. Assim, é possível tratá-lo como um módulo do projeto que o usa, mesmo que seja consumido como um serviço.


\section{Aplicações web}\label{sec:apps_web}


A popularidade das soluções Web aumentou exponencialmente na última década e todos os dias cresce o número de pessoas usuárias desse tipo de software. E seguindo um padrão próprio, Kumar et al. \citep{7813710} sugerem que para o desenvolvimento web, deve-se manter a prática eficaz de produzir diagramas UML. A abordagem baseada na web oferece uma maneira fácil e eficaz para gerenciar e controlar o processo de desenvolvimento por meio de artefatos de modelagem. Tal abordagem pode ser usada quando há uma exigência de lidar com mudanças muito rápidas e grandes em requisitos de forma muito eficaz em muito menos tempo, gerando assim um menor impacto. 


Para atender à fomentada demanda de aplicativos web, é necessário adotar métodos de desenvolvimentos que sejam ágeis, eficientes e de fácil manutenção. Yu Ping et al. \cite{1372143} propõem o uso do modelo MVC (Model, View e Controller) no desenvolvimento para softwares web. O modelo apresentado tornou-se um padrão popular e divide o software em camadas com propósito definido, tornando-o de mais fácil manutenção.


O Ajax (Asynchronous Javascript and XML) revolucionou a web. Conforme demonstrado por Yuping \citep{6845605}, ao usar a tecnologia Ajax, podemos enriquecer a experiência do usuário em aplicações baseadas em navegador de internet, e fornecer uma variedade de aplicações interativas para atender às necessidade de humanização das aplicações.
Os aplicativos Ajax em execução no navegador se comunicam com um servidor Web de forma assíncrona e atualizam apenas uma parte da página.


No artigo de Tesarik et al. \citep{4664349}, temos que o design de software SPA (Single page application) é uma maneira interessante de criar um software Web numa uma única página. Essa solução de página única sem navegação funciona apenas com base em técnicas dinâmicas e assíncronas, como o AJAX (citado no parágrafo anterior). No entanto, esta abordagem coloca o desenvolvedor antes de alguns desafios substanciais.
Para projetar a interface do usuário que mostra as informações, é importante projetar corretamente a tela para manipular os dados do aplicativo numa única página. O design da página deve ser elaborado para maximizar a decomposição da página em componentes distintos que encapsulam os principais casos de uso.
Também recomenda-se que para elaborar os artefatos visuais, explore os mais ricos recursos na implementação, como HTML5, JavaScript, Ajax, CSS3, e outras tecnologias que se apliquem. Entretanto, desenvolver uma rica interface com o uso de diversas tecnologias/frameworks pode ocasionar um esforço maior para explorar as possibilidades.
Tais componentes da interface de uma aplicação SPA normalmente são alimentados mediante o consumo de uma API Rest via requisições AJAX. Assim, é possível particionar as responsabilidades de processamento do software entre cliente e servidor.


Como é possível denotar no trabalho de Salvadori e Siqueire \citep{7195633}, REST é uma arquitetura muito popular para integração de aplicativos web, que permite compartilhar e reutilizar informações através de sistemas. Aplicações de grande escala baseadas em no REST deve ser implementadas utilizando estratégias e mecanismos para produzir sistemas que sejam fáceis de desenvolver, reutilizar e manter. As interfaces de integração fornecidas por esses sistemas, chamadas de Web API, têm uma influência importante nas características da implementação resultante, pois a sua forma de resposta deve ser casada com a interface que o consome.
 
 
Os elementos comentados nessa seção encontram-se presentes neste trabalho de conclusão de curso. Juntos, montam a estrutura tecnológica necessária aliada à arquitetura adotada. Esses elementos seguem tendências atuais dos softwares web, fazendo com que esse trabalho esteja composto por tecnologias modernas que estão com boa aceitação no mercado.

% %% RiSE Latex Template - version 0.5
%%
%% RiSE's latex template for thesis and dissertations
%% http://risetemplate.sourceforge.net
%%
%% (c) 2012 Yguaratã Cerqueira Cavalcanti (yguarata@gmail.com)
%%          Vinicius Cardoso Garcia (vinicius.garcia@gmail.com)
%%
%% This document was initially based on UFPEThesis template, from Paulo Gustavo
%% S. Fonseca.
%%
%% ACKNOWLEDGEMENTS
%%
%% We would like to thanks the RiSE's researchers community, the 
%% students from Federal University of Pernambuco, and other users that have
%% been contributing to this projects with comments and patches.
%%
%% GENERAL INSTRUCTIONS
%%
%% We strongly recommend you to compile your documents using pdflatex command.
%% It is also recommend use the texlipse plugin for Eclipse to edit your documents.
%%
%% Options for \documentclass command:
%%         * Idiom
%%           pt   - Portguese (default)
%%           en   - English
%%
%%         * Text type
%%           bsc  - B.Sc. Thesis
%%           msc  - M.Sc. Thesis (default)
%%           qual - PHD qualification (not tested yet)
%%           prop - PHD proposal (not tested yet)
%%           phd  - PHD thesis
%%
%%         * Media
%%           scr  - to eletronic version (PDF) / see the users guide
%%
%%         * Pagination
%%           oneside - unique face press
%%           twoside - two faces press
%%
%%		   * Line spacing
%%           singlespacing  - the same as using \linespread{1}
%%           onehalfspacing - the same as using \linespread{1.3}
%%           doublespacing  - the same as using \linespread{1.6}
%%
%% Reference commands. Use the following commands to make references in your
%% text:
%%          \figref  -- for Figure reference
%%          \tabref  -- for Table reference
%%          \eqnref  -- for equation reference
%%          \chapref -- for chapter reference
%%          \secref  -- for section reference
%%          \appref  -- for appendix reference
%%          \axiref  -- for axiom reference
%%          \conjref -- for conjecture reference
%%          \defref  -- for definition reference
%%          \lemref  -- for lemma reference
%%          \theoref -- for theorem reference
%%          \corref  -- for corollary reference
%%          \propref -- for proprosition reference
%%          \pgref   -- for page reference
%%
%%          Example: See \chapref{chap:introduction}. It will produce 
%%                   'See Chapter 1', in case of English language.

\documentclass[pt,twoside,onehalfspacing,bsc]{risethesis}

\usepackage[sort,square]{natbib} %package for bibliography
\bibliographystyle{ieeetr} %authordate1

\usepackage[utf8]{inputenc}
\usepackage[brazilian]{babel}
\usepackage[T1]{fontenc}

\usepackage{graphicx}

%% Change the following pdf author attribute name to your name.
\usepackage[linkcolor=blue,citecolor=blue,urlcolor=blue,colorlinks,pdfpagelabels,pdftitle={Bruno Cabral's Bachelor Thesis},pdfauthor={Bruno Cabral}]{hyperref}

\address{SALVADOR}

\universitypt{Universidade Federal da Bahia}
\universityen{Federal University of Bahia}

\departmentpt{Depertamento de Ciência da Computação}
\departmenten{Computer Science Department}

\programpt{Programa Multiinstitucional de Pós-graduação em Ciência da Computação}
\programen{Graduate in Computer Science}

\majorfieldpt{Ciência da Computação}
\majorfielden{Computer Science}

\title{Sistema de apoio à Pós graduação - UFBA}
\date{Outubro/2016}

\author{Victor de Azevedo Nunes}
\adviser{Ivan do Carmo Machado}

\begin{document}

\frontmatter
\frontpage
\presentationpage

\begin{dedicatory}
Eu dedico esta dissertação...
%I dedicate this dissertation to my family, girlfriend, friends and
%professors who gave me all necessary support to get here.
\end{dedicatory}

\acknowledgements
Meus agradecimentos...

\begin{epigraph}[]{Edward V Berard}
Walking on water and developing software from a specification are easy if both are frozen
\end{epigraph}

\resumo
% Escreva seu resumo no arquivo resumo.tex
\input{resumo}

\abstract
% Write your abstract in a file called abstract.tex
\input{abstract}

% Summary (tables of contents)
\tableofcontents

% List of figures
\listoffigures

% List of tables
\listoftables

% List of acronyms
% Acronyms manual: http://linorg.usp.br/CTAN/macros/latex/contrib/acronym/acronym.pdf
\listofacronyms
\input{acronyms}

% List of listings
%\lstlistoflistings

\mainmatter

\include{chapters/intro}
\include{chapters/referencial_teorico}

% \include{chapters/introduction/main}
% \include{chapters/background/main}
% \include{chapters/proposed_solution/main}
% \include{chapters/experiment/main}
% \include{chapters/conclusion/main}

%\bibliographystyle{natbib}
%\addcontentsline{toc}{chapter}{\bibliographytocname}
\bibliography{references}

% Appendix
\clearpage
\addappheadtotoc
\appendix
\appendixpage
% \include{appendix/experiment-instruments}

\end{document}
% %% RiSE Latex Template - version 0.5
%%
%% RiSE's latex template for thesis and dissertations
%% http://risetemplate.sourceforge.net
%%
%% (c) 2012 Yguaratã Cerqueira Cavalcanti (yguarata@gmail.com)
%%          Vinicius Cardoso Garcia (vinicius.garcia@gmail.com)
%%
%% This document was initially based on UFPEThesis template, from Paulo Gustavo
%% S. Fonseca.
%%
%% ACKNOWLEDGEMENTS
%%
%% We would like to thanks the RiSE's researchers community, the 
%% students from Federal University of Pernambuco, and other users that have
%% been contributing to this projects with comments and patches.
%%
%% GENERAL INSTRUCTIONS
%%
%% We strongly recommend you to compile your documents using pdflatex command.
%% It is also recommend use the texlipse plugin for Eclipse to edit your documents.
%%
%% Options for \documentclass command:
%%         * Idiom
%%           pt   - Portguese (default)
%%           en   - English
%%
%%         * Text type
%%           bsc  - B.Sc. Thesis
%%           msc  - M.Sc. Thesis (default)
%%           qual - PHD qualification (not tested yet)
%%           prop - PHD proposal (not tested yet)
%%           phd  - PHD thesis
%%
%%         * Media
%%           scr  - to eletronic version (PDF) / see the users guide
%%
%%         * Pagination
%%           oneside - unique face press
%%           twoside - two faces press
%%
%%		   * Line spacing
%%           singlespacing  - the same as using \linespread{1}
%%           onehalfspacing - the same as using \linespread{1.3}
%%           doublespacing  - the same as using \linespread{1.6}
%%
%% Reference commands. Use the following commands to make references in your
%% text:
%%          \figref  -- for Figure reference
%%          \tabref  -- for Table reference
%%          \eqnref  -- for equation reference
%%          \chapref -- for chapter reference
%%          \secref  -- for section reference
%%          \appref  -- for appendix reference
%%          \axiref  -- for axiom reference
%%          \conjref -- for conjecture reference
%%          \defref  -- for definition reference
%%          \lemref  -- for lemma reference
%%          \theoref -- for theorem reference
%%          \corref  -- for corollary reference
%%          \propref -- for proprosition reference
%%          \pgref   -- for page reference
%%
%%          Example: See \chapref{chap:introduction}. It will produce 
%%                   'See Chapter 1', in case of English language.

\documentclass[pt,twoside,onehalfspacing,bsc]{risethesis}

\usepackage[sort,square]{natbib} %package for bibliography
\bibliographystyle{ieeetr} %authordate1

\usepackage[utf8]{inputenc}
\usepackage[brazilian]{babel}
\usepackage[T1]{fontenc}

\usepackage{graphicx}

%% Change the following pdf author attribute name to your name.
\usepackage[linkcolor=blue,citecolor=blue,urlcolor=blue,colorlinks,pdfpagelabels,pdftitle={Bruno Cabral's Bachelor Thesis},pdfauthor={Bruno Cabral}]{hyperref}

\address{SALVADOR}

\universitypt{Universidade Federal da Bahia}
\universityen{Federal University of Bahia}

\departmentpt{Depertamento de Ciência da Computação}
\departmenten{Computer Science Department}

\programpt{Programa Multiinstitucional de Pós-graduação em Ciência da Computação}
\programen{Graduate in Computer Science}

\majorfieldpt{Ciência da Computação}
\majorfielden{Computer Science}

\title{Sistema de apoio à Pós graduação - UFBA}
\date{Outubro/2016}

\author{Victor de Azevedo Nunes}
\adviser{Ivan do Carmo Machado}

\begin{document}

\frontmatter
\frontpage
\presentationpage

\begin{dedicatory}
Eu dedico esta dissertação...
%I dedicate this dissertation to my family, girlfriend, friends and
%professors who gave me all necessary support to get here.
\end{dedicatory}

\acknowledgements
Meus agradecimentos...

\begin{epigraph}[]{Edward V Berard}
Walking on water and developing software from a specification are easy if both are frozen
\end{epigraph}

\resumo
% Escreva seu resumo no arquivo resumo.tex
\input{resumo}

\abstract
% Write your abstract in a file called abstract.tex
\input{abstract}

% Summary (tables of contents)
\tableofcontents

% List of figures
\listoffigures

% List of tables
\listoftables

% List of acronyms
% Acronyms manual: http://linorg.usp.br/CTAN/macros/latex/contrib/acronym/acronym.pdf
\listofacronyms
\input{acronyms}

% List of listings
%\lstlistoflistings

\mainmatter

\include{chapters/intro}
\include{chapters/referencial_teorico}

% \include{chapters/introduction/main}
% \include{chapters/background/main}
% \include{chapters/proposed_solution/main}
% \include{chapters/experiment/main}
% \include{chapters/conclusion/main}

%\bibliographystyle{natbib}
%\addcontentsline{toc}{chapter}{\bibliographytocname}
\bibliography{references}

% Appendix
\clearpage
\addappheadtotoc
\appendix
\appendixpage
% \include{appendix/experiment-instruments}

\end{document}
% %% RiSE Latex Template - version 0.5
%%
%% RiSE's latex template for thesis and dissertations
%% http://risetemplate.sourceforge.net
%%
%% (c) 2012 Yguaratã Cerqueira Cavalcanti (yguarata@gmail.com)
%%          Vinicius Cardoso Garcia (vinicius.garcia@gmail.com)
%%
%% This document was initially based on UFPEThesis template, from Paulo Gustavo
%% S. Fonseca.
%%
%% ACKNOWLEDGEMENTS
%%
%% We would like to thanks the RiSE's researchers community, the 
%% students from Federal University of Pernambuco, and other users that have
%% been contributing to this projects with comments and patches.
%%
%% GENERAL INSTRUCTIONS
%%
%% We strongly recommend you to compile your documents using pdflatex command.
%% It is also recommend use the texlipse plugin for Eclipse to edit your documents.
%%
%% Options for \documentclass command:
%%         * Idiom
%%           pt   - Portguese (default)
%%           en   - English
%%
%%         * Text type
%%           bsc  - B.Sc. Thesis
%%           msc  - M.Sc. Thesis (default)
%%           qual - PHD qualification (not tested yet)
%%           prop - PHD proposal (not tested yet)
%%           phd  - PHD thesis
%%
%%         * Media
%%           scr  - to eletronic version (PDF) / see the users guide
%%
%%         * Pagination
%%           oneside - unique face press
%%           twoside - two faces press
%%
%%		   * Line spacing
%%           singlespacing  - the same as using \linespread{1}
%%           onehalfspacing - the same as using \linespread{1.3}
%%           doublespacing  - the same as using \linespread{1.6}
%%
%% Reference commands. Use the following commands to make references in your
%% text:
%%          \figref  -- for Figure reference
%%          \tabref  -- for Table reference
%%          \eqnref  -- for equation reference
%%          \chapref -- for chapter reference
%%          \secref  -- for section reference
%%          \appref  -- for appendix reference
%%          \axiref  -- for axiom reference
%%          \conjref -- for conjecture reference
%%          \defref  -- for definition reference
%%          \lemref  -- for lemma reference
%%          \theoref -- for theorem reference
%%          \corref  -- for corollary reference
%%          \propref -- for proprosition reference
%%          \pgref   -- for page reference
%%
%%          Example: See \chapref{chap:introduction}. It will produce 
%%                   'See Chapter 1', in case of English language.

\documentclass[pt,twoside,onehalfspacing,bsc]{risethesis}

\usepackage[sort,square]{natbib} %package for bibliography
\bibliographystyle{ieeetr} %authordate1

\usepackage[utf8]{inputenc}
\usepackage[brazilian]{babel}
\usepackage[T1]{fontenc}

\usepackage{graphicx}

%% Change the following pdf author attribute name to your name.
\usepackage[linkcolor=blue,citecolor=blue,urlcolor=blue,colorlinks,pdfpagelabels,pdftitle={Bruno Cabral's Bachelor Thesis},pdfauthor={Bruno Cabral}]{hyperref}

\address{SALVADOR}

\universitypt{Universidade Federal da Bahia}
\universityen{Federal University of Bahia}

\departmentpt{Depertamento de Ciência da Computação}
\departmenten{Computer Science Department}

\programpt{Programa Multiinstitucional de Pós-graduação em Ciência da Computação}
\programen{Graduate in Computer Science}

\majorfieldpt{Ciência da Computação}
\majorfielden{Computer Science}

\title{Sistema de apoio à Pós graduação - UFBA}
\date{Outubro/2016}

\author{Victor de Azevedo Nunes}
\adviser{Ivan do Carmo Machado}

\begin{document}

\frontmatter
\frontpage
\presentationpage

\begin{dedicatory}
Eu dedico esta dissertação...
%I dedicate this dissertation to my family, girlfriend, friends and
%professors who gave me all necessary support to get here.
\end{dedicatory}

\acknowledgements
Meus agradecimentos...

\begin{epigraph}[]{Edward V Berard}
Walking on water and developing software from a specification are easy if both are frozen
\end{epigraph}

\resumo
% Escreva seu resumo no arquivo resumo.tex
\input{resumo}

\abstract
% Write your abstract in a file called abstract.tex
\input{abstract}

% Summary (tables of contents)
\tableofcontents

% List of figures
\listoffigures

% List of tables
\listoftables

% List of acronyms
% Acronyms manual: http://linorg.usp.br/CTAN/macros/latex/contrib/acronym/acronym.pdf
\listofacronyms
\input{acronyms}

% List of listings
%\lstlistoflistings

\mainmatter

\include{chapters/intro}
\include{chapters/referencial_teorico}

% \include{chapters/introduction/main}
% \include{chapters/background/main}
% \include{chapters/proposed_solution/main}
% \include{chapters/experiment/main}
% \include{chapters/conclusion/main}

%\bibliographystyle{natbib}
%\addcontentsline{toc}{chapter}{\bibliographytocname}
\bibliography{references}

% Appendix
\clearpage
\addappheadtotoc
\appendix
\appendixpage
% \include{appendix/experiment-instruments}

\end{document}
% %% RiSE Latex Template - version 0.5
%%
%% RiSE's latex template for thesis and dissertations
%% http://risetemplate.sourceforge.net
%%
%% (c) 2012 Yguaratã Cerqueira Cavalcanti (yguarata@gmail.com)
%%          Vinicius Cardoso Garcia (vinicius.garcia@gmail.com)
%%
%% This document was initially based on UFPEThesis template, from Paulo Gustavo
%% S. Fonseca.
%%
%% ACKNOWLEDGEMENTS
%%
%% We would like to thanks the RiSE's researchers community, the 
%% students from Federal University of Pernambuco, and other users that have
%% been contributing to this projects with comments and patches.
%%
%% GENERAL INSTRUCTIONS
%%
%% We strongly recommend you to compile your documents using pdflatex command.
%% It is also recommend use the texlipse plugin for Eclipse to edit your documents.
%%
%% Options for \documentclass command:
%%         * Idiom
%%           pt   - Portguese (default)
%%           en   - English
%%
%%         * Text type
%%           bsc  - B.Sc. Thesis
%%           msc  - M.Sc. Thesis (default)
%%           qual - PHD qualification (not tested yet)
%%           prop - PHD proposal (not tested yet)
%%           phd  - PHD thesis
%%
%%         * Media
%%           scr  - to eletronic version (PDF) / see the users guide
%%
%%         * Pagination
%%           oneside - unique face press
%%           twoside - two faces press
%%
%%		   * Line spacing
%%           singlespacing  - the same as using \linespread{1}
%%           onehalfspacing - the same as using \linespread{1.3}
%%           doublespacing  - the same as using \linespread{1.6}
%%
%% Reference commands. Use the following commands to make references in your
%% text:
%%          \figref  -- for Figure reference
%%          \tabref  -- for Table reference
%%          \eqnref  -- for equation reference
%%          \chapref -- for chapter reference
%%          \secref  -- for section reference
%%          \appref  -- for appendix reference
%%          \axiref  -- for axiom reference
%%          \conjref -- for conjecture reference
%%          \defref  -- for definition reference
%%          \lemref  -- for lemma reference
%%          \theoref -- for theorem reference
%%          \corref  -- for corollary reference
%%          \propref -- for proprosition reference
%%          \pgref   -- for page reference
%%
%%          Example: See \chapref{chap:introduction}. It will produce 
%%                   'See Chapter 1', in case of English language.

\documentclass[pt,twoside,onehalfspacing,bsc]{risethesis}

\usepackage[sort,square]{natbib} %package for bibliography
\bibliographystyle{ieeetr} %authordate1

\usepackage[utf8]{inputenc}
\usepackage[brazilian]{babel}
\usepackage[T1]{fontenc}

\usepackage{graphicx}

%% Change the following pdf author attribute name to your name.
\usepackage[linkcolor=blue,citecolor=blue,urlcolor=blue,colorlinks,pdfpagelabels,pdftitle={Bruno Cabral's Bachelor Thesis},pdfauthor={Bruno Cabral}]{hyperref}

\address{SALVADOR}

\universitypt{Universidade Federal da Bahia}
\universityen{Federal University of Bahia}

\departmentpt{Depertamento de Ciência da Computação}
\departmenten{Computer Science Department}

\programpt{Programa Multiinstitucional de Pós-graduação em Ciência da Computação}
\programen{Graduate in Computer Science}

\majorfieldpt{Ciência da Computação}
\majorfielden{Computer Science}

\title{Sistema de apoio à Pós graduação - UFBA}
\date{Outubro/2016}

\author{Victor de Azevedo Nunes}
\adviser{Ivan do Carmo Machado}

\begin{document}

\frontmatter
\frontpage
\presentationpage

\begin{dedicatory}
Eu dedico esta dissertação...
%I dedicate this dissertation to my family, girlfriend, friends and
%professors who gave me all necessary support to get here.
\end{dedicatory}

\acknowledgements
Meus agradecimentos...

\begin{epigraph}[]{Edward V Berard}
Walking on water and developing software from a specification are easy if both are frozen
\end{epigraph}

\resumo
% Escreva seu resumo no arquivo resumo.tex
\input{resumo}

\abstract
% Write your abstract in a file called abstract.tex
\input{abstract}

% Summary (tables of contents)
\tableofcontents

% List of figures
\listoffigures

% List of tables
\listoftables

% List of acronyms
% Acronyms manual: http://linorg.usp.br/CTAN/macros/latex/contrib/acronym/acronym.pdf
\listofacronyms
\input{acronyms}

% List of listings
%\lstlistoflistings

\mainmatter

\include{chapters/intro}
\include{chapters/referencial_teorico}

% \include{chapters/introduction/main}
% \include{chapters/background/main}
% \include{chapters/proposed_solution/main}
% \include{chapters/experiment/main}
% \include{chapters/conclusion/main}

%\bibliographystyle{natbib}
%\addcontentsline{toc}{chapter}{\bibliographytocname}
\bibliography{references}

% Appendix
\clearpage
\addappheadtotoc
\appendix
\appendixpage
% \include{appendix/experiment-instruments}

\end{document}
% %% RiSE Latex Template - version 0.5
%%
%% RiSE's latex template for thesis and dissertations
%% http://risetemplate.sourceforge.net
%%
%% (c) 2012 Yguaratã Cerqueira Cavalcanti (yguarata@gmail.com)
%%          Vinicius Cardoso Garcia (vinicius.garcia@gmail.com)
%%
%% This document was initially based on UFPEThesis template, from Paulo Gustavo
%% S. Fonseca.
%%
%% ACKNOWLEDGEMENTS
%%
%% We would like to thanks the RiSE's researchers community, the 
%% students from Federal University of Pernambuco, and other users that have
%% been contributing to this projects with comments and patches.
%%
%% GENERAL INSTRUCTIONS
%%
%% We strongly recommend you to compile your documents using pdflatex command.
%% It is also recommend use the texlipse plugin for Eclipse to edit your documents.
%%
%% Options for \documentclass command:
%%         * Idiom
%%           pt   - Portguese (default)
%%           en   - English
%%
%%         * Text type
%%           bsc  - B.Sc. Thesis
%%           msc  - M.Sc. Thesis (default)
%%           qual - PHD qualification (not tested yet)
%%           prop - PHD proposal (not tested yet)
%%           phd  - PHD thesis
%%
%%         * Media
%%           scr  - to eletronic version (PDF) / see the users guide
%%
%%         * Pagination
%%           oneside - unique face press
%%           twoside - two faces press
%%
%%		   * Line spacing
%%           singlespacing  - the same as using \linespread{1}
%%           onehalfspacing - the same as using \linespread{1.3}
%%           doublespacing  - the same as using \linespread{1.6}
%%
%% Reference commands. Use the following commands to make references in your
%% text:
%%          \figref  -- for Figure reference
%%          \tabref  -- for Table reference
%%          \eqnref  -- for equation reference
%%          \chapref -- for chapter reference
%%          \secref  -- for section reference
%%          \appref  -- for appendix reference
%%          \axiref  -- for axiom reference
%%          \conjref -- for conjecture reference
%%          \defref  -- for definition reference
%%          \lemref  -- for lemma reference
%%          \theoref -- for theorem reference
%%          \corref  -- for corollary reference
%%          \propref -- for proprosition reference
%%          \pgref   -- for page reference
%%
%%          Example: See \chapref{chap:introduction}. It will produce 
%%                   'See Chapter 1', in case of English language.

\documentclass[pt,twoside,onehalfspacing,bsc]{risethesis}

\usepackage[sort,square]{natbib} %package for bibliography
\bibliographystyle{ieeetr} %authordate1

\usepackage[utf8]{inputenc}
\usepackage[brazilian]{babel}
\usepackage[T1]{fontenc}

\usepackage{graphicx}

%% Change the following pdf author attribute name to your name.
\usepackage[linkcolor=blue,citecolor=blue,urlcolor=blue,colorlinks,pdfpagelabels,pdftitle={Bruno Cabral's Bachelor Thesis},pdfauthor={Bruno Cabral}]{hyperref}

\address{SALVADOR}

\universitypt{Universidade Federal da Bahia}
\universityen{Federal University of Bahia}

\departmentpt{Depertamento de Ciência da Computação}
\departmenten{Computer Science Department}

\programpt{Programa Multiinstitucional de Pós-graduação em Ciência da Computação}
\programen{Graduate in Computer Science}

\majorfieldpt{Ciência da Computação}
\majorfielden{Computer Science}

\title{Sistema de apoio à Pós graduação - UFBA}
\date{Outubro/2016}

\author{Victor de Azevedo Nunes}
\adviser{Ivan do Carmo Machado}

\begin{document}

\frontmatter
\frontpage
\presentationpage

\begin{dedicatory}
Eu dedico esta dissertação...
%I dedicate this dissertation to my family, girlfriend, friends and
%professors who gave me all necessary support to get here.
\end{dedicatory}

\acknowledgements
Meus agradecimentos...

\begin{epigraph}[]{Edward V Berard}
Walking on water and developing software from a specification are easy if both are frozen
\end{epigraph}

\resumo
% Escreva seu resumo no arquivo resumo.tex
\input{resumo}

\abstract
% Write your abstract in a file called abstract.tex
\input{abstract}

% Summary (tables of contents)
\tableofcontents

% List of figures
\listoffigures

% List of tables
\listoftables

% List of acronyms
% Acronyms manual: http://linorg.usp.br/CTAN/macros/latex/contrib/acronym/acronym.pdf
\listofacronyms
\input{acronyms}

% List of listings
%\lstlistoflistings

\mainmatter

\include{chapters/intro}
\include{chapters/referencial_teorico}

% \include{chapters/introduction/main}
% \include{chapters/background/main}
% \include{chapters/proposed_solution/main}
% \include{chapters/experiment/main}
% \include{chapters/conclusion/main}

%\bibliographystyle{natbib}
%\addcontentsline{toc}{chapter}{\bibliographytocname}
\bibliography{references}

% Appendix
\clearpage
\addappheadtotoc
\appendix
\appendixpage
% \include{appendix/experiment-instruments}

\end{document}

%\bibliographystyle{natbib}
%\addcontentsline{toc}{chapter}{\bibliographytocname}
\bibliography{references}

% Appendix
\clearpage
\addappheadtotoc
\appendix
\appendixpage
% \include{appendix/experiment-instruments}

\end{document}
% %% RiSE Latex Template - version 0.5
%%
%% RiSE's latex template for thesis and dissertations
%% http://risetemplate.sourceforge.net
%%
%% (c) 2012 Yguaratã Cerqueira Cavalcanti (yguarata@gmail.com)
%%          Vinicius Cardoso Garcia (vinicius.garcia@gmail.com)
%%
%% This document was initially based on UFPEThesis template, from Paulo Gustavo
%% S. Fonseca.
%%
%% ACKNOWLEDGEMENTS
%%
%% We would like to thanks the RiSE's researchers community, the 
%% students from Federal University of Pernambuco, and other users that have
%% been contributing to this projects with comments and patches.
%%
%% GENERAL INSTRUCTIONS
%%
%% We strongly recommend you to compile your documents using pdflatex command.
%% It is also recommend use the texlipse plugin for Eclipse to edit your documents.
%%
%% Options for \documentclass command:
%%         * Idiom
%%           pt   - Portguese (default)
%%           en   - English
%%
%%         * Text type
%%           bsc  - B.Sc. Thesis
%%           msc  - M.Sc. Thesis (default)
%%           qual - PHD qualification (not tested yet)
%%           prop - PHD proposal (not tested yet)
%%           phd  - PHD thesis
%%
%%         * Media
%%           scr  - to eletronic version (PDF) / see the users guide
%%
%%         * Pagination
%%           oneside - unique face press
%%           twoside - two faces press
%%
%%		   * Line spacing
%%           singlespacing  - the same as using \linespread{1}
%%           onehalfspacing - the same as using \linespread{1.3}
%%           doublespacing  - the same as using \linespread{1.6}
%%
%% Reference commands. Use the following commands to make references in your
%% text:
%%          \figref  -- for Figure reference
%%          \tabref  -- for Table reference
%%          \eqnref  -- for equation reference
%%          \chapref -- for chapter reference
%%          \secref  -- for section reference
%%          \appref  -- for appendix reference
%%          \axiref  -- for axiom reference
%%          \conjref -- for conjecture reference
%%          \defref  -- for definition reference
%%          \lemref  -- for lemma reference
%%          \theoref -- for theorem reference
%%          \corref  -- for corollary reference
%%          \propref -- for proprosition reference
%%          \pgref   -- for page reference
%%
%%          Example: See \chapref{chap:introduction}. It will produce 
%%                   'See Chapter 1', in case of English language.

\documentclass[pt,twoside,onehalfspacing,bsc]{risethesis}

\usepackage[sort,square]{natbib} %package for bibliography
\bibliographystyle{ieeetr} %authordate1

\usepackage[utf8]{inputenc}
\usepackage[brazilian]{babel}
\usepackage[T1]{fontenc}

\usepackage{graphicx}

%% Change the following pdf author attribute name to your name.
\usepackage[linkcolor=blue,citecolor=blue,urlcolor=blue,colorlinks,pdfpagelabels,pdftitle={Bruno Cabral's Bachelor Thesis},pdfauthor={Bruno Cabral}]{hyperref}

\address{SALVADOR}

\universitypt{Universidade Federal da Bahia}
\universityen{Federal University of Bahia}

\departmentpt{Depertamento de Ciência da Computação}
\departmenten{Computer Science Department}

\programpt{Programa Multiinstitucional de Pós-graduação em Ciência da Computação}
\programen{Graduate in Computer Science}

\majorfieldpt{Ciência da Computação}
\majorfielden{Computer Science}

\title{Sistema de apoio à Pós graduação - UFBA}
\date{Outubro/2016}

\author{Victor de Azevedo Nunes}
\adviser{Ivan do Carmo Machado}

\begin{document}

\frontmatter
\frontpage
\presentationpage

\begin{dedicatory}
Eu dedico esta dissertação...
%I dedicate this dissertation to my family, girlfriend, friends and
%professors who gave me all necessary support to get here.
\end{dedicatory}

\acknowledgements
Meus agradecimentos...

\begin{epigraph}[]{Edward V Berard}
Walking on water and developing software from a specification are easy if both are frozen
\end{epigraph}

\resumo
% Escreva seu resumo no arquivo resumo.tex
Este trabalho de conclusão de curso utiliza elementos da engenharia de software para propor um software como serviço a fim de otimizar o processo de construção e manutenção dos softwares. Assim, o objetivo deste SaaS é gerenciar as permissões de acesso de sistemas cliente, provendo o reuso de software e padronizando as soluções. O sistema proposto neste trabalho encontra-se implementado e disponível no Github, e traz fundamentos sobre a arquitetura e tecnologias utilizadas, bem como avaliações sobre possibilidades de utilização do mesmo em diversos ambientes, como Web e mobile.

\begin{keywords}
Software; Reuso; SaaS; Web; PHP
\end{keywords}

\abstract
% Write your abstract in a file called abstract.tex
My abstract...

\begin{keywords}
key words...
\end{keywords}

% Summary (tables of contents)
\tableofcontents

% List of figures
\listoffigures

% List of tables
\listoftables

% List of acronyms
% Acronyms manual: http://linorg.usp.br/CTAN/macros/latex/contrib/acronym/acronym.pdf
\listofacronyms
\begin{acronym}[ACRONYM] 
% Change the word ACRONYM above to change the acronym column width.
% The column width is equals to the width of the word that you put.
% Read the manual about acronym package for more examples:
%   http://linorg.usp.br/CTAN/macros/latex/contrib/acronym/acronym.pdf

\acro{SPA}{Single Page Application}
\acro{JSON}{Javascript Object Notation}
\acro{PHP}{PHP: Hypertext Preprocessor}
\acro{SaaS}{Software as a Service}
\acro{ERP}{Enterprise Resource Planning}
\acro{QoS}{Quality of Service}
\acro{UML}{Unified Modeling Language}
\acro{MVC}{Model-View-Controller}
\acro{Ajax}{Asynchronous Javascript and XML}
\acro{HTML}{HyperText Markup Language}
\acro{CSS}{Cascading Style Sheets}
\acro{API}{Application Programming Interface}
\acro{DOM}{Document Object Model}
\acro{BPMN}{Business Process Model and Notation}
\acro{REST}{Representational State Transfer}

\end{acronym}

% List of listings
%\lstlistoflistings

\mainmatter

\chapter{Introdução}

\section{Motivação}

Organizar os procedimentos de um processo sempre nos traz vantagens. Apesar de no processo de implantação de um sistema, o mesmo burocratizar o processo, com o tempo temos o retorno da dedicação para a inserção dos dados. Com um certo volume de dados, é possível estruturar informações que num processo manual são difíceis de serem enxergadas. Assim, é possível depender menos das pessoas que organizam o processo, pois o legado de informações não estará mais somente na mente de alguns, mas sim documentado nos dados do sistema.

Além de colaborar na organização, também haverá uma grande colaboração no tempo gasto na gestão. Lidar com muitos papéis e confiar na mente humana para guardar informações, não é uma alternativa muito segura devido ao fato que as pessoas sempre estão sujeitas a sair do processo e levar contigo a experiência obtida. Experiência essa que faz com que os procedimentos sejam executados de forma mais eficiente. Entretanto, com um sistema inteligente, é possível auxiliar e tornar mais ágil a execução das tarefas.


\section{Problema}


De acordo com funcionários ligados ao o setor de pós graduação da UFBA, entrevistados a fim de um maior entendimento do cenário, apesar das semelhanças estruturais, a pós graduação gerida de forma diferencia da graduação. FULANO afirma que devido ao fato de não ter a mesma visibilidade, não tem acesso aos mesmos recursos de gestão acadêmica da graduação. O professores não executam somente atividades dentro da sala de aula, também tem diversas outras ocupações no setor. E muitos procedimentos realizados extra classe ainda se encontram sendo realizados de forma manual, estando mais vulnerável ao erro ou até mesmo à violação do processo. Também ocorre um grande desperdício de tempo pelos professores e gestores da área, devido ao diversos processos ainda realizados de forma manual, sem a devida documentação. Segundo FULANO, também entrevistado, esse tempo perdido implica numa redução da eficiência na sala de aula, pois o professor acaba por ter menos tempo disponível para o planejamento das atividades, o que gera impactos negativos aos alunos.


\section{Objetivos} %<o que deve ser feito/entregue>


Devido aos muitos processos sendo resolvidos de forma manual, propõe-se com solução um sistema moderno, arquitetado para ter funcionamento na web e com um módulo mobile, a fim de fornecer informações de forma rápida e eficiente para os professores através de notificações, já que o acesso à internet móvel é comum entre os possíveis usuários do sistema em questão.
O principal requisito para o sistema seria dispor recursos para reduzir o tempo desperdiçado pelos professores durante as atividades extra classe.


\section{Metodologia} %<como será feito | como resolver o problema apontado inicialmente>


%<analise de literatura | design | implementação | validação>
Baseando-se nas tecnologias gratuitas em alta no cenário atual do desenvolvimento web, dispomos de algumas opções eficientes para a implementação da solução. Dentre as possibilidades, considerando a facilidade para futura manutenção e continuidade do projeto, tende-se a optar por uma tecnologia popular. Como linguagem de programação, adota-se o PHP. A escolha é fundamentada de acordo com a pesquisa da RedMonk de 2015, que evidencia o uso das linguagens de programação de acordo com as discussões no StackOverflow e repositórios no GitHub. É possível constatar a popularidade do PHP no cenário atual com o gráfico da pesquisa citada, na qual o PHP é apresentado na terceira colocação, apenas atrás do lider JavaScript e do segundo colocado, o Java.

\begin{figure}
	\label{fig:graficoRedmonk}
	\includegraphics[width=1\textwidth]{img/grafico_redmonk}
	\caption{Ranking das liguagens de programação no Stack Overflow e Github}
\end{figure}


Ainda assim, para compor a interface do dado projeto, também ocorrerá o uso do líder JavaScript de forma intensa, provendo o elo com o as informações gerenciadas pelo PHP.


Entretanto, não seria inteligente desenvolver um sistema completo sem o auxílio de um framework. Dentre os frameworks disponíveis para PHP, hoje o destaque está com o Laravel, que se encontra no topo dentre os mais utilizados no momento. 


A WebHostFace, uma empresa de hospedagem, compilou várias estatísticas para criar um infográfico mostrando os frameworks PHP mais populares de 2015. Utilizando informações sobre os próprios clientes, o Google Trends, estatísticas de repositórios do GitHub e a pesquisa do SitePoint “Best PHP Frameworks 2015”, a WebHostFace elaborou o seguinte infográfico: 

\begin{figure}
	\label{fig:graficoWebhostface}
	\includegraphics[width=1\textwidth]{img/infografico_webhostface}
	\caption{Infográfico da WebhostFace, exibindo a popularidade dos Frameworks PHP em 2015}
\end{figure}

Assim, tem-se a evidência que o Laravel em 2015 teve a maior popularidade em projetos pessoais e tem a maior comunidade entre os concorrentes, o que o torna uma boa escolha para a escrita de um software que será continuado por terceiros.


Para elaborar os recursos de interface e integrar ao back-end PHP do sistema, será adotado o já conhecido AngularJS, ferramenta sólida e conhecida no aspecto em questão. 


Dados coletados via Google Trends, que propõe comparações entre termos pesquisados, revela a popularidade do AngularJs diante de alguns dos principais concorrentes. O gráfico abaixo evidencia o cenário.


%Como mostra a Figura \ref{fig:graficoGoogleTrendsFerramentasFront}. 
\begin{figure}
	\label{fig:graficoGoogleTrendsFerramentasFront}
	\includegraphics[width=1\textwidth]{img/grafico_ferramentas_front}
	\caption{Gráfico do Google Trends exibindo as pesquisas por ferramentas front-end}
\end{figure}


Junto ao Angular JS, será utilizada a agradável tendência de interface do Material Design da Google, que propõe layouts limpos e otimizados já conhecidos pelos usuários de smartphones Android. 


Para a elaboração da plataforma mobile do projeto, será utilizado o Ionic Framework, muito difundido e bastante pesquisado na área, o que fica evidenciado com o gráfico de pesquisbaixo, coletado via Google Trends buscando por frameworks de desenvolvimento híbrido mobile.


\begin{figure}
	\label{fig:graficoGoogleTrendsFerramentasHibridasMobile}
	\includegraphics[width=1\textwidth]{img/grafico_ferramentas_hibridas_mobile}
	\caption{Gráfico do Google Trends exibindo as pesquisas por Frameworks híbridos mobile}
\end{figure}	

Para layout da interface mobile, também será aplicado a tendência do Material Design, a fim de propor uma harmonia entre o módulo web e mobile para os usuários


\section{Resultados Esperados}


Como fruto de um sistema para pós-graduação da UFBA, espera-se que os professores tenham mais recursos para integrar as atividades e também prover melhores condições para acompanhamento da vida acadêmica dos alunos em questão. Também, que os novos colaboradores que entrarem no processo tenham facilidade de compreender o fluxo do setor ao navegar pelo sistema proposto.


\section{Fora de Escopo}


Interação com os alunos devido às complicações para realizar a integração com o sistema empregado na UFBA, gerenciado pela XXXXXX, o que causaria uma inviabilidade no projeto devido à necessidade de entrega do produto ser mais forte que o tempo necessário para executar o processo de obtenção de acesso ao sistema legado para realizar a integração.


\section{Estrutura do Trabalho}


<breve resumo sobre os capítulos do TCC>
\chapter{Referencial Teórico}


Projetar o desenvolvimento de um software requer muito planejamento, pois as falhas iniciais podem custar bastante caro ou até mesmo inviabilizar a continuação de um projeto. Assim, a escolha da arquitetura ideal para a aplicabilidade é essencial na concepção de um produto de software. 
De todo o modo, sempre busca-se fazer mais com menos. Diante de tal filosofia, temos neste capítulo, uma breve discussão sobre alguns elementos de projeto e arquitetura de software, a fim de contextualizar este trabalho de conclusão de curso.
O capítulo corrente é composto por quatro seções. A \ref{sec:saas} trata de Software como serviço, discutindo alguns elementos do contexto que são relevantes para o trabalho proposto. A \ref{sec:reuso}  discute sobre a empregabilidade do reuso de software. A \ref{sec:modularizacao} seção, trata sobre aspectos envolvidos na modularização dos softwares. Por fim, a \ref{sec:apps_web} aborda as aplicações web, discutindo sobre aspectos relevantes sobre a aplicação web que compõe este trabalho.


 \section{Software como serviço}\label{sec:saas}


Segundo La e Chun \citep{La2009Systematic}, o princípio da definição de Software como um Serviço (Sofware as a Service - SaaS) é um serviço complementar para aplicações da computação em nuvem (cloud computing). No entanto, as áreas não se confundem. SaaS deve ser entendido como um mecanismo de suporte às soluções existentes na cloud. Os SaaS existem justamente para maximizar o reuso de serviços repetidos e não centrais em uma aplicação remota.


Como vantagens, diversos fatores podem ser favoráveis para a adoção de um SaaS, como custo e manutenção dentre outros fatores aplicáveis a soluções específicas. Lechesaet al. \citep{LechesaSS11}, quantificam índices sobre os fatores determinantes para adoção ou não de um SaaS voltado para ERP na África do Sul. Os principais fatores determinantes para adoção desse mecanismo de software são sua fluidez quanto à rede e a segurança. Esses fatores estão presentes na aplicação desenvolvida neste trabalho de conclusão de curso.


Devido ao fato de ter um serviço constantemente na nuvem, fica o questionamento sobre a segurança da informação manipulada. Sabe-se que a vulnerabilidade na Web não é restrita ao SaaS, atingindo diversos âmbitos. Rai et al. \cite{journals/corr/RaiSM13} consideram que o avanço da computação em nuvem não é um problema apenas para os serviços Web do ponto de vista da segurança, pois muitos trabalhos na literatura mostram a área como mais um ponto de vulnerabilidade para diversos setores, a exemplo de infraestrutura. No mesmo artigo mencionado de Rai et al. \cite{journals/corr/RaiSM13}, também realizaram-se estudos exploratórios junto a empresas usuárias de serviços em computação em nuvem e consideram que a perspectiva de SaaS também pode fortalecer a segurança nas aplicações de cloud computing, pois o software de autenticação compartilhado por várias aplicações em nuvem, oferece uma melhor padronização e consequente facilidade de prevenção a erros de vulnerabilidade específicas de cada módulo da pesquisa. Esse ponto de vista é fundamenta para a compreensão de SaaS.


A arquitetura de armazenamento de dados de um Saas pode variar de acordo com a necessidade do contexto. Huixin \cite{7586486} descreve possíveis modelagens nesse sentido. Essa abordagem pode ser com um banco de dados único, fazendo com que diferentes clientes compartilhem o mesmo banco, diferindo os dados através de controle de usuário, ou isolando os diferentes clientes através de bancos de dados exclusivos para cada um. Esse fator também pode ser combinado com a arquitetura da aplicação, caso ofereça aplicação única para todos os clientes ou aplicação compartilhada. Diante das possíveis abordagens, a modelagem de dados do software pode ser decidida pela regra de negócio. Este trabalho optou por aplicação única e banco de dados compartilhado.

Devido ao conceito particular de obtenção de software, tanto pela visão do cliente como pela visão do vendedor, é necessário compreender os diversos aspectos que podem ser relevantes ao orçar um Saas. O recente trabalho de T. Kaur et al. \citep{6949281} orienta um modelo para compor o custo de um Saas. O custo total seria composto pelos fatores que dão suporte ao funcionamento do software. Tais fatores incluem infra-estrutura, configurabilidade, customização, parâmetros de QoS (Quality of service) como escalabilidade, disponibilidade, usabilidade, pontualidade e desempenho da resposta, portabilidade, custo total de propriedade e retorno do investimento. Esses fatores caracterizam o custo de forma eficaz, possibilitando ao fornecedor, prover um Serviço de acordo com a exigência do consumidor em vários pacotes de serviços.


O conceito de software como serviço se aplica neste trabalho de conclusão de curso, pois o mesmo estará disponível na web com alta disponibilidade adotando as características apresentadas para qualquer pessoa que desejar usá-lo. Assim, basta se cadastrar, configurá-lo e usar como um serviço, sem preocupação com a manutenção do mesmo.


\section{Reuso de software}\label{sec:reuso} %CRUISE BOOK CAPITULO 2


De acordo com o livro Software pratical reuse \citep{Ezran2002}, o reuso de software é a utilização de qualquer informação que um desenvolvedor pode necessitar no processo de criação de software. O livro de Basili e Rombach \citep{Basili1991} define reutilização de software como o uso de tudo o que está associado a um projeto de conhecimento.
Assim, o objetivo da reutilização de software é reciclar o design, código e outros componentes de um produto de software e assim reduzir o custo, o tempo e melhorar a qualidade do produto.
Segundo Keswani et al. \cite{6783445}, o componente reutilizável de software pode ser qualquer parte de seu desenvolvimento, como um fragmento de código, design, casos de teste, ou até mesmo a especificação de requisitos de uma funcionalidade do software. 

O reuso de software pode ter impacto positivo em diversos aspectos do software, vejamos alguns, conforme apresentados no C.R.U.I.S.E Book \citep{cruise2007} :

\begin{itemize}

\item Qualidade: As correções de erro tornam-se úteis em todos os locais em que ocorreu, padronizando e facilitando a manutenção.

\item Produtividade: O ganho de produtividade é alcançado devido ao menor número de artefatos desenvolvido. Isso resulta em menor esforço de teste e também análise e design, reduzindo custos.

\item Confiabilidade: A utilização de componentes bem testados aumenta a
confiança no software. Além disso, a utilização de um mesmo componente em vários sistemas, aumenta a possibilidade de detecção de erros e reforça a confiança no componente.

\item Redução do Esforço: A reutilização de software proporciona uma redução do tempo de desenvolvimento, o que reduz o tempo necessário para o produto ser disponibilizado no mercado para trazer rentabilidade.

\item Trabalho redundante e tempo de desenvolvimento: Desenvolver um sistema do
zero significa desenvolvimento redundante de muitos componentes, como requisitos, especificações, casos de uso, arquitetura, etc. Isso pode ser evitado quando estes estão disponíveis como componentes reutilizáveis e podem ser compartilhados, resultando em um processo de desenvolvimento otimizado.

\item Documentação: Embora a documentação seja muito importante para a
manutenção de um sistema, muitas vezes é negligenciada. A reutilização de componentes de software reduz a quantidade de documentação a ser escrita, entretanto depende da qualidade do que está escrito. Assim, apenas a estrutura do sistema e os novos artefatos desenvolvidos necessitam ser documentados.

\item Custo de manutenção: Menos defeitos e manutenções são esperados quando tem-se comprovada a qualidade dos componentes utilizados.

\item Tamanho da equipe: É comum haver casos em que a equipe de desenvolvimento sofre sobrecarga. Entretanto, dobrar o tamanho da equipe de desenvolvimento não necessariamente duplica produtividade. Se muitos componentes podem ser reutilizados, é possível desenvolver com equipes menores, levando a melhor comunicação e aumento da produtividade.

\end{itemize}

Apesar dos benefícios da reutilização de software, ela não é suficientemente aproveitada. Existem fatores que influenciam direta ou indiretamente na sua adoção. Esses fatores podem ser de aspecto gerencial, organizacional, econômico, conceitual ou técnico. Veremos a seguir alguns aspectos que podem gerar conflito com a cultura de reuso de software, segundo o C.R.U.I.S.E Book \citep{cruise2007} :
%(Sametinger, 1997). REVER

\begin{itemize}
	
\item Falta de apoio da gestão: Como a reutilização de software gera custos iniciais,
a medida pode não ser amplamente alcançada em uma organização sem o apoio de alto nível de gestão. Os gestores têm de ser informados sobre os custos iniciais e serem convencidos sobre economias futuras.

\item Gerenciamento do Projeto: Gerenciar projetos tradicionais é uma tarefa árdua, principalmente, os que praticam a reutilização de software. Utilizando a técnica em larga escala, tem-se impacto sobre todo o ciclo de vida do software.

\item Estruturas organizacionais inadequadas: As estruturas organizacionais devem
considerar diferentes necessidades que surgem quando a reutilização em larga escala está sendo adotada. Por exemplo, uma equipe particionada pode ser alocada somente para desenvolver, manter e certificar componentes reutilizáveis de software.

\item Incentivos de gestão: É comum a falta de incentivo para deixar os desenvolvedores gastarem tempo elaborando componentes do sistemas. A produtividade é muitas vezes medida apenas no tempo necessário para concluir um projeto. Assim, fazer qualquer trabalho além disso, embora benéfico para a empresa como um todo, diminui o seu sucesso. Mesmo quando os componentes reutilizáveis são utilizados, os benefícios obtidos são uma pequena fração do que poderia ser alcançado caso houvesse reutilização explícita, planejada e organizada.

\item Dificuldade de encontrar software reutilizável: Para reutilizar os componentes, devem existir formas eficientes de busca. Além disso, é importante ter um repositório bem organizado contendo componentes com um eficiente meio de acesso.

\item Não reutilização do software encontrado. O acesso fácil ao software existente
não necessariamente aumentar a reutilização. Os componentes reutilizáveis devem ser cuidadosamente especificados, projetados, implementados e documentados, pois em alguns casos, modificar e adaptar o código  pode ser mais custoso que a programação da funcionalidade necessária a partir do zero.

\item Modificação: É muito difícil encontrar um componente que funcione
exatamente da mesma maneira que queremos. Desta forma, são necessárias modificações e devem existir formas de determinar os seus efeitos sobre o componente.


\end{itemize}


A definição do reuso de software pode ser associada a este trabalho, pois ainda que como um serviço, estaremos adotando uma aplicação única para servir aos mais diversos softwares de uma empresa, por exemplo. Logo, ao adotar o projeto, pode ser interessante aplicar algumas métricas para tomar conhecimento de possíveis vantanjens como as citadas nessa seção, a exemplo de economia de tempo de desenvolvimento, e custo do projeto.


%Outra diretriz importante para a reutilização de software é reduzir o risco na criação de novos softwares. O risco tende a ser bastante reduzido se os componentes que estão sendo reutilizados têm as documentação, interfaces necessárias e devidamente testadas, fatores que contibruem para uma fácil integração.
%De acordo com Keswani et al. \citep{6783445}, para o reuso de software dar retornos apropriados, o processo deve ser sistemático e planejado. Qualquer organização que implemente a reutilização de software deve identificar os melhores métodos e estratégias de reutilização para obter a máxima produtividade. A reutilização de software ajuda a evitar software de engenharia a partir do zero, pois usa módulos de software existentes. A reutilização de software, embora seja uma tarefa difícil, especialmente para softwares antigos sem padrões de projeto, pode melhorar significativamente a produtividade e a qualidade de um produto de software. Embora a reutilização de software não seja um novo campo, ela pode dar grandes retornos em curto período de tempo.


\section{Modularização}\label{sec:modularizacao} %artigo de claudio pagina 222 introdução


%A modularidade vem desempenhando um papel predominante estágios emergentes das disciplinas de arquitetura de software [13]. Engenheiros de software consideram modularidade como princípio base na comparação entre arquiteturas alternativas  e arquitetura degeneração [9]. De fato, os engenheiros de software são incentivados a arquitecturas, baseando-se numa multiplicidade de mecanismos de modularidade disponíveis em: 
%(i) Linguagens de descrição de arquitetura (ADLs), como ACME [8], 
%(ii) catálogos de arquitetônicos [2, 13], e 
%(iii) conhecem bem princípios de alto nível, como interfaces de componentes estreitos, acoplamento arquitectónico reduzido e semelhantes.


Conforme é frisado por Wickramaarachchi e Lai \citep{7062705}, o conceito de modularização na indústria de software tem uma longa história e tem sido utilizado para melhorar o processo de desenvolvimento de software em diferentes estágios. Os principais conceitos por trás da modularização do software foram introduzidos por pesquisadores pioneiros há quarenta anos, com uma notável contribuição feita por Melvin Conway e David Parnas, que tem representação notável na engenharia de software.


Modularizar um software é um bom padrão a ser adotado. Segundo Wickramaarachchi e Lai \citep{7062705}, a modularização é importante na identificação de dependências e reduz as dificuldades diante de uma possível necessidade de grandes alterações. De uma perspectiva da engenharia de software, uma modularização geralmente tem várias vantagens, tais como: tornar a complexidade do software mais gerenciável, facilitar o trabalho paralelo e tornar o software mais maleável para acomodar o futuro incerto que um software pode ter. O objetivo final da modularização do software é aumentar a produtividade ea qualidade do software. Tal conceito encontra-se bastante difundido e estái incorporado em linguagens de programação e ferramentas de software. O presente trabalho favorece ao uso da modularização de um software e até mesmo pode ser considerado um módulo a ser acoplado a qualquer software, mediante a compatibilidade.


Ao realizar a adoção do trabalho proposto, fica bastante evidente a "responsabilidade" do mesmo no escopo do projeto. Assim, é possível tratá-lo como um módulo do projeto que o usa, mesmo que seja consumido como um serviço.


\section{Aplicações web}\label{sec:apps_web}


A popularidade das soluções Web aumentou exponencialmente na última década e todos os dias cresce o número de pessoas usuárias desse tipo de software. E seguindo um padrão próprio, Kumar et al. \citep{7813710} sugerem que para o desenvolvimento web, deve-se manter a prática eficaz de produzir diagramas UML. A abordagem baseada na web oferece uma maneira fácil e eficaz para gerenciar e controlar o processo de desenvolvimento por meio de artefatos de modelagem. Tal abordagem pode ser usada quando há uma exigência de lidar com mudanças muito rápidas e grandes em requisitos de forma muito eficaz em muito menos tempo, gerando assim um menor impacto. 


Para atender à fomentada demanda de aplicativos web, é necessário adotar métodos de desenvolvimentos que sejam ágeis, eficientes e de fácil manutenção. Yu Ping et al. \cite{1372143} propõem o uso do modelo MVC (Model, View e Controller) no desenvolvimento para softwares web. O modelo apresentado tornou-se um padrão popular e divide o software em camadas com propósito definido, tornando-o de mais fácil manutenção.


O Ajax (Asynchronous Javascript and XML) revolucionou a web. Conforme demonstrado por Yuping \citep{6845605}, ao usar a tecnologia Ajax, podemos enriquecer a experiência do usuário em aplicações baseadas em navegador de internet, e fornecer uma variedade de aplicações interativas para atender às necessidade de humanização das aplicações.
Os aplicativos Ajax em execução no navegador se comunicam com um servidor Web de forma assíncrona e atualizam apenas uma parte da página.


No artigo de Tesarik et al. \citep{4664349}, temos que o design de software SPA (Single page application) é uma maneira interessante de criar um software Web numa uma única página. Essa solução de página única sem navegação funciona apenas com base em técnicas dinâmicas e assíncronas, como o AJAX (citado no parágrafo anterior). No entanto, esta abordagem coloca o desenvolvedor antes de alguns desafios substanciais.
Para projetar a interface do usuário que mostra as informações, é importante projetar corretamente a tela para manipular os dados do aplicativo numa única página. O design da página deve ser elaborado para maximizar a decomposição da página em componentes distintos que encapsulam os principais casos de uso.
Também recomenda-se que para elaborar os artefatos visuais, explore os mais ricos recursos na implementação, como HTML5, JavaScript, Ajax, CSS3, e outras tecnologias que se apliquem. Entretanto, desenvolver uma rica interface com o uso de diversas tecnologias/frameworks pode ocasionar um esforço maior para explorar as possibilidades.
Tais componentes da interface de uma aplicação SPA normalmente são alimentados mediante o consumo de uma API Rest via requisições AJAX. Assim, é possível particionar as responsabilidades de processamento do software entre cliente e servidor.


Como é possível denotar no trabalho de Salvadori e Siqueire \citep{7195633}, REST é uma arquitetura muito popular para integração de aplicativos web, que permite compartilhar e reutilizar informações através de sistemas. Aplicações de grande escala baseadas em no REST deve ser implementadas utilizando estratégias e mecanismos para produzir sistemas que sejam fáceis de desenvolver, reutilizar e manter. As interfaces de integração fornecidas por esses sistemas, chamadas de Web API, têm uma influência importante nas características da implementação resultante, pois a sua forma de resposta deve ser casada com a interface que o consome.
 
 
Os elementos comentados nessa seção encontram-se presentes neste trabalho de conclusão de curso. Juntos, montam a estrutura tecnológica necessária aliada à arquitetura adotada. Esses elementos seguem tendências atuais dos softwares web, fazendo com que esse trabalho esteja composto por tecnologias modernas que estão com boa aceitação no mercado.

% %% RiSE Latex Template - version 0.5
%%
%% RiSE's latex template for thesis and dissertations
%% http://risetemplate.sourceforge.net
%%
%% (c) 2012 Yguaratã Cerqueira Cavalcanti (yguarata@gmail.com)
%%          Vinicius Cardoso Garcia (vinicius.garcia@gmail.com)
%%
%% This document was initially based on UFPEThesis template, from Paulo Gustavo
%% S. Fonseca.
%%
%% ACKNOWLEDGEMENTS
%%
%% We would like to thanks the RiSE's researchers community, the 
%% students from Federal University of Pernambuco, and other users that have
%% been contributing to this projects with comments and patches.
%%
%% GENERAL INSTRUCTIONS
%%
%% We strongly recommend you to compile your documents using pdflatex command.
%% It is also recommend use the texlipse plugin for Eclipse to edit your documents.
%%
%% Options for \documentclass command:
%%         * Idiom
%%           pt   - Portguese (default)
%%           en   - English
%%
%%         * Text type
%%           bsc  - B.Sc. Thesis
%%           msc  - M.Sc. Thesis (default)
%%           qual - PHD qualification (not tested yet)
%%           prop - PHD proposal (not tested yet)
%%           phd  - PHD thesis
%%
%%         * Media
%%           scr  - to eletronic version (PDF) / see the users guide
%%
%%         * Pagination
%%           oneside - unique face press
%%           twoside - two faces press
%%
%%		   * Line spacing
%%           singlespacing  - the same as using \linespread{1}
%%           onehalfspacing - the same as using \linespread{1.3}
%%           doublespacing  - the same as using \linespread{1.6}
%%
%% Reference commands. Use the following commands to make references in your
%% text:
%%          \figref  -- for Figure reference
%%          \tabref  -- for Table reference
%%          \eqnref  -- for equation reference
%%          \chapref -- for chapter reference
%%          \secref  -- for section reference
%%          \appref  -- for appendix reference
%%          \axiref  -- for axiom reference
%%          \conjref -- for conjecture reference
%%          \defref  -- for definition reference
%%          \lemref  -- for lemma reference
%%          \theoref -- for theorem reference
%%          \corref  -- for corollary reference
%%          \propref -- for proprosition reference
%%          \pgref   -- for page reference
%%
%%          Example: See \chapref{chap:introduction}. It will produce 
%%                   'See Chapter 1', in case of English language.

\documentclass[pt,twoside,onehalfspacing,bsc]{risethesis}

\usepackage[sort,square]{natbib} %package for bibliography
\bibliographystyle{ieeetr} %authordate1

\usepackage[utf8]{inputenc}
\usepackage[brazilian]{babel}
\usepackage[T1]{fontenc}

\usepackage{graphicx}

%% Change the following pdf author attribute name to your name.
\usepackage[linkcolor=blue,citecolor=blue,urlcolor=blue,colorlinks,pdfpagelabels,pdftitle={Bruno Cabral's Bachelor Thesis},pdfauthor={Bruno Cabral}]{hyperref}

\address{SALVADOR}

\universitypt{Universidade Federal da Bahia}
\universityen{Federal University of Bahia}

\departmentpt{Depertamento de Ciência da Computação}
\departmenten{Computer Science Department}

\programpt{Programa Multiinstitucional de Pós-graduação em Ciência da Computação}
\programen{Graduate in Computer Science}

\majorfieldpt{Ciência da Computação}
\majorfielden{Computer Science}

\title{Sistema de apoio à Pós graduação - UFBA}
\date{Outubro/2016}

\author{Victor de Azevedo Nunes}
\adviser{Ivan do Carmo Machado}

\begin{document}

\frontmatter
\frontpage
\presentationpage

\begin{dedicatory}
Eu dedico esta dissertação...
%I dedicate this dissertation to my family, girlfriend, friends and
%professors who gave me all necessary support to get here.
\end{dedicatory}

\acknowledgements
Meus agradecimentos...

\begin{epigraph}[]{Edward V Berard}
Walking on water and developing software from a specification are easy if both are frozen
\end{epigraph}

\resumo
% Escreva seu resumo no arquivo resumo.tex
\input{resumo}

\abstract
% Write your abstract in a file called abstract.tex
\input{abstract}

% Summary (tables of contents)
\tableofcontents

% List of figures
\listoffigures

% List of tables
\listoftables

% List of acronyms
% Acronyms manual: http://linorg.usp.br/CTAN/macros/latex/contrib/acronym/acronym.pdf
\listofacronyms
\input{acronyms}

% List of listings
%\lstlistoflistings

\mainmatter

\include{chapters/intro}
\include{chapters/referencial_teorico}

% \include{chapters/introduction/main}
% \include{chapters/background/main}
% \include{chapters/proposed_solution/main}
% \include{chapters/experiment/main}
% \include{chapters/conclusion/main}

%\bibliographystyle{natbib}
%\addcontentsline{toc}{chapter}{\bibliographytocname}
\bibliography{references}

% Appendix
\clearpage
\addappheadtotoc
\appendix
\appendixpage
% \include{appendix/experiment-instruments}

\end{document}
% %% RiSE Latex Template - version 0.5
%%
%% RiSE's latex template for thesis and dissertations
%% http://risetemplate.sourceforge.net
%%
%% (c) 2012 Yguaratã Cerqueira Cavalcanti (yguarata@gmail.com)
%%          Vinicius Cardoso Garcia (vinicius.garcia@gmail.com)
%%
%% This document was initially based on UFPEThesis template, from Paulo Gustavo
%% S. Fonseca.
%%
%% ACKNOWLEDGEMENTS
%%
%% We would like to thanks the RiSE's researchers community, the 
%% students from Federal University of Pernambuco, and other users that have
%% been contributing to this projects with comments and patches.
%%
%% GENERAL INSTRUCTIONS
%%
%% We strongly recommend you to compile your documents using pdflatex command.
%% It is also recommend use the texlipse plugin for Eclipse to edit your documents.
%%
%% Options for \documentclass command:
%%         * Idiom
%%           pt   - Portguese (default)
%%           en   - English
%%
%%         * Text type
%%           bsc  - B.Sc. Thesis
%%           msc  - M.Sc. Thesis (default)
%%           qual - PHD qualification (not tested yet)
%%           prop - PHD proposal (not tested yet)
%%           phd  - PHD thesis
%%
%%         * Media
%%           scr  - to eletronic version (PDF) / see the users guide
%%
%%         * Pagination
%%           oneside - unique face press
%%           twoside - two faces press
%%
%%		   * Line spacing
%%           singlespacing  - the same as using \linespread{1}
%%           onehalfspacing - the same as using \linespread{1.3}
%%           doublespacing  - the same as using \linespread{1.6}
%%
%% Reference commands. Use the following commands to make references in your
%% text:
%%          \figref  -- for Figure reference
%%          \tabref  -- for Table reference
%%          \eqnref  -- for equation reference
%%          \chapref -- for chapter reference
%%          \secref  -- for section reference
%%          \appref  -- for appendix reference
%%          \axiref  -- for axiom reference
%%          \conjref -- for conjecture reference
%%          \defref  -- for definition reference
%%          \lemref  -- for lemma reference
%%          \theoref -- for theorem reference
%%          \corref  -- for corollary reference
%%          \propref -- for proprosition reference
%%          \pgref   -- for page reference
%%
%%          Example: See \chapref{chap:introduction}. It will produce 
%%                   'See Chapter 1', in case of English language.

\documentclass[pt,twoside,onehalfspacing,bsc]{risethesis}

\usepackage[sort,square]{natbib} %package for bibliography
\bibliographystyle{ieeetr} %authordate1

\usepackage[utf8]{inputenc}
\usepackage[brazilian]{babel}
\usepackage[T1]{fontenc}

\usepackage{graphicx}

%% Change the following pdf author attribute name to your name.
\usepackage[linkcolor=blue,citecolor=blue,urlcolor=blue,colorlinks,pdfpagelabels,pdftitle={Bruno Cabral's Bachelor Thesis},pdfauthor={Bruno Cabral}]{hyperref}

\address{SALVADOR}

\universitypt{Universidade Federal da Bahia}
\universityen{Federal University of Bahia}

\departmentpt{Depertamento de Ciência da Computação}
\departmenten{Computer Science Department}

\programpt{Programa Multiinstitucional de Pós-graduação em Ciência da Computação}
\programen{Graduate in Computer Science}

\majorfieldpt{Ciência da Computação}
\majorfielden{Computer Science}

\title{Sistema de apoio à Pós graduação - UFBA}
\date{Outubro/2016}

\author{Victor de Azevedo Nunes}
\adviser{Ivan do Carmo Machado}

\begin{document}

\frontmatter
\frontpage
\presentationpage

\begin{dedicatory}
Eu dedico esta dissertação...
%I dedicate this dissertation to my family, girlfriend, friends and
%professors who gave me all necessary support to get here.
\end{dedicatory}

\acknowledgements
Meus agradecimentos...

\begin{epigraph}[]{Edward V Berard}
Walking on water and developing software from a specification are easy if both are frozen
\end{epigraph}

\resumo
% Escreva seu resumo no arquivo resumo.tex
\input{resumo}

\abstract
% Write your abstract in a file called abstract.tex
\input{abstract}

% Summary (tables of contents)
\tableofcontents

% List of figures
\listoffigures

% List of tables
\listoftables

% List of acronyms
% Acronyms manual: http://linorg.usp.br/CTAN/macros/latex/contrib/acronym/acronym.pdf
\listofacronyms
\input{acronyms}

% List of listings
%\lstlistoflistings

\mainmatter

\include{chapters/intro}
\include{chapters/referencial_teorico}

% \include{chapters/introduction/main}
% \include{chapters/background/main}
% \include{chapters/proposed_solution/main}
% \include{chapters/experiment/main}
% \include{chapters/conclusion/main}

%\bibliographystyle{natbib}
%\addcontentsline{toc}{chapter}{\bibliographytocname}
\bibliography{references}

% Appendix
\clearpage
\addappheadtotoc
\appendix
\appendixpage
% \include{appendix/experiment-instruments}

\end{document}
% %% RiSE Latex Template - version 0.5
%%
%% RiSE's latex template for thesis and dissertations
%% http://risetemplate.sourceforge.net
%%
%% (c) 2012 Yguaratã Cerqueira Cavalcanti (yguarata@gmail.com)
%%          Vinicius Cardoso Garcia (vinicius.garcia@gmail.com)
%%
%% This document was initially based on UFPEThesis template, from Paulo Gustavo
%% S. Fonseca.
%%
%% ACKNOWLEDGEMENTS
%%
%% We would like to thanks the RiSE's researchers community, the 
%% students from Federal University of Pernambuco, and other users that have
%% been contributing to this projects with comments and patches.
%%
%% GENERAL INSTRUCTIONS
%%
%% We strongly recommend you to compile your documents using pdflatex command.
%% It is also recommend use the texlipse plugin for Eclipse to edit your documents.
%%
%% Options for \documentclass command:
%%         * Idiom
%%           pt   - Portguese (default)
%%           en   - English
%%
%%         * Text type
%%           bsc  - B.Sc. Thesis
%%           msc  - M.Sc. Thesis (default)
%%           qual - PHD qualification (not tested yet)
%%           prop - PHD proposal (not tested yet)
%%           phd  - PHD thesis
%%
%%         * Media
%%           scr  - to eletronic version (PDF) / see the users guide
%%
%%         * Pagination
%%           oneside - unique face press
%%           twoside - two faces press
%%
%%		   * Line spacing
%%           singlespacing  - the same as using \linespread{1}
%%           onehalfspacing - the same as using \linespread{1.3}
%%           doublespacing  - the same as using \linespread{1.6}
%%
%% Reference commands. Use the following commands to make references in your
%% text:
%%          \figref  -- for Figure reference
%%          \tabref  -- for Table reference
%%          \eqnref  -- for equation reference
%%          \chapref -- for chapter reference
%%          \secref  -- for section reference
%%          \appref  -- for appendix reference
%%          \axiref  -- for axiom reference
%%          \conjref -- for conjecture reference
%%          \defref  -- for definition reference
%%          \lemref  -- for lemma reference
%%          \theoref -- for theorem reference
%%          \corref  -- for corollary reference
%%          \propref -- for proprosition reference
%%          \pgref   -- for page reference
%%
%%          Example: See \chapref{chap:introduction}. It will produce 
%%                   'See Chapter 1', in case of English language.

\documentclass[pt,twoside,onehalfspacing,bsc]{risethesis}

\usepackage[sort,square]{natbib} %package for bibliography
\bibliographystyle{ieeetr} %authordate1

\usepackage[utf8]{inputenc}
\usepackage[brazilian]{babel}
\usepackage[T1]{fontenc}

\usepackage{graphicx}

%% Change the following pdf author attribute name to your name.
\usepackage[linkcolor=blue,citecolor=blue,urlcolor=blue,colorlinks,pdfpagelabels,pdftitle={Bruno Cabral's Bachelor Thesis},pdfauthor={Bruno Cabral}]{hyperref}

\address{SALVADOR}

\universitypt{Universidade Federal da Bahia}
\universityen{Federal University of Bahia}

\departmentpt{Depertamento de Ciência da Computação}
\departmenten{Computer Science Department}

\programpt{Programa Multiinstitucional de Pós-graduação em Ciência da Computação}
\programen{Graduate in Computer Science}

\majorfieldpt{Ciência da Computação}
\majorfielden{Computer Science}

\title{Sistema de apoio à Pós graduação - UFBA}
\date{Outubro/2016}

\author{Victor de Azevedo Nunes}
\adviser{Ivan do Carmo Machado}

\begin{document}

\frontmatter
\frontpage
\presentationpage

\begin{dedicatory}
Eu dedico esta dissertação...
%I dedicate this dissertation to my family, girlfriend, friends and
%professors who gave me all necessary support to get here.
\end{dedicatory}

\acknowledgements
Meus agradecimentos...

\begin{epigraph}[]{Edward V Berard}
Walking on water and developing software from a specification are easy if both are frozen
\end{epigraph}

\resumo
% Escreva seu resumo no arquivo resumo.tex
\input{resumo}

\abstract
% Write your abstract in a file called abstract.tex
\input{abstract}

% Summary (tables of contents)
\tableofcontents

% List of figures
\listoffigures

% List of tables
\listoftables

% List of acronyms
% Acronyms manual: http://linorg.usp.br/CTAN/macros/latex/contrib/acronym/acronym.pdf
\listofacronyms
\input{acronyms}

% List of listings
%\lstlistoflistings

\mainmatter

\include{chapters/intro}
\include{chapters/referencial_teorico}

% \include{chapters/introduction/main}
% \include{chapters/background/main}
% \include{chapters/proposed_solution/main}
% \include{chapters/experiment/main}
% \include{chapters/conclusion/main}

%\bibliographystyle{natbib}
%\addcontentsline{toc}{chapter}{\bibliographytocname}
\bibliography{references}

% Appendix
\clearpage
\addappheadtotoc
\appendix
\appendixpage
% \include{appendix/experiment-instruments}

\end{document}
% %% RiSE Latex Template - version 0.5
%%
%% RiSE's latex template for thesis and dissertations
%% http://risetemplate.sourceforge.net
%%
%% (c) 2012 Yguaratã Cerqueira Cavalcanti (yguarata@gmail.com)
%%          Vinicius Cardoso Garcia (vinicius.garcia@gmail.com)
%%
%% This document was initially based on UFPEThesis template, from Paulo Gustavo
%% S. Fonseca.
%%
%% ACKNOWLEDGEMENTS
%%
%% We would like to thanks the RiSE's researchers community, the 
%% students from Federal University of Pernambuco, and other users that have
%% been contributing to this projects with comments and patches.
%%
%% GENERAL INSTRUCTIONS
%%
%% We strongly recommend you to compile your documents using pdflatex command.
%% It is also recommend use the texlipse plugin for Eclipse to edit your documents.
%%
%% Options for \documentclass command:
%%         * Idiom
%%           pt   - Portguese (default)
%%           en   - English
%%
%%         * Text type
%%           bsc  - B.Sc. Thesis
%%           msc  - M.Sc. Thesis (default)
%%           qual - PHD qualification (not tested yet)
%%           prop - PHD proposal (not tested yet)
%%           phd  - PHD thesis
%%
%%         * Media
%%           scr  - to eletronic version (PDF) / see the users guide
%%
%%         * Pagination
%%           oneside - unique face press
%%           twoside - two faces press
%%
%%		   * Line spacing
%%           singlespacing  - the same as using \linespread{1}
%%           onehalfspacing - the same as using \linespread{1.3}
%%           doublespacing  - the same as using \linespread{1.6}
%%
%% Reference commands. Use the following commands to make references in your
%% text:
%%          \figref  -- for Figure reference
%%          \tabref  -- for Table reference
%%          \eqnref  -- for equation reference
%%          \chapref -- for chapter reference
%%          \secref  -- for section reference
%%          \appref  -- for appendix reference
%%          \axiref  -- for axiom reference
%%          \conjref -- for conjecture reference
%%          \defref  -- for definition reference
%%          \lemref  -- for lemma reference
%%          \theoref -- for theorem reference
%%          \corref  -- for corollary reference
%%          \propref -- for proprosition reference
%%          \pgref   -- for page reference
%%
%%          Example: See \chapref{chap:introduction}. It will produce 
%%                   'See Chapter 1', in case of English language.

\documentclass[pt,twoside,onehalfspacing,bsc]{risethesis}

\usepackage[sort,square]{natbib} %package for bibliography
\bibliographystyle{ieeetr} %authordate1

\usepackage[utf8]{inputenc}
\usepackage[brazilian]{babel}
\usepackage[T1]{fontenc}

\usepackage{graphicx}

%% Change the following pdf author attribute name to your name.
\usepackage[linkcolor=blue,citecolor=blue,urlcolor=blue,colorlinks,pdfpagelabels,pdftitle={Bruno Cabral's Bachelor Thesis},pdfauthor={Bruno Cabral}]{hyperref}

\address{SALVADOR}

\universitypt{Universidade Federal da Bahia}
\universityen{Federal University of Bahia}

\departmentpt{Depertamento de Ciência da Computação}
\departmenten{Computer Science Department}

\programpt{Programa Multiinstitucional de Pós-graduação em Ciência da Computação}
\programen{Graduate in Computer Science}

\majorfieldpt{Ciência da Computação}
\majorfielden{Computer Science}

\title{Sistema de apoio à Pós graduação - UFBA}
\date{Outubro/2016}

\author{Victor de Azevedo Nunes}
\adviser{Ivan do Carmo Machado}

\begin{document}

\frontmatter
\frontpage
\presentationpage

\begin{dedicatory}
Eu dedico esta dissertação...
%I dedicate this dissertation to my family, girlfriend, friends and
%professors who gave me all necessary support to get here.
\end{dedicatory}

\acknowledgements
Meus agradecimentos...

\begin{epigraph}[]{Edward V Berard}
Walking on water and developing software from a specification are easy if both are frozen
\end{epigraph}

\resumo
% Escreva seu resumo no arquivo resumo.tex
\input{resumo}

\abstract
% Write your abstract in a file called abstract.tex
\input{abstract}

% Summary (tables of contents)
\tableofcontents

% List of figures
\listoffigures

% List of tables
\listoftables

% List of acronyms
% Acronyms manual: http://linorg.usp.br/CTAN/macros/latex/contrib/acronym/acronym.pdf
\listofacronyms
\input{acronyms}

% List of listings
%\lstlistoflistings

\mainmatter

\include{chapters/intro}
\include{chapters/referencial_teorico}

% \include{chapters/introduction/main}
% \include{chapters/background/main}
% \include{chapters/proposed_solution/main}
% \include{chapters/experiment/main}
% \include{chapters/conclusion/main}

%\bibliographystyle{natbib}
%\addcontentsline{toc}{chapter}{\bibliographytocname}
\bibliography{references}

% Appendix
\clearpage
\addappheadtotoc
\appendix
\appendixpage
% \include{appendix/experiment-instruments}

\end{document}
% %% RiSE Latex Template - version 0.5
%%
%% RiSE's latex template for thesis and dissertations
%% http://risetemplate.sourceforge.net
%%
%% (c) 2012 Yguaratã Cerqueira Cavalcanti (yguarata@gmail.com)
%%          Vinicius Cardoso Garcia (vinicius.garcia@gmail.com)
%%
%% This document was initially based on UFPEThesis template, from Paulo Gustavo
%% S. Fonseca.
%%
%% ACKNOWLEDGEMENTS
%%
%% We would like to thanks the RiSE's researchers community, the 
%% students from Federal University of Pernambuco, and other users that have
%% been contributing to this projects with comments and patches.
%%
%% GENERAL INSTRUCTIONS
%%
%% We strongly recommend you to compile your documents using pdflatex command.
%% It is also recommend use the texlipse plugin for Eclipse to edit your documents.
%%
%% Options for \documentclass command:
%%         * Idiom
%%           pt   - Portguese (default)
%%           en   - English
%%
%%         * Text type
%%           bsc  - B.Sc. Thesis
%%           msc  - M.Sc. Thesis (default)
%%           qual - PHD qualification (not tested yet)
%%           prop - PHD proposal (not tested yet)
%%           phd  - PHD thesis
%%
%%         * Media
%%           scr  - to eletronic version (PDF) / see the users guide
%%
%%         * Pagination
%%           oneside - unique face press
%%           twoside - two faces press
%%
%%		   * Line spacing
%%           singlespacing  - the same as using \linespread{1}
%%           onehalfspacing - the same as using \linespread{1.3}
%%           doublespacing  - the same as using \linespread{1.6}
%%
%% Reference commands. Use the following commands to make references in your
%% text:
%%          \figref  -- for Figure reference
%%          \tabref  -- for Table reference
%%          \eqnref  -- for equation reference
%%          \chapref -- for chapter reference
%%          \secref  -- for section reference
%%          \appref  -- for appendix reference
%%          \axiref  -- for axiom reference
%%          \conjref -- for conjecture reference
%%          \defref  -- for definition reference
%%          \lemref  -- for lemma reference
%%          \theoref -- for theorem reference
%%          \corref  -- for corollary reference
%%          \propref -- for proprosition reference
%%          \pgref   -- for page reference
%%
%%          Example: See \chapref{chap:introduction}. It will produce 
%%                   'See Chapter 1', in case of English language.

\documentclass[pt,twoside,onehalfspacing,bsc]{risethesis}

\usepackage[sort,square]{natbib} %package for bibliography
\bibliographystyle{ieeetr} %authordate1

\usepackage[utf8]{inputenc}
\usepackage[brazilian]{babel}
\usepackage[T1]{fontenc}

\usepackage{graphicx}

%% Change the following pdf author attribute name to your name.
\usepackage[linkcolor=blue,citecolor=blue,urlcolor=blue,colorlinks,pdfpagelabels,pdftitle={Bruno Cabral's Bachelor Thesis},pdfauthor={Bruno Cabral}]{hyperref}

\address{SALVADOR}

\universitypt{Universidade Federal da Bahia}
\universityen{Federal University of Bahia}

\departmentpt{Depertamento de Ciência da Computação}
\departmenten{Computer Science Department}

\programpt{Programa Multiinstitucional de Pós-graduação em Ciência da Computação}
\programen{Graduate in Computer Science}

\majorfieldpt{Ciência da Computação}
\majorfielden{Computer Science}

\title{Sistema de apoio à Pós graduação - UFBA}
\date{Outubro/2016}

\author{Victor de Azevedo Nunes}
\adviser{Ivan do Carmo Machado}

\begin{document}

\frontmatter
\frontpage
\presentationpage

\begin{dedicatory}
Eu dedico esta dissertação...
%I dedicate this dissertation to my family, girlfriend, friends and
%professors who gave me all necessary support to get here.
\end{dedicatory}

\acknowledgements
Meus agradecimentos...

\begin{epigraph}[]{Edward V Berard}
Walking on water and developing software from a specification are easy if both are frozen
\end{epigraph}

\resumo
% Escreva seu resumo no arquivo resumo.tex
\input{resumo}

\abstract
% Write your abstract in a file called abstract.tex
\input{abstract}

% Summary (tables of contents)
\tableofcontents

% List of figures
\listoffigures

% List of tables
\listoftables

% List of acronyms
% Acronyms manual: http://linorg.usp.br/CTAN/macros/latex/contrib/acronym/acronym.pdf
\listofacronyms
\input{acronyms}

% List of listings
%\lstlistoflistings

\mainmatter

\include{chapters/intro}
\include{chapters/referencial_teorico}

% \include{chapters/introduction/main}
% \include{chapters/background/main}
% \include{chapters/proposed_solution/main}
% \include{chapters/experiment/main}
% \include{chapters/conclusion/main}

%\bibliographystyle{natbib}
%\addcontentsline{toc}{chapter}{\bibliographytocname}
\bibliography{references}

% Appendix
\clearpage
\addappheadtotoc
\appendix
\appendixpage
% \include{appendix/experiment-instruments}

\end{document}

%\bibliographystyle{natbib}
%\addcontentsline{toc}{chapter}{\bibliographytocname}
\bibliography{references}

% Appendix
\clearpage
\addappheadtotoc
\appendix
\appendixpage
% \include{appendix/experiment-instruments}

\end{document}
% %% RiSE Latex Template - version 0.5
%%
%% RiSE's latex template for thesis and dissertations
%% http://risetemplate.sourceforge.net
%%
%% (c) 2012 Yguaratã Cerqueira Cavalcanti (yguarata@gmail.com)
%%          Vinicius Cardoso Garcia (vinicius.garcia@gmail.com)
%%
%% This document was initially based on UFPEThesis template, from Paulo Gustavo
%% S. Fonseca.
%%
%% ACKNOWLEDGEMENTS
%%
%% We would like to thanks the RiSE's researchers community, the 
%% students from Federal University of Pernambuco, and other users that have
%% been contributing to this projects with comments and patches.
%%
%% GENERAL INSTRUCTIONS
%%
%% We strongly recommend you to compile your documents using pdflatex command.
%% It is also recommend use the texlipse plugin for Eclipse to edit your documents.
%%
%% Options for \documentclass command:
%%         * Idiom
%%           pt   - Portguese (default)
%%           en   - English
%%
%%         * Text type
%%           bsc  - B.Sc. Thesis
%%           msc  - M.Sc. Thesis (default)
%%           qual - PHD qualification (not tested yet)
%%           prop - PHD proposal (not tested yet)
%%           phd  - PHD thesis
%%
%%         * Media
%%           scr  - to eletronic version (PDF) / see the users guide
%%
%%         * Pagination
%%           oneside - unique face press
%%           twoside - two faces press
%%
%%		   * Line spacing
%%           singlespacing  - the same as using \linespread{1}
%%           onehalfspacing - the same as using \linespread{1.3}
%%           doublespacing  - the same as using \linespread{1.6}
%%
%% Reference commands. Use the following commands to make references in your
%% text:
%%          \figref  -- for Figure reference
%%          \tabref  -- for Table reference
%%          \eqnref  -- for equation reference
%%          \chapref -- for chapter reference
%%          \secref  -- for section reference
%%          \appref  -- for appendix reference
%%          \axiref  -- for axiom reference
%%          \conjref -- for conjecture reference
%%          \defref  -- for definition reference
%%          \lemref  -- for lemma reference
%%          \theoref -- for theorem reference
%%          \corref  -- for corollary reference
%%          \propref -- for proprosition reference
%%          \pgref   -- for page reference
%%
%%          Example: See \chapref{chap:introduction}. It will produce 
%%                   'See Chapter 1', in case of English language.

\documentclass[pt,twoside,onehalfspacing,bsc]{risethesis}

\usepackage[sort,square]{natbib} %package for bibliography
\bibliographystyle{ieeetr} %authordate1

\usepackage[utf8]{inputenc}
\usepackage[brazilian]{babel}
\usepackage[T1]{fontenc}

\usepackage{graphicx}

%% Change the following pdf author attribute name to your name.
\usepackage[linkcolor=blue,citecolor=blue,urlcolor=blue,colorlinks,pdfpagelabels,pdftitle={Bruno Cabral's Bachelor Thesis},pdfauthor={Bruno Cabral}]{hyperref}

\address{SALVADOR}

\universitypt{Universidade Federal da Bahia}
\universityen{Federal University of Bahia}

\departmentpt{Depertamento de Ciência da Computação}
\departmenten{Computer Science Department}

\programpt{Programa Multiinstitucional de Pós-graduação em Ciência da Computação}
\programen{Graduate in Computer Science}

\majorfieldpt{Ciência da Computação}
\majorfielden{Computer Science}

\title{Sistema de apoio à Pós graduação - UFBA}
\date{Outubro/2016}

\author{Victor de Azevedo Nunes}
\adviser{Ivan do Carmo Machado}

\begin{document}

\frontmatter
\frontpage
\presentationpage

\begin{dedicatory}
Eu dedico esta dissertação...
%I dedicate this dissertation to my family, girlfriend, friends and
%professors who gave me all necessary support to get here.
\end{dedicatory}

\acknowledgements
Meus agradecimentos...

\begin{epigraph}[]{Edward V Berard}
Walking on water and developing software from a specification are easy if both are frozen
\end{epigraph}

\resumo
% Escreva seu resumo no arquivo resumo.tex
Este trabalho de conclusão de curso utiliza elementos da engenharia de software para propor um software como serviço a fim de otimizar o processo de construção e manutenção dos softwares. Assim, o objetivo deste SaaS é gerenciar as permissões de acesso de sistemas cliente, provendo o reuso de software e padronizando as soluções. O sistema proposto neste trabalho encontra-se implementado e disponível no Github, e traz fundamentos sobre a arquitetura e tecnologias utilizadas, bem como avaliações sobre possibilidades de utilização do mesmo em diversos ambientes, como Web e mobile.

\begin{keywords}
Software; Reuso; SaaS; Web; PHP
\end{keywords}

\abstract
% Write your abstract in a file called abstract.tex
My abstract...

\begin{keywords}
key words...
\end{keywords}

% Summary (tables of contents)
\tableofcontents

% List of figures
\listoffigures

% List of tables
\listoftables

% List of acronyms
% Acronyms manual: http://linorg.usp.br/CTAN/macros/latex/contrib/acronym/acronym.pdf
\listofacronyms
\begin{acronym}[ACRONYM] 
% Change the word ACRONYM above to change the acronym column width.
% The column width is equals to the width of the word that you put.
% Read the manual about acronym package for more examples:
%   http://linorg.usp.br/CTAN/macros/latex/contrib/acronym/acronym.pdf

\acro{SPA}{Single Page Application}
\acro{JSON}{Javascript Object Notation}
\acro{PHP}{PHP: Hypertext Preprocessor}
\acro{SaaS}{Software as a Service}
\acro{ERP}{Enterprise Resource Planning}
\acro{QoS}{Quality of Service}
\acro{UML}{Unified Modeling Language}
\acro{MVC}{Model-View-Controller}
\acro{Ajax}{Asynchronous Javascript and XML}
\acro{HTML}{HyperText Markup Language}
\acro{CSS}{Cascading Style Sheets}
\acro{API}{Application Programming Interface}
\acro{DOM}{Document Object Model}
\acro{BPMN}{Business Process Model and Notation}
\acro{REST}{Representational State Transfer}

\end{acronym}

% List of listings
%\lstlistoflistings

\mainmatter

\chapter{Introdução}

\section{Motivação}

Organizar os procedimentos de um processo sempre nos traz vantagens. Apesar de no processo de implantação de um sistema, o mesmo burocratizar o processo, com o tempo temos o retorno da dedicação para a inserção dos dados. Com um certo volume de dados, é possível estruturar informações que num processo manual são difíceis de serem enxergadas. Assim, é possível depender menos das pessoas que organizam o processo, pois o legado de informações não estará mais somente na mente de alguns, mas sim documentado nos dados do sistema.

Além de colaborar na organização, também haverá uma grande colaboração no tempo gasto na gestão. Lidar com muitos papéis e confiar na mente humana para guardar informações, não é uma alternativa muito segura devido ao fato que as pessoas sempre estão sujeitas a sair do processo e levar contigo a experiência obtida. Experiência essa que faz com que os procedimentos sejam executados de forma mais eficiente. Entretanto, com um sistema inteligente, é possível auxiliar e tornar mais ágil a execução das tarefas.


\section{Problema}


De acordo com funcionários ligados ao o setor de pós graduação da UFBA, entrevistados a fim de um maior entendimento do cenário, apesar das semelhanças estruturais, a pós graduação gerida de forma diferencia da graduação. FULANO afirma que devido ao fato de não ter a mesma visibilidade, não tem acesso aos mesmos recursos de gestão acadêmica da graduação. O professores não executam somente atividades dentro da sala de aula, também tem diversas outras ocupações no setor. E muitos procedimentos realizados extra classe ainda se encontram sendo realizados de forma manual, estando mais vulnerável ao erro ou até mesmo à violação do processo. Também ocorre um grande desperdício de tempo pelos professores e gestores da área, devido ao diversos processos ainda realizados de forma manual, sem a devida documentação. Segundo FULANO, também entrevistado, esse tempo perdido implica numa redução da eficiência na sala de aula, pois o professor acaba por ter menos tempo disponível para o planejamento das atividades, o que gera impactos negativos aos alunos.


\section{Objetivos} %<o que deve ser feito/entregue>


Devido aos muitos processos sendo resolvidos de forma manual, propõe-se com solução um sistema moderno, arquitetado para ter funcionamento na web e com um módulo mobile, a fim de fornecer informações de forma rápida e eficiente para os professores através de notificações, já que o acesso à internet móvel é comum entre os possíveis usuários do sistema em questão.
O principal requisito para o sistema seria dispor recursos para reduzir o tempo desperdiçado pelos professores durante as atividades extra classe.


\section{Metodologia} %<como será feito | como resolver o problema apontado inicialmente>


%<analise de literatura | design | implementação | validação>
Baseando-se nas tecnologias gratuitas em alta no cenário atual do desenvolvimento web, dispomos de algumas opções eficientes para a implementação da solução. Dentre as possibilidades, considerando a facilidade para futura manutenção e continuidade do projeto, tende-se a optar por uma tecnologia popular. Como linguagem de programação, adota-se o PHP. A escolha é fundamentada de acordo com a pesquisa da RedMonk de 2015, que evidencia o uso das linguagens de programação de acordo com as discussões no StackOverflow e repositórios no GitHub. É possível constatar a popularidade do PHP no cenário atual com o gráfico da pesquisa citada, na qual o PHP é apresentado na terceira colocação, apenas atrás do lider JavaScript e do segundo colocado, o Java.

\begin{figure}
	\label{fig:graficoRedmonk}
	\includegraphics[width=1\textwidth]{img/grafico_redmonk}
	\caption{Ranking das liguagens de programação no Stack Overflow e Github}
\end{figure}


Ainda assim, para compor a interface do dado projeto, também ocorrerá o uso do líder JavaScript de forma intensa, provendo o elo com o as informações gerenciadas pelo PHP.


Entretanto, não seria inteligente desenvolver um sistema completo sem o auxílio de um framework. Dentre os frameworks disponíveis para PHP, hoje o destaque está com o Laravel, que se encontra no topo dentre os mais utilizados no momento. 


A WebHostFace, uma empresa de hospedagem, compilou várias estatísticas para criar um infográfico mostrando os frameworks PHP mais populares de 2015. Utilizando informações sobre os próprios clientes, o Google Trends, estatísticas de repositórios do GitHub e a pesquisa do SitePoint “Best PHP Frameworks 2015”, a WebHostFace elaborou o seguinte infográfico: 

\begin{figure}
	\label{fig:graficoWebhostface}
	\includegraphics[width=1\textwidth]{img/infografico_webhostface}
	\caption{Infográfico da WebhostFace, exibindo a popularidade dos Frameworks PHP em 2015}
\end{figure}

Assim, tem-se a evidência que o Laravel em 2015 teve a maior popularidade em projetos pessoais e tem a maior comunidade entre os concorrentes, o que o torna uma boa escolha para a escrita de um software que será continuado por terceiros.


Para elaborar os recursos de interface e integrar ao back-end PHP do sistema, será adotado o já conhecido AngularJS, ferramenta sólida e conhecida no aspecto em questão. 


Dados coletados via Google Trends, que propõe comparações entre termos pesquisados, revela a popularidade do AngularJs diante de alguns dos principais concorrentes. O gráfico abaixo evidencia o cenário.


%Como mostra a Figura \ref{fig:graficoGoogleTrendsFerramentasFront}. 
\begin{figure}
	\label{fig:graficoGoogleTrendsFerramentasFront}
	\includegraphics[width=1\textwidth]{img/grafico_ferramentas_front}
	\caption{Gráfico do Google Trends exibindo as pesquisas por ferramentas front-end}
\end{figure}


Junto ao Angular JS, será utilizada a agradável tendência de interface do Material Design da Google, que propõe layouts limpos e otimizados já conhecidos pelos usuários de smartphones Android. 


Para a elaboração da plataforma mobile do projeto, será utilizado o Ionic Framework, muito difundido e bastante pesquisado na área, o que fica evidenciado com o gráfico de pesquisbaixo, coletado via Google Trends buscando por frameworks de desenvolvimento híbrido mobile.


\begin{figure}
	\label{fig:graficoGoogleTrendsFerramentasHibridasMobile}
	\includegraphics[width=1\textwidth]{img/grafico_ferramentas_hibridas_mobile}
	\caption{Gráfico do Google Trends exibindo as pesquisas por Frameworks híbridos mobile}
\end{figure}	

Para layout da interface mobile, também será aplicado a tendência do Material Design, a fim de propor uma harmonia entre o módulo web e mobile para os usuários


\section{Resultados Esperados}


Como fruto de um sistema para pós-graduação da UFBA, espera-se que os professores tenham mais recursos para integrar as atividades e também prover melhores condições para acompanhamento da vida acadêmica dos alunos em questão. Também, que os novos colaboradores que entrarem no processo tenham facilidade de compreender o fluxo do setor ao navegar pelo sistema proposto.


\section{Fora de Escopo}


Interação com os alunos devido às complicações para realizar a integração com o sistema empregado na UFBA, gerenciado pela XXXXXX, o que causaria uma inviabilidade no projeto devido à necessidade de entrega do produto ser mais forte que o tempo necessário para executar o processo de obtenção de acesso ao sistema legado para realizar a integração.


\section{Estrutura do Trabalho}


<breve resumo sobre os capítulos do TCC>
\chapter{Referencial Teórico}


Projetar o desenvolvimento de um software requer muito planejamento, pois as falhas iniciais podem custar bastante caro ou até mesmo inviabilizar a continuação de um projeto. Assim, a escolha da arquitetura ideal para a aplicabilidade é essencial na concepção de um produto de software. 
De todo o modo, sempre busca-se fazer mais com menos. Diante de tal filosofia, temos neste capítulo, uma breve discussão sobre alguns elementos de projeto e arquitetura de software, a fim de contextualizar este trabalho de conclusão de curso.
O capítulo corrente é composto por quatro seções. A \ref{sec:saas} trata de Software como serviço, discutindo alguns elementos do contexto que são relevantes para o trabalho proposto. A \ref{sec:reuso}  discute sobre a empregabilidade do reuso de software. A \ref{sec:modularizacao} seção, trata sobre aspectos envolvidos na modularização dos softwares. Por fim, a \ref{sec:apps_web} aborda as aplicações web, discutindo sobre aspectos relevantes sobre a aplicação web que compõe este trabalho.


 \section{Software como serviço}\label{sec:saas}


Segundo La e Chun \citep{La2009Systematic}, o princípio da definição de Software como um Serviço (Sofware as a Service - SaaS) é um serviço complementar para aplicações da computação em nuvem (cloud computing). No entanto, as áreas não se confundem. SaaS deve ser entendido como um mecanismo de suporte às soluções existentes na cloud. Os SaaS existem justamente para maximizar o reuso de serviços repetidos e não centrais em uma aplicação remota.


Como vantagens, diversos fatores podem ser favoráveis para a adoção de um SaaS, como custo e manutenção dentre outros fatores aplicáveis a soluções específicas. Lechesaet al. \citep{LechesaSS11}, quantificam índices sobre os fatores determinantes para adoção ou não de um SaaS voltado para ERP na África do Sul. Os principais fatores determinantes para adoção desse mecanismo de software são sua fluidez quanto à rede e a segurança. Esses fatores estão presentes na aplicação desenvolvida neste trabalho de conclusão de curso.


Devido ao fato de ter um serviço constantemente na nuvem, fica o questionamento sobre a segurança da informação manipulada. Sabe-se que a vulnerabilidade na Web não é restrita ao SaaS, atingindo diversos âmbitos. Rai et al. \cite{journals/corr/RaiSM13} consideram que o avanço da computação em nuvem não é um problema apenas para os serviços Web do ponto de vista da segurança, pois muitos trabalhos na literatura mostram a área como mais um ponto de vulnerabilidade para diversos setores, a exemplo de infraestrutura. No mesmo artigo mencionado de Rai et al. \cite{journals/corr/RaiSM13}, também realizaram-se estudos exploratórios junto a empresas usuárias de serviços em computação em nuvem e consideram que a perspectiva de SaaS também pode fortalecer a segurança nas aplicações de cloud computing, pois o software de autenticação compartilhado por várias aplicações em nuvem, oferece uma melhor padronização e consequente facilidade de prevenção a erros de vulnerabilidade específicas de cada módulo da pesquisa. Esse ponto de vista é fundamenta para a compreensão de SaaS.


A arquitetura de armazenamento de dados de um Saas pode variar de acordo com a necessidade do contexto. Huixin \cite{7586486} descreve possíveis modelagens nesse sentido. Essa abordagem pode ser com um banco de dados único, fazendo com que diferentes clientes compartilhem o mesmo banco, diferindo os dados através de controle de usuário, ou isolando os diferentes clientes através de bancos de dados exclusivos para cada um. Esse fator também pode ser combinado com a arquitetura da aplicação, caso ofereça aplicação única para todos os clientes ou aplicação compartilhada. Diante das possíveis abordagens, a modelagem de dados do software pode ser decidida pela regra de negócio. Este trabalho optou por aplicação única e banco de dados compartilhado.

Devido ao conceito particular de obtenção de software, tanto pela visão do cliente como pela visão do vendedor, é necessário compreender os diversos aspectos que podem ser relevantes ao orçar um Saas. O recente trabalho de T. Kaur et al. \citep{6949281} orienta um modelo para compor o custo de um Saas. O custo total seria composto pelos fatores que dão suporte ao funcionamento do software. Tais fatores incluem infra-estrutura, configurabilidade, customização, parâmetros de QoS (Quality of service) como escalabilidade, disponibilidade, usabilidade, pontualidade e desempenho da resposta, portabilidade, custo total de propriedade e retorno do investimento. Esses fatores caracterizam o custo de forma eficaz, possibilitando ao fornecedor, prover um Serviço de acordo com a exigência do consumidor em vários pacotes de serviços.


O conceito de software como serviço se aplica neste trabalho de conclusão de curso, pois o mesmo estará disponível na web com alta disponibilidade adotando as características apresentadas para qualquer pessoa que desejar usá-lo. Assim, basta se cadastrar, configurá-lo e usar como um serviço, sem preocupação com a manutenção do mesmo.


\section{Reuso de software}\label{sec:reuso} %CRUISE BOOK CAPITULO 2


De acordo com o livro Software pratical reuse \citep{Ezran2002}, o reuso de software é a utilização de qualquer informação que um desenvolvedor pode necessitar no processo de criação de software. O livro de Basili e Rombach \citep{Basili1991} define reutilização de software como o uso de tudo o que está associado a um projeto de conhecimento.
Assim, o objetivo da reutilização de software é reciclar o design, código e outros componentes de um produto de software e assim reduzir o custo, o tempo e melhorar a qualidade do produto.
Segundo Keswani et al. \cite{6783445}, o componente reutilizável de software pode ser qualquer parte de seu desenvolvimento, como um fragmento de código, design, casos de teste, ou até mesmo a especificação de requisitos de uma funcionalidade do software. 

O reuso de software pode ter impacto positivo em diversos aspectos do software, vejamos alguns, conforme apresentados no C.R.U.I.S.E Book \citep{cruise2007} :

\begin{itemize}

\item Qualidade: As correções de erro tornam-se úteis em todos os locais em que ocorreu, padronizando e facilitando a manutenção.

\item Produtividade: O ganho de produtividade é alcançado devido ao menor número de artefatos desenvolvido. Isso resulta em menor esforço de teste e também análise e design, reduzindo custos.

\item Confiabilidade: A utilização de componentes bem testados aumenta a
confiança no software. Além disso, a utilização de um mesmo componente em vários sistemas, aumenta a possibilidade de detecção de erros e reforça a confiança no componente.

\item Redução do Esforço: A reutilização de software proporciona uma redução do tempo de desenvolvimento, o que reduz o tempo necessário para o produto ser disponibilizado no mercado para trazer rentabilidade.

\item Trabalho redundante e tempo de desenvolvimento: Desenvolver um sistema do
zero significa desenvolvimento redundante de muitos componentes, como requisitos, especificações, casos de uso, arquitetura, etc. Isso pode ser evitado quando estes estão disponíveis como componentes reutilizáveis e podem ser compartilhados, resultando em um processo de desenvolvimento otimizado.

\item Documentação: Embora a documentação seja muito importante para a
manutenção de um sistema, muitas vezes é negligenciada. A reutilização de componentes de software reduz a quantidade de documentação a ser escrita, entretanto depende da qualidade do que está escrito. Assim, apenas a estrutura do sistema e os novos artefatos desenvolvidos necessitam ser documentados.

\item Custo de manutenção: Menos defeitos e manutenções são esperados quando tem-se comprovada a qualidade dos componentes utilizados.

\item Tamanho da equipe: É comum haver casos em que a equipe de desenvolvimento sofre sobrecarga. Entretanto, dobrar o tamanho da equipe de desenvolvimento não necessariamente duplica produtividade. Se muitos componentes podem ser reutilizados, é possível desenvolver com equipes menores, levando a melhor comunicação e aumento da produtividade.

\end{itemize}

Apesar dos benefícios da reutilização de software, ela não é suficientemente aproveitada. Existem fatores que influenciam direta ou indiretamente na sua adoção. Esses fatores podem ser de aspecto gerencial, organizacional, econômico, conceitual ou técnico. Veremos a seguir alguns aspectos que podem gerar conflito com a cultura de reuso de software, segundo o C.R.U.I.S.E Book \citep{cruise2007} :
%(Sametinger, 1997). REVER

\begin{itemize}
	
\item Falta de apoio da gestão: Como a reutilização de software gera custos iniciais,
a medida pode não ser amplamente alcançada em uma organização sem o apoio de alto nível de gestão. Os gestores têm de ser informados sobre os custos iniciais e serem convencidos sobre economias futuras.

\item Gerenciamento do Projeto: Gerenciar projetos tradicionais é uma tarefa árdua, principalmente, os que praticam a reutilização de software. Utilizando a técnica em larga escala, tem-se impacto sobre todo o ciclo de vida do software.

\item Estruturas organizacionais inadequadas: As estruturas organizacionais devem
considerar diferentes necessidades que surgem quando a reutilização em larga escala está sendo adotada. Por exemplo, uma equipe particionada pode ser alocada somente para desenvolver, manter e certificar componentes reutilizáveis de software.

\item Incentivos de gestão: É comum a falta de incentivo para deixar os desenvolvedores gastarem tempo elaborando componentes do sistemas. A produtividade é muitas vezes medida apenas no tempo necessário para concluir um projeto. Assim, fazer qualquer trabalho além disso, embora benéfico para a empresa como um todo, diminui o seu sucesso. Mesmo quando os componentes reutilizáveis são utilizados, os benefícios obtidos são uma pequena fração do que poderia ser alcançado caso houvesse reutilização explícita, planejada e organizada.

\item Dificuldade de encontrar software reutilizável: Para reutilizar os componentes, devem existir formas eficientes de busca. Além disso, é importante ter um repositório bem organizado contendo componentes com um eficiente meio de acesso.

\item Não reutilização do software encontrado. O acesso fácil ao software existente
não necessariamente aumentar a reutilização. Os componentes reutilizáveis devem ser cuidadosamente especificados, projetados, implementados e documentados, pois em alguns casos, modificar e adaptar o código  pode ser mais custoso que a programação da funcionalidade necessária a partir do zero.

\item Modificação: É muito difícil encontrar um componente que funcione
exatamente da mesma maneira que queremos. Desta forma, são necessárias modificações e devem existir formas de determinar os seus efeitos sobre o componente.


\end{itemize}


A definição do reuso de software pode ser associada a este trabalho, pois ainda que como um serviço, estaremos adotando uma aplicação única para servir aos mais diversos softwares de uma empresa, por exemplo. Logo, ao adotar o projeto, pode ser interessante aplicar algumas métricas para tomar conhecimento de possíveis vantanjens como as citadas nessa seção, a exemplo de economia de tempo de desenvolvimento, e custo do projeto.


%Outra diretriz importante para a reutilização de software é reduzir o risco na criação de novos softwares. O risco tende a ser bastante reduzido se os componentes que estão sendo reutilizados têm as documentação, interfaces necessárias e devidamente testadas, fatores que contibruem para uma fácil integração.
%De acordo com Keswani et al. \citep{6783445}, para o reuso de software dar retornos apropriados, o processo deve ser sistemático e planejado. Qualquer organização que implemente a reutilização de software deve identificar os melhores métodos e estratégias de reutilização para obter a máxima produtividade. A reutilização de software ajuda a evitar software de engenharia a partir do zero, pois usa módulos de software existentes. A reutilização de software, embora seja uma tarefa difícil, especialmente para softwares antigos sem padrões de projeto, pode melhorar significativamente a produtividade e a qualidade de um produto de software. Embora a reutilização de software não seja um novo campo, ela pode dar grandes retornos em curto período de tempo.


\section{Modularização}\label{sec:modularizacao} %artigo de claudio pagina 222 introdução


%A modularidade vem desempenhando um papel predominante estágios emergentes das disciplinas de arquitetura de software [13]. Engenheiros de software consideram modularidade como princípio base na comparação entre arquiteturas alternativas  e arquitetura degeneração [9]. De fato, os engenheiros de software são incentivados a arquitecturas, baseando-se numa multiplicidade de mecanismos de modularidade disponíveis em: 
%(i) Linguagens de descrição de arquitetura (ADLs), como ACME [8], 
%(ii) catálogos de arquitetônicos [2, 13], e 
%(iii) conhecem bem princípios de alto nível, como interfaces de componentes estreitos, acoplamento arquitectónico reduzido e semelhantes.


Conforme é frisado por Wickramaarachchi e Lai \citep{7062705}, o conceito de modularização na indústria de software tem uma longa história e tem sido utilizado para melhorar o processo de desenvolvimento de software em diferentes estágios. Os principais conceitos por trás da modularização do software foram introduzidos por pesquisadores pioneiros há quarenta anos, com uma notável contribuição feita por Melvin Conway e David Parnas, que tem representação notável na engenharia de software.


Modularizar um software é um bom padrão a ser adotado. Segundo Wickramaarachchi e Lai \citep{7062705}, a modularização é importante na identificação de dependências e reduz as dificuldades diante de uma possível necessidade de grandes alterações. De uma perspectiva da engenharia de software, uma modularização geralmente tem várias vantagens, tais como: tornar a complexidade do software mais gerenciável, facilitar o trabalho paralelo e tornar o software mais maleável para acomodar o futuro incerto que um software pode ter. O objetivo final da modularização do software é aumentar a produtividade ea qualidade do software. Tal conceito encontra-se bastante difundido e estái incorporado em linguagens de programação e ferramentas de software. O presente trabalho favorece ao uso da modularização de um software e até mesmo pode ser considerado um módulo a ser acoplado a qualquer software, mediante a compatibilidade.


Ao realizar a adoção do trabalho proposto, fica bastante evidente a "responsabilidade" do mesmo no escopo do projeto. Assim, é possível tratá-lo como um módulo do projeto que o usa, mesmo que seja consumido como um serviço.


\section{Aplicações web}\label{sec:apps_web}


A popularidade das soluções Web aumentou exponencialmente na última década e todos os dias cresce o número de pessoas usuárias desse tipo de software. E seguindo um padrão próprio, Kumar et al. \citep{7813710} sugerem que para o desenvolvimento web, deve-se manter a prática eficaz de produzir diagramas UML. A abordagem baseada na web oferece uma maneira fácil e eficaz para gerenciar e controlar o processo de desenvolvimento por meio de artefatos de modelagem. Tal abordagem pode ser usada quando há uma exigência de lidar com mudanças muito rápidas e grandes em requisitos de forma muito eficaz em muito menos tempo, gerando assim um menor impacto. 


Para atender à fomentada demanda de aplicativos web, é necessário adotar métodos de desenvolvimentos que sejam ágeis, eficientes e de fácil manutenção. Yu Ping et al. \cite{1372143} propõem o uso do modelo MVC (Model, View e Controller) no desenvolvimento para softwares web. O modelo apresentado tornou-se um padrão popular e divide o software em camadas com propósito definido, tornando-o de mais fácil manutenção.


O Ajax (Asynchronous Javascript and XML) revolucionou a web. Conforme demonstrado por Yuping \citep{6845605}, ao usar a tecnologia Ajax, podemos enriquecer a experiência do usuário em aplicações baseadas em navegador de internet, e fornecer uma variedade de aplicações interativas para atender às necessidade de humanização das aplicações.
Os aplicativos Ajax em execução no navegador se comunicam com um servidor Web de forma assíncrona e atualizam apenas uma parte da página.


No artigo de Tesarik et al. \citep{4664349}, temos que o design de software SPA (Single page application) é uma maneira interessante de criar um software Web numa uma única página. Essa solução de página única sem navegação funciona apenas com base em técnicas dinâmicas e assíncronas, como o AJAX (citado no parágrafo anterior). No entanto, esta abordagem coloca o desenvolvedor antes de alguns desafios substanciais.
Para projetar a interface do usuário que mostra as informações, é importante projetar corretamente a tela para manipular os dados do aplicativo numa única página. O design da página deve ser elaborado para maximizar a decomposição da página em componentes distintos que encapsulam os principais casos de uso.
Também recomenda-se que para elaborar os artefatos visuais, explore os mais ricos recursos na implementação, como HTML5, JavaScript, Ajax, CSS3, e outras tecnologias que se apliquem. Entretanto, desenvolver uma rica interface com o uso de diversas tecnologias/frameworks pode ocasionar um esforço maior para explorar as possibilidades.
Tais componentes da interface de uma aplicação SPA normalmente são alimentados mediante o consumo de uma API Rest via requisições AJAX. Assim, é possível particionar as responsabilidades de processamento do software entre cliente e servidor.


Como é possível denotar no trabalho de Salvadori e Siqueire \citep{7195633}, REST é uma arquitetura muito popular para integração de aplicativos web, que permite compartilhar e reutilizar informações através de sistemas. Aplicações de grande escala baseadas em no REST deve ser implementadas utilizando estratégias e mecanismos para produzir sistemas que sejam fáceis de desenvolver, reutilizar e manter. As interfaces de integração fornecidas por esses sistemas, chamadas de Web API, têm uma influência importante nas características da implementação resultante, pois a sua forma de resposta deve ser casada com a interface que o consome.
 
 
Os elementos comentados nessa seção encontram-se presentes neste trabalho de conclusão de curso. Juntos, montam a estrutura tecnológica necessária aliada à arquitetura adotada. Esses elementos seguem tendências atuais dos softwares web, fazendo com que esse trabalho esteja composto por tecnologias modernas que estão com boa aceitação no mercado.

% %% RiSE Latex Template - version 0.5
%%
%% RiSE's latex template for thesis and dissertations
%% http://risetemplate.sourceforge.net
%%
%% (c) 2012 Yguaratã Cerqueira Cavalcanti (yguarata@gmail.com)
%%          Vinicius Cardoso Garcia (vinicius.garcia@gmail.com)
%%
%% This document was initially based on UFPEThesis template, from Paulo Gustavo
%% S. Fonseca.
%%
%% ACKNOWLEDGEMENTS
%%
%% We would like to thanks the RiSE's researchers community, the 
%% students from Federal University of Pernambuco, and other users that have
%% been contributing to this projects with comments and patches.
%%
%% GENERAL INSTRUCTIONS
%%
%% We strongly recommend you to compile your documents using pdflatex command.
%% It is also recommend use the texlipse plugin for Eclipse to edit your documents.
%%
%% Options for \documentclass command:
%%         * Idiom
%%           pt   - Portguese (default)
%%           en   - English
%%
%%         * Text type
%%           bsc  - B.Sc. Thesis
%%           msc  - M.Sc. Thesis (default)
%%           qual - PHD qualification (not tested yet)
%%           prop - PHD proposal (not tested yet)
%%           phd  - PHD thesis
%%
%%         * Media
%%           scr  - to eletronic version (PDF) / see the users guide
%%
%%         * Pagination
%%           oneside - unique face press
%%           twoside - two faces press
%%
%%		   * Line spacing
%%           singlespacing  - the same as using \linespread{1}
%%           onehalfspacing - the same as using \linespread{1.3}
%%           doublespacing  - the same as using \linespread{1.6}
%%
%% Reference commands. Use the following commands to make references in your
%% text:
%%          \figref  -- for Figure reference
%%          \tabref  -- for Table reference
%%          \eqnref  -- for equation reference
%%          \chapref -- for chapter reference
%%          \secref  -- for section reference
%%          \appref  -- for appendix reference
%%          \axiref  -- for axiom reference
%%          \conjref -- for conjecture reference
%%          \defref  -- for definition reference
%%          \lemref  -- for lemma reference
%%          \theoref -- for theorem reference
%%          \corref  -- for corollary reference
%%          \propref -- for proprosition reference
%%          \pgref   -- for page reference
%%
%%          Example: See \chapref{chap:introduction}. It will produce 
%%                   'See Chapter 1', in case of English language.

\documentclass[pt,twoside,onehalfspacing,bsc]{risethesis}

\usepackage[sort,square]{natbib} %package for bibliography
\bibliographystyle{ieeetr} %authordate1

\usepackage[utf8]{inputenc}
\usepackage[brazilian]{babel}
\usepackage[T1]{fontenc}

\usepackage{graphicx}

%% Change the following pdf author attribute name to your name.
\usepackage[linkcolor=blue,citecolor=blue,urlcolor=blue,colorlinks,pdfpagelabels,pdftitle={Bruno Cabral's Bachelor Thesis},pdfauthor={Bruno Cabral}]{hyperref}

\address{SALVADOR}

\universitypt{Universidade Federal da Bahia}
\universityen{Federal University of Bahia}

\departmentpt{Depertamento de Ciência da Computação}
\departmenten{Computer Science Department}

\programpt{Programa Multiinstitucional de Pós-graduação em Ciência da Computação}
\programen{Graduate in Computer Science}

\majorfieldpt{Ciência da Computação}
\majorfielden{Computer Science}

\title{Sistema de apoio à Pós graduação - UFBA}
\date{Outubro/2016}

\author{Victor de Azevedo Nunes}
\adviser{Ivan do Carmo Machado}

\begin{document}

\frontmatter
\frontpage
\presentationpage

\begin{dedicatory}
Eu dedico esta dissertação...
%I dedicate this dissertation to my family, girlfriend, friends and
%professors who gave me all necessary support to get here.
\end{dedicatory}

\acknowledgements
Meus agradecimentos...

\begin{epigraph}[]{Edward V Berard}
Walking on water and developing software from a specification are easy if both are frozen
\end{epigraph}

\resumo
% Escreva seu resumo no arquivo resumo.tex
\input{resumo}

\abstract
% Write your abstract in a file called abstract.tex
\input{abstract}

% Summary (tables of contents)
\tableofcontents

% List of figures
\listoffigures

% List of tables
\listoftables

% List of acronyms
% Acronyms manual: http://linorg.usp.br/CTAN/macros/latex/contrib/acronym/acronym.pdf
\listofacronyms
\input{acronyms}

% List of listings
%\lstlistoflistings

\mainmatter

\include{chapters/intro}
\include{chapters/referencial_teorico}

% \include{chapters/introduction/main}
% \include{chapters/background/main}
% \include{chapters/proposed_solution/main}
% \include{chapters/experiment/main}
% \include{chapters/conclusion/main}

%\bibliographystyle{natbib}
%\addcontentsline{toc}{chapter}{\bibliographytocname}
\bibliography{references}

% Appendix
\clearpage
\addappheadtotoc
\appendix
\appendixpage
% \include{appendix/experiment-instruments}

\end{document}
% %% RiSE Latex Template - version 0.5
%%
%% RiSE's latex template for thesis and dissertations
%% http://risetemplate.sourceforge.net
%%
%% (c) 2012 Yguaratã Cerqueira Cavalcanti (yguarata@gmail.com)
%%          Vinicius Cardoso Garcia (vinicius.garcia@gmail.com)
%%
%% This document was initially based on UFPEThesis template, from Paulo Gustavo
%% S. Fonseca.
%%
%% ACKNOWLEDGEMENTS
%%
%% We would like to thanks the RiSE's researchers community, the 
%% students from Federal University of Pernambuco, and other users that have
%% been contributing to this projects with comments and patches.
%%
%% GENERAL INSTRUCTIONS
%%
%% We strongly recommend you to compile your documents using pdflatex command.
%% It is also recommend use the texlipse plugin for Eclipse to edit your documents.
%%
%% Options for \documentclass command:
%%         * Idiom
%%           pt   - Portguese (default)
%%           en   - English
%%
%%         * Text type
%%           bsc  - B.Sc. Thesis
%%           msc  - M.Sc. Thesis (default)
%%           qual - PHD qualification (not tested yet)
%%           prop - PHD proposal (not tested yet)
%%           phd  - PHD thesis
%%
%%         * Media
%%           scr  - to eletronic version (PDF) / see the users guide
%%
%%         * Pagination
%%           oneside - unique face press
%%           twoside - two faces press
%%
%%		   * Line spacing
%%           singlespacing  - the same as using \linespread{1}
%%           onehalfspacing - the same as using \linespread{1.3}
%%           doublespacing  - the same as using \linespread{1.6}
%%
%% Reference commands. Use the following commands to make references in your
%% text:
%%          \figref  -- for Figure reference
%%          \tabref  -- for Table reference
%%          \eqnref  -- for equation reference
%%          \chapref -- for chapter reference
%%          \secref  -- for section reference
%%          \appref  -- for appendix reference
%%          \axiref  -- for axiom reference
%%          \conjref -- for conjecture reference
%%          \defref  -- for definition reference
%%          \lemref  -- for lemma reference
%%          \theoref -- for theorem reference
%%          \corref  -- for corollary reference
%%          \propref -- for proprosition reference
%%          \pgref   -- for page reference
%%
%%          Example: See \chapref{chap:introduction}. It will produce 
%%                   'See Chapter 1', in case of English language.

\documentclass[pt,twoside,onehalfspacing,bsc]{risethesis}

\usepackage[sort,square]{natbib} %package for bibliography
\bibliographystyle{ieeetr} %authordate1

\usepackage[utf8]{inputenc}
\usepackage[brazilian]{babel}
\usepackage[T1]{fontenc}

\usepackage{graphicx}

%% Change the following pdf author attribute name to your name.
\usepackage[linkcolor=blue,citecolor=blue,urlcolor=blue,colorlinks,pdfpagelabels,pdftitle={Bruno Cabral's Bachelor Thesis},pdfauthor={Bruno Cabral}]{hyperref}

\address{SALVADOR}

\universitypt{Universidade Federal da Bahia}
\universityen{Federal University of Bahia}

\departmentpt{Depertamento de Ciência da Computação}
\departmenten{Computer Science Department}

\programpt{Programa Multiinstitucional de Pós-graduação em Ciência da Computação}
\programen{Graduate in Computer Science}

\majorfieldpt{Ciência da Computação}
\majorfielden{Computer Science}

\title{Sistema de apoio à Pós graduação - UFBA}
\date{Outubro/2016}

\author{Victor de Azevedo Nunes}
\adviser{Ivan do Carmo Machado}

\begin{document}

\frontmatter
\frontpage
\presentationpage

\begin{dedicatory}
Eu dedico esta dissertação...
%I dedicate this dissertation to my family, girlfriend, friends and
%professors who gave me all necessary support to get here.
\end{dedicatory}

\acknowledgements
Meus agradecimentos...

\begin{epigraph}[]{Edward V Berard}
Walking on water and developing software from a specification are easy if both are frozen
\end{epigraph}

\resumo
% Escreva seu resumo no arquivo resumo.tex
\input{resumo}

\abstract
% Write your abstract in a file called abstract.tex
\input{abstract}

% Summary (tables of contents)
\tableofcontents

% List of figures
\listoffigures

% List of tables
\listoftables

% List of acronyms
% Acronyms manual: http://linorg.usp.br/CTAN/macros/latex/contrib/acronym/acronym.pdf
\listofacronyms
\input{acronyms}

% List of listings
%\lstlistoflistings

\mainmatter

\include{chapters/intro}
\include{chapters/referencial_teorico}

% \include{chapters/introduction/main}
% \include{chapters/background/main}
% \include{chapters/proposed_solution/main}
% \include{chapters/experiment/main}
% \include{chapters/conclusion/main}

%\bibliographystyle{natbib}
%\addcontentsline{toc}{chapter}{\bibliographytocname}
\bibliography{references}

% Appendix
\clearpage
\addappheadtotoc
\appendix
\appendixpage
% \include{appendix/experiment-instruments}

\end{document}
% %% RiSE Latex Template - version 0.5
%%
%% RiSE's latex template for thesis and dissertations
%% http://risetemplate.sourceforge.net
%%
%% (c) 2012 Yguaratã Cerqueira Cavalcanti (yguarata@gmail.com)
%%          Vinicius Cardoso Garcia (vinicius.garcia@gmail.com)
%%
%% This document was initially based on UFPEThesis template, from Paulo Gustavo
%% S. Fonseca.
%%
%% ACKNOWLEDGEMENTS
%%
%% We would like to thanks the RiSE's researchers community, the 
%% students from Federal University of Pernambuco, and other users that have
%% been contributing to this projects with comments and patches.
%%
%% GENERAL INSTRUCTIONS
%%
%% We strongly recommend you to compile your documents using pdflatex command.
%% It is also recommend use the texlipse plugin for Eclipse to edit your documents.
%%
%% Options for \documentclass command:
%%         * Idiom
%%           pt   - Portguese (default)
%%           en   - English
%%
%%         * Text type
%%           bsc  - B.Sc. Thesis
%%           msc  - M.Sc. Thesis (default)
%%           qual - PHD qualification (not tested yet)
%%           prop - PHD proposal (not tested yet)
%%           phd  - PHD thesis
%%
%%         * Media
%%           scr  - to eletronic version (PDF) / see the users guide
%%
%%         * Pagination
%%           oneside - unique face press
%%           twoside - two faces press
%%
%%		   * Line spacing
%%           singlespacing  - the same as using \linespread{1}
%%           onehalfspacing - the same as using \linespread{1.3}
%%           doublespacing  - the same as using \linespread{1.6}
%%
%% Reference commands. Use the following commands to make references in your
%% text:
%%          \figref  -- for Figure reference
%%          \tabref  -- for Table reference
%%          \eqnref  -- for equation reference
%%          \chapref -- for chapter reference
%%          \secref  -- for section reference
%%          \appref  -- for appendix reference
%%          \axiref  -- for axiom reference
%%          \conjref -- for conjecture reference
%%          \defref  -- for definition reference
%%          \lemref  -- for lemma reference
%%          \theoref -- for theorem reference
%%          \corref  -- for corollary reference
%%          \propref -- for proprosition reference
%%          \pgref   -- for page reference
%%
%%          Example: See \chapref{chap:introduction}. It will produce 
%%                   'See Chapter 1', in case of English language.

\documentclass[pt,twoside,onehalfspacing,bsc]{risethesis}

\usepackage[sort,square]{natbib} %package for bibliography
\bibliographystyle{ieeetr} %authordate1

\usepackage[utf8]{inputenc}
\usepackage[brazilian]{babel}
\usepackage[T1]{fontenc}

\usepackage{graphicx}

%% Change the following pdf author attribute name to your name.
\usepackage[linkcolor=blue,citecolor=blue,urlcolor=blue,colorlinks,pdfpagelabels,pdftitle={Bruno Cabral's Bachelor Thesis},pdfauthor={Bruno Cabral}]{hyperref}

\address{SALVADOR}

\universitypt{Universidade Federal da Bahia}
\universityen{Federal University of Bahia}

\departmentpt{Depertamento de Ciência da Computação}
\departmenten{Computer Science Department}

\programpt{Programa Multiinstitucional de Pós-graduação em Ciência da Computação}
\programen{Graduate in Computer Science}

\majorfieldpt{Ciência da Computação}
\majorfielden{Computer Science}

\title{Sistema de apoio à Pós graduação - UFBA}
\date{Outubro/2016}

\author{Victor de Azevedo Nunes}
\adviser{Ivan do Carmo Machado}

\begin{document}

\frontmatter
\frontpage
\presentationpage

\begin{dedicatory}
Eu dedico esta dissertação...
%I dedicate this dissertation to my family, girlfriend, friends and
%professors who gave me all necessary support to get here.
\end{dedicatory}

\acknowledgements
Meus agradecimentos...

\begin{epigraph}[]{Edward V Berard}
Walking on water and developing software from a specification are easy if both are frozen
\end{epigraph}

\resumo
% Escreva seu resumo no arquivo resumo.tex
\input{resumo}

\abstract
% Write your abstract in a file called abstract.tex
\input{abstract}

% Summary (tables of contents)
\tableofcontents

% List of figures
\listoffigures

% List of tables
\listoftables

% List of acronyms
% Acronyms manual: http://linorg.usp.br/CTAN/macros/latex/contrib/acronym/acronym.pdf
\listofacronyms
\input{acronyms}

% List of listings
%\lstlistoflistings

\mainmatter

\include{chapters/intro}
\include{chapters/referencial_teorico}

% \include{chapters/introduction/main}
% \include{chapters/background/main}
% \include{chapters/proposed_solution/main}
% \include{chapters/experiment/main}
% \include{chapters/conclusion/main}

%\bibliographystyle{natbib}
%\addcontentsline{toc}{chapter}{\bibliographytocname}
\bibliography{references}

% Appendix
\clearpage
\addappheadtotoc
\appendix
\appendixpage
% \include{appendix/experiment-instruments}

\end{document}
% %% RiSE Latex Template - version 0.5
%%
%% RiSE's latex template for thesis and dissertations
%% http://risetemplate.sourceforge.net
%%
%% (c) 2012 Yguaratã Cerqueira Cavalcanti (yguarata@gmail.com)
%%          Vinicius Cardoso Garcia (vinicius.garcia@gmail.com)
%%
%% This document was initially based on UFPEThesis template, from Paulo Gustavo
%% S. Fonseca.
%%
%% ACKNOWLEDGEMENTS
%%
%% We would like to thanks the RiSE's researchers community, the 
%% students from Federal University of Pernambuco, and other users that have
%% been contributing to this projects with comments and patches.
%%
%% GENERAL INSTRUCTIONS
%%
%% We strongly recommend you to compile your documents using pdflatex command.
%% It is also recommend use the texlipse plugin for Eclipse to edit your documents.
%%
%% Options for \documentclass command:
%%         * Idiom
%%           pt   - Portguese (default)
%%           en   - English
%%
%%         * Text type
%%           bsc  - B.Sc. Thesis
%%           msc  - M.Sc. Thesis (default)
%%           qual - PHD qualification (not tested yet)
%%           prop - PHD proposal (not tested yet)
%%           phd  - PHD thesis
%%
%%         * Media
%%           scr  - to eletronic version (PDF) / see the users guide
%%
%%         * Pagination
%%           oneside - unique face press
%%           twoside - two faces press
%%
%%		   * Line spacing
%%           singlespacing  - the same as using \linespread{1}
%%           onehalfspacing - the same as using \linespread{1.3}
%%           doublespacing  - the same as using \linespread{1.6}
%%
%% Reference commands. Use the following commands to make references in your
%% text:
%%          \figref  -- for Figure reference
%%          \tabref  -- for Table reference
%%          \eqnref  -- for equation reference
%%          \chapref -- for chapter reference
%%          \secref  -- for section reference
%%          \appref  -- for appendix reference
%%          \axiref  -- for axiom reference
%%          \conjref -- for conjecture reference
%%          \defref  -- for definition reference
%%          \lemref  -- for lemma reference
%%          \theoref -- for theorem reference
%%          \corref  -- for corollary reference
%%          \propref -- for proprosition reference
%%          \pgref   -- for page reference
%%
%%          Example: See \chapref{chap:introduction}. It will produce 
%%                   'See Chapter 1', in case of English language.

\documentclass[pt,twoside,onehalfspacing,bsc]{risethesis}

\usepackage[sort,square]{natbib} %package for bibliography
\bibliographystyle{ieeetr} %authordate1

\usepackage[utf8]{inputenc}
\usepackage[brazilian]{babel}
\usepackage[T1]{fontenc}

\usepackage{graphicx}

%% Change the following pdf author attribute name to your name.
\usepackage[linkcolor=blue,citecolor=blue,urlcolor=blue,colorlinks,pdfpagelabels,pdftitle={Bruno Cabral's Bachelor Thesis},pdfauthor={Bruno Cabral}]{hyperref}

\address{SALVADOR}

\universitypt{Universidade Federal da Bahia}
\universityen{Federal University of Bahia}

\departmentpt{Depertamento de Ciência da Computação}
\departmenten{Computer Science Department}

\programpt{Programa Multiinstitucional de Pós-graduação em Ciência da Computação}
\programen{Graduate in Computer Science}

\majorfieldpt{Ciência da Computação}
\majorfielden{Computer Science}

\title{Sistema de apoio à Pós graduação - UFBA}
\date{Outubro/2016}

\author{Victor de Azevedo Nunes}
\adviser{Ivan do Carmo Machado}

\begin{document}

\frontmatter
\frontpage
\presentationpage

\begin{dedicatory}
Eu dedico esta dissertação...
%I dedicate this dissertation to my family, girlfriend, friends and
%professors who gave me all necessary support to get here.
\end{dedicatory}

\acknowledgements
Meus agradecimentos...

\begin{epigraph}[]{Edward V Berard}
Walking on water and developing software from a specification are easy if both are frozen
\end{epigraph}

\resumo
% Escreva seu resumo no arquivo resumo.tex
\input{resumo}

\abstract
% Write your abstract in a file called abstract.tex
\input{abstract}

% Summary (tables of contents)
\tableofcontents

% List of figures
\listoffigures

% List of tables
\listoftables

% List of acronyms
% Acronyms manual: http://linorg.usp.br/CTAN/macros/latex/contrib/acronym/acronym.pdf
\listofacronyms
\input{acronyms}

% List of listings
%\lstlistoflistings

\mainmatter

\include{chapters/intro}
\include{chapters/referencial_teorico}

% \include{chapters/introduction/main}
% \include{chapters/background/main}
% \include{chapters/proposed_solution/main}
% \include{chapters/experiment/main}
% \include{chapters/conclusion/main}

%\bibliographystyle{natbib}
%\addcontentsline{toc}{chapter}{\bibliographytocname}
\bibliography{references}

% Appendix
\clearpage
\addappheadtotoc
\appendix
\appendixpage
% \include{appendix/experiment-instruments}

\end{document}
% %% RiSE Latex Template - version 0.5
%%
%% RiSE's latex template for thesis and dissertations
%% http://risetemplate.sourceforge.net
%%
%% (c) 2012 Yguaratã Cerqueira Cavalcanti (yguarata@gmail.com)
%%          Vinicius Cardoso Garcia (vinicius.garcia@gmail.com)
%%
%% This document was initially based on UFPEThesis template, from Paulo Gustavo
%% S. Fonseca.
%%
%% ACKNOWLEDGEMENTS
%%
%% We would like to thanks the RiSE's researchers community, the 
%% students from Federal University of Pernambuco, and other users that have
%% been contributing to this projects with comments and patches.
%%
%% GENERAL INSTRUCTIONS
%%
%% We strongly recommend you to compile your documents using pdflatex command.
%% It is also recommend use the texlipse plugin for Eclipse to edit your documents.
%%
%% Options for \documentclass command:
%%         * Idiom
%%           pt   - Portguese (default)
%%           en   - English
%%
%%         * Text type
%%           bsc  - B.Sc. Thesis
%%           msc  - M.Sc. Thesis (default)
%%           qual - PHD qualification (not tested yet)
%%           prop - PHD proposal (not tested yet)
%%           phd  - PHD thesis
%%
%%         * Media
%%           scr  - to eletronic version (PDF) / see the users guide
%%
%%         * Pagination
%%           oneside - unique face press
%%           twoside - two faces press
%%
%%		   * Line spacing
%%           singlespacing  - the same as using \linespread{1}
%%           onehalfspacing - the same as using \linespread{1.3}
%%           doublespacing  - the same as using \linespread{1.6}
%%
%% Reference commands. Use the following commands to make references in your
%% text:
%%          \figref  -- for Figure reference
%%          \tabref  -- for Table reference
%%          \eqnref  -- for equation reference
%%          \chapref -- for chapter reference
%%          \secref  -- for section reference
%%          \appref  -- for appendix reference
%%          \axiref  -- for axiom reference
%%          \conjref -- for conjecture reference
%%          \defref  -- for definition reference
%%          \lemref  -- for lemma reference
%%          \theoref -- for theorem reference
%%          \corref  -- for corollary reference
%%          \propref -- for proprosition reference
%%          \pgref   -- for page reference
%%
%%          Example: See \chapref{chap:introduction}. It will produce 
%%                   'See Chapter 1', in case of English language.

\documentclass[pt,twoside,onehalfspacing,bsc]{risethesis}

\usepackage[sort,square]{natbib} %package for bibliography
\bibliographystyle{ieeetr} %authordate1

\usepackage[utf8]{inputenc}
\usepackage[brazilian]{babel}
\usepackage[T1]{fontenc}

\usepackage{graphicx}

%% Change the following pdf author attribute name to your name.
\usepackage[linkcolor=blue,citecolor=blue,urlcolor=blue,colorlinks,pdfpagelabels,pdftitle={Bruno Cabral's Bachelor Thesis},pdfauthor={Bruno Cabral}]{hyperref}

\address{SALVADOR}

\universitypt{Universidade Federal da Bahia}
\universityen{Federal University of Bahia}

\departmentpt{Depertamento de Ciência da Computação}
\departmenten{Computer Science Department}

\programpt{Programa Multiinstitucional de Pós-graduação em Ciência da Computação}
\programen{Graduate in Computer Science}

\majorfieldpt{Ciência da Computação}
\majorfielden{Computer Science}

\title{Sistema de apoio à Pós graduação - UFBA}
\date{Outubro/2016}

\author{Victor de Azevedo Nunes}
\adviser{Ivan do Carmo Machado}

\begin{document}

\frontmatter
\frontpage
\presentationpage

\begin{dedicatory}
Eu dedico esta dissertação...
%I dedicate this dissertation to my family, girlfriend, friends and
%professors who gave me all necessary support to get here.
\end{dedicatory}

\acknowledgements
Meus agradecimentos...

\begin{epigraph}[]{Edward V Berard}
Walking on water and developing software from a specification are easy if both are frozen
\end{epigraph}

\resumo
% Escreva seu resumo no arquivo resumo.tex
\input{resumo}

\abstract
% Write your abstract in a file called abstract.tex
\input{abstract}

% Summary (tables of contents)
\tableofcontents

% List of figures
\listoffigures

% List of tables
\listoftables

% List of acronyms
% Acronyms manual: http://linorg.usp.br/CTAN/macros/latex/contrib/acronym/acronym.pdf
\listofacronyms
\input{acronyms}

% List of listings
%\lstlistoflistings

\mainmatter

\include{chapters/intro}
\include{chapters/referencial_teorico}

% \include{chapters/introduction/main}
% \include{chapters/background/main}
% \include{chapters/proposed_solution/main}
% \include{chapters/experiment/main}
% \include{chapters/conclusion/main}

%\bibliographystyle{natbib}
%\addcontentsline{toc}{chapter}{\bibliographytocname}
\bibliography{references}

% Appendix
\clearpage
\addappheadtotoc
\appendix
\appendixpage
% \include{appendix/experiment-instruments}

\end{document}

%\bibliographystyle{natbib}
%\addcontentsline{toc}{chapter}{\bibliographytocname}
\bibliography{references}

% Appendix
\clearpage
\addappheadtotoc
\appendix
\appendixpage
% \include{appendix/experiment-instruments}

\end{document}
% %% RiSE Latex Template - version 0.5
%%
%% RiSE's latex template for thesis and dissertations
%% http://risetemplate.sourceforge.net
%%
%% (c) 2012 Yguaratã Cerqueira Cavalcanti (yguarata@gmail.com)
%%          Vinicius Cardoso Garcia (vinicius.garcia@gmail.com)
%%
%% This document was initially based on UFPEThesis template, from Paulo Gustavo
%% S. Fonseca.
%%
%% ACKNOWLEDGEMENTS
%%
%% We would like to thanks the RiSE's researchers community, the 
%% students from Federal University of Pernambuco, and other users that have
%% been contributing to this projects with comments and patches.
%%
%% GENERAL INSTRUCTIONS
%%
%% We strongly recommend you to compile your documents using pdflatex command.
%% It is also recommend use the texlipse plugin for Eclipse to edit your documents.
%%
%% Options for \documentclass command:
%%         * Idiom
%%           pt   - Portguese (default)
%%           en   - English
%%
%%         * Text type
%%           bsc  - B.Sc. Thesis
%%           msc  - M.Sc. Thesis (default)
%%           qual - PHD qualification (not tested yet)
%%           prop - PHD proposal (not tested yet)
%%           phd  - PHD thesis
%%
%%         * Media
%%           scr  - to eletronic version (PDF) / see the users guide
%%
%%         * Pagination
%%           oneside - unique face press
%%           twoside - two faces press
%%
%%		   * Line spacing
%%           singlespacing  - the same as using \linespread{1}
%%           onehalfspacing - the same as using \linespread{1.3}
%%           doublespacing  - the same as using \linespread{1.6}
%%
%% Reference commands. Use the following commands to make references in your
%% text:
%%          \figref  -- for Figure reference
%%          \tabref  -- for Table reference
%%          \eqnref  -- for equation reference
%%          \chapref -- for chapter reference
%%          \secref  -- for section reference
%%          \appref  -- for appendix reference
%%          \axiref  -- for axiom reference
%%          \conjref -- for conjecture reference
%%          \defref  -- for definition reference
%%          \lemref  -- for lemma reference
%%          \theoref -- for theorem reference
%%          \corref  -- for corollary reference
%%          \propref -- for proprosition reference
%%          \pgref   -- for page reference
%%
%%          Example: See \chapref{chap:introduction}. It will produce 
%%                   'See Chapter 1', in case of English language.

\documentclass[pt,twoside,onehalfspacing,bsc]{risethesis}

\usepackage[sort,square]{natbib} %package for bibliography
\bibliographystyle{ieeetr} %authordate1

\usepackage[utf8]{inputenc}
\usepackage[brazilian]{babel}
\usepackage[T1]{fontenc}

\usepackage{graphicx}

%% Change the following pdf author attribute name to your name.
\usepackage[linkcolor=blue,citecolor=blue,urlcolor=blue,colorlinks,pdfpagelabels,pdftitle={Bruno Cabral's Bachelor Thesis},pdfauthor={Bruno Cabral}]{hyperref}

\address{SALVADOR}

\universitypt{Universidade Federal da Bahia}
\universityen{Federal University of Bahia}

\departmentpt{Depertamento de Ciência da Computação}
\departmenten{Computer Science Department}

\programpt{Programa Multiinstitucional de Pós-graduação em Ciência da Computação}
\programen{Graduate in Computer Science}

\majorfieldpt{Ciência da Computação}
\majorfielden{Computer Science}

\title{Sistema de apoio à Pós graduação - UFBA}
\date{Outubro/2016}

\author{Victor de Azevedo Nunes}
\adviser{Ivan do Carmo Machado}

\begin{document}

\frontmatter
\frontpage
\presentationpage

\begin{dedicatory}
Eu dedico esta dissertação...
%I dedicate this dissertation to my family, girlfriend, friends and
%professors who gave me all necessary support to get here.
\end{dedicatory}

\acknowledgements
Meus agradecimentos...

\begin{epigraph}[]{Edward V Berard}
Walking on water and developing software from a specification are easy if both are frozen
\end{epigraph}

\resumo
% Escreva seu resumo no arquivo resumo.tex
Este trabalho de conclusão de curso utiliza elementos da engenharia de software para propor um software como serviço a fim de otimizar o processo de construção e manutenção dos softwares. Assim, o objetivo deste SaaS é gerenciar as permissões de acesso de sistemas cliente, provendo o reuso de software e padronizando as soluções. O sistema proposto neste trabalho encontra-se implementado e disponível no Github, e traz fundamentos sobre a arquitetura e tecnologias utilizadas, bem como avaliações sobre possibilidades de utilização do mesmo em diversos ambientes, como Web e mobile.

\begin{keywords}
Software; Reuso; SaaS; Web; PHP
\end{keywords}

\abstract
% Write your abstract in a file called abstract.tex
My abstract...

\begin{keywords}
key words...
\end{keywords}

% Summary (tables of contents)
\tableofcontents

% List of figures
\listoffigures

% List of tables
\listoftables

% List of acronyms
% Acronyms manual: http://linorg.usp.br/CTAN/macros/latex/contrib/acronym/acronym.pdf
\listofacronyms
\begin{acronym}[ACRONYM] 
% Change the word ACRONYM above to change the acronym column width.
% The column width is equals to the width of the word that you put.
% Read the manual about acronym package for more examples:
%   http://linorg.usp.br/CTAN/macros/latex/contrib/acronym/acronym.pdf

\acro{SPA}{Single Page Application}
\acro{JSON}{Javascript Object Notation}
\acro{PHP}{PHP: Hypertext Preprocessor}
\acro{SaaS}{Software as a Service}
\acro{ERP}{Enterprise Resource Planning}
\acro{QoS}{Quality of Service}
\acro{UML}{Unified Modeling Language}
\acro{MVC}{Model-View-Controller}
\acro{Ajax}{Asynchronous Javascript and XML}
\acro{HTML}{HyperText Markup Language}
\acro{CSS}{Cascading Style Sheets}
\acro{API}{Application Programming Interface}
\acro{DOM}{Document Object Model}
\acro{BPMN}{Business Process Model and Notation}
\acro{REST}{Representational State Transfer}

\end{acronym}

% List of listings
%\lstlistoflistings

\mainmatter

\chapter{Introdução}

\section{Motivação}

Organizar os procedimentos de um processo sempre nos traz vantagens. Apesar de no processo de implantação de um sistema, o mesmo burocratizar o processo, com o tempo temos o retorno da dedicação para a inserção dos dados. Com um certo volume de dados, é possível estruturar informações que num processo manual são difíceis de serem enxergadas. Assim, é possível depender menos das pessoas que organizam o processo, pois o legado de informações não estará mais somente na mente de alguns, mas sim documentado nos dados do sistema.

Além de colaborar na organização, também haverá uma grande colaboração no tempo gasto na gestão. Lidar com muitos papéis e confiar na mente humana para guardar informações, não é uma alternativa muito segura devido ao fato que as pessoas sempre estão sujeitas a sair do processo e levar contigo a experiência obtida. Experiência essa que faz com que os procedimentos sejam executados de forma mais eficiente. Entretanto, com um sistema inteligente, é possível auxiliar e tornar mais ágil a execução das tarefas.


\section{Problema}


De acordo com funcionários ligados ao o setor de pós graduação da UFBA, entrevistados a fim de um maior entendimento do cenário, apesar das semelhanças estruturais, a pós graduação gerida de forma diferencia da graduação. FULANO afirma que devido ao fato de não ter a mesma visibilidade, não tem acesso aos mesmos recursos de gestão acadêmica da graduação. O professores não executam somente atividades dentro da sala de aula, também tem diversas outras ocupações no setor. E muitos procedimentos realizados extra classe ainda se encontram sendo realizados de forma manual, estando mais vulnerável ao erro ou até mesmo à violação do processo. Também ocorre um grande desperdício de tempo pelos professores e gestores da área, devido ao diversos processos ainda realizados de forma manual, sem a devida documentação. Segundo FULANO, também entrevistado, esse tempo perdido implica numa redução da eficiência na sala de aula, pois o professor acaba por ter menos tempo disponível para o planejamento das atividades, o que gera impactos negativos aos alunos.


\section{Objetivos} %<o que deve ser feito/entregue>


Devido aos muitos processos sendo resolvidos de forma manual, propõe-se com solução um sistema moderno, arquitetado para ter funcionamento na web e com um módulo mobile, a fim de fornecer informações de forma rápida e eficiente para os professores através de notificações, já que o acesso à internet móvel é comum entre os possíveis usuários do sistema em questão.
O principal requisito para o sistema seria dispor recursos para reduzir o tempo desperdiçado pelos professores durante as atividades extra classe.


\section{Metodologia} %<como será feito | como resolver o problema apontado inicialmente>


%<analise de literatura | design | implementação | validação>
Baseando-se nas tecnologias gratuitas em alta no cenário atual do desenvolvimento web, dispomos de algumas opções eficientes para a implementação da solução. Dentre as possibilidades, considerando a facilidade para futura manutenção e continuidade do projeto, tende-se a optar por uma tecnologia popular. Como linguagem de programação, adota-se o PHP. A escolha é fundamentada de acordo com a pesquisa da RedMonk de 2015, que evidencia o uso das linguagens de programação de acordo com as discussões no StackOverflow e repositórios no GitHub. É possível constatar a popularidade do PHP no cenário atual com o gráfico da pesquisa citada, na qual o PHP é apresentado na terceira colocação, apenas atrás do lider JavaScript e do segundo colocado, o Java.

\begin{figure}
	\label{fig:graficoRedmonk}
	\includegraphics[width=1\textwidth]{img/grafico_redmonk}
	\caption{Ranking das liguagens de programação no Stack Overflow e Github}
\end{figure}


Ainda assim, para compor a interface do dado projeto, também ocorrerá o uso do líder JavaScript de forma intensa, provendo o elo com o as informações gerenciadas pelo PHP.


Entretanto, não seria inteligente desenvolver um sistema completo sem o auxílio de um framework. Dentre os frameworks disponíveis para PHP, hoje o destaque está com o Laravel, que se encontra no topo dentre os mais utilizados no momento. 


A WebHostFace, uma empresa de hospedagem, compilou várias estatísticas para criar um infográfico mostrando os frameworks PHP mais populares de 2015. Utilizando informações sobre os próprios clientes, o Google Trends, estatísticas de repositórios do GitHub e a pesquisa do SitePoint “Best PHP Frameworks 2015”, a WebHostFace elaborou o seguinte infográfico: 

\begin{figure}
	\label{fig:graficoWebhostface}
	\includegraphics[width=1\textwidth]{img/infografico_webhostface}
	\caption{Infográfico da WebhostFace, exibindo a popularidade dos Frameworks PHP em 2015}
\end{figure}

Assim, tem-se a evidência que o Laravel em 2015 teve a maior popularidade em projetos pessoais e tem a maior comunidade entre os concorrentes, o que o torna uma boa escolha para a escrita de um software que será continuado por terceiros.


Para elaborar os recursos de interface e integrar ao back-end PHP do sistema, será adotado o já conhecido AngularJS, ferramenta sólida e conhecida no aspecto em questão. 


Dados coletados via Google Trends, que propõe comparações entre termos pesquisados, revela a popularidade do AngularJs diante de alguns dos principais concorrentes. O gráfico abaixo evidencia o cenário.


%Como mostra a Figura \ref{fig:graficoGoogleTrendsFerramentasFront}. 
\begin{figure}
	\label{fig:graficoGoogleTrendsFerramentasFront}
	\includegraphics[width=1\textwidth]{img/grafico_ferramentas_front}
	\caption{Gráfico do Google Trends exibindo as pesquisas por ferramentas front-end}
\end{figure}


Junto ao Angular JS, será utilizada a agradável tendência de interface do Material Design da Google, que propõe layouts limpos e otimizados já conhecidos pelos usuários de smartphones Android. 


Para a elaboração da plataforma mobile do projeto, será utilizado o Ionic Framework, muito difundido e bastante pesquisado na área, o que fica evidenciado com o gráfico de pesquisbaixo, coletado via Google Trends buscando por frameworks de desenvolvimento híbrido mobile.


\begin{figure}
	\label{fig:graficoGoogleTrendsFerramentasHibridasMobile}
	\includegraphics[width=1\textwidth]{img/grafico_ferramentas_hibridas_mobile}
	\caption{Gráfico do Google Trends exibindo as pesquisas por Frameworks híbridos mobile}
\end{figure}	

Para layout da interface mobile, também será aplicado a tendência do Material Design, a fim de propor uma harmonia entre o módulo web e mobile para os usuários


\section{Resultados Esperados}


Como fruto de um sistema para pós-graduação da UFBA, espera-se que os professores tenham mais recursos para integrar as atividades e também prover melhores condições para acompanhamento da vida acadêmica dos alunos em questão. Também, que os novos colaboradores que entrarem no processo tenham facilidade de compreender o fluxo do setor ao navegar pelo sistema proposto.


\section{Fora de Escopo}


Interação com os alunos devido às complicações para realizar a integração com o sistema empregado na UFBA, gerenciado pela XXXXXX, o que causaria uma inviabilidade no projeto devido à necessidade de entrega do produto ser mais forte que o tempo necessário para executar o processo de obtenção de acesso ao sistema legado para realizar a integração.


\section{Estrutura do Trabalho}


<breve resumo sobre os capítulos do TCC>
\chapter{Referencial Teórico}


Projetar o desenvolvimento de um software requer muito planejamento, pois as falhas iniciais podem custar bastante caro ou até mesmo inviabilizar a continuação de um projeto. Assim, a escolha da arquitetura ideal para a aplicabilidade é essencial na concepção de um produto de software. 
De todo o modo, sempre busca-se fazer mais com menos. Diante de tal filosofia, temos neste capítulo, uma breve discussão sobre alguns elementos de projeto e arquitetura de software, a fim de contextualizar este trabalho de conclusão de curso.
O capítulo corrente é composto por quatro seções. A \ref{sec:saas} trata de Software como serviço, discutindo alguns elementos do contexto que são relevantes para o trabalho proposto. A \ref{sec:reuso}  discute sobre a empregabilidade do reuso de software. A \ref{sec:modularizacao} seção, trata sobre aspectos envolvidos na modularização dos softwares. Por fim, a \ref{sec:apps_web} aborda as aplicações web, discutindo sobre aspectos relevantes sobre a aplicação web que compõe este trabalho.


 \section{Software como serviço}\label{sec:saas}


Segundo La e Chun \citep{La2009Systematic}, o princípio da definição de Software como um Serviço (Sofware as a Service - SaaS) é um serviço complementar para aplicações da computação em nuvem (cloud computing). No entanto, as áreas não se confundem. SaaS deve ser entendido como um mecanismo de suporte às soluções existentes na cloud. Os SaaS existem justamente para maximizar o reuso de serviços repetidos e não centrais em uma aplicação remota.


Como vantagens, diversos fatores podem ser favoráveis para a adoção de um SaaS, como custo e manutenção dentre outros fatores aplicáveis a soluções específicas. Lechesaet al. \citep{LechesaSS11}, quantificam índices sobre os fatores determinantes para adoção ou não de um SaaS voltado para ERP na África do Sul. Os principais fatores determinantes para adoção desse mecanismo de software são sua fluidez quanto à rede e a segurança. Esses fatores estão presentes na aplicação desenvolvida neste trabalho de conclusão de curso.


Devido ao fato de ter um serviço constantemente na nuvem, fica o questionamento sobre a segurança da informação manipulada. Sabe-se que a vulnerabilidade na Web não é restrita ao SaaS, atingindo diversos âmbitos. Rai et al. \cite{journals/corr/RaiSM13} consideram que o avanço da computação em nuvem não é um problema apenas para os serviços Web do ponto de vista da segurança, pois muitos trabalhos na literatura mostram a área como mais um ponto de vulnerabilidade para diversos setores, a exemplo de infraestrutura. No mesmo artigo mencionado de Rai et al. \cite{journals/corr/RaiSM13}, também realizaram-se estudos exploratórios junto a empresas usuárias de serviços em computação em nuvem e consideram que a perspectiva de SaaS também pode fortalecer a segurança nas aplicações de cloud computing, pois o software de autenticação compartilhado por várias aplicações em nuvem, oferece uma melhor padronização e consequente facilidade de prevenção a erros de vulnerabilidade específicas de cada módulo da pesquisa. Esse ponto de vista é fundamenta para a compreensão de SaaS.


A arquitetura de armazenamento de dados de um Saas pode variar de acordo com a necessidade do contexto. Huixin \cite{7586486} descreve possíveis modelagens nesse sentido. Essa abordagem pode ser com um banco de dados único, fazendo com que diferentes clientes compartilhem o mesmo banco, diferindo os dados através de controle de usuário, ou isolando os diferentes clientes através de bancos de dados exclusivos para cada um. Esse fator também pode ser combinado com a arquitetura da aplicação, caso ofereça aplicação única para todos os clientes ou aplicação compartilhada. Diante das possíveis abordagens, a modelagem de dados do software pode ser decidida pela regra de negócio. Este trabalho optou por aplicação única e banco de dados compartilhado.

Devido ao conceito particular de obtenção de software, tanto pela visão do cliente como pela visão do vendedor, é necessário compreender os diversos aspectos que podem ser relevantes ao orçar um Saas. O recente trabalho de T. Kaur et al. \citep{6949281} orienta um modelo para compor o custo de um Saas. O custo total seria composto pelos fatores que dão suporte ao funcionamento do software. Tais fatores incluem infra-estrutura, configurabilidade, customização, parâmetros de QoS (Quality of service) como escalabilidade, disponibilidade, usabilidade, pontualidade e desempenho da resposta, portabilidade, custo total de propriedade e retorno do investimento. Esses fatores caracterizam o custo de forma eficaz, possibilitando ao fornecedor, prover um Serviço de acordo com a exigência do consumidor em vários pacotes de serviços.


O conceito de software como serviço se aplica neste trabalho de conclusão de curso, pois o mesmo estará disponível na web com alta disponibilidade adotando as características apresentadas para qualquer pessoa que desejar usá-lo. Assim, basta se cadastrar, configurá-lo e usar como um serviço, sem preocupação com a manutenção do mesmo.


\section{Reuso de software}\label{sec:reuso} %CRUISE BOOK CAPITULO 2


De acordo com o livro Software pratical reuse \citep{Ezran2002}, o reuso de software é a utilização de qualquer informação que um desenvolvedor pode necessitar no processo de criação de software. O livro de Basili e Rombach \citep{Basili1991} define reutilização de software como o uso de tudo o que está associado a um projeto de conhecimento.
Assim, o objetivo da reutilização de software é reciclar o design, código e outros componentes de um produto de software e assim reduzir o custo, o tempo e melhorar a qualidade do produto.
Segundo Keswani et al. \cite{6783445}, o componente reutilizável de software pode ser qualquer parte de seu desenvolvimento, como um fragmento de código, design, casos de teste, ou até mesmo a especificação de requisitos de uma funcionalidade do software. 

O reuso de software pode ter impacto positivo em diversos aspectos do software, vejamos alguns, conforme apresentados no C.R.U.I.S.E Book \citep{cruise2007} :

\begin{itemize}

\item Qualidade: As correções de erro tornam-se úteis em todos os locais em que ocorreu, padronizando e facilitando a manutenção.

\item Produtividade: O ganho de produtividade é alcançado devido ao menor número de artefatos desenvolvido. Isso resulta em menor esforço de teste e também análise e design, reduzindo custos.

\item Confiabilidade: A utilização de componentes bem testados aumenta a
confiança no software. Além disso, a utilização de um mesmo componente em vários sistemas, aumenta a possibilidade de detecção de erros e reforça a confiança no componente.

\item Redução do Esforço: A reutilização de software proporciona uma redução do tempo de desenvolvimento, o que reduz o tempo necessário para o produto ser disponibilizado no mercado para trazer rentabilidade.

\item Trabalho redundante e tempo de desenvolvimento: Desenvolver um sistema do
zero significa desenvolvimento redundante de muitos componentes, como requisitos, especificações, casos de uso, arquitetura, etc. Isso pode ser evitado quando estes estão disponíveis como componentes reutilizáveis e podem ser compartilhados, resultando em um processo de desenvolvimento otimizado.

\item Documentação: Embora a documentação seja muito importante para a
manutenção de um sistema, muitas vezes é negligenciada. A reutilização de componentes de software reduz a quantidade de documentação a ser escrita, entretanto depende da qualidade do que está escrito. Assim, apenas a estrutura do sistema e os novos artefatos desenvolvidos necessitam ser documentados.

\item Custo de manutenção: Menos defeitos e manutenções são esperados quando tem-se comprovada a qualidade dos componentes utilizados.

\item Tamanho da equipe: É comum haver casos em que a equipe de desenvolvimento sofre sobrecarga. Entretanto, dobrar o tamanho da equipe de desenvolvimento não necessariamente duplica produtividade. Se muitos componentes podem ser reutilizados, é possível desenvolver com equipes menores, levando a melhor comunicação e aumento da produtividade.

\end{itemize}

Apesar dos benefícios da reutilização de software, ela não é suficientemente aproveitada. Existem fatores que influenciam direta ou indiretamente na sua adoção. Esses fatores podem ser de aspecto gerencial, organizacional, econômico, conceitual ou técnico. Veremos a seguir alguns aspectos que podem gerar conflito com a cultura de reuso de software, segundo o C.R.U.I.S.E Book \citep{cruise2007} :
%(Sametinger, 1997). REVER

\begin{itemize}
	
\item Falta de apoio da gestão: Como a reutilização de software gera custos iniciais,
a medida pode não ser amplamente alcançada em uma organização sem o apoio de alto nível de gestão. Os gestores têm de ser informados sobre os custos iniciais e serem convencidos sobre economias futuras.

\item Gerenciamento do Projeto: Gerenciar projetos tradicionais é uma tarefa árdua, principalmente, os que praticam a reutilização de software. Utilizando a técnica em larga escala, tem-se impacto sobre todo o ciclo de vida do software.

\item Estruturas organizacionais inadequadas: As estruturas organizacionais devem
considerar diferentes necessidades que surgem quando a reutilização em larga escala está sendo adotada. Por exemplo, uma equipe particionada pode ser alocada somente para desenvolver, manter e certificar componentes reutilizáveis de software.

\item Incentivos de gestão: É comum a falta de incentivo para deixar os desenvolvedores gastarem tempo elaborando componentes do sistemas. A produtividade é muitas vezes medida apenas no tempo necessário para concluir um projeto. Assim, fazer qualquer trabalho além disso, embora benéfico para a empresa como um todo, diminui o seu sucesso. Mesmo quando os componentes reutilizáveis são utilizados, os benefícios obtidos são uma pequena fração do que poderia ser alcançado caso houvesse reutilização explícita, planejada e organizada.

\item Dificuldade de encontrar software reutilizável: Para reutilizar os componentes, devem existir formas eficientes de busca. Além disso, é importante ter um repositório bem organizado contendo componentes com um eficiente meio de acesso.

\item Não reutilização do software encontrado. O acesso fácil ao software existente
não necessariamente aumentar a reutilização. Os componentes reutilizáveis devem ser cuidadosamente especificados, projetados, implementados e documentados, pois em alguns casos, modificar e adaptar o código  pode ser mais custoso que a programação da funcionalidade necessária a partir do zero.

\item Modificação: É muito difícil encontrar um componente que funcione
exatamente da mesma maneira que queremos. Desta forma, são necessárias modificações e devem existir formas de determinar os seus efeitos sobre o componente.


\end{itemize}


A definição do reuso de software pode ser associada a este trabalho, pois ainda que como um serviço, estaremos adotando uma aplicação única para servir aos mais diversos softwares de uma empresa, por exemplo. Logo, ao adotar o projeto, pode ser interessante aplicar algumas métricas para tomar conhecimento de possíveis vantanjens como as citadas nessa seção, a exemplo de economia de tempo de desenvolvimento, e custo do projeto.


%Outra diretriz importante para a reutilização de software é reduzir o risco na criação de novos softwares. O risco tende a ser bastante reduzido se os componentes que estão sendo reutilizados têm as documentação, interfaces necessárias e devidamente testadas, fatores que contibruem para uma fácil integração.
%De acordo com Keswani et al. \citep{6783445}, para o reuso de software dar retornos apropriados, o processo deve ser sistemático e planejado. Qualquer organização que implemente a reutilização de software deve identificar os melhores métodos e estratégias de reutilização para obter a máxima produtividade. A reutilização de software ajuda a evitar software de engenharia a partir do zero, pois usa módulos de software existentes. A reutilização de software, embora seja uma tarefa difícil, especialmente para softwares antigos sem padrões de projeto, pode melhorar significativamente a produtividade e a qualidade de um produto de software. Embora a reutilização de software não seja um novo campo, ela pode dar grandes retornos em curto período de tempo.


\section{Modularização}\label{sec:modularizacao} %artigo de claudio pagina 222 introdução


%A modularidade vem desempenhando um papel predominante estágios emergentes das disciplinas de arquitetura de software [13]. Engenheiros de software consideram modularidade como princípio base na comparação entre arquiteturas alternativas  e arquitetura degeneração [9]. De fato, os engenheiros de software são incentivados a arquitecturas, baseando-se numa multiplicidade de mecanismos de modularidade disponíveis em: 
%(i) Linguagens de descrição de arquitetura (ADLs), como ACME [8], 
%(ii) catálogos de arquitetônicos [2, 13], e 
%(iii) conhecem bem princípios de alto nível, como interfaces de componentes estreitos, acoplamento arquitectónico reduzido e semelhantes.


Conforme é frisado por Wickramaarachchi e Lai \citep{7062705}, o conceito de modularização na indústria de software tem uma longa história e tem sido utilizado para melhorar o processo de desenvolvimento de software em diferentes estágios. Os principais conceitos por trás da modularização do software foram introduzidos por pesquisadores pioneiros há quarenta anos, com uma notável contribuição feita por Melvin Conway e David Parnas, que tem representação notável na engenharia de software.


Modularizar um software é um bom padrão a ser adotado. Segundo Wickramaarachchi e Lai \citep{7062705}, a modularização é importante na identificação de dependências e reduz as dificuldades diante de uma possível necessidade de grandes alterações. De uma perspectiva da engenharia de software, uma modularização geralmente tem várias vantagens, tais como: tornar a complexidade do software mais gerenciável, facilitar o trabalho paralelo e tornar o software mais maleável para acomodar o futuro incerto que um software pode ter. O objetivo final da modularização do software é aumentar a produtividade ea qualidade do software. Tal conceito encontra-se bastante difundido e estái incorporado em linguagens de programação e ferramentas de software. O presente trabalho favorece ao uso da modularização de um software e até mesmo pode ser considerado um módulo a ser acoplado a qualquer software, mediante a compatibilidade.


Ao realizar a adoção do trabalho proposto, fica bastante evidente a "responsabilidade" do mesmo no escopo do projeto. Assim, é possível tratá-lo como um módulo do projeto que o usa, mesmo que seja consumido como um serviço.


\section{Aplicações web}\label{sec:apps_web}


A popularidade das soluções Web aumentou exponencialmente na última década e todos os dias cresce o número de pessoas usuárias desse tipo de software. E seguindo um padrão próprio, Kumar et al. \citep{7813710} sugerem que para o desenvolvimento web, deve-se manter a prática eficaz de produzir diagramas UML. A abordagem baseada na web oferece uma maneira fácil e eficaz para gerenciar e controlar o processo de desenvolvimento por meio de artefatos de modelagem. Tal abordagem pode ser usada quando há uma exigência de lidar com mudanças muito rápidas e grandes em requisitos de forma muito eficaz em muito menos tempo, gerando assim um menor impacto. 


Para atender à fomentada demanda de aplicativos web, é necessário adotar métodos de desenvolvimentos que sejam ágeis, eficientes e de fácil manutenção. Yu Ping et al. \cite{1372143} propõem o uso do modelo MVC (Model, View e Controller) no desenvolvimento para softwares web. O modelo apresentado tornou-se um padrão popular e divide o software em camadas com propósito definido, tornando-o de mais fácil manutenção.


O Ajax (Asynchronous Javascript and XML) revolucionou a web. Conforme demonstrado por Yuping \citep{6845605}, ao usar a tecnologia Ajax, podemos enriquecer a experiência do usuário em aplicações baseadas em navegador de internet, e fornecer uma variedade de aplicações interativas para atender às necessidade de humanização das aplicações.
Os aplicativos Ajax em execução no navegador se comunicam com um servidor Web de forma assíncrona e atualizam apenas uma parte da página.


No artigo de Tesarik et al. \citep{4664349}, temos que o design de software SPA (Single page application) é uma maneira interessante de criar um software Web numa uma única página. Essa solução de página única sem navegação funciona apenas com base em técnicas dinâmicas e assíncronas, como o AJAX (citado no parágrafo anterior). No entanto, esta abordagem coloca o desenvolvedor antes de alguns desafios substanciais.
Para projetar a interface do usuário que mostra as informações, é importante projetar corretamente a tela para manipular os dados do aplicativo numa única página. O design da página deve ser elaborado para maximizar a decomposição da página em componentes distintos que encapsulam os principais casos de uso.
Também recomenda-se que para elaborar os artefatos visuais, explore os mais ricos recursos na implementação, como HTML5, JavaScript, Ajax, CSS3, e outras tecnologias que se apliquem. Entretanto, desenvolver uma rica interface com o uso de diversas tecnologias/frameworks pode ocasionar um esforço maior para explorar as possibilidades.
Tais componentes da interface de uma aplicação SPA normalmente são alimentados mediante o consumo de uma API Rest via requisições AJAX. Assim, é possível particionar as responsabilidades de processamento do software entre cliente e servidor.


Como é possível denotar no trabalho de Salvadori e Siqueire \citep{7195633}, REST é uma arquitetura muito popular para integração de aplicativos web, que permite compartilhar e reutilizar informações através de sistemas. Aplicações de grande escala baseadas em no REST deve ser implementadas utilizando estratégias e mecanismos para produzir sistemas que sejam fáceis de desenvolver, reutilizar e manter. As interfaces de integração fornecidas por esses sistemas, chamadas de Web API, têm uma influência importante nas características da implementação resultante, pois a sua forma de resposta deve ser casada com a interface que o consome.
 
 
Os elementos comentados nessa seção encontram-se presentes neste trabalho de conclusão de curso. Juntos, montam a estrutura tecnológica necessária aliada à arquitetura adotada. Esses elementos seguem tendências atuais dos softwares web, fazendo com que esse trabalho esteja composto por tecnologias modernas que estão com boa aceitação no mercado.

% %% RiSE Latex Template - version 0.5
%%
%% RiSE's latex template for thesis and dissertations
%% http://risetemplate.sourceforge.net
%%
%% (c) 2012 Yguaratã Cerqueira Cavalcanti (yguarata@gmail.com)
%%          Vinicius Cardoso Garcia (vinicius.garcia@gmail.com)
%%
%% This document was initially based on UFPEThesis template, from Paulo Gustavo
%% S. Fonseca.
%%
%% ACKNOWLEDGEMENTS
%%
%% We would like to thanks the RiSE's researchers community, the 
%% students from Federal University of Pernambuco, and other users that have
%% been contributing to this projects with comments and patches.
%%
%% GENERAL INSTRUCTIONS
%%
%% We strongly recommend you to compile your documents using pdflatex command.
%% It is also recommend use the texlipse plugin for Eclipse to edit your documents.
%%
%% Options for \documentclass command:
%%         * Idiom
%%           pt   - Portguese (default)
%%           en   - English
%%
%%         * Text type
%%           bsc  - B.Sc. Thesis
%%           msc  - M.Sc. Thesis (default)
%%           qual - PHD qualification (not tested yet)
%%           prop - PHD proposal (not tested yet)
%%           phd  - PHD thesis
%%
%%         * Media
%%           scr  - to eletronic version (PDF) / see the users guide
%%
%%         * Pagination
%%           oneside - unique face press
%%           twoside - two faces press
%%
%%		   * Line spacing
%%           singlespacing  - the same as using \linespread{1}
%%           onehalfspacing - the same as using \linespread{1.3}
%%           doublespacing  - the same as using \linespread{1.6}
%%
%% Reference commands. Use the following commands to make references in your
%% text:
%%          \figref  -- for Figure reference
%%          \tabref  -- for Table reference
%%          \eqnref  -- for equation reference
%%          \chapref -- for chapter reference
%%          \secref  -- for section reference
%%          \appref  -- for appendix reference
%%          \axiref  -- for axiom reference
%%          \conjref -- for conjecture reference
%%          \defref  -- for definition reference
%%          \lemref  -- for lemma reference
%%          \theoref -- for theorem reference
%%          \corref  -- for corollary reference
%%          \propref -- for proprosition reference
%%          \pgref   -- for page reference
%%
%%          Example: See \chapref{chap:introduction}. It will produce 
%%                   'See Chapter 1', in case of English language.

\documentclass[pt,twoside,onehalfspacing,bsc]{risethesis}

\usepackage[sort,square]{natbib} %package for bibliography
\bibliographystyle{ieeetr} %authordate1

\usepackage[utf8]{inputenc}
\usepackage[brazilian]{babel}
\usepackage[T1]{fontenc}

\usepackage{graphicx}

%% Change the following pdf author attribute name to your name.
\usepackage[linkcolor=blue,citecolor=blue,urlcolor=blue,colorlinks,pdfpagelabels,pdftitle={Bruno Cabral's Bachelor Thesis},pdfauthor={Bruno Cabral}]{hyperref}

\address{SALVADOR}

\universitypt{Universidade Federal da Bahia}
\universityen{Federal University of Bahia}

\departmentpt{Depertamento de Ciência da Computação}
\departmenten{Computer Science Department}

\programpt{Programa Multiinstitucional de Pós-graduação em Ciência da Computação}
\programen{Graduate in Computer Science}

\majorfieldpt{Ciência da Computação}
\majorfielden{Computer Science}

\title{Sistema de apoio à Pós graduação - UFBA}
\date{Outubro/2016}

\author{Victor de Azevedo Nunes}
\adviser{Ivan do Carmo Machado}

\begin{document}

\frontmatter
\frontpage
\presentationpage

\begin{dedicatory}
Eu dedico esta dissertação...
%I dedicate this dissertation to my family, girlfriend, friends and
%professors who gave me all necessary support to get here.
\end{dedicatory}

\acknowledgements
Meus agradecimentos...

\begin{epigraph}[]{Edward V Berard}
Walking on water and developing software from a specification are easy if both are frozen
\end{epigraph}

\resumo
% Escreva seu resumo no arquivo resumo.tex
\input{resumo}

\abstract
% Write your abstract in a file called abstract.tex
\input{abstract}

% Summary (tables of contents)
\tableofcontents

% List of figures
\listoffigures

% List of tables
\listoftables

% List of acronyms
% Acronyms manual: http://linorg.usp.br/CTAN/macros/latex/contrib/acronym/acronym.pdf
\listofacronyms
\input{acronyms}

% List of listings
%\lstlistoflistings

\mainmatter

\include{chapters/intro}
\include{chapters/referencial_teorico}

% \include{chapters/introduction/main}
% \include{chapters/background/main}
% \include{chapters/proposed_solution/main}
% \include{chapters/experiment/main}
% \include{chapters/conclusion/main}

%\bibliographystyle{natbib}
%\addcontentsline{toc}{chapter}{\bibliographytocname}
\bibliography{references}

% Appendix
\clearpage
\addappheadtotoc
\appendix
\appendixpage
% \include{appendix/experiment-instruments}

\end{document}
% %% RiSE Latex Template - version 0.5
%%
%% RiSE's latex template for thesis and dissertations
%% http://risetemplate.sourceforge.net
%%
%% (c) 2012 Yguaratã Cerqueira Cavalcanti (yguarata@gmail.com)
%%          Vinicius Cardoso Garcia (vinicius.garcia@gmail.com)
%%
%% This document was initially based on UFPEThesis template, from Paulo Gustavo
%% S. Fonseca.
%%
%% ACKNOWLEDGEMENTS
%%
%% We would like to thanks the RiSE's researchers community, the 
%% students from Federal University of Pernambuco, and other users that have
%% been contributing to this projects with comments and patches.
%%
%% GENERAL INSTRUCTIONS
%%
%% We strongly recommend you to compile your documents using pdflatex command.
%% It is also recommend use the texlipse plugin for Eclipse to edit your documents.
%%
%% Options for \documentclass command:
%%         * Idiom
%%           pt   - Portguese (default)
%%           en   - English
%%
%%         * Text type
%%           bsc  - B.Sc. Thesis
%%           msc  - M.Sc. Thesis (default)
%%           qual - PHD qualification (not tested yet)
%%           prop - PHD proposal (not tested yet)
%%           phd  - PHD thesis
%%
%%         * Media
%%           scr  - to eletronic version (PDF) / see the users guide
%%
%%         * Pagination
%%           oneside - unique face press
%%           twoside - two faces press
%%
%%		   * Line spacing
%%           singlespacing  - the same as using \linespread{1}
%%           onehalfspacing - the same as using \linespread{1.3}
%%           doublespacing  - the same as using \linespread{1.6}
%%
%% Reference commands. Use the following commands to make references in your
%% text:
%%          \figref  -- for Figure reference
%%          \tabref  -- for Table reference
%%          \eqnref  -- for equation reference
%%          \chapref -- for chapter reference
%%          \secref  -- for section reference
%%          \appref  -- for appendix reference
%%          \axiref  -- for axiom reference
%%          \conjref -- for conjecture reference
%%          \defref  -- for definition reference
%%          \lemref  -- for lemma reference
%%          \theoref -- for theorem reference
%%          \corref  -- for corollary reference
%%          \propref -- for proprosition reference
%%          \pgref   -- for page reference
%%
%%          Example: See \chapref{chap:introduction}. It will produce 
%%                   'See Chapter 1', in case of English language.

\documentclass[pt,twoside,onehalfspacing,bsc]{risethesis}

\usepackage[sort,square]{natbib} %package for bibliography
\bibliographystyle{ieeetr} %authordate1

\usepackage[utf8]{inputenc}
\usepackage[brazilian]{babel}
\usepackage[T1]{fontenc}

\usepackage{graphicx}

%% Change the following pdf author attribute name to your name.
\usepackage[linkcolor=blue,citecolor=blue,urlcolor=blue,colorlinks,pdfpagelabels,pdftitle={Bruno Cabral's Bachelor Thesis},pdfauthor={Bruno Cabral}]{hyperref}

\address{SALVADOR}

\universitypt{Universidade Federal da Bahia}
\universityen{Federal University of Bahia}

\departmentpt{Depertamento de Ciência da Computação}
\departmenten{Computer Science Department}

\programpt{Programa Multiinstitucional de Pós-graduação em Ciência da Computação}
\programen{Graduate in Computer Science}

\majorfieldpt{Ciência da Computação}
\majorfielden{Computer Science}

\title{Sistema de apoio à Pós graduação - UFBA}
\date{Outubro/2016}

\author{Victor de Azevedo Nunes}
\adviser{Ivan do Carmo Machado}

\begin{document}

\frontmatter
\frontpage
\presentationpage

\begin{dedicatory}
Eu dedico esta dissertação...
%I dedicate this dissertation to my family, girlfriend, friends and
%professors who gave me all necessary support to get here.
\end{dedicatory}

\acknowledgements
Meus agradecimentos...

\begin{epigraph}[]{Edward V Berard}
Walking on water and developing software from a specification are easy if both are frozen
\end{epigraph}

\resumo
% Escreva seu resumo no arquivo resumo.tex
\input{resumo}

\abstract
% Write your abstract in a file called abstract.tex
\input{abstract}

% Summary (tables of contents)
\tableofcontents

% List of figures
\listoffigures

% List of tables
\listoftables

% List of acronyms
% Acronyms manual: http://linorg.usp.br/CTAN/macros/latex/contrib/acronym/acronym.pdf
\listofacronyms
\input{acronyms}

% List of listings
%\lstlistoflistings

\mainmatter

\include{chapters/intro}
\include{chapters/referencial_teorico}

% \include{chapters/introduction/main}
% \include{chapters/background/main}
% \include{chapters/proposed_solution/main}
% \include{chapters/experiment/main}
% \include{chapters/conclusion/main}

%\bibliographystyle{natbib}
%\addcontentsline{toc}{chapter}{\bibliographytocname}
\bibliography{references}

% Appendix
\clearpage
\addappheadtotoc
\appendix
\appendixpage
% \include{appendix/experiment-instruments}

\end{document}
% %% RiSE Latex Template - version 0.5
%%
%% RiSE's latex template for thesis and dissertations
%% http://risetemplate.sourceforge.net
%%
%% (c) 2012 Yguaratã Cerqueira Cavalcanti (yguarata@gmail.com)
%%          Vinicius Cardoso Garcia (vinicius.garcia@gmail.com)
%%
%% This document was initially based on UFPEThesis template, from Paulo Gustavo
%% S. Fonseca.
%%
%% ACKNOWLEDGEMENTS
%%
%% We would like to thanks the RiSE's researchers community, the 
%% students from Federal University of Pernambuco, and other users that have
%% been contributing to this projects with comments and patches.
%%
%% GENERAL INSTRUCTIONS
%%
%% We strongly recommend you to compile your documents using pdflatex command.
%% It is also recommend use the texlipse plugin for Eclipse to edit your documents.
%%
%% Options for \documentclass command:
%%         * Idiom
%%           pt   - Portguese (default)
%%           en   - English
%%
%%         * Text type
%%           bsc  - B.Sc. Thesis
%%           msc  - M.Sc. Thesis (default)
%%           qual - PHD qualification (not tested yet)
%%           prop - PHD proposal (not tested yet)
%%           phd  - PHD thesis
%%
%%         * Media
%%           scr  - to eletronic version (PDF) / see the users guide
%%
%%         * Pagination
%%           oneside - unique face press
%%           twoside - two faces press
%%
%%		   * Line spacing
%%           singlespacing  - the same as using \linespread{1}
%%           onehalfspacing - the same as using \linespread{1.3}
%%           doublespacing  - the same as using \linespread{1.6}
%%
%% Reference commands. Use the following commands to make references in your
%% text:
%%          \figref  -- for Figure reference
%%          \tabref  -- for Table reference
%%          \eqnref  -- for equation reference
%%          \chapref -- for chapter reference
%%          \secref  -- for section reference
%%          \appref  -- for appendix reference
%%          \axiref  -- for axiom reference
%%          \conjref -- for conjecture reference
%%          \defref  -- for definition reference
%%          \lemref  -- for lemma reference
%%          \theoref -- for theorem reference
%%          \corref  -- for corollary reference
%%          \propref -- for proprosition reference
%%          \pgref   -- for page reference
%%
%%          Example: See \chapref{chap:introduction}. It will produce 
%%                   'See Chapter 1', in case of English language.

\documentclass[pt,twoside,onehalfspacing,bsc]{risethesis}

\usepackage[sort,square]{natbib} %package for bibliography
\bibliographystyle{ieeetr} %authordate1

\usepackage[utf8]{inputenc}
\usepackage[brazilian]{babel}
\usepackage[T1]{fontenc}

\usepackage{graphicx}

%% Change the following pdf author attribute name to your name.
\usepackage[linkcolor=blue,citecolor=blue,urlcolor=blue,colorlinks,pdfpagelabels,pdftitle={Bruno Cabral's Bachelor Thesis},pdfauthor={Bruno Cabral}]{hyperref}

\address{SALVADOR}

\universitypt{Universidade Federal da Bahia}
\universityen{Federal University of Bahia}

\departmentpt{Depertamento de Ciência da Computação}
\departmenten{Computer Science Department}

\programpt{Programa Multiinstitucional de Pós-graduação em Ciência da Computação}
\programen{Graduate in Computer Science}

\majorfieldpt{Ciência da Computação}
\majorfielden{Computer Science}

\title{Sistema de apoio à Pós graduação - UFBA}
\date{Outubro/2016}

\author{Victor de Azevedo Nunes}
\adviser{Ivan do Carmo Machado}

\begin{document}

\frontmatter
\frontpage
\presentationpage

\begin{dedicatory}
Eu dedico esta dissertação...
%I dedicate this dissertation to my family, girlfriend, friends and
%professors who gave me all necessary support to get here.
\end{dedicatory}

\acknowledgements
Meus agradecimentos...

\begin{epigraph}[]{Edward V Berard}
Walking on water and developing software from a specification are easy if both are frozen
\end{epigraph}

\resumo
% Escreva seu resumo no arquivo resumo.tex
\input{resumo}

\abstract
% Write your abstract in a file called abstract.tex
\input{abstract}

% Summary (tables of contents)
\tableofcontents

% List of figures
\listoffigures

% List of tables
\listoftables

% List of acronyms
% Acronyms manual: http://linorg.usp.br/CTAN/macros/latex/contrib/acronym/acronym.pdf
\listofacronyms
\input{acronyms}

% List of listings
%\lstlistoflistings

\mainmatter

\include{chapters/intro}
\include{chapters/referencial_teorico}

% \include{chapters/introduction/main}
% \include{chapters/background/main}
% \include{chapters/proposed_solution/main}
% \include{chapters/experiment/main}
% \include{chapters/conclusion/main}

%\bibliographystyle{natbib}
%\addcontentsline{toc}{chapter}{\bibliographytocname}
\bibliography{references}

% Appendix
\clearpage
\addappheadtotoc
\appendix
\appendixpage
% \include{appendix/experiment-instruments}

\end{document}
% %% RiSE Latex Template - version 0.5
%%
%% RiSE's latex template for thesis and dissertations
%% http://risetemplate.sourceforge.net
%%
%% (c) 2012 Yguaratã Cerqueira Cavalcanti (yguarata@gmail.com)
%%          Vinicius Cardoso Garcia (vinicius.garcia@gmail.com)
%%
%% This document was initially based on UFPEThesis template, from Paulo Gustavo
%% S. Fonseca.
%%
%% ACKNOWLEDGEMENTS
%%
%% We would like to thanks the RiSE's researchers community, the 
%% students from Federal University of Pernambuco, and other users that have
%% been contributing to this projects with comments and patches.
%%
%% GENERAL INSTRUCTIONS
%%
%% We strongly recommend you to compile your documents using pdflatex command.
%% It is also recommend use the texlipse plugin for Eclipse to edit your documents.
%%
%% Options for \documentclass command:
%%         * Idiom
%%           pt   - Portguese (default)
%%           en   - English
%%
%%         * Text type
%%           bsc  - B.Sc. Thesis
%%           msc  - M.Sc. Thesis (default)
%%           qual - PHD qualification (not tested yet)
%%           prop - PHD proposal (not tested yet)
%%           phd  - PHD thesis
%%
%%         * Media
%%           scr  - to eletronic version (PDF) / see the users guide
%%
%%         * Pagination
%%           oneside - unique face press
%%           twoside - two faces press
%%
%%		   * Line spacing
%%           singlespacing  - the same as using \linespread{1}
%%           onehalfspacing - the same as using \linespread{1.3}
%%           doublespacing  - the same as using \linespread{1.6}
%%
%% Reference commands. Use the following commands to make references in your
%% text:
%%          \figref  -- for Figure reference
%%          \tabref  -- for Table reference
%%          \eqnref  -- for equation reference
%%          \chapref -- for chapter reference
%%          \secref  -- for section reference
%%          \appref  -- for appendix reference
%%          \axiref  -- for axiom reference
%%          \conjref -- for conjecture reference
%%          \defref  -- for definition reference
%%          \lemref  -- for lemma reference
%%          \theoref -- for theorem reference
%%          \corref  -- for corollary reference
%%          \propref -- for proprosition reference
%%          \pgref   -- for page reference
%%
%%          Example: See \chapref{chap:introduction}. It will produce 
%%                   'See Chapter 1', in case of English language.

\documentclass[pt,twoside,onehalfspacing,bsc]{risethesis}

\usepackage[sort,square]{natbib} %package for bibliography
\bibliographystyle{ieeetr} %authordate1

\usepackage[utf8]{inputenc}
\usepackage[brazilian]{babel}
\usepackage[T1]{fontenc}

\usepackage{graphicx}

%% Change the following pdf author attribute name to your name.
\usepackage[linkcolor=blue,citecolor=blue,urlcolor=blue,colorlinks,pdfpagelabels,pdftitle={Bruno Cabral's Bachelor Thesis},pdfauthor={Bruno Cabral}]{hyperref}

\address{SALVADOR}

\universitypt{Universidade Federal da Bahia}
\universityen{Federal University of Bahia}

\departmentpt{Depertamento de Ciência da Computação}
\departmenten{Computer Science Department}

\programpt{Programa Multiinstitucional de Pós-graduação em Ciência da Computação}
\programen{Graduate in Computer Science}

\majorfieldpt{Ciência da Computação}
\majorfielden{Computer Science}

\title{Sistema de apoio à Pós graduação - UFBA}
\date{Outubro/2016}

\author{Victor de Azevedo Nunes}
\adviser{Ivan do Carmo Machado}

\begin{document}

\frontmatter
\frontpage
\presentationpage

\begin{dedicatory}
Eu dedico esta dissertação...
%I dedicate this dissertation to my family, girlfriend, friends and
%professors who gave me all necessary support to get here.
\end{dedicatory}

\acknowledgements
Meus agradecimentos...

\begin{epigraph}[]{Edward V Berard}
Walking on water and developing software from a specification are easy if both are frozen
\end{epigraph}

\resumo
% Escreva seu resumo no arquivo resumo.tex
\input{resumo}

\abstract
% Write your abstract in a file called abstract.tex
\input{abstract}

% Summary (tables of contents)
\tableofcontents

% List of figures
\listoffigures

% List of tables
\listoftables

% List of acronyms
% Acronyms manual: http://linorg.usp.br/CTAN/macros/latex/contrib/acronym/acronym.pdf
\listofacronyms
\input{acronyms}

% List of listings
%\lstlistoflistings

\mainmatter

\include{chapters/intro}
\include{chapters/referencial_teorico}

% \include{chapters/introduction/main}
% \include{chapters/background/main}
% \include{chapters/proposed_solution/main}
% \include{chapters/experiment/main}
% \include{chapters/conclusion/main}

%\bibliographystyle{natbib}
%\addcontentsline{toc}{chapter}{\bibliographytocname}
\bibliography{references}

% Appendix
\clearpage
\addappheadtotoc
\appendix
\appendixpage
% \include{appendix/experiment-instruments}

\end{document}
% %% RiSE Latex Template - version 0.5
%%
%% RiSE's latex template for thesis and dissertations
%% http://risetemplate.sourceforge.net
%%
%% (c) 2012 Yguaratã Cerqueira Cavalcanti (yguarata@gmail.com)
%%          Vinicius Cardoso Garcia (vinicius.garcia@gmail.com)
%%
%% This document was initially based on UFPEThesis template, from Paulo Gustavo
%% S. Fonseca.
%%
%% ACKNOWLEDGEMENTS
%%
%% We would like to thanks the RiSE's researchers community, the 
%% students from Federal University of Pernambuco, and other users that have
%% been contributing to this projects with comments and patches.
%%
%% GENERAL INSTRUCTIONS
%%
%% We strongly recommend you to compile your documents using pdflatex command.
%% It is also recommend use the texlipse plugin for Eclipse to edit your documents.
%%
%% Options for \documentclass command:
%%         * Idiom
%%           pt   - Portguese (default)
%%           en   - English
%%
%%         * Text type
%%           bsc  - B.Sc. Thesis
%%           msc  - M.Sc. Thesis (default)
%%           qual - PHD qualification (not tested yet)
%%           prop - PHD proposal (not tested yet)
%%           phd  - PHD thesis
%%
%%         * Media
%%           scr  - to eletronic version (PDF) / see the users guide
%%
%%         * Pagination
%%           oneside - unique face press
%%           twoside - two faces press
%%
%%		   * Line spacing
%%           singlespacing  - the same as using \linespread{1}
%%           onehalfspacing - the same as using \linespread{1.3}
%%           doublespacing  - the same as using \linespread{1.6}
%%
%% Reference commands. Use the following commands to make references in your
%% text:
%%          \figref  -- for Figure reference
%%          \tabref  -- for Table reference
%%          \eqnref  -- for equation reference
%%          \chapref -- for chapter reference
%%          \secref  -- for section reference
%%          \appref  -- for appendix reference
%%          \axiref  -- for axiom reference
%%          \conjref -- for conjecture reference
%%          \defref  -- for definition reference
%%          \lemref  -- for lemma reference
%%          \theoref -- for theorem reference
%%          \corref  -- for corollary reference
%%          \propref -- for proprosition reference
%%          \pgref   -- for page reference
%%
%%          Example: See \chapref{chap:introduction}. It will produce 
%%                   'See Chapter 1', in case of English language.

\documentclass[pt,twoside,onehalfspacing,bsc]{risethesis}

\usepackage[sort,square]{natbib} %package for bibliography
\bibliographystyle{ieeetr} %authordate1

\usepackage[utf8]{inputenc}
\usepackage[brazilian]{babel}
\usepackage[T1]{fontenc}

\usepackage{graphicx}

%% Change the following pdf author attribute name to your name.
\usepackage[linkcolor=blue,citecolor=blue,urlcolor=blue,colorlinks,pdfpagelabels,pdftitle={Bruno Cabral's Bachelor Thesis},pdfauthor={Bruno Cabral}]{hyperref}

\address{SALVADOR}

\universitypt{Universidade Federal da Bahia}
\universityen{Federal University of Bahia}

\departmentpt{Depertamento de Ciência da Computação}
\departmenten{Computer Science Department}

\programpt{Programa Multiinstitucional de Pós-graduação em Ciência da Computação}
\programen{Graduate in Computer Science}

\majorfieldpt{Ciência da Computação}
\majorfielden{Computer Science}

\title{Sistema de apoio à Pós graduação - UFBA}
\date{Outubro/2016}

\author{Victor de Azevedo Nunes}
\adviser{Ivan do Carmo Machado}

\begin{document}

\frontmatter
\frontpage
\presentationpage

\begin{dedicatory}
Eu dedico esta dissertação...
%I dedicate this dissertation to my family, girlfriend, friends and
%professors who gave me all necessary support to get here.
\end{dedicatory}

\acknowledgements
Meus agradecimentos...

\begin{epigraph}[]{Edward V Berard}
Walking on water and developing software from a specification are easy if both are frozen
\end{epigraph}

\resumo
% Escreva seu resumo no arquivo resumo.tex
\input{resumo}

\abstract
% Write your abstract in a file called abstract.tex
\input{abstract}

% Summary (tables of contents)
\tableofcontents

% List of figures
\listoffigures

% List of tables
\listoftables

% List of acronyms
% Acronyms manual: http://linorg.usp.br/CTAN/macros/latex/contrib/acronym/acronym.pdf
\listofacronyms
\input{acronyms}

% List of listings
%\lstlistoflistings

\mainmatter

\include{chapters/intro}
\include{chapters/referencial_teorico}

% \include{chapters/introduction/main}
% \include{chapters/background/main}
% \include{chapters/proposed_solution/main}
% \include{chapters/experiment/main}
% \include{chapters/conclusion/main}

%\bibliographystyle{natbib}
%\addcontentsline{toc}{chapter}{\bibliographytocname}
\bibliography{references}

% Appendix
\clearpage
\addappheadtotoc
\appendix
\appendixpage
% \include{appendix/experiment-instruments}

\end{document}

%\bibliographystyle{natbib}
%\addcontentsline{toc}{chapter}{\bibliographytocname}
\bibliography{references}

% Appendix
\clearpage
\addappheadtotoc
\appendix
\appendixpage
% \include{appendix/experiment-instruments}

\end{document}
% %% RiSE Latex Template - version 0.5
%%
%% RiSE's latex template for thesis and dissertations
%% http://risetemplate.sourceforge.net
%%
%% (c) 2012 Yguaratã Cerqueira Cavalcanti (yguarata@gmail.com)
%%          Vinicius Cardoso Garcia (vinicius.garcia@gmail.com)
%%
%% This document was initially based on UFPEThesis template, from Paulo Gustavo
%% S. Fonseca.
%%
%% ACKNOWLEDGEMENTS
%%
%% We would like to thanks the RiSE's researchers community, the 
%% students from Federal University of Pernambuco, and other users that have
%% been contributing to this projects with comments and patches.
%%
%% GENERAL INSTRUCTIONS
%%
%% We strongly recommend you to compile your documents using pdflatex command.
%% It is also recommend use the texlipse plugin for Eclipse to edit your documents.
%%
%% Options for \documentclass command:
%%         * Idiom
%%           pt   - Portguese (default)
%%           en   - English
%%
%%         * Text type
%%           bsc  - B.Sc. Thesis
%%           msc  - M.Sc. Thesis (default)
%%           qual - PHD qualification (not tested yet)
%%           prop - PHD proposal (not tested yet)
%%           phd  - PHD thesis
%%
%%         * Media
%%           scr  - to eletronic version (PDF) / see the users guide
%%
%%         * Pagination
%%           oneside - unique face press
%%           twoside - two faces press
%%
%%		   * Line spacing
%%           singlespacing  - the same as using \linespread{1}
%%           onehalfspacing - the same as using \linespread{1.3}
%%           doublespacing  - the same as using \linespread{1.6}
%%
%% Reference commands. Use the following commands to make references in your
%% text:
%%          \figref  -- for Figure reference
%%          \tabref  -- for Table reference
%%          \eqnref  -- for equation reference
%%          \chapref -- for chapter reference
%%          \secref  -- for section reference
%%          \appref  -- for appendix reference
%%          \axiref  -- for axiom reference
%%          \conjref -- for conjecture reference
%%          \defref  -- for definition reference
%%          \lemref  -- for lemma reference
%%          \theoref -- for theorem reference
%%          \corref  -- for corollary reference
%%          \propref -- for proprosition reference
%%          \pgref   -- for page reference
%%
%%          Example: See \chapref{chap:introduction}. It will produce 
%%                   'See Chapter 1', in case of English language.

\documentclass[pt,twoside,onehalfspacing,bsc]{risethesis}

\usepackage[sort,square]{natbib} %package for bibliography
\bibliographystyle{ieeetr} %authordate1

\usepackage[utf8]{inputenc}
\usepackage[brazilian]{babel}
\usepackage[T1]{fontenc}

\usepackage{graphicx}

%% Change the following pdf author attribute name to your name.
\usepackage[linkcolor=blue,citecolor=blue,urlcolor=blue,colorlinks,pdfpagelabels,pdftitle={Bruno Cabral's Bachelor Thesis},pdfauthor={Bruno Cabral}]{hyperref}

\address{SALVADOR}

\universitypt{Universidade Federal da Bahia}
\universityen{Federal University of Bahia}

\departmentpt{Depertamento de Ciência da Computação}
\departmenten{Computer Science Department}

\programpt{Programa Multiinstitucional de Pós-graduação em Ciência da Computação}
\programen{Graduate in Computer Science}

\majorfieldpt{Ciência da Computação}
\majorfielden{Computer Science}

\title{Sistema de apoio à Pós graduação - UFBA}
\date{Outubro/2016}

\author{Victor de Azevedo Nunes}
\adviser{Ivan do Carmo Machado}

\begin{document}

\frontmatter
\frontpage
\presentationpage

\begin{dedicatory}
Eu dedico esta dissertação...
%I dedicate this dissertation to my family, girlfriend, friends and
%professors who gave me all necessary support to get here.
\end{dedicatory}

\acknowledgements
Meus agradecimentos...

\begin{epigraph}[]{Edward V Berard}
Walking on water and developing software from a specification are easy if both are frozen
\end{epigraph}

\resumo
% Escreva seu resumo no arquivo resumo.tex
Este trabalho de conclusão de curso utiliza elementos da engenharia de software para propor um software como serviço a fim de otimizar o processo de construção e manutenção dos softwares. Assim, o objetivo deste SaaS é gerenciar as permissões de acesso de sistemas cliente, provendo o reuso de software e padronizando as soluções. O sistema proposto neste trabalho encontra-se implementado e disponível no Github, e traz fundamentos sobre a arquitetura e tecnologias utilizadas, bem como avaliações sobre possibilidades de utilização do mesmo em diversos ambientes, como Web e mobile.

\begin{keywords}
Software; Reuso; SaaS; Web; PHP
\end{keywords}

\abstract
% Write your abstract in a file called abstract.tex
My abstract...

\begin{keywords}
key words...
\end{keywords}

% Summary (tables of contents)
\tableofcontents

% List of figures
\listoffigures

% List of tables
\listoftables

% List of acronyms
% Acronyms manual: http://linorg.usp.br/CTAN/macros/latex/contrib/acronym/acronym.pdf
\listofacronyms
\begin{acronym}[ACRONYM] 
% Change the word ACRONYM above to change the acronym column width.
% The column width is equals to the width of the word that you put.
% Read the manual about acronym package for more examples:
%   http://linorg.usp.br/CTAN/macros/latex/contrib/acronym/acronym.pdf

\acro{SPA}{Single Page Application}
\acro{JSON}{Javascript Object Notation}
\acro{PHP}{PHP: Hypertext Preprocessor}
\acro{SaaS}{Software as a Service}
\acro{ERP}{Enterprise Resource Planning}
\acro{QoS}{Quality of Service}
\acro{UML}{Unified Modeling Language}
\acro{MVC}{Model-View-Controller}
\acro{Ajax}{Asynchronous Javascript and XML}
\acro{HTML}{HyperText Markup Language}
\acro{CSS}{Cascading Style Sheets}
\acro{API}{Application Programming Interface}
\acro{DOM}{Document Object Model}
\acro{BPMN}{Business Process Model and Notation}
\acro{REST}{Representational State Transfer}

\end{acronym}

% List of listings
%\lstlistoflistings

\mainmatter

\chapter{Introdução}

\section{Motivação}

Organizar os procedimentos de um processo sempre nos traz vantagens. Apesar de no processo de implantação de um sistema, o mesmo burocratizar o processo, com o tempo temos o retorno da dedicação para a inserção dos dados. Com um certo volume de dados, é possível estruturar informações que num processo manual são difíceis de serem enxergadas. Assim, é possível depender menos das pessoas que organizam o processo, pois o legado de informações não estará mais somente na mente de alguns, mas sim documentado nos dados do sistema.

Além de colaborar na organização, também haverá uma grande colaboração no tempo gasto na gestão. Lidar com muitos papéis e confiar na mente humana para guardar informações, não é uma alternativa muito segura devido ao fato que as pessoas sempre estão sujeitas a sair do processo e levar contigo a experiência obtida. Experiência essa que faz com que os procedimentos sejam executados de forma mais eficiente. Entretanto, com um sistema inteligente, é possível auxiliar e tornar mais ágil a execução das tarefas.


\section{Problema}


De acordo com funcionários ligados ao o setor de pós graduação da UFBA, entrevistados a fim de um maior entendimento do cenário, apesar das semelhanças estruturais, a pós graduação gerida de forma diferencia da graduação. FULANO afirma que devido ao fato de não ter a mesma visibilidade, não tem acesso aos mesmos recursos de gestão acadêmica da graduação. O professores não executam somente atividades dentro da sala de aula, também tem diversas outras ocupações no setor. E muitos procedimentos realizados extra classe ainda se encontram sendo realizados de forma manual, estando mais vulnerável ao erro ou até mesmo à violação do processo. Também ocorre um grande desperdício de tempo pelos professores e gestores da área, devido ao diversos processos ainda realizados de forma manual, sem a devida documentação. Segundo FULANO, também entrevistado, esse tempo perdido implica numa redução da eficiência na sala de aula, pois o professor acaba por ter menos tempo disponível para o planejamento das atividades, o que gera impactos negativos aos alunos.


\section{Objetivos} %<o que deve ser feito/entregue>


Devido aos muitos processos sendo resolvidos de forma manual, propõe-se com solução um sistema moderno, arquitetado para ter funcionamento na web e com um módulo mobile, a fim de fornecer informações de forma rápida e eficiente para os professores através de notificações, já que o acesso à internet móvel é comum entre os possíveis usuários do sistema em questão.
O principal requisito para o sistema seria dispor recursos para reduzir o tempo desperdiçado pelos professores durante as atividades extra classe.


\section{Metodologia} %<como será feito | como resolver o problema apontado inicialmente>


%<analise de literatura | design | implementação | validação>
Baseando-se nas tecnologias gratuitas em alta no cenário atual do desenvolvimento web, dispomos de algumas opções eficientes para a implementação da solução. Dentre as possibilidades, considerando a facilidade para futura manutenção e continuidade do projeto, tende-se a optar por uma tecnologia popular. Como linguagem de programação, adota-se o PHP. A escolha é fundamentada de acordo com a pesquisa da RedMonk de 2015, que evidencia o uso das linguagens de programação de acordo com as discussões no StackOverflow e repositórios no GitHub. É possível constatar a popularidade do PHP no cenário atual com o gráfico da pesquisa citada, na qual o PHP é apresentado na terceira colocação, apenas atrás do lider JavaScript e do segundo colocado, o Java.

\begin{figure}
	\label{fig:graficoRedmonk}
	\includegraphics[width=1\textwidth]{img/grafico_redmonk}
	\caption{Ranking das liguagens de programação no Stack Overflow e Github}
\end{figure}


Ainda assim, para compor a interface do dado projeto, também ocorrerá o uso do líder JavaScript de forma intensa, provendo o elo com o as informações gerenciadas pelo PHP.


Entretanto, não seria inteligente desenvolver um sistema completo sem o auxílio de um framework. Dentre os frameworks disponíveis para PHP, hoje o destaque está com o Laravel, que se encontra no topo dentre os mais utilizados no momento. 


A WebHostFace, uma empresa de hospedagem, compilou várias estatísticas para criar um infográfico mostrando os frameworks PHP mais populares de 2015. Utilizando informações sobre os próprios clientes, o Google Trends, estatísticas de repositórios do GitHub e a pesquisa do SitePoint “Best PHP Frameworks 2015”, a WebHostFace elaborou o seguinte infográfico: 

\begin{figure}
	\label{fig:graficoWebhostface}
	\includegraphics[width=1\textwidth]{img/infografico_webhostface}
	\caption{Infográfico da WebhostFace, exibindo a popularidade dos Frameworks PHP em 2015}
\end{figure}

Assim, tem-se a evidência que o Laravel em 2015 teve a maior popularidade em projetos pessoais e tem a maior comunidade entre os concorrentes, o que o torna uma boa escolha para a escrita de um software que será continuado por terceiros.


Para elaborar os recursos de interface e integrar ao back-end PHP do sistema, será adotado o já conhecido AngularJS, ferramenta sólida e conhecida no aspecto em questão. 


Dados coletados via Google Trends, que propõe comparações entre termos pesquisados, revela a popularidade do AngularJs diante de alguns dos principais concorrentes. O gráfico abaixo evidencia o cenário.


%Como mostra a Figura \ref{fig:graficoGoogleTrendsFerramentasFront}. 
\begin{figure}
	\label{fig:graficoGoogleTrendsFerramentasFront}
	\includegraphics[width=1\textwidth]{img/grafico_ferramentas_front}
	\caption{Gráfico do Google Trends exibindo as pesquisas por ferramentas front-end}
\end{figure}


Junto ao Angular JS, será utilizada a agradável tendência de interface do Material Design da Google, que propõe layouts limpos e otimizados já conhecidos pelos usuários de smartphones Android. 


Para a elaboração da plataforma mobile do projeto, será utilizado o Ionic Framework, muito difundido e bastante pesquisado na área, o que fica evidenciado com o gráfico de pesquisbaixo, coletado via Google Trends buscando por frameworks de desenvolvimento híbrido mobile.


\begin{figure}
	\label{fig:graficoGoogleTrendsFerramentasHibridasMobile}
	\includegraphics[width=1\textwidth]{img/grafico_ferramentas_hibridas_mobile}
	\caption{Gráfico do Google Trends exibindo as pesquisas por Frameworks híbridos mobile}
\end{figure}	

Para layout da interface mobile, também será aplicado a tendência do Material Design, a fim de propor uma harmonia entre o módulo web e mobile para os usuários


\section{Resultados Esperados}


Como fruto de um sistema para pós-graduação da UFBA, espera-se que os professores tenham mais recursos para integrar as atividades e também prover melhores condições para acompanhamento da vida acadêmica dos alunos em questão. Também, que os novos colaboradores que entrarem no processo tenham facilidade de compreender o fluxo do setor ao navegar pelo sistema proposto.


\section{Fora de Escopo}


Interação com os alunos devido às complicações para realizar a integração com o sistema empregado na UFBA, gerenciado pela XXXXXX, o que causaria uma inviabilidade no projeto devido à necessidade de entrega do produto ser mais forte que o tempo necessário para executar o processo de obtenção de acesso ao sistema legado para realizar a integração.


\section{Estrutura do Trabalho}


<breve resumo sobre os capítulos do TCC>
\chapter{Referencial Teórico}


Projetar o desenvolvimento de um software requer muito planejamento, pois as falhas iniciais podem custar bastante caro ou até mesmo inviabilizar a continuação de um projeto. Assim, a escolha da arquitetura ideal para a aplicabilidade é essencial na concepção de um produto de software. 
De todo o modo, sempre busca-se fazer mais com menos. Diante de tal filosofia, temos neste capítulo, uma breve discussão sobre alguns elementos de projeto e arquitetura de software, a fim de contextualizar este trabalho de conclusão de curso.
O capítulo corrente é composto por quatro seções. A \ref{sec:saas} trata de Software como serviço, discutindo alguns elementos do contexto que são relevantes para o trabalho proposto. A \ref{sec:reuso}  discute sobre a empregabilidade do reuso de software. A \ref{sec:modularizacao} seção, trata sobre aspectos envolvidos na modularização dos softwares. Por fim, a \ref{sec:apps_web} aborda as aplicações web, discutindo sobre aspectos relevantes sobre a aplicação web que compõe este trabalho.


 \section{Software como serviço}\label{sec:saas}


Segundo La e Chun \citep{La2009Systematic}, o princípio da definição de Software como um Serviço (Sofware as a Service - SaaS) é um serviço complementar para aplicações da computação em nuvem (cloud computing). No entanto, as áreas não se confundem. SaaS deve ser entendido como um mecanismo de suporte às soluções existentes na cloud. Os SaaS existem justamente para maximizar o reuso de serviços repetidos e não centrais em uma aplicação remota.


Como vantagens, diversos fatores podem ser favoráveis para a adoção de um SaaS, como custo e manutenção dentre outros fatores aplicáveis a soluções específicas. Lechesaet al. \citep{LechesaSS11}, quantificam índices sobre os fatores determinantes para adoção ou não de um SaaS voltado para ERP na África do Sul. Os principais fatores determinantes para adoção desse mecanismo de software são sua fluidez quanto à rede e a segurança. Esses fatores estão presentes na aplicação desenvolvida neste trabalho de conclusão de curso.


Devido ao fato de ter um serviço constantemente na nuvem, fica o questionamento sobre a segurança da informação manipulada. Sabe-se que a vulnerabilidade na Web não é restrita ao SaaS, atingindo diversos âmbitos. Rai et al. \cite{journals/corr/RaiSM13} consideram que o avanço da computação em nuvem não é um problema apenas para os serviços Web do ponto de vista da segurança, pois muitos trabalhos na literatura mostram a área como mais um ponto de vulnerabilidade para diversos setores, a exemplo de infraestrutura. No mesmo artigo mencionado de Rai et al. \cite{journals/corr/RaiSM13}, também realizaram-se estudos exploratórios junto a empresas usuárias de serviços em computação em nuvem e consideram que a perspectiva de SaaS também pode fortalecer a segurança nas aplicações de cloud computing, pois o software de autenticação compartilhado por várias aplicações em nuvem, oferece uma melhor padronização e consequente facilidade de prevenção a erros de vulnerabilidade específicas de cada módulo da pesquisa. Esse ponto de vista é fundamenta para a compreensão de SaaS.


A arquitetura de armazenamento de dados de um Saas pode variar de acordo com a necessidade do contexto. Huixin \cite{7586486} descreve possíveis modelagens nesse sentido. Essa abordagem pode ser com um banco de dados único, fazendo com que diferentes clientes compartilhem o mesmo banco, diferindo os dados através de controle de usuário, ou isolando os diferentes clientes através de bancos de dados exclusivos para cada um. Esse fator também pode ser combinado com a arquitetura da aplicação, caso ofereça aplicação única para todos os clientes ou aplicação compartilhada. Diante das possíveis abordagens, a modelagem de dados do software pode ser decidida pela regra de negócio. Este trabalho optou por aplicação única e banco de dados compartilhado.

Devido ao conceito particular de obtenção de software, tanto pela visão do cliente como pela visão do vendedor, é necessário compreender os diversos aspectos que podem ser relevantes ao orçar um Saas. O recente trabalho de T. Kaur et al. \citep{6949281} orienta um modelo para compor o custo de um Saas. O custo total seria composto pelos fatores que dão suporte ao funcionamento do software. Tais fatores incluem infra-estrutura, configurabilidade, customização, parâmetros de QoS (Quality of service) como escalabilidade, disponibilidade, usabilidade, pontualidade e desempenho da resposta, portabilidade, custo total de propriedade e retorno do investimento. Esses fatores caracterizam o custo de forma eficaz, possibilitando ao fornecedor, prover um Serviço de acordo com a exigência do consumidor em vários pacotes de serviços.


O conceito de software como serviço se aplica neste trabalho de conclusão de curso, pois o mesmo estará disponível na web com alta disponibilidade adotando as características apresentadas para qualquer pessoa que desejar usá-lo. Assim, basta se cadastrar, configurá-lo e usar como um serviço, sem preocupação com a manutenção do mesmo.


\section{Reuso de software}\label{sec:reuso} %CRUISE BOOK CAPITULO 2


De acordo com o livro Software pratical reuse \citep{Ezran2002}, o reuso de software é a utilização de qualquer informação que um desenvolvedor pode necessitar no processo de criação de software. O livro de Basili e Rombach \citep{Basili1991} define reutilização de software como o uso de tudo o que está associado a um projeto de conhecimento.
Assim, o objetivo da reutilização de software é reciclar o design, código e outros componentes de um produto de software e assim reduzir o custo, o tempo e melhorar a qualidade do produto.
Segundo Keswani et al. \cite{6783445}, o componente reutilizável de software pode ser qualquer parte de seu desenvolvimento, como um fragmento de código, design, casos de teste, ou até mesmo a especificação de requisitos de uma funcionalidade do software. 

O reuso de software pode ter impacto positivo em diversos aspectos do software, vejamos alguns, conforme apresentados no C.R.U.I.S.E Book \citep{cruise2007} :

\begin{itemize}

\item Qualidade: As correções de erro tornam-se úteis em todos os locais em que ocorreu, padronizando e facilitando a manutenção.

\item Produtividade: O ganho de produtividade é alcançado devido ao menor número de artefatos desenvolvido. Isso resulta em menor esforço de teste e também análise e design, reduzindo custos.

\item Confiabilidade: A utilização de componentes bem testados aumenta a
confiança no software. Além disso, a utilização de um mesmo componente em vários sistemas, aumenta a possibilidade de detecção de erros e reforça a confiança no componente.

\item Redução do Esforço: A reutilização de software proporciona uma redução do tempo de desenvolvimento, o que reduz o tempo necessário para o produto ser disponibilizado no mercado para trazer rentabilidade.

\item Trabalho redundante e tempo de desenvolvimento: Desenvolver um sistema do
zero significa desenvolvimento redundante de muitos componentes, como requisitos, especificações, casos de uso, arquitetura, etc. Isso pode ser evitado quando estes estão disponíveis como componentes reutilizáveis e podem ser compartilhados, resultando em um processo de desenvolvimento otimizado.

\item Documentação: Embora a documentação seja muito importante para a
manutenção de um sistema, muitas vezes é negligenciada. A reutilização de componentes de software reduz a quantidade de documentação a ser escrita, entretanto depende da qualidade do que está escrito. Assim, apenas a estrutura do sistema e os novos artefatos desenvolvidos necessitam ser documentados.

\item Custo de manutenção: Menos defeitos e manutenções são esperados quando tem-se comprovada a qualidade dos componentes utilizados.

\item Tamanho da equipe: É comum haver casos em que a equipe de desenvolvimento sofre sobrecarga. Entretanto, dobrar o tamanho da equipe de desenvolvimento não necessariamente duplica produtividade. Se muitos componentes podem ser reutilizados, é possível desenvolver com equipes menores, levando a melhor comunicação e aumento da produtividade.

\end{itemize}

Apesar dos benefícios da reutilização de software, ela não é suficientemente aproveitada. Existem fatores que influenciam direta ou indiretamente na sua adoção. Esses fatores podem ser de aspecto gerencial, organizacional, econômico, conceitual ou técnico. Veremos a seguir alguns aspectos que podem gerar conflito com a cultura de reuso de software, segundo o C.R.U.I.S.E Book \citep{cruise2007} :
%(Sametinger, 1997). REVER

\begin{itemize}
	
\item Falta de apoio da gestão: Como a reutilização de software gera custos iniciais,
a medida pode não ser amplamente alcançada em uma organização sem o apoio de alto nível de gestão. Os gestores têm de ser informados sobre os custos iniciais e serem convencidos sobre economias futuras.

\item Gerenciamento do Projeto: Gerenciar projetos tradicionais é uma tarefa árdua, principalmente, os que praticam a reutilização de software. Utilizando a técnica em larga escala, tem-se impacto sobre todo o ciclo de vida do software.

\item Estruturas organizacionais inadequadas: As estruturas organizacionais devem
considerar diferentes necessidades que surgem quando a reutilização em larga escala está sendo adotada. Por exemplo, uma equipe particionada pode ser alocada somente para desenvolver, manter e certificar componentes reutilizáveis de software.

\item Incentivos de gestão: É comum a falta de incentivo para deixar os desenvolvedores gastarem tempo elaborando componentes do sistemas. A produtividade é muitas vezes medida apenas no tempo necessário para concluir um projeto. Assim, fazer qualquer trabalho além disso, embora benéfico para a empresa como um todo, diminui o seu sucesso. Mesmo quando os componentes reutilizáveis são utilizados, os benefícios obtidos são uma pequena fração do que poderia ser alcançado caso houvesse reutilização explícita, planejada e organizada.

\item Dificuldade de encontrar software reutilizável: Para reutilizar os componentes, devem existir formas eficientes de busca. Além disso, é importante ter um repositório bem organizado contendo componentes com um eficiente meio de acesso.

\item Não reutilização do software encontrado. O acesso fácil ao software existente
não necessariamente aumentar a reutilização. Os componentes reutilizáveis devem ser cuidadosamente especificados, projetados, implementados e documentados, pois em alguns casos, modificar e adaptar o código  pode ser mais custoso que a programação da funcionalidade necessária a partir do zero.

\item Modificação: É muito difícil encontrar um componente que funcione
exatamente da mesma maneira que queremos. Desta forma, são necessárias modificações e devem existir formas de determinar os seus efeitos sobre o componente.


\end{itemize}


A definição do reuso de software pode ser associada a este trabalho, pois ainda que como um serviço, estaremos adotando uma aplicação única para servir aos mais diversos softwares de uma empresa, por exemplo. Logo, ao adotar o projeto, pode ser interessante aplicar algumas métricas para tomar conhecimento de possíveis vantanjens como as citadas nessa seção, a exemplo de economia de tempo de desenvolvimento, e custo do projeto.


%Outra diretriz importante para a reutilização de software é reduzir o risco na criação de novos softwares. O risco tende a ser bastante reduzido se os componentes que estão sendo reutilizados têm as documentação, interfaces necessárias e devidamente testadas, fatores que contibruem para uma fácil integração.
%De acordo com Keswani et al. \citep{6783445}, para o reuso de software dar retornos apropriados, o processo deve ser sistemático e planejado. Qualquer organização que implemente a reutilização de software deve identificar os melhores métodos e estratégias de reutilização para obter a máxima produtividade. A reutilização de software ajuda a evitar software de engenharia a partir do zero, pois usa módulos de software existentes. A reutilização de software, embora seja uma tarefa difícil, especialmente para softwares antigos sem padrões de projeto, pode melhorar significativamente a produtividade e a qualidade de um produto de software. Embora a reutilização de software não seja um novo campo, ela pode dar grandes retornos em curto período de tempo.


\section{Modularização}\label{sec:modularizacao} %artigo de claudio pagina 222 introdução


%A modularidade vem desempenhando um papel predominante estágios emergentes das disciplinas de arquitetura de software [13]. Engenheiros de software consideram modularidade como princípio base na comparação entre arquiteturas alternativas  e arquitetura degeneração [9]. De fato, os engenheiros de software são incentivados a arquitecturas, baseando-se numa multiplicidade de mecanismos de modularidade disponíveis em: 
%(i) Linguagens de descrição de arquitetura (ADLs), como ACME [8], 
%(ii) catálogos de arquitetônicos [2, 13], e 
%(iii) conhecem bem princípios de alto nível, como interfaces de componentes estreitos, acoplamento arquitectónico reduzido e semelhantes.


Conforme é frisado por Wickramaarachchi e Lai \citep{7062705}, o conceito de modularização na indústria de software tem uma longa história e tem sido utilizado para melhorar o processo de desenvolvimento de software em diferentes estágios. Os principais conceitos por trás da modularização do software foram introduzidos por pesquisadores pioneiros há quarenta anos, com uma notável contribuição feita por Melvin Conway e David Parnas, que tem representação notável na engenharia de software.


Modularizar um software é um bom padrão a ser adotado. Segundo Wickramaarachchi e Lai \citep{7062705}, a modularização é importante na identificação de dependências e reduz as dificuldades diante de uma possível necessidade de grandes alterações. De uma perspectiva da engenharia de software, uma modularização geralmente tem várias vantagens, tais como: tornar a complexidade do software mais gerenciável, facilitar o trabalho paralelo e tornar o software mais maleável para acomodar o futuro incerto que um software pode ter. O objetivo final da modularização do software é aumentar a produtividade ea qualidade do software. Tal conceito encontra-se bastante difundido e estái incorporado em linguagens de programação e ferramentas de software. O presente trabalho favorece ao uso da modularização de um software e até mesmo pode ser considerado um módulo a ser acoplado a qualquer software, mediante a compatibilidade.


Ao realizar a adoção do trabalho proposto, fica bastante evidente a "responsabilidade" do mesmo no escopo do projeto. Assim, é possível tratá-lo como um módulo do projeto que o usa, mesmo que seja consumido como um serviço.


\section{Aplicações web}\label{sec:apps_web}


A popularidade das soluções Web aumentou exponencialmente na última década e todos os dias cresce o número de pessoas usuárias desse tipo de software. E seguindo um padrão próprio, Kumar et al. \citep{7813710} sugerem que para o desenvolvimento web, deve-se manter a prática eficaz de produzir diagramas UML. A abordagem baseada na web oferece uma maneira fácil e eficaz para gerenciar e controlar o processo de desenvolvimento por meio de artefatos de modelagem. Tal abordagem pode ser usada quando há uma exigência de lidar com mudanças muito rápidas e grandes em requisitos de forma muito eficaz em muito menos tempo, gerando assim um menor impacto. 


Para atender à fomentada demanda de aplicativos web, é necessário adotar métodos de desenvolvimentos que sejam ágeis, eficientes e de fácil manutenção. Yu Ping et al. \cite{1372143} propõem o uso do modelo MVC (Model, View e Controller) no desenvolvimento para softwares web. O modelo apresentado tornou-se um padrão popular e divide o software em camadas com propósito definido, tornando-o de mais fácil manutenção.


O Ajax (Asynchronous Javascript and XML) revolucionou a web. Conforme demonstrado por Yuping \citep{6845605}, ao usar a tecnologia Ajax, podemos enriquecer a experiência do usuário em aplicações baseadas em navegador de internet, e fornecer uma variedade de aplicações interativas para atender às necessidade de humanização das aplicações.
Os aplicativos Ajax em execução no navegador se comunicam com um servidor Web de forma assíncrona e atualizam apenas uma parte da página.


No artigo de Tesarik et al. \citep{4664349}, temos que o design de software SPA (Single page application) é uma maneira interessante de criar um software Web numa uma única página. Essa solução de página única sem navegação funciona apenas com base em técnicas dinâmicas e assíncronas, como o AJAX (citado no parágrafo anterior). No entanto, esta abordagem coloca o desenvolvedor antes de alguns desafios substanciais.
Para projetar a interface do usuário que mostra as informações, é importante projetar corretamente a tela para manipular os dados do aplicativo numa única página. O design da página deve ser elaborado para maximizar a decomposição da página em componentes distintos que encapsulam os principais casos de uso.
Também recomenda-se que para elaborar os artefatos visuais, explore os mais ricos recursos na implementação, como HTML5, JavaScript, Ajax, CSS3, e outras tecnologias que se apliquem. Entretanto, desenvolver uma rica interface com o uso de diversas tecnologias/frameworks pode ocasionar um esforço maior para explorar as possibilidades.
Tais componentes da interface de uma aplicação SPA normalmente são alimentados mediante o consumo de uma API Rest via requisições AJAX. Assim, é possível particionar as responsabilidades de processamento do software entre cliente e servidor.


Como é possível denotar no trabalho de Salvadori e Siqueire \citep{7195633}, REST é uma arquitetura muito popular para integração de aplicativos web, que permite compartilhar e reutilizar informações através de sistemas. Aplicações de grande escala baseadas em no REST deve ser implementadas utilizando estratégias e mecanismos para produzir sistemas que sejam fáceis de desenvolver, reutilizar e manter. As interfaces de integração fornecidas por esses sistemas, chamadas de Web API, têm uma influência importante nas características da implementação resultante, pois a sua forma de resposta deve ser casada com a interface que o consome.
 
 
Os elementos comentados nessa seção encontram-se presentes neste trabalho de conclusão de curso. Juntos, montam a estrutura tecnológica necessária aliada à arquitetura adotada. Esses elementos seguem tendências atuais dos softwares web, fazendo com que esse trabalho esteja composto por tecnologias modernas que estão com boa aceitação no mercado.

% %% RiSE Latex Template - version 0.5
%%
%% RiSE's latex template for thesis and dissertations
%% http://risetemplate.sourceforge.net
%%
%% (c) 2012 Yguaratã Cerqueira Cavalcanti (yguarata@gmail.com)
%%          Vinicius Cardoso Garcia (vinicius.garcia@gmail.com)
%%
%% This document was initially based on UFPEThesis template, from Paulo Gustavo
%% S. Fonseca.
%%
%% ACKNOWLEDGEMENTS
%%
%% We would like to thanks the RiSE's researchers community, the 
%% students from Federal University of Pernambuco, and other users that have
%% been contributing to this projects with comments and patches.
%%
%% GENERAL INSTRUCTIONS
%%
%% We strongly recommend you to compile your documents using pdflatex command.
%% It is also recommend use the texlipse plugin for Eclipse to edit your documents.
%%
%% Options for \documentclass command:
%%         * Idiom
%%           pt   - Portguese (default)
%%           en   - English
%%
%%         * Text type
%%           bsc  - B.Sc. Thesis
%%           msc  - M.Sc. Thesis (default)
%%           qual - PHD qualification (not tested yet)
%%           prop - PHD proposal (not tested yet)
%%           phd  - PHD thesis
%%
%%         * Media
%%           scr  - to eletronic version (PDF) / see the users guide
%%
%%         * Pagination
%%           oneside - unique face press
%%           twoside - two faces press
%%
%%		   * Line spacing
%%           singlespacing  - the same as using \linespread{1}
%%           onehalfspacing - the same as using \linespread{1.3}
%%           doublespacing  - the same as using \linespread{1.6}
%%
%% Reference commands. Use the following commands to make references in your
%% text:
%%          \figref  -- for Figure reference
%%          \tabref  -- for Table reference
%%          \eqnref  -- for equation reference
%%          \chapref -- for chapter reference
%%          \secref  -- for section reference
%%          \appref  -- for appendix reference
%%          \axiref  -- for axiom reference
%%          \conjref -- for conjecture reference
%%          \defref  -- for definition reference
%%          \lemref  -- for lemma reference
%%          \theoref -- for theorem reference
%%          \corref  -- for corollary reference
%%          \propref -- for proprosition reference
%%          \pgref   -- for page reference
%%
%%          Example: See \chapref{chap:introduction}. It will produce 
%%                   'See Chapter 1', in case of English language.

\documentclass[pt,twoside,onehalfspacing,bsc]{risethesis}

\usepackage[sort,square]{natbib} %package for bibliography
\bibliographystyle{ieeetr} %authordate1

\usepackage[utf8]{inputenc}
\usepackage[brazilian]{babel}
\usepackage[T1]{fontenc}

\usepackage{graphicx}

%% Change the following pdf author attribute name to your name.
\usepackage[linkcolor=blue,citecolor=blue,urlcolor=blue,colorlinks,pdfpagelabels,pdftitle={Bruno Cabral's Bachelor Thesis},pdfauthor={Bruno Cabral}]{hyperref}

\address{SALVADOR}

\universitypt{Universidade Federal da Bahia}
\universityen{Federal University of Bahia}

\departmentpt{Depertamento de Ciência da Computação}
\departmenten{Computer Science Department}

\programpt{Programa Multiinstitucional de Pós-graduação em Ciência da Computação}
\programen{Graduate in Computer Science}

\majorfieldpt{Ciência da Computação}
\majorfielden{Computer Science}

\title{Sistema de apoio à Pós graduação - UFBA}
\date{Outubro/2016}

\author{Victor de Azevedo Nunes}
\adviser{Ivan do Carmo Machado}

\begin{document}

\frontmatter
\frontpage
\presentationpage

\begin{dedicatory}
Eu dedico esta dissertação...
%I dedicate this dissertation to my family, girlfriend, friends and
%professors who gave me all necessary support to get here.
\end{dedicatory}

\acknowledgements
Meus agradecimentos...

\begin{epigraph}[]{Edward V Berard}
Walking on water and developing software from a specification are easy if both are frozen
\end{epigraph}

\resumo
% Escreva seu resumo no arquivo resumo.tex
\input{resumo}

\abstract
% Write your abstract in a file called abstract.tex
\input{abstract}

% Summary (tables of contents)
\tableofcontents

% List of figures
\listoffigures

% List of tables
\listoftables

% List of acronyms
% Acronyms manual: http://linorg.usp.br/CTAN/macros/latex/contrib/acronym/acronym.pdf
\listofacronyms
\input{acronyms}

% List of listings
%\lstlistoflistings

\mainmatter

\include{chapters/intro}
\include{chapters/referencial_teorico}

% \include{chapters/introduction/main}
% \include{chapters/background/main}
% \include{chapters/proposed_solution/main}
% \include{chapters/experiment/main}
% \include{chapters/conclusion/main}

%\bibliographystyle{natbib}
%\addcontentsline{toc}{chapter}{\bibliographytocname}
\bibliography{references}

% Appendix
\clearpage
\addappheadtotoc
\appendix
\appendixpage
% \include{appendix/experiment-instruments}

\end{document}
% %% RiSE Latex Template - version 0.5
%%
%% RiSE's latex template for thesis and dissertations
%% http://risetemplate.sourceforge.net
%%
%% (c) 2012 Yguaratã Cerqueira Cavalcanti (yguarata@gmail.com)
%%          Vinicius Cardoso Garcia (vinicius.garcia@gmail.com)
%%
%% This document was initially based on UFPEThesis template, from Paulo Gustavo
%% S. Fonseca.
%%
%% ACKNOWLEDGEMENTS
%%
%% We would like to thanks the RiSE's researchers community, the 
%% students from Federal University of Pernambuco, and other users that have
%% been contributing to this projects with comments and patches.
%%
%% GENERAL INSTRUCTIONS
%%
%% We strongly recommend you to compile your documents using pdflatex command.
%% It is also recommend use the texlipse plugin for Eclipse to edit your documents.
%%
%% Options for \documentclass command:
%%         * Idiom
%%           pt   - Portguese (default)
%%           en   - English
%%
%%         * Text type
%%           bsc  - B.Sc. Thesis
%%           msc  - M.Sc. Thesis (default)
%%           qual - PHD qualification (not tested yet)
%%           prop - PHD proposal (not tested yet)
%%           phd  - PHD thesis
%%
%%         * Media
%%           scr  - to eletronic version (PDF) / see the users guide
%%
%%         * Pagination
%%           oneside - unique face press
%%           twoside - two faces press
%%
%%		   * Line spacing
%%           singlespacing  - the same as using \linespread{1}
%%           onehalfspacing - the same as using \linespread{1.3}
%%           doublespacing  - the same as using \linespread{1.6}
%%
%% Reference commands. Use the following commands to make references in your
%% text:
%%          \figref  -- for Figure reference
%%          \tabref  -- for Table reference
%%          \eqnref  -- for equation reference
%%          \chapref -- for chapter reference
%%          \secref  -- for section reference
%%          \appref  -- for appendix reference
%%          \axiref  -- for axiom reference
%%          \conjref -- for conjecture reference
%%          \defref  -- for definition reference
%%          \lemref  -- for lemma reference
%%          \theoref -- for theorem reference
%%          \corref  -- for corollary reference
%%          \propref -- for proprosition reference
%%          \pgref   -- for page reference
%%
%%          Example: See \chapref{chap:introduction}. It will produce 
%%                   'See Chapter 1', in case of English language.

\documentclass[pt,twoside,onehalfspacing,bsc]{risethesis}

\usepackage[sort,square]{natbib} %package for bibliography
\bibliographystyle{ieeetr} %authordate1

\usepackage[utf8]{inputenc}
\usepackage[brazilian]{babel}
\usepackage[T1]{fontenc}

\usepackage{graphicx}

%% Change the following pdf author attribute name to your name.
\usepackage[linkcolor=blue,citecolor=blue,urlcolor=blue,colorlinks,pdfpagelabels,pdftitle={Bruno Cabral's Bachelor Thesis},pdfauthor={Bruno Cabral}]{hyperref}

\address{SALVADOR}

\universitypt{Universidade Federal da Bahia}
\universityen{Federal University of Bahia}

\departmentpt{Depertamento de Ciência da Computação}
\departmenten{Computer Science Department}

\programpt{Programa Multiinstitucional de Pós-graduação em Ciência da Computação}
\programen{Graduate in Computer Science}

\majorfieldpt{Ciência da Computação}
\majorfielden{Computer Science}

\title{Sistema de apoio à Pós graduação - UFBA}
\date{Outubro/2016}

\author{Victor de Azevedo Nunes}
\adviser{Ivan do Carmo Machado}

\begin{document}

\frontmatter
\frontpage
\presentationpage

\begin{dedicatory}
Eu dedico esta dissertação...
%I dedicate this dissertation to my family, girlfriend, friends and
%professors who gave me all necessary support to get here.
\end{dedicatory}

\acknowledgements
Meus agradecimentos...

\begin{epigraph}[]{Edward V Berard}
Walking on water and developing software from a specification are easy if both are frozen
\end{epigraph}

\resumo
% Escreva seu resumo no arquivo resumo.tex
\input{resumo}

\abstract
% Write your abstract in a file called abstract.tex
\input{abstract}

% Summary (tables of contents)
\tableofcontents

% List of figures
\listoffigures

% List of tables
\listoftables

% List of acronyms
% Acronyms manual: http://linorg.usp.br/CTAN/macros/latex/contrib/acronym/acronym.pdf
\listofacronyms
\input{acronyms}

% List of listings
%\lstlistoflistings

\mainmatter

\include{chapters/intro}
\include{chapters/referencial_teorico}

% \include{chapters/introduction/main}
% \include{chapters/background/main}
% \include{chapters/proposed_solution/main}
% \include{chapters/experiment/main}
% \include{chapters/conclusion/main}

%\bibliographystyle{natbib}
%\addcontentsline{toc}{chapter}{\bibliographytocname}
\bibliography{references}

% Appendix
\clearpage
\addappheadtotoc
\appendix
\appendixpage
% \include{appendix/experiment-instruments}

\end{document}
% %% RiSE Latex Template - version 0.5
%%
%% RiSE's latex template for thesis and dissertations
%% http://risetemplate.sourceforge.net
%%
%% (c) 2012 Yguaratã Cerqueira Cavalcanti (yguarata@gmail.com)
%%          Vinicius Cardoso Garcia (vinicius.garcia@gmail.com)
%%
%% This document was initially based on UFPEThesis template, from Paulo Gustavo
%% S. Fonseca.
%%
%% ACKNOWLEDGEMENTS
%%
%% We would like to thanks the RiSE's researchers community, the 
%% students from Federal University of Pernambuco, and other users that have
%% been contributing to this projects with comments and patches.
%%
%% GENERAL INSTRUCTIONS
%%
%% We strongly recommend you to compile your documents using pdflatex command.
%% It is also recommend use the texlipse plugin for Eclipse to edit your documents.
%%
%% Options for \documentclass command:
%%         * Idiom
%%           pt   - Portguese (default)
%%           en   - English
%%
%%         * Text type
%%           bsc  - B.Sc. Thesis
%%           msc  - M.Sc. Thesis (default)
%%           qual - PHD qualification (not tested yet)
%%           prop - PHD proposal (not tested yet)
%%           phd  - PHD thesis
%%
%%         * Media
%%           scr  - to eletronic version (PDF) / see the users guide
%%
%%         * Pagination
%%           oneside - unique face press
%%           twoside - two faces press
%%
%%		   * Line spacing
%%           singlespacing  - the same as using \linespread{1}
%%           onehalfspacing - the same as using \linespread{1.3}
%%           doublespacing  - the same as using \linespread{1.6}
%%
%% Reference commands. Use the following commands to make references in your
%% text:
%%          \figref  -- for Figure reference
%%          \tabref  -- for Table reference
%%          \eqnref  -- for equation reference
%%          \chapref -- for chapter reference
%%          \secref  -- for section reference
%%          \appref  -- for appendix reference
%%          \axiref  -- for axiom reference
%%          \conjref -- for conjecture reference
%%          \defref  -- for definition reference
%%          \lemref  -- for lemma reference
%%          \theoref -- for theorem reference
%%          \corref  -- for corollary reference
%%          \propref -- for proprosition reference
%%          \pgref   -- for page reference
%%
%%          Example: See \chapref{chap:introduction}. It will produce 
%%                   'See Chapter 1', in case of English language.

\documentclass[pt,twoside,onehalfspacing,bsc]{risethesis}

\usepackage[sort,square]{natbib} %package for bibliography
\bibliographystyle{ieeetr} %authordate1

\usepackage[utf8]{inputenc}
\usepackage[brazilian]{babel}
\usepackage[T1]{fontenc}

\usepackage{graphicx}

%% Change the following pdf author attribute name to your name.
\usepackage[linkcolor=blue,citecolor=blue,urlcolor=blue,colorlinks,pdfpagelabels,pdftitle={Bruno Cabral's Bachelor Thesis},pdfauthor={Bruno Cabral}]{hyperref}

\address{SALVADOR}

\universitypt{Universidade Federal da Bahia}
\universityen{Federal University of Bahia}

\departmentpt{Depertamento de Ciência da Computação}
\departmenten{Computer Science Department}

\programpt{Programa Multiinstitucional de Pós-graduação em Ciência da Computação}
\programen{Graduate in Computer Science}

\majorfieldpt{Ciência da Computação}
\majorfielden{Computer Science}

\title{Sistema de apoio à Pós graduação - UFBA}
\date{Outubro/2016}

\author{Victor de Azevedo Nunes}
\adviser{Ivan do Carmo Machado}

\begin{document}

\frontmatter
\frontpage
\presentationpage

\begin{dedicatory}
Eu dedico esta dissertação...
%I dedicate this dissertation to my family, girlfriend, friends and
%professors who gave me all necessary support to get here.
\end{dedicatory}

\acknowledgements
Meus agradecimentos...

\begin{epigraph}[]{Edward V Berard}
Walking on water and developing software from a specification are easy if both are frozen
\end{epigraph}

\resumo
% Escreva seu resumo no arquivo resumo.tex
\input{resumo}

\abstract
% Write your abstract in a file called abstract.tex
\input{abstract}

% Summary (tables of contents)
\tableofcontents

% List of figures
\listoffigures

% List of tables
\listoftables

% List of acronyms
% Acronyms manual: http://linorg.usp.br/CTAN/macros/latex/contrib/acronym/acronym.pdf
\listofacronyms
\input{acronyms}

% List of listings
%\lstlistoflistings

\mainmatter

\include{chapters/intro}
\include{chapters/referencial_teorico}

% \include{chapters/introduction/main}
% \include{chapters/background/main}
% \include{chapters/proposed_solution/main}
% \include{chapters/experiment/main}
% \include{chapters/conclusion/main}

%\bibliographystyle{natbib}
%\addcontentsline{toc}{chapter}{\bibliographytocname}
\bibliography{references}

% Appendix
\clearpage
\addappheadtotoc
\appendix
\appendixpage
% \include{appendix/experiment-instruments}

\end{document}
% %% RiSE Latex Template - version 0.5
%%
%% RiSE's latex template for thesis and dissertations
%% http://risetemplate.sourceforge.net
%%
%% (c) 2012 Yguaratã Cerqueira Cavalcanti (yguarata@gmail.com)
%%          Vinicius Cardoso Garcia (vinicius.garcia@gmail.com)
%%
%% This document was initially based on UFPEThesis template, from Paulo Gustavo
%% S. Fonseca.
%%
%% ACKNOWLEDGEMENTS
%%
%% We would like to thanks the RiSE's researchers community, the 
%% students from Federal University of Pernambuco, and other users that have
%% been contributing to this projects with comments and patches.
%%
%% GENERAL INSTRUCTIONS
%%
%% We strongly recommend you to compile your documents using pdflatex command.
%% It is also recommend use the texlipse plugin for Eclipse to edit your documents.
%%
%% Options for \documentclass command:
%%         * Idiom
%%           pt   - Portguese (default)
%%           en   - English
%%
%%         * Text type
%%           bsc  - B.Sc. Thesis
%%           msc  - M.Sc. Thesis (default)
%%           qual - PHD qualification (not tested yet)
%%           prop - PHD proposal (not tested yet)
%%           phd  - PHD thesis
%%
%%         * Media
%%           scr  - to eletronic version (PDF) / see the users guide
%%
%%         * Pagination
%%           oneside - unique face press
%%           twoside - two faces press
%%
%%		   * Line spacing
%%           singlespacing  - the same as using \linespread{1}
%%           onehalfspacing - the same as using \linespread{1.3}
%%           doublespacing  - the same as using \linespread{1.6}
%%
%% Reference commands. Use the following commands to make references in your
%% text:
%%          \figref  -- for Figure reference
%%          \tabref  -- for Table reference
%%          \eqnref  -- for equation reference
%%          \chapref -- for chapter reference
%%          \secref  -- for section reference
%%          \appref  -- for appendix reference
%%          \axiref  -- for axiom reference
%%          \conjref -- for conjecture reference
%%          \defref  -- for definition reference
%%          \lemref  -- for lemma reference
%%          \theoref -- for theorem reference
%%          \corref  -- for corollary reference
%%          \propref -- for proprosition reference
%%          \pgref   -- for page reference
%%
%%          Example: See \chapref{chap:introduction}. It will produce 
%%                   'See Chapter 1', in case of English language.

\documentclass[pt,twoside,onehalfspacing,bsc]{risethesis}

\usepackage[sort,square]{natbib} %package for bibliography
\bibliographystyle{ieeetr} %authordate1

\usepackage[utf8]{inputenc}
\usepackage[brazilian]{babel}
\usepackage[T1]{fontenc}

\usepackage{graphicx}

%% Change the following pdf author attribute name to your name.
\usepackage[linkcolor=blue,citecolor=blue,urlcolor=blue,colorlinks,pdfpagelabels,pdftitle={Bruno Cabral's Bachelor Thesis},pdfauthor={Bruno Cabral}]{hyperref}

\address{SALVADOR}

\universitypt{Universidade Federal da Bahia}
\universityen{Federal University of Bahia}

\departmentpt{Depertamento de Ciência da Computação}
\departmenten{Computer Science Department}

\programpt{Programa Multiinstitucional de Pós-graduação em Ciência da Computação}
\programen{Graduate in Computer Science}

\majorfieldpt{Ciência da Computação}
\majorfielden{Computer Science}

\title{Sistema de apoio à Pós graduação - UFBA}
\date{Outubro/2016}

\author{Victor de Azevedo Nunes}
\adviser{Ivan do Carmo Machado}

\begin{document}

\frontmatter
\frontpage
\presentationpage

\begin{dedicatory}
Eu dedico esta dissertação...
%I dedicate this dissertation to my family, girlfriend, friends and
%professors who gave me all necessary support to get here.
\end{dedicatory}

\acknowledgements
Meus agradecimentos...

\begin{epigraph}[]{Edward V Berard}
Walking on water and developing software from a specification are easy if both are frozen
\end{epigraph}

\resumo
% Escreva seu resumo no arquivo resumo.tex
\input{resumo}

\abstract
% Write your abstract in a file called abstract.tex
\input{abstract}

% Summary (tables of contents)
\tableofcontents

% List of figures
\listoffigures

% List of tables
\listoftables

% List of acronyms
% Acronyms manual: http://linorg.usp.br/CTAN/macros/latex/contrib/acronym/acronym.pdf
\listofacronyms
\input{acronyms}

% List of listings
%\lstlistoflistings

\mainmatter

\include{chapters/intro}
\include{chapters/referencial_teorico}

% \include{chapters/introduction/main}
% \include{chapters/background/main}
% \include{chapters/proposed_solution/main}
% \include{chapters/experiment/main}
% \include{chapters/conclusion/main}

%\bibliographystyle{natbib}
%\addcontentsline{toc}{chapter}{\bibliographytocname}
\bibliography{references}

% Appendix
\clearpage
\addappheadtotoc
\appendix
\appendixpage
% \include{appendix/experiment-instruments}

\end{document}
% %% RiSE Latex Template - version 0.5
%%
%% RiSE's latex template for thesis and dissertations
%% http://risetemplate.sourceforge.net
%%
%% (c) 2012 Yguaratã Cerqueira Cavalcanti (yguarata@gmail.com)
%%          Vinicius Cardoso Garcia (vinicius.garcia@gmail.com)
%%
%% This document was initially based on UFPEThesis template, from Paulo Gustavo
%% S. Fonseca.
%%
%% ACKNOWLEDGEMENTS
%%
%% We would like to thanks the RiSE's researchers community, the 
%% students from Federal University of Pernambuco, and other users that have
%% been contributing to this projects with comments and patches.
%%
%% GENERAL INSTRUCTIONS
%%
%% We strongly recommend you to compile your documents using pdflatex command.
%% It is also recommend use the texlipse plugin for Eclipse to edit your documents.
%%
%% Options for \documentclass command:
%%         * Idiom
%%           pt   - Portguese (default)
%%           en   - English
%%
%%         * Text type
%%           bsc  - B.Sc. Thesis
%%           msc  - M.Sc. Thesis (default)
%%           qual - PHD qualification (not tested yet)
%%           prop - PHD proposal (not tested yet)
%%           phd  - PHD thesis
%%
%%         * Media
%%           scr  - to eletronic version (PDF) / see the users guide
%%
%%         * Pagination
%%           oneside - unique face press
%%           twoside - two faces press
%%
%%		   * Line spacing
%%           singlespacing  - the same as using \linespread{1}
%%           onehalfspacing - the same as using \linespread{1.3}
%%           doublespacing  - the same as using \linespread{1.6}
%%
%% Reference commands. Use the following commands to make references in your
%% text:
%%          \figref  -- for Figure reference
%%          \tabref  -- for Table reference
%%          \eqnref  -- for equation reference
%%          \chapref -- for chapter reference
%%          \secref  -- for section reference
%%          \appref  -- for appendix reference
%%          \axiref  -- for axiom reference
%%          \conjref -- for conjecture reference
%%          \defref  -- for definition reference
%%          \lemref  -- for lemma reference
%%          \theoref -- for theorem reference
%%          \corref  -- for corollary reference
%%          \propref -- for proprosition reference
%%          \pgref   -- for page reference
%%
%%          Example: See \chapref{chap:introduction}. It will produce 
%%                   'See Chapter 1', in case of English language.

\documentclass[pt,twoside,onehalfspacing,bsc]{risethesis}

\usepackage[sort,square]{natbib} %package for bibliography
\bibliographystyle{ieeetr} %authordate1

\usepackage[utf8]{inputenc}
\usepackage[brazilian]{babel}
\usepackage[T1]{fontenc}

\usepackage{graphicx}

%% Change the following pdf author attribute name to your name.
\usepackage[linkcolor=blue,citecolor=blue,urlcolor=blue,colorlinks,pdfpagelabels,pdftitle={Bruno Cabral's Bachelor Thesis},pdfauthor={Bruno Cabral}]{hyperref}

\address{SALVADOR}

\universitypt{Universidade Federal da Bahia}
\universityen{Federal University of Bahia}

\departmentpt{Depertamento de Ciência da Computação}
\departmenten{Computer Science Department}

\programpt{Programa Multiinstitucional de Pós-graduação em Ciência da Computação}
\programen{Graduate in Computer Science}

\majorfieldpt{Ciência da Computação}
\majorfielden{Computer Science}

\title{Sistema de apoio à Pós graduação - UFBA}
\date{Outubro/2016}

\author{Victor de Azevedo Nunes}
\adviser{Ivan do Carmo Machado}

\begin{document}

\frontmatter
\frontpage
\presentationpage

\begin{dedicatory}
Eu dedico esta dissertação...
%I dedicate this dissertation to my family, girlfriend, friends and
%professors who gave me all necessary support to get here.
\end{dedicatory}

\acknowledgements
Meus agradecimentos...

\begin{epigraph}[]{Edward V Berard}
Walking on water and developing software from a specification are easy if both are frozen
\end{epigraph}

\resumo
% Escreva seu resumo no arquivo resumo.tex
\input{resumo}

\abstract
% Write your abstract in a file called abstract.tex
\input{abstract}

% Summary (tables of contents)
\tableofcontents

% List of figures
\listoffigures

% List of tables
\listoftables

% List of acronyms
% Acronyms manual: http://linorg.usp.br/CTAN/macros/latex/contrib/acronym/acronym.pdf
\listofacronyms
\input{acronyms}

% List of listings
%\lstlistoflistings

\mainmatter

\include{chapters/intro}
\include{chapters/referencial_teorico}

% \include{chapters/introduction/main}
% \include{chapters/background/main}
% \include{chapters/proposed_solution/main}
% \include{chapters/experiment/main}
% \include{chapters/conclusion/main}

%\bibliographystyle{natbib}
%\addcontentsline{toc}{chapter}{\bibliographytocname}
\bibliography{references}

% Appendix
\clearpage
\addappheadtotoc
\appendix
\appendixpage
% \include{appendix/experiment-instruments}

\end{document}

%\bibliographystyle{natbib}
%\addcontentsline{toc}{chapter}{\bibliographytocname}
\bibliography{references}

% Appendix
\clearpage
\addappheadtotoc
\appendix
\appendixpage
% \include{appendix/experiment-instruments}

\end{document}

%\bibliographystyle{natbib}
%\addcontentsline{toc}{chapter}{\bibliographytocname}
\bibliography{references}

% Appendix
\clearpage
\addappheadtotoc
\appendix
\appendixpage
% \include{appendix/experiment-instruments}

\end{document}
% %% RiSE Latex Template - version 0.5
%%
%% RiSE's latex template for thesis and dissertations
%% http://risetemplate.sourceforge.net
%%
%% (c) 2012 Yguaratã Cerqueira Cavalcanti (yguarata@gmail.com)
%%          Vinicius Cardoso Garcia (vinicius.garcia@gmail.com)
%%
%% This document was initially based on UFPEThesis template, from Paulo Gustavo
%% S. Fonseca.
%%
%% ACKNOWLEDGEMENTS
%%
%% We would like to thanks the RiSE's researchers community, the 
%% students from Federal University of Pernambuco, and other users that have
%% been contributing to this projects with comments and patches.
%%
%% GENERAL INSTRUCTIONS
%%
%% We strongly recommend you to compile your documents using pdflatex command.
%% It is also recommend use the texlipse plugin for Eclipse to edit your documents.
%%
%% Options for \documentclass command:
%%         * Idiom
%%           pt   - Portguese (default)
%%           en   - English
%%
%%         * Text type
%%           bsc  - B.Sc. Thesis
%%           msc  - M.Sc. Thesis (default)
%%           qual - PHD qualification (not tested yet)
%%           prop - PHD proposal (not tested yet)
%%           phd  - PHD thesis
%%
%%         * Media
%%           scr  - to eletronic version (PDF) / see the users guide
%%
%%         * Pagination
%%           oneside - unique face press
%%           twoside - two faces press
%%
%%		   * Line spacing
%%           singlespacing  - the same as using \linespread{1}
%%           onehalfspacing - the same as using \linespread{1.3}
%%           doublespacing  - the same as using \linespread{1.6}
%%
%% Reference commands. Use the following commands to make references in your
%% text:
%%          \figref  -- for Figure reference
%%          \tabref  -- for Table reference
%%          \eqnref  -- for equation reference
%%          \chapref -- for chapter reference
%%          \secref  -- for section reference
%%          \appref  -- for appendix reference
%%          \axiref  -- for axiom reference
%%          \conjref -- for conjecture reference
%%          \defref  -- for definition reference
%%          \lemref  -- for lemma reference
%%          \theoref -- for theorem reference
%%          \corref  -- for corollary reference
%%          \propref -- for proprosition reference
%%          \pgref   -- for page reference
%%
%%          Example: See \chapref{chap:introduction}. It will produce 
%%                   'See Chapter 1', in case of English language.

\documentclass[pt,twoside,onehalfspacing,bsc]{risethesis}

\usepackage[sort,square]{natbib} %package for bibliography
\bibliographystyle{ieeetr} %authordate1

\usepackage[utf8]{inputenc}
\usepackage[brazilian]{babel}
\usepackage[T1]{fontenc}

\usepackage{graphicx}

%% Change the following pdf author attribute name to your name.
\usepackage[linkcolor=blue,citecolor=blue,urlcolor=blue,colorlinks,pdfpagelabels,pdftitle={Bruno Cabral's Bachelor Thesis},pdfauthor={Bruno Cabral}]{hyperref}

\address{SALVADOR}

\universitypt{Universidade Federal da Bahia}
\universityen{Federal University of Bahia}

\departmentpt{Depertamento de Ciência da Computação}
\departmenten{Computer Science Department}

\programpt{Programa Multiinstitucional de Pós-graduação em Ciência da Computação}
\programen{Graduate in Computer Science}

\majorfieldpt{Ciência da Computação}
\majorfielden{Computer Science}

\title{Sistema de apoio à Pós graduação - UFBA}
\date{Outubro/2016}

\author{Victor de Azevedo Nunes}
\adviser{Ivan do Carmo Machado}

\begin{document}

\frontmatter
\frontpage
\presentationpage

\begin{dedicatory}
Eu dedico esta dissertação...
%I dedicate this dissertation to my family, girlfriend, friends and
%professors who gave me all necessary support to get here.
\end{dedicatory}

\acknowledgements
Meus agradecimentos...

\begin{epigraph}[]{Edward V Berard}
Walking on water and developing software from a specification are easy if both are frozen
\end{epigraph}

\resumo
% Escreva seu resumo no arquivo resumo.tex
Este trabalho de conclusão de curso utiliza elementos da engenharia de software para propor um software como serviço a fim de otimizar o processo de construção e manutenção dos softwares. Assim, o objetivo deste SaaS é gerenciar as permissões de acesso de sistemas cliente, provendo o reuso de software e padronizando as soluções. O sistema proposto neste trabalho encontra-se implementado e disponível no Github, e traz fundamentos sobre a arquitetura e tecnologias utilizadas, bem como avaliações sobre possibilidades de utilização do mesmo em diversos ambientes, como Web e mobile.

\begin{keywords}
Software; Reuso; SaaS; Web; PHP
\end{keywords}

\abstract
% Write your abstract in a file called abstract.tex
My abstract...

\begin{keywords}
key words...
\end{keywords}

% Summary (tables of contents)
\tableofcontents

% List of figures
\listoffigures

% List of tables
\listoftables

% List of acronyms
% Acronyms manual: http://linorg.usp.br/CTAN/macros/latex/contrib/acronym/acronym.pdf
\listofacronyms
\begin{acronym}[ACRONYM] 
% Change the word ACRONYM above to change the acronym column width.
% The column width is equals to the width of the word that you put.
% Read the manual about acronym package for more examples:
%   http://linorg.usp.br/CTAN/macros/latex/contrib/acronym/acronym.pdf

\acro{SPA}{Single Page Application}
\acro{JSON}{Javascript Object Notation}
\acro{PHP}{PHP: Hypertext Preprocessor}
\acro{SaaS}{Software as a Service}
\acro{ERP}{Enterprise Resource Planning}
\acro{QoS}{Quality of Service}
\acro{UML}{Unified Modeling Language}
\acro{MVC}{Model-View-Controller}
\acro{Ajax}{Asynchronous Javascript and XML}
\acro{HTML}{HyperText Markup Language}
\acro{CSS}{Cascading Style Sheets}
\acro{API}{Application Programming Interface}
\acro{DOM}{Document Object Model}
\acro{BPMN}{Business Process Model and Notation}
\acro{REST}{Representational State Transfer}

\end{acronym}

% List of listings
%\lstlistoflistings

\mainmatter

\chapter{Introdução}

\section{Motivação}

Organizar os procedimentos de um processo sempre nos traz vantagens. Apesar de no processo de implantação de um sistema, o mesmo burocratizar o processo, com o tempo temos o retorno da dedicação para a inserção dos dados. Com um certo volume de dados, é possível estruturar informações que num processo manual são difíceis de serem enxergadas. Assim, é possível depender menos das pessoas que organizam o processo, pois o legado de informações não estará mais somente na mente de alguns, mas sim documentado nos dados do sistema.

Além de colaborar na organização, também haverá uma grande colaboração no tempo gasto na gestão. Lidar com muitos papéis e confiar na mente humana para guardar informações, não é uma alternativa muito segura devido ao fato que as pessoas sempre estão sujeitas a sair do processo e levar contigo a experiência obtida. Experiência essa que faz com que os procedimentos sejam executados de forma mais eficiente. Entretanto, com um sistema inteligente, é possível auxiliar e tornar mais ágil a execução das tarefas.


\section{Problema}


De acordo com funcionários ligados ao o setor de pós graduação da UFBA, entrevistados a fim de um maior entendimento do cenário, apesar das semelhanças estruturais, a pós graduação gerida de forma diferencia da graduação. FULANO afirma que devido ao fato de não ter a mesma visibilidade, não tem acesso aos mesmos recursos de gestão acadêmica da graduação. O professores não executam somente atividades dentro da sala de aula, também tem diversas outras ocupações no setor. E muitos procedimentos realizados extra classe ainda se encontram sendo realizados de forma manual, estando mais vulnerável ao erro ou até mesmo à violação do processo. Também ocorre um grande desperdício de tempo pelos professores e gestores da área, devido ao diversos processos ainda realizados de forma manual, sem a devida documentação. Segundo FULANO, também entrevistado, esse tempo perdido implica numa redução da eficiência na sala de aula, pois o professor acaba por ter menos tempo disponível para o planejamento das atividades, o que gera impactos negativos aos alunos.


\section{Objetivos} %<o que deve ser feito/entregue>


Devido aos muitos processos sendo resolvidos de forma manual, propõe-se com solução um sistema moderno, arquitetado para ter funcionamento na web e com um módulo mobile, a fim de fornecer informações de forma rápida e eficiente para os professores através de notificações, já que o acesso à internet móvel é comum entre os possíveis usuários do sistema em questão.
O principal requisito para o sistema seria dispor recursos para reduzir o tempo desperdiçado pelos professores durante as atividades extra classe.


\section{Metodologia} %<como será feito | como resolver o problema apontado inicialmente>


%<analise de literatura | design | implementação | validação>
Baseando-se nas tecnologias gratuitas em alta no cenário atual do desenvolvimento web, dispomos de algumas opções eficientes para a implementação da solução. Dentre as possibilidades, considerando a facilidade para futura manutenção e continuidade do projeto, tende-se a optar por uma tecnologia popular. Como linguagem de programação, adota-se o PHP. A escolha é fundamentada de acordo com a pesquisa da RedMonk de 2015, que evidencia o uso das linguagens de programação de acordo com as discussões no StackOverflow e repositórios no GitHub. É possível constatar a popularidade do PHP no cenário atual com o gráfico da pesquisa citada, na qual o PHP é apresentado na terceira colocação, apenas atrás do lider JavaScript e do segundo colocado, o Java.

\begin{figure}
	\label{fig:graficoRedmonk}
	\includegraphics[width=1\textwidth]{img/grafico_redmonk}
	\caption{Ranking das liguagens de programação no Stack Overflow e Github}
\end{figure}


Ainda assim, para compor a interface do dado projeto, também ocorrerá o uso do líder JavaScript de forma intensa, provendo o elo com o as informações gerenciadas pelo PHP.


Entretanto, não seria inteligente desenvolver um sistema completo sem o auxílio de um framework. Dentre os frameworks disponíveis para PHP, hoje o destaque está com o Laravel, que se encontra no topo dentre os mais utilizados no momento. 


A WebHostFace, uma empresa de hospedagem, compilou várias estatísticas para criar um infográfico mostrando os frameworks PHP mais populares de 2015. Utilizando informações sobre os próprios clientes, o Google Trends, estatísticas de repositórios do GitHub e a pesquisa do SitePoint “Best PHP Frameworks 2015”, a WebHostFace elaborou o seguinte infográfico: 

\begin{figure}
	\label{fig:graficoWebhostface}
	\includegraphics[width=1\textwidth]{img/infografico_webhostface}
	\caption{Infográfico da WebhostFace, exibindo a popularidade dos Frameworks PHP em 2015}
\end{figure}

Assim, tem-se a evidência que o Laravel em 2015 teve a maior popularidade em projetos pessoais e tem a maior comunidade entre os concorrentes, o que o torna uma boa escolha para a escrita de um software que será continuado por terceiros.


Para elaborar os recursos de interface e integrar ao back-end PHP do sistema, será adotado o já conhecido AngularJS, ferramenta sólida e conhecida no aspecto em questão. 


Dados coletados via Google Trends, que propõe comparações entre termos pesquisados, revela a popularidade do AngularJs diante de alguns dos principais concorrentes. O gráfico abaixo evidencia o cenário.


%Como mostra a Figura \ref{fig:graficoGoogleTrendsFerramentasFront}. 
\begin{figure}
	\label{fig:graficoGoogleTrendsFerramentasFront}
	\includegraphics[width=1\textwidth]{img/grafico_ferramentas_front}
	\caption{Gráfico do Google Trends exibindo as pesquisas por ferramentas front-end}
\end{figure}


Junto ao Angular JS, será utilizada a agradável tendência de interface do Material Design da Google, que propõe layouts limpos e otimizados já conhecidos pelos usuários de smartphones Android. 


Para a elaboração da plataforma mobile do projeto, será utilizado o Ionic Framework, muito difundido e bastante pesquisado na área, o que fica evidenciado com o gráfico de pesquisbaixo, coletado via Google Trends buscando por frameworks de desenvolvimento híbrido mobile.


\begin{figure}
	\label{fig:graficoGoogleTrendsFerramentasHibridasMobile}
	\includegraphics[width=1\textwidth]{img/grafico_ferramentas_hibridas_mobile}
	\caption{Gráfico do Google Trends exibindo as pesquisas por Frameworks híbridos mobile}
\end{figure}	

Para layout da interface mobile, também será aplicado a tendência do Material Design, a fim de propor uma harmonia entre o módulo web e mobile para os usuários


\section{Resultados Esperados}


Como fruto de um sistema para pós-graduação da UFBA, espera-se que os professores tenham mais recursos para integrar as atividades e também prover melhores condições para acompanhamento da vida acadêmica dos alunos em questão. Também, que os novos colaboradores que entrarem no processo tenham facilidade de compreender o fluxo do setor ao navegar pelo sistema proposto.


\section{Fora de Escopo}


Interação com os alunos devido às complicações para realizar a integração com o sistema empregado na UFBA, gerenciado pela XXXXXX, o que causaria uma inviabilidade no projeto devido à necessidade de entrega do produto ser mais forte que o tempo necessário para executar o processo de obtenção de acesso ao sistema legado para realizar a integração.


\section{Estrutura do Trabalho}


<breve resumo sobre os capítulos do TCC>
\chapter{Referencial Teórico}


Projetar o desenvolvimento de um software requer muito planejamento, pois as falhas iniciais podem custar bastante caro ou até mesmo inviabilizar a continuação de um projeto. Assim, a escolha da arquitetura ideal para a aplicabilidade é essencial na concepção de um produto de software. 
De todo o modo, sempre busca-se fazer mais com menos. Diante de tal filosofia, temos neste capítulo, uma breve discussão sobre alguns elementos de projeto e arquitetura de software, a fim de contextualizar este trabalho de conclusão de curso.
O capítulo corrente é composto por quatro seções. A \ref{sec:saas} trata de Software como serviço, discutindo alguns elementos do contexto que são relevantes para o trabalho proposto. A \ref{sec:reuso}  discute sobre a empregabilidade do reuso de software. A \ref{sec:modularizacao} seção, trata sobre aspectos envolvidos na modularização dos softwares. Por fim, a \ref{sec:apps_web} aborda as aplicações web, discutindo sobre aspectos relevantes sobre a aplicação web que compõe este trabalho.


 \section{Software como serviço}\label{sec:saas}


Segundo La e Chun \citep{La2009Systematic}, o princípio da definição de Software como um Serviço (Sofware as a Service - SaaS) é um serviço complementar para aplicações da computação em nuvem (cloud computing). No entanto, as áreas não se confundem. SaaS deve ser entendido como um mecanismo de suporte às soluções existentes na cloud. Os SaaS existem justamente para maximizar o reuso de serviços repetidos e não centrais em uma aplicação remota.


Como vantagens, diversos fatores podem ser favoráveis para a adoção de um SaaS, como custo e manutenção dentre outros fatores aplicáveis a soluções específicas. Lechesaet al. \citep{LechesaSS11}, quantificam índices sobre os fatores determinantes para adoção ou não de um SaaS voltado para ERP na África do Sul. Os principais fatores determinantes para adoção desse mecanismo de software são sua fluidez quanto à rede e a segurança. Esses fatores estão presentes na aplicação desenvolvida neste trabalho de conclusão de curso.


Devido ao fato de ter um serviço constantemente na nuvem, fica o questionamento sobre a segurança da informação manipulada. Sabe-se que a vulnerabilidade na Web não é restrita ao SaaS, atingindo diversos âmbitos. Rai et al. \cite{journals/corr/RaiSM13} consideram que o avanço da computação em nuvem não é um problema apenas para os serviços Web do ponto de vista da segurança, pois muitos trabalhos na literatura mostram a área como mais um ponto de vulnerabilidade para diversos setores, a exemplo de infraestrutura. No mesmo artigo mencionado de Rai et al. \cite{journals/corr/RaiSM13}, também realizaram-se estudos exploratórios junto a empresas usuárias de serviços em computação em nuvem e consideram que a perspectiva de SaaS também pode fortalecer a segurança nas aplicações de cloud computing, pois o software de autenticação compartilhado por várias aplicações em nuvem, oferece uma melhor padronização e consequente facilidade de prevenção a erros de vulnerabilidade específicas de cada módulo da pesquisa. Esse ponto de vista é fundamenta para a compreensão de SaaS.


A arquitetura de armazenamento de dados de um Saas pode variar de acordo com a necessidade do contexto. Huixin \cite{7586486} descreve possíveis modelagens nesse sentido. Essa abordagem pode ser com um banco de dados único, fazendo com que diferentes clientes compartilhem o mesmo banco, diferindo os dados através de controle de usuário, ou isolando os diferentes clientes através de bancos de dados exclusivos para cada um. Esse fator também pode ser combinado com a arquitetura da aplicação, caso ofereça aplicação única para todos os clientes ou aplicação compartilhada. Diante das possíveis abordagens, a modelagem de dados do software pode ser decidida pela regra de negócio. Este trabalho optou por aplicação única e banco de dados compartilhado.

Devido ao conceito particular de obtenção de software, tanto pela visão do cliente como pela visão do vendedor, é necessário compreender os diversos aspectos que podem ser relevantes ao orçar um Saas. O recente trabalho de T. Kaur et al. \citep{6949281} orienta um modelo para compor o custo de um Saas. O custo total seria composto pelos fatores que dão suporte ao funcionamento do software. Tais fatores incluem infra-estrutura, configurabilidade, customização, parâmetros de QoS (Quality of service) como escalabilidade, disponibilidade, usabilidade, pontualidade e desempenho da resposta, portabilidade, custo total de propriedade e retorno do investimento. Esses fatores caracterizam o custo de forma eficaz, possibilitando ao fornecedor, prover um Serviço de acordo com a exigência do consumidor em vários pacotes de serviços.


O conceito de software como serviço se aplica neste trabalho de conclusão de curso, pois o mesmo estará disponível na web com alta disponibilidade adotando as características apresentadas para qualquer pessoa que desejar usá-lo. Assim, basta se cadastrar, configurá-lo e usar como um serviço, sem preocupação com a manutenção do mesmo.


\section{Reuso de software}\label{sec:reuso} %CRUISE BOOK CAPITULO 2


De acordo com o livro Software pratical reuse \citep{Ezran2002}, o reuso de software é a utilização de qualquer informação que um desenvolvedor pode necessitar no processo de criação de software. O livro de Basili e Rombach \citep{Basili1991} define reutilização de software como o uso de tudo o que está associado a um projeto de conhecimento.
Assim, o objetivo da reutilização de software é reciclar o design, código e outros componentes de um produto de software e assim reduzir o custo, o tempo e melhorar a qualidade do produto.
Segundo Keswani et al. \cite{6783445}, o componente reutilizável de software pode ser qualquer parte de seu desenvolvimento, como um fragmento de código, design, casos de teste, ou até mesmo a especificação de requisitos de uma funcionalidade do software. 

O reuso de software pode ter impacto positivo em diversos aspectos do software, vejamos alguns, conforme apresentados no C.R.U.I.S.E Book \citep{cruise2007} :

\begin{itemize}

\item Qualidade: As correções de erro tornam-se úteis em todos os locais em que ocorreu, padronizando e facilitando a manutenção.

\item Produtividade: O ganho de produtividade é alcançado devido ao menor número de artefatos desenvolvido. Isso resulta em menor esforço de teste e também análise e design, reduzindo custos.

\item Confiabilidade: A utilização de componentes bem testados aumenta a
confiança no software. Além disso, a utilização de um mesmo componente em vários sistemas, aumenta a possibilidade de detecção de erros e reforça a confiança no componente.

\item Redução do Esforço: A reutilização de software proporciona uma redução do tempo de desenvolvimento, o que reduz o tempo necessário para o produto ser disponibilizado no mercado para trazer rentabilidade.

\item Trabalho redundante e tempo de desenvolvimento: Desenvolver um sistema do
zero significa desenvolvimento redundante de muitos componentes, como requisitos, especificações, casos de uso, arquitetura, etc. Isso pode ser evitado quando estes estão disponíveis como componentes reutilizáveis e podem ser compartilhados, resultando em um processo de desenvolvimento otimizado.

\item Documentação: Embora a documentação seja muito importante para a
manutenção de um sistema, muitas vezes é negligenciada. A reutilização de componentes de software reduz a quantidade de documentação a ser escrita, entretanto depende da qualidade do que está escrito. Assim, apenas a estrutura do sistema e os novos artefatos desenvolvidos necessitam ser documentados.

\item Custo de manutenção: Menos defeitos e manutenções são esperados quando tem-se comprovada a qualidade dos componentes utilizados.

\item Tamanho da equipe: É comum haver casos em que a equipe de desenvolvimento sofre sobrecarga. Entretanto, dobrar o tamanho da equipe de desenvolvimento não necessariamente duplica produtividade. Se muitos componentes podem ser reutilizados, é possível desenvolver com equipes menores, levando a melhor comunicação e aumento da produtividade.

\end{itemize}

Apesar dos benefícios da reutilização de software, ela não é suficientemente aproveitada. Existem fatores que influenciam direta ou indiretamente na sua adoção. Esses fatores podem ser de aspecto gerencial, organizacional, econômico, conceitual ou técnico. Veremos a seguir alguns aspectos que podem gerar conflito com a cultura de reuso de software, segundo o C.R.U.I.S.E Book \citep{cruise2007} :
%(Sametinger, 1997). REVER

\begin{itemize}
	
\item Falta de apoio da gestão: Como a reutilização de software gera custos iniciais,
a medida pode não ser amplamente alcançada em uma organização sem o apoio de alto nível de gestão. Os gestores têm de ser informados sobre os custos iniciais e serem convencidos sobre economias futuras.

\item Gerenciamento do Projeto: Gerenciar projetos tradicionais é uma tarefa árdua, principalmente, os que praticam a reutilização de software. Utilizando a técnica em larga escala, tem-se impacto sobre todo o ciclo de vida do software.

\item Estruturas organizacionais inadequadas: As estruturas organizacionais devem
considerar diferentes necessidades que surgem quando a reutilização em larga escala está sendo adotada. Por exemplo, uma equipe particionada pode ser alocada somente para desenvolver, manter e certificar componentes reutilizáveis de software.

\item Incentivos de gestão: É comum a falta de incentivo para deixar os desenvolvedores gastarem tempo elaborando componentes do sistemas. A produtividade é muitas vezes medida apenas no tempo necessário para concluir um projeto. Assim, fazer qualquer trabalho além disso, embora benéfico para a empresa como um todo, diminui o seu sucesso. Mesmo quando os componentes reutilizáveis são utilizados, os benefícios obtidos são uma pequena fração do que poderia ser alcançado caso houvesse reutilização explícita, planejada e organizada.

\item Dificuldade de encontrar software reutilizável: Para reutilizar os componentes, devem existir formas eficientes de busca. Além disso, é importante ter um repositório bem organizado contendo componentes com um eficiente meio de acesso.

\item Não reutilização do software encontrado. O acesso fácil ao software existente
não necessariamente aumentar a reutilização. Os componentes reutilizáveis devem ser cuidadosamente especificados, projetados, implementados e documentados, pois em alguns casos, modificar e adaptar o código  pode ser mais custoso que a programação da funcionalidade necessária a partir do zero.

\item Modificação: É muito difícil encontrar um componente que funcione
exatamente da mesma maneira que queremos. Desta forma, são necessárias modificações e devem existir formas de determinar os seus efeitos sobre o componente.


\end{itemize}


A definição do reuso de software pode ser associada a este trabalho, pois ainda que como um serviço, estaremos adotando uma aplicação única para servir aos mais diversos softwares de uma empresa, por exemplo. Logo, ao adotar o projeto, pode ser interessante aplicar algumas métricas para tomar conhecimento de possíveis vantanjens como as citadas nessa seção, a exemplo de economia de tempo de desenvolvimento, e custo do projeto.


%Outra diretriz importante para a reutilização de software é reduzir o risco na criação de novos softwares. O risco tende a ser bastante reduzido se os componentes que estão sendo reutilizados têm as documentação, interfaces necessárias e devidamente testadas, fatores que contibruem para uma fácil integração.
%De acordo com Keswani et al. \citep{6783445}, para o reuso de software dar retornos apropriados, o processo deve ser sistemático e planejado. Qualquer organização que implemente a reutilização de software deve identificar os melhores métodos e estratégias de reutilização para obter a máxima produtividade. A reutilização de software ajuda a evitar software de engenharia a partir do zero, pois usa módulos de software existentes. A reutilização de software, embora seja uma tarefa difícil, especialmente para softwares antigos sem padrões de projeto, pode melhorar significativamente a produtividade e a qualidade de um produto de software. Embora a reutilização de software não seja um novo campo, ela pode dar grandes retornos em curto período de tempo.


\section{Modularização}\label{sec:modularizacao} %artigo de claudio pagina 222 introdução


%A modularidade vem desempenhando um papel predominante estágios emergentes das disciplinas de arquitetura de software [13]. Engenheiros de software consideram modularidade como princípio base na comparação entre arquiteturas alternativas  e arquitetura degeneração [9]. De fato, os engenheiros de software são incentivados a arquitecturas, baseando-se numa multiplicidade de mecanismos de modularidade disponíveis em: 
%(i) Linguagens de descrição de arquitetura (ADLs), como ACME [8], 
%(ii) catálogos de arquitetônicos [2, 13], e 
%(iii) conhecem bem princípios de alto nível, como interfaces de componentes estreitos, acoplamento arquitectónico reduzido e semelhantes.


Conforme é frisado por Wickramaarachchi e Lai \citep{7062705}, o conceito de modularização na indústria de software tem uma longa história e tem sido utilizado para melhorar o processo de desenvolvimento de software em diferentes estágios. Os principais conceitos por trás da modularização do software foram introduzidos por pesquisadores pioneiros há quarenta anos, com uma notável contribuição feita por Melvin Conway e David Parnas, que tem representação notável na engenharia de software.


Modularizar um software é um bom padrão a ser adotado. Segundo Wickramaarachchi e Lai \citep{7062705}, a modularização é importante na identificação de dependências e reduz as dificuldades diante de uma possível necessidade de grandes alterações. De uma perspectiva da engenharia de software, uma modularização geralmente tem várias vantagens, tais como: tornar a complexidade do software mais gerenciável, facilitar o trabalho paralelo e tornar o software mais maleável para acomodar o futuro incerto que um software pode ter. O objetivo final da modularização do software é aumentar a produtividade ea qualidade do software. Tal conceito encontra-se bastante difundido e estái incorporado em linguagens de programação e ferramentas de software. O presente trabalho favorece ao uso da modularização de um software e até mesmo pode ser considerado um módulo a ser acoplado a qualquer software, mediante a compatibilidade.


Ao realizar a adoção do trabalho proposto, fica bastante evidente a "responsabilidade" do mesmo no escopo do projeto. Assim, é possível tratá-lo como um módulo do projeto que o usa, mesmo que seja consumido como um serviço.


\section{Aplicações web}\label{sec:apps_web}


A popularidade das soluções Web aumentou exponencialmente na última década e todos os dias cresce o número de pessoas usuárias desse tipo de software. E seguindo um padrão próprio, Kumar et al. \citep{7813710} sugerem que para o desenvolvimento web, deve-se manter a prática eficaz de produzir diagramas UML. A abordagem baseada na web oferece uma maneira fácil e eficaz para gerenciar e controlar o processo de desenvolvimento por meio de artefatos de modelagem. Tal abordagem pode ser usada quando há uma exigência de lidar com mudanças muito rápidas e grandes em requisitos de forma muito eficaz em muito menos tempo, gerando assim um menor impacto. 


Para atender à fomentada demanda de aplicativos web, é necessário adotar métodos de desenvolvimentos que sejam ágeis, eficientes e de fácil manutenção. Yu Ping et al. \cite{1372143} propõem o uso do modelo MVC (Model, View e Controller) no desenvolvimento para softwares web. O modelo apresentado tornou-se um padrão popular e divide o software em camadas com propósito definido, tornando-o de mais fácil manutenção.


O Ajax (Asynchronous Javascript and XML) revolucionou a web. Conforme demonstrado por Yuping \citep{6845605}, ao usar a tecnologia Ajax, podemos enriquecer a experiência do usuário em aplicações baseadas em navegador de internet, e fornecer uma variedade de aplicações interativas para atender às necessidade de humanização das aplicações.
Os aplicativos Ajax em execução no navegador se comunicam com um servidor Web de forma assíncrona e atualizam apenas uma parte da página.


No artigo de Tesarik et al. \citep{4664349}, temos que o design de software SPA (Single page application) é uma maneira interessante de criar um software Web numa uma única página. Essa solução de página única sem navegação funciona apenas com base em técnicas dinâmicas e assíncronas, como o AJAX (citado no parágrafo anterior). No entanto, esta abordagem coloca o desenvolvedor antes de alguns desafios substanciais.
Para projetar a interface do usuário que mostra as informações, é importante projetar corretamente a tela para manipular os dados do aplicativo numa única página. O design da página deve ser elaborado para maximizar a decomposição da página em componentes distintos que encapsulam os principais casos de uso.
Também recomenda-se que para elaborar os artefatos visuais, explore os mais ricos recursos na implementação, como HTML5, JavaScript, Ajax, CSS3, e outras tecnologias que se apliquem. Entretanto, desenvolver uma rica interface com o uso de diversas tecnologias/frameworks pode ocasionar um esforço maior para explorar as possibilidades.
Tais componentes da interface de uma aplicação SPA normalmente são alimentados mediante o consumo de uma API Rest via requisições AJAX. Assim, é possível particionar as responsabilidades de processamento do software entre cliente e servidor.


Como é possível denotar no trabalho de Salvadori e Siqueire \citep{7195633}, REST é uma arquitetura muito popular para integração de aplicativos web, que permite compartilhar e reutilizar informações através de sistemas. Aplicações de grande escala baseadas em no REST deve ser implementadas utilizando estratégias e mecanismos para produzir sistemas que sejam fáceis de desenvolver, reutilizar e manter. As interfaces de integração fornecidas por esses sistemas, chamadas de Web API, têm uma influência importante nas características da implementação resultante, pois a sua forma de resposta deve ser casada com a interface que o consome.
 
 
Os elementos comentados nessa seção encontram-se presentes neste trabalho de conclusão de curso. Juntos, montam a estrutura tecnológica necessária aliada à arquitetura adotada. Esses elementos seguem tendências atuais dos softwares web, fazendo com que esse trabalho esteja composto por tecnologias modernas que estão com boa aceitação no mercado.

% %% RiSE Latex Template - version 0.5
%%
%% RiSE's latex template for thesis and dissertations
%% http://risetemplate.sourceforge.net
%%
%% (c) 2012 Yguaratã Cerqueira Cavalcanti (yguarata@gmail.com)
%%          Vinicius Cardoso Garcia (vinicius.garcia@gmail.com)
%%
%% This document was initially based on UFPEThesis template, from Paulo Gustavo
%% S. Fonseca.
%%
%% ACKNOWLEDGEMENTS
%%
%% We would like to thanks the RiSE's researchers community, the 
%% students from Federal University of Pernambuco, and other users that have
%% been contributing to this projects with comments and patches.
%%
%% GENERAL INSTRUCTIONS
%%
%% We strongly recommend you to compile your documents using pdflatex command.
%% It is also recommend use the texlipse plugin for Eclipse to edit your documents.
%%
%% Options for \documentclass command:
%%         * Idiom
%%           pt   - Portguese (default)
%%           en   - English
%%
%%         * Text type
%%           bsc  - B.Sc. Thesis
%%           msc  - M.Sc. Thesis (default)
%%           qual - PHD qualification (not tested yet)
%%           prop - PHD proposal (not tested yet)
%%           phd  - PHD thesis
%%
%%         * Media
%%           scr  - to eletronic version (PDF) / see the users guide
%%
%%         * Pagination
%%           oneside - unique face press
%%           twoside - two faces press
%%
%%		   * Line spacing
%%           singlespacing  - the same as using \linespread{1}
%%           onehalfspacing - the same as using \linespread{1.3}
%%           doublespacing  - the same as using \linespread{1.6}
%%
%% Reference commands. Use the following commands to make references in your
%% text:
%%          \figref  -- for Figure reference
%%          \tabref  -- for Table reference
%%          \eqnref  -- for equation reference
%%          \chapref -- for chapter reference
%%          \secref  -- for section reference
%%          \appref  -- for appendix reference
%%          \axiref  -- for axiom reference
%%          \conjref -- for conjecture reference
%%          \defref  -- for definition reference
%%          \lemref  -- for lemma reference
%%          \theoref -- for theorem reference
%%          \corref  -- for corollary reference
%%          \propref -- for proprosition reference
%%          \pgref   -- for page reference
%%
%%          Example: See \chapref{chap:introduction}. It will produce 
%%                   'See Chapter 1', in case of English language.

\documentclass[pt,twoside,onehalfspacing,bsc]{risethesis}

\usepackage[sort,square]{natbib} %package for bibliography
\bibliographystyle{ieeetr} %authordate1

\usepackage[utf8]{inputenc}
\usepackage[brazilian]{babel}
\usepackage[T1]{fontenc}

\usepackage{graphicx}

%% Change the following pdf author attribute name to your name.
\usepackage[linkcolor=blue,citecolor=blue,urlcolor=blue,colorlinks,pdfpagelabels,pdftitle={Bruno Cabral's Bachelor Thesis},pdfauthor={Bruno Cabral}]{hyperref}

\address{SALVADOR}

\universitypt{Universidade Federal da Bahia}
\universityen{Federal University of Bahia}

\departmentpt{Depertamento de Ciência da Computação}
\departmenten{Computer Science Department}

\programpt{Programa Multiinstitucional de Pós-graduação em Ciência da Computação}
\programen{Graduate in Computer Science}

\majorfieldpt{Ciência da Computação}
\majorfielden{Computer Science}

\title{Sistema de apoio à Pós graduação - UFBA}
\date{Outubro/2016}

\author{Victor de Azevedo Nunes}
\adviser{Ivan do Carmo Machado}

\begin{document}

\frontmatter
\frontpage
\presentationpage

\begin{dedicatory}
Eu dedico esta dissertação...
%I dedicate this dissertation to my family, girlfriend, friends and
%professors who gave me all necessary support to get here.
\end{dedicatory}

\acknowledgements
Meus agradecimentos...

\begin{epigraph}[]{Edward V Berard}
Walking on water and developing software from a specification are easy if both are frozen
\end{epigraph}

\resumo
% Escreva seu resumo no arquivo resumo.tex
Este trabalho de conclusão de curso utiliza elementos da engenharia de software para propor um software como serviço a fim de otimizar o processo de construção e manutenção dos softwares. Assim, o objetivo deste SaaS é gerenciar as permissões de acesso de sistemas cliente, provendo o reuso de software e padronizando as soluções. O sistema proposto neste trabalho encontra-se implementado e disponível no Github, e traz fundamentos sobre a arquitetura e tecnologias utilizadas, bem como avaliações sobre possibilidades de utilização do mesmo em diversos ambientes, como Web e mobile.

\begin{keywords}
Software; Reuso; SaaS; Web; PHP
\end{keywords}

\abstract
% Write your abstract in a file called abstract.tex
My abstract...

\begin{keywords}
key words...
\end{keywords}

% Summary (tables of contents)
\tableofcontents

% List of figures
\listoffigures

% List of tables
\listoftables

% List of acronyms
% Acronyms manual: http://linorg.usp.br/CTAN/macros/latex/contrib/acronym/acronym.pdf
\listofacronyms
\begin{acronym}[ACRONYM] 
% Change the word ACRONYM above to change the acronym column width.
% The column width is equals to the width of the word that you put.
% Read the manual about acronym package for more examples:
%   http://linorg.usp.br/CTAN/macros/latex/contrib/acronym/acronym.pdf

\acro{SPA}{Single Page Application}
\acro{JSON}{Javascript Object Notation}
\acro{PHP}{PHP: Hypertext Preprocessor}
\acro{SaaS}{Software as a Service}
\acro{ERP}{Enterprise Resource Planning}
\acro{QoS}{Quality of Service}
\acro{UML}{Unified Modeling Language}
\acro{MVC}{Model-View-Controller}
\acro{Ajax}{Asynchronous Javascript and XML}
\acro{HTML}{HyperText Markup Language}
\acro{CSS}{Cascading Style Sheets}
\acro{API}{Application Programming Interface}
\acro{DOM}{Document Object Model}
\acro{BPMN}{Business Process Model and Notation}
\acro{REST}{Representational State Transfer}

\end{acronym}

% List of listings
%\lstlistoflistings

\mainmatter

\chapter{Introdução}

\section{Motivação}

Organizar os procedimentos de um processo sempre nos traz vantagens. Apesar de no processo de implantação de um sistema, o mesmo burocratizar o processo, com o tempo temos o retorno da dedicação para a inserção dos dados. Com um certo volume de dados, é possível estruturar informações que num processo manual são difíceis de serem enxergadas. Assim, é possível depender menos das pessoas que organizam o processo, pois o legado de informações não estará mais somente na mente de alguns, mas sim documentado nos dados do sistema.

Além de colaborar na organização, também haverá uma grande colaboração no tempo gasto na gestão. Lidar com muitos papéis e confiar na mente humana para guardar informações, não é uma alternativa muito segura devido ao fato que as pessoas sempre estão sujeitas a sair do processo e levar contigo a experiência obtida. Experiência essa que faz com que os procedimentos sejam executados de forma mais eficiente. Entretanto, com um sistema inteligente, é possível auxiliar e tornar mais ágil a execução das tarefas.


\section{Problema}


De acordo com funcionários ligados ao o setor de pós graduação da UFBA, entrevistados a fim de um maior entendimento do cenário, apesar das semelhanças estruturais, a pós graduação gerida de forma diferencia da graduação. FULANO afirma que devido ao fato de não ter a mesma visibilidade, não tem acesso aos mesmos recursos de gestão acadêmica da graduação. O professores não executam somente atividades dentro da sala de aula, também tem diversas outras ocupações no setor. E muitos procedimentos realizados extra classe ainda se encontram sendo realizados de forma manual, estando mais vulnerável ao erro ou até mesmo à violação do processo. Também ocorre um grande desperdício de tempo pelos professores e gestores da área, devido ao diversos processos ainda realizados de forma manual, sem a devida documentação. Segundo FULANO, também entrevistado, esse tempo perdido implica numa redução da eficiência na sala de aula, pois o professor acaba por ter menos tempo disponível para o planejamento das atividades, o que gera impactos negativos aos alunos.


\section{Objetivos} %<o que deve ser feito/entregue>


Devido aos muitos processos sendo resolvidos de forma manual, propõe-se com solução um sistema moderno, arquitetado para ter funcionamento na web e com um módulo mobile, a fim de fornecer informações de forma rápida e eficiente para os professores através de notificações, já que o acesso à internet móvel é comum entre os possíveis usuários do sistema em questão.
O principal requisito para o sistema seria dispor recursos para reduzir o tempo desperdiçado pelos professores durante as atividades extra classe.


\section{Metodologia} %<como será feito | como resolver o problema apontado inicialmente>


%<analise de literatura | design | implementação | validação>
Baseando-se nas tecnologias gratuitas em alta no cenário atual do desenvolvimento web, dispomos de algumas opções eficientes para a implementação da solução. Dentre as possibilidades, considerando a facilidade para futura manutenção e continuidade do projeto, tende-se a optar por uma tecnologia popular. Como linguagem de programação, adota-se o PHP. A escolha é fundamentada de acordo com a pesquisa da RedMonk de 2015, que evidencia o uso das linguagens de programação de acordo com as discussões no StackOverflow e repositórios no GitHub. É possível constatar a popularidade do PHP no cenário atual com o gráfico da pesquisa citada, na qual o PHP é apresentado na terceira colocação, apenas atrás do lider JavaScript e do segundo colocado, o Java.

\begin{figure}
	\label{fig:graficoRedmonk}
	\includegraphics[width=1\textwidth]{img/grafico_redmonk}
	\caption{Ranking das liguagens de programação no Stack Overflow e Github}
\end{figure}


Ainda assim, para compor a interface do dado projeto, também ocorrerá o uso do líder JavaScript de forma intensa, provendo o elo com o as informações gerenciadas pelo PHP.


Entretanto, não seria inteligente desenvolver um sistema completo sem o auxílio de um framework. Dentre os frameworks disponíveis para PHP, hoje o destaque está com o Laravel, que se encontra no topo dentre os mais utilizados no momento. 


A WebHostFace, uma empresa de hospedagem, compilou várias estatísticas para criar um infográfico mostrando os frameworks PHP mais populares de 2015. Utilizando informações sobre os próprios clientes, o Google Trends, estatísticas de repositórios do GitHub e a pesquisa do SitePoint “Best PHP Frameworks 2015”, a WebHostFace elaborou o seguinte infográfico: 

\begin{figure}
	\label{fig:graficoWebhostface}
	\includegraphics[width=1\textwidth]{img/infografico_webhostface}
	\caption{Infográfico da WebhostFace, exibindo a popularidade dos Frameworks PHP em 2015}
\end{figure}

Assim, tem-se a evidência que o Laravel em 2015 teve a maior popularidade em projetos pessoais e tem a maior comunidade entre os concorrentes, o que o torna uma boa escolha para a escrita de um software que será continuado por terceiros.


Para elaborar os recursos de interface e integrar ao back-end PHP do sistema, será adotado o já conhecido AngularJS, ferramenta sólida e conhecida no aspecto em questão. 


Dados coletados via Google Trends, que propõe comparações entre termos pesquisados, revela a popularidade do AngularJs diante de alguns dos principais concorrentes. O gráfico abaixo evidencia o cenário.


%Como mostra a Figura \ref{fig:graficoGoogleTrendsFerramentasFront}. 
\begin{figure}
	\label{fig:graficoGoogleTrendsFerramentasFront}
	\includegraphics[width=1\textwidth]{img/grafico_ferramentas_front}
	\caption{Gráfico do Google Trends exibindo as pesquisas por ferramentas front-end}
\end{figure}


Junto ao Angular JS, será utilizada a agradável tendência de interface do Material Design da Google, que propõe layouts limpos e otimizados já conhecidos pelos usuários de smartphones Android. 


Para a elaboração da plataforma mobile do projeto, será utilizado o Ionic Framework, muito difundido e bastante pesquisado na área, o que fica evidenciado com o gráfico de pesquisbaixo, coletado via Google Trends buscando por frameworks de desenvolvimento híbrido mobile.


\begin{figure}
	\label{fig:graficoGoogleTrendsFerramentasHibridasMobile}
	\includegraphics[width=1\textwidth]{img/grafico_ferramentas_hibridas_mobile}
	\caption{Gráfico do Google Trends exibindo as pesquisas por Frameworks híbridos mobile}
\end{figure}	

Para layout da interface mobile, também será aplicado a tendência do Material Design, a fim de propor uma harmonia entre o módulo web e mobile para os usuários


\section{Resultados Esperados}


Como fruto de um sistema para pós-graduação da UFBA, espera-se que os professores tenham mais recursos para integrar as atividades e também prover melhores condições para acompanhamento da vida acadêmica dos alunos em questão. Também, que os novos colaboradores que entrarem no processo tenham facilidade de compreender o fluxo do setor ao navegar pelo sistema proposto.


\section{Fora de Escopo}


Interação com os alunos devido às complicações para realizar a integração com o sistema empregado na UFBA, gerenciado pela XXXXXX, o que causaria uma inviabilidade no projeto devido à necessidade de entrega do produto ser mais forte que o tempo necessário para executar o processo de obtenção de acesso ao sistema legado para realizar a integração.


\section{Estrutura do Trabalho}


<breve resumo sobre os capítulos do TCC>
\chapter{Referencial Teórico}


Projetar o desenvolvimento de um software requer muito planejamento, pois as falhas iniciais podem custar bastante caro ou até mesmo inviabilizar a continuação de um projeto. Assim, a escolha da arquitetura ideal para a aplicabilidade é essencial na concepção de um produto de software. 
De todo o modo, sempre busca-se fazer mais com menos. Diante de tal filosofia, temos neste capítulo, uma breve discussão sobre alguns elementos de projeto e arquitetura de software, a fim de contextualizar este trabalho de conclusão de curso.
O capítulo corrente é composto por quatro seções. A \ref{sec:saas} trata de Software como serviço, discutindo alguns elementos do contexto que são relevantes para o trabalho proposto. A \ref{sec:reuso}  discute sobre a empregabilidade do reuso de software. A \ref{sec:modularizacao} seção, trata sobre aspectos envolvidos na modularização dos softwares. Por fim, a \ref{sec:apps_web} aborda as aplicações web, discutindo sobre aspectos relevantes sobre a aplicação web que compõe este trabalho.


 \section{Software como serviço}\label{sec:saas}


Segundo La e Chun \citep{La2009Systematic}, o princípio da definição de Software como um Serviço (Sofware as a Service - SaaS) é um serviço complementar para aplicações da computação em nuvem (cloud computing). No entanto, as áreas não se confundem. SaaS deve ser entendido como um mecanismo de suporte às soluções existentes na cloud. Os SaaS existem justamente para maximizar o reuso de serviços repetidos e não centrais em uma aplicação remota.


Como vantagens, diversos fatores podem ser favoráveis para a adoção de um SaaS, como custo e manutenção dentre outros fatores aplicáveis a soluções específicas. Lechesaet al. \citep{LechesaSS11}, quantificam índices sobre os fatores determinantes para adoção ou não de um SaaS voltado para ERP na África do Sul. Os principais fatores determinantes para adoção desse mecanismo de software são sua fluidez quanto à rede e a segurança. Esses fatores estão presentes na aplicação desenvolvida neste trabalho de conclusão de curso.


Devido ao fato de ter um serviço constantemente na nuvem, fica o questionamento sobre a segurança da informação manipulada. Sabe-se que a vulnerabilidade na Web não é restrita ao SaaS, atingindo diversos âmbitos. Rai et al. \cite{journals/corr/RaiSM13} consideram que o avanço da computação em nuvem não é um problema apenas para os serviços Web do ponto de vista da segurança, pois muitos trabalhos na literatura mostram a área como mais um ponto de vulnerabilidade para diversos setores, a exemplo de infraestrutura. No mesmo artigo mencionado de Rai et al. \cite{journals/corr/RaiSM13}, também realizaram-se estudos exploratórios junto a empresas usuárias de serviços em computação em nuvem e consideram que a perspectiva de SaaS também pode fortalecer a segurança nas aplicações de cloud computing, pois o software de autenticação compartilhado por várias aplicações em nuvem, oferece uma melhor padronização e consequente facilidade de prevenção a erros de vulnerabilidade específicas de cada módulo da pesquisa. Esse ponto de vista é fundamenta para a compreensão de SaaS.


A arquitetura de armazenamento de dados de um Saas pode variar de acordo com a necessidade do contexto. Huixin \cite{7586486} descreve possíveis modelagens nesse sentido. Essa abordagem pode ser com um banco de dados único, fazendo com que diferentes clientes compartilhem o mesmo banco, diferindo os dados através de controle de usuário, ou isolando os diferentes clientes através de bancos de dados exclusivos para cada um. Esse fator também pode ser combinado com a arquitetura da aplicação, caso ofereça aplicação única para todos os clientes ou aplicação compartilhada. Diante das possíveis abordagens, a modelagem de dados do software pode ser decidida pela regra de negócio. Este trabalho optou por aplicação única e banco de dados compartilhado.

Devido ao conceito particular de obtenção de software, tanto pela visão do cliente como pela visão do vendedor, é necessário compreender os diversos aspectos que podem ser relevantes ao orçar um Saas. O recente trabalho de T. Kaur et al. \citep{6949281} orienta um modelo para compor o custo de um Saas. O custo total seria composto pelos fatores que dão suporte ao funcionamento do software. Tais fatores incluem infra-estrutura, configurabilidade, customização, parâmetros de QoS (Quality of service) como escalabilidade, disponibilidade, usabilidade, pontualidade e desempenho da resposta, portabilidade, custo total de propriedade e retorno do investimento. Esses fatores caracterizam o custo de forma eficaz, possibilitando ao fornecedor, prover um Serviço de acordo com a exigência do consumidor em vários pacotes de serviços.


O conceito de software como serviço se aplica neste trabalho de conclusão de curso, pois o mesmo estará disponível na web com alta disponibilidade adotando as características apresentadas para qualquer pessoa que desejar usá-lo. Assim, basta se cadastrar, configurá-lo e usar como um serviço, sem preocupação com a manutenção do mesmo.


\section{Reuso de software}\label{sec:reuso} %CRUISE BOOK CAPITULO 2


De acordo com o livro Software pratical reuse \citep{Ezran2002}, o reuso de software é a utilização de qualquer informação que um desenvolvedor pode necessitar no processo de criação de software. O livro de Basili e Rombach \citep{Basili1991} define reutilização de software como o uso de tudo o que está associado a um projeto de conhecimento.
Assim, o objetivo da reutilização de software é reciclar o design, código e outros componentes de um produto de software e assim reduzir o custo, o tempo e melhorar a qualidade do produto.
Segundo Keswani et al. \cite{6783445}, o componente reutilizável de software pode ser qualquer parte de seu desenvolvimento, como um fragmento de código, design, casos de teste, ou até mesmo a especificação de requisitos de uma funcionalidade do software. 

O reuso de software pode ter impacto positivo em diversos aspectos do software, vejamos alguns, conforme apresentados no C.R.U.I.S.E Book \citep{cruise2007} :

\begin{itemize}

\item Qualidade: As correções de erro tornam-se úteis em todos os locais em que ocorreu, padronizando e facilitando a manutenção.

\item Produtividade: O ganho de produtividade é alcançado devido ao menor número de artefatos desenvolvido. Isso resulta em menor esforço de teste e também análise e design, reduzindo custos.

\item Confiabilidade: A utilização de componentes bem testados aumenta a
confiança no software. Além disso, a utilização de um mesmo componente em vários sistemas, aumenta a possibilidade de detecção de erros e reforça a confiança no componente.

\item Redução do Esforço: A reutilização de software proporciona uma redução do tempo de desenvolvimento, o que reduz o tempo necessário para o produto ser disponibilizado no mercado para trazer rentabilidade.

\item Trabalho redundante e tempo de desenvolvimento: Desenvolver um sistema do
zero significa desenvolvimento redundante de muitos componentes, como requisitos, especificações, casos de uso, arquitetura, etc. Isso pode ser evitado quando estes estão disponíveis como componentes reutilizáveis e podem ser compartilhados, resultando em um processo de desenvolvimento otimizado.

\item Documentação: Embora a documentação seja muito importante para a
manutenção de um sistema, muitas vezes é negligenciada. A reutilização de componentes de software reduz a quantidade de documentação a ser escrita, entretanto depende da qualidade do que está escrito. Assim, apenas a estrutura do sistema e os novos artefatos desenvolvidos necessitam ser documentados.

\item Custo de manutenção: Menos defeitos e manutenções são esperados quando tem-se comprovada a qualidade dos componentes utilizados.

\item Tamanho da equipe: É comum haver casos em que a equipe de desenvolvimento sofre sobrecarga. Entretanto, dobrar o tamanho da equipe de desenvolvimento não necessariamente duplica produtividade. Se muitos componentes podem ser reutilizados, é possível desenvolver com equipes menores, levando a melhor comunicação e aumento da produtividade.

\end{itemize}

Apesar dos benefícios da reutilização de software, ela não é suficientemente aproveitada. Existem fatores que influenciam direta ou indiretamente na sua adoção. Esses fatores podem ser de aspecto gerencial, organizacional, econômico, conceitual ou técnico. Veremos a seguir alguns aspectos que podem gerar conflito com a cultura de reuso de software, segundo o C.R.U.I.S.E Book \citep{cruise2007} :
%(Sametinger, 1997). REVER

\begin{itemize}
	
\item Falta de apoio da gestão: Como a reutilização de software gera custos iniciais,
a medida pode não ser amplamente alcançada em uma organização sem o apoio de alto nível de gestão. Os gestores têm de ser informados sobre os custos iniciais e serem convencidos sobre economias futuras.

\item Gerenciamento do Projeto: Gerenciar projetos tradicionais é uma tarefa árdua, principalmente, os que praticam a reutilização de software. Utilizando a técnica em larga escala, tem-se impacto sobre todo o ciclo de vida do software.

\item Estruturas organizacionais inadequadas: As estruturas organizacionais devem
considerar diferentes necessidades que surgem quando a reutilização em larga escala está sendo adotada. Por exemplo, uma equipe particionada pode ser alocada somente para desenvolver, manter e certificar componentes reutilizáveis de software.

\item Incentivos de gestão: É comum a falta de incentivo para deixar os desenvolvedores gastarem tempo elaborando componentes do sistemas. A produtividade é muitas vezes medida apenas no tempo necessário para concluir um projeto. Assim, fazer qualquer trabalho além disso, embora benéfico para a empresa como um todo, diminui o seu sucesso. Mesmo quando os componentes reutilizáveis são utilizados, os benefícios obtidos são uma pequena fração do que poderia ser alcançado caso houvesse reutilização explícita, planejada e organizada.

\item Dificuldade de encontrar software reutilizável: Para reutilizar os componentes, devem existir formas eficientes de busca. Além disso, é importante ter um repositório bem organizado contendo componentes com um eficiente meio de acesso.

\item Não reutilização do software encontrado. O acesso fácil ao software existente
não necessariamente aumentar a reutilização. Os componentes reutilizáveis devem ser cuidadosamente especificados, projetados, implementados e documentados, pois em alguns casos, modificar e adaptar o código  pode ser mais custoso que a programação da funcionalidade necessária a partir do zero.

\item Modificação: É muito difícil encontrar um componente que funcione
exatamente da mesma maneira que queremos. Desta forma, são necessárias modificações e devem existir formas de determinar os seus efeitos sobre o componente.


\end{itemize}


A definição do reuso de software pode ser associada a este trabalho, pois ainda que como um serviço, estaremos adotando uma aplicação única para servir aos mais diversos softwares de uma empresa, por exemplo. Logo, ao adotar o projeto, pode ser interessante aplicar algumas métricas para tomar conhecimento de possíveis vantanjens como as citadas nessa seção, a exemplo de economia de tempo de desenvolvimento, e custo do projeto.


%Outra diretriz importante para a reutilização de software é reduzir o risco na criação de novos softwares. O risco tende a ser bastante reduzido se os componentes que estão sendo reutilizados têm as documentação, interfaces necessárias e devidamente testadas, fatores que contibruem para uma fácil integração.
%De acordo com Keswani et al. \citep{6783445}, para o reuso de software dar retornos apropriados, o processo deve ser sistemático e planejado. Qualquer organização que implemente a reutilização de software deve identificar os melhores métodos e estratégias de reutilização para obter a máxima produtividade. A reutilização de software ajuda a evitar software de engenharia a partir do zero, pois usa módulos de software existentes. A reutilização de software, embora seja uma tarefa difícil, especialmente para softwares antigos sem padrões de projeto, pode melhorar significativamente a produtividade e a qualidade de um produto de software. Embora a reutilização de software não seja um novo campo, ela pode dar grandes retornos em curto período de tempo.


\section{Modularização}\label{sec:modularizacao} %artigo de claudio pagina 222 introdução


%A modularidade vem desempenhando um papel predominante estágios emergentes das disciplinas de arquitetura de software [13]. Engenheiros de software consideram modularidade como princípio base na comparação entre arquiteturas alternativas  e arquitetura degeneração [9]. De fato, os engenheiros de software são incentivados a arquitecturas, baseando-se numa multiplicidade de mecanismos de modularidade disponíveis em: 
%(i) Linguagens de descrição de arquitetura (ADLs), como ACME [8], 
%(ii) catálogos de arquitetônicos [2, 13], e 
%(iii) conhecem bem princípios de alto nível, como interfaces de componentes estreitos, acoplamento arquitectónico reduzido e semelhantes.


Conforme é frisado por Wickramaarachchi e Lai \citep{7062705}, o conceito de modularização na indústria de software tem uma longa história e tem sido utilizado para melhorar o processo de desenvolvimento de software em diferentes estágios. Os principais conceitos por trás da modularização do software foram introduzidos por pesquisadores pioneiros há quarenta anos, com uma notável contribuição feita por Melvin Conway e David Parnas, que tem representação notável na engenharia de software.


Modularizar um software é um bom padrão a ser adotado. Segundo Wickramaarachchi e Lai \citep{7062705}, a modularização é importante na identificação de dependências e reduz as dificuldades diante de uma possível necessidade de grandes alterações. De uma perspectiva da engenharia de software, uma modularização geralmente tem várias vantagens, tais como: tornar a complexidade do software mais gerenciável, facilitar o trabalho paralelo e tornar o software mais maleável para acomodar o futuro incerto que um software pode ter. O objetivo final da modularização do software é aumentar a produtividade ea qualidade do software. Tal conceito encontra-se bastante difundido e estái incorporado em linguagens de programação e ferramentas de software. O presente trabalho favorece ao uso da modularização de um software e até mesmo pode ser considerado um módulo a ser acoplado a qualquer software, mediante a compatibilidade.


Ao realizar a adoção do trabalho proposto, fica bastante evidente a "responsabilidade" do mesmo no escopo do projeto. Assim, é possível tratá-lo como um módulo do projeto que o usa, mesmo que seja consumido como um serviço.


\section{Aplicações web}\label{sec:apps_web}


A popularidade das soluções Web aumentou exponencialmente na última década e todos os dias cresce o número de pessoas usuárias desse tipo de software. E seguindo um padrão próprio, Kumar et al. \citep{7813710} sugerem que para o desenvolvimento web, deve-se manter a prática eficaz de produzir diagramas UML. A abordagem baseada na web oferece uma maneira fácil e eficaz para gerenciar e controlar o processo de desenvolvimento por meio de artefatos de modelagem. Tal abordagem pode ser usada quando há uma exigência de lidar com mudanças muito rápidas e grandes em requisitos de forma muito eficaz em muito menos tempo, gerando assim um menor impacto. 


Para atender à fomentada demanda de aplicativos web, é necessário adotar métodos de desenvolvimentos que sejam ágeis, eficientes e de fácil manutenção. Yu Ping et al. \cite{1372143} propõem o uso do modelo MVC (Model, View e Controller) no desenvolvimento para softwares web. O modelo apresentado tornou-se um padrão popular e divide o software em camadas com propósito definido, tornando-o de mais fácil manutenção.


O Ajax (Asynchronous Javascript and XML) revolucionou a web. Conforme demonstrado por Yuping \citep{6845605}, ao usar a tecnologia Ajax, podemos enriquecer a experiência do usuário em aplicações baseadas em navegador de internet, e fornecer uma variedade de aplicações interativas para atender às necessidade de humanização das aplicações.
Os aplicativos Ajax em execução no navegador se comunicam com um servidor Web de forma assíncrona e atualizam apenas uma parte da página.


No artigo de Tesarik et al. \citep{4664349}, temos que o design de software SPA (Single page application) é uma maneira interessante de criar um software Web numa uma única página. Essa solução de página única sem navegação funciona apenas com base em técnicas dinâmicas e assíncronas, como o AJAX (citado no parágrafo anterior). No entanto, esta abordagem coloca o desenvolvedor antes de alguns desafios substanciais.
Para projetar a interface do usuário que mostra as informações, é importante projetar corretamente a tela para manipular os dados do aplicativo numa única página. O design da página deve ser elaborado para maximizar a decomposição da página em componentes distintos que encapsulam os principais casos de uso.
Também recomenda-se que para elaborar os artefatos visuais, explore os mais ricos recursos na implementação, como HTML5, JavaScript, Ajax, CSS3, e outras tecnologias que se apliquem. Entretanto, desenvolver uma rica interface com o uso de diversas tecnologias/frameworks pode ocasionar um esforço maior para explorar as possibilidades.
Tais componentes da interface de uma aplicação SPA normalmente são alimentados mediante o consumo de uma API Rest via requisições AJAX. Assim, é possível particionar as responsabilidades de processamento do software entre cliente e servidor.


Como é possível denotar no trabalho de Salvadori e Siqueire \citep{7195633}, REST é uma arquitetura muito popular para integração de aplicativos web, que permite compartilhar e reutilizar informações através de sistemas. Aplicações de grande escala baseadas em no REST deve ser implementadas utilizando estratégias e mecanismos para produzir sistemas que sejam fáceis de desenvolver, reutilizar e manter. As interfaces de integração fornecidas por esses sistemas, chamadas de Web API, têm uma influência importante nas características da implementação resultante, pois a sua forma de resposta deve ser casada com a interface que o consome.
 
 
Os elementos comentados nessa seção encontram-se presentes neste trabalho de conclusão de curso. Juntos, montam a estrutura tecnológica necessária aliada à arquitetura adotada. Esses elementos seguem tendências atuais dos softwares web, fazendo com que esse trabalho esteja composto por tecnologias modernas que estão com boa aceitação no mercado.

% %% RiSE Latex Template - version 0.5
%%
%% RiSE's latex template for thesis and dissertations
%% http://risetemplate.sourceforge.net
%%
%% (c) 2012 Yguaratã Cerqueira Cavalcanti (yguarata@gmail.com)
%%          Vinicius Cardoso Garcia (vinicius.garcia@gmail.com)
%%
%% This document was initially based on UFPEThesis template, from Paulo Gustavo
%% S. Fonseca.
%%
%% ACKNOWLEDGEMENTS
%%
%% We would like to thanks the RiSE's researchers community, the 
%% students from Federal University of Pernambuco, and other users that have
%% been contributing to this projects with comments and patches.
%%
%% GENERAL INSTRUCTIONS
%%
%% We strongly recommend you to compile your documents using pdflatex command.
%% It is also recommend use the texlipse plugin for Eclipse to edit your documents.
%%
%% Options for \documentclass command:
%%         * Idiom
%%           pt   - Portguese (default)
%%           en   - English
%%
%%         * Text type
%%           bsc  - B.Sc. Thesis
%%           msc  - M.Sc. Thesis (default)
%%           qual - PHD qualification (not tested yet)
%%           prop - PHD proposal (not tested yet)
%%           phd  - PHD thesis
%%
%%         * Media
%%           scr  - to eletronic version (PDF) / see the users guide
%%
%%         * Pagination
%%           oneside - unique face press
%%           twoside - two faces press
%%
%%		   * Line spacing
%%           singlespacing  - the same as using \linespread{1}
%%           onehalfspacing - the same as using \linespread{1.3}
%%           doublespacing  - the same as using \linespread{1.6}
%%
%% Reference commands. Use the following commands to make references in your
%% text:
%%          \figref  -- for Figure reference
%%          \tabref  -- for Table reference
%%          \eqnref  -- for equation reference
%%          \chapref -- for chapter reference
%%          \secref  -- for section reference
%%          \appref  -- for appendix reference
%%          \axiref  -- for axiom reference
%%          \conjref -- for conjecture reference
%%          \defref  -- for definition reference
%%          \lemref  -- for lemma reference
%%          \theoref -- for theorem reference
%%          \corref  -- for corollary reference
%%          \propref -- for proprosition reference
%%          \pgref   -- for page reference
%%
%%          Example: See \chapref{chap:introduction}. It will produce 
%%                   'See Chapter 1', in case of English language.

\documentclass[pt,twoside,onehalfspacing,bsc]{risethesis}

\usepackage[sort,square]{natbib} %package for bibliography
\bibliographystyle{ieeetr} %authordate1

\usepackage[utf8]{inputenc}
\usepackage[brazilian]{babel}
\usepackage[T1]{fontenc}

\usepackage{graphicx}

%% Change the following pdf author attribute name to your name.
\usepackage[linkcolor=blue,citecolor=blue,urlcolor=blue,colorlinks,pdfpagelabels,pdftitle={Bruno Cabral's Bachelor Thesis},pdfauthor={Bruno Cabral}]{hyperref}

\address{SALVADOR}

\universitypt{Universidade Federal da Bahia}
\universityen{Federal University of Bahia}

\departmentpt{Depertamento de Ciência da Computação}
\departmenten{Computer Science Department}

\programpt{Programa Multiinstitucional de Pós-graduação em Ciência da Computação}
\programen{Graduate in Computer Science}

\majorfieldpt{Ciência da Computação}
\majorfielden{Computer Science}

\title{Sistema de apoio à Pós graduação - UFBA}
\date{Outubro/2016}

\author{Victor de Azevedo Nunes}
\adviser{Ivan do Carmo Machado}

\begin{document}

\frontmatter
\frontpage
\presentationpage

\begin{dedicatory}
Eu dedico esta dissertação...
%I dedicate this dissertation to my family, girlfriend, friends and
%professors who gave me all necessary support to get here.
\end{dedicatory}

\acknowledgements
Meus agradecimentos...

\begin{epigraph}[]{Edward V Berard}
Walking on water and developing software from a specification are easy if both are frozen
\end{epigraph}

\resumo
% Escreva seu resumo no arquivo resumo.tex
\input{resumo}

\abstract
% Write your abstract in a file called abstract.tex
\input{abstract}

% Summary (tables of contents)
\tableofcontents

% List of figures
\listoffigures

% List of tables
\listoftables

% List of acronyms
% Acronyms manual: http://linorg.usp.br/CTAN/macros/latex/contrib/acronym/acronym.pdf
\listofacronyms
\input{acronyms}

% List of listings
%\lstlistoflistings

\mainmatter

\include{chapters/intro}
\include{chapters/referencial_teorico}

% \include{chapters/introduction/main}
% \include{chapters/background/main}
% \include{chapters/proposed_solution/main}
% \include{chapters/experiment/main}
% \include{chapters/conclusion/main}

%\bibliographystyle{natbib}
%\addcontentsline{toc}{chapter}{\bibliographytocname}
\bibliography{references}

% Appendix
\clearpage
\addappheadtotoc
\appendix
\appendixpage
% \include{appendix/experiment-instruments}

\end{document}
% %% RiSE Latex Template - version 0.5
%%
%% RiSE's latex template for thesis and dissertations
%% http://risetemplate.sourceforge.net
%%
%% (c) 2012 Yguaratã Cerqueira Cavalcanti (yguarata@gmail.com)
%%          Vinicius Cardoso Garcia (vinicius.garcia@gmail.com)
%%
%% This document was initially based on UFPEThesis template, from Paulo Gustavo
%% S. Fonseca.
%%
%% ACKNOWLEDGEMENTS
%%
%% We would like to thanks the RiSE's researchers community, the 
%% students from Federal University of Pernambuco, and other users that have
%% been contributing to this projects with comments and patches.
%%
%% GENERAL INSTRUCTIONS
%%
%% We strongly recommend you to compile your documents using pdflatex command.
%% It is also recommend use the texlipse plugin for Eclipse to edit your documents.
%%
%% Options for \documentclass command:
%%         * Idiom
%%           pt   - Portguese (default)
%%           en   - English
%%
%%         * Text type
%%           bsc  - B.Sc. Thesis
%%           msc  - M.Sc. Thesis (default)
%%           qual - PHD qualification (not tested yet)
%%           prop - PHD proposal (not tested yet)
%%           phd  - PHD thesis
%%
%%         * Media
%%           scr  - to eletronic version (PDF) / see the users guide
%%
%%         * Pagination
%%           oneside - unique face press
%%           twoside - two faces press
%%
%%		   * Line spacing
%%           singlespacing  - the same as using \linespread{1}
%%           onehalfspacing - the same as using \linespread{1.3}
%%           doublespacing  - the same as using \linespread{1.6}
%%
%% Reference commands. Use the following commands to make references in your
%% text:
%%          \figref  -- for Figure reference
%%          \tabref  -- for Table reference
%%          \eqnref  -- for equation reference
%%          \chapref -- for chapter reference
%%          \secref  -- for section reference
%%          \appref  -- for appendix reference
%%          \axiref  -- for axiom reference
%%          \conjref -- for conjecture reference
%%          \defref  -- for definition reference
%%          \lemref  -- for lemma reference
%%          \theoref -- for theorem reference
%%          \corref  -- for corollary reference
%%          \propref -- for proprosition reference
%%          \pgref   -- for page reference
%%
%%          Example: See \chapref{chap:introduction}. It will produce 
%%                   'See Chapter 1', in case of English language.

\documentclass[pt,twoside,onehalfspacing,bsc]{risethesis}

\usepackage[sort,square]{natbib} %package for bibliography
\bibliographystyle{ieeetr} %authordate1

\usepackage[utf8]{inputenc}
\usepackage[brazilian]{babel}
\usepackage[T1]{fontenc}

\usepackage{graphicx}

%% Change the following pdf author attribute name to your name.
\usepackage[linkcolor=blue,citecolor=blue,urlcolor=blue,colorlinks,pdfpagelabels,pdftitle={Bruno Cabral's Bachelor Thesis},pdfauthor={Bruno Cabral}]{hyperref}

\address{SALVADOR}

\universitypt{Universidade Federal da Bahia}
\universityen{Federal University of Bahia}

\departmentpt{Depertamento de Ciência da Computação}
\departmenten{Computer Science Department}

\programpt{Programa Multiinstitucional de Pós-graduação em Ciência da Computação}
\programen{Graduate in Computer Science}

\majorfieldpt{Ciência da Computação}
\majorfielden{Computer Science}

\title{Sistema de apoio à Pós graduação - UFBA}
\date{Outubro/2016}

\author{Victor de Azevedo Nunes}
\adviser{Ivan do Carmo Machado}

\begin{document}

\frontmatter
\frontpage
\presentationpage

\begin{dedicatory}
Eu dedico esta dissertação...
%I dedicate this dissertation to my family, girlfriend, friends and
%professors who gave me all necessary support to get here.
\end{dedicatory}

\acknowledgements
Meus agradecimentos...

\begin{epigraph}[]{Edward V Berard}
Walking on water and developing software from a specification are easy if both are frozen
\end{epigraph}

\resumo
% Escreva seu resumo no arquivo resumo.tex
\input{resumo}

\abstract
% Write your abstract in a file called abstract.tex
\input{abstract}

% Summary (tables of contents)
\tableofcontents

% List of figures
\listoffigures

% List of tables
\listoftables

% List of acronyms
% Acronyms manual: http://linorg.usp.br/CTAN/macros/latex/contrib/acronym/acronym.pdf
\listofacronyms
\input{acronyms}

% List of listings
%\lstlistoflistings

\mainmatter

\include{chapters/intro}
\include{chapters/referencial_teorico}

% \include{chapters/introduction/main}
% \include{chapters/background/main}
% \include{chapters/proposed_solution/main}
% \include{chapters/experiment/main}
% \include{chapters/conclusion/main}

%\bibliographystyle{natbib}
%\addcontentsline{toc}{chapter}{\bibliographytocname}
\bibliography{references}

% Appendix
\clearpage
\addappheadtotoc
\appendix
\appendixpage
% \include{appendix/experiment-instruments}

\end{document}
% %% RiSE Latex Template - version 0.5
%%
%% RiSE's latex template for thesis and dissertations
%% http://risetemplate.sourceforge.net
%%
%% (c) 2012 Yguaratã Cerqueira Cavalcanti (yguarata@gmail.com)
%%          Vinicius Cardoso Garcia (vinicius.garcia@gmail.com)
%%
%% This document was initially based on UFPEThesis template, from Paulo Gustavo
%% S. Fonseca.
%%
%% ACKNOWLEDGEMENTS
%%
%% We would like to thanks the RiSE's researchers community, the 
%% students from Federal University of Pernambuco, and other users that have
%% been contributing to this projects with comments and patches.
%%
%% GENERAL INSTRUCTIONS
%%
%% We strongly recommend you to compile your documents using pdflatex command.
%% It is also recommend use the texlipse plugin for Eclipse to edit your documents.
%%
%% Options for \documentclass command:
%%         * Idiom
%%           pt   - Portguese (default)
%%           en   - English
%%
%%         * Text type
%%           bsc  - B.Sc. Thesis
%%           msc  - M.Sc. Thesis (default)
%%           qual - PHD qualification (not tested yet)
%%           prop - PHD proposal (not tested yet)
%%           phd  - PHD thesis
%%
%%         * Media
%%           scr  - to eletronic version (PDF) / see the users guide
%%
%%         * Pagination
%%           oneside - unique face press
%%           twoside - two faces press
%%
%%		   * Line spacing
%%           singlespacing  - the same as using \linespread{1}
%%           onehalfspacing - the same as using \linespread{1.3}
%%           doublespacing  - the same as using \linespread{1.6}
%%
%% Reference commands. Use the following commands to make references in your
%% text:
%%          \figref  -- for Figure reference
%%          \tabref  -- for Table reference
%%          \eqnref  -- for equation reference
%%          \chapref -- for chapter reference
%%          \secref  -- for section reference
%%          \appref  -- for appendix reference
%%          \axiref  -- for axiom reference
%%          \conjref -- for conjecture reference
%%          \defref  -- for definition reference
%%          \lemref  -- for lemma reference
%%          \theoref -- for theorem reference
%%          \corref  -- for corollary reference
%%          \propref -- for proprosition reference
%%          \pgref   -- for page reference
%%
%%          Example: See \chapref{chap:introduction}. It will produce 
%%                   'See Chapter 1', in case of English language.

\documentclass[pt,twoside,onehalfspacing,bsc]{risethesis}

\usepackage[sort,square]{natbib} %package for bibliography
\bibliographystyle{ieeetr} %authordate1

\usepackage[utf8]{inputenc}
\usepackage[brazilian]{babel}
\usepackage[T1]{fontenc}

\usepackage{graphicx}

%% Change the following pdf author attribute name to your name.
\usepackage[linkcolor=blue,citecolor=blue,urlcolor=blue,colorlinks,pdfpagelabels,pdftitle={Bruno Cabral's Bachelor Thesis},pdfauthor={Bruno Cabral}]{hyperref}

\address{SALVADOR}

\universitypt{Universidade Federal da Bahia}
\universityen{Federal University of Bahia}

\departmentpt{Depertamento de Ciência da Computação}
\departmenten{Computer Science Department}

\programpt{Programa Multiinstitucional de Pós-graduação em Ciência da Computação}
\programen{Graduate in Computer Science}

\majorfieldpt{Ciência da Computação}
\majorfielden{Computer Science}

\title{Sistema de apoio à Pós graduação - UFBA}
\date{Outubro/2016}

\author{Victor de Azevedo Nunes}
\adviser{Ivan do Carmo Machado}

\begin{document}

\frontmatter
\frontpage
\presentationpage

\begin{dedicatory}
Eu dedico esta dissertação...
%I dedicate this dissertation to my family, girlfriend, friends and
%professors who gave me all necessary support to get here.
\end{dedicatory}

\acknowledgements
Meus agradecimentos...

\begin{epigraph}[]{Edward V Berard}
Walking on water and developing software from a specification are easy if both are frozen
\end{epigraph}

\resumo
% Escreva seu resumo no arquivo resumo.tex
\input{resumo}

\abstract
% Write your abstract in a file called abstract.tex
\input{abstract}

% Summary (tables of contents)
\tableofcontents

% List of figures
\listoffigures

% List of tables
\listoftables

% List of acronyms
% Acronyms manual: http://linorg.usp.br/CTAN/macros/latex/contrib/acronym/acronym.pdf
\listofacronyms
\input{acronyms}

% List of listings
%\lstlistoflistings

\mainmatter

\include{chapters/intro}
\include{chapters/referencial_teorico}

% \include{chapters/introduction/main}
% \include{chapters/background/main}
% \include{chapters/proposed_solution/main}
% \include{chapters/experiment/main}
% \include{chapters/conclusion/main}

%\bibliographystyle{natbib}
%\addcontentsline{toc}{chapter}{\bibliographytocname}
\bibliography{references}

% Appendix
\clearpage
\addappheadtotoc
\appendix
\appendixpage
% \include{appendix/experiment-instruments}

\end{document}
% %% RiSE Latex Template - version 0.5
%%
%% RiSE's latex template for thesis and dissertations
%% http://risetemplate.sourceforge.net
%%
%% (c) 2012 Yguaratã Cerqueira Cavalcanti (yguarata@gmail.com)
%%          Vinicius Cardoso Garcia (vinicius.garcia@gmail.com)
%%
%% This document was initially based on UFPEThesis template, from Paulo Gustavo
%% S. Fonseca.
%%
%% ACKNOWLEDGEMENTS
%%
%% We would like to thanks the RiSE's researchers community, the 
%% students from Federal University of Pernambuco, and other users that have
%% been contributing to this projects with comments and patches.
%%
%% GENERAL INSTRUCTIONS
%%
%% We strongly recommend you to compile your documents using pdflatex command.
%% It is also recommend use the texlipse plugin for Eclipse to edit your documents.
%%
%% Options for \documentclass command:
%%         * Idiom
%%           pt   - Portguese (default)
%%           en   - English
%%
%%         * Text type
%%           bsc  - B.Sc. Thesis
%%           msc  - M.Sc. Thesis (default)
%%           qual - PHD qualification (not tested yet)
%%           prop - PHD proposal (not tested yet)
%%           phd  - PHD thesis
%%
%%         * Media
%%           scr  - to eletronic version (PDF) / see the users guide
%%
%%         * Pagination
%%           oneside - unique face press
%%           twoside - two faces press
%%
%%		   * Line spacing
%%           singlespacing  - the same as using \linespread{1}
%%           onehalfspacing - the same as using \linespread{1.3}
%%           doublespacing  - the same as using \linespread{1.6}
%%
%% Reference commands. Use the following commands to make references in your
%% text:
%%          \figref  -- for Figure reference
%%          \tabref  -- for Table reference
%%          \eqnref  -- for equation reference
%%          \chapref -- for chapter reference
%%          \secref  -- for section reference
%%          \appref  -- for appendix reference
%%          \axiref  -- for axiom reference
%%          \conjref -- for conjecture reference
%%          \defref  -- for definition reference
%%          \lemref  -- for lemma reference
%%          \theoref -- for theorem reference
%%          \corref  -- for corollary reference
%%          \propref -- for proprosition reference
%%          \pgref   -- for page reference
%%
%%          Example: See \chapref{chap:introduction}. It will produce 
%%                   'See Chapter 1', in case of English language.

\documentclass[pt,twoside,onehalfspacing,bsc]{risethesis}

\usepackage[sort,square]{natbib} %package for bibliography
\bibliographystyle{ieeetr} %authordate1

\usepackage[utf8]{inputenc}
\usepackage[brazilian]{babel}
\usepackage[T1]{fontenc}

\usepackage{graphicx}

%% Change the following pdf author attribute name to your name.
\usepackage[linkcolor=blue,citecolor=blue,urlcolor=blue,colorlinks,pdfpagelabels,pdftitle={Bruno Cabral's Bachelor Thesis},pdfauthor={Bruno Cabral}]{hyperref}

\address{SALVADOR}

\universitypt{Universidade Federal da Bahia}
\universityen{Federal University of Bahia}

\departmentpt{Depertamento de Ciência da Computação}
\departmenten{Computer Science Department}

\programpt{Programa Multiinstitucional de Pós-graduação em Ciência da Computação}
\programen{Graduate in Computer Science}

\majorfieldpt{Ciência da Computação}
\majorfielden{Computer Science}

\title{Sistema de apoio à Pós graduação - UFBA}
\date{Outubro/2016}

\author{Victor de Azevedo Nunes}
\adviser{Ivan do Carmo Machado}

\begin{document}

\frontmatter
\frontpage
\presentationpage

\begin{dedicatory}
Eu dedico esta dissertação...
%I dedicate this dissertation to my family, girlfriend, friends and
%professors who gave me all necessary support to get here.
\end{dedicatory}

\acknowledgements
Meus agradecimentos...

\begin{epigraph}[]{Edward V Berard}
Walking on water and developing software from a specification are easy if both are frozen
\end{epigraph}

\resumo
% Escreva seu resumo no arquivo resumo.tex
\input{resumo}

\abstract
% Write your abstract in a file called abstract.tex
\input{abstract}

% Summary (tables of contents)
\tableofcontents

% List of figures
\listoffigures

% List of tables
\listoftables

% List of acronyms
% Acronyms manual: http://linorg.usp.br/CTAN/macros/latex/contrib/acronym/acronym.pdf
\listofacronyms
\input{acronyms}

% List of listings
%\lstlistoflistings

\mainmatter

\include{chapters/intro}
\include{chapters/referencial_teorico}

% \include{chapters/introduction/main}
% \include{chapters/background/main}
% \include{chapters/proposed_solution/main}
% \include{chapters/experiment/main}
% \include{chapters/conclusion/main}

%\bibliographystyle{natbib}
%\addcontentsline{toc}{chapter}{\bibliographytocname}
\bibliography{references}

% Appendix
\clearpage
\addappheadtotoc
\appendix
\appendixpage
% \include{appendix/experiment-instruments}

\end{document}
% %% RiSE Latex Template - version 0.5
%%
%% RiSE's latex template for thesis and dissertations
%% http://risetemplate.sourceforge.net
%%
%% (c) 2012 Yguaratã Cerqueira Cavalcanti (yguarata@gmail.com)
%%          Vinicius Cardoso Garcia (vinicius.garcia@gmail.com)
%%
%% This document was initially based on UFPEThesis template, from Paulo Gustavo
%% S. Fonseca.
%%
%% ACKNOWLEDGEMENTS
%%
%% We would like to thanks the RiSE's researchers community, the 
%% students from Federal University of Pernambuco, and other users that have
%% been contributing to this projects with comments and patches.
%%
%% GENERAL INSTRUCTIONS
%%
%% We strongly recommend you to compile your documents using pdflatex command.
%% It is also recommend use the texlipse plugin for Eclipse to edit your documents.
%%
%% Options for \documentclass command:
%%         * Idiom
%%           pt   - Portguese (default)
%%           en   - English
%%
%%         * Text type
%%           bsc  - B.Sc. Thesis
%%           msc  - M.Sc. Thesis (default)
%%           qual - PHD qualification (not tested yet)
%%           prop - PHD proposal (not tested yet)
%%           phd  - PHD thesis
%%
%%         * Media
%%           scr  - to eletronic version (PDF) / see the users guide
%%
%%         * Pagination
%%           oneside - unique face press
%%           twoside - two faces press
%%
%%		   * Line spacing
%%           singlespacing  - the same as using \linespread{1}
%%           onehalfspacing - the same as using \linespread{1.3}
%%           doublespacing  - the same as using \linespread{1.6}
%%
%% Reference commands. Use the following commands to make references in your
%% text:
%%          \figref  -- for Figure reference
%%          \tabref  -- for Table reference
%%          \eqnref  -- for equation reference
%%          \chapref -- for chapter reference
%%          \secref  -- for section reference
%%          \appref  -- for appendix reference
%%          \axiref  -- for axiom reference
%%          \conjref -- for conjecture reference
%%          \defref  -- for definition reference
%%          \lemref  -- for lemma reference
%%          \theoref -- for theorem reference
%%          \corref  -- for corollary reference
%%          \propref -- for proprosition reference
%%          \pgref   -- for page reference
%%
%%          Example: See \chapref{chap:introduction}. It will produce 
%%                   'See Chapter 1', in case of English language.

\documentclass[pt,twoside,onehalfspacing,bsc]{risethesis}

\usepackage[sort,square]{natbib} %package for bibliography
\bibliographystyle{ieeetr} %authordate1

\usepackage[utf8]{inputenc}
\usepackage[brazilian]{babel}
\usepackage[T1]{fontenc}

\usepackage{graphicx}

%% Change the following pdf author attribute name to your name.
\usepackage[linkcolor=blue,citecolor=blue,urlcolor=blue,colorlinks,pdfpagelabels,pdftitle={Bruno Cabral's Bachelor Thesis},pdfauthor={Bruno Cabral}]{hyperref}

\address{SALVADOR}

\universitypt{Universidade Federal da Bahia}
\universityen{Federal University of Bahia}

\departmentpt{Depertamento de Ciência da Computação}
\departmenten{Computer Science Department}

\programpt{Programa Multiinstitucional de Pós-graduação em Ciência da Computação}
\programen{Graduate in Computer Science}

\majorfieldpt{Ciência da Computação}
\majorfielden{Computer Science}

\title{Sistema de apoio à Pós graduação - UFBA}
\date{Outubro/2016}

\author{Victor de Azevedo Nunes}
\adviser{Ivan do Carmo Machado}

\begin{document}

\frontmatter
\frontpage
\presentationpage

\begin{dedicatory}
Eu dedico esta dissertação...
%I dedicate this dissertation to my family, girlfriend, friends and
%professors who gave me all necessary support to get here.
\end{dedicatory}

\acknowledgements
Meus agradecimentos...

\begin{epigraph}[]{Edward V Berard}
Walking on water and developing software from a specification are easy if both are frozen
\end{epigraph}

\resumo
% Escreva seu resumo no arquivo resumo.tex
\input{resumo}

\abstract
% Write your abstract in a file called abstract.tex
\input{abstract}

% Summary (tables of contents)
\tableofcontents

% List of figures
\listoffigures

% List of tables
\listoftables

% List of acronyms
% Acronyms manual: http://linorg.usp.br/CTAN/macros/latex/contrib/acronym/acronym.pdf
\listofacronyms
\input{acronyms}

% List of listings
%\lstlistoflistings

\mainmatter

\include{chapters/intro}
\include{chapters/referencial_teorico}

% \include{chapters/introduction/main}
% \include{chapters/background/main}
% \include{chapters/proposed_solution/main}
% \include{chapters/experiment/main}
% \include{chapters/conclusion/main}

%\bibliographystyle{natbib}
%\addcontentsline{toc}{chapter}{\bibliographytocname}
\bibliography{references}

% Appendix
\clearpage
\addappheadtotoc
\appendix
\appendixpage
% \include{appendix/experiment-instruments}

\end{document}

%\bibliographystyle{natbib}
%\addcontentsline{toc}{chapter}{\bibliographytocname}
\bibliography{references}

% Appendix
\clearpage
\addappheadtotoc
\appendix
\appendixpage
% \include{appendix/experiment-instruments}

\end{document}
% %% RiSE Latex Template - version 0.5
%%
%% RiSE's latex template for thesis and dissertations
%% http://risetemplate.sourceforge.net
%%
%% (c) 2012 Yguaratã Cerqueira Cavalcanti (yguarata@gmail.com)
%%          Vinicius Cardoso Garcia (vinicius.garcia@gmail.com)
%%
%% This document was initially based on UFPEThesis template, from Paulo Gustavo
%% S. Fonseca.
%%
%% ACKNOWLEDGEMENTS
%%
%% We would like to thanks the RiSE's researchers community, the 
%% students from Federal University of Pernambuco, and other users that have
%% been contributing to this projects with comments and patches.
%%
%% GENERAL INSTRUCTIONS
%%
%% We strongly recommend you to compile your documents using pdflatex command.
%% It is also recommend use the texlipse plugin for Eclipse to edit your documents.
%%
%% Options for \documentclass command:
%%         * Idiom
%%           pt   - Portguese (default)
%%           en   - English
%%
%%         * Text type
%%           bsc  - B.Sc. Thesis
%%           msc  - M.Sc. Thesis (default)
%%           qual - PHD qualification (not tested yet)
%%           prop - PHD proposal (not tested yet)
%%           phd  - PHD thesis
%%
%%         * Media
%%           scr  - to eletronic version (PDF) / see the users guide
%%
%%         * Pagination
%%           oneside - unique face press
%%           twoside - two faces press
%%
%%		   * Line spacing
%%           singlespacing  - the same as using \linespread{1}
%%           onehalfspacing - the same as using \linespread{1.3}
%%           doublespacing  - the same as using \linespread{1.6}
%%
%% Reference commands. Use the following commands to make references in your
%% text:
%%          \figref  -- for Figure reference
%%          \tabref  -- for Table reference
%%          \eqnref  -- for equation reference
%%          \chapref -- for chapter reference
%%          \secref  -- for section reference
%%          \appref  -- for appendix reference
%%          \axiref  -- for axiom reference
%%          \conjref -- for conjecture reference
%%          \defref  -- for definition reference
%%          \lemref  -- for lemma reference
%%          \theoref -- for theorem reference
%%          \corref  -- for corollary reference
%%          \propref -- for proprosition reference
%%          \pgref   -- for page reference
%%
%%          Example: See \chapref{chap:introduction}. It will produce 
%%                   'See Chapter 1', in case of English language.

\documentclass[pt,twoside,onehalfspacing,bsc]{risethesis}

\usepackage[sort,square]{natbib} %package for bibliography
\bibliographystyle{ieeetr} %authordate1

\usepackage[utf8]{inputenc}
\usepackage[brazilian]{babel}
\usepackage[T1]{fontenc}

\usepackage{graphicx}

%% Change the following pdf author attribute name to your name.
\usepackage[linkcolor=blue,citecolor=blue,urlcolor=blue,colorlinks,pdfpagelabels,pdftitle={Bruno Cabral's Bachelor Thesis},pdfauthor={Bruno Cabral}]{hyperref}

\address{SALVADOR}

\universitypt{Universidade Federal da Bahia}
\universityen{Federal University of Bahia}

\departmentpt{Depertamento de Ciência da Computação}
\departmenten{Computer Science Department}

\programpt{Programa Multiinstitucional de Pós-graduação em Ciência da Computação}
\programen{Graduate in Computer Science}

\majorfieldpt{Ciência da Computação}
\majorfielden{Computer Science}

\title{Sistema de apoio à Pós graduação - UFBA}
\date{Outubro/2016}

\author{Victor de Azevedo Nunes}
\adviser{Ivan do Carmo Machado}

\begin{document}

\frontmatter
\frontpage
\presentationpage

\begin{dedicatory}
Eu dedico esta dissertação...
%I dedicate this dissertation to my family, girlfriend, friends and
%professors who gave me all necessary support to get here.
\end{dedicatory}

\acknowledgements
Meus agradecimentos...

\begin{epigraph}[]{Edward V Berard}
Walking on water and developing software from a specification are easy if both are frozen
\end{epigraph}

\resumo
% Escreva seu resumo no arquivo resumo.tex
Este trabalho de conclusão de curso utiliza elementos da engenharia de software para propor um software como serviço a fim de otimizar o processo de construção e manutenção dos softwares. Assim, o objetivo deste SaaS é gerenciar as permissões de acesso de sistemas cliente, provendo o reuso de software e padronizando as soluções. O sistema proposto neste trabalho encontra-se implementado e disponível no Github, e traz fundamentos sobre a arquitetura e tecnologias utilizadas, bem como avaliações sobre possibilidades de utilização do mesmo em diversos ambientes, como Web e mobile.

\begin{keywords}
Software; Reuso; SaaS; Web; PHP
\end{keywords}

\abstract
% Write your abstract in a file called abstract.tex
My abstract...

\begin{keywords}
key words...
\end{keywords}

% Summary (tables of contents)
\tableofcontents

% List of figures
\listoffigures

% List of tables
\listoftables

% List of acronyms
% Acronyms manual: http://linorg.usp.br/CTAN/macros/latex/contrib/acronym/acronym.pdf
\listofacronyms
\begin{acronym}[ACRONYM] 
% Change the word ACRONYM above to change the acronym column width.
% The column width is equals to the width of the word that you put.
% Read the manual about acronym package for more examples:
%   http://linorg.usp.br/CTAN/macros/latex/contrib/acronym/acronym.pdf

\acro{SPA}{Single Page Application}
\acro{JSON}{Javascript Object Notation}
\acro{PHP}{PHP: Hypertext Preprocessor}
\acro{SaaS}{Software as a Service}
\acro{ERP}{Enterprise Resource Planning}
\acro{QoS}{Quality of Service}
\acro{UML}{Unified Modeling Language}
\acro{MVC}{Model-View-Controller}
\acro{Ajax}{Asynchronous Javascript and XML}
\acro{HTML}{HyperText Markup Language}
\acro{CSS}{Cascading Style Sheets}
\acro{API}{Application Programming Interface}
\acro{DOM}{Document Object Model}
\acro{BPMN}{Business Process Model and Notation}
\acro{REST}{Representational State Transfer}

\end{acronym}

% List of listings
%\lstlistoflistings

\mainmatter

\chapter{Introdução}

\section{Motivação}

Organizar os procedimentos de um processo sempre nos traz vantagens. Apesar de no processo de implantação de um sistema, o mesmo burocratizar o processo, com o tempo temos o retorno da dedicação para a inserção dos dados. Com um certo volume de dados, é possível estruturar informações que num processo manual são difíceis de serem enxergadas. Assim, é possível depender menos das pessoas que organizam o processo, pois o legado de informações não estará mais somente na mente de alguns, mas sim documentado nos dados do sistema.

Além de colaborar na organização, também haverá uma grande colaboração no tempo gasto na gestão. Lidar com muitos papéis e confiar na mente humana para guardar informações, não é uma alternativa muito segura devido ao fato que as pessoas sempre estão sujeitas a sair do processo e levar contigo a experiência obtida. Experiência essa que faz com que os procedimentos sejam executados de forma mais eficiente. Entretanto, com um sistema inteligente, é possível auxiliar e tornar mais ágil a execução das tarefas.


\section{Problema}


De acordo com funcionários ligados ao o setor de pós graduação da UFBA, entrevistados a fim de um maior entendimento do cenário, apesar das semelhanças estruturais, a pós graduação gerida de forma diferencia da graduação. FULANO afirma que devido ao fato de não ter a mesma visibilidade, não tem acesso aos mesmos recursos de gestão acadêmica da graduação. O professores não executam somente atividades dentro da sala de aula, também tem diversas outras ocupações no setor. E muitos procedimentos realizados extra classe ainda se encontram sendo realizados de forma manual, estando mais vulnerável ao erro ou até mesmo à violação do processo. Também ocorre um grande desperdício de tempo pelos professores e gestores da área, devido ao diversos processos ainda realizados de forma manual, sem a devida documentação. Segundo FULANO, também entrevistado, esse tempo perdido implica numa redução da eficiência na sala de aula, pois o professor acaba por ter menos tempo disponível para o planejamento das atividades, o que gera impactos negativos aos alunos.


\section{Objetivos} %<o que deve ser feito/entregue>


Devido aos muitos processos sendo resolvidos de forma manual, propõe-se com solução um sistema moderno, arquitetado para ter funcionamento na web e com um módulo mobile, a fim de fornecer informações de forma rápida e eficiente para os professores através de notificações, já que o acesso à internet móvel é comum entre os possíveis usuários do sistema em questão.
O principal requisito para o sistema seria dispor recursos para reduzir o tempo desperdiçado pelos professores durante as atividades extra classe.


\section{Metodologia} %<como será feito | como resolver o problema apontado inicialmente>


%<analise de literatura | design | implementação | validação>
Baseando-se nas tecnologias gratuitas em alta no cenário atual do desenvolvimento web, dispomos de algumas opções eficientes para a implementação da solução. Dentre as possibilidades, considerando a facilidade para futura manutenção e continuidade do projeto, tende-se a optar por uma tecnologia popular. Como linguagem de programação, adota-se o PHP. A escolha é fundamentada de acordo com a pesquisa da RedMonk de 2015, que evidencia o uso das linguagens de programação de acordo com as discussões no StackOverflow e repositórios no GitHub. É possível constatar a popularidade do PHP no cenário atual com o gráfico da pesquisa citada, na qual o PHP é apresentado na terceira colocação, apenas atrás do lider JavaScript e do segundo colocado, o Java.

\begin{figure}
	\label{fig:graficoRedmonk}
	\includegraphics[width=1\textwidth]{img/grafico_redmonk}
	\caption{Ranking das liguagens de programação no Stack Overflow e Github}
\end{figure}


Ainda assim, para compor a interface do dado projeto, também ocorrerá o uso do líder JavaScript de forma intensa, provendo o elo com o as informações gerenciadas pelo PHP.


Entretanto, não seria inteligente desenvolver um sistema completo sem o auxílio de um framework. Dentre os frameworks disponíveis para PHP, hoje o destaque está com o Laravel, que se encontra no topo dentre os mais utilizados no momento. 


A WebHostFace, uma empresa de hospedagem, compilou várias estatísticas para criar um infográfico mostrando os frameworks PHP mais populares de 2015. Utilizando informações sobre os próprios clientes, o Google Trends, estatísticas de repositórios do GitHub e a pesquisa do SitePoint “Best PHP Frameworks 2015”, a WebHostFace elaborou o seguinte infográfico: 

\begin{figure}
	\label{fig:graficoWebhostface}
	\includegraphics[width=1\textwidth]{img/infografico_webhostface}
	\caption{Infográfico da WebhostFace, exibindo a popularidade dos Frameworks PHP em 2015}
\end{figure}

Assim, tem-se a evidência que o Laravel em 2015 teve a maior popularidade em projetos pessoais e tem a maior comunidade entre os concorrentes, o que o torna uma boa escolha para a escrita de um software que será continuado por terceiros.


Para elaborar os recursos de interface e integrar ao back-end PHP do sistema, será adotado o já conhecido AngularJS, ferramenta sólida e conhecida no aspecto em questão. 


Dados coletados via Google Trends, que propõe comparações entre termos pesquisados, revela a popularidade do AngularJs diante de alguns dos principais concorrentes. O gráfico abaixo evidencia o cenário.


%Como mostra a Figura \ref{fig:graficoGoogleTrendsFerramentasFront}. 
\begin{figure}
	\label{fig:graficoGoogleTrendsFerramentasFront}
	\includegraphics[width=1\textwidth]{img/grafico_ferramentas_front}
	\caption{Gráfico do Google Trends exibindo as pesquisas por ferramentas front-end}
\end{figure}


Junto ao Angular JS, será utilizada a agradável tendência de interface do Material Design da Google, que propõe layouts limpos e otimizados já conhecidos pelos usuários de smartphones Android. 


Para a elaboração da plataforma mobile do projeto, será utilizado o Ionic Framework, muito difundido e bastante pesquisado na área, o que fica evidenciado com o gráfico de pesquisbaixo, coletado via Google Trends buscando por frameworks de desenvolvimento híbrido mobile.


\begin{figure}
	\label{fig:graficoGoogleTrendsFerramentasHibridasMobile}
	\includegraphics[width=1\textwidth]{img/grafico_ferramentas_hibridas_mobile}
	\caption{Gráfico do Google Trends exibindo as pesquisas por Frameworks híbridos mobile}
\end{figure}	

Para layout da interface mobile, também será aplicado a tendência do Material Design, a fim de propor uma harmonia entre o módulo web e mobile para os usuários


\section{Resultados Esperados}


Como fruto de um sistema para pós-graduação da UFBA, espera-se que os professores tenham mais recursos para integrar as atividades e também prover melhores condições para acompanhamento da vida acadêmica dos alunos em questão. Também, que os novos colaboradores que entrarem no processo tenham facilidade de compreender o fluxo do setor ao navegar pelo sistema proposto.


\section{Fora de Escopo}


Interação com os alunos devido às complicações para realizar a integração com o sistema empregado na UFBA, gerenciado pela XXXXXX, o que causaria uma inviabilidade no projeto devido à necessidade de entrega do produto ser mais forte que o tempo necessário para executar o processo de obtenção de acesso ao sistema legado para realizar a integração.


\section{Estrutura do Trabalho}


<breve resumo sobre os capítulos do TCC>
\chapter{Referencial Teórico}


Projetar o desenvolvimento de um software requer muito planejamento, pois as falhas iniciais podem custar bastante caro ou até mesmo inviabilizar a continuação de um projeto. Assim, a escolha da arquitetura ideal para a aplicabilidade é essencial na concepção de um produto de software. 
De todo o modo, sempre busca-se fazer mais com menos. Diante de tal filosofia, temos neste capítulo, uma breve discussão sobre alguns elementos de projeto e arquitetura de software, a fim de contextualizar este trabalho de conclusão de curso.
O capítulo corrente é composto por quatro seções. A \ref{sec:saas} trata de Software como serviço, discutindo alguns elementos do contexto que são relevantes para o trabalho proposto. A \ref{sec:reuso}  discute sobre a empregabilidade do reuso de software. A \ref{sec:modularizacao} seção, trata sobre aspectos envolvidos na modularização dos softwares. Por fim, a \ref{sec:apps_web} aborda as aplicações web, discutindo sobre aspectos relevantes sobre a aplicação web que compõe este trabalho.


 \section{Software como serviço}\label{sec:saas}


Segundo La e Chun \citep{La2009Systematic}, o princípio da definição de Software como um Serviço (Sofware as a Service - SaaS) é um serviço complementar para aplicações da computação em nuvem (cloud computing). No entanto, as áreas não se confundem. SaaS deve ser entendido como um mecanismo de suporte às soluções existentes na cloud. Os SaaS existem justamente para maximizar o reuso de serviços repetidos e não centrais em uma aplicação remota.


Como vantagens, diversos fatores podem ser favoráveis para a adoção de um SaaS, como custo e manutenção dentre outros fatores aplicáveis a soluções específicas. Lechesaet al. \citep{LechesaSS11}, quantificam índices sobre os fatores determinantes para adoção ou não de um SaaS voltado para ERP na África do Sul. Os principais fatores determinantes para adoção desse mecanismo de software são sua fluidez quanto à rede e a segurança. Esses fatores estão presentes na aplicação desenvolvida neste trabalho de conclusão de curso.


Devido ao fato de ter um serviço constantemente na nuvem, fica o questionamento sobre a segurança da informação manipulada. Sabe-se que a vulnerabilidade na Web não é restrita ao SaaS, atingindo diversos âmbitos. Rai et al. \cite{journals/corr/RaiSM13} consideram que o avanço da computação em nuvem não é um problema apenas para os serviços Web do ponto de vista da segurança, pois muitos trabalhos na literatura mostram a área como mais um ponto de vulnerabilidade para diversos setores, a exemplo de infraestrutura. No mesmo artigo mencionado de Rai et al. \cite{journals/corr/RaiSM13}, também realizaram-se estudos exploratórios junto a empresas usuárias de serviços em computação em nuvem e consideram que a perspectiva de SaaS também pode fortalecer a segurança nas aplicações de cloud computing, pois o software de autenticação compartilhado por várias aplicações em nuvem, oferece uma melhor padronização e consequente facilidade de prevenção a erros de vulnerabilidade específicas de cada módulo da pesquisa. Esse ponto de vista é fundamenta para a compreensão de SaaS.


A arquitetura de armazenamento de dados de um Saas pode variar de acordo com a necessidade do contexto. Huixin \cite{7586486} descreve possíveis modelagens nesse sentido. Essa abordagem pode ser com um banco de dados único, fazendo com que diferentes clientes compartilhem o mesmo banco, diferindo os dados através de controle de usuário, ou isolando os diferentes clientes através de bancos de dados exclusivos para cada um. Esse fator também pode ser combinado com a arquitetura da aplicação, caso ofereça aplicação única para todos os clientes ou aplicação compartilhada. Diante das possíveis abordagens, a modelagem de dados do software pode ser decidida pela regra de negócio. Este trabalho optou por aplicação única e banco de dados compartilhado.

Devido ao conceito particular de obtenção de software, tanto pela visão do cliente como pela visão do vendedor, é necessário compreender os diversos aspectos que podem ser relevantes ao orçar um Saas. O recente trabalho de T. Kaur et al. \citep{6949281} orienta um modelo para compor o custo de um Saas. O custo total seria composto pelos fatores que dão suporte ao funcionamento do software. Tais fatores incluem infra-estrutura, configurabilidade, customização, parâmetros de QoS (Quality of service) como escalabilidade, disponibilidade, usabilidade, pontualidade e desempenho da resposta, portabilidade, custo total de propriedade e retorno do investimento. Esses fatores caracterizam o custo de forma eficaz, possibilitando ao fornecedor, prover um Serviço de acordo com a exigência do consumidor em vários pacotes de serviços.


O conceito de software como serviço se aplica neste trabalho de conclusão de curso, pois o mesmo estará disponível na web com alta disponibilidade adotando as características apresentadas para qualquer pessoa que desejar usá-lo. Assim, basta se cadastrar, configurá-lo e usar como um serviço, sem preocupação com a manutenção do mesmo.


\section{Reuso de software}\label{sec:reuso} %CRUISE BOOK CAPITULO 2


De acordo com o livro Software pratical reuse \citep{Ezran2002}, o reuso de software é a utilização de qualquer informação que um desenvolvedor pode necessitar no processo de criação de software. O livro de Basili e Rombach \citep{Basili1991} define reutilização de software como o uso de tudo o que está associado a um projeto de conhecimento.
Assim, o objetivo da reutilização de software é reciclar o design, código e outros componentes de um produto de software e assim reduzir o custo, o tempo e melhorar a qualidade do produto.
Segundo Keswani et al. \cite{6783445}, o componente reutilizável de software pode ser qualquer parte de seu desenvolvimento, como um fragmento de código, design, casos de teste, ou até mesmo a especificação de requisitos de uma funcionalidade do software. 

O reuso de software pode ter impacto positivo em diversos aspectos do software, vejamos alguns, conforme apresentados no C.R.U.I.S.E Book \citep{cruise2007} :

\begin{itemize}

\item Qualidade: As correções de erro tornam-se úteis em todos os locais em que ocorreu, padronizando e facilitando a manutenção.

\item Produtividade: O ganho de produtividade é alcançado devido ao menor número de artefatos desenvolvido. Isso resulta em menor esforço de teste e também análise e design, reduzindo custos.

\item Confiabilidade: A utilização de componentes bem testados aumenta a
confiança no software. Além disso, a utilização de um mesmo componente em vários sistemas, aumenta a possibilidade de detecção de erros e reforça a confiança no componente.

\item Redução do Esforço: A reutilização de software proporciona uma redução do tempo de desenvolvimento, o que reduz o tempo necessário para o produto ser disponibilizado no mercado para trazer rentabilidade.

\item Trabalho redundante e tempo de desenvolvimento: Desenvolver um sistema do
zero significa desenvolvimento redundante de muitos componentes, como requisitos, especificações, casos de uso, arquitetura, etc. Isso pode ser evitado quando estes estão disponíveis como componentes reutilizáveis e podem ser compartilhados, resultando em um processo de desenvolvimento otimizado.

\item Documentação: Embora a documentação seja muito importante para a
manutenção de um sistema, muitas vezes é negligenciada. A reutilização de componentes de software reduz a quantidade de documentação a ser escrita, entretanto depende da qualidade do que está escrito. Assim, apenas a estrutura do sistema e os novos artefatos desenvolvidos necessitam ser documentados.

\item Custo de manutenção: Menos defeitos e manutenções são esperados quando tem-se comprovada a qualidade dos componentes utilizados.

\item Tamanho da equipe: É comum haver casos em que a equipe de desenvolvimento sofre sobrecarga. Entretanto, dobrar o tamanho da equipe de desenvolvimento não necessariamente duplica produtividade. Se muitos componentes podem ser reutilizados, é possível desenvolver com equipes menores, levando a melhor comunicação e aumento da produtividade.

\end{itemize}

Apesar dos benefícios da reutilização de software, ela não é suficientemente aproveitada. Existem fatores que influenciam direta ou indiretamente na sua adoção. Esses fatores podem ser de aspecto gerencial, organizacional, econômico, conceitual ou técnico. Veremos a seguir alguns aspectos que podem gerar conflito com a cultura de reuso de software, segundo o C.R.U.I.S.E Book \citep{cruise2007} :
%(Sametinger, 1997). REVER

\begin{itemize}
	
\item Falta de apoio da gestão: Como a reutilização de software gera custos iniciais,
a medida pode não ser amplamente alcançada em uma organização sem o apoio de alto nível de gestão. Os gestores têm de ser informados sobre os custos iniciais e serem convencidos sobre economias futuras.

\item Gerenciamento do Projeto: Gerenciar projetos tradicionais é uma tarefa árdua, principalmente, os que praticam a reutilização de software. Utilizando a técnica em larga escala, tem-se impacto sobre todo o ciclo de vida do software.

\item Estruturas organizacionais inadequadas: As estruturas organizacionais devem
considerar diferentes necessidades que surgem quando a reutilização em larga escala está sendo adotada. Por exemplo, uma equipe particionada pode ser alocada somente para desenvolver, manter e certificar componentes reutilizáveis de software.

\item Incentivos de gestão: É comum a falta de incentivo para deixar os desenvolvedores gastarem tempo elaborando componentes do sistemas. A produtividade é muitas vezes medida apenas no tempo necessário para concluir um projeto. Assim, fazer qualquer trabalho além disso, embora benéfico para a empresa como um todo, diminui o seu sucesso. Mesmo quando os componentes reutilizáveis são utilizados, os benefícios obtidos são uma pequena fração do que poderia ser alcançado caso houvesse reutilização explícita, planejada e organizada.

\item Dificuldade de encontrar software reutilizável: Para reutilizar os componentes, devem existir formas eficientes de busca. Além disso, é importante ter um repositório bem organizado contendo componentes com um eficiente meio de acesso.

\item Não reutilização do software encontrado. O acesso fácil ao software existente
não necessariamente aumentar a reutilização. Os componentes reutilizáveis devem ser cuidadosamente especificados, projetados, implementados e documentados, pois em alguns casos, modificar e adaptar o código  pode ser mais custoso que a programação da funcionalidade necessária a partir do zero.

\item Modificação: É muito difícil encontrar um componente que funcione
exatamente da mesma maneira que queremos. Desta forma, são necessárias modificações e devem existir formas de determinar os seus efeitos sobre o componente.


\end{itemize}


A definição do reuso de software pode ser associada a este trabalho, pois ainda que como um serviço, estaremos adotando uma aplicação única para servir aos mais diversos softwares de uma empresa, por exemplo. Logo, ao adotar o projeto, pode ser interessante aplicar algumas métricas para tomar conhecimento de possíveis vantanjens como as citadas nessa seção, a exemplo de economia de tempo de desenvolvimento, e custo do projeto.


%Outra diretriz importante para a reutilização de software é reduzir o risco na criação de novos softwares. O risco tende a ser bastante reduzido se os componentes que estão sendo reutilizados têm as documentação, interfaces necessárias e devidamente testadas, fatores que contibruem para uma fácil integração.
%De acordo com Keswani et al. \citep{6783445}, para o reuso de software dar retornos apropriados, o processo deve ser sistemático e planejado. Qualquer organização que implemente a reutilização de software deve identificar os melhores métodos e estratégias de reutilização para obter a máxima produtividade. A reutilização de software ajuda a evitar software de engenharia a partir do zero, pois usa módulos de software existentes. A reutilização de software, embora seja uma tarefa difícil, especialmente para softwares antigos sem padrões de projeto, pode melhorar significativamente a produtividade e a qualidade de um produto de software. Embora a reutilização de software não seja um novo campo, ela pode dar grandes retornos em curto período de tempo.


\section{Modularização}\label{sec:modularizacao} %artigo de claudio pagina 222 introdução


%A modularidade vem desempenhando um papel predominante estágios emergentes das disciplinas de arquitetura de software [13]. Engenheiros de software consideram modularidade como princípio base na comparação entre arquiteturas alternativas  e arquitetura degeneração [9]. De fato, os engenheiros de software são incentivados a arquitecturas, baseando-se numa multiplicidade de mecanismos de modularidade disponíveis em: 
%(i) Linguagens de descrição de arquitetura (ADLs), como ACME [8], 
%(ii) catálogos de arquitetônicos [2, 13], e 
%(iii) conhecem bem princípios de alto nível, como interfaces de componentes estreitos, acoplamento arquitectónico reduzido e semelhantes.


Conforme é frisado por Wickramaarachchi e Lai \citep{7062705}, o conceito de modularização na indústria de software tem uma longa história e tem sido utilizado para melhorar o processo de desenvolvimento de software em diferentes estágios. Os principais conceitos por trás da modularização do software foram introduzidos por pesquisadores pioneiros há quarenta anos, com uma notável contribuição feita por Melvin Conway e David Parnas, que tem representação notável na engenharia de software.


Modularizar um software é um bom padrão a ser adotado. Segundo Wickramaarachchi e Lai \citep{7062705}, a modularização é importante na identificação de dependências e reduz as dificuldades diante de uma possível necessidade de grandes alterações. De uma perspectiva da engenharia de software, uma modularização geralmente tem várias vantagens, tais como: tornar a complexidade do software mais gerenciável, facilitar o trabalho paralelo e tornar o software mais maleável para acomodar o futuro incerto que um software pode ter. O objetivo final da modularização do software é aumentar a produtividade ea qualidade do software. Tal conceito encontra-se bastante difundido e estái incorporado em linguagens de programação e ferramentas de software. O presente trabalho favorece ao uso da modularização de um software e até mesmo pode ser considerado um módulo a ser acoplado a qualquer software, mediante a compatibilidade.


Ao realizar a adoção do trabalho proposto, fica bastante evidente a "responsabilidade" do mesmo no escopo do projeto. Assim, é possível tratá-lo como um módulo do projeto que o usa, mesmo que seja consumido como um serviço.


\section{Aplicações web}\label{sec:apps_web}


A popularidade das soluções Web aumentou exponencialmente na última década e todos os dias cresce o número de pessoas usuárias desse tipo de software. E seguindo um padrão próprio, Kumar et al. \citep{7813710} sugerem que para o desenvolvimento web, deve-se manter a prática eficaz de produzir diagramas UML. A abordagem baseada na web oferece uma maneira fácil e eficaz para gerenciar e controlar o processo de desenvolvimento por meio de artefatos de modelagem. Tal abordagem pode ser usada quando há uma exigência de lidar com mudanças muito rápidas e grandes em requisitos de forma muito eficaz em muito menos tempo, gerando assim um menor impacto. 


Para atender à fomentada demanda de aplicativos web, é necessário adotar métodos de desenvolvimentos que sejam ágeis, eficientes e de fácil manutenção. Yu Ping et al. \cite{1372143} propõem o uso do modelo MVC (Model, View e Controller) no desenvolvimento para softwares web. O modelo apresentado tornou-se um padrão popular e divide o software em camadas com propósito definido, tornando-o de mais fácil manutenção.


O Ajax (Asynchronous Javascript and XML) revolucionou a web. Conforme demonstrado por Yuping \citep{6845605}, ao usar a tecnologia Ajax, podemos enriquecer a experiência do usuário em aplicações baseadas em navegador de internet, e fornecer uma variedade de aplicações interativas para atender às necessidade de humanização das aplicações.
Os aplicativos Ajax em execução no navegador se comunicam com um servidor Web de forma assíncrona e atualizam apenas uma parte da página.


No artigo de Tesarik et al. \citep{4664349}, temos que o design de software SPA (Single page application) é uma maneira interessante de criar um software Web numa uma única página. Essa solução de página única sem navegação funciona apenas com base em técnicas dinâmicas e assíncronas, como o AJAX (citado no parágrafo anterior). No entanto, esta abordagem coloca o desenvolvedor antes de alguns desafios substanciais.
Para projetar a interface do usuário que mostra as informações, é importante projetar corretamente a tela para manipular os dados do aplicativo numa única página. O design da página deve ser elaborado para maximizar a decomposição da página em componentes distintos que encapsulam os principais casos de uso.
Também recomenda-se que para elaborar os artefatos visuais, explore os mais ricos recursos na implementação, como HTML5, JavaScript, Ajax, CSS3, e outras tecnologias que se apliquem. Entretanto, desenvolver uma rica interface com o uso de diversas tecnologias/frameworks pode ocasionar um esforço maior para explorar as possibilidades.
Tais componentes da interface de uma aplicação SPA normalmente são alimentados mediante o consumo de uma API Rest via requisições AJAX. Assim, é possível particionar as responsabilidades de processamento do software entre cliente e servidor.


Como é possível denotar no trabalho de Salvadori e Siqueire \citep{7195633}, REST é uma arquitetura muito popular para integração de aplicativos web, que permite compartilhar e reutilizar informações através de sistemas. Aplicações de grande escala baseadas em no REST deve ser implementadas utilizando estratégias e mecanismos para produzir sistemas que sejam fáceis de desenvolver, reutilizar e manter. As interfaces de integração fornecidas por esses sistemas, chamadas de Web API, têm uma influência importante nas características da implementação resultante, pois a sua forma de resposta deve ser casada com a interface que o consome.
 
 
Os elementos comentados nessa seção encontram-se presentes neste trabalho de conclusão de curso. Juntos, montam a estrutura tecnológica necessária aliada à arquitetura adotada. Esses elementos seguem tendências atuais dos softwares web, fazendo com que esse trabalho esteja composto por tecnologias modernas que estão com boa aceitação no mercado.

% %% RiSE Latex Template - version 0.5
%%
%% RiSE's latex template for thesis and dissertations
%% http://risetemplate.sourceforge.net
%%
%% (c) 2012 Yguaratã Cerqueira Cavalcanti (yguarata@gmail.com)
%%          Vinicius Cardoso Garcia (vinicius.garcia@gmail.com)
%%
%% This document was initially based on UFPEThesis template, from Paulo Gustavo
%% S. Fonseca.
%%
%% ACKNOWLEDGEMENTS
%%
%% We would like to thanks the RiSE's researchers community, the 
%% students from Federal University of Pernambuco, and other users that have
%% been contributing to this projects with comments and patches.
%%
%% GENERAL INSTRUCTIONS
%%
%% We strongly recommend you to compile your documents using pdflatex command.
%% It is also recommend use the texlipse plugin for Eclipse to edit your documents.
%%
%% Options for \documentclass command:
%%         * Idiom
%%           pt   - Portguese (default)
%%           en   - English
%%
%%         * Text type
%%           bsc  - B.Sc. Thesis
%%           msc  - M.Sc. Thesis (default)
%%           qual - PHD qualification (not tested yet)
%%           prop - PHD proposal (not tested yet)
%%           phd  - PHD thesis
%%
%%         * Media
%%           scr  - to eletronic version (PDF) / see the users guide
%%
%%         * Pagination
%%           oneside - unique face press
%%           twoside - two faces press
%%
%%		   * Line spacing
%%           singlespacing  - the same as using \linespread{1}
%%           onehalfspacing - the same as using \linespread{1.3}
%%           doublespacing  - the same as using \linespread{1.6}
%%
%% Reference commands. Use the following commands to make references in your
%% text:
%%          \figref  -- for Figure reference
%%          \tabref  -- for Table reference
%%          \eqnref  -- for equation reference
%%          \chapref -- for chapter reference
%%          \secref  -- for section reference
%%          \appref  -- for appendix reference
%%          \axiref  -- for axiom reference
%%          \conjref -- for conjecture reference
%%          \defref  -- for definition reference
%%          \lemref  -- for lemma reference
%%          \theoref -- for theorem reference
%%          \corref  -- for corollary reference
%%          \propref -- for proprosition reference
%%          \pgref   -- for page reference
%%
%%          Example: See \chapref{chap:introduction}. It will produce 
%%                   'See Chapter 1', in case of English language.

\documentclass[pt,twoside,onehalfspacing,bsc]{risethesis}

\usepackage[sort,square]{natbib} %package for bibliography
\bibliographystyle{ieeetr} %authordate1

\usepackage[utf8]{inputenc}
\usepackage[brazilian]{babel}
\usepackage[T1]{fontenc}

\usepackage{graphicx}

%% Change the following pdf author attribute name to your name.
\usepackage[linkcolor=blue,citecolor=blue,urlcolor=blue,colorlinks,pdfpagelabels,pdftitle={Bruno Cabral's Bachelor Thesis},pdfauthor={Bruno Cabral}]{hyperref}

\address{SALVADOR}

\universitypt{Universidade Federal da Bahia}
\universityen{Federal University of Bahia}

\departmentpt{Depertamento de Ciência da Computação}
\departmenten{Computer Science Department}

\programpt{Programa Multiinstitucional de Pós-graduação em Ciência da Computação}
\programen{Graduate in Computer Science}

\majorfieldpt{Ciência da Computação}
\majorfielden{Computer Science}

\title{Sistema de apoio à Pós graduação - UFBA}
\date{Outubro/2016}

\author{Victor de Azevedo Nunes}
\adviser{Ivan do Carmo Machado}

\begin{document}

\frontmatter
\frontpage
\presentationpage

\begin{dedicatory}
Eu dedico esta dissertação...
%I dedicate this dissertation to my family, girlfriend, friends and
%professors who gave me all necessary support to get here.
\end{dedicatory}

\acknowledgements
Meus agradecimentos...

\begin{epigraph}[]{Edward V Berard}
Walking on water and developing software from a specification are easy if both are frozen
\end{epigraph}

\resumo
% Escreva seu resumo no arquivo resumo.tex
\input{resumo}

\abstract
% Write your abstract in a file called abstract.tex
\input{abstract}

% Summary (tables of contents)
\tableofcontents

% List of figures
\listoffigures

% List of tables
\listoftables

% List of acronyms
% Acronyms manual: http://linorg.usp.br/CTAN/macros/latex/contrib/acronym/acronym.pdf
\listofacronyms
\input{acronyms}

% List of listings
%\lstlistoflistings

\mainmatter

\include{chapters/intro}
\include{chapters/referencial_teorico}

% \include{chapters/introduction/main}
% \include{chapters/background/main}
% \include{chapters/proposed_solution/main}
% \include{chapters/experiment/main}
% \include{chapters/conclusion/main}

%\bibliographystyle{natbib}
%\addcontentsline{toc}{chapter}{\bibliographytocname}
\bibliography{references}

% Appendix
\clearpage
\addappheadtotoc
\appendix
\appendixpage
% \include{appendix/experiment-instruments}

\end{document}
% %% RiSE Latex Template - version 0.5
%%
%% RiSE's latex template for thesis and dissertations
%% http://risetemplate.sourceforge.net
%%
%% (c) 2012 Yguaratã Cerqueira Cavalcanti (yguarata@gmail.com)
%%          Vinicius Cardoso Garcia (vinicius.garcia@gmail.com)
%%
%% This document was initially based on UFPEThesis template, from Paulo Gustavo
%% S. Fonseca.
%%
%% ACKNOWLEDGEMENTS
%%
%% We would like to thanks the RiSE's researchers community, the 
%% students from Federal University of Pernambuco, and other users that have
%% been contributing to this projects with comments and patches.
%%
%% GENERAL INSTRUCTIONS
%%
%% We strongly recommend you to compile your documents using pdflatex command.
%% It is also recommend use the texlipse plugin for Eclipse to edit your documents.
%%
%% Options for \documentclass command:
%%         * Idiom
%%           pt   - Portguese (default)
%%           en   - English
%%
%%         * Text type
%%           bsc  - B.Sc. Thesis
%%           msc  - M.Sc. Thesis (default)
%%           qual - PHD qualification (not tested yet)
%%           prop - PHD proposal (not tested yet)
%%           phd  - PHD thesis
%%
%%         * Media
%%           scr  - to eletronic version (PDF) / see the users guide
%%
%%         * Pagination
%%           oneside - unique face press
%%           twoside - two faces press
%%
%%		   * Line spacing
%%           singlespacing  - the same as using \linespread{1}
%%           onehalfspacing - the same as using \linespread{1.3}
%%           doublespacing  - the same as using \linespread{1.6}
%%
%% Reference commands. Use the following commands to make references in your
%% text:
%%          \figref  -- for Figure reference
%%          \tabref  -- for Table reference
%%          \eqnref  -- for equation reference
%%          \chapref -- for chapter reference
%%          \secref  -- for section reference
%%          \appref  -- for appendix reference
%%          \axiref  -- for axiom reference
%%          \conjref -- for conjecture reference
%%          \defref  -- for definition reference
%%          \lemref  -- for lemma reference
%%          \theoref -- for theorem reference
%%          \corref  -- for corollary reference
%%          \propref -- for proprosition reference
%%          \pgref   -- for page reference
%%
%%          Example: See \chapref{chap:introduction}. It will produce 
%%                   'See Chapter 1', in case of English language.

\documentclass[pt,twoside,onehalfspacing,bsc]{risethesis}

\usepackage[sort,square]{natbib} %package for bibliography
\bibliographystyle{ieeetr} %authordate1

\usepackage[utf8]{inputenc}
\usepackage[brazilian]{babel}
\usepackage[T1]{fontenc}

\usepackage{graphicx}

%% Change the following pdf author attribute name to your name.
\usepackage[linkcolor=blue,citecolor=blue,urlcolor=blue,colorlinks,pdfpagelabels,pdftitle={Bruno Cabral's Bachelor Thesis},pdfauthor={Bruno Cabral}]{hyperref}

\address{SALVADOR}

\universitypt{Universidade Federal da Bahia}
\universityen{Federal University of Bahia}

\departmentpt{Depertamento de Ciência da Computação}
\departmenten{Computer Science Department}

\programpt{Programa Multiinstitucional de Pós-graduação em Ciência da Computação}
\programen{Graduate in Computer Science}

\majorfieldpt{Ciência da Computação}
\majorfielden{Computer Science}

\title{Sistema de apoio à Pós graduação - UFBA}
\date{Outubro/2016}

\author{Victor de Azevedo Nunes}
\adviser{Ivan do Carmo Machado}

\begin{document}

\frontmatter
\frontpage
\presentationpage

\begin{dedicatory}
Eu dedico esta dissertação...
%I dedicate this dissertation to my family, girlfriend, friends and
%professors who gave me all necessary support to get here.
\end{dedicatory}

\acknowledgements
Meus agradecimentos...

\begin{epigraph}[]{Edward V Berard}
Walking on water and developing software from a specification are easy if both are frozen
\end{epigraph}

\resumo
% Escreva seu resumo no arquivo resumo.tex
\input{resumo}

\abstract
% Write your abstract in a file called abstract.tex
\input{abstract}

% Summary (tables of contents)
\tableofcontents

% List of figures
\listoffigures

% List of tables
\listoftables

% List of acronyms
% Acronyms manual: http://linorg.usp.br/CTAN/macros/latex/contrib/acronym/acronym.pdf
\listofacronyms
\input{acronyms}

% List of listings
%\lstlistoflistings

\mainmatter

\include{chapters/intro}
\include{chapters/referencial_teorico}

% \include{chapters/introduction/main}
% \include{chapters/background/main}
% \include{chapters/proposed_solution/main}
% \include{chapters/experiment/main}
% \include{chapters/conclusion/main}

%\bibliographystyle{natbib}
%\addcontentsline{toc}{chapter}{\bibliographytocname}
\bibliography{references}

% Appendix
\clearpage
\addappheadtotoc
\appendix
\appendixpage
% \include{appendix/experiment-instruments}

\end{document}
% %% RiSE Latex Template - version 0.5
%%
%% RiSE's latex template for thesis and dissertations
%% http://risetemplate.sourceforge.net
%%
%% (c) 2012 Yguaratã Cerqueira Cavalcanti (yguarata@gmail.com)
%%          Vinicius Cardoso Garcia (vinicius.garcia@gmail.com)
%%
%% This document was initially based on UFPEThesis template, from Paulo Gustavo
%% S. Fonseca.
%%
%% ACKNOWLEDGEMENTS
%%
%% We would like to thanks the RiSE's researchers community, the 
%% students from Federal University of Pernambuco, and other users that have
%% been contributing to this projects with comments and patches.
%%
%% GENERAL INSTRUCTIONS
%%
%% We strongly recommend you to compile your documents using pdflatex command.
%% It is also recommend use the texlipse plugin for Eclipse to edit your documents.
%%
%% Options for \documentclass command:
%%         * Idiom
%%           pt   - Portguese (default)
%%           en   - English
%%
%%         * Text type
%%           bsc  - B.Sc. Thesis
%%           msc  - M.Sc. Thesis (default)
%%           qual - PHD qualification (not tested yet)
%%           prop - PHD proposal (not tested yet)
%%           phd  - PHD thesis
%%
%%         * Media
%%           scr  - to eletronic version (PDF) / see the users guide
%%
%%         * Pagination
%%           oneside - unique face press
%%           twoside - two faces press
%%
%%		   * Line spacing
%%           singlespacing  - the same as using \linespread{1}
%%           onehalfspacing - the same as using \linespread{1.3}
%%           doublespacing  - the same as using \linespread{1.6}
%%
%% Reference commands. Use the following commands to make references in your
%% text:
%%          \figref  -- for Figure reference
%%          \tabref  -- for Table reference
%%          \eqnref  -- for equation reference
%%          \chapref -- for chapter reference
%%          \secref  -- for section reference
%%          \appref  -- for appendix reference
%%          \axiref  -- for axiom reference
%%          \conjref -- for conjecture reference
%%          \defref  -- for definition reference
%%          \lemref  -- for lemma reference
%%          \theoref -- for theorem reference
%%          \corref  -- for corollary reference
%%          \propref -- for proprosition reference
%%          \pgref   -- for page reference
%%
%%          Example: See \chapref{chap:introduction}. It will produce 
%%                   'See Chapter 1', in case of English language.

\documentclass[pt,twoside,onehalfspacing,bsc]{risethesis}

\usepackage[sort,square]{natbib} %package for bibliography
\bibliographystyle{ieeetr} %authordate1

\usepackage[utf8]{inputenc}
\usepackage[brazilian]{babel}
\usepackage[T1]{fontenc}

\usepackage{graphicx}

%% Change the following pdf author attribute name to your name.
\usepackage[linkcolor=blue,citecolor=blue,urlcolor=blue,colorlinks,pdfpagelabels,pdftitle={Bruno Cabral's Bachelor Thesis},pdfauthor={Bruno Cabral}]{hyperref}

\address{SALVADOR}

\universitypt{Universidade Federal da Bahia}
\universityen{Federal University of Bahia}

\departmentpt{Depertamento de Ciência da Computação}
\departmenten{Computer Science Department}

\programpt{Programa Multiinstitucional de Pós-graduação em Ciência da Computação}
\programen{Graduate in Computer Science}

\majorfieldpt{Ciência da Computação}
\majorfielden{Computer Science}

\title{Sistema de apoio à Pós graduação - UFBA}
\date{Outubro/2016}

\author{Victor de Azevedo Nunes}
\adviser{Ivan do Carmo Machado}

\begin{document}

\frontmatter
\frontpage
\presentationpage

\begin{dedicatory}
Eu dedico esta dissertação...
%I dedicate this dissertation to my family, girlfriend, friends and
%professors who gave me all necessary support to get here.
\end{dedicatory}

\acknowledgements
Meus agradecimentos...

\begin{epigraph}[]{Edward V Berard}
Walking on water and developing software from a specification are easy if both are frozen
\end{epigraph}

\resumo
% Escreva seu resumo no arquivo resumo.tex
\input{resumo}

\abstract
% Write your abstract in a file called abstract.tex
\input{abstract}

% Summary (tables of contents)
\tableofcontents

% List of figures
\listoffigures

% List of tables
\listoftables

% List of acronyms
% Acronyms manual: http://linorg.usp.br/CTAN/macros/latex/contrib/acronym/acronym.pdf
\listofacronyms
\input{acronyms}

% List of listings
%\lstlistoflistings

\mainmatter

\include{chapters/intro}
\include{chapters/referencial_teorico}

% \include{chapters/introduction/main}
% \include{chapters/background/main}
% \include{chapters/proposed_solution/main}
% \include{chapters/experiment/main}
% \include{chapters/conclusion/main}

%\bibliographystyle{natbib}
%\addcontentsline{toc}{chapter}{\bibliographytocname}
\bibliography{references}

% Appendix
\clearpage
\addappheadtotoc
\appendix
\appendixpage
% \include{appendix/experiment-instruments}

\end{document}
% %% RiSE Latex Template - version 0.5
%%
%% RiSE's latex template for thesis and dissertations
%% http://risetemplate.sourceforge.net
%%
%% (c) 2012 Yguaratã Cerqueira Cavalcanti (yguarata@gmail.com)
%%          Vinicius Cardoso Garcia (vinicius.garcia@gmail.com)
%%
%% This document was initially based on UFPEThesis template, from Paulo Gustavo
%% S. Fonseca.
%%
%% ACKNOWLEDGEMENTS
%%
%% We would like to thanks the RiSE's researchers community, the 
%% students from Federal University of Pernambuco, and other users that have
%% been contributing to this projects with comments and patches.
%%
%% GENERAL INSTRUCTIONS
%%
%% We strongly recommend you to compile your documents using pdflatex command.
%% It is also recommend use the texlipse plugin for Eclipse to edit your documents.
%%
%% Options for \documentclass command:
%%         * Idiom
%%           pt   - Portguese (default)
%%           en   - English
%%
%%         * Text type
%%           bsc  - B.Sc. Thesis
%%           msc  - M.Sc. Thesis (default)
%%           qual - PHD qualification (not tested yet)
%%           prop - PHD proposal (not tested yet)
%%           phd  - PHD thesis
%%
%%         * Media
%%           scr  - to eletronic version (PDF) / see the users guide
%%
%%         * Pagination
%%           oneside - unique face press
%%           twoside - two faces press
%%
%%		   * Line spacing
%%           singlespacing  - the same as using \linespread{1}
%%           onehalfspacing - the same as using \linespread{1.3}
%%           doublespacing  - the same as using \linespread{1.6}
%%
%% Reference commands. Use the following commands to make references in your
%% text:
%%          \figref  -- for Figure reference
%%          \tabref  -- for Table reference
%%          \eqnref  -- for equation reference
%%          \chapref -- for chapter reference
%%          \secref  -- for section reference
%%          \appref  -- for appendix reference
%%          \axiref  -- for axiom reference
%%          \conjref -- for conjecture reference
%%          \defref  -- for definition reference
%%          \lemref  -- for lemma reference
%%          \theoref -- for theorem reference
%%          \corref  -- for corollary reference
%%          \propref -- for proprosition reference
%%          \pgref   -- for page reference
%%
%%          Example: See \chapref{chap:introduction}. It will produce 
%%                   'See Chapter 1', in case of English language.

\documentclass[pt,twoside,onehalfspacing,bsc]{risethesis}

\usepackage[sort,square]{natbib} %package for bibliography
\bibliographystyle{ieeetr} %authordate1

\usepackage[utf8]{inputenc}
\usepackage[brazilian]{babel}
\usepackage[T1]{fontenc}

\usepackage{graphicx}

%% Change the following pdf author attribute name to your name.
\usepackage[linkcolor=blue,citecolor=blue,urlcolor=blue,colorlinks,pdfpagelabels,pdftitle={Bruno Cabral's Bachelor Thesis},pdfauthor={Bruno Cabral}]{hyperref}

\address{SALVADOR}

\universitypt{Universidade Federal da Bahia}
\universityen{Federal University of Bahia}

\departmentpt{Depertamento de Ciência da Computação}
\departmenten{Computer Science Department}

\programpt{Programa Multiinstitucional de Pós-graduação em Ciência da Computação}
\programen{Graduate in Computer Science}

\majorfieldpt{Ciência da Computação}
\majorfielden{Computer Science}

\title{Sistema de apoio à Pós graduação - UFBA}
\date{Outubro/2016}

\author{Victor de Azevedo Nunes}
\adviser{Ivan do Carmo Machado}

\begin{document}

\frontmatter
\frontpage
\presentationpage

\begin{dedicatory}
Eu dedico esta dissertação...
%I dedicate this dissertation to my family, girlfriend, friends and
%professors who gave me all necessary support to get here.
\end{dedicatory}

\acknowledgements
Meus agradecimentos...

\begin{epigraph}[]{Edward V Berard}
Walking on water and developing software from a specification are easy if both are frozen
\end{epigraph}

\resumo
% Escreva seu resumo no arquivo resumo.tex
\input{resumo}

\abstract
% Write your abstract in a file called abstract.tex
\input{abstract}

% Summary (tables of contents)
\tableofcontents

% List of figures
\listoffigures

% List of tables
\listoftables

% List of acronyms
% Acronyms manual: http://linorg.usp.br/CTAN/macros/latex/contrib/acronym/acronym.pdf
\listofacronyms
\input{acronyms}

% List of listings
%\lstlistoflistings

\mainmatter

\include{chapters/intro}
\include{chapters/referencial_teorico}

% \include{chapters/introduction/main}
% \include{chapters/background/main}
% \include{chapters/proposed_solution/main}
% \include{chapters/experiment/main}
% \include{chapters/conclusion/main}

%\bibliographystyle{natbib}
%\addcontentsline{toc}{chapter}{\bibliographytocname}
\bibliography{references}

% Appendix
\clearpage
\addappheadtotoc
\appendix
\appendixpage
% \include{appendix/experiment-instruments}

\end{document}
% %% RiSE Latex Template - version 0.5
%%
%% RiSE's latex template for thesis and dissertations
%% http://risetemplate.sourceforge.net
%%
%% (c) 2012 Yguaratã Cerqueira Cavalcanti (yguarata@gmail.com)
%%          Vinicius Cardoso Garcia (vinicius.garcia@gmail.com)
%%
%% This document was initially based on UFPEThesis template, from Paulo Gustavo
%% S. Fonseca.
%%
%% ACKNOWLEDGEMENTS
%%
%% We would like to thanks the RiSE's researchers community, the 
%% students from Federal University of Pernambuco, and other users that have
%% been contributing to this projects with comments and patches.
%%
%% GENERAL INSTRUCTIONS
%%
%% We strongly recommend you to compile your documents using pdflatex command.
%% It is also recommend use the texlipse plugin for Eclipse to edit your documents.
%%
%% Options for \documentclass command:
%%         * Idiom
%%           pt   - Portguese (default)
%%           en   - English
%%
%%         * Text type
%%           bsc  - B.Sc. Thesis
%%           msc  - M.Sc. Thesis (default)
%%           qual - PHD qualification (not tested yet)
%%           prop - PHD proposal (not tested yet)
%%           phd  - PHD thesis
%%
%%         * Media
%%           scr  - to eletronic version (PDF) / see the users guide
%%
%%         * Pagination
%%           oneside - unique face press
%%           twoside - two faces press
%%
%%		   * Line spacing
%%           singlespacing  - the same as using \linespread{1}
%%           onehalfspacing - the same as using \linespread{1.3}
%%           doublespacing  - the same as using \linespread{1.6}
%%
%% Reference commands. Use the following commands to make references in your
%% text:
%%          \figref  -- for Figure reference
%%          \tabref  -- for Table reference
%%          \eqnref  -- for equation reference
%%          \chapref -- for chapter reference
%%          \secref  -- for section reference
%%          \appref  -- for appendix reference
%%          \axiref  -- for axiom reference
%%          \conjref -- for conjecture reference
%%          \defref  -- for definition reference
%%          \lemref  -- for lemma reference
%%          \theoref -- for theorem reference
%%          \corref  -- for corollary reference
%%          \propref -- for proprosition reference
%%          \pgref   -- for page reference
%%
%%          Example: See \chapref{chap:introduction}. It will produce 
%%                   'See Chapter 1', in case of English language.

\documentclass[pt,twoside,onehalfspacing,bsc]{risethesis}

\usepackage[sort,square]{natbib} %package for bibliography
\bibliographystyle{ieeetr} %authordate1

\usepackage[utf8]{inputenc}
\usepackage[brazilian]{babel}
\usepackage[T1]{fontenc}

\usepackage{graphicx}

%% Change the following pdf author attribute name to your name.
\usepackage[linkcolor=blue,citecolor=blue,urlcolor=blue,colorlinks,pdfpagelabels,pdftitle={Bruno Cabral's Bachelor Thesis},pdfauthor={Bruno Cabral}]{hyperref}

\address{SALVADOR}

\universitypt{Universidade Federal da Bahia}
\universityen{Federal University of Bahia}

\departmentpt{Depertamento de Ciência da Computação}
\departmenten{Computer Science Department}

\programpt{Programa Multiinstitucional de Pós-graduação em Ciência da Computação}
\programen{Graduate in Computer Science}

\majorfieldpt{Ciência da Computação}
\majorfielden{Computer Science}

\title{Sistema de apoio à Pós graduação - UFBA}
\date{Outubro/2016}

\author{Victor de Azevedo Nunes}
\adviser{Ivan do Carmo Machado}

\begin{document}

\frontmatter
\frontpage
\presentationpage

\begin{dedicatory}
Eu dedico esta dissertação...
%I dedicate this dissertation to my family, girlfriend, friends and
%professors who gave me all necessary support to get here.
\end{dedicatory}

\acknowledgements
Meus agradecimentos...

\begin{epigraph}[]{Edward V Berard}
Walking on water and developing software from a specification are easy if both are frozen
\end{epigraph}

\resumo
% Escreva seu resumo no arquivo resumo.tex
\input{resumo}

\abstract
% Write your abstract in a file called abstract.tex
\input{abstract}

% Summary (tables of contents)
\tableofcontents

% List of figures
\listoffigures

% List of tables
\listoftables

% List of acronyms
% Acronyms manual: http://linorg.usp.br/CTAN/macros/latex/contrib/acronym/acronym.pdf
\listofacronyms
\input{acronyms}

% List of listings
%\lstlistoflistings

\mainmatter

\include{chapters/intro}
\include{chapters/referencial_teorico}

% \include{chapters/introduction/main}
% \include{chapters/background/main}
% \include{chapters/proposed_solution/main}
% \include{chapters/experiment/main}
% \include{chapters/conclusion/main}

%\bibliographystyle{natbib}
%\addcontentsline{toc}{chapter}{\bibliographytocname}
\bibliography{references}

% Appendix
\clearpage
\addappheadtotoc
\appendix
\appendixpage
% \include{appendix/experiment-instruments}

\end{document}

%\bibliographystyle{natbib}
%\addcontentsline{toc}{chapter}{\bibliographytocname}
\bibliography{references}

% Appendix
\clearpage
\addappheadtotoc
\appendix
\appendixpage
% \include{appendix/experiment-instruments}

\end{document}
% %% RiSE Latex Template - version 0.5
%%
%% RiSE's latex template for thesis and dissertations
%% http://risetemplate.sourceforge.net
%%
%% (c) 2012 Yguaratã Cerqueira Cavalcanti (yguarata@gmail.com)
%%          Vinicius Cardoso Garcia (vinicius.garcia@gmail.com)
%%
%% This document was initially based on UFPEThesis template, from Paulo Gustavo
%% S. Fonseca.
%%
%% ACKNOWLEDGEMENTS
%%
%% We would like to thanks the RiSE's researchers community, the 
%% students from Federal University of Pernambuco, and other users that have
%% been contributing to this projects with comments and patches.
%%
%% GENERAL INSTRUCTIONS
%%
%% We strongly recommend you to compile your documents using pdflatex command.
%% It is also recommend use the texlipse plugin for Eclipse to edit your documents.
%%
%% Options for \documentclass command:
%%         * Idiom
%%           pt   - Portguese (default)
%%           en   - English
%%
%%         * Text type
%%           bsc  - B.Sc. Thesis
%%           msc  - M.Sc. Thesis (default)
%%           qual - PHD qualification (not tested yet)
%%           prop - PHD proposal (not tested yet)
%%           phd  - PHD thesis
%%
%%         * Media
%%           scr  - to eletronic version (PDF) / see the users guide
%%
%%         * Pagination
%%           oneside - unique face press
%%           twoside - two faces press
%%
%%		   * Line spacing
%%           singlespacing  - the same as using \linespread{1}
%%           onehalfspacing - the same as using \linespread{1.3}
%%           doublespacing  - the same as using \linespread{1.6}
%%
%% Reference commands. Use the following commands to make references in your
%% text:
%%          \figref  -- for Figure reference
%%          \tabref  -- for Table reference
%%          \eqnref  -- for equation reference
%%          \chapref -- for chapter reference
%%          \secref  -- for section reference
%%          \appref  -- for appendix reference
%%          \axiref  -- for axiom reference
%%          \conjref -- for conjecture reference
%%          \defref  -- for definition reference
%%          \lemref  -- for lemma reference
%%          \theoref -- for theorem reference
%%          \corref  -- for corollary reference
%%          \propref -- for proprosition reference
%%          \pgref   -- for page reference
%%
%%          Example: See \chapref{chap:introduction}. It will produce 
%%                   'See Chapter 1', in case of English language.

\documentclass[pt,twoside,onehalfspacing,bsc]{risethesis}

\usepackage[sort,square]{natbib} %package for bibliography
\bibliographystyle{ieeetr} %authordate1

\usepackage[utf8]{inputenc}
\usepackage[brazilian]{babel}
\usepackage[T1]{fontenc}

\usepackage{graphicx}

%% Change the following pdf author attribute name to your name.
\usepackage[linkcolor=blue,citecolor=blue,urlcolor=blue,colorlinks,pdfpagelabels,pdftitle={Bruno Cabral's Bachelor Thesis},pdfauthor={Bruno Cabral}]{hyperref}

\address{SALVADOR}

\universitypt{Universidade Federal da Bahia}
\universityen{Federal University of Bahia}

\departmentpt{Depertamento de Ciência da Computação}
\departmenten{Computer Science Department}

\programpt{Programa Multiinstitucional de Pós-graduação em Ciência da Computação}
\programen{Graduate in Computer Science}

\majorfieldpt{Ciência da Computação}
\majorfielden{Computer Science}

\title{Sistema de apoio à Pós graduação - UFBA}
\date{Outubro/2016}

\author{Victor de Azevedo Nunes}
\adviser{Ivan do Carmo Machado}

\begin{document}

\frontmatter
\frontpage
\presentationpage

\begin{dedicatory}
Eu dedico esta dissertação...
%I dedicate this dissertation to my family, girlfriend, friends and
%professors who gave me all necessary support to get here.
\end{dedicatory}

\acknowledgements
Meus agradecimentos...

\begin{epigraph}[]{Edward V Berard}
Walking on water and developing software from a specification are easy if both are frozen
\end{epigraph}

\resumo
% Escreva seu resumo no arquivo resumo.tex
Este trabalho de conclusão de curso utiliza elementos da engenharia de software para propor um software como serviço a fim de otimizar o processo de construção e manutenção dos softwares. Assim, o objetivo deste SaaS é gerenciar as permissões de acesso de sistemas cliente, provendo o reuso de software e padronizando as soluções. O sistema proposto neste trabalho encontra-se implementado e disponível no Github, e traz fundamentos sobre a arquitetura e tecnologias utilizadas, bem como avaliações sobre possibilidades de utilização do mesmo em diversos ambientes, como Web e mobile.

\begin{keywords}
Software; Reuso; SaaS; Web; PHP
\end{keywords}

\abstract
% Write your abstract in a file called abstract.tex
My abstract...

\begin{keywords}
key words...
\end{keywords}

% Summary (tables of contents)
\tableofcontents

% List of figures
\listoffigures

% List of tables
\listoftables

% List of acronyms
% Acronyms manual: http://linorg.usp.br/CTAN/macros/latex/contrib/acronym/acronym.pdf
\listofacronyms
\begin{acronym}[ACRONYM] 
% Change the word ACRONYM above to change the acronym column width.
% The column width is equals to the width of the word that you put.
% Read the manual about acronym package for more examples:
%   http://linorg.usp.br/CTAN/macros/latex/contrib/acronym/acronym.pdf

\acro{SPA}{Single Page Application}
\acro{JSON}{Javascript Object Notation}
\acro{PHP}{PHP: Hypertext Preprocessor}
\acro{SaaS}{Software as a Service}
\acro{ERP}{Enterprise Resource Planning}
\acro{QoS}{Quality of Service}
\acro{UML}{Unified Modeling Language}
\acro{MVC}{Model-View-Controller}
\acro{Ajax}{Asynchronous Javascript and XML}
\acro{HTML}{HyperText Markup Language}
\acro{CSS}{Cascading Style Sheets}
\acro{API}{Application Programming Interface}
\acro{DOM}{Document Object Model}
\acro{BPMN}{Business Process Model and Notation}
\acro{REST}{Representational State Transfer}

\end{acronym}

% List of listings
%\lstlistoflistings

\mainmatter

\chapter{Introdução}

\section{Motivação}

Organizar os procedimentos de um processo sempre nos traz vantagens. Apesar de no processo de implantação de um sistema, o mesmo burocratizar o processo, com o tempo temos o retorno da dedicação para a inserção dos dados. Com um certo volume de dados, é possível estruturar informações que num processo manual são difíceis de serem enxergadas. Assim, é possível depender menos das pessoas que organizam o processo, pois o legado de informações não estará mais somente na mente de alguns, mas sim documentado nos dados do sistema.

Além de colaborar na organização, também haverá uma grande colaboração no tempo gasto na gestão. Lidar com muitos papéis e confiar na mente humana para guardar informações, não é uma alternativa muito segura devido ao fato que as pessoas sempre estão sujeitas a sair do processo e levar contigo a experiência obtida. Experiência essa que faz com que os procedimentos sejam executados de forma mais eficiente. Entretanto, com um sistema inteligente, é possível auxiliar e tornar mais ágil a execução das tarefas.


\section{Problema}


De acordo com funcionários ligados ao o setor de pós graduação da UFBA, entrevistados a fim de um maior entendimento do cenário, apesar das semelhanças estruturais, a pós graduação gerida de forma diferencia da graduação. FULANO afirma que devido ao fato de não ter a mesma visibilidade, não tem acesso aos mesmos recursos de gestão acadêmica da graduação. O professores não executam somente atividades dentro da sala de aula, também tem diversas outras ocupações no setor. E muitos procedimentos realizados extra classe ainda se encontram sendo realizados de forma manual, estando mais vulnerável ao erro ou até mesmo à violação do processo. Também ocorre um grande desperdício de tempo pelos professores e gestores da área, devido ao diversos processos ainda realizados de forma manual, sem a devida documentação. Segundo FULANO, também entrevistado, esse tempo perdido implica numa redução da eficiência na sala de aula, pois o professor acaba por ter menos tempo disponível para o planejamento das atividades, o que gera impactos negativos aos alunos.


\section{Objetivos} %<o que deve ser feito/entregue>


Devido aos muitos processos sendo resolvidos de forma manual, propõe-se com solução um sistema moderno, arquitetado para ter funcionamento na web e com um módulo mobile, a fim de fornecer informações de forma rápida e eficiente para os professores através de notificações, já que o acesso à internet móvel é comum entre os possíveis usuários do sistema em questão.
O principal requisito para o sistema seria dispor recursos para reduzir o tempo desperdiçado pelos professores durante as atividades extra classe.


\section{Metodologia} %<como será feito | como resolver o problema apontado inicialmente>


%<analise de literatura | design | implementação | validação>
Baseando-se nas tecnologias gratuitas em alta no cenário atual do desenvolvimento web, dispomos de algumas opções eficientes para a implementação da solução. Dentre as possibilidades, considerando a facilidade para futura manutenção e continuidade do projeto, tende-se a optar por uma tecnologia popular. Como linguagem de programação, adota-se o PHP. A escolha é fundamentada de acordo com a pesquisa da RedMonk de 2015, que evidencia o uso das linguagens de programação de acordo com as discussões no StackOverflow e repositórios no GitHub. É possível constatar a popularidade do PHP no cenário atual com o gráfico da pesquisa citada, na qual o PHP é apresentado na terceira colocação, apenas atrás do lider JavaScript e do segundo colocado, o Java.

\begin{figure}
	\label{fig:graficoRedmonk}
	\includegraphics[width=1\textwidth]{img/grafico_redmonk}
	\caption{Ranking das liguagens de programação no Stack Overflow e Github}
\end{figure}


Ainda assim, para compor a interface do dado projeto, também ocorrerá o uso do líder JavaScript de forma intensa, provendo o elo com o as informações gerenciadas pelo PHP.


Entretanto, não seria inteligente desenvolver um sistema completo sem o auxílio de um framework. Dentre os frameworks disponíveis para PHP, hoje o destaque está com o Laravel, que se encontra no topo dentre os mais utilizados no momento. 


A WebHostFace, uma empresa de hospedagem, compilou várias estatísticas para criar um infográfico mostrando os frameworks PHP mais populares de 2015. Utilizando informações sobre os próprios clientes, o Google Trends, estatísticas de repositórios do GitHub e a pesquisa do SitePoint “Best PHP Frameworks 2015”, a WebHostFace elaborou o seguinte infográfico: 

\begin{figure}
	\label{fig:graficoWebhostface}
	\includegraphics[width=1\textwidth]{img/infografico_webhostface}
	\caption{Infográfico da WebhostFace, exibindo a popularidade dos Frameworks PHP em 2015}
\end{figure}

Assim, tem-se a evidência que o Laravel em 2015 teve a maior popularidade em projetos pessoais e tem a maior comunidade entre os concorrentes, o que o torna uma boa escolha para a escrita de um software que será continuado por terceiros.


Para elaborar os recursos de interface e integrar ao back-end PHP do sistema, será adotado o já conhecido AngularJS, ferramenta sólida e conhecida no aspecto em questão. 


Dados coletados via Google Trends, que propõe comparações entre termos pesquisados, revela a popularidade do AngularJs diante de alguns dos principais concorrentes. O gráfico abaixo evidencia o cenário.


%Como mostra a Figura \ref{fig:graficoGoogleTrendsFerramentasFront}. 
\begin{figure}
	\label{fig:graficoGoogleTrendsFerramentasFront}
	\includegraphics[width=1\textwidth]{img/grafico_ferramentas_front}
	\caption{Gráfico do Google Trends exibindo as pesquisas por ferramentas front-end}
\end{figure}


Junto ao Angular JS, será utilizada a agradável tendência de interface do Material Design da Google, que propõe layouts limpos e otimizados já conhecidos pelos usuários de smartphones Android. 


Para a elaboração da plataforma mobile do projeto, será utilizado o Ionic Framework, muito difundido e bastante pesquisado na área, o que fica evidenciado com o gráfico de pesquisbaixo, coletado via Google Trends buscando por frameworks de desenvolvimento híbrido mobile.


\begin{figure}
	\label{fig:graficoGoogleTrendsFerramentasHibridasMobile}
	\includegraphics[width=1\textwidth]{img/grafico_ferramentas_hibridas_mobile}
	\caption{Gráfico do Google Trends exibindo as pesquisas por Frameworks híbridos mobile}
\end{figure}	

Para layout da interface mobile, também será aplicado a tendência do Material Design, a fim de propor uma harmonia entre o módulo web e mobile para os usuários


\section{Resultados Esperados}


Como fruto de um sistema para pós-graduação da UFBA, espera-se que os professores tenham mais recursos para integrar as atividades e também prover melhores condições para acompanhamento da vida acadêmica dos alunos em questão. Também, que os novos colaboradores que entrarem no processo tenham facilidade de compreender o fluxo do setor ao navegar pelo sistema proposto.


\section{Fora de Escopo}


Interação com os alunos devido às complicações para realizar a integração com o sistema empregado na UFBA, gerenciado pela XXXXXX, o que causaria uma inviabilidade no projeto devido à necessidade de entrega do produto ser mais forte que o tempo necessário para executar o processo de obtenção de acesso ao sistema legado para realizar a integração.


\section{Estrutura do Trabalho}


<breve resumo sobre os capítulos do TCC>
\chapter{Referencial Teórico}


Projetar o desenvolvimento de um software requer muito planejamento, pois as falhas iniciais podem custar bastante caro ou até mesmo inviabilizar a continuação de um projeto. Assim, a escolha da arquitetura ideal para a aplicabilidade é essencial na concepção de um produto de software. 
De todo o modo, sempre busca-se fazer mais com menos. Diante de tal filosofia, temos neste capítulo, uma breve discussão sobre alguns elementos de projeto e arquitetura de software, a fim de contextualizar este trabalho de conclusão de curso.
O capítulo corrente é composto por quatro seções. A \ref{sec:saas} trata de Software como serviço, discutindo alguns elementos do contexto que são relevantes para o trabalho proposto. A \ref{sec:reuso}  discute sobre a empregabilidade do reuso de software. A \ref{sec:modularizacao} seção, trata sobre aspectos envolvidos na modularização dos softwares. Por fim, a \ref{sec:apps_web} aborda as aplicações web, discutindo sobre aspectos relevantes sobre a aplicação web que compõe este trabalho.


 \section{Software como serviço}\label{sec:saas}


Segundo La e Chun \citep{La2009Systematic}, o princípio da definição de Software como um Serviço (Sofware as a Service - SaaS) é um serviço complementar para aplicações da computação em nuvem (cloud computing). No entanto, as áreas não se confundem. SaaS deve ser entendido como um mecanismo de suporte às soluções existentes na cloud. Os SaaS existem justamente para maximizar o reuso de serviços repetidos e não centrais em uma aplicação remota.


Como vantagens, diversos fatores podem ser favoráveis para a adoção de um SaaS, como custo e manutenção dentre outros fatores aplicáveis a soluções específicas. Lechesaet al. \citep{LechesaSS11}, quantificam índices sobre os fatores determinantes para adoção ou não de um SaaS voltado para ERP na África do Sul. Os principais fatores determinantes para adoção desse mecanismo de software são sua fluidez quanto à rede e a segurança. Esses fatores estão presentes na aplicação desenvolvida neste trabalho de conclusão de curso.


Devido ao fato de ter um serviço constantemente na nuvem, fica o questionamento sobre a segurança da informação manipulada. Sabe-se que a vulnerabilidade na Web não é restrita ao SaaS, atingindo diversos âmbitos. Rai et al. \cite{journals/corr/RaiSM13} consideram que o avanço da computação em nuvem não é um problema apenas para os serviços Web do ponto de vista da segurança, pois muitos trabalhos na literatura mostram a área como mais um ponto de vulnerabilidade para diversos setores, a exemplo de infraestrutura. No mesmo artigo mencionado de Rai et al. \cite{journals/corr/RaiSM13}, também realizaram-se estudos exploratórios junto a empresas usuárias de serviços em computação em nuvem e consideram que a perspectiva de SaaS também pode fortalecer a segurança nas aplicações de cloud computing, pois o software de autenticação compartilhado por várias aplicações em nuvem, oferece uma melhor padronização e consequente facilidade de prevenção a erros de vulnerabilidade específicas de cada módulo da pesquisa. Esse ponto de vista é fundamenta para a compreensão de SaaS.


A arquitetura de armazenamento de dados de um Saas pode variar de acordo com a necessidade do contexto. Huixin \cite{7586486} descreve possíveis modelagens nesse sentido. Essa abordagem pode ser com um banco de dados único, fazendo com que diferentes clientes compartilhem o mesmo banco, diferindo os dados através de controle de usuário, ou isolando os diferentes clientes através de bancos de dados exclusivos para cada um. Esse fator também pode ser combinado com a arquitetura da aplicação, caso ofereça aplicação única para todos os clientes ou aplicação compartilhada. Diante das possíveis abordagens, a modelagem de dados do software pode ser decidida pela regra de negócio. Este trabalho optou por aplicação única e banco de dados compartilhado.

Devido ao conceito particular de obtenção de software, tanto pela visão do cliente como pela visão do vendedor, é necessário compreender os diversos aspectos que podem ser relevantes ao orçar um Saas. O recente trabalho de T. Kaur et al. \citep{6949281} orienta um modelo para compor o custo de um Saas. O custo total seria composto pelos fatores que dão suporte ao funcionamento do software. Tais fatores incluem infra-estrutura, configurabilidade, customização, parâmetros de QoS (Quality of service) como escalabilidade, disponibilidade, usabilidade, pontualidade e desempenho da resposta, portabilidade, custo total de propriedade e retorno do investimento. Esses fatores caracterizam o custo de forma eficaz, possibilitando ao fornecedor, prover um Serviço de acordo com a exigência do consumidor em vários pacotes de serviços.


O conceito de software como serviço se aplica neste trabalho de conclusão de curso, pois o mesmo estará disponível na web com alta disponibilidade adotando as características apresentadas para qualquer pessoa que desejar usá-lo. Assim, basta se cadastrar, configurá-lo e usar como um serviço, sem preocupação com a manutenção do mesmo.


\section{Reuso de software}\label{sec:reuso} %CRUISE BOOK CAPITULO 2


De acordo com o livro Software pratical reuse \citep{Ezran2002}, o reuso de software é a utilização de qualquer informação que um desenvolvedor pode necessitar no processo de criação de software. O livro de Basili e Rombach \citep{Basili1991} define reutilização de software como o uso de tudo o que está associado a um projeto de conhecimento.
Assim, o objetivo da reutilização de software é reciclar o design, código e outros componentes de um produto de software e assim reduzir o custo, o tempo e melhorar a qualidade do produto.
Segundo Keswani et al. \cite{6783445}, o componente reutilizável de software pode ser qualquer parte de seu desenvolvimento, como um fragmento de código, design, casos de teste, ou até mesmo a especificação de requisitos de uma funcionalidade do software. 

O reuso de software pode ter impacto positivo em diversos aspectos do software, vejamos alguns, conforme apresentados no C.R.U.I.S.E Book \citep{cruise2007} :

\begin{itemize}

\item Qualidade: As correções de erro tornam-se úteis em todos os locais em que ocorreu, padronizando e facilitando a manutenção.

\item Produtividade: O ganho de produtividade é alcançado devido ao menor número de artefatos desenvolvido. Isso resulta em menor esforço de teste e também análise e design, reduzindo custos.

\item Confiabilidade: A utilização de componentes bem testados aumenta a
confiança no software. Além disso, a utilização de um mesmo componente em vários sistemas, aumenta a possibilidade de detecção de erros e reforça a confiança no componente.

\item Redução do Esforço: A reutilização de software proporciona uma redução do tempo de desenvolvimento, o que reduz o tempo necessário para o produto ser disponibilizado no mercado para trazer rentabilidade.

\item Trabalho redundante e tempo de desenvolvimento: Desenvolver um sistema do
zero significa desenvolvimento redundante de muitos componentes, como requisitos, especificações, casos de uso, arquitetura, etc. Isso pode ser evitado quando estes estão disponíveis como componentes reutilizáveis e podem ser compartilhados, resultando em um processo de desenvolvimento otimizado.

\item Documentação: Embora a documentação seja muito importante para a
manutenção de um sistema, muitas vezes é negligenciada. A reutilização de componentes de software reduz a quantidade de documentação a ser escrita, entretanto depende da qualidade do que está escrito. Assim, apenas a estrutura do sistema e os novos artefatos desenvolvidos necessitam ser documentados.

\item Custo de manutenção: Menos defeitos e manutenções são esperados quando tem-se comprovada a qualidade dos componentes utilizados.

\item Tamanho da equipe: É comum haver casos em que a equipe de desenvolvimento sofre sobrecarga. Entretanto, dobrar o tamanho da equipe de desenvolvimento não necessariamente duplica produtividade. Se muitos componentes podem ser reutilizados, é possível desenvolver com equipes menores, levando a melhor comunicação e aumento da produtividade.

\end{itemize}

Apesar dos benefícios da reutilização de software, ela não é suficientemente aproveitada. Existem fatores que influenciam direta ou indiretamente na sua adoção. Esses fatores podem ser de aspecto gerencial, organizacional, econômico, conceitual ou técnico. Veremos a seguir alguns aspectos que podem gerar conflito com a cultura de reuso de software, segundo o C.R.U.I.S.E Book \citep{cruise2007} :
%(Sametinger, 1997). REVER

\begin{itemize}
	
\item Falta de apoio da gestão: Como a reutilização de software gera custos iniciais,
a medida pode não ser amplamente alcançada em uma organização sem o apoio de alto nível de gestão. Os gestores têm de ser informados sobre os custos iniciais e serem convencidos sobre economias futuras.

\item Gerenciamento do Projeto: Gerenciar projetos tradicionais é uma tarefa árdua, principalmente, os que praticam a reutilização de software. Utilizando a técnica em larga escala, tem-se impacto sobre todo o ciclo de vida do software.

\item Estruturas organizacionais inadequadas: As estruturas organizacionais devem
considerar diferentes necessidades que surgem quando a reutilização em larga escala está sendo adotada. Por exemplo, uma equipe particionada pode ser alocada somente para desenvolver, manter e certificar componentes reutilizáveis de software.

\item Incentivos de gestão: É comum a falta de incentivo para deixar os desenvolvedores gastarem tempo elaborando componentes do sistemas. A produtividade é muitas vezes medida apenas no tempo necessário para concluir um projeto. Assim, fazer qualquer trabalho além disso, embora benéfico para a empresa como um todo, diminui o seu sucesso. Mesmo quando os componentes reutilizáveis são utilizados, os benefícios obtidos são uma pequena fração do que poderia ser alcançado caso houvesse reutilização explícita, planejada e organizada.

\item Dificuldade de encontrar software reutilizável: Para reutilizar os componentes, devem existir formas eficientes de busca. Além disso, é importante ter um repositório bem organizado contendo componentes com um eficiente meio de acesso.

\item Não reutilização do software encontrado. O acesso fácil ao software existente
não necessariamente aumentar a reutilização. Os componentes reutilizáveis devem ser cuidadosamente especificados, projetados, implementados e documentados, pois em alguns casos, modificar e adaptar o código  pode ser mais custoso que a programação da funcionalidade necessária a partir do zero.

\item Modificação: É muito difícil encontrar um componente que funcione
exatamente da mesma maneira que queremos. Desta forma, são necessárias modificações e devem existir formas de determinar os seus efeitos sobre o componente.


\end{itemize}


A definição do reuso de software pode ser associada a este trabalho, pois ainda que como um serviço, estaremos adotando uma aplicação única para servir aos mais diversos softwares de uma empresa, por exemplo. Logo, ao adotar o projeto, pode ser interessante aplicar algumas métricas para tomar conhecimento de possíveis vantanjens como as citadas nessa seção, a exemplo de economia de tempo de desenvolvimento, e custo do projeto.


%Outra diretriz importante para a reutilização de software é reduzir o risco na criação de novos softwares. O risco tende a ser bastante reduzido se os componentes que estão sendo reutilizados têm as documentação, interfaces necessárias e devidamente testadas, fatores que contibruem para uma fácil integração.
%De acordo com Keswani et al. \citep{6783445}, para o reuso de software dar retornos apropriados, o processo deve ser sistemático e planejado. Qualquer organização que implemente a reutilização de software deve identificar os melhores métodos e estratégias de reutilização para obter a máxima produtividade. A reutilização de software ajuda a evitar software de engenharia a partir do zero, pois usa módulos de software existentes. A reutilização de software, embora seja uma tarefa difícil, especialmente para softwares antigos sem padrões de projeto, pode melhorar significativamente a produtividade e a qualidade de um produto de software. Embora a reutilização de software não seja um novo campo, ela pode dar grandes retornos em curto período de tempo.


\section{Modularização}\label{sec:modularizacao} %artigo de claudio pagina 222 introdução


%A modularidade vem desempenhando um papel predominante estágios emergentes das disciplinas de arquitetura de software [13]. Engenheiros de software consideram modularidade como princípio base na comparação entre arquiteturas alternativas  e arquitetura degeneração [9]. De fato, os engenheiros de software são incentivados a arquitecturas, baseando-se numa multiplicidade de mecanismos de modularidade disponíveis em: 
%(i) Linguagens de descrição de arquitetura (ADLs), como ACME [8], 
%(ii) catálogos de arquitetônicos [2, 13], e 
%(iii) conhecem bem princípios de alto nível, como interfaces de componentes estreitos, acoplamento arquitectónico reduzido e semelhantes.


Conforme é frisado por Wickramaarachchi e Lai \citep{7062705}, o conceito de modularização na indústria de software tem uma longa história e tem sido utilizado para melhorar o processo de desenvolvimento de software em diferentes estágios. Os principais conceitos por trás da modularização do software foram introduzidos por pesquisadores pioneiros há quarenta anos, com uma notável contribuição feita por Melvin Conway e David Parnas, que tem representação notável na engenharia de software.


Modularizar um software é um bom padrão a ser adotado. Segundo Wickramaarachchi e Lai \citep{7062705}, a modularização é importante na identificação de dependências e reduz as dificuldades diante de uma possível necessidade de grandes alterações. De uma perspectiva da engenharia de software, uma modularização geralmente tem várias vantagens, tais como: tornar a complexidade do software mais gerenciável, facilitar o trabalho paralelo e tornar o software mais maleável para acomodar o futuro incerto que um software pode ter. O objetivo final da modularização do software é aumentar a produtividade ea qualidade do software. Tal conceito encontra-se bastante difundido e estái incorporado em linguagens de programação e ferramentas de software. O presente trabalho favorece ao uso da modularização de um software e até mesmo pode ser considerado um módulo a ser acoplado a qualquer software, mediante a compatibilidade.


Ao realizar a adoção do trabalho proposto, fica bastante evidente a "responsabilidade" do mesmo no escopo do projeto. Assim, é possível tratá-lo como um módulo do projeto que o usa, mesmo que seja consumido como um serviço.


\section{Aplicações web}\label{sec:apps_web}


A popularidade das soluções Web aumentou exponencialmente na última década e todos os dias cresce o número de pessoas usuárias desse tipo de software. E seguindo um padrão próprio, Kumar et al. \citep{7813710} sugerem que para o desenvolvimento web, deve-se manter a prática eficaz de produzir diagramas UML. A abordagem baseada na web oferece uma maneira fácil e eficaz para gerenciar e controlar o processo de desenvolvimento por meio de artefatos de modelagem. Tal abordagem pode ser usada quando há uma exigência de lidar com mudanças muito rápidas e grandes em requisitos de forma muito eficaz em muito menos tempo, gerando assim um menor impacto. 


Para atender à fomentada demanda de aplicativos web, é necessário adotar métodos de desenvolvimentos que sejam ágeis, eficientes e de fácil manutenção. Yu Ping et al. \cite{1372143} propõem o uso do modelo MVC (Model, View e Controller) no desenvolvimento para softwares web. O modelo apresentado tornou-se um padrão popular e divide o software em camadas com propósito definido, tornando-o de mais fácil manutenção.


O Ajax (Asynchronous Javascript and XML) revolucionou a web. Conforme demonstrado por Yuping \citep{6845605}, ao usar a tecnologia Ajax, podemos enriquecer a experiência do usuário em aplicações baseadas em navegador de internet, e fornecer uma variedade de aplicações interativas para atender às necessidade de humanização das aplicações.
Os aplicativos Ajax em execução no navegador se comunicam com um servidor Web de forma assíncrona e atualizam apenas uma parte da página.


No artigo de Tesarik et al. \citep{4664349}, temos que o design de software SPA (Single page application) é uma maneira interessante de criar um software Web numa uma única página. Essa solução de página única sem navegação funciona apenas com base em técnicas dinâmicas e assíncronas, como o AJAX (citado no parágrafo anterior). No entanto, esta abordagem coloca o desenvolvedor antes de alguns desafios substanciais.
Para projetar a interface do usuário que mostra as informações, é importante projetar corretamente a tela para manipular os dados do aplicativo numa única página. O design da página deve ser elaborado para maximizar a decomposição da página em componentes distintos que encapsulam os principais casos de uso.
Também recomenda-se que para elaborar os artefatos visuais, explore os mais ricos recursos na implementação, como HTML5, JavaScript, Ajax, CSS3, e outras tecnologias que se apliquem. Entretanto, desenvolver uma rica interface com o uso de diversas tecnologias/frameworks pode ocasionar um esforço maior para explorar as possibilidades.
Tais componentes da interface de uma aplicação SPA normalmente são alimentados mediante o consumo de uma API Rest via requisições AJAX. Assim, é possível particionar as responsabilidades de processamento do software entre cliente e servidor.


Como é possível denotar no trabalho de Salvadori e Siqueire \citep{7195633}, REST é uma arquitetura muito popular para integração de aplicativos web, que permite compartilhar e reutilizar informações através de sistemas. Aplicações de grande escala baseadas em no REST deve ser implementadas utilizando estratégias e mecanismos para produzir sistemas que sejam fáceis de desenvolver, reutilizar e manter. As interfaces de integração fornecidas por esses sistemas, chamadas de Web API, têm uma influência importante nas características da implementação resultante, pois a sua forma de resposta deve ser casada com a interface que o consome.
 
 
Os elementos comentados nessa seção encontram-se presentes neste trabalho de conclusão de curso. Juntos, montam a estrutura tecnológica necessária aliada à arquitetura adotada. Esses elementos seguem tendências atuais dos softwares web, fazendo com que esse trabalho esteja composto por tecnologias modernas que estão com boa aceitação no mercado.

% %% RiSE Latex Template - version 0.5
%%
%% RiSE's latex template for thesis and dissertations
%% http://risetemplate.sourceforge.net
%%
%% (c) 2012 Yguaratã Cerqueira Cavalcanti (yguarata@gmail.com)
%%          Vinicius Cardoso Garcia (vinicius.garcia@gmail.com)
%%
%% This document was initially based on UFPEThesis template, from Paulo Gustavo
%% S. Fonseca.
%%
%% ACKNOWLEDGEMENTS
%%
%% We would like to thanks the RiSE's researchers community, the 
%% students from Federal University of Pernambuco, and other users that have
%% been contributing to this projects with comments and patches.
%%
%% GENERAL INSTRUCTIONS
%%
%% We strongly recommend you to compile your documents using pdflatex command.
%% It is also recommend use the texlipse plugin for Eclipse to edit your documents.
%%
%% Options for \documentclass command:
%%         * Idiom
%%           pt   - Portguese (default)
%%           en   - English
%%
%%         * Text type
%%           bsc  - B.Sc. Thesis
%%           msc  - M.Sc. Thesis (default)
%%           qual - PHD qualification (not tested yet)
%%           prop - PHD proposal (not tested yet)
%%           phd  - PHD thesis
%%
%%         * Media
%%           scr  - to eletronic version (PDF) / see the users guide
%%
%%         * Pagination
%%           oneside - unique face press
%%           twoside - two faces press
%%
%%		   * Line spacing
%%           singlespacing  - the same as using \linespread{1}
%%           onehalfspacing - the same as using \linespread{1.3}
%%           doublespacing  - the same as using \linespread{1.6}
%%
%% Reference commands. Use the following commands to make references in your
%% text:
%%          \figref  -- for Figure reference
%%          \tabref  -- for Table reference
%%          \eqnref  -- for equation reference
%%          \chapref -- for chapter reference
%%          \secref  -- for section reference
%%          \appref  -- for appendix reference
%%          \axiref  -- for axiom reference
%%          \conjref -- for conjecture reference
%%          \defref  -- for definition reference
%%          \lemref  -- for lemma reference
%%          \theoref -- for theorem reference
%%          \corref  -- for corollary reference
%%          \propref -- for proprosition reference
%%          \pgref   -- for page reference
%%
%%          Example: See \chapref{chap:introduction}. It will produce 
%%                   'See Chapter 1', in case of English language.

\documentclass[pt,twoside,onehalfspacing,bsc]{risethesis}

\usepackage[sort,square]{natbib} %package for bibliography
\bibliographystyle{ieeetr} %authordate1

\usepackage[utf8]{inputenc}
\usepackage[brazilian]{babel}
\usepackage[T1]{fontenc}

\usepackage{graphicx}

%% Change the following pdf author attribute name to your name.
\usepackage[linkcolor=blue,citecolor=blue,urlcolor=blue,colorlinks,pdfpagelabels,pdftitle={Bruno Cabral's Bachelor Thesis},pdfauthor={Bruno Cabral}]{hyperref}

\address{SALVADOR}

\universitypt{Universidade Federal da Bahia}
\universityen{Federal University of Bahia}

\departmentpt{Depertamento de Ciência da Computação}
\departmenten{Computer Science Department}

\programpt{Programa Multiinstitucional de Pós-graduação em Ciência da Computação}
\programen{Graduate in Computer Science}

\majorfieldpt{Ciência da Computação}
\majorfielden{Computer Science}

\title{Sistema de apoio à Pós graduação - UFBA}
\date{Outubro/2016}

\author{Victor de Azevedo Nunes}
\adviser{Ivan do Carmo Machado}

\begin{document}

\frontmatter
\frontpage
\presentationpage

\begin{dedicatory}
Eu dedico esta dissertação...
%I dedicate this dissertation to my family, girlfriend, friends and
%professors who gave me all necessary support to get here.
\end{dedicatory}

\acknowledgements
Meus agradecimentos...

\begin{epigraph}[]{Edward V Berard}
Walking on water and developing software from a specification are easy if both are frozen
\end{epigraph}

\resumo
% Escreva seu resumo no arquivo resumo.tex
\input{resumo}

\abstract
% Write your abstract in a file called abstract.tex
\input{abstract}

% Summary (tables of contents)
\tableofcontents

% List of figures
\listoffigures

% List of tables
\listoftables

% List of acronyms
% Acronyms manual: http://linorg.usp.br/CTAN/macros/latex/contrib/acronym/acronym.pdf
\listofacronyms
\input{acronyms}

% List of listings
%\lstlistoflistings

\mainmatter

\include{chapters/intro}
\include{chapters/referencial_teorico}

% \include{chapters/introduction/main}
% \include{chapters/background/main}
% \include{chapters/proposed_solution/main}
% \include{chapters/experiment/main}
% \include{chapters/conclusion/main}

%\bibliographystyle{natbib}
%\addcontentsline{toc}{chapter}{\bibliographytocname}
\bibliography{references}

% Appendix
\clearpage
\addappheadtotoc
\appendix
\appendixpage
% \include{appendix/experiment-instruments}

\end{document}
% %% RiSE Latex Template - version 0.5
%%
%% RiSE's latex template for thesis and dissertations
%% http://risetemplate.sourceforge.net
%%
%% (c) 2012 Yguaratã Cerqueira Cavalcanti (yguarata@gmail.com)
%%          Vinicius Cardoso Garcia (vinicius.garcia@gmail.com)
%%
%% This document was initially based on UFPEThesis template, from Paulo Gustavo
%% S. Fonseca.
%%
%% ACKNOWLEDGEMENTS
%%
%% We would like to thanks the RiSE's researchers community, the 
%% students from Federal University of Pernambuco, and other users that have
%% been contributing to this projects with comments and patches.
%%
%% GENERAL INSTRUCTIONS
%%
%% We strongly recommend you to compile your documents using pdflatex command.
%% It is also recommend use the texlipse plugin for Eclipse to edit your documents.
%%
%% Options for \documentclass command:
%%         * Idiom
%%           pt   - Portguese (default)
%%           en   - English
%%
%%         * Text type
%%           bsc  - B.Sc. Thesis
%%           msc  - M.Sc. Thesis (default)
%%           qual - PHD qualification (not tested yet)
%%           prop - PHD proposal (not tested yet)
%%           phd  - PHD thesis
%%
%%         * Media
%%           scr  - to eletronic version (PDF) / see the users guide
%%
%%         * Pagination
%%           oneside - unique face press
%%           twoside - two faces press
%%
%%		   * Line spacing
%%           singlespacing  - the same as using \linespread{1}
%%           onehalfspacing - the same as using \linespread{1.3}
%%           doublespacing  - the same as using \linespread{1.6}
%%
%% Reference commands. Use the following commands to make references in your
%% text:
%%          \figref  -- for Figure reference
%%          \tabref  -- for Table reference
%%          \eqnref  -- for equation reference
%%          \chapref -- for chapter reference
%%          \secref  -- for section reference
%%          \appref  -- for appendix reference
%%          \axiref  -- for axiom reference
%%          \conjref -- for conjecture reference
%%          \defref  -- for definition reference
%%          \lemref  -- for lemma reference
%%          \theoref -- for theorem reference
%%          \corref  -- for corollary reference
%%          \propref -- for proprosition reference
%%          \pgref   -- for page reference
%%
%%          Example: See \chapref{chap:introduction}. It will produce 
%%                   'See Chapter 1', in case of English language.

\documentclass[pt,twoside,onehalfspacing,bsc]{risethesis}

\usepackage[sort,square]{natbib} %package for bibliography
\bibliographystyle{ieeetr} %authordate1

\usepackage[utf8]{inputenc}
\usepackage[brazilian]{babel}
\usepackage[T1]{fontenc}

\usepackage{graphicx}

%% Change the following pdf author attribute name to your name.
\usepackage[linkcolor=blue,citecolor=blue,urlcolor=blue,colorlinks,pdfpagelabels,pdftitle={Bruno Cabral's Bachelor Thesis},pdfauthor={Bruno Cabral}]{hyperref}

\address{SALVADOR}

\universitypt{Universidade Federal da Bahia}
\universityen{Federal University of Bahia}

\departmentpt{Depertamento de Ciência da Computação}
\departmenten{Computer Science Department}

\programpt{Programa Multiinstitucional de Pós-graduação em Ciência da Computação}
\programen{Graduate in Computer Science}

\majorfieldpt{Ciência da Computação}
\majorfielden{Computer Science}

\title{Sistema de apoio à Pós graduação - UFBA}
\date{Outubro/2016}

\author{Victor de Azevedo Nunes}
\adviser{Ivan do Carmo Machado}

\begin{document}

\frontmatter
\frontpage
\presentationpage

\begin{dedicatory}
Eu dedico esta dissertação...
%I dedicate this dissertation to my family, girlfriend, friends and
%professors who gave me all necessary support to get here.
\end{dedicatory}

\acknowledgements
Meus agradecimentos...

\begin{epigraph}[]{Edward V Berard}
Walking on water and developing software from a specification are easy if both are frozen
\end{epigraph}

\resumo
% Escreva seu resumo no arquivo resumo.tex
\input{resumo}

\abstract
% Write your abstract in a file called abstract.tex
\input{abstract}

% Summary (tables of contents)
\tableofcontents

% List of figures
\listoffigures

% List of tables
\listoftables

% List of acronyms
% Acronyms manual: http://linorg.usp.br/CTAN/macros/latex/contrib/acronym/acronym.pdf
\listofacronyms
\input{acronyms}

% List of listings
%\lstlistoflistings

\mainmatter

\include{chapters/intro}
\include{chapters/referencial_teorico}

% \include{chapters/introduction/main}
% \include{chapters/background/main}
% \include{chapters/proposed_solution/main}
% \include{chapters/experiment/main}
% \include{chapters/conclusion/main}

%\bibliographystyle{natbib}
%\addcontentsline{toc}{chapter}{\bibliographytocname}
\bibliography{references}

% Appendix
\clearpage
\addappheadtotoc
\appendix
\appendixpage
% \include{appendix/experiment-instruments}

\end{document}
% %% RiSE Latex Template - version 0.5
%%
%% RiSE's latex template for thesis and dissertations
%% http://risetemplate.sourceforge.net
%%
%% (c) 2012 Yguaratã Cerqueira Cavalcanti (yguarata@gmail.com)
%%          Vinicius Cardoso Garcia (vinicius.garcia@gmail.com)
%%
%% This document was initially based on UFPEThesis template, from Paulo Gustavo
%% S. Fonseca.
%%
%% ACKNOWLEDGEMENTS
%%
%% We would like to thanks the RiSE's researchers community, the 
%% students from Federal University of Pernambuco, and other users that have
%% been contributing to this projects with comments and patches.
%%
%% GENERAL INSTRUCTIONS
%%
%% We strongly recommend you to compile your documents using pdflatex command.
%% It is also recommend use the texlipse plugin for Eclipse to edit your documents.
%%
%% Options for \documentclass command:
%%         * Idiom
%%           pt   - Portguese (default)
%%           en   - English
%%
%%         * Text type
%%           bsc  - B.Sc. Thesis
%%           msc  - M.Sc. Thesis (default)
%%           qual - PHD qualification (not tested yet)
%%           prop - PHD proposal (not tested yet)
%%           phd  - PHD thesis
%%
%%         * Media
%%           scr  - to eletronic version (PDF) / see the users guide
%%
%%         * Pagination
%%           oneside - unique face press
%%           twoside - two faces press
%%
%%		   * Line spacing
%%           singlespacing  - the same as using \linespread{1}
%%           onehalfspacing - the same as using \linespread{1.3}
%%           doublespacing  - the same as using \linespread{1.6}
%%
%% Reference commands. Use the following commands to make references in your
%% text:
%%          \figref  -- for Figure reference
%%          \tabref  -- for Table reference
%%          \eqnref  -- for equation reference
%%          \chapref -- for chapter reference
%%          \secref  -- for section reference
%%          \appref  -- for appendix reference
%%          \axiref  -- for axiom reference
%%          \conjref -- for conjecture reference
%%          \defref  -- for definition reference
%%          \lemref  -- for lemma reference
%%          \theoref -- for theorem reference
%%          \corref  -- for corollary reference
%%          \propref -- for proprosition reference
%%          \pgref   -- for page reference
%%
%%          Example: See \chapref{chap:introduction}. It will produce 
%%                   'See Chapter 1', in case of English language.

\documentclass[pt,twoside,onehalfspacing,bsc]{risethesis}

\usepackage[sort,square]{natbib} %package for bibliography
\bibliographystyle{ieeetr} %authordate1

\usepackage[utf8]{inputenc}
\usepackage[brazilian]{babel}
\usepackage[T1]{fontenc}

\usepackage{graphicx}

%% Change the following pdf author attribute name to your name.
\usepackage[linkcolor=blue,citecolor=blue,urlcolor=blue,colorlinks,pdfpagelabels,pdftitle={Bruno Cabral's Bachelor Thesis},pdfauthor={Bruno Cabral}]{hyperref}

\address{SALVADOR}

\universitypt{Universidade Federal da Bahia}
\universityen{Federal University of Bahia}

\departmentpt{Depertamento de Ciência da Computação}
\departmenten{Computer Science Department}

\programpt{Programa Multiinstitucional de Pós-graduação em Ciência da Computação}
\programen{Graduate in Computer Science}

\majorfieldpt{Ciência da Computação}
\majorfielden{Computer Science}

\title{Sistema de apoio à Pós graduação - UFBA}
\date{Outubro/2016}

\author{Victor de Azevedo Nunes}
\adviser{Ivan do Carmo Machado}

\begin{document}

\frontmatter
\frontpage
\presentationpage

\begin{dedicatory}
Eu dedico esta dissertação...
%I dedicate this dissertation to my family, girlfriend, friends and
%professors who gave me all necessary support to get here.
\end{dedicatory}

\acknowledgements
Meus agradecimentos...

\begin{epigraph}[]{Edward V Berard}
Walking on water and developing software from a specification are easy if both are frozen
\end{epigraph}

\resumo
% Escreva seu resumo no arquivo resumo.tex
\input{resumo}

\abstract
% Write your abstract in a file called abstract.tex
\input{abstract}

% Summary (tables of contents)
\tableofcontents

% List of figures
\listoffigures

% List of tables
\listoftables

% List of acronyms
% Acronyms manual: http://linorg.usp.br/CTAN/macros/latex/contrib/acronym/acronym.pdf
\listofacronyms
\input{acronyms}

% List of listings
%\lstlistoflistings

\mainmatter

\include{chapters/intro}
\include{chapters/referencial_teorico}

% \include{chapters/introduction/main}
% \include{chapters/background/main}
% \include{chapters/proposed_solution/main}
% \include{chapters/experiment/main}
% \include{chapters/conclusion/main}

%\bibliographystyle{natbib}
%\addcontentsline{toc}{chapter}{\bibliographytocname}
\bibliography{references}

% Appendix
\clearpage
\addappheadtotoc
\appendix
\appendixpage
% \include{appendix/experiment-instruments}

\end{document}
% %% RiSE Latex Template - version 0.5
%%
%% RiSE's latex template for thesis and dissertations
%% http://risetemplate.sourceforge.net
%%
%% (c) 2012 Yguaratã Cerqueira Cavalcanti (yguarata@gmail.com)
%%          Vinicius Cardoso Garcia (vinicius.garcia@gmail.com)
%%
%% This document was initially based on UFPEThesis template, from Paulo Gustavo
%% S. Fonseca.
%%
%% ACKNOWLEDGEMENTS
%%
%% We would like to thanks the RiSE's researchers community, the 
%% students from Federal University of Pernambuco, and other users that have
%% been contributing to this projects with comments and patches.
%%
%% GENERAL INSTRUCTIONS
%%
%% We strongly recommend you to compile your documents using pdflatex command.
%% It is also recommend use the texlipse plugin for Eclipse to edit your documents.
%%
%% Options for \documentclass command:
%%         * Idiom
%%           pt   - Portguese (default)
%%           en   - English
%%
%%         * Text type
%%           bsc  - B.Sc. Thesis
%%           msc  - M.Sc. Thesis (default)
%%           qual - PHD qualification (not tested yet)
%%           prop - PHD proposal (not tested yet)
%%           phd  - PHD thesis
%%
%%         * Media
%%           scr  - to eletronic version (PDF) / see the users guide
%%
%%         * Pagination
%%           oneside - unique face press
%%           twoside - two faces press
%%
%%		   * Line spacing
%%           singlespacing  - the same as using \linespread{1}
%%           onehalfspacing - the same as using \linespread{1.3}
%%           doublespacing  - the same as using \linespread{1.6}
%%
%% Reference commands. Use the following commands to make references in your
%% text:
%%          \figref  -- for Figure reference
%%          \tabref  -- for Table reference
%%          \eqnref  -- for equation reference
%%          \chapref -- for chapter reference
%%          \secref  -- for section reference
%%          \appref  -- for appendix reference
%%          \axiref  -- for axiom reference
%%          \conjref -- for conjecture reference
%%          \defref  -- for definition reference
%%          \lemref  -- for lemma reference
%%          \theoref -- for theorem reference
%%          \corref  -- for corollary reference
%%          \propref -- for proprosition reference
%%          \pgref   -- for page reference
%%
%%          Example: See \chapref{chap:introduction}. It will produce 
%%                   'See Chapter 1', in case of English language.

\documentclass[pt,twoside,onehalfspacing,bsc]{risethesis}

\usepackage[sort,square]{natbib} %package for bibliography
\bibliographystyle{ieeetr} %authordate1

\usepackage[utf8]{inputenc}
\usepackage[brazilian]{babel}
\usepackage[T1]{fontenc}

\usepackage{graphicx}

%% Change the following pdf author attribute name to your name.
\usepackage[linkcolor=blue,citecolor=blue,urlcolor=blue,colorlinks,pdfpagelabels,pdftitle={Bruno Cabral's Bachelor Thesis},pdfauthor={Bruno Cabral}]{hyperref}

\address{SALVADOR}

\universitypt{Universidade Federal da Bahia}
\universityen{Federal University of Bahia}

\departmentpt{Depertamento de Ciência da Computação}
\departmenten{Computer Science Department}

\programpt{Programa Multiinstitucional de Pós-graduação em Ciência da Computação}
\programen{Graduate in Computer Science}

\majorfieldpt{Ciência da Computação}
\majorfielden{Computer Science}

\title{Sistema de apoio à Pós graduação - UFBA}
\date{Outubro/2016}

\author{Victor de Azevedo Nunes}
\adviser{Ivan do Carmo Machado}

\begin{document}

\frontmatter
\frontpage
\presentationpage

\begin{dedicatory}
Eu dedico esta dissertação...
%I dedicate this dissertation to my family, girlfriend, friends and
%professors who gave me all necessary support to get here.
\end{dedicatory}

\acknowledgements
Meus agradecimentos...

\begin{epigraph}[]{Edward V Berard}
Walking on water and developing software from a specification are easy if both are frozen
\end{epigraph}

\resumo
% Escreva seu resumo no arquivo resumo.tex
\input{resumo}

\abstract
% Write your abstract in a file called abstract.tex
\input{abstract}

% Summary (tables of contents)
\tableofcontents

% List of figures
\listoffigures

% List of tables
\listoftables

% List of acronyms
% Acronyms manual: http://linorg.usp.br/CTAN/macros/latex/contrib/acronym/acronym.pdf
\listofacronyms
\input{acronyms}

% List of listings
%\lstlistoflistings

\mainmatter

\include{chapters/intro}
\include{chapters/referencial_teorico}

% \include{chapters/introduction/main}
% \include{chapters/background/main}
% \include{chapters/proposed_solution/main}
% \include{chapters/experiment/main}
% \include{chapters/conclusion/main}

%\bibliographystyle{natbib}
%\addcontentsline{toc}{chapter}{\bibliographytocname}
\bibliography{references}

% Appendix
\clearpage
\addappheadtotoc
\appendix
\appendixpage
% \include{appendix/experiment-instruments}

\end{document}
% %% RiSE Latex Template - version 0.5
%%
%% RiSE's latex template for thesis and dissertations
%% http://risetemplate.sourceforge.net
%%
%% (c) 2012 Yguaratã Cerqueira Cavalcanti (yguarata@gmail.com)
%%          Vinicius Cardoso Garcia (vinicius.garcia@gmail.com)
%%
%% This document was initially based on UFPEThesis template, from Paulo Gustavo
%% S. Fonseca.
%%
%% ACKNOWLEDGEMENTS
%%
%% We would like to thanks the RiSE's researchers community, the 
%% students from Federal University of Pernambuco, and other users that have
%% been contributing to this projects with comments and patches.
%%
%% GENERAL INSTRUCTIONS
%%
%% We strongly recommend you to compile your documents using pdflatex command.
%% It is also recommend use the texlipse plugin for Eclipse to edit your documents.
%%
%% Options for \documentclass command:
%%         * Idiom
%%           pt   - Portguese (default)
%%           en   - English
%%
%%         * Text type
%%           bsc  - B.Sc. Thesis
%%           msc  - M.Sc. Thesis (default)
%%           qual - PHD qualification (not tested yet)
%%           prop - PHD proposal (not tested yet)
%%           phd  - PHD thesis
%%
%%         * Media
%%           scr  - to eletronic version (PDF) / see the users guide
%%
%%         * Pagination
%%           oneside - unique face press
%%           twoside - two faces press
%%
%%		   * Line spacing
%%           singlespacing  - the same as using \linespread{1}
%%           onehalfspacing - the same as using \linespread{1.3}
%%           doublespacing  - the same as using \linespread{1.6}
%%
%% Reference commands. Use the following commands to make references in your
%% text:
%%          \figref  -- for Figure reference
%%          \tabref  -- for Table reference
%%          \eqnref  -- for equation reference
%%          \chapref -- for chapter reference
%%          \secref  -- for section reference
%%          \appref  -- for appendix reference
%%          \axiref  -- for axiom reference
%%          \conjref -- for conjecture reference
%%          \defref  -- for definition reference
%%          \lemref  -- for lemma reference
%%          \theoref -- for theorem reference
%%          \corref  -- for corollary reference
%%          \propref -- for proprosition reference
%%          \pgref   -- for page reference
%%
%%          Example: See \chapref{chap:introduction}. It will produce 
%%                   'See Chapter 1', in case of English language.

\documentclass[pt,twoside,onehalfspacing,bsc]{risethesis}

\usepackage[sort,square]{natbib} %package for bibliography
\bibliographystyle{ieeetr} %authordate1

\usepackage[utf8]{inputenc}
\usepackage[brazilian]{babel}
\usepackage[T1]{fontenc}

\usepackage{graphicx}

%% Change the following pdf author attribute name to your name.
\usepackage[linkcolor=blue,citecolor=blue,urlcolor=blue,colorlinks,pdfpagelabels,pdftitle={Bruno Cabral's Bachelor Thesis},pdfauthor={Bruno Cabral}]{hyperref}

\address{SALVADOR}

\universitypt{Universidade Federal da Bahia}
\universityen{Federal University of Bahia}

\departmentpt{Depertamento de Ciência da Computação}
\departmenten{Computer Science Department}

\programpt{Programa Multiinstitucional de Pós-graduação em Ciência da Computação}
\programen{Graduate in Computer Science}

\majorfieldpt{Ciência da Computação}
\majorfielden{Computer Science}

\title{Sistema de apoio à Pós graduação - UFBA}
\date{Outubro/2016}

\author{Victor de Azevedo Nunes}
\adviser{Ivan do Carmo Machado}

\begin{document}

\frontmatter
\frontpage
\presentationpage

\begin{dedicatory}
Eu dedico esta dissertação...
%I dedicate this dissertation to my family, girlfriend, friends and
%professors who gave me all necessary support to get here.
\end{dedicatory}

\acknowledgements
Meus agradecimentos...

\begin{epigraph}[]{Edward V Berard}
Walking on water and developing software from a specification are easy if both are frozen
\end{epigraph}

\resumo
% Escreva seu resumo no arquivo resumo.tex
\input{resumo}

\abstract
% Write your abstract in a file called abstract.tex
\input{abstract}

% Summary (tables of contents)
\tableofcontents

% List of figures
\listoffigures

% List of tables
\listoftables

% List of acronyms
% Acronyms manual: http://linorg.usp.br/CTAN/macros/latex/contrib/acronym/acronym.pdf
\listofacronyms
\input{acronyms}

% List of listings
%\lstlistoflistings

\mainmatter

\include{chapters/intro}
\include{chapters/referencial_teorico}

% \include{chapters/introduction/main}
% \include{chapters/background/main}
% \include{chapters/proposed_solution/main}
% \include{chapters/experiment/main}
% \include{chapters/conclusion/main}

%\bibliographystyle{natbib}
%\addcontentsline{toc}{chapter}{\bibliographytocname}
\bibliography{references}

% Appendix
\clearpage
\addappheadtotoc
\appendix
\appendixpage
% \include{appendix/experiment-instruments}

\end{document}

%\bibliographystyle{natbib}
%\addcontentsline{toc}{chapter}{\bibliographytocname}
\bibliography{references}

% Appendix
\clearpage
\addappheadtotoc
\appendix
\appendixpage
% \include{appendix/experiment-instruments}

\end{document}
% %% RiSE Latex Template - version 0.5
%%
%% RiSE's latex template for thesis and dissertations
%% http://risetemplate.sourceforge.net
%%
%% (c) 2012 Yguaratã Cerqueira Cavalcanti (yguarata@gmail.com)
%%          Vinicius Cardoso Garcia (vinicius.garcia@gmail.com)
%%
%% This document was initially based on UFPEThesis template, from Paulo Gustavo
%% S. Fonseca.
%%
%% ACKNOWLEDGEMENTS
%%
%% We would like to thanks the RiSE's researchers community, the 
%% students from Federal University of Pernambuco, and other users that have
%% been contributing to this projects with comments and patches.
%%
%% GENERAL INSTRUCTIONS
%%
%% We strongly recommend you to compile your documents using pdflatex command.
%% It is also recommend use the texlipse plugin for Eclipse to edit your documents.
%%
%% Options for \documentclass command:
%%         * Idiom
%%           pt   - Portguese (default)
%%           en   - English
%%
%%         * Text type
%%           bsc  - B.Sc. Thesis
%%           msc  - M.Sc. Thesis (default)
%%           qual - PHD qualification (not tested yet)
%%           prop - PHD proposal (not tested yet)
%%           phd  - PHD thesis
%%
%%         * Media
%%           scr  - to eletronic version (PDF) / see the users guide
%%
%%         * Pagination
%%           oneside - unique face press
%%           twoside - two faces press
%%
%%		   * Line spacing
%%           singlespacing  - the same as using \linespread{1}
%%           onehalfspacing - the same as using \linespread{1.3}
%%           doublespacing  - the same as using \linespread{1.6}
%%
%% Reference commands. Use the following commands to make references in your
%% text:
%%          \figref  -- for Figure reference
%%          \tabref  -- for Table reference
%%          \eqnref  -- for equation reference
%%          \chapref -- for chapter reference
%%          \secref  -- for section reference
%%          \appref  -- for appendix reference
%%          \axiref  -- for axiom reference
%%          \conjref -- for conjecture reference
%%          \defref  -- for definition reference
%%          \lemref  -- for lemma reference
%%          \theoref -- for theorem reference
%%          \corref  -- for corollary reference
%%          \propref -- for proprosition reference
%%          \pgref   -- for page reference
%%
%%          Example: See \chapref{chap:introduction}. It will produce 
%%                   'See Chapter 1', in case of English language.

\documentclass[pt,twoside,onehalfspacing,bsc]{risethesis}

\usepackage[sort,square]{natbib} %package for bibliography
\bibliographystyle{ieeetr} %authordate1

\usepackage[utf8]{inputenc}
\usepackage[brazilian]{babel}
\usepackage[T1]{fontenc}

\usepackage{graphicx}

%% Change the following pdf author attribute name to your name.
\usepackage[linkcolor=blue,citecolor=blue,urlcolor=blue,colorlinks,pdfpagelabels,pdftitle={Bruno Cabral's Bachelor Thesis},pdfauthor={Bruno Cabral}]{hyperref}

\address{SALVADOR}

\universitypt{Universidade Federal da Bahia}
\universityen{Federal University of Bahia}

\departmentpt{Depertamento de Ciência da Computação}
\departmenten{Computer Science Department}

\programpt{Programa Multiinstitucional de Pós-graduação em Ciência da Computação}
\programen{Graduate in Computer Science}

\majorfieldpt{Ciência da Computação}
\majorfielden{Computer Science}

\title{Sistema de apoio à Pós graduação - UFBA}
\date{Outubro/2016}

\author{Victor de Azevedo Nunes}
\adviser{Ivan do Carmo Machado}

\begin{document}

\frontmatter
\frontpage
\presentationpage

\begin{dedicatory}
Eu dedico esta dissertação...
%I dedicate this dissertation to my family, girlfriend, friends and
%professors who gave me all necessary support to get here.
\end{dedicatory}

\acknowledgements
Meus agradecimentos...

\begin{epigraph}[]{Edward V Berard}
Walking on water and developing software from a specification are easy if both are frozen
\end{epigraph}

\resumo
% Escreva seu resumo no arquivo resumo.tex
Este trabalho de conclusão de curso utiliza elementos da engenharia de software para propor um software como serviço a fim de otimizar o processo de construção e manutenção dos softwares. Assim, o objetivo deste SaaS é gerenciar as permissões de acesso de sistemas cliente, provendo o reuso de software e padronizando as soluções. O sistema proposto neste trabalho encontra-se implementado e disponível no Github, e traz fundamentos sobre a arquitetura e tecnologias utilizadas, bem como avaliações sobre possibilidades de utilização do mesmo em diversos ambientes, como Web e mobile.

\begin{keywords}
Software; Reuso; SaaS; Web; PHP
\end{keywords}

\abstract
% Write your abstract in a file called abstract.tex
My abstract...

\begin{keywords}
key words...
\end{keywords}

% Summary (tables of contents)
\tableofcontents

% List of figures
\listoffigures

% List of tables
\listoftables

% List of acronyms
% Acronyms manual: http://linorg.usp.br/CTAN/macros/latex/contrib/acronym/acronym.pdf
\listofacronyms
\begin{acronym}[ACRONYM] 
% Change the word ACRONYM above to change the acronym column width.
% The column width is equals to the width of the word that you put.
% Read the manual about acronym package for more examples:
%   http://linorg.usp.br/CTAN/macros/latex/contrib/acronym/acronym.pdf

\acro{SPA}{Single Page Application}
\acro{JSON}{Javascript Object Notation}
\acro{PHP}{PHP: Hypertext Preprocessor}
\acro{SaaS}{Software as a Service}
\acro{ERP}{Enterprise Resource Planning}
\acro{QoS}{Quality of Service}
\acro{UML}{Unified Modeling Language}
\acro{MVC}{Model-View-Controller}
\acro{Ajax}{Asynchronous Javascript and XML}
\acro{HTML}{HyperText Markup Language}
\acro{CSS}{Cascading Style Sheets}
\acro{API}{Application Programming Interface}
\acro{DOM}{Document Object Model}
\acro{BPMN}{Business Process Model and Notation}
\acro{REST}{Representational State Transfer}

\end{acronym}

% List of listings
%\lstlistoflistings

\mainmatter

\chapter{Introdução}

\section{Motivação}

Organizar os procedimentos de um processo sempre nos traz vantagens. Apesar de no processo de implantação de um sistema, o mesmo burocratizar o processo, com o tempo temos o retorno da dedicação para a inserção dos dados. Com um certo volume de dados, é possível estruturar informações que num processo manual são difíceis de serem enxergadas. Assim, é possível depender menos das pessoas que organizam o processo, pois o legado de informações não estará mais somente na mente de alguns, mas sim documentado nos dados do sistema.

Além de colaborar na organização, também haverá uma grande colaboração no tempo gasto na gestão. Lidar com muitos papéis e confiar na mente humana para guardar informações, não é uma alternativa muito segura devido ao fato que as pessoas sempre estão sujeitas a sair do processo e levar contigo a experiência obtida. Experiência essa que faz com que os procedimentos sejam executados de forma mais eficiente. Entretanto, com um sistema inteligente, é possível auxiliar e tornar mais ágil a execução das tarefas.


\section{Problema}


De acordo com funcionários ligados ao o setor de pós graduação da UFBA, entrevistados a fim de um maior entendimento do cenário, apesar das semelhanças estruturais, a pós graduação gerida de forma diferencia da graduação. FULANO afirma que devido ao fato de não ter a mesma visibilidade, não tem acesso aos mesmos recursos de gestão acadêmica da graduação. O professores não executam somente atividades dentro da sala de aula, também tem diversas outras ocupações no setor. E muitos procedimentos realizados extra classe ainda se encontram sendo realizados de forma manual, estando mais vulnerável ao erro ou até mesmo à violação do processo. Também ocorre um grande desperdício de tempo pelos professores e gestores da área, devido ao diversos processos ainda realizados de forma manual, sem a devida documentação. Segundo FULANO, também entrevistado, esse tempo perdido implica numa redução da eficiência na sala de aula, pois o professor acaba por ter menos tempo disponível para o planejamento das atividades, o que gera impactos negativos aos alunos.


\section{Objetivos} %<o que deve ser feito/entregue>


Devido aos muitos processos sendo resolvidos de forma manual, propõe-se com solução um sistema moderno, arquitetado para ter funcionamento na web e com um módulo mobile, a fim de fornecer informações de forma rápida e eficiente para os professores através de notificações, já que o acesso à internet móvel é comum entre os possíveis usuários do sistema em questão.
O principal requisito para o sistema seria dispor recursos para reduzir o tempo desperdiçado pelos professores durante as atividades extra classe.


\section{Metodologia} %<como será feito | como resolver o problema apontado inicialmente>


%<analise de literatura | design | implementação | validação>
Baseando-se nas tecnologias gratuitas em alta no cenário atual do desenvolvimento web, dispomos de algumas opções eficientes para a implementação da solução. Dentre as possibilidades, considerando a facilidade para futura manutenção e continuidade do projeto, tende-se a optar por uma tecnologia popular. Como linguagem de programação, adota-se o PHP. A escolha é fundamentada de acordo com a pesquisa da RedMonk de 2015, que evidencia o uso das linguagens de programação de acordo com as discussões no StackOverflow e repositórios no GitHub. É possível constatar a popularidade do PHP no cenário atual com o gráfico da pesquisa citada, na qual o PHP é apresentado na terceira colocação, apenas atrás do lider JavaScript e do segundo colocado, o Java.

\begin{figure}
	\label{fig:graficoRedmonk}
	\includegraphics[width=1\textwidth]{img/grafico_redmonk}
	\caption{Ranking das liguagens de programação no Stack Overflow e Github}
\end{figure}


Ainda assim, para compor a interface do dado projeto, também ocorrerá o uso do líder JavaScript de forma intensa, provendo o elo com o as informações gerenciadas pelo PHP.


Entretanto, não seria inteligente desenvolver um sistema completo sem o auxílio de um framework. Dentre os frameworks disponíveis para PHP, hoje o destaque está com o Laravel, que se encontra no topo dentre os mais utilizados no momento. 


A WebHostFace, uma empresa de hospedagem, compilou várias estatísticas para criar um infográfico mostrando os frameworks PHP mais populares de 2015. Utilizando informações sobre os próprios clientes, o Google Trends, estatísticas de repositórios do GitHub e a pesquisa do SitePoint “Best PHP Frameworks 2015”, a WebHostFace elaborou o seguinte infográfico: 

\begin{figure}
	\label{fig:graficoWebhostface}
	\includegraphics[width=1\textwidth]{img/infografico_webhostface}
	\caption{Infográfico da WebhostFace, exibindo a popularidade dos Frameworks PHP em 2015}
\end{figure}

Assim, tem-se a evidência que o Laravel em 2015 teve a maior popularidade em projetos pessoais e tem a maior comunidade entre os concorrentes, o que o torna uma boa escolha para a escrita de um software que será continuado por terceiros.


Para elaborar os recursos de interface e integrar ao back-end PHP do sistema, será adotado o já conhecido AngularJS, ferramenta sólida e conhecida no aspecto em questão. 


Dados coletados via Google Trends, que propõe comparações entre termos pesquisados, revela a popularidade do AngularJs diante de alguns dos principais concorrentes. O gráfico abaixo evidencia o cenário.


%Como mostra a Figura \ref{fig:graficoGoogleTrendsFerramentasFront}. 
\begin{figure}
	\label{fig:graficoGoogleTrendsFerramentasFront}
	\includegraphics[width=1\textwidth]{img/grafico_ferramentas_front}
	\caption{Gráfico do Google Trends exibindo as pesquisas por ferramentas front-end}
\end{figure}


Junto ao Angular JS, será utilizada a agradável tendência de interface do Material Design da Google, que propõe layouts limpos e otimizados já conhecidos pelos usuários de smartphones Android. 


Para a elaboração da plataforma mobile do projeto, será utilizado o Ionic Framework, muito difundido e bastante pesquisado na área, o que fica evidenciado com o gráfico de pesquisbaixo, coletado via Google Trends buscando por frameworks de desenvolvimento híbrido mobile.


\begin{figure}
	\label{fig:graficoGoogleTrendsFerramentasHibridasMobile}
	\includegraphics[width=1\textwidth]{img/grafico_ferramentas_hibridas_mobile}
	\caption{Gráfico do Google Trends exibindo as pesquisas por Frameworks híbridos mobile}
\end{figure}	

Para layout da interface mobile, também será aplicado a tendência do Material Design, a fim de propor uma harmonia entre o módulo web e mobile para os usuários


\section{Resultados Esperados}


Como fruto de um sistema para pós-graduação da UFBA, espera-se que os professores tenham mais recursos para integrar as atividades e também prover melhores condições para acompanhamento da vida acadêmica dos alunos em questão. Também, que os novos colaboradores que entrarem no processo tenham facilidade de compreender o fluxo do setor ao navegar pelo sistema proposto.


\section{Fora de Escopo}


Interação com os alunos devido às complicações para realizar a integração com o sistema empregado na UFBA, gerenciado pela XXXXXX, o que causaria uma inviabilidade no projeto devido à necessidade de entrega do produto ser mais forte que o tempo necessário para executar o processo de obtenção de acesso ao sistema legado para realizar a integração.


\section{Estrutura do Trabalho}


<breve resumo sobre os capítulos do TCC>
\chapter{Referencial Teórico}


Projetar o desenvolvimento de um software requer muito planejamento, pois as falhas iniciais podem custar bastante caro ou até mesmo inviabilizar a continuação de um projeto. Assim, a escolha da arquitetura ideal para a aplicabilidade é essencial na concepção de um produto de software. 
De todo o modo, sempre busca-se fazer mais com menos. Diante de tal filosofia, temos neste capítulo, uma breve discussão sobre alguns elementos de projeto e arquitetura de software, a fim de contextualizar este trabalho de conclusão de curso.
O capítulo corrente é composto por quatro seções. A \ref{sec:saas} trata de Software como serviço, discutindo alguns elementos do contexto que são relevantes para o trabalho proposto. A \ref{sec:reuso}  discute sobre a empregabilidade do reuso de software. A \ref{sec:modularizacao} seção, trata sobre aspectos envolvidos na modularização dos softwares. Por fim, a \ref{sec:apps_web} aborda as aplicações web, discutindo sobre aspectos relevantes sobre a aplicação web que compõe este trabalho.


 \section{Software como serviço}\label{sec:saas}


Segundo La e Chun \citep{La2009Systematic}, o princípio da definição de Software como um Serviço (Sofware as a Service - SaaS) é um serviço complementar para aplicações da computação em nuvem (cloud computing). No entanto, as áreas não se confundem. SaaS deve ser entendido como um mecanismo de suporte às soluções existentes na cloud. Os SaaS existem justamente para maximizar o reuso de serviços repetidos e não centrais em uma aplicação remota.


Como vantagens, diversos fatores podem ser favoráveis para a adoção de um SaaS, como custo e manutenção dentre outros fatores aplicáveis a soluções específicas. Lechesaet al. \citep{LechesaSS11}, quantificam índices sobre os fatores determinantes para adoção ou não de um SaaS voltado para ERP na África do Sul. Os principais fatores determinantes para adoção desse mecanismo de software são sua fluidez quanto à rede e a segurança. Esses fatores estão presentes na aplicação desenvolvida neste trabalho de conclusão de curso.


Devido ao fato de ter um serviço constantemente na nuvem, fica o questionamento sobre a segurança da informação manipulada. Sabe-se que a vulnerabilidade na Web não é restrita ao SaaS, atingindo diversos âmbitos. Rai et al. \cite{journals/corr/RaiSM13} consideram que o avanço da computação em nuvem não é um problema apenas para os serviços Web do ponto de vista da segurança, pois muitos trabalhos na literatura mostram a área como mais um ponto de vulnerabilidade para diversos setores, a exemplo de infraestrutura. No mesmo artigo mencionado de Rai et al. \cite{journals/corr/RaiSM13}, também realizaram-se estudos exploratórios junto a empresas usuárias de serviços em computação em nuvem e consideram que a perspectiva de SaaS também pode fortalecer a segurança nas aplicações de cloud computing, pois o software de autenticação compartilhado por várias aplicações em nuvem, oferece uma melhor padronização e consequente facilidade de prevenção a erros de vulnerabilidade específicas de cada módulo da pesquisa. Esse ponto de vista é fundamenta para a compreensão de SaaS.


A arquitetura de armazenamento de dados de um Saas pode variar de acordo com a necessidade do contexto. Huixin \cite{7586486} descreve possíveis modelagens nesse sentido. Essa abordagem pode ser com um banco de dados único, fazendo com que diferentes clientes compartilhem o mesmo banco, diferindo os dados através de controle de usuário, ou isolando os diferentes clientes através de bancos de dados exclusivos para cada um. Esse fator também pode ser combinado com a arquitetura da aplicação, caso ofereça aplicação única para todos os clientes ou aplicação compartilhada. Diante das possíveis abordagens, a modelagem de dados do software pode ser decidida pela regra de negócio. Este trabalho optou por aplicação única e banco de dados compartilhado.

Devido ao conceito particular de obtenção de software, tanto pela visão do cliente como pela visão do vendedor, é necessário compreender os diversos aspectos que podem ser relevantes ao orçar um Saas. O recente trabalho de T. Kaur et al. \citep{6949281} orienta um modelo para compor o custo de um Saas. O custo total seria composto pelos fatores que dão suporte ao funcionamento do software. Tais fatores incluem infra-estrutura, configurabilidade, customização, parâmetros de QoS (Quality of service) como escalabilidade, disponibilidade, usabilidade, pontualidade e desempenho da resposta, portabilidade, custo total de propriedade e retorno do investimento. Esses fatores caracterizam o custo de forma eficaz, possibilitando ao fornecedor, prover um Serviço de acordo com a exigência do consumidor em vários pacotes de serviços.


O conceito de software como serviço se aplica neste trabalho de conclusão de curso, pois o mesmo estará disponível na web com alta disponibilidade adotando as características apresentadas para qualquer pessoa que desejar usá-lo. Assim, basta se cadastrar, configurá-lo e usar como um serviço, sem preocupação com a manutenção do mesmo.


\section{Reuso de software}\label{sec:reuso} %CRUISE BOOK CAPITULO 2


De acordo com o livro Software pratical reuse \citep{Ezran2002}, o reuso de software é a utilização de qualquer informação que um desenvolvedor pode necessitar no processo de criação de software. O livro de Basili e Rombach \citep{Basili1991} define reutilização de software como o uso de tudo o que está associado a um projeto de conhecimento.
Assim, o objetivo da reutilização de software é reciclar o design, código e outros componentes de um produto de software e assim reduzir o custo, o tempo e melhorar a qualidade do produto.
Segundo Keswani et al. \cite{6783445}, o componente reutilizável de software pode ser qualquer parte de seu desenvolvimento, como um fragmento de código, design, casos de teste, ou até mesmo a especificação de requisitos de uma funcionalidade do software. 

O reuso de software pode ter impacto positivo em diversos aspectos do software, vejamos alguns, conforme apresentados no C.R.U.I.S.E Book \citep{cruise2007} :

\begin{itemize}

\item Qualidade: As correções de erro tornam-se úteis em todos os locais em que ocorreu, padronizando e facilitando a manutenção.

\item Produtividade: O ganho de produtividade é alcançado devido ao menor número de artefatos desenvolvido. Isso resulta em menor esforço de teste e também análise e design, reduzindo custos.

\item Confiabilidade: A utilização de componentes bem testados aumenta a
confiança no software. Além disso, a utilização de um mesmo componente em vários sistemas, aumenta a possibilidade de detecção de erros e reforça a confiança no componente.

\item Redução do Esforço: A reutilização de software proporciona uma redução do tempo de desenvolvimento, o que reduz o tempo necessário para o produto ser disponibilizado no mercado para trazer rentabilidade.

\item Trabalho redundante e tempo de desenvolvimento: Desenvolver um sistema do
zero significa desenvolvimento redundante de muitos componentes, como requisitos, especificações, casos de uso, arquitetura, etc. Isso pode ser evitado quando estes estão disponíveis como componentes reutilizáveis e podem ser compartilhados, resultando em um processo de desenvolvimento otimizado.

\item Documentação: Embora a documentação seja muito importante para a
manutenção de um sistema, muitas vezes é negligenciada. A reutilização de componentes de software reduz a quantidade de documentação a ser escrita, entretanto depende da qualidade do que está escrito. Assim, apenas a estrutura do sistema e os novos artefatos desenvolvidos necessitam ser documentados.

\item Custo de manutenção: Menos defeitos e manutenções são esperados quando tem-se comprovada a qualidade dos componentes utilizados.

\item Tamanho da equipe: É comum haver casos em que a equipe de desenvolvimento sofre sobrecarga. Entretanto, dobrar o tamanho da equipe de desenvolvimento não necessariamente duplica produtividade. Se muitos componentes podem ser reutilizados, é possível desenvolver com equipes menores, levando a melhor comunicação e aumento da produtividade.

\end{itemize}

Apesar dos benefícios da reutilização de software, ela não é suficientemente aproveitada. Existem fatores que influenciam direta ou indiretamente na sua adoção. Esses fatores podem ser de aspecto gerencial, organizacional, econômico, conceitual ou técnico. Veremos a seguir alguns aspectos que podem gerar conflito com a cultura de reuso de software, segundo o C.R.U.I.S.E Book \citep{cruise2007} :
%(Sametinger, 1997). REVER

\begin{itemize}
	
\item Falta de apoio da gestão: Como a reutilização de software gera custos iniciais,
a medida pode não ser amplamente alcançada em uma organização sem o apoio de alto nível de gestão. Os gestores têm de ser informados sobre os custos iniciais e serem convencidos sobre economias futuras.

\item Gerenciamento do Projeto: Gerenciar projetos tradicionais é uma tarefa árdua, principalmente, os que praticam a reutilização de software. Utilizando a técnica em larga escala, tem-se impacto sobre todo o ciclo de vida do software.

\item Estruturas organizacionais inadequadas: As estruturas organizacionais devem
considerar diferentes necessidades que surgem quando a reutilização em larga escala está sendo adotada. Por exemplo, uma equipe particionada pode ser alocada somente para desenvolver, manter e certificar componentes reutilizáveis de software.

\item Incentivos de gestão: É comum a falta de incentivo para deixar os desenvolvedores gastarem tempo elaborando componentes do sistemas. A produtividade é muitas vezes medida apenas no tempo necessário para concluir um projeto. Assim, fazer qualquer trabalho além disso, embora benéfico para a empresa como um todo, diminui o seu sucesso. Mesmo quando os componentes reutilizáveis são utilizados, os benefícios obtidos são uma pequena fração do que poderia ser alcançado caso houvesse reutilização explícita, planejada e organizada.

\item Dificuldade de encontrar software reutilizável: Para reutilizar os componentes, devem existir formas eficientes de busca. Além disso, é importante ter um repositório bem organizado contendo componentes com um eficiente meio de acesso.

\item Não reutilização do software encontrado. O acesso fácil ao software existente
não necessariamente aumentar a reutilização. Os componentes reutilizáveis devem ser cuidadosamente especificados, projetados, implementados e documentados, pois em alguns casos, modificar e adaptar o código  pode ser mais custoso que a programação da funcionalidade necessária a partir do zero.

\item Modificação: É muito difícil encontrar um componente que funcione
exatamente da mesma maneira que queremos. Desta forma, são necessárias modificações e devem existir formas de determinar os seus efeitos sobre o componente.


\end{itemize}


A definição do reuso de software pode ser associada a este trabalho, pois ainda que como um serviço, estaremos adotando uma aplicação única para servir aos mais diversos softwares de uma empresa, por exemplo. Logo, ao adotar o projeto, pode ser interessante aplicar algumas métricas para tomar conhecimento de possíveis vantanjens como as citadas nessa seção, a exemplo de economia de tempo de desenvolvimento, e custo do projeto.


%Outra diretriz importante para a reutilização de software é reduzir o risco na criação de novos softwares. O risco tende a ser bastante reduzido se os componentes que estão sendo reutilizados têm as documentação, interfaces necessárias e devidamente testadas, fatores que contibruem para uma fácil integração.
%De acordo com Keswani et al. \citep{6783445}, para o reuso de software dar retornos apropriados, o processo deve ser sistemático e planejado. Qualquer organização que implemente a reutilização de software deve identificar os melhores métodos e estratégias de reutilização para obter a máxima produtividade. A reutilização de software ajuda a evitar software de engenharia a partir do zero, pois usa módulos de software existentes. A reutilização de software, embora seja uma tarefa difícil, especialmente para softwares antigos sem padrões de projeto, pode melhorar significativamente a produtividade e a qualidade de um produto de software. Embora a reutilização de software não seja um novo campo, ela pode dar grandes retornos em curto período de tempo.


\section{Modularização}\label{sec:modularizacao} %artigo de claudio pagina 222 introdução


%A modularidade vem desempenhando um papel predominante estágios emergentes das disciplinas de arquitetura de software [13]. Engenheiros de software consideram modularidade como princípio base na comparação entre arquiteturas alternativas  e arquitetura degeneração [9]. De fato, os engenheiros de software são incentivados a arquitecturas, baseando-se numa multiplicidade de mecanismos de modularidade disponíveis em: 
%(i) Linguagens de descrição de arquitetura (ADLs), como ACME [8], 
%(ii) catálogos de arquitetônicos [2, 13], e 
%(iii) conhecem bem princípios de alto nível, como interfaces de componentes estreitos, acoplamento arquitectónico reduzido e semelhantes.


Conforme é frisado por Wickramaarachchi e Lai \citep{7062705}, o conceito de modularização na indústria de software tem uma longa história e tem sido utilizado para melhorar o processo de desenvolvimento de software em diferentes estágios. Os principais conceitos por trás da modularização do software foram introduzidos por pesquisadores pioneiros há quarenta anos, com uma notável contribuição feita por Melvin Conway e David Parnas, que tem representação notável na engenharia de software.


Modularizar um software é um bom padrão a ser adotado. Segundo Wickramaarachchi e Lai \citep{7062705}, a modularização é importante na identificação de dependências e reduz as dificuldades diante de uma possível necessidade de grandes alterações. De uma perspectiva da engenharia de software, uma modularização geralmente tem várias vantagens, tais como: tornar a complexidade do software mais gerenciável, facilitar o trabalho paralelo e tornar o software mais maleável para acomodar o futuro incerto que um software pode ter. O objetivo final da modularização do software é aumentar a produtividade ea qualidade do software. Tal conceito encontra-se bastante difundido e estái incorporado em linguagens de programação e ferramentas de software. O presente trabalho favorece ao uso da modularização de um software e até mesmo pode ser considerado um módulo a ser acoplado a qualquer software, mediante a compatibilidade.


Ao realizar a adoção do trabalho proposto, fica bastante evidente a "responsabilidade" do mesmo no escopo do projeto. Assim, é possível tratá-lo como um módulo do projeto que o usa, mesmo que seja consumido como um serviço.


\section{Aplicações web}\label{sec:apps_web}


A popularidade das soluções Web aumentou exponencialmente na última década e todos os dias cresce o número de pessoas usuárias desse tipo de software. E seguindo um padrão próprio, Kumar et al. \citep{7813710} sugerem que para o desenvolvimento web, deve-se manter a prática eficaz de produzir diagramas UML. A abordagem baseada na web oferece uma maneira fácil e eficaz para gerenciar e controlar o processo de desenvolvimento por meio de artefatos de modelagem. Tal abordagem pode ser usada quando há uma exigência de lidar com mudanças muito rápidas e grandes em requisitos de forma muito eficaz em muito menos tempo, gerando assim um menor impacto. 


Para atender à fomentada demanda de aplicativos web, é necessário adotar métodos de desenvolvimentos que sejam ágeis, eficientes e de fácil manutenção. Yu Ping et al. \cite{1372143} propõem o uso do modelo MVC (Model, View e Controller) no desenvolvimento para softwares web. O modelo apresentado tornou-se um padrão popular e divide o software em camadas com propósito definido, tornando-o de mais fácil manutenção.


O Ajax (Asynchronous Javascript and XML) revolucionou a web. Conforme demonstrado por Yuping \citep{6845605}, ao usar a tecnologia Ajax, podemos enriquecer a experiência do usuário em aplicações baseadas em navegador de internet, e fornecer uma variedade de aplicações interativas para atender às necessidade de humanização das aplicações.
Os aplicativos Ajax em execução no navegador se comunicam com um servidor Web de forma assíncrona e atualizam apenas uma parte da página.


No artigo de Tesarik et al. \citep{4664349}, temos que o design de software SPA (Single page application) é uma maneira interessante de criar um software Web numa uma única página. Essa solução de página única sem navegação funciona apenas com base em técnicas dinâmicas e assíncronas, como o AJAX (citado no parágrafo anterior). No entanto, esta abordagem coloca o desenvolvedor antes de alguns desafios substanciais.
Para projetar a interface do usuário que mostra as informações, é importante projetar corretamente a tela para manipular os dados do aplicativo numa única página. O design da página deve ser elaborado para maximizar a decomposição da página em componentes distintos que encapsulam os principais casos de uso.
Também recomenda-se que para elaborar os artefatos visuais, explore os mais ricos recursos na implementação, como HTML5, JavaScript, Ajax, CSS3, e outras tecnologias que se apliquem. Entretanto, desenvolver uma rica interface com o uso de diversas tecnologias/frameworks pode ocasionar um esforço maior para explorar as possibilidades.
Tais componentes da interface de uma aplicação SPA normalmente são alimentados mediante o consumo de uma API Rest via requisições AJAX. Assim, é possível particionar as responsabilidades de processamento do software entre cliente e servidor.


Como é possível denotar no trabalho de Salvadori e Siqueire \citep{7195633}, REST é uma arquitetura muito popular para integração de aplicativos web, que permite compartilhar e reutilizar informações através de sistemas. Aplicações de grande escala baseadas em no REST deve ser implementadas utilizando estratégias e mecanismos para produzir sistemas que sejam fáceis de desenvolver, reutilizar e manter. As interfaces de integração fornecidas por esses sistemas, chamadas de Web API, têm uma influência importante nas características da implementação resultante, pois a sua forma de resposta deve ser casada com a interface que o consome.
 
 
Os elementos comentados nessa seção encontram-se presentes neste trabalho de conclusão de curso. Juntos, montam a estrutura tecnológica necessária aliada à arquitetura adotada. Esses elementos seguem tendências atuais dos softwares web, fazendo com que esse trabalho esteja composto por tecnologias modernas que estão com boa aceitação no mercado.

% %% RiSE Latex Template - version 0.5
%%
%% RiSE's latex template for thesis and dissertations
%% http://risetemplate.sourceforge.net
%%
%% (c) 2012 Yguaratã Cerqueira Cavalcanti (yguarata@gmail.com)
%%          Vinicius Cardoso Garcia (vinicius.garcia@gmail.com)
%%
%% This document was initially based on UFPEThesis template, from Paulo Gustavo
%% S. Fonseca.
%%
%% ACKNOWLEDGEMENTS
%%
%% We would like to thanks the RiSE's researchers community, the 
%% students from Federal University of Pernambuco, and other users that have
%% been contributing to this projects with comments and patches.
%%
%% GENERAL INSTRUCTIONS
%%
%% We strongly recommend you to compile your documents using pdflatex command.
%% It is also recommend use the texlipse plugin for Eclipse to edit your documents.
%%
%% Options for \documentclass command:
%%         * Idiom
%%           pt   - Portguese (default)
%%           en   - English
%%
%%         * Text type
%%           bsc  - B.Sc. Thesis
%%           msc  - M.Sc. Thesis (default)
%%           qual - PHD qualification (not tested yet)
%%           prop - PHD proposal (not tested yet)
%%           phd  - PHD thesis
%%
%%         * Media
%%           scr  - to eletronic version (PDF) / see the users guide
%%
%%         * Pagination
%%           oneside - unique face press
%%           twoside - two faces press
%%
%%		   * Line spacing
%%           singlespacing  - the same as using \linespread{1}
%%           onehalfspacing - the same as using \linespread{1.3}
%%           doublespacing  - the same as using \linespread{1.6}
%%
%% Reference commands. Use the following commands to make references in your
%% text:
%%          \figref  -- for Figure reference
%%          \tabref  -- for Table reference
%%          \eqnref  -- for equation reference
%%          \chapref -- for chapter reference
%%          \secref  -- for section reference
%%          \appref  -- for appendix reference
%%          \axiref  -- for axiom reference
%%          \conjref -- for conjecture reference
%%          \defref  -- for definition reference
%%          \lemref  -- for lemma reference
%%          \theoref -- for theorem reference
%%          \corref  -- for corollary reference
%%          \propref -- for proprosition reference
%%          \pgref   -- for page reference
%%
%%          Example: See \chapref{chap:introduction}. It will produce 
%%                   'See Chapter 1', in case of English language.

\documentclass[pt,twoside,onehalfspacing,bsc]{risethesis}

\usepackage[sort,square]{natbib} %package for bibliography
\bibliographystyle{ieeetr} %authordate1

\usepackage[utf8]{inputenc}
\usepackage[brazilian]{babel}
\usepackage[T1]{fontenc}

\usepackage{graphicx}

%% Change the following pdf author attribute name to your name.
\usepackage[linkcolor=blue,citecolor=blue,urlcolor=blue,colorlinks,pdfpagelabels,pdftitle={Bruno Cabral's Bachelor Thesis},pdfauthor={Bruno Cabral}]{hyperref}

\address{SALVADOR}

\universitypt{Universidade Federal da Bahia}
\universityen{Federal University of Bahia}

\departmentpt{Depertamento de Ciência da Computação}
\departmenten{Computer Science Department}

\programpt{Programa Multiinstitucional de Pós-graduação em Ciência da Computação}
\programen{Graduate in Computer Science}

\majorfieldpt{Ciência da Computação}
\majorfielden{Computer Science}

\title{Sistema de apoio à Pós graduação - UFBA}
\date{Outubro/2016}

\author{Victor de Azevedo Nunes}
\adviser{Ivan do Carmo Machado}

\begin{document}

\frontmatter
\frontpage
\presentationpage

\begin{dedicatory}
Eu dedico esta dissertação...
%I dedicate this dissertation to my family, girlfriend, friends and
%professors who gave me all necessary support to get here.
\end{dedicatory}

\acknowledgements
Meus agradecimentos...

\begin{epigraph}[]{Edward V Berard}
Walking on water and developing software from a specification are easy if both are frozen
\end{epigraph}

\resumo
% Escreva seu resumo no arquivo resumo.tex
\input{resumo}

\abstract
% Write your abstract in a file called abstract.tex
\input{abstract}

% Summary (tables of contents)
\tableofcontents

% List of figures
\listoffigures

% List of tables
\listoftables

% List of acronyms
% Acronyms manual: http://linorg.usp.br/CTAN/macros/latex/contrib/acronym/acronym.pdf
\listofacronyms
\input{acronyms}

% List of listings
%\lstlistoflistings

\mainmatter

\include{chapters/intro}
\include{chapters/referencial_teorico}

% \include{chapters/introduction/main}
% \include{chapters/background/main}
% \include{chapters/proposed_solution/main}
% \include{chapters/experiment/main}
% \include{chapters/conclusion/main}

%\bibliographystyle{natbib}
%\addcontentsline{toc}{chapter}{\bibliographytocname}
\bibliography{references}

% Appendix
\clearpage
\addappheadtotoc
\appendix
\appendixpage
% \include{appendix/experiment-instruments}

\end{document}
% %% RiSE Latex Template - version 0.5
%%
%% RiSE's latex template for thesis and dissertations
%% http://risetemplate.sourceforge.net
%%
%% (c) 2012 Yguaratã Cerqueira Cavalcanti (yguarata@gmail.com)
%%          Vinicius Cardoso Garcia (vinicius.garcia@gmail.com)
%%
%% This document was initially based on UFPEThesis template, from Paulo Gustavo
%% S. Fonseca.
%%
%% ACKNOWLEDGEMENTS
%%
%% We would like to thanks the RiSE's researchers community, the 
%% students from Federal University of Pernambuco, and other users that have
%% been contributing to this projects with comments and patches.
%%
%% GENERAL INSTRUCTIONS
%%
%% We strongly recommend you to compile your documents using pdflatex command.
%% It is also recommend use the texlipse plugin for Eclipse to edit your documents.
%%
%% Options for \documentclass command:
%%         * Idiom
%%           pt   - Portguese (default)
%%           en   - English
%%
%%         * Text type
%%           bsc  - B.Sc. Thesis
%%           msc  - M.Sc. Thesis (default)
%%           qual - PHD qualification (not tested yet)
%%           prop - PHD proposal (not tested yet)
%%           phd  - PHD thesis
%%
%%         * Media
%%           scr  - to eletronic version (PDF) / see the users guide
%%
%%         * Pagination
%%           oneside - unique face press
%%           twoside - two faces press
%%
%%		   * Line spacing
%%           singlespacing  - the same as using \linespread{1}
%%           onehalfspacing - the same as using \linespread{1.3}
%%           doublespacing  - the same as using \linespread{1.6}
%%
%% Reference commands. Use the following commands to make references in your
%% text:
%%          \figref  -- for Figure reference
%%          \tabref  -- for Table reference
%%          \eqnref  -- for equation reference
%%          \chapref -- for chapter reference
%%          \secref  -- for section reference
%%          \appref  -- for appendix reference
%%          \axiref  -- for axiom reference
%%          \conjref -- for conjecture reference
%%          \defref  -- for definition reference
%%          \lemref  -- for lemma reference
%%          \theoref -- for theorem reference
%%          \corref  -- for corollary reference
%%          \propref -- for proprosition reference
%%          \pgref   -- for page reference
%%
%%          Example: See \chapref{chap:introduction}. It will produce 
%%                   'See Chapter 1', in case of English language.

\documentclass[pt,twoside,onehalfspacing,bsc]{risethesis}

\usepackage[sort,square]{natbib} %package for bibliography
\bibliographystyle{ieeetr} %authordate1

\usepackage[utf8]{inputenc}
\usepackage[brazilian]{babel}
\usepackage[T1]{fontenc}

\usepackage{graphicx}

%% Change the following pdf author attribute name to your name.
\usepackage[linkcolor=blue,citecolor=blue,urlcolor=blue,colorlinks,pdfpagelabels,pdftitle={Bruno Cabral's Bachelor Thesis},pdfauthor={Bruno Cabral}]{hyperref}

\address{SALVADOR}

\universitypt{Universidade Federal da Bahia}
\universityen{Federal University of Bahia}

\departmentpt{Depertamento de Ciência da Computação}
\departmenten{Computer Science Department}

\programpt{Programa Multiinstitucional de Pós-graduação em Ciência da Computação}
\programen{Graduate in Computer Science}

\majorfieldpt{Ciência da Computação}
\majorfielden{Computer Science}

\title{Sistema de apoio à Pós graduação - UFBA}
\date{Outubro/2016}

\author{Victor de Azevedo Nunes}
\adviser{Ivan do Carmo Machado}

\begin{document}

\frontmatter
\frontpage
\presentationpage

\begin{dedicatory}
Eu dedico esta dissertação...
%I dedicate this dissertation to my family, girlfriend, friends and
%professors who gave me all necessary support to get here.
\end{dedicatory}

\acknowledgements
Meus agradecimentos...

\begin{epigraph}[]{Edward V Berard}
Walking on water and developing software from a specification are easy if both are frozen
\end{epigraph}

\resumo
% Escreva seu resumo no arquivo resumo.tex
\input{resumo}

\abstract
% Write your abstract in a file called abstract.tex
\input{abstract}

% Summary (tables of contents)
\tableofcontents

% List of figures
\listoffigures

% List of tables
\listoftables

% List of acronyms
% Acronyms manual: http://linorg.usp.br/CTAN/macros/latex/contrib/acronym/acronym.pdf
\listofacronyms
\input{acronyms}

% List of listings
%\lstlistoflistings

\mainmatter

\include{chapters/intro}
\include{chapters/referencial_teorico}

% \include{chapters/introduction/main}
% \include{chapters/background/main}
% \include{chapters/proposed_solution/main}
% \include{chapters/experiment/main}
% \include{chapters/conclusion/main}

%\bibliographystyle{natbib}
%\addcontentsline{toc}{chapter}{\bibliographytocname}
\bibliography{references}

% Appendix
\clearpage
\addappheadtotoc
\appendix
\appendixpage
% \include{appendix/experiment-instruments}

\end{document}
% %% RiSE Latex Template - version 0.5
%%
%% RiSE's latex template for thesis and dissertations
%% http://risetemplate.sourceforge.net
%%
%% (c) 2012 Yguaratã Cerqueira Cavalcanti (yguarata@gmail.com)
%%          Vinicius Cardoso Garcia (vinicius.garcia@gmail.com)
%%
%% This document was initially based on UFPEThesis template, from Paulo Gustavo
%% S. Fonseca.
%%
%% ACKNOWLEDGEMENTS
%%
%% We would like to thanks the RiSE's researchers community, the 
%% students from Federal University of Pernambuco, and other users that have
%% been contributing to this projects with comments and patches.
%%
%% GENERAL INSTRUCTIONS
%%
%% We strongly recommend you to compile your documents using pdflatex command.
%% It is also recommend use the texlipse plugin for Eclipse to edit your documents.
%%
%% Options for \documentclass command:
%%         * Idiom
%%           pt   - Portguese (default)
%%           en   - English
%%
%%         * Text type
%%           bsc  - B.Sc. Thesis
%%           msc  - M.Sc. Thesis (default)
%%           qual - PHD qualification (not tested yet)
%%           prop - PHD proposal (not tested yet)
%%           phd  - PHD thesis
%%
%%         * Media
%%           scr  - to eletronic version (PDF) / see the users guide
%%
%%         * Pagination
%%           oneside - unique face press
%%           twoside - two faces press
%%
%%		   * Line spacing
%%           singlespacing  - the same as using \linespread{1}
%%           onehalfspacing - the same as using \linespread{1.3}
%%           doublespacing  - the same as using \linespread{1.6}
%%
%% Reference commands. Use the following commands to make references in your
%% text:
%%          \figref  -- for Figure reference
%%          \tabref  -- for Table reference
%%          \eqnref  -- for equation reference
%%          \chapref -- for chapter reference
%%          \secref  -- for section reference
%%          \appref  -- for appendix reference
%%          \axiref  -- for axiom reference
%%          \conjref -- for conjecture reference
%%          \defref  -- for definition reference
%%          \lemref  -- for lemma reference
%%          \theoref -- for theorem reference
%%          \corref  -- for corollary reference
%%          \propref -- for proprosition reference
%%          \pgref   -- for page reference
%%
%%          Example: See \chapref{chap:introduction}. It will produce 
%%                   'See Chapter 1', in case of English language.

\documentclass[pt,twoside,onehalfspacing,bsc]{risethesis}

\usepackage[sort,square]{natbib} %package for bibliography
\bibliographystyle{ieeetr} %authordate1

\usepackage[utf8]{inputenc}
\usepackage[brazilian]{babel}
\usepackage[T1]{fontenc}

\usepackage{graphicx}

%% Change the following pdf author attribute name to your name.
\usepackage[linkcolor=blue,citecolor=blue,urlcolor=blue,colorlinks,pdfpagelabels,pdftitle={Bruno Cabral's Bachelor Thesis},pdfauthor={Bruno Cabral}]{hyperref}

\address{SALVADOR}

\universitypt{Universidade Federal da Bahia}
\universityen{Federal University of Bahia}

\departmentpt{Depertamento de Ciência da Computação}
\departmenten{Computer Science Department}

\programpt{Programa Multiinstitucional de Pós-graduação em Ciência da Computação}
\programen{Graduate in Computer Science}

\majorfieldpt{Ciência da Computação}
\majorfielden{Computer Science}

\title{Sistema de apoio à Pós graduação - UFBA}
\date{Outubro/2016}

\author{Victor de Azevedo Nunes}
\adviser{Ivan do Carmo Machado}

\begin{document}

\frontmatter
\frontpage
\presentationpage

\begin{dedicatory}
Eu dedico esta dissertação...
%I dedicate this dissertation to my family, girlfriend, friends and
%professors who gave me all necessary support to get here.
\end{dedicatory}

\acknowledgements
Meus agradecimentos...

\begin{epigraph}[]{Edward V Berard}
Walking on water and developing software from a specification are easy if both are frozen
\end{epigraph}

\resumo
% Escreva seu resumo no arquivo resumo.tex
\input{resumo}

\abstract
% Write your abstract in a file called abstract.tex
\input{abstract}

% Summary (tables of contents)
\tableofcontents

% List of figures
\listoffigures

% List of tables
\listoftables

% List of acronyms
% Acronyms manual: http://linorg.usp.br/CTAN/macros/latex/contrib/acronym/acronym.pdf
\listofacronyms
\input{acronyms}

% List of listings
%\lstlistoflistings

\mainmatter

\include{chapters/intro}
\include{chapters/referencial_teorico}

% \include{chapters/introduction/main}
% \include{chapters/background/main}
% \include{chapters/proposed_solution/main}
% \include{chapters/experiment/main}
% \include{chapters/conclusion/main}

%\bibliographystyle{natbib}
%\addcontentsline{toc}{chapter}{\bibliographytocname}
\bibliography{references}

% Appendix
\clearpage
\addappheadtotoc
\appendix
\appendixpage
% \include{appendix/experiment-instruments}

\end{document}
% %% RiSE Latex Template - version 0.5
%%
%% RiSE's latex template for thesis and dissertations
%% http://risetemplate.sourceforge.net
%%
%% (c) 2012 Yguaratã Cerqueira Cavalcanti (yguarata@gmail.com)
%%          Vinicius Cardoso Garcia (vinicius.garcia@gmail.com)
%%
%% This document was initially based on UFPEThesis template, from Paulo Gustavo
%% S. Fonseca.
%%
%% ACKNOWLEDGEMENTS
%%
%% We would like to thanks the RiSE's researchers community, the 
%% students from Federal University of Pernambuco, and other users that have
%% been contributing to this projects with comments and patches.
%%
%% GENERAL INSTRUCTIONS
%%
%% We strongly recommend you to compile your documents using pdflatex command.
%% It is also recommend use the texlipse plugin for Eclipse to edit your documents.
%%
%% Options for \documentclass command:
%%         * Idiom
%%           pt   - Portguese (default)
%%           en   - English
%%
%%         * Text type
%%           bsc  - B.Sc. Thesis
%%           msc  - M.Sc. Thesis (default)
%%           qual - PHD qualification (not tested yet)
%%           prop - PHD proposal (not tested yet)
%%           phd  - PHD thesis
%%
%%         * Media
%%           scr  - to eletronic version (PDF) / see the users guide
%%
%%         * Pagination
%%           oneside - unique face press
%%           twoside - two faces press
%%
%%		   * Line spacing
%%           singlespacing  - the same as using \linespread{1}
%%           onehalfspacing - the same as using \linespread{1.3}
%%           doublespacing  - the same as using \linespread{1.6}
%%
%% Reference commands. Use the following commands to make references in your
%% text:
%%          \figref  -- for Figure reference
%%          \tabref  -- for Table reference
%%          \eqnref  -- for equation reference
%%          \chapref -- for chapter reference
%%          \secref  -- for section reference
%%          \appref  -- for appendix reference
%%          \axiref  -- for axiom reference
%%          \conjref -- for conjecture reference
%%          \defref  -- for definition reference
%%          \lemref  -- for lemma reference
%%          \theoref -- for theorem reference
%%          \corref  -- for corollary reference
%%          \propref -- for proprosition reference
%%          \pgref   -- for page reference
%%
%%          Example: See \chapref{chap:introduction}. It will produce 
%%                   'See Chapter 1', in case of English language.

\documentclass[pt,twoside,onehalfspacing,bsc]{risethesis}

\usepackage[sort,square]{natbib} %package for bibliography
\bibliographystyle{ieeetr} %authordate1

\usepackage[utf8]{inputenc}
\usepackage[brazilian]{babel}
\usepackage[T1]{fontenc}

\usepackage{graphicx}

%% Change the following pdf author attribute name to your name.
\usepackage[linkcolor=blue,citecolor=blue,urlcolor=blue,colorlinks,pdfpagelabels,pdftitle={Bruno Cabral's Bachelor Thesis},pdfauthor={Bruno Cabral}]{hyperref}

\address{SALVADOR}

\universitypt{Universidade Federal da Bahia}
\universityen{Federal University of Bahia}

\departmentpt{Depertamento de Ciência da Computação}
\departmenten{Computer Science Department}

\programpt{Programa Multiinstitucional de Pós-graduação em Ciência da Computação}
\programen{Graduate in Computer Science}

\majorfieldpt{Ciência da Computação}
\majorfielden{Computer Science}

\title{Sistema de apoio à Pós graduação - UFBA}
\date{Outubro/2016}

\author{Victor de Azevedo Nunes}
\adviser{Ivan do Carmo Machado}

\begin{document}

\frontmatter
\frontpage
\presentationpage

\begin{dedicatory}
Eu dedico esta dissertação...
%I dedicate this dissertation to my family, girlfriend, friends and
%professors who gave me all necessary support to get here.
\end{dedicatory}

\acknowledgements
Meus agradecimentos...

\begin{epigraph}[]{Edward V Berard}
Walking on water and developing software from a specification are easy if both are frozen
\end{epigraph}

\resumo
% Escreva seu resumo no arquivo resumo.tex
\input{resumo}

\abstract
% Write your abstract in a file called abstract.tex
\input{abstract}

% Summary (tables of contents)
\tableofcontents

% List of figures
\listoffigures

% List of tables
\listoftables

% List of acronyms
% Acronyms manual: http://linorg.usp.br/CTAN/macros/latex/contrib/acronym/acronym.pdf
\listofacronyms
\input{acronyms}

% List of listings
%\lstlistoflistings

\mainmatter

\include{chapters/intro}
\include{chapters/referencial_teorico}

% \include{chapters/introduction/main}
% \include{chapters/background/main}
% \include{chapters/proposed_solution/main}
% \include{chapters/experiment/main}
% \include{chapters/conclusion/main}

%\bibliographystyle{natbib}
%\addcontentsline{toc}{chapter}{\bibliographytocname}
\bibliography{references}

% Appendix
\clearpage
\addappheadtotoc
\appendix
\appendixpage
% \include{appendix/experiment-instruments}

\end{document}
% %% RiSE Latex Template - version 0.5
%%
%% RiSE's latex template for thesis and dissertations
%% http://risetemplate.sourceforge.net
%%
%% (c) 2012 Yguaratã Cerqueira Cavalcanti (yguarata@gmail.com)
%%          Vinicius Cardoso Garcia (vinicius.garcia@gmail.com)
%%
%% This document was initially based on UFPEThesis template, from Paulo Gustavo
%% S. Fonseca.
%%
%% ACKNOWLEDGEMENTS
%%
%% We would like to thanks the RiSE's researchers community, the 
%% students from Federal University of Pernambuco, and other users that have
%% been contributing to this projects with comments and patches.
%%
%% GENERAL INSTRUCTIONS
%%
%% We strongly recommend you to compile your documents using pdflatex command.
%% It is also recommend use the texlipse plugin for Eclipse to edit your documents.
%%
%% Options for \documentclass command:
%%         * Idiom
%%           pt   - Portguese (default)
%%           en   - English
%%
%%         * Text type
%%           bsc  - B.Sc. Thesis
%%           msc  - M.Sc. Thesis (default)
%%           qual - PHD qualification (not tested yet)
%%           prop - PHD proposal (not tested yet)
%%           phd  - PHD thesis
%%
%%         * Media
%%           scr  - to eletronic version (PDF) / see the users guide
%%
%%         * Pagination
%%           oneside - unique face press
%%           twoside - two faces press
%%
%%		   * Line spacing
%%           singlespacing  - the same as using \linespread{1}
%%           onehalfspacing - the same as using \linespread{1.3}
%%           doublespacing  - the same as using \linespread{1.6}
%%
%% Reference commands. Use the following commands to make references in your
%% text:
%%          \figref  -- for Figure reference
%%          \tabref  -- for Table reference
%%          \eqnref  -- for equation reference
%%          \chapref -- for chapter reference
%%          \secref  -- for section reference
%%          \appref  -- for appendix reference
%%          \axiref  -- for axiom reference
%%          \conjref -- for conjecture reference
%%          \defref  -- for definition reference
%%          \lemref  -- for lemma reference
%%          \theoref -- for theorem reference
%%          \corref  -- for corollary reference
%%          \propref -- for proprosition reference
%%          \pgref   -- for page reference
%%
%%          Example: See \chapref{chap:introduction}. It will produce 
%%                   'See Chapter 1', in case of English language.

\documentclass[pt,twoside,onehalfspacing,bsc]{risethesis}

\usepackage[sort,square]{natbib} %package for bibliography
\bibliographystyle{ieeetr} %authordate1

\usepackage[utf8]{inputenc}
\usepackage[brazilian]{babel}
\usepackage[T1]{fontenc}

\usepackage{graphicx}

%% Change the following pdf author attribute name to your name.
\usepackage[linkcolor=blue,citecolor=blue,urlcolor=blue,colorlinks,pdfpagelabels,pdftitle={Bruno Cabral's Bachelor Thesis},pdfauthor={Bruno Cabral}]{hyperref}

\address{SALVADOR}

\universitypt{Universidade Federal da Bahia}
\universityen{Federal University of Bahia}

\departmentpt{Depertamento de Ciência da Computação}
\departmenten{Computer Science Department}

\programpt{Programa Multiinstitucional de Pós-graduação em Ciência da Computação}
\programen{Graduate in Computer Science}

\majorfieldpt{Ciência da Computação}
\majorfielden{Computer Science}

\title{Sistema de apoio à Pós graduação - UFBA}
\date{Outubro/2016}

\author{Victor de Azevedo Nunes}
\adviser{Ivan do Carmo Machado}

\begin{document}

\frontmatter
\frontpage
\presentationpage

\begin{dedicatory}
Eu dedico esta dissertação...
%I dedicate this dissertation to my family, girlfriend, friends and
%professors who gave me all necessary support to get here.
\end{dedicatory}

\acknowledgements
Meus agradecimentos...

\begin{epigraph}[]{Edward V Berard}
Walking on water and developing software from a specification are easy if both are frozen
\end{epigraph}

\resumo
% Escreva seu resumo no arquivo resumo.tex
\input{resumo}

\abstract
% Write your abstract in a file called abstract.tex
\input{abstract}

% Summary (tables of contents)
\tableofcontents

% List of figures
\listoffigures

% List of tables
\listoftables

% List of acronyms
% Acronyms manual: http://linorg.usp.br/CTAN/macros/latex/contrib/acronym/acronym.pdf
\listofacronyms
\input{acronyms}

% List of listings
%\lstlistoflistings

\mainmatter

\include{chapters/intro}
\include{chapters/referencial_teorico}

% \include{chapters/introduction/main}
% \include{chapters/background/main}
% \include{chapters/proposed_solution/main}
% \include{chapters/experiment/main}
% \include{chapters/conclusion/main}

%\bibliographystyle{natbib}
%\addcontentsline{toc}{chapter}{\bibliographytocname}
\bibliography{references}

% Appendix
\clearpage
\addappheadtotoc
\appendix
\appendixpage
% \include{appendix/experiment-instruments}

\end{document}

%\bibliographystyle{natbib}
%\addcontentsline{toc}{chapter}{\bibliographytocname}
\bibliography{references}

% Appendix
\clearpage
\addappheadtotoc
\appendix
\appendixpage
% \include{appendix/experiment-instruments}

\end{document}
% %% RiSE Latex Template - version 0.5
%%
%% RiSE's latex template for thesis and dissertations
%% http://risetemplate.sourceforge.net
%%
%% (c) 2012 Yguaratã Cerqueira Cavalcanti (yguarata@gmail.com)
%%          Vinicius Cardoso Garcia (vinicius.garcia@gmail.com)
%%
%% This document was initially based on UFPEThesis template, from Paulo Gustavo
%% S. Fonseca.
%%
%% ACKNOWLEDGEMENTS
%%
%% We would like to thanks the RiSE's researchers community, the 
%% students from Federal University of Pernambuco, and other users that have
%% been contributing to this projects with comments and patches.
%%
%% GENERAL INSTRUCTIONS
%%
%% We strongly recommend you to compile your documents using pdflatex command.
%% It is also recommend use the texlipse plugin for Eclipse to edit your documents.
%%
%% Options for \documentclass command:
%%         * Idiom
%%           pt   - Portguese (default)
%%           en   - English
%%
%%         * Text type
%%           bsc  - B.Sc. Thesis
%%           msc  - M.Sc. Thesis (default)
%%           qual - PHD qualification (not tested yet)
%%           prop - PHD proposal (not tested yet)
%%           phd  - PHD thesis
%%
%%         * Media
%%           scr  - to eletronic version (PDF) / see the users guide
%%
%%         * Pagination
%%           oneside - unique face press
%%           twoside - two faces press
%%
%%		   * Line spacing
%%           singlespacing  - the same as using \linespread{1}
%%           onehalfspacing - the same as using \linespread{1.3}
%%           doublespacing  - the same as using \linespread{1.6}
%%
%% Reference commands. Use the following commands to make references in your
%% text:
%%          \figref  -- for Figure reference
%%          \tabref  -- for Table reference
%%          \eqnref  -- for equation reference
%%          \chapref -- for chapter reference
%%          \secref  -- for section reference
%%          \appref  -- for appendix reference
%%          \axiref  -- for axiom reference
%%          \conjref -- for conjecture reference
%%          \defref  -- for definition reference
%%          \lemref  -- for lemma reference
%%          \theoref -- for theorem reference
%%          \corref  -- for corollary reference
%%          \propref -- for proprosition reference
%%          \pgref   -- for page reference
%%
%%          Example: See \chapref{chap:introduction}. It will produce 
%%                   'See Chapter 1', in case of English language.

\documentclass[pt,twoside,onehalfspacing,bsc]{risethesis}

\usepackage[sort,square]{natbib} %package for bibliography
\bibliographystyle{ieeetr} %authordate1

\usepackage[utf8]{inputenc}
\usepackage[brazilian]{babel}
\usepackage[T1]{fontenc}

\usepackage{graphicx}

%% Change the following pdf author attribute name to your name.
\usepackage[linkcolor=blue,citecolor=blue,urlcolor=blue,colorlinks,pdfpagelabels,pdftitle={Bruno Cabral's Bachelor Thesis},pdfauthor={Bruno Cabral}]{hyperref}

\address{SALVADOR}

\universitypt{Universidade Federal da Bahia}
\universityen{Federal University of Bahia}

\departmentpt{Depertamento de Ciência da Computação}
\departmenten{Computer Science Department}

\programpt{Programa Multiinstitucional de Pós-graduação em Ciência da Computação}
\programen{Graduate in Computer Science}

\majorfieldpt{Ciência da Computação}
\majorfielden{Computer Science}

\title{Sistema de apoio à Pós graduação - UFBA}
\date{Outubro/2016}

\author{Victor de Azevedo Nunes}
\adviser{Ivan do Carmo Machado}

\begin{document}

\frontmatter
\frontpage
\presentationpage

\begin{dedicatory}
Eu dedico esta dissertação...
%I dedicate this dissertation to my family, girlfriend, friends and
%professors who gave me all necessary support to get here.
\end{dedicatory}

\acknowledgements
Meus agradecimentos...

\begin{epigraph}[]{Edward V Berard}
Walking on water and developing software from a specification are easy if both are frozen
\end{epigraph}

\resumo
% Escreva seu resumo no arquivo resumo.tex
Este trabalho de conclusão de curso utiliza elementos da engenharia de software para propor um software como serviço a fim de otimizar o processo de construção e manutenção dos softwares. Assim, o objetivo deste SaaS é gerenciar as permissões de acesso de sistemas cliente, provendo o reuso de software e padronizando as soluções. O sistema proposto neste trabalho encontra-se implementado e disponível no Github, e traz fundamentos sobre a arquitetura e tecnologias utilizadas, bem como avaliações sobre possibilidades de utilização do mesmo em diversos ambientes, como Web e mobile.

\begin{keywords}
Software; Reuso; SaaS; Web; PHP
\end{keywords}

\abstract
% Write your abstract in a file called abstract.tex
My abstract...

\begin{keywords}
key words...
\end{keywords}

% Summary (tables of contents)
\tableofcontents

% List of figures
\listoffigures

% List of tables
\listoftables

% List of acronyms
% Acronyms manual: http://linorg.usp.br/CTAN/macros/latex/contrib/acronym/acronym.pdf
\listofacronyms
\begin{acronym}[ACRONYM] 
% Change the word ACRONYM above to change the acronym column width.
% The column width is equals to the width of the word that you put.
% Read the manual about acronym package for more examples:
%   http://linorg.usp.br/CTAN/macros/latex/contrib/acronym/acronym.pdf

\acro{SPA}{Single Page Application}
\acro{JSON}{Javascript Object Notation}
\acro{PHP}{PHP: Hypertext Preprocessor}
\acro{SaaS}{Software as a Service}
\acro{ERP}{Enterprise Resource Planning}
\acro{QoS}{Quality of Service}
\acro{UML}{Unified Modeling Language}
\acro{MVC}{Model-View-Controller}
\acro{Ajax}{Asynchronous Javascript and XML}
\acro{HTML}{HyperText Markup Language}
\acro{CSS}{Cascading Style Sheets}
\acro{API}{Application Programming Interface}
\acro{DOM}{Document Object Model}
\acro{BPMN}{Business Process Model and Notation}
\acro{REST}{Representational State Transfer}

\end{acronym}

% List of listings
%\lstlistoflistings

\mainmatter

\chapter{Introdução}

\section{Motivação}

Organizar os procedimentos de um processo sempre nos traz vantagens. Apesar de no processo de implantação de um sistema, o mesmo burocratizar o processo, com o tempo temos o retorno da dedicação para a inserção dos dados. Com um certo volume de dados, é possível estruturar informações que num processo manual são difíceis de serem enxergadas. Assim, é possível depender menos das pessoas que organizam o processo, pois o legado de informações não estará mais somente na mente de alguns, mas sim documentado nos dados do sistema.

Além de colaborar na organização, também haverá uma grande colaboração no tempo gasto na gestão. Lidar com muitos papéis e confiar na mente humana para guardar informações, não é uma alternativa muito segura devido ao fato que as pessoas sempre estão sujeitas a sair do processo e levar contigo a experiência obtida. Experiência essa que faz com que os procedimentos sejam executados de forma mais eficiente. Entretanto, com um sistema inteligente, é possível auxiliar e tornar mais ágil a execução das tarefas.


\section{Problema}


De acordo com funcionários ligados ao o setor de pós graduação da UFBA, entrevistados a fim de um maior entendimento do cenário, apesar das semelhanças estruturais, a pós graduação gerida de forma diferencia da graduação. FULANO afirma que devido ao fato de não ter a mesma visibilidade, não tem acesso aos mesmos recursos de gestão acadêmica da graduação. O professores não executam somente atividades dentro da sala de aula, também tem diversas outras ocupações no setor. E muitos procedimentos realizados extra classe ainda se encontram sendo realizados de forma manual, estando mais vulnerável ao erro ou até mesmo à violação do processo. Também ocorre um grande desperdício de tempo pelos professores e gestores da área, devido ao diversos processos ainda realizados de forma manual, sem a devida documentação. Segundo FULANO, também entrevistado, esse tempo perdido implica numa redução da eficiência na sala de aula, pois o professor acaba por ter menos tempo disponível para o planejamento das atividades, o que gera impactos negativos aos alunos.


\section{Objetivos} %<o que deve ser feito/entregue>


Devido aos muitos processos sendo resolvidos de forma manual, propõe-se com solução um sistema moderno, arquitetado para ter funcionamento na web e com um módulo mobile, a fim de fornecer informações de forma rápida e eficiente para os professores através de notificações, já que o acesso à internet móvel é comum entre os possíveis usuários do sistema em questão.
O principal requisito para o sistema seria dispor recursos para reduzir o tempo desperdiçado pelos professores durante as atividades extra classe.


\section{Metodologia} %<como será feito | como resolver o problema apontado inicialmente>


%<analise de literatura | design | implementação | validação>
Baseando-se nas tecnologias gratuitas em alta no cenário atual do desenvolvimento web, dispomos de algumas opções eficientes para a implementação da solução. Dentre as possibilidades, considerando a facilidade para futura manutenção e continuidade do projeto, tende-se a optar por uma tecnologia popular. Como linguagem de programação, adota-se o PHP. A escolha é fundamentada de acordo com a pesquisa da RedMonk de 2015, que evidencia o uso das linguagens de programação de acordo com as discussões no StackOverflow e repositórios no GitHub. É possível constatar a popularidade do PHP no cenário atual com o gráfico da pesquisa citada, na qual o PHP é apresentado na terceira colocação, apenas atrás do lider JavaScript e do segundo colocado, o Java.

\begin{figure}
	\label{fig:graficoRedmonk}
	\includegraphics[width=1\textwidth]{img/grafico_redmonk}
	\caption{Ranking das liguagens de programação no Stack Overflow e Github}
\end{figure}


Ainda assim, para compor a interface do dado projeto, também ocorrerá o uso do líder JavaScript de forma intensa, provendo o elo com o as informações gerenciadas pelo PHP.


Entretanto, não seria inteligente desenvolver um sistema completo sem o auxílio de um framework. Dentre os frameworks disponíveis para PHP, hoje o destaque está com o Laravel, que se encontra no topo dentre os mais utilizados no momento. 


A WebHostFace, uma empresa de hospedagem, compilou várias estatísticas para criar um infográfico mostrando os frameworks PHP mais populares de 2015. Utilizando informações sobre os próprios clientes, o Google Trends, estatísticas de repositórios do GitHub e a pesquisa do SitePoint “Best PHP Frameworks 2015”, a WebHostFace elaborou o seguinte infográfico: 

\begin{figure}
	\label{fig:graficoWebhostface}
	\includegraphics[width=1\textwidth]{img/infografico_webhostface}
	\caption{Infográfico da WebhostFace, exibindo a popularidade dos Frameworks PHP em 2015}
\end{figure}

Assim, tem-se a evidência que o Laravel em 2015 teve a maior popularidade em projetos pessoais e tem a maior comunidade entre os concorrentes, o que o torna uma boa escolha para a escrita de um software que será continuado por terceiros.


Para elaborar os recursos de interface e integrar ao back-end PHP do sistema, será adotado o já conhecido AngularJS, ferramenta sólida e conhecida no aspecto em questão. 


Dados coletados via Google Trends, que propõe comparações entre termos pesquisados, revela a popularidade do AngularJs diante de alguns dos principais concorrentes. O gráfico abaixo evidencia o cenário.


%Como mostra a Figura \ref{fig:graficoGoogleTrendsFerramentasFront}. 
\begin{figure}
	\label{fig:graficoGoogleTrendsFerramentasFront}
	\includegraphics[width=1\textwidth]{img/grafico_ferramentas_front}
	\caption{Gráfico do Google Trends exibindo as pesquisas por ferramentas front-end}
\end{figure}


Junto ao Angular JS, será utilizada a agradável tendência de interface do Material Design da Google, que propõe layouts limpos e otimizados já conhecidos pelos usuários de smartphones Android. 


Para a elaboração da plataforma mobile do projeto, será utilizado o Ionic Framework, muito difundido e bastante pesquisado na área, o que fica evidenciado com o gráfico de pesquisbaixo, coletado via Google Trends buscando por frameworks de desenvolvimento híbrido mobile.


\begin{figure}
	\label{fig:graficoGoogleTrendsFerramentasHibridasMobile}
	\includegraphics[width=1\textwidth]{img/grafico_ferramentas_hibridas_mobile}
	\caption{Gráfico do Google Trends exibindo as pesquisas por Frameworks híbridos mobile}
\end{figure}	

Para layout da interface mobile, também será aplicado a tendência do Material Design, a fim de propor uma harmonia entre o módulo web e mobile para os usuários


\section{Resultados Esperados}


Como fruto de um sistema para pós-graduação da UFBA, espera-se que os professores tenham mais recursos para integrar as atividades e também prover melhores condições para acompanhamento da vida acadêmica dos alunos em questão. Também, que os novos colaboradores que entrarem no processo tenham facilidade de compreender o fluxo do setor ao navegar pelo sistema proposto.


\section{Fora de Escopo}


Interação com os alunos devido às complicações para realizar a integração com o sistema empregado na UFBA, gerenciado pela XXXXXX, o que causaria uma inviabilidade no projeto devido à necessidade de entrega do produto ser mais forte que o tempo necessário para executar o processo de obtenção de acesso ao sistema legado para realizar a integração.


\section{Estrutura do Trabalho}


<breve resumo sobre os capítulos do TCC>
\chapter{Referencial Teórico}


Projetar o desenvolvimento de um software requer muito planejamento, pois as falhas iniciais podem custar bastante caro ou até mesmo inviabilizar a continuação de um projeto. Assim, a escolha da arquitetura ideal para a aplicabilidade é essencial na concepção de um produto de software. 
De todo o modo, sempre busca-se fazer mais com menos. Diante de tal filosofia, temos neste capítulo, uma breve discussão sobre alguns elementos de projeto e arquitetura de software, a fim de contextualizar este trabalho de conclusão de curso.
O capítulo corrente é composto por quatro seções. A \ref{sec:saas} trata de Software como serviço, discutindo alguns elementos do contexto que são relevantes para o trabalho proposto. A \ref{sec:reuso}  discute sobre a empregabilidade do reuso de software. A \ref{sec:modularizacao} seção, trata sobre aspectos envolvidos na modularização dos softwares. Por fim, a \ref{sec:apps_web} aborda as aplicações web, discutindo sobre aspectos relevantes sobre a aplicação web que compõe este trabalho.


 \section{Software como serviço}\label{sec:saas}


Segundo La e Chun \citep{La2009Systematic}, o princípio da definição de Software como um Serviço (Sofware as a Service - SaaS) é um serviço complementar para aplicações da computação em nuvem (cloud computing). No entanto, as áreas não se confundem. SaaS deve ser entendido como um mecanismo de suporte às soluções existentes na cloud. Os SaaS existem justamente para maximizar o reuso de serviços repetidos e não centrais em uma aplicação remota.


Como vantagens, diversos fatores podem ser favoráveis para a adoção de um SaaS, como custo e manutenção dentre outros fatores aplicáveis a soluções específicas. Lechesaet al. \citep{LechesaSS11}, quantificam índices sobre os fatores determinantes para adoção ou não de um SaaS voltado para ERP na África do Sul. Os principais fatores determinantes para adoção desse mecanismo de software são sua fluidez quanto à rede e a segurança. Esses fatores estão presentes na aplicação desenvolvida neste trabalho de conclusão de curso.


Devido ao fato de ter um serviço constantemente na nuvem, fica o questionamento sobre a segurança da informação manipulada. Sabe-se que a vulnerabilidade na Web não é restrita ao SaaS, atingindo diversos âmbitos. Rai et al. \cite{journals/corr/RaiSM13} consideram que o avanço da computação em nuvem não é um problema apenas para os serviços Web do ponto de vista da segurança, pois muitos trabalhos na literatura mostram a área como mais um ponto de vulnerabilidade para diversos setores, a exemplo de infraestrutura. No mesmo artigo mencionado de Rai et al. \cite{journals/corr/RaiSM13}, também realizaram-se estudos exploratórios junto a empresas usuárias de serviços em computação em nuvem e consideram que a perspectiva de SaaS também pode fortalecer a segurança nas aplicações de cloud computing, pois o software de autenticação compartilhado por várias aplicações em nuvem, oferece uma melhor padronização e consequente facilidade de prevenção a erros de vulnerabilidade específicas de cada módulo da pesquisa. Esse ponto de vista é fundamenta para a compreensão de SaaS.


A arquitetura de armazenamento de dados de um Saas pode variar de acordo com a necessidade do contexto. Huixin \cite{7586486} descreve possíveis modelagens nesse sentido. Essa abordagem pode ser com um banco de dados único, fazendo com que diferentes clientes compartilhem o mesmo banco, diferindo os dados através de controle de usuário, ou isolando os diferentes clientes através de bancos de dados exclusivos para cada um. Esse fator também pode ser combinado com a arquitetura da aplicação, caso ofereça aplicação única para todos os clientes ou aplicação compartilhada. Diante das possíveis abordagens, a modelagem de dados do software pode ser decidida pela regra de negócio. Este trabalho optou por aplicação única e banco de dados compartilhado.

Devido ao conceito particular de obtenção de software, tanto pela visão do cliente como pela visão do vendedor, é necessário compreender os diversos aspectos que podem ser relevantes ao orçar um Saas. O recente trabalho de T. Kaur et al. \citep{6949281} orienta um modelo para compor o custo de um Saas. O custo total seria composto pelos fatores que dão suporte ao funcionamento do software. Tais fatores incluem infra-estrutura, configurabilidade, customização, parâmetros de QoS (Quality of service) como escalabilidade, disponibilidade, usabilidade, pontualidade e desempenho da resposta, portabilidade, custo total de propriedade e retorno do investimento. Esses fatores caracterizam o custo de forma eficaz, possibilitando ao fornecedor, prover um Serviço de acordo com a exigência do consumidor em vários pacotes de serviços.


O conceito de software como serviço se aplica neste trabalho de conclusão de curso, pois o mesmo estará disponível na web com alta disponibilidade adotando as características apresentadas para qualquer pessoa que desejar usá-lo. Assim, basta se cadastrar, configurá-lo e usar como um serviço, sem preocupação com a manutenção do mesmo.


\section{Reuso de software}\label{sec:reuso} %CRUISE BOOK CAPITULO 2


De acordo com o livro Software pratical reuse \citep{Ezran2002}, o reuso de software é a utilização de qualquer informação que um desenvolvedor pode necessitar no processo de criação de software. O livro de Basili e Rombach \citep{Basili1991} define reutilização de software como o uso de tudo o que está associado a um projeto de conhecimento.
Assim, o objetivo da reutilização de software é reciclar o design, código e outros componentes de um produto de software e assim reduzir o custo, o tempo e melhorar a qualidade do produto.
Segundo Keswani et al. \cite{6783445}, o componente reutilizável de software pode ser qualquer parte de seu desenvolvimento, como um fragmento de código, design, casos de teste, ou até mesmo a especificação de requisitos de uma funcionalidade do software. 

O reuso de software pode ter impacto positivo em diversos aspectos do software, vejamos alguns, conforme apresentados no C.R.U.I.S.E Book \citep{cruise2007} :

\begin{itemize}

\item Qualidade: As correções de erro tornam-se úteis em todos os locais em que ocorreu, padronizando e facilitando a manutenção.

\item Produtividade: O ganho de produtividade é alcançado devido ao menor número de artefatos desenvolvido. Isso resulta em menor esforço de teste e também análise e design, reduzindo custos.

\item Confiabilidade: A utilização de componentes bem testados aumenta a
confiança no software. Além disso, a utilização de um mesmo componente em vários sistemas, aumenta a possibilidade de detecção de erros e reforça a confiança no componente.

\item Redução do Esforço: A reutilização de software proporciona uma redução do tempo de desenvolvimento, o que reduz o tempo necessário para o produto ser disponibilizado no mercado para trazer rentabilidade.

\item Trabalho redundante e tempo de desenvolvimento: Desenvolver um sistema do
zero significa desenvolvimento redundante de muitos componentes, como requisitos, especificações, casos de uso, arquitetura, etc. Isso pode ser evitado quando estes estão disponíveis como componentes reutilizáveis e podem ser compartilhados, resultando em um processo de desenvolvimento otimizado.

\item Documentação: Embora a documentação seja muito importante para a
manutenção de um sistema, muitas vezes é negligenciada. A reutilização de componentes de software reduz a quantidade de documentação a ser escrita, entretanto depende da qualidade do que está escrito. Assim, apenas a estrutura do sistema e os novos artefatos desenvolvidos necessitam ser documentados.

\item Custo de manutenção: Menos defeitos e manutenções são esperados quando tem-se comprovada a qualidade dos componentes utilizados.

\item Tamanho da equipe: É comum haver casos em que a equipe de desenvolvimento sofre sobrecarga. Entretanto, dobrar o tamanho da equipe de desenvolvimento não necessariamente duplica produtividade. Se muitos componentes podem ser reutilizados, é possível desenvolver com equipes menores, levando a melhor comunicação e aumento da produtividade.

\end{itemize}

Apesar dos benefícios da reutilização de software, ela não é suficientemente aproveitada. Existem fatores que influenciam direta ou indiretamente na sua adoção. Esses fatores podem ser de aspecto gerencial, organizacional, econômico, conceitual ou técnico. Veremos a seguir alguns aspectos que podem gerar conflito com a cultura de reuso de software, segundo o C.R.U.I.S.E Book \citep{cruise2007} :
%(Sametinger, 1997). REVER

\begin{itemize}
	
\item Falta de apoio da gestão: Como a reutilização de software gera custos iniciais,
a medida pode não ser amplamente alcançada em uma organização sem o apoio de alto nível de gestão. Os gestores têm de ser informados sobre os custos iniciais e serem convencidos sobre economias futuras.

\item Gerenciamento do Projeto: Gerenciar projetos tradicionais é uma tarefa árdua, principalmente, os que praticam a reutilização de software. Utilizando a técnica em larga escala, tem-se impacto sobre todo o ciclo de vida do software.

\item Estruturas organizacionais inadequadas: As estruturas organizacionais devem
considerar diferentes necessidades que surgem quando a reutilização em larga escala está sendo adotada. Por exemplo, uma equipe particionada pode ser alocada somente para desenvolver, manter e certificar componentes reutilizáveis de software.

\item Incentivos de gestão: É comum a falta de incentivo para deixar os desenvolvedores gastarem tempo elaborando componentes do sistemas. A produtividade é muitas vezes medida apenas no tempo necessário para concluir um projeto. Assim, fazer qualquer trabalho além disso, embora benéfico para a empresa como um todo, diminui o seu sucesso. Mesmo quando os componentes reutilizáveis são utilizados, os benefícios obtidos são uma pequena fração do que poderia ser alcançado caso houvesse reutilização explícita, planejada e organizada.

\item Dificuldade de encontrar software reutilizável: Para reutilizar os componentes, devem existir formas eficientes de busca. Além disso, é importante ter um repositório bem organizado contendo componentes com um eficiente meio de acesso.

\item Não reutilização do software encontrado. O acesso fácil ao software existente
não necessariamente aumentar a reutilização. Os componentes reutilizáveis devem ser cuidadosamente especificados, projetados, implementados e documentados, pois em alguns casos, modificar e adaptar o código  pode ser mais custoso que a programação da funcionalidade necessária a partir do zero.

\item Modificação: É muito difícil encontrar um componente que funcione
exatamente da mesma maneira que queremos. Desta forma, são necessárias modificações e devem existir formas de determinar os seus efeitos sobre o componente.


\end{itemize}


A definição do reuso de software pode ser associada a este trabalho, pois ainda que como um serviço, estaremos adotando uma aplicação única para servir aos mais diversos softwares de uma empresa, por exemplo. Logo, ao adotar o projeto, pode ser interessante aplicar algumas métricas para tomar conhecimento de possíveis vantanjens como as citadas nessa seção, a exemplo de economia de tempo de desenvolvimento, e custo do projeto.


%Outra diretriz importante para a reutilização de software é reduzir o risco na criação de novos softwares. O risco tende a ser bastante reduzido se os componentes que estão sendo reutilizados têm as documentação, interfaces necessárias e devidamente testadas, fatores que contibruem para uma fácil integração.
%De acordo com Keswani et al. \citep{6783445}, para o reuso de software dar retornos apropriados, o processo deve ser sistemático e planejado. Qualquer organização que implemente a reutilização de software deve identificar os melhores métodos e estratégias de reutilização para obter a máxima produtividade. A reutilização de software ajuda a evitar software de engenharia a partir do zero, pois usa módulos de software existentes. A reutilização de software, embora seja uma tarefa difícil, especialmente para softwares antigos sem padrões de projeto, pode melhorar significativamente a produtividade e a qualidade de um produto de software. Embora a reutilização de software não seja um novo campo, ela pode dar grandes retornos em curto período de tempo.


\section{Modularização}\label{sec:modularizacao} %artigo de claudio pagina 222 introdução


%A modularidade vem desempenhando um papel predominante estágios emergentes das disciplinas de arquitetura de software [13]. Engenheiros de software consideram modularidade como princípio base na comparação entre arquiteturas alternativas  e arquitetura degeneração [9]. De fato, os engenheiros de software são incentivados a arquitecturas, baseando-se numa multiplicidade de mecanismos de modularidade disponíveis em: 
%(i) Linguagens de descrição de arquitetura (ADLs), como ACME [8], 
%(ii) catálogos de arquitetônicos [2, 13], e 
%(iii) conhecem bem princípios de alto nível, como interfaces de componentes estreitos, acoplamento arquitectónico reduzido e semelhantes.


Conforme é frisado por Wickramaarachchi e Lai \citep{7062705}, o conceito de modularização na indústria de software tem uma longa história e tem sido utilizado para melhorar o processo de desenvolvimento de software em diferentes estágios. Os principais conceitos por trás da modularização do software foram introduzidos por pesquisadores pioneiros há quarenta anos, com uma notável contribuição feita por Melvin Conway e David Parnas, que tem representação notável na engenharia de software.


Modularizar um software é um bom padrão a ser adotado. Segundo Wickramaarachchi e Lai \citep{7062705}, a modularização é importante na identificação de dependências e reduz as dificuldades diante de uma possível necessidade de grandes alterações. De uma perspectiva da engenharia de software, uma modularização geralmente tem várias vantagens, tais como: tornar a complexidade do software mais gerenciável, facilitar o trabalho paralelo e tornar o software mais maleável para acomodar o futuro incerto que um software pode ter. O objetivo final da modularização do software é aumentar a produtividade ea qualidade do software. Tal conceito encontra-se bastante difundido e estái incorporado em linguagens de programação e ferramentas de software. O presente trabalho favorece ao uso da modularização de um software e até mesmo pode ser considerado um módulo a ser acoplado a qualquer software, mediante a compatibilidade.


Ao realizar a adoção do trabalho proposto, fica bastante evidente a "responsabilidade" do mesmo no escopo do projeto. Assim, é possível tratá-lo como um módulo do projeto que o usa, mesmo que seja consumido como um serviço.


\section{Aplicações web}\label{sec:apps_web}


A popularidade das soluções Web aumentou exponencialmente na última década e todos os dias cresce o número de pessoas usuárias desse tipo de software. E seguindo um padrão próprio, Kumar et al. \citep{7813710} sugerem que para o desenvolvimento web, deve-se manter a prática eficaz de produzir diagramas UML. A abordagem baseada na web oferece uma maneira fácil e eficaz para gerenciar e controlar o processo de desenvolvimento por meio de artefatos de modelagem. Tal abordagem pode ser usada quando há uma exigência de lidar com mudanças muito rápidas e grandes em requisitos de forma muito eficaz em muito menos tempo, gerando assim um menor impacto. 


Para atender à fomentada demanda de aplicativos web, é necessário adotar métodos de desenvolvimentos que sejam ágeis, eficientes e de fácil manutenção. Yu Ping et al. \cite{1372143} propõem o uso do modelo MVC (Model, View e Controller) no desenvolvimento para softwares web. O modelo apresentado tornou-se um padrão popular e divide o software em camadas com propósito definido, tornando-o de mais fácil manutenção.


O Ajax (Asynchronous Javascript and XML) revolucionou a web. Conforme demonstrado por Yuping \citep{6845605}, ao usar a tecnologia Ajax, podemos enriquecer a experiência do usuário em aplicações baseadas em navegador de internet, e fornecer uma variedade de aplicações interativas para atender às necessidade de humanização das aplicações.
Os aplicativos Ajax em execução no navegador se comunicam com um servidor Web de forma assíncrona e atualizam apenas uma parte da página.


No artigo de Tesarik et al. \citep{4664349}, temos que o design de software SPA (Single page application) é uma maneira interessante de criar um software Web numa uma única página. Essa solução de página única sem navegação funciona apenas com base em técnicas dinâmicas e assíncronas, como o AJAX (citado no parágrafo anterior). No entanto, esta abordagem coloca o desenvolvedor antes de alguns desafios substanciais.
Para projetar a interface do usuário que mostra as informações, é importante projetar corretamente a tela para manipular os dados do aplicativo numa única página. O design da página deve ser elaborado para maximizar a decomposição da página em componentes distintos que encapsulam os principais casos de uso.
Também recomenda-se que para elaborar os artefatos visuais, explore os mais ricos recursos na implementação, como HTML5, JavaScript, Ajax, CSS3, e outras tecnologias que se apliquem. Entretanto, desenvolver uma rica interface com o uso de diversas tecnologias/frameworks pode ocasionar um esforço maior para explorar as possibilidades.
Tais componentes da interface de uma aplicação SPA normalmente são alimentados mediante o consumo de uma API Rest via requisições AJAX. Assim, é possível particionar as responsabilidades de processamento do software entre cliente e servidor.


Como é possível denotar no trabalho de Salvadori e Siqueire \citep{7195633}, REST é uma arquitetura muito popular para integração de aplicativos web, que permite compartilhar e reutilizar informações através de sistemas. Aplicações de grande escala baseadas em no REST deve ser implementadas utilizando estratégias e mecanismos para produzir sistemas que sejam fáceis de desenvolver, reutilizar e manter. As interfaces de integração fornecidas por esses sistemas, chamadas de Web API, têm uma influência importante nas características da implementação resultante, pois a sua forma de resposta deve ser casada com a interface que o consome.
 
 
Os elementos comentados nessa seção encontram-se presentes neste trabalho de conclusão de curso. Juntos, montam a estrutura tecnológica necessária aliada à arquitetura adotada. Esses elementos seguem tendências atuais dos softwares web, fazendo com que esse trabalho esteja composto por tecnologias modernas que estão com boa aceitação no mercado.

% %% RiSE Latex Template - version 0.5
%%
%% RiSE's latex template for thesis and dissertations
%% http://risetemplate.sourceforge.net
%%
%% (c) 2012 Yguaratã Cerqueira Cavalcanti (yguarata@gmail.com)
%%          Vinicius Cardoso Garcia (vinicius.garcia@gmail.com)
%%
%% This document was initially based on UFPEThesis template, from Paulo Gustavo
%% S. Fonseca.
%%
%% ACKNOWLEDGEMENTS
%%
%% We would like to thanks the RiSE's researchers community, the 
%% students from Federal University of Pernambuco, and other users that have
%% been contributing to this projects with comments and patches.
%%
%% GENERAL INSTRUCTIONS
%%
%% We strongly recommend you to compile your documents using pdflatex command.
%% It is also recommend use the texlipse plugin for Eclipse to edit your documents.
%%
%% Options for \documentclass command:
%%         * Idiom
%%           pt   - Portguese (default)
%%           en   - English
%%
%%         * Text type
%%           bsc  - B.Sc. Thesis
%%           msc  - M.Sc. Thesis (default)
%%           qual - PHD qualification (not tested yet)
%%           prop - PHD proposal (not tested yet)
%%           phd  - PHD thesis
%%
%%         * Media
%%           scr  - to eletronic version (PDF) / see the users guide
%%
%%         * Pagination
%%           oneside - unique face press
%%           twoside - two faces press
%%
%%		   * Line spacing
%%           singlespacing  - the same as using \linespread{1}
%%           onehalfspacing - the same as using \linespread{1.3}
%%           doublespacing  - the same as using \linespread{1.6}
%%
%% Reference commands. Use the following commands to make references in your
%% text:
%%          \figref  -- for Figure reference
%%          \tabref  -- for Table reference
%%          \eqnref  -- for equation reference
%%          \chapref -- for chapter reference
%%          \secref  -- for section reference
%%          \appref  -- for appendix reference
%%          \axiref  -- for axiom reference
%%          \conjref -- for conjecture reference
%%          \defref  -- for definition reference
%%          \lemref  -- for lemma reference
%%          \theoref -- for theorem reference
%%          \corref  -- for corollary reference
%%          \propref -- for proprosition reference
%%          \pgref   -- for page reference
%%
%%          Example: See \chapref{chap:introduction}. It will produce 
%%                   'See Chapter 1', in case of English language.

\documentclass[pt,twoside,onehalfspacing,bsc]{risethesis}

\usepackage[sort,square]{natbib} %package for bibliography
\bibliographystyle{ieeetr} %authordate1

\usepackage[utf8]{inputenc}
\usepackage[brazilian]{babel}
\usepackage[T1]{fontenc}

\usepackage{graphicx}

%% Change the following pdf author attribute name to your name.
\usepackage[linkcolor=blue,citecolor=blue,urlcolor=blue,colorlinks,pdfpagelabels,pdftitle={Bruno Cabral's Bachelor Thesis},pdfauthor={Bruno Cabral}]{hyperref}

\address{SALVADOR}

\universitypt{Universidade Federal da Bahia}
\universityen{Federal University of Bahia}

\departmentpt{Depertamento de Ciência da Computação}
\departmenten{Computer Science Department}

\programpt{Programa Multiinstitucional de Pós-graduação em Ciência da Computação}
\programen{Graduate in Computer Science}

\majorfieldpt{Ciência da Computação}
\majorfielden{Computer Science}

\title{Sistema de apoio à Pós graduação - UFBA}
\date{Outubro/2016}

\author{Victor de Azevedo Nunes}
\adviser{Ivan do Carmo Machado}

\begin{document}

\frontmatter
\frontpage
\presentationpage

\begin{dedicatory}
Eu dedico esta dissertação...
%I dedicate this dissertation to my family, girlfriend, friends and
%professors who gave me all necessary support to get here.
\end{dedicatory}

\acknowledgements
Meus agradecimentos...

\begin{epigraph}[]{Edward V Berard}
Walking on water and developing software from a specification are easy if both are frozen
\end{epigraph}

\resumo
% Escreva seu resumo no arquivo resumo.tex
\input{resumo}

\abstract
% Write your abstract in a file called abstract.tex
\input{abstract}

% Summary (tables of contents)
\tableofcontents

% List of figures
\listoffigures

% List of tables
\listoftables

% List of acronyms
% Acronyms manual: http://linorg.usp.br/CTAN/macros/latex/contrib/acronym/acronym.pdf
\listofacronyms
\input{acronyms}

% List of listings
%\lstlistoflistings

\mainmatter

\include{chapters/intro}
\include{chapters/referencial_teorico}

% \include{chapters/introduction/main}
% \include{chapters/background/main}
% \include{chapters/proposed_solution/main}
% \include{chapters/experiment/main}
% \include{chapters/conclusion/main}

%\bibliographystyle{natbib}
%\addcontentsline{toc}{chapter}{\bibliographytocname}
\bibliography{references}

% Appendix
\clearpage
\addappheadtotoc
\appendix
\appendixpage
% \include{appendix/experiment-instruments}

\end{document}
% %% RiSE Latex Template - version 0.5
%%
%% RiSE's latex template for thesis and dissertations
%% http://risetemplate.sourceforge.net
%%
%% (c) 2012 Yguaratã Cerqueira Cavalcanti (yguarata@gmail.com)
%%          Vinicius Cardoso Garcia (vinicius.garcia@gmail.com)
%%
%% This document was initially based on UFPEThesis template, from Paulo Gustavo
%% S. Fonseca.
%%
%% ACKNOWLEDGEMENTS
%%
%% We would like to thanks the RiSE's researchers community, the 
%% students from Federal University of Pernambuco, and other users that have
%% been contributing to this projects with comments and patches.
%%
%% GENERAL INSTRUCTIONS
%%
%% We strongly recommend you to compile your documents using pdflatex command.
%% It is also recommend use the texlipse plugin for Eclipse to edit your documents.
%%
%% Options for \documentclass command:
%%         * Idiom
%%           pt   - Portguese (default)
%%           en   - English
%%
%%         * Text type
%%           bsc  - B.Sc. Thesis
%%           msc  - M.Sc. Thesis (default)
%%           qual - PHD qualification (not tested yet)
%%           prop - PHD proposal (not tested yet)
%%           phd  - PHD thesis
%%
%%         * Media
%%           scr  - to eletronic version (PDF) / see the users guide
%%
%%         * Pagination
%%           oneside - unique face press
%%           twoside - two faces press
%%
%%		   * Line spacing
%%           singlespacing  - the same as using \linespread{1}
%%           onehalfspacing - the same as using \linespread{1.3}
%%           doublespacing  - the same as using \linespread{1.6}
%%
%% Reference commands. Use the following commands to make references in your
%% text:
%%          \figref  -- for Figure reference
%%          \tabref  -- for Table reference
%%          \eqnref  -- for equation reference
%%          \chapref -- for chapter reference
%%          \secref  -- for section reference
%%          \appref  -- for appendix reference
%%          \axiref  -- for axiom reference
%%          \conjref -- for conjecture reference
%%          \defref  -- for definition reference
%%          \lemref  -- for lemma reference
%%          \theoref -- for theorem reference
%%          \corref  -- for corollary reference
%%          \propref -- for proprosition reference
%%          \pgref   -- for page reference
%%
%%          Example: See \chapref{chap:introduction}. It will produce 
%%                   'See Chapter 1', in case of English language.

\documentclass[pt,twoside,onehalfspacing,bsc]{risethesis}

\usepackage[sort,square]{natbib} %package for bibliography
\bibliographystyle{ieeetr} %authordate1

\usepackage[utf8]{inputenc}
\usepackage[brazilian]{babel}
\usepackage[T1]{fontenc}

\usepackage{graphicx}

%% Change the following pdf author attribute name to your name.
\usepackage[linkcolor=blue,citecolor=blue,urlcolor=blue,colorlinks,pdfpagelabels,pdftitle={Bruno Cabral's Bachelor Thesis},pdfauthor={Bruno Cabral}]{hyperref}

\address{SALVADOR}

\universitypt{Universidade Federal da Bahia}
\universityen{Federal University of Bahia}

\departmentpt{Depertamento de Ciência da Computação}
\departmenten{Computer Science Department}

\programpt{Programa Multiinstitucional de Pós-graduação em Ciência da Computação}
\programen{Graduate in Computer Science}

\majorfieldpt{Ciência da Computação}
\majorfielden{Computer Science}

\title{Sistema de apoio à Pós graduação - UFBA}
\date{Outubro/2016}

\author{Victor de Azevedo Nunes}
\adviser{Ivan do Carmo Machado}

\begin{document}

\frontmatter
\frontpage
\presentationpage

\begin{dedicatory}
Eu dedico esta dissertação...
%I dedicate this dissertation to my family, girlfriend, friends and
%professors who gave me all necessary support to get here.
\end{dedicatory}

\acknowledgements
Meus agradecimentos...

\begin{epigraph}[]{Edward V Berard}
Walking on water and developing software from a specification are easy if both are frozen
\end{epigraph}

\resumo
% Escreva seu resumo no arquivo resumo.tex
\input{resumo}

\abstract
% Write your abstract in a file called abstract.tex
\input{abstract}

% Summary (tables of contents)
\tableofcontents

% List of figures
\listoffigures

% List of tables
\listoftables

% List of acronyms
% Acronyms manual: http://linorg.usp.br/CTAN/macros/latex/contrib/acronym/acronym.pdf
\listofacronyms
\input{acronyms}

% List of listings
%\lstlistoflistings

\mainmatter

\include{chapters/intro}
\include{chapters/referencial_teorico}

% \include{chapters/introduction/main}
% \include{chapters/background/main}
% \include{chapters/proposed_solution/main}
% \include{chapters/experiment/main}
% \include{chapters/conclusion/main}

%\bibliographystyle{natbib}
%\addcontentsline{toc}{chapter}{\bibliographytocname}
\bibliography{references}

% Appendix
\clearpage
\addappheadtotoc
\appendix
\appendixpage
% \include{appendix/experiment-instruments}

\end{document}
% %% RiSE Latex Template - version 0.5
%%
%% RiSE's latex template for thesis and dissertations
%% http://risetemplate.sourceforge.net
%%
%% (c) 2012 Yguaratã Cerqueira Cavalcanti (yguarata@gmail.com)
%%          Vinicius Cardoso Garcia (vinicius.garcia@gmail.com)
%%
%% This document was initially based on UFPEThesis template, from Paulo Gustavo
%% S. Fonseca.
%%
%% ACKNOWLEDGEMENTS
%%
%% We would like to thanks the RiSE's researchers community, the 
%% students from Federal University of Pernambuco, and other users that have
%% been contributing to this projects with comments and patches.
%%
%% GENERAL INSTRUCTIONS
%%
%% We strongly recommend you to compile your documents using pdflatex command.
%% It is also recommend use the texlipse plugin for Eclipse to edit your documents.
%%
%% Options for \documentclass command:
%%         * Idiom
%%           pt   - Portguese (default)
%%           en   - English
%%
%%         * Text type
%%           bsc  - B.Sc. Thesis
%%           msc  - M.Sc. Thesis (default)
%%           qual - PHD qualification (not tested yet)
%%           prop - PHD proposal (not tested yet)
%%           phd  - PHD thesis
%%
%%         * Media
%%           scr  - to eletronic version (PDF) / see the users guide
%%
%%         * Pagination
%%           oneside - unique face press
%%           twoside - two faces press
%%
%%		   * Line spacing
%%           singlespacing  - the same as using \linespread{1}
%%           onehalfspacing - the same as using \linespread{1.3}
%%           doublespacing  - the same as using \linespread{1.6}
%%
%% Reference commands. Use the following commands to make references in your
%% text:
%%          \figref  -- for Figure reference
%%          \tabref  -- for Table reference
%%          \eqnref  -- for equation reference
%%          \chapref -- for chapter reference
%%          \secref  -- for section reference
%%          \appref  -- for appendix reference
%%          \axiref  -- for axiom reference
%%          \conjref -- for conjecture reference
%%          \defref  -- for definition reference
%%          \lemref  -- for lemma reference
%%          \theoref -- for theorem reference
%%          \corref  -- for corollary reference
%%          \propref -- for proprosition reference
%%          \pgref   -- for page reference
%%
%%          Example: See \chapref{chap:introduction}. It will produce 
%%                   'See Chapter 1', in case of English language.

\documentclass[pt,twoside,onehalfspacing,bsc]{risethesis}

\usepackage[sort,square]{natbib} %package for bibliography
\bibliographystyle{ieeetr} %authordate1

\usepackage[utf8]{inputenc}
\usepackage[brazilian]{babel}
\usepackage[T1]{fontenc}

\usepackage{graphicx}

%% Change the following pdf author attribute name to your name.
\usepackage[linkcolor=blue,citecolor=blue,urlcolor=blue,colorlinks,pdfpagelabels,pdftitle={Bruno Cabral's Bachelor Thesis},pdfauthor={Bruno Cabral}]{hyperref}

\address{SALVADOR}

\universitypt{Universidade Federal da Bahia}
\universityen{Federal University of Bahia}

\departmentpt{Depertamento de Ciência da Computação}
\departmenten{Computer Science Department}

\programpt{Programa Multiinstitucional de Pós-graduação em Ciência da Computação}
\programen{Graduate in Computer Science}

\majorfieldpt{Ciência da Computação}
\majorfielden{Computer Science}

\title{Sistema de apoio à Pós graduação - UFBA}
\date{Outubro/2016}

\author{Victor de Azevedo Nunes}
\adviser{Ivan do Carmo Machado}

\begin{document}

\frontmatter
\frontpage
\presentationpage

\begin{dedicatory}
Eu dedico esta dissertação...
%I dedicate this dissertation to my family, girlfriend, friends and
%professors who gave me all necessary support to get here.
\end{dedicatory}

\acknowledgements
Meus agradecimentos...

\begin{epigraph}[]{Edward V Berard}
Walking on water and developing software from a specification are easy if both are frozen
\end{epigraph}

\resumo
% Escreva seu resumo no arquivo resumo.tex
\input{resumo}

\abstract
% Write your abstract in a file called abstract.tex
\input{abstract}

% Summary (tables of contents)
\tableofcontents

% List of figures
\listoffigures

% List of tables
\listoftables

% List of acronyms
% Acronyms manual: http://linorg.usp.br/CTAN/macros/latex/contrib/acronym/acronym.pdf
\listofacronyms
\input{acronyms}

% List of listings
%\lstlistoflistings

\mainmatter

\include{chapters/intro}
\include{chapters/referencial_teorico}

% \include{chapters/introduction/main}
% \include{chapters/background/main}
% \include{chapters/proposed_solution/main}
% \include{chapters/experiment/main}
% \include{chapters/conclusion/main}

%\bibliographystyle{natbib}
%\addcontentsline{toc}{chapter}{\bibliographytocname}
\bibliography{references}

% Appendix
\clearpage
\addappheadtotoc
\appendix
\appendixpage
% \include{appendix/experiment-instruments}

\end{document}
% %% RiSE Latex Template - version 0.5
%%
%% RiSE's latex template for thesis and dissertations
%% http://risetemplate.sourceforge.net
%%
%% (c) 2012 Yguaratã Cerqueira Cavalcanti (yguarata@gmail.com)
%%          Vinicius Cardoso Garcia (vinicius.garcia@gmail.com)
%%
%% This document was initially based on UFPEThesis template, from Paulo Gustavo
%% S. Fonseca.
%%
%% ACKNOWLEDGEMENTS
%%
%% We would like to thanks the RiSE's researchers community, the 
%% students from Federal University of Pernambuco, and other users that have
%% been contributing to this projects with comments and patches.
%%
%% GENERAL INSTRUCTIONS
%%
%% We strongly recommend you to compile your documents using pdflatex command.
%% It is also recommend use the texlipse plugin for Eclipse to edit your documents.
%%
%% Options for \documentclass command:
%%         * Idiom
%%           pt   - Portguese (default)
%%           en   - English
%%
%%         * Text type
%%           bsc  - B.Sc. Thesis
%%           msc  - M.Sc. Thesis (default)
%%           qual - PHD qualification (not tested yet)
%%           prop - PHD proposal (not tested yet)
%%           phd  - PHD thesis
%%
%%         * Media
%%           scr  - to eletronic version (PDF) / see the users guide
%%
%%         * Pagination
%%           oneside - unique face press
%%           twoside - two faces press
%%
%%		   * Line spacing
%%           singlespacing  - the same as using \linespread{1}
%%           onehalfspacing - the same as using \linespread{1.3}
%%           doublespacing  - the same as using \linespread{1.6}
%%
%% Reference commands. Use the following commands to make references in your
%% text:
%%          \figref  -- for Figure reference
%%          \tabref  -- for Table reference
%%          \eqnref  -- for equation reference
%%          \chapref -- for chapter reference
%%          \secref  -- for section reference
%%          \appref  -- for appendix reference
%%          \axiref  -- for axiom reference
%%          \conjref -- for conjecture reference
%%          \defref  -- for definition reference
%%          \lemref  -- for lemma reference
%%          \theoref -- for theorem reference
%%          \corref  -- for corollary reference
%%          \propref -- for proprosition reference
%%          \pgref   -- for page reference
%%
%%          Example: See \chapref{chap:introduction}. It will produce 
%%                   'See Chapter 1', in case of English language.

\documentclass[pt,twoside,onehalfspacing,bsc]{risethesis}

\usepackage[sort,square]{natbib} %package for bibliography
\bibliographystyle{ieeetr} %authordate1

\usepackage[utf8]{inputenc}
\usepackage[brazilian]{babel}
\usepackage[T1]{fontenc}

\usepackage{graphicx}

%% Change the following pdf author attribute name to your name.
\usepackage[linkcolor=blue,citecolor=blue,urlcolor=blue,colorlinks,pdfpagelabels,pdftitle={Bruno Cabral's Bachelor Thesis},pdfauthor={Bruno Cabral}]{hyperref}

\address{SALVADOR}

\universitypt{Universidade Federal da Bahia}
\universityen{Federal University of Bahia}

\departmentpt{Depertamento de Ciência da Computação}
\departmenten{Computer Science Department}

\programpt{Programa Multiinstitucional de Pós-graduação em Ciência da Computação}
\programen{Graduate in Computer Science}

\majorfieldpt{Ciência da Computação}
\majorfielden{Computer Science}

\title{Sistema de apoio à Pós graduação - UFBA}
\date{Outubro/2016}

\author{Victor de Azevedo Nunes}
\adviser{Ivan do Carmo Machado}

\begin{document}

\frontmatter
\frontpage
\presentationpage

\begin{dedicatory}
Eu dedico esta dissertação...
%I dedicate this dissertation to my family, girlfriend, friends and
%professors who gave me all necessary support to get here.
\end{dedicatory}

\acknowledgements
Meus agradecimentos...

\begin{epigraph}[]{Edward V Berard}
Walking on water and developing software from a specification are easy if both are frozen
\end{epigraph}

\resumo
% Escreva seu resumo no arquivo resumo.tex
\input{resumo}

\abstract
% Write your abstract in a file called abstract.tex
\input{abstract}

% Summary (tables of contents)
\tableofcontents

% List of figures
\listoffigures

% List of tables
\listoftables

% List of acronyms
% Acronyms manual: http://linorg.usp.br/CTAN/macros/latex/contrib/acronym/acronym.pdf
\listofacronyms
\input{acronyms}

% List of listings
%\lstlistoflistings

\mainmatter

\include{chapters/intro}
\include{chapters/referencial_teorico}

% \include{chapters/introduction/main}
% \include{chapters/background/main}
% \include{chapters/proposed_solution/main}
% \include{chapters/experiment/main}
% \include{chapters/conclusion/main}

%\bibliographystyle{natbib}
%\addcontentsline{toc}{chapter}{\bibliographytocname}
\bibliography{references}

% Appendix
\clearpage
\addappheadtotoc
\appendix
\appendixpage
% \include{appendix/experiment-instruments}

\end{document}
% %% RiSE Latex Template - version 0.5
%%
%% RiSE's latex template for thesis and dissertations
%% http://risetemplate.sourceforge.net
%%
%% (c) 2012 Yguaratã Cerqueira Cavalcanti (yguarata@gmail.com)
%%          Vinicius Cardoso Garcia (vinicius.garcia@gmail.com)
%%
%% This document was initially based on UFPEThesis template, from Paulo Gustavo
%% S. Fonseca.
%%
%% ACKNOWLEDGEMENTS
%%
%% We would like to thanks the RiSE's researchers community, the 
%% students from Federal University of Pernambuco, and other users that have
%% been contributing to this projects with comments and patches.
%%
%% GENERAL INSTRUCTIONS
%%
%% We strongly recommend you to compile your documents using pdflatex command.
%% It is also recommend use the texlipse plugin for Eclipse to edit your documents.
%%
%% Options for \documentclass command:
%%         * Idiom
%%           pt   - Portguese (default)
%%           en   - English
%%
%%         * Text type
%%           bsc  - B.Sc. Thesis
%%           msc  - M.Sc. Thesis (default)
%%           qual - PHD qualification (not tested yet)
%%           prop - PHD proposal (not tested yet)
%%           phd  - PHD thesis
%%
%%         * Media
%%           scr  - to eletronic version (PDF) / see the users guide
%%
%%         * Pagination
%%           oneside - unique face press
%%           twoside - two faces press
%%
%%		   * Line spacing
%%           singlespacing  - the same as using \linespread{1}
%%           onehalfspacing - the same as using \linespread{1.3}
%%           doublespacing  - the same as using \linespread{1.6}
%%
%% Reference commands. Use the following commands to make references in your
%% text:
%%          \figref  -- for Figure reference
%%          \tabref  -- for Table reference
%%          \eqnref  -- for equation reference
%%          \chapref -- for chapter reference
%%          \secref  -- for section reference
%%          \appref  -- for appendix reference
%%          \axiref  -- for axiom reference
%%          \conjref -- for conjecture reference
%%          \defref  -- for definition reference
%%          \lemref  -- for lemma reference
%%          \theoref -- for theorem reference
%%          \corref  -- for corollary reference
%%          \propref -- for proprosition reference
%%          \pgref   -- for page reference
%%
%%          Example: See \chapref{chap:introduction}. It will produce 
%%                   'See Chapter 1', in case of English language.

\documentclass[pt,twoside,onehalfspacing,bsc]{risethesis}

\usepackage[sort,square]{natbib} %package for bibliography
\bibliographystyle{ieeetr} %authordate1

\usepackage[utf8]{inputenc}
\usepackage[brazilian]{babel}
\usepackage[T1]{fontenc}

\usepackage{graphicx}

%% Change the following pdf author attribute name to your name.
\usepackage[linkcolor=blue,citecolor=blue,urlcolor=blue,colorlinks,pdfpagelabels,pdftitle={Bruno Cabral's Bachelor Thesis},pdfauthor={Bruno Cabral}]{hyperref}

\address{SALVADOR}

\universitypt{Universidade Federal da Bahia}
\universityen{Federal University of Bahia}

\departmentpt{Depertamento de Ciência da Computação}
\departmenten{Computer Science Department}

\programpt{Programa Multiinstitucional de Pós-graduação em Ciência da Computação}
\programen{Graduate in Computer Science}

\majorfieldpt{Ciência da Computação}
\majorfielden{Computer Science}

\title{Sistema de apoio à Pós graduação - UFBA}
\date{Outubro/2016}

\author{Victor de Azevedo Nunes}
\adviser{Ivan do Carmo Machado}

\begin{document}

\frontmatter
\frontpage
\presentationpage

\begin{dedicatory}
Eu dedico esta dissertação...
%I dedicate this dissertation to my family, girlfriend, friends and
%professors who gave me all necessary support to get here.
\end{dedicatory}

\acknowledgements
Meus agradecimentos...

\begin{epigraph}[]{Edward V Berard}
Walking on water and developing software from a specification are easy if both are frozen
\end{epigraph}

\resumo
% Escreva seu resumo no arquivo resumo.tex
\input{resumo}

\abstract
% Write your abstract in a file called abstract.tex
\input{abstract}

% Summary (tables of contents)
\tableofcontents

% List of figures
\listoffigures

% List of tables
\listoftables

% List of acronyms
% Acronyms manual: http://linorg.usp.br/CTAN/macros/latex/contrib/acronym/acronym.pdf
\listofacronyms
\input{acronyms}

% List of listings
%\lstlistoflistings

\mainmatter

\include{chapters/intro}
\include{chapters/referencial_teorico}

% \include{chapters/introduction/main}
% \include{chapters/background/main}
% \include{chapters/proposed_solution/main}
% \include{chapters/experiment/main}
% \include{chapters/conclusion/main}

%\bibliographystyle{natbib}
%\addcontentsline{toc}{chapter}{\bibliographytocname}
\bibliography{references}

% Appendix
\clearpage
\addappheadtotoc
\appendix
\appendixpage
% \include{appendix/experiment-instruments}

\end{document}

%\bibliographystyle{natbib}
%\addcontentsline{toc}{chapter}{\bibliographytocname}
\bibliography{references}

% Appendix
\clearpage
\addappheadtotoc
\appendix
\appendixpage
% \include{appendix/experiment-instruments}

\end{document}

%\bibliographystyle{natbib}
%\addcontentsline{toc}{chapter}{\bibliographytocname}
\bibliography{references}

% Appendix
\clearpage
\addappheadtotoc
\appendix
\appendixpage
% \include{appendix/experiment-instruments}

\end{document}

%\bibliographystyle{natbib}
%\addcontentsline{toc}{chapter}{\bibliographytocname}
\bibliography{references}

% Appendix
\clearpage
\addappheadtotoc
\appendix
\appendixpage
% \include{appendix/experiment-instruments}

\end{document}